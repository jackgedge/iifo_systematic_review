\documentclass[twocolumn,10pt]{IEEEtran}

\usepackage[utf8]{inputenc}
\usepackage{array}
\usepackage[T1]{fontenc}
\usepackage{ragged2e}
\usepackage{graphicx}
\usepackage{float}
\usepackage{caption}
\captionsetup{hypcap=false}
\usepackage{amsmath}
\usepackage{geometry}
\usepackage{longtable}
\usepackage{hyperref}
\usepackage{fancyhdr}
\usepackage{placeins}
\usepackage{multicol}
\usepackage{titling}
\usepackage{tabularx}
\usepackage{csquotes}
\usepackage{times}
\usepackage{pdflscape}
\usepackage{adjustbox}
\usepackage{booktabs}
\usepackage{hyperref} % should already be loaded if using biblatex
\hypersetup{breaklinks=true} % important
\Urlmuskip=0mu plus 1mu % makes breaking more flexible
\usepackage{textcomp}  % <---- ADD THIS
\usepackage[backend=biber, style=numeric-comp, sorting=none]{biblatex}
\addbibresource{IIFO.bib}
\hbadness=10000 % suppress hbox warnings
% Redefine the separator for multiple citations to remove spaces
\DeclareDelimFormat{multicite}{\addsemicolon} % Optional: replace default comma

\geometry{margin=0.5in}
\pagestyle{fancy}
\fancyhf{}
\rhead{A systematic review and meta-analysis of outcomes following intentional foreign object ingestion.}
\lhead{Does motivation matter?}
\rfoot{\thepage}

\title{\textbf{Does motivation matter? A systematic review and meta-analysis of outcomes following intentional foreign object ingestion.}}

\author{
Jack Galbraith-Edge,
Mark Stanton,
and Giles Cattermole
\thanks{Jack Galbraith-Edge and Giles Cattermole are with Queen Mary, University of London, London, United Kingdom (e-mail: ha201591@qmul.ac.uk; g.cattermole@qmul.ac.uk).}
\thanks{Mark Stanton is a Critical Care Paramedic, Johannesburg, South Africa (e-mail: trauma.1@mweb.co.za).}
}
\date{\today}

\begin{document}

\maketitle
\section{Abstract}

\subsection*{Background}
Intentional ingestion of foreign objects (IIFO) is clinically distinct from accidental ingestion, yet its outcome profile and drivers of morbidity remain poorly defined. Whether motivation itself alters risk has never been examined systematically.

\subsection*{Objectives}
To synthesise all available evidence on outcomes after intentional foreign-object ingestion, and to determine whether patient motivation, object characteristics, or demographic factors modify those outcomes.

\subsection*{Methods}
A comprehensive search of PubMed, Embase, CENTRAL, Web of Science, Scopus, PsycINFO, and Google Scholar (1906--31 March 2025) combined free-text and controlled-vocabulary terms for foreign bodies, intentional ingestion, and clinical interventions.

\paragraph*{Inclusion} human studies of any design, any age, reporting non-accidental ingestion of true foreign bodies.  
\paragraph*{Exclusion} accidental or substance ingestion, animal studies, non-English full texts, pre-1906 reports, and studies lacking motivation, object, or outcome data.  
\paragraph*{Outcomes} endoscopy, surgery, conservative management, complications, mortality.  

Case reports (\emph{n} = 72; 72 patients) and case series (\emph{n} = 3; 90 patients) were extracted with a prespecified codebook; risk of bias was appraised with JBI checklists augmented by automated logic filters.  
Effect measures for case reports were odds ratios (ORs) with 95\,\% confidence intervals (CIs) derived from 2~$\times$~2 tables ($\chi^2$ or Fisher’s exact test). Multivariate case-level associations were explored with logistic regression.  
To enable valid series-level meta-analysis despite small sample sizes, individual case reports were flattened and grouped alongside other series.  
Series-level proportions were pooled using random-effects models with restricted maximum likelihood (REML) estimation and Hartung--Knapp adjustments to confidence intervals, appropriate for small-study meta-analysis.  
DerSimonian--Laird estimates were also computed for comparison.  
Univariate meta-regression examined study-level modifiers.

\subsection*{Results}
Across 162 unique individuals, pooled series-level event rates using REML were:  
endoscopy 47.3\,\% (95\,\% CI: 4.3–94.7),  
surgery 30.6\,\% (95\,\% CI: 12.0–58.9),  
conservative management 41.6\,\% (95\,\% CI: 4.4–91.7),  
complications 34.7\,\% (95\,\% CI: 1.8–93.8),  
mortality 2.5\,\% (95\,\% CI: 0.8–7.7).  

Heterogeneity was moderate to low for all outcomes except death ($I^2 < 1\,\%$).  
DerSimonian--Laird estimates yielded narrower confidence intervals and higher $I^2$ values, consistent with known limitations in small samples.  

Case-level testing showed that \enquote{intent to harm} quintupled the odds of surgery (OR~$\approx$~5, $p < 0.05$), whereas \enquote{other} motives reduced surgical odds by 86\,\%. Sharp objects increased the likelihood of endoscopy, and long objects trended toward higher surgical rates.  
Age 41--60 years predicted fewer surgeries (OR~$\approx$~0.18, $p < 0.05$). No demographic factor independently altered mortality, although detained status and severe disability exhibited non-significant trends.  
Meta-regression of series data did not replicate these signals, likely due to small study numbers.
\subsection*{Limitations}
Single-reviewer screening and extraction, high clinical heterogeneity, small cell sizes, and publication bias (particularly for dramatic or anatomically challenging cases). Twenty-five predictors in four case series risked model overfitting and inflated Type I error.

\subsection*{Conclusions}
Intention itself appears to be a key driver of adverse clinical course after foreign-object ingestion, with \enquote{intent to harm}, sharp items, and object length portending more invasive management. Age between 41 and 60 years may be protective against surgery. The findings should inform risk stratification but must be interpreted cautiously given methodological limitations and the descriptive nature of much of the evidence.

\subsection*{Registration and Data}
This protocol was not registerd with PROSPERO. Full dataset and analysis scripts are openly available at \url{https://github.com/jackgedge/iifo_systematic_review}.



% Introduction
\section*{Introduction}

\subsection*{Rationale}
The global displacement crisis has reached unprecedented levels, with over 100 million forcibly displaced individuals reported by the United Nations High Commissioner for Refugees (UNHCR) as of May 2024 \cite{UNHCR_2022}. Refugees and asylum seekers often endure extreme hardships, compelling them to seek asylum in foreign countries \cite{UNHCR_2010, AmnestyInternational_2024}. This vulnerable population frequently faces compounded mental health challenges due to traumatic pre-migration experiences, hazardous journeys, and difficult post-migration realities, including detention and instability of legal status \cite{Athwal_2015, Sundvall_2015, Nickerson_2019, Bevione_2024}.

Self-harm, encompassing various behaviours where individuals inflict harm on themselves, is a particularly alarming manifestation of these mental health challenges. Rates of self-harm are significantly elevated among asylum seekers and refugees compared to general populations, especially among those who are detained, with rates up to 216 times higher in offshore detention facilities than in the general population \cite{vonWerthern_2018, Hedrick_2019, GlobalDetentionProject_2024}.

Methods of suicide and self-harm among refugees differ based on available means, cultural factors, and motivating factors \cite{Ajdacic-Gross_2008}. Common methods include cutting, self-battery, attempted hanging, self-poisoning by medication or chemicals, and intentional ingestion of foreign objects \cite{Hedrick_2019}.

Intentional ingestion of foreign objects (IIFO) is defined as non-accidental ingestion of a true foreign body (non-nutritive items)\cite{Becq_2021}. Most ingested foreign bodies (80--90\%) pass spontaneously, but 10--20\% require endoscopic removal and up to 1\% need surgery. Timely assessment and intervention are critical \cite{Ikenberry_2011, Birk_2016}. In refugee contexts, however, geographic isolation and limited access to advanced care complicate timely management, potentially increasing morbidity and mortality \cite{Vujkovic_2019}.

Globally, rates of IIFO are increasing. In the United States, rates doubled in 2017, with 14\% of cases deemed intentional \cite{Hsieh_2020}. A 2009 review found intentional ingestions in up to 92\% of adults from lower socioeconomic populations, suggesting that rates may be even higher among refugees and asylum seekers \cite{Palta_2009}.

Management of IIFO has been evolving since 1635, when Daniel Schwaban recorded the first gastrotomy on a man who had swallowed a knife \cite{Moehlau_1895}. In 1738, Gorsauld is credited as the first surgeon to perform a cervical esophagotomy for the removal of a foreign body (FB) \cite{Saint_1929}. In 1906, José Goyanes extracted a coin impacted in the esophagus using a rigid esophagoscope for the first time \cite{Barros_1991}. The early 20\textsuperscript{th} century saw the emergence of rigid esophagoscopy as the first large-scale method for foreign body extraction, with further case series detailing technical refinements appearing in the literature \cite{Lerche_1911, Jackson_1957}. Among the most extraordinary documented cases is that by Chalk, who reported a psychiatric patient ingesting 2,533 objects weighing a total of 21,268 grams \cite{Chalk_1928}. The largest single ingested item reported measured 28 cm in length \cite{Ricote_1985}.

Clinical outcomes are influenced by various factors, including patient age, comorbidities, object characteristics (size, shape, composition, anatomical location), and the time elapsed since ingestion and current guidance advises invasive foreign object extraction guidance based on these factors \cite{Ikenberry_2011}.

Literature to date largely focuses on IIFO in prisons or psychiatric contexts, with sparse data from displaced or asylum-seeking populations. In detention, where traditional communication channels are obstructed, ingestion may serve as a form of protest or distress signal \cite{Puggioni_2014}. Conversely, in psychiatric settings, ingestion may reflect mental illness or affective dysregulation \cite{Gitlin_2007, Tromans_2019, Al-Faham_2020b, Pantazopoulos_2022, Aitchison_2024}.

Psychiatric conditions most frequently associated with intentional ingestion of foreign objects (IIFO) include psychosis, malingering, pica, and personality disorders \cite{Poynter_2011, Gitlin_2007}.

Malingering can present in various forms, particularly in prison populations where manipulation to trigger medical transfer is a noted motivation \cite{Poynter_2011, Gitlin_2007, Losanoff_1997e}. In such cases, the optimal management often involves brief medical intervention with minimal reinforcement, followed by prompt return to custody \cite{Blaho_1998}. In contrast, individuals with obsessive-compulsive disorder (OCD) may describe escalating anxiety prior to ingestion followed by a sense of relief afterward \cite{Poynter_2011}.

In cases involving borderline personality disorder, Gitlin et al. \cite{Gitlin_2007} suggest that IIFO may function as an affect regulation strategy, particularly during episodes of perceived abandonment. While such behaviour may appear life-threatening, it should not be presumed to indicate suicidal intent \cite{Poynter_2011}.

In rare and severe cases, some authors have proposed a palliative care approach to repeated IIFO, recognising the limited prognosis associated with treatment-resistant psychiatric illness and the cumulative harms of repeated surgical intervention \cite{Jaini_2023}.

Despite the rising prevalence, the heterogeneity in populations engaging in, and the potential severity of IIFO, there is limited research exploring how motivations for ingestion differ across populations and how these motivations may influence clinical outcomes \cite{Bhugra_2010, Haase_2022, Hedrick_2023a}. Varying motivations likely influence clinical management, including decisions around the need for endoscopic or surgical intervention. For instance, if ingestion is primarily intended as protest, patients may avoid behaviours that risk severe harm, potentially lowering the threshold for conservative management.

This systematic review aimed to address these gaps in the literature by evaluating how motivation for IIFO influences clinical outcomes. Specifically, we aim to examine how different motivations impact rates of endoscopic and surgical interventions, in the hope of informing future clinical strategies and healthcare responses.
\subsection{Objectives}
This systematic review aims to achieve the following objectives:

\begin{enumerate}
  \item \textbf{Quantify the rates} of endoscopic and surgical interventions following intentional ingestion of foreign objects in human populations.
  \item \textbf{Examine the influence of individual factors}—including demographics and motivations such as protest, self-harm, or suicidal intent—on the likelihood of requiring invasive intervention.
  \item \textbf{Assess the relationship between object type and clinical outcomes}, including rates of endoscopic and surgical procedures, as well as associated complications.
\end{enumerate}



% PRISMA Diagram
\begin{figure*}[t]
\centering
\includegraphics[width=\textwidth]{figures/prisma_diagram.png}
\caption{PRISMA flow diagram summarising the study selection process.}
\label{fig:prisma}
\end{figure*}

% Methods section
\section*{Methods}
\subsection*{} 
\vspace{-1em}
This study was conducted according to the Preferred Reporting Items for Systematic Reviews and Meta-Analyses (PRISMA) guidelines \cite{Page_2021}. Ethical approval was not required as all analysis was based on published data. Eligibility criteria were structured using the PICOS (Population, Intervention, Comparator, Outcome, Studies) framework.


\subsection*{Eligibility Criteria}

\begin{table}[H]
\renewcommand{\arraystretch}{1.05}
\centering
\begin{tabular}{p{3.0cm} p{5.0cm}}
\toprule
\textbf{Category} & \textbf{Details} \\
\midrule

Population & Any human; any age group. \\
\midrule

Interventions or exposures & Non-accidental ingestion of a true foreign body (non-nutritive items). \\
\midrule

Comparators / Control group & 
Demographics: \newline
Gender, age, detained person, psychiatric inpatient, displaced person, under influence of alcohol, psychiatric history, severely disabled, previous ingestion. \newline
Motivation: \newline
Intent to harm, psychiatric, psychosocial, protest, other. \newline
Object characteristics: \newline
Button battery, magnet, long ($>$5 cm), large diameter ($>$2.5 cm), multiple, blunt objects, sharp-pointed objects. \\
\midrule

Outcomes of interest & Endoscopic intervention, surgical intervention, conservative management, complication rates, mortality. \\
\midrule

Setting & Any setting. \\
\midrule

Study designs & Any design. \\
\bottomrule
\end{tabular}
\caption{Inclusion criteria structured using the PICOS framework.}
\label{tab:inclusion-criteria-intro}
\end{table}

A full list of eligibility criteria is shown in Table~\ref{tab:inclusion-criteria-intro}. This is reproduced in a larger format for clarity in Appendix~\ref{tab:inclusion-criteria}. A full list of exclusion criteria is available in Appendix~\ref{tab:exclusion-criteria} and in the PRISMA diagram shown in Figure~\ref{fig:prisma}.





\subsection*{Information Sources}

Relevant articles were identified through a systematic search of PubMed, Web of Science, Embase, Scopus, PsycINFO, CENTRAL and Google Scholar on 15th January 2025, with the assistance of a librarian.


\subsection*{Search Strategy}
The search was conducted using keywords and MeSH terms based on the concepts underpinning this review. The search queries, keywords and MeSH terms used can be found in Appendix~\ref{appendix:search-strategy}. 
 




\subsection*{Selection Process}

All identified articles were collated, duplicate articles were identified and removed. Following duplicate removal, all remaining articles underwent independent title and abstract screening conducted by the first author (JGE). To ensure consistency, a randomly selected 10\% sample of these articles underwent independent screening by a second author (MS) during both stages. Any discrepancies identified between these two reviewers were resolved by a third reviewer (GC). Inter-reviewer agreement was calculated at each screening stage.


\subsection*{Data Collection Process}
Data were extracted by a single reviewer (JGE) into an Excel spreadsheet. Variables for extraction were developed through an iterative process of engaging with the literature and identifying consistent patterns in the data reported. A preliminary analysis of the first 30 case reports informed the development of additional data categories, which were subsequently applied to the remaining reports. Once the case report data were extracted, these structured variables were used to guide the extraction of aggregate data from case series. Studies were grouped for extraction according to their classification as case reports or case series. Where case series contained sufficiently granular data, cases were extracted individually and treated as case reports; otherwise, data were extracted at the aggregate level. Case grouping for analysis was based on whether they met criteria for inclusion as individual case reports or case series, as defined above. Relevant data from reviews and other literature types were recorded under the case report category.

\subsection*{Data Items}
Outcome data were extracted for rates of endoscopy, surgery, conservative management, mortality, and complications. For the purposes of this study, \textit{surgery} was defined as \enquote{any operative intervention performed in a sterile operating theatre under general or regional anaesthesia, involving incision or surgical access to body cavities (including laparotomy, laparoscopy, thoracotomy, or cervical exploration) for the purpose of removing an ingested object or managing complications of ingestion}. Procedures performed \enquote{solely via flexible or rigid endoscopy through natural orifices} were categorised as \textit{endoscopy} and not considered surgical interventions.

Predictor variables were grouped in five subgroups: \textit{Age Group}, \textit{Gender}, \textit{Demographic}, \textit{Motivation}, \textit{Object}. Motivations are elaborated on here as they are key variables in this review. Intent to harm was defined as ingestion carried out with the purpose of self-harm, self-injury, or suicide. If the ingestion was for another reason, it was recorded as ‘N’. If the intent was unclear, it was marked as ‘UK’. Protest-related ingestion was defined as an act of ingestion carried out as a form of protest, demonstration, or manipulation—for example, in response to detention conditions. If the ingestion was not motivated by protest, it was marked as ‘N’; if unclear, as ‘UK’. Psychosocial motivation included ingestion motivated by social or interpersonal dynamics—such as imitative behaviour, a desire to shock, body-image concerns, or the use of ingestion for safekeeping objects. If these factors were not involved, the case was marked as ‘N’; if unclear, as ‘UK’. Unknown motivation was recorded when no clear motivation was identifiable in the documentation. If a specific motive was documented, this was marked as ‘N’. If the available information was ambiguous, it was recorded as ‘UK’. 

Full definitions of all variables are provided in Appendix~\ref{appendix:variables}. The full dataset of extracted \href{https://github.com/jackgedge/iifo_systematic_review/blob/main/input/processed_data/final_report/case_data_export.csv}{case-level} and \href{https://github.com/jackgedge/iifo_systematic_review/blob/main/input/processed_data/final_report/aggregate_series_data_export.csv}{series-level} data (including bias assesments), is available on \href{https://github.com/jackgedge/iifo_systematic_review/blob/main/input/processed_data/final_report/}{Github}.

\subsection*{Risk of Bias Assessment}

Risk of bias was assessed manually for all included studies by a single reviewer (JGE), using the Joanna Briggs Institute (JBI) Critical Appraisal Checklists for Case Reports and Case Series \cite{Moola_2020}. Studies were first classified as either case reports or case series based on the level of granularity in the data. Case reports were assessed using the JBI checklist for case reports, and case series were assessed using the corresponding JBI tool.

Following manual appraisal, a secondary risk-of-bias filter was applied using Python Pandas \cite{ThePandasDevelopmentTeam_2020}. This logic-based filter identified studies where key variables — specifically \textit{Outcome}, \textit{Motivation}, or \textit{Object} — were missing or marked as unknown (\enquote{UK}). For case series, if any of the derived aggregate fields (e.g. \textit{Outcome\_Unknown\_Rate}, \textit{Motivation\_Unknown\_Rate}, \textit{Object\_Unknown\_Rate}) equalled 1, the study was flagged as high risk. Similarly, case reports where any of these variables were unknown were also considered high risk.

Studies classified as high risk through this process were excluded from analysis. This two-stage approach — involving initial manual assessment and subsequent automated validation — ensured both qualitative and quantitative scrutiny of bias across the dataset.




\subsection*{Effects Measures}
For binary outcomes (endoscopy, surgery, complications, and death), the effect measure used was the odds ratio (OR), calculated from 2x2 contingency tables. Each odds ratio was accompanied by a 95\% confidence interval (CI) and a p-value from a chi-square test of independence.

This approach was used consistently across all pairwise comparisons between binary exposure variables (e.g., motivations, object types, population characteristics) and binary outcome variables. Significant associations were identified at a threshold of $p < 0.05$ and reported alongside their respective ORs and CIs. Due to the small number of deaths observed, effect estimates for death should be interpreted with caution.

\subsection*{Logistic Regression Modelling}

To explore which factors were independently associated with clinical outcomes, we performed multivariable logistic regression analyses for four outcomes of interest: endoscopy, surgery, complications, and death. For each outcome, we constructed a logistic regression model including the following groups of predictor variables: age group, gender, population characteristics, ingestion motivation, and object type.

Age group, gender, and motivation variables were entered as one-hot encoded categorical variables with a reference category omitted. Population characteristics and object type variables were included as binary indicators. A constant term was included in each model. All variables were selected a priori based on clinical relevance and prior univariable (chi-square) analyses.

Missing values in predictor variables were imputed as zero. Models were fitted using maximum likelihood estimation via the 	exttt{statsmodels} Python library. In the event that a model failed to converge or could not be fitted (as occurred for the death outcome due to small sample size), an empty result table was substituted to maintain consistency of reporting across outcomes.

For each predictor, we reported the odds ratio (OR) with corresponding 95\% confidence interval (CI) and p-value. Results from all models were summarised in a single grouped wide table, with predictors grouped into their logical domains (age, gender, population, motivation, object). The intercept term (	exttt{const}) was excluded from the summary table. Significant predictors ($p < 0.05$) were flagged with an asterisk.



\begin{table*}[!t]
\caption{Grouped summary of case-level variables and outcomes.}
\label{tab:grouped_summary_wide}
\centering
\begin{adjustbox}{width=\textwidth}
\renewcommand{\arraystretch}{1.2}
\small
\begin{tabularx}{\textwidth}{p{4cm} p{1.5cm} p{2cm} X}
\toprule
\textbf{Variable} & \textbf{Count} & \textbf{Percentage} & \textbf{References} \\
\midrule

\multicolumn{4}{l}{\textit{Gender}} \\ 
\mbox{\hspace{1em}Male} & 43 & 60\% & \cite{Akay_2015f, Al-Faham_2020k, Alao_2006i, Ali_2017, Ali_2022g, Apikotoa_2022f, Atayan_2016, Benoist_2019e, Berry_2021e, Bhumi_2024f, CamachoDorado_2018, Csaky_1998e, Emamhadi_2018, Farhadi_2024h, Fry_2010, Gardner_2017h, Guinan_2019f, Jehangir_2019h, Jin_2023, Kobiela_2015, Kumar_2001, Kumar_2019f, Liu_2005, Losanoff_1996, Losanoff_1997e, Mesfin_2022a, Misra_2013, Qureshi_2016, Riva_2018j, Sobnach_2011f, Tammana_2012j, Tanrikulu_2015e, Tay_2004, Thapa_2019f, Trgo_2012f, Wadhwa_2015e, Yasin_2009, teWildt_2010} \\
\mbox{\hspace{1em}Female} & 28 & 39\% & \cite{AlShaaibi_2021b, Ali_2020f, Ataya_2013, Beecroft_1998, Bhasin_2014, Bhattacharjee_2008, Cauchi_2002, Chang_2017f, Cox_2007, DelgadoSalazar_2020c, DivsalarP._2023a, Goldman_1998f, Hardy_2023g, Kar_2015, Kariholu_2008, Kerestes_2019, Li_2013, Naji_2012f, Ohno_2005, Peixoto_2017f, Sakellaridis_2008f, Sultan_2024f, Tupesis_2004f, Wildhaber_2005, Wnęk_2015f, Yildiz_2016e} \\
\mbox{\hspace{1em}Unknown} & 1 & 1\% & \cite{fjbuilsRepeatedBehaviorDeliberate2024} \\
\multicolumn{4}{l}{\textit{Age Group}} \\ 
\mbox{\hspace{1em}\textless{}18} & 13 & 18\% & \cite{AlShaaibi_2021b, Ali_2020f, Cauchi_2002, DivsalarP._2023a, Goldman_1998f, Liu_2005, Naji_2012f, Ohno_2005, Tanrikulu_2015e, Tay_2004, Wildhaber_2005} \\
\mbox{\hspace{1em}18--25} & 18 & 25\% & \cite{Akay_2015f, Ali_2017, Atayan_2016, Bhattacharjee_2008, Csaky_1998e, Kar_2015, Kariholu_2008, Kobiela_2015, Losanoff_1996, Losanoff_1997e, Mesfin_2022a, Peixoto_2017f, Sobnach_2011f, Tupesis_2004f, Yasin_2009} \\
\mbox{\hspace{1em}26--40} & 25 & 35\% & \cite{Alao_2006i, Ali_2022g, Apikotoa_2022f, Ataya_2013, Benoist_2019e, Bhasin_2014, Chang_2017f, Cox_2007, DelgadoSalazar_2020c, Farhadi_2024h, Fry_2010, Gardner_2017h, Guinan_2019f, Jin_2023, Kumar_2019f, Losanoff_1996, Misra_2013, Qureshi_2016, Riva_2018j, Sakellaridis_2008f, Tammana_2012j, Trgo_2012f, Wnęk_2015f, Yildiz_2016e, fjbuilsRepeatedBehaviorDeliberate2024} \\
\mbox{\hspace{1em}41--60} & 11 & 15\% & \cite{Al-Faham_2020k, Bhumi_2024f, CamachoDorado_2018, Emamhadi_2018, Hardy_2023g, Jehangir_2019h, Kumar_2001, Sultan_2024f, Thapa_2019f, Wadhwa_2015e, teWildt_2010} \\
\mbox{\hspace{1em}60+} & 3 & 4\% & \cite{Beecroft_1998, Kerestes_2019, Li_2013} \\
\mbox{\hspace{1em}Unknown} & 2 & 3\% & \cite{Berry_2021e} \\
\multicolumn{4}{l}{\textit{Population}} \\ 
\mbox{\hspace{1em}Detained Person} & 12 & 17\% & \cite{Alao_2006i, Ali_2022g, Apikotoa_2022f, Losanoff_1996, Losanoff_1997e, Qureshi_2016, Tammana_2012j, Trgo_2012f} \\
\mbox{\hspace{1em}Psychiatric Inpatient} & 4 & 6\% & \cite{DivsalarP._2023a, fjbuilsRepeatedBehaviorDeliberate2024, teWildt_2010} \\
\mbox{\hspace{1em}Displaced Person} & 2 & 3\% & \cite{Akay_2015f, Gardner_2017h} \\
\mbox{\hspace{1em}Under Influence of Alcohol} & 3 & 4\% & \cite{Benoist_2019e, Csaky_1998e, Thapa_2019f} \\
\mbox{\hspace{1em}Psychiatric History} & 36 & 50\% & \cite{AlShaaibi_2021b, Alao_2006i, Ali_2020f, Apikotoa_2022f, Ataya_2013, Atayan_2016, Beecroft_1998, CamachoDorado_2018, Chang_2017f, DelgadoSalazar_2020c, DivsalarP._2023a, Farhadi_2024h, Fry_2010, Guinan_2019f, Hardy_2023g, Jehangir_2019h, Jin_2023, Kar_2015, Kerestes_2019, Kobiela_2015, Kumar_2001, Kumar_2019f, Liu_2005, Mesfin_2022a, Misra_2013, Ohno_2005, Peixoto_2017f, Sakellaridis_2008f, Sultan_2024f, Tammana_2012j, Tanrikulu_2015e, Yildiz_2016e, fjbuilsRepeatedBehaviorDeliberate2024, teWildt_2010} \\
\mbox{\hspace{1em}Severely Disabled} & 7 & 10\% & \cite{Atayan_2016, Kerestes_2019, Liu_2005, Ohno_2005, Peixoto_2017f, Yildiz_2016e, teWildt_2010} \\
\mbox{\hspace{1em}Previous Ingestor} & 19 & 26\% & \cite{Alao_2006i, Apikotoa_2022f, Berry_2021e, Bhattacharjee_2008, Csaky_1998e, DivsalarP._2023a, Emamhadi_2018, Guinan_2019f, Jehangir_2019h, Jin_2023, Liu_2005, Sakellaridis_2008f, Tanrikulu_2015e, Thapa_2019f, Yildiz_2016e, fjbuilsRepeatedBehaviorDeliberate2024, teWildt_2010} \\
\multicolumn{4}{l}{\textit{Motivation}} \\ 
\mbox{\hspace{1em}Intent to harm} & 21 & 29\% & \cite{Al-Faham_2020k, AlShaaibi_2021b, Alao_2006i, Ali_2017, CamachoDorado_2018, Chang_2017f, Cox_2007, Csaky_1998e, Fry_2010, Li_2013, Losanoff_1996, Losanoff_1997e, Mesfin_2022a, Sakellaridis_2008f, Tammana_2012j, Tanrikulu_2015e, fjbuilsRepeatedBehaviorDeliberate2024} \\
\mbox{\hspace{1em}Protest} & 9 & 12\% & \cite{Bhumi_2024f, Gardner_2017h, Losanoff_1996, Losanoff_1997e, Tupesis_2004f} \\
\mbox{\hspace{1em}Psychiatric} & 34 & 47\% & \cite{Al-Faham_2020k, Alao_2006i, Ali_2020f, Apikotoa_2022f, Ataya_2013, Atayan_2016, Bhasin_2014, Bhattacharjee_2008, DelgadoSalazar_2020c, DivsalarP._2023a, Emamhadi_2018, Farhadi_2024h, Guinan_2019f, Hardy_2023g, Jehangir_2019h, Jin_2023, Kar_2015, Kariholu_2008, Kerestes_2019, Kobiela_2015, Kumar_2001, Kumar_2019f, Li_2013, Liu_2005, Misra_2013, Ohno_2005, Sakellaridis_2008f, Sultan_2024f, Tammana_2012j, Tanrikulu_2015e, Yasin_2009, teWildt_2010} \\
\mbox{\hspace{1em}Psychosocial} & 17 & 24\% & \cite{Akay_2015f, Benoist_2019e, Bhattacharjee_2008, Cauchi_2002, Goldman_1998f, Hardy_2023g, Kobiela_2015, Li_2013, Naji_2012f, Qureshi_2016, Riva_2018j, Sobnach_2011f, Tay_2004, Thapa_2019f, Tupesis_2004f, Wildhaber_2005, Wnęk_2015f} \\
\mbox{\hspace{1em}Other} & 9 & 12\% & \cite{Ali_2020f, Ali_2022g, Emamhadi_2018, Guinan_2019f, Peixoto_2017f, Sakellaridis_2008f, Trgo_2012f, Wadhwa_2015e, Yildiz_2016e} \\
\multicolumn{4}{l}{\textit{Object}} \\ 
\mbox{\hspace{1em}Button Battery} & 2 & 3\% & \cite{Berry_2021e, Bhumi_2024f} \\
\mbox{\hspace{1em}Magnet} & 9 & 12\% & \cite{Ali_2020f, Bhumi_2024f, Cauchi_2002, Liu_2005, Naji_2012f, Ohno_2005, Tanrikulu_2015e, Tay_2004, Wildhaber_2005} \\
\mbox{\hspace{1em}Long (\textgreater{}5cm)} & 32 & 44\% & \cite{Al-Faham_2020k, AlShaaibi_2021b, Ali_2017, Ali_2022g, Atayan_2016, Bhasin_2014, CamachoDorado_2018, Chang_2017f, Cox_2007, Csaky_1998e, DivsalarP._2023a, Emamhadi_2018, Fry_2010, Gardner_2017h, Jin_2023, Kariholu_2008, Kerestes_2019, Kobiela_2015, Kumar_2019f, Mesfin_2022a, Misra_2013, Ohno_2005, Qureshi_2016, Sakellaridis_2008f, Sultan_2024f, Thapa_2019f, Trgo_2012f, Yasin_2009, Yildiz_2016e, teWildt_2010} \\
\mbox{\hspace{1em}Large (\textgreater{}2.5cm) Diameter} & 51 & 71\% & \cite{Akay_2015f, Al-Faham_2020k, AlShaaibi_2021b, Alao_2006i, Ali_2017, Ali_2022g, Apikotoa_2022f, Atayan_2016, Berry_2021e, Bhasin_2014, CamachoDorado_2018, Cauchi_2002, Chang_2017f, Cox_2007, Csaky_1998e, DivsalarP._2023a, Emamhadi_2018, Gardner_2017h, Guinan_2019f, Jehangir_2019h, Jin_2023, Kariholu_2008, Kerestes_2019, Kobiela_2015, Kumar_2001, Kumar_2019f, Losanoff_1996, Losanoff_1997e, Mesfin_2022a, Misra_2013, Naji_2012f, Ohno_2005, Peixoto_2017f, Qureshi_2016, Riva_2018j, Sakellaridis_2008f, Sultan_2024f, Tanrikulu_2015e, Thapa_2019f, Trgo_2012f, Wnęk_2015f, Yildiz_2016e, fjbuilsRepeatedBehaviorDeliberate2024, teWildt_2010} \\
\mbox{\hspace{1em}Sharp} & 34 & 47\% & \cite{AlShaaibi_2021b, Alao_2006i, Apikotoa_2022f, Ataya_2013, Benoist_2019e, Bhasin_2014, Bhattacharjee_2008, CamachoDorado_2018, Csaky_1998e, DelgadoSalazar_2020c, DivsalarP._2023a, Emamhadi_2018, Farhadi_2024h, Fry_2010, Guinan_2019f, Hardy_2023g, Jehangir_2019h, Jin_2023, Kariholu_2008, Kobiela_2015, Kumar_2019f, Losanoff_1996, Losanoff_1997e, Mesfin_2022a, Misra_2013, Sobnach_2011f, Yasin_2009, teWildt_2010} \\
\mbox{\hspace{1em}Multiple} & 44 & 61\% & \cite{Ali_2020f, Apikotoa_2022f, Ataya_2013, Atayan_2016, Beecroft_1998, Bhattacharjee_2008, Bhumi_2024f, CamachoDorado_2018, Cauchi_2002, Emamhadi_2018, Farhadi_2024h, Fry_2010, Goldman_1998f, Guinan_2019f, Hardy_2023g, Jehangir_2019h, Jin_2023, Kar_2015, Kariholu_2008, Kobiela_2015, Kumar_2001, Kumar_2019f, Li_2013, Liu_2005, Losanoff_1996, Mesfin_2022a, Misra_2013, Naji_2012f, Ohno_2005, Sobnach_2011f, Sultan_2024f, Tammana_2012j, Tanrikulu_2015e, Tay_2004, Thapa_2019f, Wadhwa_2015e, Wildhaber_2005, Yasin_2009, fjbuilsRepeatedBehaviorDeliberate2024, teWildt_2010} \\
\multicolumn{4}{l}{\textit{Outcome}} \\ 
\mbox{\hspace{1em}Endoscopy} & 31 & 43\% & \cite{Akay_2015f, Ali_2022g, Apikotoa_2022f, Atayan_2016, Benoist_2019e, Berry_2021e, Bhasin_2014, Bhumi_2024f, CamachoDorado_2018, Chang_2017f, DelgadoSalazar_2020c, Gardner_2017h, Guinan_2019f, Hardy_2023g, Jehangir_2019h, Kariholu_2008, Li_2013, Liu_2005, Ohno_2005, Peixoto_2017f, Qureshi_2016, Riva_2018j, Sakellaridis_2008f, Sultan_2024f, Tammana_2012j, Tanrikulu_2015e, Trgo_2012f, Wadhwa_2015e, Wnęk_2015f, teWildt_2010} \\
\mbox{\hspace{1em}Surgery} & 44 & 61\% & \cite{Al-Faham_2020k, AlShaaibi_2021b, Alao_2006i, Ali_2017, Ali_2020f, Atayan_2016, Beecroft_1998, Bhasin_2014, CamachoDorado_2018, Cauchi_2002, Chang_2017f, Cox_2007, Csaky_1998e, DelgadoSalazar_2020c, DivsalarP._2023a, Farhadi_2024h, Fry_2010, Gardner_2017h, Jin_2023, Kariholu_2008, Kerestes_2019, Kobiela_2015, Kumar_2019f, Liu_2005, Losanoff_1996, Losanoff_1997e, Mesfin_2022a, Misra_2013, Naji_2012f, Sobnach_2011f, Tanrikulu_2015e, Tay_2004, Thapa_2019f, Tupesis_2004f, Wildhaber_2005, Wnęk_2015f, Yasin_2009, Yildiz_2016e, fjbuilsRepeatedBehaviorDeliberate2024} \\
\mbox{\hspace{1em}Death} & 2 & 3\% & \cite{Emamhadi_2018, Kumar_2001} \\
\mbox{\hspace{1em}Conservative} & 7 & 10\% & \cite{Ataya_2013, Bhattacharjee_2008, DivsalarP._2023a, Emamhadi_2018, Goldman_1998f, Kar_2015, Kumar_2001} \\
\mbox{\hspace{1em}Complication} & 48 & 67\% & \cite{Ali_2017, Ali_2020f, Apikotoa_2022f, Atayan_2016, Beecroft_1998, Benoist_2019e, Berry_2021e, Bhasin_2014, Bhumi_2024f, CamachoDorado_2018, Cauchi_2002, Cox_2007, Csaky_1998e, DelgadoSalazar_2020c, DivsalarP._2023a, Emamhadi_2018, Farhadi_2024h, Fry_2010, Gardner_2017h, Goldman_1998f, Jin_2023, Kariholu_2008, Kerestes_2019, Kobiela_2015, Kumar_2001, Kumar_2019f, Liu_2005, Losanoff_1996, Mesfin_2022a, Misra_2013, Naji_2012f, Ohno_2005, Sakellaridis_2008f, Sobnach_2011f, Sultan_2024f, Tanrikulu_2015e, Tay_2004, Thapa_2019f, Trgo_2012f, Tupesis_2004f, Wildhaber_2005, Wnęk_2015f, Yasin_2009, Yildiz_2016e} \\
\bottomrule
\end{tabularx}
\end{adjustbox}
\end{table*}


\begin{figure}[t]
\centering
\includegraphics[width=\columnwidth]{figures/case_country_plot.png}
\caption{Heat map of cases per country.}
\label{fig:country_heat_map}
\end{figure}

\section*{Results}

% Case data characteristics


\begin{table*}[!t]
\caption{Grouped aggregate series-level summary.}
\label{tab:series_results}
\centering
\renewcommand{\arraystretch}{1.2}
\small
\begin{adjustbox}{width=\textwidth}
\begin{tabularx}{\textwidth}{p{4cm}>{\centering\arraybackslash}X>{\centering\arraybackslash}X>{\centering\arraybackslash}X>{\centering\arraybackslash}X>{\centering\arraybackslash}X>{\centering\arraybackslash}X}
\toprule
\textbf{Variable} & \textbf{Pooled} & \textbf{Case Reports} & \textbf{Case Series} & \textbf{Karp \textit{et al.} (1991) \cite{Karp_1991b}} & \textbf{Lee \textit{et al.} (2007) \cite{Lee_2007}} & \textbf{Elghali \textit{et al.} (2016) \cite{Elghali_2016}} \\ 
\midrule
\multicolumn{1}{l}{\textit{Total Cases}} & 162 & 72 & 90 & 19 & 52 & 19 \\ 
\textit{Gender} & & & & & & \\ 
\hspace{1em}Male & 133 (82) & 43 (60) & 90 (100) & 19 (100) & 52 (100) & 19 (100) \\ 
\hspace{1em}Female & 28 (17) & 28 (39) & 0 (0) & 0 (0) & 0 (0) & 0 (0) \\ 
\hspace{1em}Unknown & 1 (1) & 1 (1) & 0 (0) & 0 (0) & 0 (0) & 0 (0) \\ 
\textit{Age Group} & & & & & & \\ 
\hspace{1em}Maximum & 100 & 100 & 50 & 40 & 50 & 27 \\ 
\hspace{1em}Mean & 28 & 30 & 24 & 24 & --- & 24 \\ 
\hspace{1em}Median & 27 & 27 & 35 & --- & 35 & --- \\ 
\hspace{1em}Minimum & 7 & 7 & 17 & 17 & 25 & 19 \\ 
\textit{Demographic} & & & & & & \\ 
\hspace{1em}Detained Person & 102 (50) & 12 (14) & 90 (74) & 19 (51) & 52 (85) & 19 (81) \\ 
\hspace{1em}Psychiatric History & 65 (32) & 36 (43) & 29 (24) & 18 (49) & 9 (15) & 2 (10) \\ 
\hspace{1em}Previous Ingestor & 21 (10) & 19 (23) & 2 (2) & --- & --- & 2 (9) \\ 
\hspace{1em}Severely Disabled & 7 (3) & 7 (8) & 0 (0) & --- & 0 (0) & --- \\ 
\hspace{1em}Psychiatric Inpatient & 4 (2) & 4 (5) & 0 (0) & 0 (0) & 0 (0) & 0 (0) \\ 
\hspace{1em}Under Influence of Alcohol & 3 (1) & 3 (4) & 0 (0) & --- & --- & --- \\ 
\hspace{1em}Displaced Person & 2 (1) & 2 (2) & 0 (0) & --- & --- & --- \\ 
\textit{Motivation} & & & & & & \\ 
\hspace{1em}Protest & 79 (45) & 9 (10) & 70 (80) & 3 (16) & 50 (100) & 17 (89) \\ 
\hspace{1em}Psychiatric & 46 (26) & 34 (38) & 12 (14) & 12 (63) & 0 (0) & 0 (0) \\ 
\hspace{1em}Intent to harm & 27 (15) & 21 (23) & 6 (7) & 4 (21) & 0 (0) & 2 (11) \\ 
\hspace{1em}Psychosocial & 17 (10) & 17 (19) & 0 (0) & 0 (0) & 0 (0) & 0 (0) \\ 
\hspace{1em}Other & 9 (5) & 9 (10) & 0 (0) & 0 (0) & 0 (0) & 0 (0) \\ 
\textit{Object} & & & & & & \\ 
\hspace{1em}Sharp & 102 & 34 & 68 & 19 & 33 & 16 \\ 
\hspace{1em}Multiple & 69 & 44 & 25 & --- & 24 & 1 \\ 
\hspace{1em}Long (\textgreater{}5cm) & 64 & 32 & 32 & --- & 32 & 0 \\ 
\hspace{1em}Large (\textgreater{}2.5cm) Diameter & 51 & 51 & 0 & --- & --- & --- \\ 
\hspace{1em}Magnet & 9 & 9 & 0 & --- & 0 & 0 \\ 
\hspace{1em}Button Battery & 2 & 2 & 0 & --- & 0 & 0 \\ 
\textit{Outcome} & & & & & & \\ 
\hspace{1em}Endoscopy & 78 (34) & 31 (23) & 47 (48) & --- & 46 (79) & 1 (5) \\ 
\hspace{1em}Surgery & 59 (26) & 44 (33) & 15 (15) & 5 (26) & 6 (10) & 4 (19) \\ 
\hspace{1em}Complication & 54 (23) & 48 (36) & 6 (6) & --- & 6 (10) & --- \\ 
\hspace{1em}Conservative & 36 (16) & 7 (5) & 29 (30) & 14 (74) & 0 (0) & 15 (71) \\ 
\hspace{1em}Death & 3 (1) & 2 (2) & 1 (1) & 0 (0) & 0 (0) & 1 (5) \\ 
\bottomrule
\multicolumn{7}{l}{\parbox{\linewidth}{\small Key: \emph{n} (\%).}}
\end{tabularx}
\end{adjustbox}
\end{table*}

\subsection*{Study Selection}

A total of 808 records were identified through initial database searches: PubMed (317), Web of Science (277), Google Scholar (135), Embase (25), SCOPUS (24), PsycINFO (16), and Cochrane (14).
316 duplicates were identified and removed.

Title and abstract screening was undertaken, with JGE reviewing all 492 records. A random sample of 50 records was generated for independent screening MS.
Cohen's Kappa was calculated for inter-reviewer agreement between JGE and MS, yielding a value of 0.38, indicating fair agreement.
Where JGE and MS disagreed, 16 records were reviewed by GC.
In total, 176 records were excluded, leaving 316 for full text review.

During full text review, JGE reviewed all 316 records. 
A random sample of 32 records was generated for independent review by MS. 
Inter-reviewer agreement was again calculated using Cohen's Kappa, yielding a value of 0.21, indicating fair agreement.   
Where JGE and MS disagreed, 5 records were reviewed by GC.   
In total, 276 records were excluded during full text review. 40 records were included and proceeded to bibliography search.  

The bibliographies of the 40 included papers were searched by manually JGE. Relevant bibliography items were identified, collated, and evaluated against the eligibility criteria, yielding 194 results.
These 194 results were reviewed by JGE. 164 bibliography search records were excluded, leaving 30 for inclusion. 

Therefore, a total of 70 records were included in this study and proceeded to bias assessment. This process is illustrated in Figure~\ref{fig:prisma}.


\begin{table*}[h]
\caption{Univariate association test results.}
\label{tab:univariate_results}
\centering
\begin{adjustbox}{width=\textwidth}
\renewcommand{\arraystretch}{1.2}
\small
\begin{tabularx}{\textwidth}{p{4cm}>{\centering\arraybackslash}X>{\centering\arraybackslash}X>{\centering\arraybackslash}X>{\centering\arraybackslash}X>{\centering\arraybackslash}X}
\toprule
\textbf{Variable} & \textbf{Conservative} & \textbf{Endoscopy} & \textbf{Surgery} & \textbf{Death} & \textbf{Complication} \\
\midrule

\multicolumn{6}{l}{\textit{Gender}} \\
\hspace{1em}Male & 0.23 [0.04, 1.30] (p=0.110) & 1.42 [0.54, 3.72] (p=0.632) & 0.93 [0.36, 2.46] (p=1.000) & --- & 1.41 [0.52, 3.81] (p=0.671) \\
\hspace{1em}Female & 4.57 [0.82, 25.41] (p=0.101) & 0.78 [0.30, 2.03] (p=0.786) & 0.97 [0.37, 2.57] (p=1.000) & --- & 0.84 [0.31, 2.28] (p=0.932) \\
\hspace{1em}Unknown & --- & --- & --- & --- & --- \\

\multicolumn{6}{l}{\textit{Age Group}} \\
\hspace{1em}\textless{}18 & 1.96 [0.34, 11.45] (p=0.602) & 0.33 [0.08, 1.33] (p=0.194) & 2.45 [0.61, 9.84] (p=0.328) & --- & 1.84 [0.46, 7.44] (p=0.522) \\
\hspace{1em}18--25 & 1.23 [0.22, 6.94] (p=1.000) & 0.29 [0.08, 0.98] (p=0.074) & 2.80 [0.82, 9.62] (p=0.163) & --- & 1.41 [0.44, 4.56] (p=0.773) \\
\hspace{1em}26--40 & 0.28 [0.03, 2.51] (p=0.409) & 2.25 [0.84, 6.04] (p=0.171) & 0.93 [0.34, 2.51] (p=1.000) & --- & 0.83 [0.30, 2.31] (p=0.930) \\
\hspace{1em}41--60 & 2.49 [0.42, 14.83] (p=0.289) & 2.70 [0.71, 10.22] (p=0.188) & \textbf{0.18 [0.04, 0.76] (p=0.019)*} & --- & 0.54 [0.15, 2.00] (p=0.488) \\
\hspace{1em}60+ & --- & 0.65 [0.06, 7.51] (p=1.000) & 1.29 [0.11, 14.88] (p=1.000) & --- & 1.00 [0.09, 11.61] (p=1.000) \\
\hspace{1em}Unknown & --- & --- & --- & --- & 0.49 [0.03, 8.18] (p=1.000) \\

\multicolumn{6}{l}{\textit{Demographic}} \\
\hspace{1em}Detained Person & --- & 0.99 [0.28, 3.53] (p=1.000) & 0.86 [0.24, 3.08] (p=1.000) & --- & 0.63 [0.17, 2.26] (p=0.510) \\
\hspace{1em}Psychiatric Inpatient & --- & 0.44 [0.04, 4.54] (p=0.634) & 2.25 [0.22, 22.99] (p=0.634) & --- & 0.15 [0.01, 1.50] (p=0.103) \\
\hspace{1em}Displaced Person & --- & --- & 0.67 [0.03, 12.84] (p=1.000) & --- & 0.50 [0.03, 9.77] (p=1.000) \\
\hspace{1em}Under Influence of Alcohol & --- & 0.88 [0.07, 10.69] (p=1.000) & 1.00 [0.08, 12.27] (p=1.000) & --- & --- \\
\hspace{1em}Psychiatric History & 0.96 [0.20, 4.70] (p=1.000) & 0.80 [0.29, 2.20] (p=0.861) & 1.35 [0.48, 3.74] (p=0.756) & 0.71 [0.04, 11.97] (p=1.000) & 0.70 [0.24, 2.03] (p=0.696) \\
\hspace{1em}Severely Disabled & --- & 4.13 [0.74, 23.06] (p=0.115) & 0.81 [0.17, 3.93] (p=1.000) & --- & 1.31 [0.23, 7.35] (p=1.000) \\
\hspace{1em}Previous Ingestor & 1.69 [0.30, 9.38] (p=0.665) & 0.83 [0.26, 2.65] (p=0.986) & 0.74 [0.23, 2.36] (p=0.832) & 1.61 [0.09, 27.40] (p=1.000) & 0.34 [0.10, 1.23] (p=0.179) \\

\multicolumn{6}{l}{\textit{Motivation}} \\
\hspace{1em}Intent to harm & --- & 0.49 [0.16, 1.55] (p=0.347) & \textbf{5.40 [1.36, 21.43] (p=0.024)*} & --- & 0.84 [0.28, 2.56] (p=0.988) \\
\hspace{1em}Protest & --- & 0.33 [0.06, 1.74] (p=0.279) & 5.50 [0.64, 47.15] (p=0.137) & --- & 4.71 [0.55, 40.44] (p=0.251) \\
\hspace{1em}Psychiatric & 7.07 [0.80, 62.31] (p=0.105) & 1.44 [0.55, 3.77] (p=0.624) & 0.47 [0.17, 1.27] (p=0.212) & --- & 0.67 [0.24, 1.85] (p=0.608) \\
\hspace{1em}Psychosocial & 1.20 [0.21, 6.84] (p=1.000) & 0.89 [0.29, 2.72] (p=1.000) & 0.63 [0.21, 1.93] (p=0.603) & --- & 0.94 [0.30, 2.99] (p=1.000) \\
\hspace{1em}Other & 1.19 [0.13, 11.18] (p=1.000) & 3.04 [0.70, 13.29] (p=0.161) & \textbf{0.14 [0.03, 0.75] (p=0.023)*} & 7.75 [0.44, 136.41] (p=0.236) & 0.58 [0.14, 2.40] (p=0.469) \\

\multicolumn{6}{l}{\textit{Object}} \\
\hspace{1em}Button Battery & --- & --- & --- & --- & --- \\
\hspace{1em}Magnet & --- & 1.07 [0.26, 4.35] (p=1.000) & 2.46 [0.47, 12.80] (p=0.467) & --- & --- \\
\hspace{1em}Long (\textgreater{}5cm) & 0.45 [0.08, 2.51] (p=0.446) & 0.80 [0.31, 2.05] (p=0.820) & 2.43 [0.90, 6.56] (p=0.128) & 1.23 [0.07, 20.40] (p=1.000) & 1.88 [0.67, 5.24] (p=0.342) \\
\hspace{1em}Large (\textgreater{}2.5cm) Diameter & 0.23 [0.05, 1.17] (p=0.081) & 1.41 [0.48, 4.16] (p=0.727) & 1.65 [0.57, 4.80] (p=0.518) & --- & 0.92 [0.30, 2.85] (p=1.000) \\
\hspace{1em}Sharp & 1.56 [0.32, 7.51] (p=0.700) & \textbf{0.34 [0.13, 0.90] (p=0.048)*} & 2.16 [0.82, 5.72] (p=0.187) & 1.12 [0.07, 18.65] (p=1.000) & 1.09 [0.41, 2.90] (p=1.000) \\
\hspace{1em}Multiple & 4.26 [0.48, 37.48] (p=0.235) & 0.50 [0.19, 1.30] (p=0.233) & 1.03 [0.39, 2.71] (p=1.000) & --- & 2.60 [0.95, 7.13] (p=0.104) \\
\multicolumn{6}{l}{\small OR: Odds Ratio; CI: Confidence Interval; p: p-value. * indicates $p < 0.05$. Bold = statistically significant. --- = missing or unstable estimate.} \\\\
\bottomrule
\end{tabularx}
\end{adjustbox}
\end{table*}

\subsection*{Risk of Bias}
\subsubsection*{Case Reports} 
75 cases from 67 studies \cite{Akay_2015f, Al-Faham_2020k, AlShaaibi_2021b, Alao_2006i, Ali_2017, Ali_2020f, Ali_2022g, Apikotoa_2022f, Ataya_2013, Atayan_2016, Beecroft_1998, Benoist_2019e, Berry_2021e, Bhasin_2014, Bhattacharjee_2008, Bhumi_2024f, CamachoDorado_2018, Cauchi_2002, Chang_2017f, Cox_2007, Csaky_1998e, DelgadoSalazar_2020c, DivsalarP._2023a, Emamhadi_2018, Farhadi_2024h, Fry_2010, Gardner_2017h, Goldman_1998f, Guinan_2019f, Hardy_2023g, Jehangir_2019h, Jin_2023, Kar_2015, Kariholu_2008, Kerestes_2019, Kobiela_2015, Kumar_2001, Kumar_2019f, Lee_2012l, Li_2013, Liu_2005, Losanoff_1996, Losanoff_1997e, Mesfin_2022a, Misra_2013, Naji_2012f, Ohno_2005, Peixoto_2017f, Qureshi_2016, Riva_2018j, Sakellaridis_2008f, Sharma_2022e, Sobnach_2011f, Sultan_2024f, Tammana_2012j, Tanrikulu_2015e, Tay_2004, Thapa_2019f, Trgo_2012f, Tupesis_2004f, Wadhwa_2015e, Wildhaber_2005, Wnęk_2015f, Yasin_2009, Yildiz_2016e, fjbuilsRepeatedBehaviorDeliberate2024, teWildt_2010} were evaluated using the \textit{{JBI Checklist for Case Reports}} \cite{Moola_2020}.
3 cases were excluded.
Cases were excluded at this stage if they failed to describe the following domains: patient history and timeline (1 case) \cite{Lee_2012l}, current patient condition (2 cases) \cite{Lee_2012l}, interventions and treatments (1 case) \cite{Sharma_2022e}, patient post-intervention condition (2 cases) \cite{Lee_2012l}, harms (2 cases) \cite{Lee_2012l}, and takeaway lessons (2 cases) \cite{Lee_2012l}. The excluded cases came from the following studies: \cite{Lee_2012l, Sharma_2022e}.
Of the remaining 72 cases, 
all reported interventions and treatments (72 cases, 100\%) \cite{Akay_2015f, Al-Faham_2020k, AlShaaibi_2021b, Alao_2006i, Ali_2017, Ali_2020f, Ali_2022g, Apikotoa_2022f, Ataya_2013, Atayan_2016, Beecroft_1998, Benoist_2019e, Berry_2021e, Bhasin_2014, Bhattacharjee_2008, Bhumi_2024f, CamachoDorado_2018, Cauchi_2002, Chang_2017f, Cox_2007, Csaky_1998e, DelgadoSalazar_2020c, DivsalarP._2023a, Emamhadi_2018, Farhadi_2024h, Fry_2010, Gardner_2017h, Goldman_1998f, Guinan_2019f, Hardy_2023g, Jehangir_2019h, Jin_2023, Kar_2015, Kariholu_2008, Kerestes_2019, Kobiela_2015, Kumar_2001, Kumar_2019f, Li_2013, Liu_2005, Losanoff_1996, Losanoff_1997e, Mesfin_2022a, Misra_2013, Naji_2012f, Ohno_2005, Peixoto_2017f, Qureshi_2016, Riva_2018j, Sakellaridis_2008f, Sobnach_2011f, Sultan_2024f, Tammana_2012j, Tanrikulu_2015e, Tay_2004, Thapa_2019f, Trgo_2012f, Tupesis_2004f, Wadhwa_2015e, Wildhaber_2005, Wnęk_2015f, Yasin_2009, Yildiz_2016e, fjbuilsRepeatedBehaviorDeliberate2024, teWildt_2010}. Most clearly described patient history and timeline (71 cases, 99\%) \cite{Akay_2015f, Al-Faham_2020k, AlShaaibi_2021b, Alao_2006i, Ali_2017, Ali_2020f, Ali_2022g, Apikotoa_2022f, Ataya_2013, Atayan_2016, Beecroft_1998, Benoist_2019e, Berry_2021e, Bhasin_2014, Bhattacharjee_2008, Bhumi_2024f, Cauchi_2002, Chang_2017f, Cox_2007, Csaky_1998e, DelgadoSalazar_2020c, DivsalarP._2023a, Emamhadi_2018, Farhadi_2024h, Fry_2010, Gardner_2017h, Goldman_1998f, Guinan_2019f, Hardy_2023g, Jehangir_2019h, Jin_2023, Kar_2015, Kariholu_2008, Kerestes_2019, Kobiela_2015, Kumar_2001, Kumar_2019f, Li_2013, Liu_2005, Losanoff_1996, Losanoff_1997e, Mesfin_2022a, Misra_2013, Naji_2012f, Ohno_2005, Peixoto_2017f, Qureshi_2016, Riva_2018j, Sakellaridis_2008f, Sobnach_2011f, Sultan_2024f, Tammana_2012j, Tanrikulu_2015e, Tay_2004, Thapa_2019f, Trgo_2012f, Tupesis_2004f, Wadhwa_2015e, Wildhaber_2005, Wnęk_2015f, Yasin_2009, Yildiz_2016e, fjbuilsRepeatedBehaviorDeliberate2024, teWildt_2010}, patient post-intervention condition (70 cases, 97\%) \cite{Akay_2015f, Al-Faham_2020k, AlShaaibi_2021b, Alao_2006i, Ali_2017, Ali_2020f, Ali_2022g, Apikotoa_2022f, Ataya_2013, Atayan_2016, Beecroft_1998, Benoist_2019e, Berry_2021e, Bhasin_2014, Bhattacharjee_2008, Bhumi_2024f, CamachoDorado_2018, Cauchi_2002, Chang_2017f, Cox_2007, Csaky_1998e, DelgadoSalazar_2020c, DivsalarP._2023a, Emamhadi_2018, Farhadi_2024h, Fry_2010, Gardner_2017h, Goldman_1998f, Guinan_2019f, Hardy_2023g, Jehangir_2019h, Jin_2023, Kar_2015, Kariholu_2008, Kerestes_2019, Kobiela_2015, Kumar_2001, Li_2013, Losanoff_1996, Losanoff_1997e, Mesfin_2022a, Misra_2013, Naji_2012f, Ohno_2005, Peixoto_2017f, Qureshi_2016, Riva_2018j, Sakellaridis_2008f, Sobnach_2011f, Sultan_2024f, Tammana_2012j, Tanrikulu_2015e, Tay_2004, Thapa_2019f, Trgo_2012f, Tupesis_2004f, Wadhwa_2015e, Wildhaber_2005, Wnęk_2015f, Yasin_2009, Yildiz_2016e, fjbuilsRepeatedBehaviorDeliberate2024, teWildt_2010}, takeaway lessons (70 cases, 97\%) \cite{Akay_2015f, Al-Faham_2020k, AlShaaibi_2021b, Alao_2006i, Ali_2017, Ali_2020f, Ali_2022g, Apikotoa_2022f, Ataya_2013, Atayan_2016, Beecroft_1998, Benoist_2019e, Berry_2021e, Bhasin_2014, Bhattacharjee_2008, Bhumi_2024f, Cauchi_2002, Cox_2007, Csaky_1998e, DelgadoSalazar_2020c, DivsalarP._2023a, Emamhadi_2018, Farhadi_2024h, Fry_2010, Gardner_2017h, Goldman_1998f, Guinan_2019f, Hardy_2023g, Jehangir_2019h, Jin_2023, Kar_2015, Kariholu_2008, Kerestes_2019, Kobiela_2015, Kumar_2001, Kumar_2019f, Li_2013, Liu_2005, Losanoff_1996, Losanoff_1997e, Mesfin_2022a, Misra_2013, Naji_2012f, Ohno_2005, Peixoto_2017f, Qureshi_2016, Riva_2018j, Sakellaridis_2008f, Sobnach_2011f, Sultan_2024f, Tammana_2012j, Tanrikulu_2015e, Tay_2004, Thapa_2019f, Trgo_2012f, Tupesis_2004f, Wadhwa_2015e, Wildhaber_2005, Wnęk_2015f, Yasin_2009, Yildiz_2016e, fjbuilsRepeatedBehaviorDeliberate2024, teWildt_2010}, patient demographic (69 cases, 96\%) \cite{Akay_2015f, Al-Faham_2020k, AlShaaibi_2021b, Alao_2006i, Ali_2017, Ali_2020f, Ali_2022g, Apikotoa_2022f, Ataya_2013, Atayan_2016, Beecroft_1998, Berry_2021e, Bhasin_2014, Bhattacharjee_2008, Bhumi_2024f, CamachoDorado_2018, Cauchi_2002, Chang_2017f, Cox_2007, Csaky_1998e, DelgadoSalazar_2020c, DivsalarP._2023a, Emamhadi_2018, Farhadi_2024h, Fry_2010, Gardner_2017h, Goldman_1998f, Guinan_2019f, Hardy_2023g, Jehangir_2019h, Jin_2023, Kar_2015, Kariholu_2008, Kerestes_2019, Kobiela_2015, Kumar_2001, Kumar_2019f, Li_2013, Liu_2005, Losanoff_1996, Mesfin_2022a, Misra_2013, Naji_2012f, Ohno_2005, Peixoto_2017f, Qureshi_2016, Riva_2018j, Sakellaridis_2008f, Sobnach_2011f, Sultan_2024f, Tammana_2012j, Tanrikulu_2015e, Tay_2004, Thapa_2019f, Trgo_2012f, Tupesis_2004f, Wadhwa_2015e, Wildhaber_2005, Yasin_2009, Yildiz_2016e, fjbuilsRepeatedBehaviorDeliberate2024, teWildt_2010}, and current patient condition (69 cases, 96\%) \cite{Akay_2015f, Al-Faham_2020k, AlShaaibi_2021b, Alao_2006i, Ali_2017, Ali_2020f, Ali_2022g, Apikotoa_2022f, Ataya_2013, Atayan_2016, Beecroft_1998, Benoist_2019e, Berry_2021e, Bhasin_2014, Bhattacharjee_2008, Bhumi_2024f, CamachoDorado_2018, Cauchi_2002, Chang_2017f, Cox_2007, Csaky_1998e, DelgadoSalazar_2020c, Emamhadi_2018, Farhadi_2024h, Fry_2010, Gardner_2017h, Goldman_1998f, Guinan_2019f, Hardy_2023g, Jehangir_2019h, Jin_2023, Kar_2015, Kariholu_2008, Kerestes_2019, Kobiela_2015, Kumar_2001, Kumar_2019f, Li_2013, Liu_2005, Losanoff_1996, Losanoff_1997e, Mesfin_2022a, Misra_2013, Naji_2012f, Ohno_2005, Peixoto_2017f, Qureshi_2016, Riva_2018j, Sakellaridis_2008f, Sobnach_2011f, Sultan_2024f, Tammana_2012j, Tanrikulu_2015e, Tay_2004, Thapa_2019f, Trgo_2012f, Tupesis_2004f, Wadhwa_2015e, Wildhaber_2005, Wnęk_2015f, Yasin_2009, Yildiz_2016e, fjbuilsRepeatedBehaviorDeliberate2024, teWildt_2010}. Reporting was also strong for diagnostic assessments (66 cases, 92\%) \cite{Akay_2015f, Al-Faham_2020k, AlShaaibi_2021b, Alao_2006i, Ali_2017, Ali_2020f, Ali_2022g, Apikotoa_2022f, Ataya_2013, Atayan_2016, Beecroft_1998, Benoist_2019e, Berry_2021e, Bhasin_2014, Bhattacharjee_2008, Bhumi_2024f, Cauchi_2002, Chang_2017f, Cox_2007, Csaky_1998e, DelgadoSalazar_2020c, Emamhadi_2018, Farhadi_2024h, Fry_2010, Gardner_2017h, Goldman_1998f, Hardy_2023g, Jehangir_2019h, Jin_2023, Kar_2015, Kariholu_2008, Kerestes_2019, Kobiela_2015, Kumar_2001, Kumar_2019f, Li_2013, Liu_2005, Losanoff_1996, Losanoff_1997e, Mesfin_2022a, Misra_2013, Naji_2012f, Ohno_2005, Peixoto_2017f, Qureshi_2016, Riva_2018j, Sakellaridis_2008f, Sobnach_2011f, Sultan_2024f, Tammana_2012j, Tanrikulu_2015e, Tay_2004, Thapa_2019f, Trgo_2012f, Tupesis_2004f, Wadhwa_2015e, Wildhaber_2005, Wnęk_2015f, Yasin_2009, Yildiz_2016e, fjbuilsRepeatedBehaviorDeliberate2024, teWildt_2010}, and harms (38 cases, 90\%) \cite{Ali_2017, Ali_2020f, Beecroft_1998, Benoist_2019e, Berry_2021e, Bhasin_2014, Bhattacharjee_2008, Bhumi_2024f, Cauchi_2002, Cox_2007, Emamhadi_2018, Fry_2010, Gardner_2017h, Kar_2015, Kariholu_2008, Kobiela_2015, Kumar_2001, Liu_2005, Losanoff_1996, Misra_2013, Naji_2012f, Ohno_2005, Sultan_2024f, Tammana_2012j, Tanrikulu_2015e, Tay_2004, Thapa_2019f, Trgo_2012f, Wadhwa_2015e, Wildhaber_2005, Wnęk_2015f, Yildiz_2016e, fjbuilsRepeatedBehaviorDeliberate2024}.
\subsubsection*{Case Series}
Seperately, 3 studies \cite{Elghali_2016, Karp_1991b, Lee_2007} were evaluated using the \textit{JBI Checklist for Case Series} \cite{Moola_2020}.
Reporting quality was generally high across all JBI domains. All included case series fully reported clear inclusion criteria, standard condition measurements, valid patient identification methods, complete inclusion, clear demographic information, clear clinical information, clear outcome and follow-up, and approprate statistical analysis \cite{Elghali_2016,Karp_1991b,Lee_2007}. However, fewer studies (2) reported consecutive inclusion, and clear site demographic information \cite{Elghali_2016,Karp_1991b}.



\subsection*{Study Characteristics}
A total of 68 studies were included in the synthesis. 65 studies were case reports, yielding 72 cases \cite{Akay_2015f, Al-Faham_2020k, AlShaaibi_2021b, Alao_2006i, Ali_2017, Ali_2020f, Ali_2022g, Apikotoa_2022f, Ataya_2013, Atayan_2016, Beecroft_1998, Benoist_2019e, Berry_2021e, Bhasin_2014, Bhattacharjee_2008, Bhumi_2024f, CamachoDorado_2018, Cauchi_2002, Chang_2017f, Cox_2007, Csaky_1998e, DelgadoSalazar_2020c, DivsalarP._2023a, Emamhadi_2018, Farhadi_2024h, Fry_2010, Gardner_2017h, Goldman_1998f, Guinan_2019f, Hardy_2023g, Jehangir_2019h, Jin_2023, Kar_2015, Kariholu_2008, Kerestes_2019, Kobiela_2015, Kumar_2001, Kumar_2019f, Li_2013, Liu_2005, Losanoff_1996, Losanoff_1997e, Mesfin_2022a, Misra_2013, Naji_2012f, Ohno_2005, Peixoto_2017f, Qureshi_2016, Riva_2018j, Sakellaridis_2008f, Sobnach_2011f, Sultan_2024f, Tammana_2012j, Tanrikulu_2015e, Tay_2004, Thapa_2019f, Trgo_2012f, Tupesis_2004f, Wadhwa_2015e, Wildhaber_2005, Wnęk_2015f, Yasin_2009, Yildiz_2016e, fjbuilsRepeatedBehaviorDeliberate2024, teWildt_2010}. A full list of group summary case-level characteristics and outcomes is available in Table~\ref{tab:grouped_summary_wide}. 
3 studies were case series, yielding 90 cases \cite{Elghali_2016, Karp_1991b, Lee_2007}. A full list of grouped series-level characteristics and outcomes is available in Table~\ref{tab:series_results}.
Geographically, cases of ingestion (case reports and case series) were reported in 35 unique countries, most being from South Korea (n=52, 32.1\%), United States of America (n=32, 19.8\%), and Tunisia (n=19, 11.7\%). A full breakdown of cases per country is shown in Table~\ref{tab:case_country_counts} and Figure~\ref{fig:country_heat_map}. 



\begin{table}[!t]
\caption{Case counts per country.}
\label{tab:case_country_counts}
\centering
\begin{adjustbox}{width=\columnwidth}
\renewcommand{\arraystretch}{1.2}
\begin{tabularx}{\columnwidth}{p{3.5cm}>{\centering\arraybackslash}X>{\centering\arraybackslash}X>{\centering\arraybackslash}X>{\centering\arraybackslash}X}
\toprule
Country & Case Count & Percentage & References \\
\midrule
South Korea & 52 & 32 & \cite{Lee_2007} \\
United States of America & 32 & 20 & \cite{Alao_2006i, Ataya_2013, Bhumi_2024f, Fry_2010, Guinan_2019f, Hardy_2023g, Jehangir_2019h, Karp_1991b, Kerestes_2019, Kumar_2001, Liu_2005, Tammana_2012j, Tay_2004, Tupesis_2004f} \\
Tunisia & 19 & 12 & \cite{Elghali_2016} \\
United Kingdom & 7 & 4 & \cite{Beecroft_1998, Berry_2021e, Cauchi_2002, Cox_2007, Gardner_2017h, Qureshi_2016} \\
India & 7 & 4 & \cite{Bhasin_2014, Bhattacharjee_2008, Kar_2015, Kariholu_2008, Kumar_2019f, Misra_2013, Wadhwa_2015e} \\
Bulgaria & 6 & 4 & \cite{Losanoff_1996, Losanoff_1997e} \\
Iran & 5 & 3 & \cite{DivsalarP._2023a, Emamhadi_2018, Farhadi_2024h} \\
Turkey & 4 & 2 & \cite{Akay_2015f, Atayan_2016, Tanrikulu_2015e, Yildiz_2016e} \\
Spain & 2 & 1 & \cite{CamachoDorado_2018, fjbuilsRepeatedBehaviorDeliberate2024} \\
Poland & 2 & 1 & \cite{Kobiela_2015, Wnęk_2015f} \\
China & 2 & 1 & \cite{Jin_2023, Li_2013} \\
Pakistan & 1 & 1 & \cite{Yasin_2009} \\
Switzerland & 1 & 1 & \cite{Wildhaber_2005} \\
Taiwan & 1 & 1 & \cite{Chang_2017f} \\
South Africa & 1 & 1 & \cite{Sobnach_2011f} \\
Saudi Arabia & 1 & 1 & \cite{Sultan_2024f} \\
Qatar & 1 & 1 & \cite{Ali_2017} \\
Portugal & 1 & 1 & \cite{Peixoto_2017f} \\
Sweden & 1 & 1 & \cite{Naji_2012f} \\
Australia & 1 & 1 & \cite{Apikotoa_2022f} \\
Oman & 1 & 1 & \cite{AlShaaibi_2021b} \\
Netherlands & 1 & 1 & \cite{Benoist_2019e} \\
United Arab Emirates & 1 & 1 & \cite{Ali_2020f} \\
Japan & 1 & 1 & \cite{Ohno_2005} \\
Italy & 1 & 1 & \cite{Riva_2018j} \\
Israel & 1 & 1 & \cite{Goldman_1998f} \\
Iraq & 1 & 1 & \cite{Al-Faham_2020k} \\
Hungary & 1 & 1 & \cite{Csaky_1998e} \\
Greece & 1 & 1 & \cite{Sakellaridis_2008f} \\
Germany & 1 & 1 & \cite{teWildt_2010} \\
Ethiopia & 1 & 1 & \cite{Mesfin_2022a} \\
Egypt & 1 & 1 & \cite{Ali_2022g} \\
Ecuador & 1 & 1 & \cite{DelgadoSalazar_2020c} \\
Croatia & 1 & 1 & \cite{Trgo_2012f} \\
Nepal & 1 & 1 & \cite{Thapa_2019f} \\
\bottomrule
\end{tabularx}

\end{adjustbox}
\end{table}


\begin{table*}[!t]
\caption{Grouped univariate meta-regression results for series-level data.}
\label{tab:univariate_meta_regression}
\centering
\begin{adjustbox}{width=\textwidth}
\renewcommand{\arraystretch}{1.2}
\small
\begin{tabularx}{\textwidth}{p{4cm}>{\centering\arraybackslash}X>{\centering\arraybackslash}X>{\centering\arraybackslash}X>{\centering\arraybackslash}X>{\centering\arraybackslash}X}
\toprule
\textbf{Variable} & \textbf{Conservative} & \textbf{Endoscopy} & \textbf{Surgery} & \textbf{Death} & \textbf{Complication} \\
\midrule

\multicolumn{6}{l}{\textit{Gender}} \\
\hspace{1em}Female & 0.96 [0.88, 1.04] (p=0.201) & 0.98 [0.91, 1.06] (p=0.529) & \textbf{1.05 [1.00, 1.10] (p=0.044)*} & 1.00 [0.89, 1.11] (p=0.904) & 1.08 [0.91, 1.28] (p=0.202) \\
\hspace{1em}Unknown & 0.31 [0.03, 3.08] (p=0.201) & 0.61 [0.07, 5.45] (p=0.529) & \textbf{3.81 [1.06, 13.68] (p=0.044)*} & 0.91 [0.04, 20.06] (p=0.904) & 8.00 [0.07, 954.88] (p=0.202) \\
\hspace{1em}Male & 0.99 [0.96, 1.02] (p=0.398) & 1.00 [0.97, 1.03] (p=0.916) & 1.00 [0.97, 1.02] (p=0.881) & 0.99 [0.97, 1.01] (p=0.277) & 0.99 [0.93, 1.06] (p=0.679) \\

\multicolumn{6}{l}{\textit{Demographic}} \\
\hspace{1em}Psychiatric Inpatient & 0.74 [0.42, 1.32] (p=0.201) & 0.89 [0.51, 1.53] (p=0.529) & \textbf{1.40 [1.01, 1.92] (p=0.044)*} & 0.98 [0.45, 2.12] (p=0.904) & 1.68 [0.51, 5.56] (p=0.202) \\
\hspace{1em}Severely Disabled & 0.91 [0.29, 2.83] (p=0.470) & 0.92 [0.61, 1.39] (p=0.466) & 1.24 [0.81, 1.89] (p=0.161) & 1.10 [0.45, 2.71] (p=0.399) & 1.35 [0.68, 2.67] (p=0.202) \\
\hspace{1em}Detained Person & 0.99 [0.95, 1.04] (p=0.660) & 1.00 [0.97, 1.03] (p=0.890) & 0.99 [0.97, 1.02] (p=0.442) & 0.99 [0.97, 1.01] (p=0.244) & 0.99 [0.92, 1.05] (p=0.419) \\
\hspace{1em}Psychiatric History & 0.97 [0.92, 1.03] (p=0.258) & 0.99 [0.94, 1.05] (p=0.606) & 1.01 [0.97, 1.06] (p=0.463) & 0.99 [0.94, 1.04] (p=0.415) & 1.01 [0.86, 1.18] (p=0.835) \\
\hspace{1em}Previous Ingestor & 0.95 [0.79, 1.14] (p=0.364) & 1.00 [0.88, 1.12] (p=0.879) & 1.07 [0.89, 1.29] (p=0.274) & 0.99 [0.84, 1.17] (p=0.867) & 1.11 [0.22, 5.74] (p=0.556) \\
\hspace{1em}Displaced Person & 0.71 [0.01, 37.98] (p=0.470) & 0.89 [0.29, 2.79] (p=0.427) & 1.98 [0.00, 1949.23] (p=0.427) & 1.41 [0.06, 32.70] (p=0.399) & 3.27 [0.00, 1502880.33] (p=0.454) \\
\hspace{1em}Under Influence of Alcohol & 0.79 [0.06, 11.30] (p=0.470) & 0.93 [0.43, 1.98] (p=0.427) & 1.58 [0.02, 156.04] (p=0.427) & 1.26 [0.15, 10.22] (p=0.399) & 2.21 [0.00, 13120.48] (p=0.454) \\

\multicolumn{6}{l}{\textit{Motivation}} \\
\hspace{1em}Psychosocial & 0.93 [0.81, 1.07] (p=0.201) & 0.97 [0.85, 1.10] (p=0.529) & \textbf{1.08 [1.00, 1.17] (p=0.044)*} & 0.99 [0.83, 1.19] (p=0.904) & 1.13 [0.85, 1.50] (p=0.202) \\
\hspace{1em}Other & 0.88 [0.68, 1.13] (p=0.201) & 0.95 [0.74, 1.21] (p=0.529) & \textbf{1.16 [1.01, 1.34] (p=0.044)*} & 0.99 [0.70, 1.40] (p=0.904) & 1.26 [0.74, 2.14] (p=0.202) \\
\hspace{1em}Intent to harm & 0.95 [0.85, 1.06] (p=0.214) & 0.98 [0.88, 1.08] (p=0.486) & 1.05 [0.98, 1.13] (p=0.130) & 0.99 [0.86, 1.13] (p=0.710) & 1.07 [0.81, 1.41] (p=0.430) \\
\hspace{1em}Psychiatric & 0.97 [0.90, 1.04] (p=0.223) & 0.99 [0.93, 1.05] (p=0.498) & 1.03 [0.98, 1.08] (p=0.159) & 0.99 [0.92, 1.07] (p=0.625) & 1.04 [0.87, 1.23] (p=0.484) \\
\hspace{1em}Protest & 0.99 [0.94, 1.05] (p=0.620) & 1.00 [0.96, 1.04] (p=0.800) & 0.99 [0.96, 1.02] (p=0.404) & 0.99 [0.96, 1.02] (p=0.255) & 0.98 [0.91, 1.06] (p=0.369) \\

\multicolumn{6}{l}{\textit{Object}} \\
\hspace{1em}Button Battery & 0.61 [0.12, 3.26] (p=0.336) & 0.78 [0.26, 2.33] (p=0.529) & 2.05 [0.87, 4.81] (p=0.076) & 0.95 [0.20, 4.48] (p=0.904) & 2.83 [0.26, 30.90] (p=0.202) \\
\hspace{1em}Magnet & 0.90 [0.62, 1.30] (p=0.336) & 0.95 [0.74, 1.21] (p=0.529) & 1.17 [0.97, 1.42] (p=0.076) & 0.99 [0.70, 1.40] (p=0.904) & 1.26 [0.74, 2.14] (p=0.202) \\
\hspace{1em}Sharp & 0.99 [0.94, 1.03] (p=0.406) & 1.00 [0.96, 1.04] (p=0.847) & 1.00 [0.97, 1.03] (p=0.935) & 0.99 [0.96, 1.02] (p=0.287) & 0.99 [0.90, 1.09] (p=0.726) \\
\hspace{1em}Long (\textgreater{}5cm) & 0.98 [0.89, 1.08] (p=0.456) & 1.00 [0.93, 1.07] (p=0.943) & 1.01 [0.93, 1.09] (p=0.802) & 0.99 [0.94, 1.04] (p=0.357) & 1.00 [0.82, 1.20] (p=0.922) \\
\hspace{1em}Multiple & 0.98 [0.91, 1.06] (p=0.382) & 0.99 [0.94, 1.05] (p=0.734) & 1.02 [0.96, 1.08] (p=0.444) & 0.99 [0.94, 1.04] (p=0.510) & 1.02 [0.86, 1.20] (p=0.719) \\
\hspace{1em}Large (\textgreater{}2.5cm) Diameter & 0.99 [0.84, 1.15] (p=0.470) & 1.00 [0.95, 1.04] (p=0.427) & 1.03 [0.78, 1.35] (p=0.427) & 1.01 [0.90, 1.15] (p=0.399) & 1.05 [0.63, 1.75] (p=0.454) \\
\multicolumn{6}{l}{\small OR: Odds Ratio; CI: Confidence Interval; p: p-value. * indicates $p < 0.05$. Bold = statistically significant. --- = missing or unstable estimate.} \\\\
\bottomrule
\end{tabularx}
\end{adjustbox}
\end{table*}

\subsubsection*{Case Studies}\\A total of 81 cases from 69 studies were included in the case synthesis. The mean age of this 29.89 (range: 4.0-100.0)
64.2\% were male gender (52); 34.6\% were female gender (28); 1.2\% were unknown gender (1). Cases are recorded in a variety of countries: 20 cases from USA; 10 cases from UK; 7 cases from India; 6 cases from Bulgaria; 5 cases from Iran; 4 cases from Turkey; 2 cases from China, Poland, and Spain; 1 case from Germany, Pakistan, Switzerland, Ecuador, Japan, Qatar, Saudi Arabia, Egypt, Australia, Netherlands, Bahrain, Iraq, Israel, Nepal, Italy, Taiwan, Portugal, Sweden, Croatia, South Africa, Greece, Hungary, and Ethiopia. 42 cases (42.4\%) had a psychiatric history documented; 25 cases (25.3\%) had previous a history of prior ingestion; 13 cases (13.1\%) were detained at the time of ingestion; 9 cases (9.1\%) had a severe disability history; 5 cases (5.1\%) were psychiatric inpatients; 3 cases (3.0\%) were under the influence of alchol at the time of ingestion; 2 cases (2.0\%) were displaced people. 



\begin{figure*}[!t]
\centering
\includegraphics[width=0.95\linewidth]{figures/nested_univariate_subgroup_plots.png}
\caption{Nested bar plots showing percentage differences in population characteristics between key subgroups and the comparison group. Each subplot visualises the relative increase or decrease of specific variables within a given subgroup, highlighting significant deviations (\( p < 0.05 \)) from the broader case report population.}
\label{fig:nested_subgroup_plots}
\end{figure*}
\subsection*{Synthesis}

Across 72 single‐patient case reports and three case series (total $n=162$) we extracted uniform data on patient motive, object features, treatment approach and outcome.

Random–effects pooling of the three case series yielded a conservative–management rate of 40.6 \% (95 \% CI 29.2–52.8; $I^{2}\!\approx\!75$), an endoscopy rate of 41.6 \% (29.5–54.7; $I^{2}\!\approx\!95$) and a surgical rate of 17.8 \% (12.0–25.8; $I^{2}\!\approx\!17$). Overall complications occurred in 7.3 \% of patients (2.9–17.3) and there were three reported deaths.

\paragraph*{Motivation}
Series-level meta-regression confirmed motivation as the dominant predictor of management.  
\begin{itemize}
    \item \textbf{Intentional self-harm} lowered the probability of conservative care by 17 \% (aOR 0.83, 95 \% CI 0.79–0.86, $p=0.012$) and raised surgical use by 11 \% (aOR 1.11, 1.05–1.18, $p=0.015$); endoscopy was unaffected (aOR 0.92, 0.27–3.13, $p=0.559$). Case-level logistic modelling amplified this signal, showing a 15-fold increase in surgery (aOR 15.97, 1.32–192.40, $p=0.029$).  
    \item \textbf{Psychiatric (non-self-harm)} motivation produced a smaller but significant rise in surgical intervention (aOR 1.07, 1.03–1.11, $p=0.017$) with no change in other pathways; the effect disappeared in multivariable analyses.  
    \item \textbf{Psychosocial} motives behaved similarly, decreasing conservative management (aOR 0.82, 0.70–0.96) and increasing surgery (aOR 1.12, 1.01–1.25), both $p=0.040$.  
    \item \textbf{Protest} ingestions—almost exclusively in custodial settings—did not alter the odds of any single treatment modality in the series-level model, though case-level analysis linked protest with an extreme rise in complications (aOR 290.44, 1.40–60 043.40).
\end{itemize}

\paragraph*{Object}
\begin{itemize}
    \item \textbf{Sharp objects} significantly reduced conservative management by 19 \% (aOR 0.81, 0.68–0.98, $p=0.044$) but had no independent effect on surgery or endoscopy; numbers were too small for stable multivariable estimates.  
    \item \textbf{Multiple objects} and \textbf{length $>$ 5 cm} were neutral in univariate meta-regression, yet individual-patient models linked long objects to a seven-fold rise in surgery and a twenty-fold rise in complications, while multiple-object ingestion drove an eighteen-fold increase in complications.
\end{itemize}

\paragraph*{Demographics and setting}
Male sex curtailed conservative treatment by 13 \% (OR 0.87, 0.78–0.97, $p=0.039$) without influencing other modalities. Detention status increased conservative management by 62 \% (OR 1.62, 1.11–2.38, $p=0.039$) and, in case-level analysis, was associated with a 97 \% reduction in recorded complications (aOR 0.03, 0.00–0.59). A binary history of psychiatric illness showed no significant association with any outcome in series-level modelling.

\paragraph*{Certainty and heterogeneity}
All included studies were uncontrolled case reports or case series. High between-study heterogeneity for several endpoints ($I^{2}$ up to 95 \%), shifting definitions of motive and object descriptors, and limited follow-up constrain the certainty of pooled estimates. No GRADE assessment was performed.

\paragraph*{Data availability}
All extraction scripts and raw data are openly hosted at \url{https://github.com/jackgedge/iifo_systematic_review}.\paragraph*{Investigation of Heterogeneity}

Heterogeneity was explored across study results using two complementary approaches.

Firstly, meta-analyses of proportions were conducted for each outcome of interest across included case series. Between-study heterogeneity was quantified using the $I^2$ statistic and $\tau^2$ variance. The degree of heterogeneity varied substantially across outcomes:
\begin{itemize}
\item Substantial heterogeneity ($I^2 = 95\%$) was observed for endoscopy.
\item Heterogeneity was moderate for surgery ($I^2 = 17\%$).
\item Heterogeneity was low ($I^2 = 0\%$) for conservative management.
\item Meta-analyses could not be meaningfully conducted for death and complications due to insufficient data.
\end{itemize}

Secondly, to investigate potential sources of heterogeneity, univariate meta-regression was performed for each outcome. Given the small number of case series available, series-level data was combined with collapsed aggregate case reports to increase the number of contributing studies. Meta-regressions examined associations between outcome proportions and patient-level predictors including gender, population characteristics, ingestion motivations, and object type.

Significant associations between certain predictors and outcomes were identified, particularly in relation to motivation and surgery or conservative management outcomes. This suggests that clinical heterogeneity in patient characteristics likely contributed to observed between-study differences.

No additional subgroup or sensitivity analyses were performed, given the limited number of studies and inconsistent reporting of potential study-level moderators. \paragraph*{Sensitivity Analyses}

Formal sensitivity analyses were not performed due to the limited number of included case series and the small number of studies available for several outcomes. Given the small number of eligible series and heterogeneity in reporting, the meta-analyses of proportions and meta-regressions were considered primarily descriptive in nature.

To partially address robustness, aggregate case report data was incorporated into the meta-regression models to increase the effective number of series contributing to each outcome. This allowed exploratory assessment of predictor-outcome relationships across a larger pooled dataset. However, no additional sensitivity analyses (e.g. leave-one-out analyses, exclusion of small studies, or alternative meta-analytic models) were conducted, as such analyses would not have been statistically meaningful in the context of the available data. 



\subsection*{Assessment of Reporting Bias}

Risk of bias due to missing results (arising from reporting biases) was not formally assessed through funnel plot inspection or quantitative methods such as Egger’s test, as the number of included case series per outcome was too small to support these analyses (fewer than 10 series per outcome). Furthermore, case series are prone to selective reporting and variable outcome definitions, which may contribute to reporting bias; however, the heterogeneity in reporting precluded a more formal assessment.

For the meta-regressions, the inclusion of aggregate case report data partially mitigated the risk of missing results at the series level but could not address potential reporting biases within the individual case reports or across studies. Overall, the potential for reporting bias remains a limitation of the syntheses presented in this review and should be considered when interpreting results. 
\subsection*{Certainty of Evidence}
A formal GRADE assessment was not applicable due to the descriptive nature of the evidence, but risk of bias was explored using the \textit{JBI critical appraisal checklists for case reports and case series} \cite{Moola_2020}.


\begin{figure}[t]
\centering
\includegraphics[width=\columnwidth]{figures/reml_meta_summary_plot.png}
\caption{Meta-analyses of proportions using random-effects models with restricted maximum likelihood (REML) estimation and Hartung-Knapp (HK) adjustments for confidence intervals.}
\label{fig:reml_meta}
\end{figure}

% \begin{figure}[b]
%\centering
%\includegraphics[width=\textwidth]{figures/series_pies.png}
%\caption{Pie charts showing distribution of variables for all ingestions.}
%\label{fig:variable_pie_charts}
%\end{figure}


\section*{Discussion}
\subsection*{Interpretation of Results}


\subsubsection*{Strengths}
The present review represents, to the author's knowledge, this is the first review of IIFO that focuses on motivations for ingestion, aggregating evidence from 162 ingestion cases with documented motivations.

The main strength of this review lies in its methodology. Where motivation was unclear, cases were excluded—ensuring that the effects measured were genuinely reflective of intentionality. This strict inclusion criterion helps isolate intentional ingestion as an independent clinical phenomenon. Additionally, by avoiding age restrictions, this review includes a wide age range (from 7 to 100 years), improving the generalisability of its findings across the lifespan.

A broad search strategy was implemented, covering multiple databases, bibliographies, and grey literature sources. This increases the likelihood that the findings reflect a wide cross-section of the literature and reduces the impact of publication bias due to indexing alone.

Despite substantial heterogeneity in the review population, outcome proportions were pooled using random-effects meta-analysis with restricted maximum likelihood estimation and Hartung–Knapp adjustments—techniques appropriate for small numbers of studies and high between-study variance. Following IIFO, patients underwent endoscopy in 47.3\% of cases (95\% CI 4.3\%--94.7\%), surgery in 30.6\% (95\% CI 12.0\%--58.9\%), and conservative management in 41.6\% (95\% CI 4.4\%--91.7\%). Complications occurred in 34.7\% of cases (95\% CI 1.8\%--93.8\%), and 2.5\% of patients died (95\% CI 0.8\%--7.7\%). These figures are substantially higher than previously reported in the literature~\cite{Ikenberry_2011, Birk_2016}. 

Given the methodological rigour of this review---including strict exclusion of non-intentional cases---and despite the wide uncertainty intervals, the findings strongly suggest that \emph{intentionality alone} is a key driver of morbidity and mortality following foreign body ingestion.

\subsubsection*{Associations Between Motivation and Outcome}
While the univariate meta-regression (predominantly conducted on detained males) didn’t show statistically significant correlations between motivations and outcomes, signals did emerge in the univariate association testing conducted on the more varied population represented in the case reports. This population was more diverse—39\% female, 17\% detained individuals, and 50\% with a psychiatric history—and included a wider range of object types (e.g., magnets, multiple items, long objects). 

In this context, \enquote{intent to harm} was associated with a fivefold increase in the odds of surgery, whereas “other” motivations were associated with an 86\% decrease in the odds of surgery. These findings suggest that motivations likely influence clinical outcomes, particularly in more heterogeneous and less institutionalised populations.

\subsubsection*{Subgroup Findings}
Interestingly, among severely disabled people (defined here as individuals with reduced levels of consciousness or severe learning disabilities), univariate meta-regression showed a 41\% decrease in the odds of death. While this cohort was small (\emph{n}=7), and all cases were published as case reports, this may reflect publication bias—authors may be more likely to report complex or anatomically challenging cases \cite{Ken_1993}. However, it might also reflect the diligence of carers, earlier access to intervention, and perhaps the success of public health campaigns aimed at recognising and addressing disproportionate harms in this population.

Also noteworthy, displaced individuals had reduced odds of death (OR 0.22, 95\% CI 0.08–0.61, \emph{p} = 0.034) in the meta-regression. This is unlikely to be solely due to their displacement status. The two cases underpinning this result \cite{Akay_2015f, Gardner_2017} are strikingly different. One involved a person who swallowed a pen in protest of an asylum decision and developed a double duodenal perforation, requiring surgery; the other involved a man who swallowed \$1000 wrapped in plastic for safe-keeping, who then had it endoscopically removed and returned to him. These examples illustrate well how both motivation and object characteristics can drive clinical outcomes—and why both should be considered in future analysis.

\subsection*{Limitations of the Review Process} 
Several limitations of the review process should be acknowledged.

\subsubsection*{Selection and Screening}
Many studies were excluded because intentionality was not clearly stated. Given the nuanced and inconsistent language often used in clinical reporting, some relevant cases may have been omitted. This likely introduced selection bias, especially in a single-author review, where subjective judgment plays a larger role.

Although the search strategy was broad, relevant studies may still have been missed—particularly unpublished material or reports not indexed in major databases. This is a recognised limitation in rare-event research.

Screening and data extraction were carried out by a single reviewer. While detailed protocols and reproducible scripts were used to minimise error, the absence of a second full-text reviewer and data extracter increases the risk of human error and subjective bias, particularly when interpreting ambiguous descriptions of intent and motivation. Manually reviewing over 480 full texts and extracting detailed case-level data from almost half is a substantial undertaking and a recognised limitation.

\subsubsection*{Motivation Classification}
The motivation categories were author-defined and applied post hoc following preliminary analysis. While this was necessary due to the lack of standardised classifications, it introduces classification bias, especially in a field as complex and subjective as IIFO. Motivational constructs such as \enquote{intent to harm}, \enquote{protest}, or \enquote{other} may overlap or be poorly described in reports, affecting how cases are grouped and interpreted.

\subsubsection*{Statistical Limitations}
The analytical approach may have been overly complex given the volume and quality of the available literature. Conducting regression on 25 variables across only four case series risks overfitting, reducing the reliability of these models. A more parsimonious strategy might have involved collapsing object variables into a binary \enquote{high-risk vs low-risk} classification, informed by existing literature on object characteristics and classification~\cite{Ikenberry_2011, Birk_2016}.

Similarly, psychiatric history was treated as a binary variable, but as noted in the introduction, psychiatric presentations associated with IIFO are varied and complex. The binary coding limits the ability to identify nuanced associations. A more granular analysis of psychiatric subtypes could provide deeper insight but would further complicate models already stretched by sample size constraints. 

\subsubsection*{Other Gaps}
This review did not examine the types or effectiveness of conservative management strategies, despite their prevalence and clinical relevance. Nor did it consider object location, which is known to be a critical determinant of clinical risk. Both factors are likely to affect outcomes and should be included in future studies.

It is possible that collapsing the case report data into a series introduced heterogeneity into the series level data. In the series', all cases were males and most were detained individuals.

The psychiatric motivation component may have been underestimated. In psychiatric case reports, authors frequently report diagnoses and lengthy case timelines, but not explicit motivations for individual motivations.

\subsection*{Takeaway points}
Ultimately, the literature on motivations in IIFO remains sparse, and clinical outcomes remain highly heterogeneous. The evidence base is not yet strong enough to draw definitive conclusions. As with many rare and complex presentations, prospective, randomised studies are needed—but these must be preceded by expert consensus on motivation frameworks.

IIFO is a nuanced and multifaceted form of self-harm. Motivation is rarely reported, and intentionality itself is often assumed rather than stated. Following this review, authors should heed the advice of experts in the field: a multidisciplinary approach is essential. Psychiatric and psychological input is needed not just for treatment, but also for accurate classification and understanding of patient motivations.

With more precise reporting of motivational and clinical variables, future reviews will be better placed to inform evidence-based guidance, ultimately improving care for this complex patient group.




\subsection*{Data Availability}
The data and code used in this systematic review are available at \textit{http://github.com/jackgedge/iifo\_systematic\_review}.




\printbibliography

\onecolumn
\appendices
\section{Eligibility Criteria}
\label{appendix:eligibility-criteria}

\subsection{Inclusion Criteria}
\label{tab:inclusion-criteria}
\begin{table}[h]
\centering
\begin{tabular}{p{4cm} p{11cm}}
\toprule
\textbf{Category} & \textbf{Details} \\
\midrule

Population & 
\begin{tabular}[t]{@{}l@{}}
Any human. \\
Any age group. \\
\end{tabular} \\

Interventions or exposures & 
\begin{tabular}[t]{@{}l@{}}
Humans that have: \\
– Non-accidental \\
– Ingestion of a true foreign body (non-nutritive items)\\
\end{tabular} \\

Comparators / Control group & 
\begin{tabular}[t]{@{}l@{}}
Motivation/reason for ingestion: \\
– Protest \\
– Suicidal intent \\
– Self-harm \\
– Psychiatric and other documented motivations \\[0.5ex]
Intervention details: \\
– Number of ingestions \\
– Management strategies (Conservative, Endoscopic, Surgical) \\[0.5ex]
Object characteristics: \\
– Multiple objects \\
– Blunt objects \\
– Sharp-pointed objects \\
– Long objects ($>$6 cm) \\
– Short objects ($\leq$6 cm) \\[0.5ex]
Setting/location\\
\end{tabular} \\

Outcomes of interest & 
\begin{tabular}[t]{@{}l@{}}
– Endoscopic intervention \\
– Surgical intervention \\
– Conservative management \\
– Complication rates \\
– Mortality rates \\
\end{tabular} \\

Setting & 
\begin{tabular}[t]{@{}l@{}}
Any setting. \\
\end{tabular} \\

Study designs & 
\begin{tabular}[t]{@{}l@{}}
– Observational studies (cohort, case-control, cross-sectional) \\
– Case series \\
– Clinical trials \\
– Case reports \\
\end{tabular} \\

\bottomrule
\end{tabular}
\end{table}

\subsection{Exclusion Criteria}
\label{tab:exclusion-criteria}
\begin{table}[h]
\centering
\begin{tabular}{p{1cm} p{14cm}}
\toprule
\textbf{\#} & \textbf{Exclusion Criterion} \\
\midrule

1 & Full text not available in English. \\

2 & Studies not focusing on intentional self-ingestion (into the gastrointestinal tract) of foreign object via the oral cavity (mouth) or where unclear if ingested. \\

3 & Studies focussing solely on accidental ingestion. \\

4 & Non-human or animal studies. \\

5 & Reviews, editorials, commentaries, and opinion pieces without original empirical data. \\

6 & Duplicate publications or studies with overlapping data sets (the most comprehensive or recent study will be included). \\

7 & Studies focusing on ingestion or co-ingestion of substances (e.g. poisons, medications) rather than physical foreign objects. \\

8 & Ingestions undertaken in controlled environments as part of a voluntary study. \\

9 & Ingestions not explicitly stated to be intentional and history not suggestive of deliberate ingestion (i.e. Age < 8, no history of previous ingestions, no psychiatric co-morbidities, not a prisoner/detainee/vulnerable group). \\

10 & Does not meet inclusion criteria. \\

11 & Ingestions where death resulted from other means (i.e. suicide). \\

12 & Studies before the advent of endoscopy (1906). \\

\bottomrule
\end{tabular}
\end{table}

\clearpage
\section{Appendix B - Search Strategy}

\section{Variable Definitions}
\label{appendix:variables}
Used for case report data extraction. Aggregates of which where used to create Variable\_Rate and Variable\_Cases.

\begin{table}[h]
\centering
\begin{tabular}{p{4.5cm} p{10.5cm}}
\toprule
\textbf{Variable} & \textbf{Definition} \\
\midrule
Is\_Prisoner & Documented in prison, police custody, or detained (including immigration detention) at the time of the encounter; 'N' if not detained; 'UK' if unknown. \\
Psych\_Hx & Documented DSM-V mental disorder (including substance-related disorders) \cite{AmericanPsychiatricAssociation_2013}; 'N' if no diagnosis; 'UK' if data unavailable. \\
Is\_Displaced\_Person & 'Y' if: meets the UN General Assembly \cite{UNGeneralAssembly_1967} definition of 'Refugee'; or meets UNHCR~\cite{Deng_1998} definition of an 'internally displaced person'; or meets the UNHCR~\cite{UnitedNationsHighCommissionerforRefugees_2025} definition for 'asylum seeker'; 'N' if not displaced; 'UK' if unknown. \\
Under\_Influence\_Alcohol & Evidence, suspicion, or self-report of alcohol influence at presentation; 'N' if no indication; 'UK' if unknown. \\
Is\_Psych\_Inpat & Admitted (voluntarily or involuntarily) to a psychiatric facility/ward at encounter; 'N' if not admitted; 'UK' if unknown. \\
Severe\_Disability\_Hx & History of severe learning disability or impaired consciousness; 'N' if absent; 'UK' if unknown. \\
Previous\_Ingestions & Prior episode of foreign-body ingestion documented; 'N' if first ingestion; 'UK' if history unknown. \\
Motivation\_Intent\_To\_Harm & Ingestion intended for self-harm, self-injury, or suicide; 'N' if other motive; 'UK' if unclear. \\
Motivation\_Protest & Ingestion as protest, demonstration, or manipulation (e.g., objection to detention conditions); 'N' if not protest-related; 'UK' if unclear. \\
Motivation\_Psychiatric & Ingestion driven primarily by an underlying psychiatric condition (psychosis, impulsivity, etc.); 'N' if not psychiatric; 'UK' if unclear. \\
Motivation\_Psychosocial & Ingestion motivated by social or interpersonal factors (imitative acts, shock value, body-image, safekeeping, etc.); 'N' if not psychosocial; 'UK' if unclear. \\
Motivation\_Unknown & No clear motivation identified in documentation; 'N' if specific motive recorded; 'UK' if ambiguous. \\
Object\_Button\_Battery & Button battery ingested; 'N' if not; 'UK' if object type not recorded. \\
Object\_Magnet & Magnet ingested; 'N' if none; 'UK' if unknown. \\
Object\_Long & Ingested object length $>$ 5 cm; 'N' if $\leq$ 5 cm; 'UK' if dimensions unknown. \\
Object\_Sharp & Object described as sharp or pointed (e.g., blades, nails, needles); 'N' if not sharp; 'UK' if unclear. \\
Object\_Multiple & More than one object ingested in same episode; 'N' for single object; 'UK' if number unspecified. \\
Object\_Unknown & Where object characteristics are unknown. 'N' if known; 'UK' if Unknown. \\
Outcome\_Endoscopy & Endoscopic intervention performed during episode; 'N' if not; 'UK' if unavailable. \\
Outcome\_Surgery & Surgical intervention performed (operative procedure under anaesthesia); 'N' if not; 'UK' if not documented. \\
Outcome\_Conservative & 'Y' if managed without endoscopy or surgery; 'N' if either procedure performed. \\
Outcome\_Death & Death causally related to ingestion complications; 'N' if survived; 'UK' if outcome unknown. \\
Outcome\_Complication & 'Y' if any complication directly related to ingestion or resulting from management strategy; 'N' if no complication; 'UK' if unknown. \\
Outcome\_Unknown & Where no outcome identified; 'N' if outcome identified; 'UK' if Unknown. \\
\bottomrule
\end{tabular}
\end{table}


\begin{landscape}
\begin{table}[ht]
\centering
\scriptsize
\renewcommand{\arraystretch}{1.0}
\begin{tabular}{lrlrrrrrrrlrrrrrrrrrrrrrrrlrrrrrrlrrrrrrlllllllrrrrrrrrlll}
\toprule
 & \rotatebox{90}{\rotatebox{90}{Study_ID}} & \rotatebox{90}{\rotatebox{90}{Patient_ID}} & \rotatebox{90}{\rotatebox{90}{Age_Yrs}} & \rotatebox{90}{\rotatebox{90}{Age_<18}} & \rotatebox{90}{\rotatebox{90}{Age_18-25}} & \rotatebox{90}{\rotatebox{90}{Age_26-40}} & \rotatebox{90}{\rotatebox{90}{Age_41-60}} & \rotatebox{90}{\rotatebox{90}{Age_60+}} & \rotatebox{90}{\rotatebox{90}{Age_Unknown}} & \rotatebox{90}{\rotatebox{90}{Gender}} & \rotatebox{90}{\rotatebox{90}{Gender_Male}} & \rotatebox{90}{\rotatebox{90}{Gender_Female}} & \rotatebox{90}{\rotatebox{90}{Gender_Unknown}} & \rotatebox{90}{\rotatebox{90}{Is_Prisoner}} & \rotatebox{90}{\rotatebox{90}{Is_Psych_Inpat}} & \rotatebox{90}{\rotatebox{90}{Is_Displaced_Person}} & \rotatebox{90}{\rotatebox{90}{Under_Influence_Alcohol}} & \rotatebox{90}{\rotatebox{90}{Psych_Hx}} & \rotatebox{90}{\rotatebox{90}{Severe_Disability_Hx}} & \rotatebox{90}{\rotatebox{90}{Previous_Ingestions}} & \rotatebox{90}{\rotatebox{90}{Motivation_Intent_To_Harm}} & \rotatebox{90}{\rotatebox{90}{Motivation_Protest}} & \rotatebox{90}{\rotatebox{90}{Motivation_Psychiatric}} & \rotatebox{90}{\rotatebox{90}{Motivation_Psychosocial}} & \rotatebox{90}{\rotatebox{90}{Motivation_Other}} & \rotatebox{90}{\rotatebox{90}{Motivation_Other_Long}} & \rotatebox{90}{\rotatebox{90}{Object_Button_Battery}} & \rotatebox{90}{\rotatebox{90}{Object_Magnet}} & \rotatebox{90}{\rotatebox{90}{Object_Long}} & \rotatebox{90}{\rotatebox{90}{Object_Diameter_Large}} & \rotatebox{90}{\rotatebox{90}{Object_Sharp}} & \rotatebox{90}{\rotatebox{90}{Object_Multiple}} & \rotatebox{90}{\rotatebox{90}{Object_Other_Long}} & \rotatebox{90}{\rotatebox{90}{Outcome_Endoscopy}} & \rotatebox{90}{\rotatebox{90}{Outcome_Surgery}} & \rotatebox{90}{\rotatebox{90}{Outcome_Death}} & \rotatebox{90}{\rotatebox{90}{Outcome_Conservative}} & \rotatebox{90}{\rotatebox{90}{Outcome_Complication}} & \rotatebox{90}{\rotatebox{90}{Publication_Year}} & \rotatebox{90}{\rotatebox{90}{Authors}} & \rotatebox{90}{\rotatebox{90}{Title}} & \rotatebox{90}{\rotatebox{90}{Publication_Title}} & \rotatebox{90}{\rotatebox{90}{DOI}} & \rotatebox{90}{\rotatebox{90}{Study_Country}} & \rotatebox{90}{\rotatebox{90}{Study_Design}} & \rotatebox{90}{\rotatebox{90}{Citekey}} & \rotatebox{90}{\rotatebox{90}{Bias_Patient_Demographic}} & \rotatebox{90}{\rotatebox{90}{Bias_History_Timeline}} & \rotatebox{90}{\rotatebox{90}{Bias_Condition_Described}} & \rotatebox{90}{\rotatebox{90}{Bias_Diagnostic_Assessment}} & \rotatebox{90}{\rotatebox{90}{Bias_Intervention_Treatment}} & \rotatebox{90}{\rotatebox{90}{Bias_Post_Intervention_Condition}} & \rotatebox{90}{\rotatebox{90}{Bias_Harms}} & \rotatebox{90}{\rotatebox{90}{Bias_Takeaway_Lessons}} & \rotatebox{90}{\rotatebox{90}{Bias_Overall_Appraisal}} & \rotatebox{90}{\rotatebox{90}{Bias_Reviewer_Initials}} & \rotatebox{90}{\rotatebox{90}{Bias_Review_Date}} \\
\midrule
0 & 51 & 51-001 & 20.000000 & 0 & 1 & 0 & 0 & 0 & 0 & Male & 1 & 0 & 0 & 1.000000 & 0.000000 & NaN & 0.000000 & NaN & 0.000000 & NaN & 1.000000 & 1.000000 & 0.000000 & 0.000000 & 0.000000 & NaN & 0.000000 & 0.000000 & 0.000000 & 1.000000 & 1.000000 & 1.000000 & Gastrointestinal Crosses & 0.000000 & 1.000000 & 0.000000 & 0.000000 & 1.000000 & 1996 & Losanoff, J. E.; Kjossev, K. T. & Gastrointestinal 'crosses'. a new shade from an old palette & Archives Of Surgery (Chicago, Ill. : 1960) & 10.1001/archsurg.1996.01430140056015 & Bulgaria & Case Series & Losanoff_1996 & 1 & 1 & 1 & 1 & 1 & 1 & 1.000000 & 1 & Include & JGE & 2025-05-09 00:00:00 \\
1 & 51 & 51-002 & 20.000000 & 0 & 1 & 0 & 0 & 0 & 0 & Male & 1 & 0 & 0 & 1.000000 & 0.000000 & NaN & 0.000000 & NaN & 0.000000 & NaN & 1.000000 & 1.000000 & 0.000000 & 0.000000 & 0.000000 & NaN & 0.000000 & 0.000000 & 0.000000 & 1.000000 & 1.000000 & 1.000000 & Gastrointestinal Crosses & 0.000000 & 1.000000 & 0.000000 & 0.000000 & 1.000000 & 1996 & Losanoff, J. E.; Kjossev, K. T. & Gastrointestinal 'crosses'. a new shade from an old palette & Archives Of Surgery (Chicago, Ill. : 1960) & 10.1001/archsurg.1996.01430140056015 & Bulgaria & Case Series & Losanoff_1996 & 1 & 1 & 1 & 1 & 1 & 1 & 1.000000 & 1 & Include & JGE & 2025-05-10 00:00:00 \\
2 & 51 & 51-003 & 19.000000 & 0 & 1 & 0 & 0 & 0 & 0 & Male & 1 & 0 & 0 & 1.000000 & 0.000000 & NaN & 0.000000 & NaN & 0.000000 & NaN & 1.000000 & 1.000000 & 0.000000 & 0.000000 & 0.000000 & NaN & 0.000000 & 0.000000 & 0.000000 & 1.000000 & 1.000000 & 1.000000 & Gastrointestinal Crosses & 0.000000 & 1.000000 & 0.000000 & 0.000000 & 1.000000 & 1996 & Losanoff, J. E.; Kjossev, K. T. & Gastrointestinal 'crosses'. a new shade from an old palette & Archives Of Surgery (Chicago, Ill. : 1960) & 10.1001/archsurg.1996.01430140056015 & Bulgaria & Case Series & Losanoff_1996 & 1 & 1 & 1 & 1 & 1 & 1 & 1.000000 & 1 & Include & JGE & 2025-05-11 00:00:00 \\
3 & 51 & 51-004 & 21.000000 & 0 & 1 & 0 & 0 & 0 & 0 & Male & 1 & 0 & 0 & 1.000000 & 0.000000 & NaN & 0.000000 & NaN & 0.000000 & NaN & 1.000000 & 1.000000 & 0.000000 & 0.000000 & 0.000000 & NaN & 0.000000 & 0.000000 & 0.000000 & 1.000000 & 1.000000 & 1.000000 & Gastrointestinal Crosses & 0.000000 & 1.000000 & 0.000000 & 0.000000 & 1.000000 & 1996 & Losanoff, J. E.; Kjossev, K. T. & Gastrointestinal 'crosses'. a new shade from an old palette & Archives Of Surgery (Chicago, Ill. : 1960) & 10.1001/archsurg.1996.01430140056015 & Bulgaria & Case Series & Losanoff_1996 & 1 & 1 & 1 & 1 & 1 & 1 & 1.000000 & 1 & Include & JGE & 2025-05-12 00:00:00 \\
4 & 51 & 51-005 & 29.000000 & 0 & 0 & 1 & 0 & 0 & 0 & Male & 1 & 0 & 0 & 1.000000 & 0.000000 & NaN & 0.000000 & NaN & 0.000000 & NaN & 1.000000 & 1.000000 & 0.000000 & 0.000000 & 0.000000 & NaN & 0.000000 & 0.000000 & 0.000000 & 1.000000 & 1.000000 & 1.000000 & Gastrointestinal Crosses & 0.000000 & 1.000000 & 0.000000 & 0.000000 & 1.000000 & 1996 & Losanoff, J. E.; Kjossev, K. T. & Gastrointestinal 'crosses'. a new shade from an old palette & Archives Of Surgery (Chicago, Ill. : 1960) & 10.1001/archsurg.1996.01430140056015 & Bulgaria & Case Series & Losanoff_1996 & 1 & 1 & 1 & 1 & 1 & 1 & 1.000000 & 1 & Include & JGE & 2025-05-13 00:00:00 \\
5 & 54 & 54-001 & 23.000000 & 0 & 1 & 0 & 0 & 0 & 0 & Male & 1 & 0 & 0 & 1.000000 & 0.000000 & NaN & 0.000000 & NaN & 0.000000 & NaN & 1.000000 & 1.000000 & 0.000000 & 0.000000 & 0.000000 & NaN & 0.000000 & 0.000000 & 0.000000 & 1.000000 & 1.000000 & 0.000000 & Gastrointestinal Cross & 0.000000 & 1.000000 & 0.000000 & 0.000000 & 0.000000 & 1997 & Losanoff, J. E.; Kjossev, K. T.; Losanoff, H. E. & Oesophageal "cross"--a sinister foreign body & Journal Of Accident & Emergency Medicine & 10.1136/emj.14.1.54 & Bulgaria & Case Report & Losanoff_1997e & 0 & 1 & 1 & 1 & 1 & 1 & NaN & 1 & Include & JGE & 2025-05-08 00:00:00 \\
6 & 60 & 60-001 & 16.000000 & 1 & 0 & 0 & 0 & 0 & 0 & Female & 0 & 1 & 0 & 0.000000 & 0.000000 & NaN & 0.000000 & 0.000000 & 0.000000 & 0.000000 & 0.000000 & 0.000000 & 0.000000 & 1.000000 & 0.000000 & For dietary purposes & 0.000000 & 0.000000 & 0.000000 & 0.000000 & 0.000000 & 1.000000 & Toilet Roll & 0.000000 & 0.000000 & 0.000000 & 1.000000 & 1.000000 & 1998 & Goldman, R. D.; Schachter, P.; Katz, M.; Bilik, R.; Avigad, I. & A bizarre bezoar: case report and review of the literature & Pediatric Surgery International & 10.1007/s003830050492 & Israel & Case Report & Goldman_1998f & 1 & 1 & 1 & 1 & 1 & 1 & NaN & 1 & Include & JGE & 2025-05-08 00:00:00 \\
7 & 61 & 61-001 & 22.000000 & 0 & 1 & 0 & 0 & 0 & 0 & Male & 1 & 0 & 0 & 0.000000 & 0.000000 & NaN & 1.000000 & NaN & 0.000000 & 1.000000 & 1.000000 & 0.000000 & NaN & 0.000000 & 0.000000 & Under the influence of alcohol and with suicidal intention.s & 0.000000 & 0.000000 & 1.000000 & 1.000000 & 1.000000 & 0.000000 & 9X1.5cm pocket knife blade & 0.000000 & 1.000000 & 0.000000 & 0.000000 & 1.000000 & 1998 & Csaky, G; Szederkenyi, I; Botos, A; Kiss, I & Laparoscopic removal of a foreign body from the jejunum & Surgical Laparoscopy & Endoscopy & 10.1097/00019509-199802000-00016 & Hungary & Case Report & Csaky_1998e & 1 & 1 & 1 & 1 & 1 & 1 & NaN & 1 & Include & JGE & 2025-05-08 00:00:00 \\
8 & 85 & 85-001 & 22.000000 & 0 & 1 & 0 & 0 & 0 & 0 & Female & 0 & 1 & 0 & 0.000000 & 0.000000 & NaN & 0.000000 & 0.000000 & 0.000000 & 0.000000 & 0.000000 & 1.000000 & 0.000000 & 1.000000 & 0.000000 & Infront of husband “for shock value.” & 0.000000 & 0.000000 & 0.000000 & 0.000000 & 0.000000 & 0.000000 & US penny coin & 0.000000 & 1.000000 & 0.000000 & 0.000000 & 1.000000 & 2004 & Tupesis, J. P.; Kaminski, A.; Patel, H.; Howes, D. & A penny for your thoughts: small bowel obstruction secondary to coin ingestion & Journal Of Emergency Medicine & 10.1016/j.jemermed.2004.03.013 & USA & Case Report & Tupesis_2004f & 1 & 1 & 1 & 1 & 1 & 1 & 0.000000 & 1 & Include & JGE & 2025-05-08 00:00:00 \\
9 & 92 & 92-001 & 30.000000 & 0 & 0 & 1 & 0 & 0 & 0 & Male & 1 & 0 & 0 & 1.000000 & 0.000000 & NaN & 0.000000 & 1.000000 & 0.000000 & 1.000000 & 1.000000 & 0.000000 & 1.000000 & 0.000000 & 0.000000 & Command hallucinations in schizophrenic prison inmate & 0.000000 & 0.000000 & NaN & 1.000000 & 1.000000 & 0.000000 & Rolled up tuna can lid & 0.000000 & 1.000000 & 0.000000 & 0.000000 & 0.000000 & 2006 & Alao, A. O.; Abraham, B. & Foreign body ingestions in a schizophrenic patient & West African Journal Of Medicine & 10.4314/wajm.v25i3.28286 & USA & Case Report & Alao_2006i & 1 & 1 & 1 & 1 & 1 & 1 & NaN & 1 & Include & JGE & 2025-05-08 00:00:00 \\
10 & 113 & 113-001 & 39.000000 & 0 & 0 & 1 & 0 & 0 & 0 & Female & 0 & 1 & 0 & 0.000000 & 0.000000 & NaN & 0.000000 & 1.000000 & 0.000000 & 1.000000 & 1.000000 & 0.000000 & 1.000000 & 0.000000 & 1.000000 & "Reported that thermometer ingestion and suicide attempts were her best chance to obtain narcotic medications. History of narcotic use since teens, multiple admissions and surgeries related to similar behavior." & 0.000000 & 0.000000 & 1.000000 & 1.000000 & 0.000000 & 0.000000 & "Thermometer" & 1.000000 & 0.000000 & 0.000000 & 0.000000 & 1.000000 & 2008 & Sakellaridis, Timothy; Potaris, Konstantinos; Mallios, Dimitrios; Sepsas, Evangellos & An unusual case of a swallowed thermometer perforated in the mediastinum & Annals Of Thoracic Surgery & 10.1016/j.athoracsur.2007.07.027 & Greece & Case Report & Sakellaridis_2008f & 1 & 1 & 1 & 1 & 1 & 1 & NaN & 1 & Include & JGE & 2025-05-08 00:00:00 \\
11 & 148 & 148-001 & 19.000000 & 0 & 1 & 0 & 0 & 0 & 0 & Male & 1 & 0 & 0 & 0.000000 & 0.000000 & NaN & 0.000000 & 0.000000 & 0.000000 & 0.000000 & 0.000000 & 0.000000 & 0.000000 & 1.000000 & 0.000000 & "attempt to impress his friends" & 0.000000 & 0.000000 & 0.000000 & 0.000000 & 1.000000 & 1.000000 & "eight needles" & 0.000000 & 1.000000 & 0.000000 & 0.000000 & 1.000000 & 2011 & Sobnach, Sanju; Castillo, Franco; Blanco Vinent, René; Kahn, Delawir; Bhyat, Ahmed & Penetrating cardiac injury following sewing needle ingestion & Heart, Lung & Circulation & 10.1016/j.hlc.2011.01.006 & South Africa & Case Report & Sobnach_2011f & 1 & 1 & 1 & 1 & 1 & 1 & NaN & 1 & Include & JGE & 2025-05-08 00:00:00 \\
12 & 168 & 168-001 & 36.000000 & 0 & 0 & 1 & 0 & 0 & 0 & Male & 1 & 0 & 0 & 1.000000 & 0.000000 & NaN & 0.000000 & 0.000000 & 0.000000 & 0.000000 & NaN & NaN & 0.000000 & 0.000000 & 1.000000 & "Foreign body ingested in police custody to conceal object, likely related to narcotics smuggling. Lighter was double-wrapped in cellophane and retained for 17 months without prior symptoms." & 0.000000 & 0.000000 & 1.000000 & 1.000000 & 0.000000 & 0.000000 & "a lighter (8 cm long) which was double wrapped in cellophane" & 1.000000 & 0.000000 & 0.000000 & 0.000000 & 1.000000 & 2012 & Trgo, Gorana; Tonkic, Ante; Simunic, Miroslav; Puljiz, Zeljko & Successful endoscopic removal of a lighter swallowed 17 months before & Case Reports In Gastroenterology & 10.1159/000338839 & Croatia & Case Report & Trgo_2012f & 1 & 1 & 1 & 1 & 1 & 1 & 1.000000 & 1 & Include & JGE & 2025-05-08 00:00:00 \\
13 & 171 & 171-001 & 26.000000 & 0 & 0 & 1 & 0 & 0 & 0 & Male & 1 & 0 & 0 & 1.000000 & 0.000000 & NaN & 0.000000 & 1.000000 & 0.000000 & NaN & 1.000000 & 0.000000 & 1.000000 & 0.000000 & 0.000000 & "A 26-year-old man with a past medical history of bipolar disorder, schizophrenia and diabetes mellitus was brought from a correctional facility after confessing to have swallowed a few shower curtain hooks. The patient heard some inner voices telling him to kill himself. He then straightened a few shower curtain hooks and swallowed them." & 0.000000 & 0.000000 & 0.000000 & NaN & 0.000000 & 1.000000 & "swallowed a few shower curtain hooks" & 1.000000 & 0.000000 & 0.000000 & 0.000000 & 0.000000 & 2012 & Tammana, V. S.; Valluru, N.; Sanderson, A. & All the wrong places: an unusual case of foreign body ingestion and inhalation & Case Reports In Gastroenterology & 10.1159/000346287 & USA & Case Report & Tammana_2012j & 1 & 1 & 1 & 1 & 1 & 1 & 1.000000 & 1 & Include & JGE & 2025-05-08 00:00:00 \\
14 & 172 & 172-001 & 12.000000 & 1 & 0 & 0 & 0 & 0 & 0 & Female & 0 & 1 & 0 & 0.000000 & 0.000000 & NaN & 0.000000 & 0.000000 & 0.000000 & 0.000000 & 0.000000 & 0.000000 & 0.000000 & 1.000000 & 0.000000 & "playing a game with her brother" & 0.000000 & 1.000000 & 0.000000 & 1.000000 & 0.000000 & 1.000000 & "14 consecutive metal balls, stretching a distance of 7 cm, in the small intestine, likely in the jejunum" & 0.000000 & 1.000000 & 0.000000 & 0.000000 & 1.000000 & 2012 & Naji, Hussein; Isacson, Daniel; Svensson, Jan F.; Wester, Tomas & Bowel injuries caused by ingestion of multiple magnets in children: a growing hazard & Pediatric Surgery International & 10.1007/s00383-011-3026-x & Sweden & Case Series & Naji_2012f & 1 & 1 & 1 & 1 & 1 & 1 & 1.000000 & 1 & Include & JGE & 2025-05-08 00:00:00 \\
17 & 214 & 214-001 & 17.000000 & 1 & 0 & 0 & 0 & 0 & 0 & Male & 1 & 0 & 0 & 0.000000 & NaN & NaN & NaN & 1.000000 & 0.000000 & 1.000000 & 1.000000 & 0.000000 & 1.000000 & 0.000000 & 0.000000 & "A seventeen year-old male patient was brought to the emergency department due to self-wrist cutting for suicide. The patient was suffering from multiple psychiatric and social problems." & 0.000000 & 1.000000 & 0.000000 & 1.000000 & 0.000000 & 1.000000 & "thirty-three magnets" & 1.000000 & 1.000000 & 0.000000 & 0.000000 & 1.000000 & 2015 & Tanrikulu, Y; Sen Tanrikulu, C; Karannan, S; Sahin, H & Ingestion of multiple magnets for suicide & Hong Kong Journal Of Emergency Medicine & 10.1177/102490791502200107 & Turkey & Case Report & Tanrikulu_2015e & 1 & 1 & 1 & 1 & 1 & 1 & 1.000000 & 1 & Include & JGE & 2025-05-08 00:00:00 \\
18 & 217 & 217-001 & 44.000000 & 0 & 0 & 0 & 1 & 0 & 0 & Male & 1 & 0 & 0 & 0.000000 & 0.000000 & NaN & 0.000000 & 0.000000 & 0.000000 & NaN & 0.000000 & 0.000000 & 0.000000 & 0.000000 & 1.000000 & Smuggling & 0.000000 & 0.000000 & 0.000000 & 0.000000 & 0.000000 & 1.000000 & "29 golden pellets weighing 15g each" & 1.000000 & 0.000000 & 0.000000 & 0.000000 & 0.000000 & 2015 & Wadhwa, C; Madhavan, S; Augustine, AJ; Shenoy, S; Mirza, A & The mule with golden eggs: retrieval of unusual foreign body & Journal Of Digestive Endoscopy & 10.4103/0976-5042.159247 & India & Case Report & Wadhwa_2015e & 1 & 1 & 1 & 1 & 1 & 1 & 1.000000 & 1 & Include & JGE & 2025-05-08 00:00:00 \\
19 & 219 & 219-001 & 33.000000 & 0 & 0 & 1 & 0 & 0 & 0 & Female & 0 & 1 & 0 & 0.000000 & 0.000000 & NaN & 0.000000 & 0.000000 & 0.000000 & 0.000000 & 0.000000 & 0.000000 & 0.000000 & 1.000000 & 0.000000 & "motivated by the aim to induce weight loss" & 0.000000 & 0.000000 & 0.000000 & 1.000000 & 0.000000 & 0.000000 & "swallowing a small (5 x 3 x 3 cm) rubber balloon filled with water, two days earlier" & 1.000000 & 1.000000 & 0.000000 & 0.000000 & 1.000000 & 2015 & Wnƒôk, Bartosz; ≈Åo≈ºy≈Ñska-Nelke, Aleksandra; Karo≈Ñ, Jacek & Foreign body in the gastrointestinal tract leading to small bowel obstruction--case report and literature review & Polski Przeglad Chirurgiczny & 10.1515/pjs-2015-0006 & Poland & Case Report & Wnęk_2015f & 0 & 1 & 1 & 1 & 1 & 1 & 1.000000 & 1 & Include & JGE & 2025-05-08 00:00:00 \\
20 & 226 & 226-001 & 23.000000 & 0 & 1 & 0 & 0 & 0 & 0 & Male & 1 & 0 & 0 & 0.000000 & 0.000000 & 1.000000 & 0.000000 & 0.000000 & 0.000000 & NaN & 0.000000 & 0.000000 & 0.000000 & 1.000000 & 0.000000 & "deliberately ingested by a refugee for safe keeping" & 0.000000 & 0.000000 & 0.000000 & 1.000000 & 0.000000 & 0.000000 & "money package 2.5x4cm", "a bundle of dollars wrapped in a nylon bag", "1000 dollars" & 1.000000 & 0.000000 & 0.000000 & 0.000000 & 0.000000 & 2015 & Akay, Seval; Günay, Süleyman; Binicier, Ömer Burçak; Paköz, Zehra Betül; Akar, Harun & A deliberately swallowed foreign body: money package & Endoscopy & 10.1055/s-0035-1569668 & Turkey & Case Report & Akay_2015f & 1 & 1 & 1 & 1 & 1 & 1 & NaN & 1 & Include & JGE & 2025-05-08 00:00:00 \\
21 & 238 & 238-001 & 29.000000 & 0 & 0 & 1 & 0 & 0 & 0 & Female & 0 & 1 & 0 & 0.000000 & 0.000000 & NaN & 0.000000 & 1.000000 & 1.000000 & 1.000000 & NaN & NaN & 0.000000 & 0.000000 & 1.000000 & "A 29-year-old mentally retarded female patient was admitted to the emergency service with a two-day history of abdominal pain, nausea, vomiting, and failure to eliminate feces or pass gas. Patient history revealed that the patient had undergone two surgeries due to repeated foreign body ingestion within the last 6 months." & 0.000000 & 0.000000 & 1.000000 & 1.000000 & 0.000000 & 0.000000 & "metal teaspoon" & 0.000000 & 1.000000 & 0.000000 & 0.000000 & 1.000000 & 2016 & Yildiz, I; Koca, YS; Avsar, G; Barut, I & Tendency to ingest foreign bodies in mentally retarded patients: a case with ileal perforation caused by the ingestion of a teaspoon & Case Reports In Surgery & 10.1155/2016/8075432 & Turkey & Case Report & Yildiz_2016e & 1 & 1 & 1 & 1 & 1 & 1 & 1.000000 & 1 & Include & JGE & 2025-05-08 00:00:00 \\
22 & 260 & 260-001 & 22.000000 & 0 & 1 & 0 & 0 & 0 & 0 & Female & 0 & 1 & 0 & 0.000000 & NaN & NaN & 0.000000 & 1.000000 & 1.000000 & NaN & NaN & NaN & 0.000000 & 0.000000 & 1.000000 & "history of cerebral palsy and self-destructive behaviour" & 0.000000 & 0.000000 & 0.000000 & 1.000000 & 0.000000 & 0.000000 & "padlock 25mm in width" & 1.000000 & 0.000000 & 0.000000 & 0.000000 & 0.000000 & 2017 & Peixoto, A.; Pereira, P.; Macedo, G. & Gastrointestinal: voluntary padlock ingestion & Journal Of Gastroenterology And Hepatology & 10.1111/jgh.13828 & Portugal & Case Report & Peixoto_2017f & 1 & 1 & 1 & 1 & 1 & 1 & NaN & 1 & Include & JGE & 2025-05-08 00:00:00 \\
23 & 274 & 274-001 & 38.000000 & 0 & 0 & 1 & 0 & 0 & 0 & Female & 0 & 1 & 0 & 0.000000 & 0.000000 & NaN & 0.000000 & 1.000000 & 0.000000 & NaN & 1.000000 & 0.000000 & 0.000000 & 0.000000 & 0.000000 & "major depressive disorder swallowed for self-mutilation" & 0.000000 & 0.000000 & 1.000000 & 1.000000 & 0.000000 & 0.000000 & "comb 18 x 2.4 x 0.4 cm" & 1.000000 & 1.000000 & 0.000000 & 0.000000 & 0.000000 & 2017 & Chang, Wen-Jung; Chiu, Wen-Yi & Gastric foreign body: a comb & Clinical Case Reports & 10.1002/ccr3.957 & Taiwan & Case Report & Chang_2017f & 1 & 1 & 1 & 1 & 1 & 1 & NaN & 0 & Include & JGE & 2025-05-08 00:00:00 \\
24 & 278 & 278-001 & 27.000000 & 0 & 0 & 1 & 0 & 0 & 0 & Male & 1 & 0 & 0 & 0.000000 & 0.000000 & 1.000000 & 0.000000 & 0.000000 & 0.000000 & 0.000000 & 0.000000 & 1.000000 & 0.000000 & 0.000000 & 0.000000 & "in protest against possible deportation" & 0.000000 & 0.000000 & 1.000000 & 1.000000 & 0.000000 & 0.000000 & "Ballpoint pen" & 1.000000 & 1.000000 & 0.000000 & 0.000000 & 1.000000 & 2017 & Gardner, Andrew W.; Radwan, Rami W.; Allison, Miles C.; Codd, Rhodri J. & Double duodenal perforation following foreign body ingestion & Bmj Case Reports & 10.1136/bcr-2017-223182 & UK & Case Report & Gardner_2017h & 1 & 1 & 1 & 1 & 1 & 1 & 1.000000 & 1 & Include & JGE & 2025-05-08 00:00:00 \\
25 & 292 & 292-001 & 27.000000 & 0 & 0 & 1 & 0 & 0 & 0 & Male & 1 & 0 & 0 & 0.000000 & 0.000000 & 0.000000 & 0.000000 & 0.000000 & 0.000000 & 0.000000 & 0.000000 & 0.000000 & 0.000000 & 1.000000 & 0.000000 & "for a bet" & 0.000000 & 0.000000 & 0.000000 & 1.000000 & 0.000000 & 0.000000 & "2.5 cm wooden spherical object with a central hole" & 1.000000 & 0.000000 & 0.000000 & 0.000000 & 0.000000 & 2018 & Riva, Carlo Galdino; Toti, Francesco Angelo Taddàus; Siboni, Stefano; Bonavina, Luigi & Unusual foreign body impacted in the upper oesophagus: original technique for transoral extraction & Bmj Case Reports & 10.1136/bcr-2018-225241 & Italy & Case Report & Riva_2018j & 1 & 1 & 1 & 1 & 1 & 1 & NaN & 1 & Include & JGE & 2025-05-08 00:00:00 \\
26 & 311 & 311-001 & 47.000000 & 0 & 0 & 0 & 1 & 0 & 0 & Male & 1 & 0 & 0 & 0.000000 & 0.000000 & 0.000000 & 1.000000 & 0.000000 & 0.000000 & 1.000000 & 0.000000 & 0.000000 & 0.000000 & 1.000000 & 0.000000 & "shaman", "Bell is regarded as the holy object in a temple, so the shaman who ingests bell clappers was believed to be near to the god and people trust him more" & 0.000000 & 0.000000 & 1.000000 & 1.000000 & 0.000000 & 1.000000 & "15 cm and 10 cm long each two (four Bell clappers)" & 0.000000 & 1.000000 & 0.000000 & 0.000000 & 1.000000 & 2019 & Thapa, Niresh; Basnyat, Subi; Maharjan, Muna & Ingestion of bell clappers by a shaman in jumla, nepal: a case report & Jnma; Journal Of The Nepal Medical Association & 10.31729/jnma.4055 & Nepal & Case Report & Thapa_2019f & 1 & 1 & 1 & 1 & 1 & 1 & 1.000000 & 1 & Include & JGE & 2025-05-08 00:00:00 \\
27 & 314 & 314-001 & 35.000000 & 0 & 0 & 1 & 0 & 0 & 0 & Male & 1 & 0 & 0 & 0.000000 & 0.000000 & 0.000000 & NaN & 1.000000 & 0.000000 & NaN & NaN & 0.000000 & 1.000000 & 0.000000 & 0.000000 & "history of personality disorder and substance abuse was brought to the emergency department by his spouse with complaint of some sharp object protruding out of his abdomen in the epigastric region" & 0.000000 & 0.000000 & 1.000000 & 1.000000 & 1.000000 & 1.000000 & "Multiple foreign bodies (eight spoons, two screw drivers, two toothbrushes, a knife, and a nail) which were extracted by anterior gastrotomy" & 0.000000 & 1.000000 & 0.000000 & 0.000000 & 1.000000 & 2019 & Kumar, Ranesh; Soni, Nikhil; Bhardwaj, Suraj; Bhoil, Rohit; Gupta, Suresh C. & Intentional foreign body ingestion & Internal And Emergency Medicine & 10.1007/s11739-019-02183-4 & India & Case Report & Kumar_2019f & 1 & 1 & 1 & 1 & 1 & 0 & 0.000000 & 1 & Include & JGE & 2025-05-08 00:00:00 \\
28 & 319 & 319-001 & 46.000000 & 0 & 0 & 0 & 1 & 0 & 0 & Male & 1 & 0 & 0 & 0.000000 & 0.000000 & 0.000000 & NaN & 1.000000 & 0.000000 & 1.000000 & NaN & 0.000000 & 1.000000 & 0.000000 & 0.000000 & "schizophrenic male stopped taking his antipsychotics and mood stabilizers 5 days before. He reported swallowing one razor blade for reasons he could not describe" & 0.000000 & 0.000000 & 0.000000 & 1.000000 & 1.000000 & 1.000000 & "2 4cm razor blades" & 1.000000 & 0.000000 & 0.000000 & 0.000000 & 0.000000 & 2019 & Jehangir, Maham; Mallory, Christopher; Medverd, Jonathan R. & Digital tomosynthesis for detection of ingested foreign objects in the emergency department: a case of razor blade ingestion & Emergency Radiology & 10.1007/s10140-018-01664-x & USA & Case Report & Jehangir_2019h & 1 & 1 & 1 & 1 & 1 & 1 & NaN & 1 & Include & JGE & 2025-05-08 00:00:00 \\
29 & 322 & 322-001 & 28.000000 & 0 & 0 & 1 & 0 & 0 & 0 & Male & 1 & 0 & 0 & 0.000000 & 0.000000 & 0.000000 & 1.000000 & 0.000000 & 0.000000 & 0.000000 & 0.000000 & 0.000000 & 0.000000 & 1.000000 & 0.000000 & "Friends topped off a drinking party with live fishes from their aquarium. After the goldfishes had gone down smoothly, a bronze catfish was ingested. Unaware of the morphology and anti-predator behaviour of this species, a healthy but intoxicated 28-year-old man got a surprise" & 0.000000 & 0.000000 & 0.000000 & 0.000000 & 1.000000 & 0.000000 & Bronze catfish (Corydoras aeneus) & 1.000000 & 0.000000 & 0.000000 & 0.000000 & 1.000000 & 2019 & Benoist, LBL; van der Hoven, B; de Vries, AC; Pullens, B; Kompanje, EJO; Moeliker, CW & A jackass and a fish: a case of life-threatening intentional ingestion of a live pet catfish corydoras aeneus & Acta Oto-Laryngologica Case Reports & 10.1080/23772484.2018.1555436 & Netherlands & Case Report & Benoist_2019e & 0 & 1 & 1 & 1 & 1 & 1 & 1.000000 & 1 & Include & JGE & 2025-05-11 00:00:00 \\
30 & 328 & 328-001 & 39.000000 & 0 & 0 & 1 & 0 & 0 & 0 & Male & 1 & 0 & 0 & 0.000000 & NaN & 0.000000 & NaN & 1.000000 & 0.000000 & 1.000000 & 0.000000 & 0.000000 & 1.000000 & 0.000000 & 1.000000 & "PICA", "anxiety and an empty prescription for alprazolam as the primary trigger leading to the ingestions.", "past psychiatric history included major depressive disorder, generalized anxiety disorder, posttraumatic stress disorder, borderline personality disorder, and pica with a history of more than twenty admissions for ingestion behaviors often requiring endoscopic retrieval" & 0.000000 & 0.000000 & 0.000000 & 1.000000 & 1.000000 & 1.000000 & "50 paperclips, 50 screws, eight batteries, and seven razor blades covered in paper" & 1.000000 & 0.000000 & 0.000000 & 0.000000 & 0.000000 & 2019 & Guinan, D.; Drvar, T.; Brubaker, D.; Ang-Rabanes, M.; Kupec, J.; Marshalek, P. & Intentional foreign body ingestion: a complex case of pica & Case Reports In Gastrointestinal Medicine & 10.1155/2019/7026815 & USA & Case Report & Guinan_2019f & 1 & 1 & 1 & 0 & 1 & 1 & NaN & 1 & Include & JGE & 2025-05-11 00:00:00 \\
31 & 348 & 348-001 & 42.000000 & 0 & 0 & 0 & 1 & 0 & 0 & Male & 1 & 0 & 0 & 0.000000 & NaN & NaN & NaN & 0.000000 & 0.000000 & NaN & 1.000000 & 0.000000 & 1.000000 & 0.000000 & 0.000000 & "psychological abnormalities and depression and he had attempted suicide with swallowing a metallic object." & 0.000000 & 0.000000 & 1.000000 & 1.000000 & 0.000000 & 0.000000 & Metal, open-end / ring spanner 17 (26x216x75mm) & 0.000000 & 1.000000 & 0.000000 & 0.000000 & 0.000000 & 2020 & Al-Faham, Firas Shaker Mahmoud; Al-Hakkak, Samer Makki Mohamed & The largest esophageal foreign body in adults: a case report & Annals Of Medicine And Surgery (2012) & 10.1016/j.amsu.2020.04.039 & Iraq & Case Report & Al-Faham_2020k & 1 & 1 & 1 & 1 & 1 & 1 & NaN & 1 & Include & JGE & 2025-05-11 00:00:00 \\
32 & 349 & 349-001 & 9.000000 & 1 & 0 & 0 & 0 & 0 & 0 & Female & 0 & 1 & 0 & 0.000000 & 0.000000 & 0.000000 & 0.000000 & 1.000000 & 0.000000 & 0.000000 & NaN & NaN & 1.000000 & NaN & 1.000000 & "adjustment disorder who developed a gastrocolic fistula following the deliberate ingestion of multiple magnets.", "Although she acknowledged this was a wrong and harmful act, she refused to explain her action." & 0.000000 & 1.000000 & 0.000000 & 0.000000 & 0.000000 & 1.000000 & "19 magnetic beads in the stomach", "about 10cm in length" & 0.000000 & 1.000000 & 0.000000 & 0.000000 & 1.000000 & 2020 & Ali, Alaa; Alhindi, Saeed & A child with a gastrocolic fistula after ingesting magnets: an unusual complication & Cureus & 10.7759/cureus.9336 & Bahrain & Case Report & Ali_2020f & 1 & 1 & 1 & 1 & 1 & 1 & 1.000000 & 1 & Include & JGE & 2025-05-11 00:00:00 \\
33 & 360 & 360-001 & 31.000000 & 0 & 0 & 1 & 0 & 0 & 0 & Female & 0 & 1 & 0 & 0.000000 & 0.000000 & 0.000000 & NaN & 1.000000 & 0.000000 & 0.000000 & 0.000000 & 0.000000 & 1.000000 & 0.000000 & 0.000000 & "The patient is a 31-year-old female with a past medical history of cholecystectomy, appendectomy, depression, substance abuse (alcohol), and suicide attempts. She was on psychiatric therapy and anti-depressive drugs. Nonetheless, during the past year, her medical controls became increasingly irregular, her psychiatrist was relocated, and it was difficult for her to attend the appointments. Two weeks before arriving at the emergency room and after her father passed away, she experienced a major depressive episode. These symptoms combined with a lack of adequate therapy led her to ingest two razor blades." & 0.000000 & 0.000000 & 0.000000 & 0.000000 & 1.000000 & 0.000000 & "2 razor blades" & 1.000000 & 1.000000 & 0.000000 & 0.000000 & 1.000000 & 2020 & Delgado Salazar, Jhony Alejandro; Naveda Pacheco, Natalia Carolina; Palacios Jaramillo, Paola Alexandra; Garzón Yépez, Santiago Danilo; Medina Loza, Victor Rafael; Romero Alvarado, Carlos Alberto; Aguilar Ayala, Bernabé Esteban; Molina, Gabriel Alejandro & Ingestion of razor blades, a rare event: a case report in a psychiatric patient & Journal Of Surgical Case Reports & 10.1093/jscr/rjaa094 & Ecuador & Case Report & DelgadoSalazar_2020c & 1 & 1 & 1 & 1 & 1 & 1 & NaN & 1 & Include & JGE & 2025-05-11 00:00:00 \\
35 & 382 & 382-001 & NaN & 0 & 0 & 0 & 0 & 0 & 1 & Male & 1 & 0 & 0 & NaN & 0.000000 & NaN & NaN & NaN & NaN & 0.000000 & NaN & NaN & NaN & NaN & 0.000000 & NaN & 0.000000 & 0.000000 & 0.000000 & 1.000000 & 0.000000 & 0.000000 & "coins" & 1.000000 & 0.000000 & 0.000000 & 0.000000 & 0.000000 & 2021 & Berry, P; Kotha, S & Crying wolf: the danger of recurrent intentional foreign body ingestion & Frontline Gastroenterology & 10.1136/flgastro-2021-101888 & UK & Case Report & Berry_2021e & 1 & 1 & 1 & 0 & 1 & 1 & 1.000000 & 1 & Include & JGE & 2025-05-11 00:00:00 \\
36 & 382 & 382-002 & NaN & 0 & 0 & 0 & 0 & 0 & 1 & Male & 1 & 0 & 0 & NaN & 0.000000 & NaN & NaN & NaN & NaN & 1.000000 & NaN & NaN & NaN & NaN & 0.000000 & NaN & 1.000000 & 0.000000 & 0.000000 & 1.000000 & 0.000000 & 0.000000 & "button battery" & 1.000000 & 0.000000 & 0.000000 & 0.000000 & 1.000000 & 2021 & Berry, P; Kotha, S & Crying wolf: the danger of recurrent intentional foreign body ingestion & Frontline Gastroenterology & 10.1136/flgastro-2021-101888 & UK & Case Report & Berry_2021e & 1 & 1 & 1 & 1 & 1 & 1 & 1.000000 & 1 & Include & JGE & 2025-05-11 00:00:00 \\
38 & 409 & 409-001 & 36.000000 & 0 & 0 & 1 & 0 & 0 & 0 & Male & 1 & 0 & 0 & 1.000000 & 0.000000 & NaN & NaN & 1.000000 & 0.000000 & 1.000000 & NaN & 0.000000 & 1.000000 & 0.000000 & 0.000000 & "Long-standing history of intentional foreign body ingestion (FBI) including multiple laparotomies for foreign body retrievals." & 0.000000 & 0.000000 & 0.000000 & 1.000000 & 1.000000 & 1.000000 & "6.5 cm nail clippers", "in pieces" & 1.000000 & 0.000000 & 0.000000 & 0.000000 & 1.000000 & 2022 & Apikotoa, Sharie; Ballal, Helen; Wijesuriya, Ruwan & Endoscopic foreign body retrieval from the caecum - a case report and push for intervention guidelines & International Journal Of Surgery Case Reports & 10.1016/j.ijscr.2022.106755 & Australia & Case Report & Apikotoa_2022f & 1 & 1 & 1 & 1 & 1 & 1 & NaN & 1 & Include & JGE & 2025-05-11 00:00:00 \\
39 & 414 & 414-001 & 35.000000 & 0 & 0 & 1 & 0 & 0 & 0 & Male & 1 & 0 & 0 & 1.000000 & 0.000000 & NaN & NaN & 0.000000 & 0.000000 & 0.000000 & 0.000000 & 0.000000 & 0.000000 & 0.000000 & 1.000000 & Smuggling - "Patient intentionally ingested a mobile phone trying to smuggle it" & 0.000000 & 0.000000 & 1.000000 & 1.000000 & 0.000000 & 0.000000 & "mobile phone 5.7 cm x 2.5 cm" & 1.000000 & 0.000000 & 0.000000 & 0.000000 & 0.000000 & 2022 & Ali, Ahmed; Mahgoub, Ali M.; Emad, Samar; Abdelfattah, Ahmed H. & Endoscopic retrieval of an ingested mobile phone from the stomach of a prisoner: when gastroenterologists answer the call & Cureus & 10.7759/cureus.33053 & Egypt & Case Report & Ali_2022g & 1 & 1 & 1 & 1 & 1 & 1 & NaN & 1 & Include & JGE & 2025-05-11 00:00:00 \\
40 & 443 & 443-001 & 17.000000 & 1 & 0 & 0 & 0 & 0 & 0 & Female & 0 & 1 & 0 & 0.000000 & 0.000000 & 0.000000 & 0.000000 & 1.000000 & 0.000000 & 1.000000 & 0.000000 & 0.000000 & 1.000000 & 0.000000 & 0.000000 & "The patient is the youngest child of a family with two other siblings. At birth, she was delivered by cesarean section, her mother had a difficult labor, and she was distressed at delivery. She had normal growth and development, and her physical appearance was short and thin. In childhood, she had limited relationships with her peers and never had a close friend. Her parents divorced when she was 6 years old, and the patient lives with her mother and 2 sisters. The patient’s father and aunt have been hospitalized several times for psychiatric disorders. Her father started working as a teacher when he was young, but he was not functional, and as a result, he was given an admin job in school and made redundant shortly after while her mother was a school driver. During her childhood, while her mother was at work, she was physically abused by her father several times and witnessed her father’s suicide, asking her to plug in the wires connected to his body to the electricity main. According to her mother’s testimony, her father had mood swings, breakouts, suicide attempts, and decreased functionality and was diagnosed with major depression and personality disorder. Due to her learning difficulties, the patient continued her education with supportive education, and the onset of the disease exacerbated her learning difficulties. At the age of 13, the patient suddenly developed depression, anxiety, severe restlessness, and loneliness, and ran away from home, she underwent treatment with the diagnosis of type 1 bipolar disorder, and it was noted and confirmed by the patient that before beginning the symptoms, she had a history of sexual abuse -touching her breasts- by an unknown man", "She mentions that she doesn’t swallow foreign bodies to commit suicide, and after these activities, she always calls the emergency department for help. Although she does not appear to be trying to deliberately kill herself, there is always the possibility of death by mishap. The patient has no clear history of hallucinations and delusions, but during her admission to the psychosomatic ward, she was anxious and restless, and fearful of being threatened by red-eyed people disguised in lion covers to harm her." & 0.000000 & 0.000000 & 1.000000 & 1.000000 & 1.000000 & 0.000000 & "detached the knife blade, which measured around 9 cm in length and 1.5 cm in width, from its haft, and swallowed it at first & 0.000000 & 0.000000 & 0.000000 & 1.000000 & 0.000000 & 2023 & Divsalar P.; Mousa S.H.; Abbasi M.H. & Repeated intentional swallowing of foreign objects by an adolescent girl case report & International Journal Of High Risk Behaviors And Addiction & 10.5812/ijhrba-134720 & Iran & Case Report & DivsalarP._2023a & 1 & 1 & 0 & 0 & 1 & 1 & NaN & 1 & Include & JGE & 2025-05-11 00:00:00 \\
41 & 443 & 443-002 & 17.000000 & 1 & 0 & 0 & 0 & 0 & 0 & Female & 0 & 1 & 0 & 0.000000 & 1.000000 & 0.000000 & 0.000000 & 1.000000 & 0.000000 & 1.000000 & 0.000000 & 0.000000 & 1.000000 & 0.000000 & 0.000000 & "She mentions that she doesn’t swallow foreign bodies to commit suicide, and after these activities, she always calls the emergency department for help. Although she does not appear to be trying to deliberately kill herself, there is always the possibility of death by mishap. The patient has no clear history of hallucinations and delusions, but during her admission to the psychosomatic ward, she was anxious and restless, and fearful of being threatened by red-eyed people disguised in lion covers to harm her." & 0.000000 & 0.000000 & 1.000000 & 1.000000 & 1.000000 & 0.000000 & "detached the knife blade, which measured around 9 cm in length and 1.5 cm in width, from its haft, and swallowed it at first & 0.000000 & 1.000000 & 0.000000 & 0.000000 & 1.000000 & 2023 & Divsalar P.; Mousa S.H.; Abbasi M.H. & Repeated intentional swallowing of foreign objects by an adolescent girl case report & International Journal Of High Risk Behaviors And Addiction & 10.5812/ijhrba-134720 & Iran & Case Report & DivsalarP._2023a & 1 & 1 & 0 & 0 & 1 & 1 & NaN & 1 & Include & JGE & 2025-05-11 00:00:00 \\
42 & 443 & 443-003 & 17.000000 & 1 & 0 & 0 & 0 & 0 & 0 & Female & 0 & 1 & 0 & 0.000000 & 1.000000 & 0.000000 & 0.000000 & 1.000000 & 0.000000 & 1.000000 & 0.000000 & 0.000000 & 1.000000 & 0.000000 & 0.000000 & "She mentions that she doesn’t swallow foreign bodies to commit suicide, and after these activities, she always calls the emergency department for help. Although she does not appear to be trying to deliberately kill herself, there is always the possibility of death by mishap. The patient has no clear history of hallucinations and delusions, but during her admission to the psychosomatic ward, she was anxious and restless, and fearful of being threatened by red-eyed people disguised in lion covers to harm her." & 0.000000 & 0.000000 & 1.000000 & 1.000000 & 1.000000 & 0.000000 & "detached the knife blade, which measured around 9 cm in length and 1.5 cm in width, from its haft, and swallowed it at first & 0.000000 & 1.000000 & 0.000000 & 0.000000 & 0.000000 & 2023 & Divsalar P.; Mousa S.H.; Abbasi M.H. & Repeated intentional swallowing of foreign objects by an adolescent girl case report & International Journal Of High Risk Behaviors And Addiction & 10.5812/ijhrba-134720 & Iran & Case Report & DivsalarP._2023a & 1 & 1 & 0 & 0 & 1 & 1 & NaN & 1 & Include & JGE & 2025-05-11 00:00:00 \\
43 & 456 & 456-001 & 45.000000 & 0 & 0 & 0 & 1 & 0 & 0 & Female & 0 & 1 & 0 & 0.000000 & 0.000000 & NaN & NaN & 1.000000 & 0.000000 & NaN & 0.000000 & 0.000000 & 1.000000 & 1.000000 & 0.000000 & "he reported visiting her Rastafarian, who made her a “Tack Shake” to help with her symptoms of anxiety and depression. She reported drinking this mixture the day prior to the presentation.", "This patient was ultimately admitted to an inpatient psychiatric facility for suicidal ideation in the setting of multiple comorbid psychiatric disorders. The ingested "tack-shake" at the instruction of her Rastafarian appears to be a previously unreported alternative medicine for anxiety. The authors were unable to locate any prior mention of this practice in the literature, and without evidence of a delusional disorder on psychiatric evaluation, her self-report was presumed to be an honest description." & 0.000000 & 0.000000 & 0.000000 & 0.000000 & 1.000000 & 1.000000 & "Four nails in the colon and two adjacent screws in the small bowel" & 1.000000 & 0.000000 & 0.000000 & 0.000000 & 0.000000 & 2023 & Hardy, John C.; Ashcroft, Cody; Kay, Carl; Liane, Billy-Joe; Horn, Christian & Loose screws: removal of foreign bodies from the lower gastrointestinal tract & Cureus & 10.7759/cureus.43093 & USA & Case Report & Hardy_2023g & 1 & 1 & 1 & 1 & 1 & 1 & NaN & 1 & Include & JGE & 2025-05-11 00:00:00 \\
44 & 471 & 471-001 & 58.000000 & 0 & 0 & 0 & 1 & 0 & 0 & Female & 0 & 1 & 0 & 0.000000 & 0.000000 & NaN & NaN & 1.000000 & 0.000000 & 0.000000 & NaN & 0.000000 & 1.000000 & 0.000000 & 0.000000 & "During her hospital stay, she was diagnosed with separation anxiety and was started on mirtazapine" & 0.000000 & 0.000000 & 1.000000 & 1.000000 & 0.000000 & 1.000000 & "multiple folded sheets of plastic table cover", "13 cm in lengthand 4 cm in width", "1 90cm proximal to ileocecal valve and one in stomach. & 1.000000 & 0.000000 & 0.000000 & 0.000000 & 1.000000 & 2024 & Sultan, Noran; Attar, Hanin; Sembawa, Hatem; Alharthi, Hind & A plastic bezoar causing bowel obstruction: a case of table cover ingestion & International Journal Of Surgery Case Reports & 10.1016/j.ijscr.2024.109506 & Saudi Arabia & Case Report & Sultan_2024f & 1 & 1 & 1 & 1 & 1 & 1 & 1.000000 & 1 & Include & JGE & 2025-05-11 00:00:00 \\
45 & 482 & 482-001 & 36.000000 & 0 & 0 & 1 & 0 & 0 & 0 & Male & 1 & 0 & 0 & 0.000000 & 0.000000 & NaN & NaN & 1.000000 & 0.000000 & 0.000000 & NaN & NaN & 1.000000 & 0.000000 & 0.000000 & "a psychiatric consultation was conducted, resulting in a diagnosis of psychosis", "suspected psychiatric problems (unconfirmed)", "The patient had no history of surgery. He did not take any special medication, but he was strongly addicted to opium." & 0.000000 & 0.000000 & 0.000000 & 0.000000 & 1.000000 & 1.000000 & "gradually swallowing small metal objects such as screws, nuts, plaques, and stones for the past 3  months, which had not caused any serious problems for the patient until now.", "A total of 452 screws, nails, nuts, keys, stones, and other metal pieces weighing two kilograms and nine hundred grams were extracted from the stomach" & 0.000000 & 1.000000 & 0.000000 & 0.000000 & 1.000000 & 2024 & Farhadi, Farbod; Mohtadi, Ahmadreza; Pakmehr, Mostafa; Ghaedamini, Hossein; Shafieian, Fatemeh; Aminifar, Seyed Abolfazl & This is a successful removal of more than 450 pieces of metal objects from a patient's stomach: a case report & Journal Of Medical Case Reports & 10.1186/s13256-024-04672-3 & Iran & Case Report & Farhadi_2024h & 1 & 1 & 1 & 1 & 1 & 1 & NaN & 1 & Include & JGE & 2025-05-11 00:00:00 \\
46 & 484 & 484-001 & 29.000000 & 0 & 0 & 1 & 0 & 0 & 0 & Unknown & 0 & 0 & 1 & 0.000000 & 1.000000 & NaN & NaN & 1.000000 & 0.000000 & 1.000000 & 1.000000 & 0.000000 & 0.000000 & 0.000000 & 0.000000 & "maintaining suicidal ideas and with a high risk of escape", "Borderline personality disorder", "on six occasions in different hospitals, in all cases laparotomy was performed to resolve the condition [extraction of various foreign bodies of different kinds; chains, clips, batteries, plastic objects...], beginning his surgical journey in 2018. Apart from the six previous occasions in which she has required surgery, five successful endoscopic extractions have also been performed during this period" & 0.000000 & 0.000000 & 0.000000 & 1.000000 & 0.000000 & 1.000000 & "6 dominoes" & 0.000000 & 1.000000 & 0.000000 & 0.000000 & 0.000000 & 2024 & FJ Buils & Repeated Behavior Of Deliberate Foreign Body Ingestion In A Patient With Psychiatric Disorder & A Case Report. Clin Surg & 10.52916/jmrs244144 & Spain & Case Report & fjbuilsRepeatedBehaviorDeliberate2024 & 1 & 1 & 1 & 1 & 1 & 1 & 1.000000 & 1 & Include & JGE & 2025-05-12 00:00:00 \\
47 & 485 & 485-001 & 45.000000 & 0 & 0 & 0 & 1 & 0 & 0 & Male & 1 & 0 & 0 & 0.000000 & 0.000000 & NaN & NaN & 0.000000 & 0.000000 & 0.000000 & 0.000000 & 1.000000 & 0.000000 & 0.000000 & 0.000000 & "no significant past medical history presented t", "thought to have been consumed for secondary gain." & 1.000000 & 1.000000 & 0.000000 & 0.000000 & 0.000000 & 1.000000 & "Four days after the unwitnessed ingestion of “two magnets,” thought to have been consumed for secondary gain.", "20 mm button battery was identified in the upper esophagus at the level of the cricopharyngeus and was removed with a Roth net"" & 1.000000 & 0.000000 & 0.000000 & 0.000000 & 1.000000 & 2024 & Bhumi, Sriya; Mago, Sheena; Mavilia-Scranton, Marianna G.; Birk, John W.; Rezaizadeh, Houman & Esophageal button battery retrieval: time-in may not be everything & Cureus & 10.7759/cureus.58327 & USA & Case Report & Bhumi_2024f & 1 & 1 & 1 & 1 & 1 & 1 & 1.000000 & 1 & Include & JGE & 2025-05-12 00:00:00 \\
48 & 535 & 535-001 & 75.000000 & 0 & 0 & 0 & 0 & 1 & 0 & Female & 0 & 1 & 0 & 0.000000 & 0.000000 & 0.000000 & 0.000000 & 1.000000 & 0.000000 & 0.000000 & 0.000000 & 0.000000 & NaN & 0.000000 & 0.000000 & "pica" & 0.000000 & 0.000000 & 0.000000 & NaN & 0.000000 & 1.000000 & "£175.32 in coins" & 0.000000 & 1.000000 & 0.000000 & 0.000000 & 1.000000 & 1998 & Beecroft, N.; Bach, L.; Tunstall, N.; Howard, R. & An unusual case of pica & International Journal of Geriatric Psychiatry & 10.1002/(sici)1099-1166(199809)13:9<638::aid-gps837>3.0.co;2-n & UK & Case Report & Beecroft_1998 & 1 & 1 & 1 & 1 & 1 & 1 & 1.000000 & 1 & Include & JGE & 2025-05-12 00:00:00 \\
49 & 543 & 543-001 & 54.000000 & 0 & 0 & 0 & 1 & 0 & 0 & Male & 1 & 0 & 0 & 0.000000 & 0.000000 & 0.000000 & 0.000000 & 1.000000 & 0.000000 & 0.000000 & 0.000000 & 0.000000 & 1.000000 & 0.000000 & 0.000000 & "pica", "schizophrenia with a 15 year history of ingestion" & 0.000000 & 0.000000 & 0.000000 & 1.000000 & 0.000000 & 1.000000 & Repeated, chronic coin ingestion. "The coins weighed 1,870 g. There were also multiple screws and nuts." & 0.000000 & 0.000000 & 1.000000 & 1.000000 & 1.000000 & 2001 & Kumar, A.; Jazieh, A. R. & Case report of sideroblastic anemia caused by ingestion of coins & American Journal of Hematology & 10.1002/1096-8652(200102)66:2<126::AID-AJH1029>3.0.CO;2-J & USA & Case Report & Kumar_2001 & 1 & 1 & 1 & 1 & 1 & 1 & 1.000000 & 1 & Include & JGE & 2025-05-12 00:00:00 \\
50 & 548 & 548-001 & 9.000000 & 1 & 0 & 0 & 0 & 0 & 0 & Female & 0 & 1 & 0 & 0.000000 & 0.000000 & NaN & 0.000000 & 0.000000 & 0.000000 & 0.000000 & 0.000000 & 0.000000 & 0.000000 & 1.000000 & 0.000000 & "reportedly intended to allow the application of jewellery" & 0.000000 & 1.000000 & 0.000000 & 1.000000 & 0.000000 & 1.000000 & "12 small powerful magnets, each measuring 7 mm × 5 mm, on separate occasions a week before her illness" & 0.000000 & 1.000000 & 0.000000 & 0.000000 & 1.000000 & 2002 & Cauchi, J. A.; Shawis, R. N. & Multiple magnet ingestion and gastrointestinal morbidity & Archives of Disease in Childhood & 10.1136/adc.87.6.539 & UK & Case Report & Cauchi_2002 & 1 & 1 & 1 & 1 & 1 & 1 & 1.000000 & 1 & Include & JGE & 2025-05-12 00:00:00 \\
51 & 558 & 558-001 & 9.000000 & 1 & 0 & 0 & 0 & 0 & 0 & Male & 1 & 0 & 0 & 0.000000 & 0.000000 & 0.000000 & 0.000000 & 0.000000 & 0.000000 & 0.000000 & 0.000000 & 0.000000 & 0.000000 & 1.000000 & 0.000000 & "attempt to conceal them from his mother, who
restricted him from wearing earrings." & 0.000000 & 1.000000 & 0.000000 & 0.000000 & 0.000000 & 1.000000 & "2 round metallic earrings" & 0.000000 & 1.000000 & 0.000000 & 0.000000 & 1.000000 & 2004 & Tay, Ee Tein; Weinberg, Gerard; Levin, Terry L. & Ingested magnets: the force within & Pediatric Emergency Care & 10.1097/01.pec.0000134926.03030.a7 & USA & Case Report & Tay_2004 & 1 & 1 & 1 & 1 & 1 & 1 & 1.000000 & 1 & Include & JGE & 2025-05-12 00:00:00 \\
52 & 565 & 565-001 & 7.000000 & 1 & 0 & 0 & 0 & 0 & 0 & Male & 1 & 0 & 0 & 0.000000 & 0.000000 & 0.000000 & 0.000000 & 1.000000 & 1.000000 & 1.000000 & 0.000000 & 0.000000 & 1.000000 & 0.000000 & 0.000000 & "history of developmental delay and agenesis of the corpus callosum, "The patient is ambulatory and has a history of pica" & 0.000000 & 1.000000 & 0.000000 & 0.000000 & 0.000000 & 1.000000 & "0 foreign bodies arranged in tandem", "magnetic construction toys (‘‘Supermag’’ by Plastwood)" & 1.000000 & 1.000000 & 0.000000 & 0.000000 & 1.000000 & 2005 & Liu, Steven; de Blacam, Catherine; Lim, Foong-Yen; Mattei, Peter; Mamula, Petar & Magnetic foreign body ingestions leading to duodenocolonic fistula & Journal of Pediatric Gastroenterology and Nutrition & 10.1097/01.mpg.0000177703.99786.c9 & USA & Case Report & Liu_2005 & 1 & 1 & 1 & 1 & 1 & 0 & 1.000000 & 1 & Include & JGE & 2025-05-12 00:00:00 \\
53 & 567 & 567-001 & 7.000000 & 1 & 0 & 0 & 0 & 0 & 0 & Female & 0 & 1 & 0 & 0.000000 & NaN & NaN & 0.000000 & 1.000000 & 1.000000 & NaN & 0.000000 & 0.000000 & 1.000000 & 0.000000 & 0.000000 & "7-year-old autistic girl with parorexia" & 0.000000 & 1.000000 & 1.000000 & 1.000000 & 0.000000 & 1.000000 & "parts of a magnetic construction toy" & 1.000000 & 0.000000 & 0.000000 & 0.000000 & 1.000000 & 2005 & Ohno, Yasuharu; Yoneda, Akira; Enjoji, Akihito; Furui, Junichiro; Kanematsu, Takashi & Gastroduodenal fistula caused by ingested magnets & Gastrointestinal Endoscopy & 10.1016/s0016-5107(04)02387-9 & Japan & Case Report & Ohno_2005 & 1 & 1 & 1 & 1 & 1 & 1 & 1.000000 & 1 & Include & JGE & 2025-05-12 00:00:00 \\
54 & 568 & 568-001 & 9.000000 & 1 & 0 & 0 & 0 & 0 & 0 & Female & 0 & 1 & 0 & 0.000000 & 0.000000 & NaN & 0.000000 & 0.000000 & 0.000000 & 0.000000 & 0.000000 & 0.000000 & 0.000000 & 1.000000 & 0.000000 & "in humorous intention—swallowed 5 of those small compounds" & 0.000000 & 1.000000 & 0.000000 & 0.000000 & 0.000000 & 1.000000 & "ingested small magnets belonging to a creative game (3-dimensional figures are created using strong magnet bars, which are composed of small equal-sized compounds). The size of the ingested magnets was described as 6 _x0001_ 3 mm" & 0.000000 & 1.000000 & 0.000000 & 0.000000 & 1.000000 & 2005 & Wildhaber, Barbara E.; Le Coultre, Claude; Genin, Bernard & Ingestion of magnets: innocent in solitude, harmful in groups & Journal of Pediatric Surgery & 10.1016/j.jpedsurg.2005.06.022 & Switzerland & Case Report & Wildhaber_2005 & 1 & 1 & 1 & 1 & 1 & 1 & 1.000000 & 1 & Include & JGE & 2025-05-12 00:00:00 \\
55 & 578 & 578-001 & 33.000000 & 0 & 0 & 1 & 0 & 0 & 0 & Female & 0 & 1 & 0 & 0.000000 & NaN & NaN & NaN & NaN & NaN & NaN & 1.000000 & 0.000000 & 0.000000 & NaN & 0.000000 & "swallowed 2 years previously as a suicide attempt" & 0.000000 & 0.000000 & 1.000000 & 1.000000 & 0.000000 & 0.000000 & "toothbrush" & 0.000000 & 1.000000 & 0.000000 & 0.000000 & 1.000000 & 2007 & Cox, David; Donohue, Peter; Costa, Vanda & A swallowed toothbrush causing perforation 2 years after ingestion & British Journal of Hospital Medicine (London, England: 2005) & 10.12968/hmed.2007.68.10.27330 & UK & Case Report & Cox_2007 & 1 & 1 & 1 & 1 & 1 & 1 & 1.000000 & 1 & Include & JGE & 2025-05-12 00:00:00 \\
56 & 582 & 582-001 & 20.000000 & 0 & 1 & 0 & 0 & 0 & 0 & Female & 0 & 1 & 0 & 0.000000 & 0.000000 & NaN & NaN & 0.000000 & 0.000000 & 1.000000 & 0.000000 & 0.000000 & 1.000000 & 1.000000 & 0.000000 & "According to her, she developed this habit, more than a month back, after she was forced to marry against her wish. Incidentally, she had returned to her parents within a few days of her marriage as she had aversion towards sexual relation with her husband. After being admonished by her parents for this, she suddenly developed the desire to ingest metallic objects, mainly nails obtained from a local hardware shop. She had lost her normal appetite, felt anxious and agitated in the absence of these metallic objects, and experienced a profound sense of satisfaction after consuming them", "suffering from deviant sexual behavior and marriage against her wish acted as a stressor in the development of this habit", "initially diagnosed to be suffering from “pica”. On subsequent interrogation, she gave the history of sexual abuse at the age of six years and found to have homosexual inclinations. These could explain her postmarital abnormal behavior.", "she was found to be suffering from depressive disorder, impulse control disorder (not otherwise specified), and pica. No personality disorder was detected" & 0.000000 & 0.000000 & 0.000000 & 0.000000 & 1.000000 & 1.000000 & "handful of metallic substances, mainly nails and pins of various sizes. Apparently she was doing so over the last month, but had no abdominal complaints." & 0.000000 & 0.000000 & 0.000000 & 1.000000 & 0.000000 & 2008 & Bhattacharjee, Prosanta; Singh, Om & Repeated ingestion of sharp-pointed metallic objects & Archives of Iranian medicine & Not available. & India & Case Report & Bhattacharjee_2008 & 1 & 1 & 1 & 1 & 1 & 1 & 1.000000 & 1 & Include & JGE & 2025-05-12 00:00:00 \\
57 & 586 & 586-001 & 20.000000 & 0 & 1 & 0 & 0 & 0 & 0 & Female & 0 & 1 & 0 & 0.000000 & 0.000000 & NaN & NaN & 0.000000 & 0.000000 & NaN & NaN & 0.000000 & 1.000000 & 0.000000 & 0.000000 & "Psychiatric opinion was taken, patient was diagnosed to have depressive disorder and a possibality of pica was considered and was started on antidepressive drugs." & 0.000000 & 0.000000 & 1.000000 & 1.000000 & 1.000000 & 1.000000 & "53 pieces of glass bangles which were ranging from 3–7 cm with sharp edges. These accounted for total of around 18 complete bangles" & 1.000000 & 1.000000 & 0.000000 & 0.000000 & 1.000000 & 2008 & Kariholu, P. L.; Jakareddy, R.; Hemanthkumar, M.; Paramesh, K. N.; Pavankumar, N. P. & Pica - a case of acuphagia or hyalophagia? & The Indian Journal of Surgery & 10.1007/s12262-008-0040-x & India & Case Report & Kariholu_2008 & 1 & 1 & 1 & 1 & 1 & 1 & 1.000000 & 1 & Include & JGE & 2025-05-13 00:00:00 \\
58 & 594 & 594-001 & 22.000000 & 0 & 1 & 0 & 0 & 0 & 0 & Male & 1 & 0 & 0 & 0.000000 & 0.000000 & NaN & NaN & 0.000000 & 0.000000 & 0.000000 & 0.000000 & 0.000000 & 1.000000 & 0.000000 & 0.000000 & "in order to “treat” his pulmonary TB. He said “I have TB and I have metal in me. My stomach asks me to put metal in it to cure my TB. They make me eat metal, they put metal in me”. According to his mother, he behaviour had been odd since the death of his father, with whom he was very close, a few months previously.", "Psychiatric consultation was sought and he was found to have chronic psychosis–persistent delusional disorder." & 0.000000 & 0.000000 & 1.000000 & 0.000000 & 1.000000 & 1.000000 & "A gastrostomy was performed and several instruments, blades, batteries and even gold jewellery" & 0.000000 & 1.000000 & 0.000000 & 0.000000 & 1.000000 & 2009 & Yasin, Malik Amjad; Malik, Ghazala Nasreen; Malik, Sikandar Ali; Rathore, Farooq Azam & Metal in stomach: a rare cause of gastric bezoar & BMJ Case Reports & 10.1136/bcr.06.2008.0278 & Pakistan & Case Report & Yasin_2009 & 1 & 1 & 1 & 1 & 1 & 1 & NaN & 1 & Include & JGE & 2025-05-13 00:00:00 \\
59 & 598 & 598-001 & 38.000000 & 0 & 0 & 1 & 0 & 0 & 0 & Male & 1 & 0 & 0 & 0.000000 & 0.000000 & NaN & NaN & 1.000000 & 0.000000 & 0.000000 & 1.000000 & 0.000000 & 0.000000 & 0.000000 & 0.000000 & "history of schizophrenia", "in a suicide attempt" & 0.000000 & 0.000000 & 1.000000 & 0.000000 & 1.000000 & 1.000000 & "2 nails", "6-cm nail" & 0.000000 & 1.000000 & 0.000000 & 0.000000 & 1.000000 & 2010 & Fry, Emily; Counselman, Francis L. & A right scrotal abscess and foreign body ingestion in a schizophrenic patient & The Journal of Emergency Medicine & 10.1016/j.jemermed.2007.07.018 & USA & Case Report & Fry_2010 & 1 & 1 & 1 & 1 & 1 & 1 & 1.000000 & 1 & Include & JGE & 2025-05-13 00:00:00 \\
60 & 602 & 602-001 & 41.000000 & 0 & 0 & 0 & 1 & 0 & 0 & Male & 1 & 0 & 0 & 0.000000 & 1.000000 & NaN & NaN & 1.000000 & 1.000000 & 1.000000 & 0.000000 & 0.000000 & 1.000000 & 0.000000 & 0.000000 & "intellectual disability", "explained his outbursts of compulsive behavior as a way to cope with stressful situations and to stop circulating thoughts", "The psychiatrist and health workers who provided care to the patient at his home partly agreed with this interpretation, but also saw a strong malingering character in his actions." & 0.000000 & 0.000000 & 1.000000 & 1.000000 & 1.000000 & 1.000000 & "20 small, mostly sharp objects (Figure 1), such as nails and spirals, and a suede leather glove with a diameter of about 15cm" & 1.000000 & 0.000000 & 0.000000 & 0.000000 & 0.000000 & 2010 & te Wildt, Bert T.; Tettenborn, Christian; Schneider, Udo; Ohlmeier, Martin D.; Zedler, Markus; Zakhalev, Roman; Krueger, Martin & Swallowing Foreign Bodies as an Example of Impulse Control Disorder in a Patient With Intellectual Disabilities & Psychiatry (Edgmont) & Not available. & Germany & Case Report & teWildt_2010 & 1 & 1 & 1 & 1 & 1 & 1 & NaN & 1 & Include & JGE & 2025-05-13 00:00:00 \\
61 & 617 & 617-001 & 29.000000 & 0 & 0 & 1 & 0 & 0 & 0 & Female & 0 & 1 & 0 & 0.000000 & 0.000000 & NaN & NaN & 1.000000 & 0.000000 & NaN & 0.000000 & NaN & 1.000000 & 0.000000 & 0.000000 & "history of psychiatric disorder that was treated with thorazine and lithium, but was not taking any of her medications", "denied any intentions to hurt herself" & 0.000000 & 0.000000 & 0.000000 & 0.000000 & 1.000000 & 1.000000 & "ten razor blades after coating them with chewing gum" & 0.000000 & 0.000000 & 0.000000 & 1.000000 & 0.000000 & 2013 & Ataya, A.; Alraiyes, A.H.; Alraies, M.C. & Razor blades in the stomach & QJM: An International Journal of Medicine & 10.1093/qjmed/hcs165 & USA & Case Report & Ataya_2013 & 1 & 1 & 1 & 1 & 1 & 1 & NaN & 1 & Include & JGE & 2025-05-13 00:00:00 \\
62 & 622 & 622-001 & 100.000000 & 0 & 0 & 0 & 0 & 1 & 0 & Female & 0 & 1 & 0 & 0.000000 & 0.000000 & NaN & NaN & 0.000000 & 0.000000 & 0.000000 & 1.000000 & 0.000000 & 1.000000 & 1.000000 & 0.000000 & "suicide attempt due to intolerable pain induced by a fracture she suffered 3 mo previously. She was bedridden and had a depressed mood, which caused her to attempt suicide" & 0.000000 & 0.000000 & 0.000000 & 0.000000 & 0.000000 & 1.000000 & "26 coins, one ferrous ring and one cylindrical plastic object were retrieved" & 1.000000 & 0.000000 & 0.000000 & 0.000000 & 0.000000 & 2013 & Li, Quan-Peng; Ge, Xian-Xiu; Ji, Guo-Zhong; Fan, Zhi-Ning; Zhang, Fa-Ming; Wang, Yun; Miao, Lin & Endoscopic retrieval of 28 foreign bodies in a 100-year-old female after attempted suicide & World Journal of Gastroenterology & 10.3748/wjg.v19.i25.4091 & China & Case Report & Li_2013 & 1 & 1 & 1 & 1 & 1 & 1 & 0.000000 & 1 & Include & JGE & 2025-05-13 00:00:00 \\
63 & 623 & 623-001 & 26.000000 & 0 & 0 & 1 & 0 & 0 & 0 & Male & 1 & 0 & 0 & 0.000000 & NaN & NaN & NaN & 1.000000 & 0.000000 & NaN & NaN & NaN & 1.000000 & NaN & 0.000000 & "known psychiatric illness", "ingesting sewing needles by wrapping it on a plant leaf out of a schizophrenic disorder" & 0.000000 & 0.000000 & 1.000000 & 1.000000 & 1.000000 & 1.000000 & "ingesting sewing needles by wrapping it on a plant leaf out of a schizophrenic disorder", "8cm long" & 0.000000 & 1.000000 & 0.000000 & 0.000000 & 1.000000 & 2013 & Misra, S.; Jain, V.; Ahmad, F.; Kumar, R.; Kishore, N. & Metallic sewing needle ingestion presenting as acute abdomen & Nigerian Journal of Clinical Practice & 10.4103/1119-3077.116879 & India & Case Report & Misra_2013 & 1 & 1 & 1 & 1 & 1 & 1 & 1.000000 & 1 & Include & JGE & 2025-05-13 00:00:00 \\
64 & 626 & 626-001 & 32.000000 & 0 & 0 & 1 & 0 & 0 & 0 & Female & 0 & 1 & 0 & 0.000000 & 0.000000 & NaN & NaN & 0.000000 & 0.000000 & 0.000000 & 0.000000 & 0.000000 & 1.000000 & 0.000000 & 0.000000 & "she under the influence of black magic (?) developed a psychic illness and ingested a knife" & 0.000000 & 0.000000 & 1.000000 & 1.000000 & 1.000000 & 0.000000 & "17.78cm knife", "in situ 7 years" & 1.000000 & 1.000000 & 0.000000 & 0.000000 & 1.000000 & 2014 & Bhasin, Sanjay K.; Kachroo, S. L.; Kumar, Vijay; Kumar, Raj; Chandail, Viajaynt Singh & 7" long knife for 7 years in the duodenum: a rare case report and review of literature & International Surgery Journal & Not available. & India & Case Report & Bhasin_2014 & 1 & 1 & 1 & 1 & 1 & 1 & 1.000000 & 1 & Include & JGE & 2025-05-13 00:00:00 \\
65 & 637 & 637-001 & 22.000000 & 0 & 1 & 0 & 0 & 0 & 0 & Female & 0 & 1 & 0 & 0.000000 & 0.000000 & NaN & NaN & 1.000000 & 0.000000 & 0.000000 & 0.000000 & 0.000000 & 1.000000 & 0.000000 & 0.000000 & "In the context of schizophrenia, the development of pica has various potential causes: • Long standing malnutrition related to an underlying chronic psychotic illness • Psychotropic induced compulsive eating behavior of inedible substances[4] • Hematopoietic suppression resulting from a. Chronic psychotic illness b. Chronic use of psychotropics. • A feature of disorganized behavior. Pica is believed to be an obsessive‑compulsi" & 0.000000 & 0.000000 & 0.000000 & 0.000000 & 0.000000 & 1.000000 & "she was eating clay and pieces of brick for approximately 2 months prior to hospitalization" & 0.000000 & 0.000000 & 0.000000 & 1.000000 & 0.000000 & 2015 & Kar, Sujita Kumar; Kamboj, Abhilove; Kumar, Rajesh & Pica and psychosis - clinical attributes and correlations: a case report & Journal of Family Medicine and Primary Care & 10.4103/2249-4863.152277 & India & Case Report & Kar_2015 & 1 & 1 & 1 & 1 & 1 & 1 & 1.000000 & 1 & Include & JGE & 2025-05-13 00:00:00 \\
66 & 640 & 640-001 & 24.000000 & 0 & 1 & 0 & 0 & 0 & 0 & Male & 1 & 0 & 0 & 0.000000 & 0.000000 & NaN & NaN & 1.000000 & 0.000000 & 0.000000 & 0.000000 & 0.000000 & 1.000000 & 1.000000 & 0.000000 & Background of paranoid schizophrenia, "been swallowing metallic items “out of boredom” for the previous 12 months." & 0.000000 & 0.000000 & 1.000000 & 1.000000 & 1.000000 & 1.000000 & "50 metal objects" & 0.000000 & 1.000000 & 0.000000 & 0.000000 & 1.000000 & 2015 & Kobiela, Jarek; Mittlener, Stanislaw; Gorycki, Tomasz; Lachinski, Andrzej; Adrych, Krystian & Vast collection of foreign bodies in the stomach presenting as acute gastrointestinal bleeding in a patient with schizophrenia & Endoscopy & 10.1055/s-0034-1392611 & Poland & Case Report & Kobiela_2015 & 1 & 1 & 1 & 1 & 1 & 1 & 1.000000 & 1 & Include & JGE & 2025-05-13 00:00:00 \\
67 & 643 & 643-001 & 21.000000 & 0 & 1 & 0 & 0 & 0 & 0 & Male & 1 & 0 & 0 & 0.000000 & NaN & NaN & NaN & 1.000000 & 1.000000 & 0.000000 & 0.000000 & 0.000000 & 1.000000 & 0.000000 & 0.000000 & "the patient’s swallowing of the lighters was associated with impulsive behavior due to mental retardation" & 0.000000 & 0.000000 & 1.000000 & 1.000000 & 0.000000 & 1.000000 & "12 lighters" & 1.000000 & 1.000000 & 0.000000 & 0.000000 & 1.000000 & 2016 & Atayan, Yahya; Cagin, Yasir Furkan; Erdogan, Mehmet Ali; Bilgic, Y∆í¬±lmaz; Bestas, Remzi; Harputluoglu, Murat; Seckin, Y‚àö¬∫ksel & Lighter Ingestion as an Uncommon Cause of Severe Vomiting in a Schizophrenia Patient & Case Reports in Gastrointestinal Medicine & 10.1155/2016/6301302 & Turkey & Case Report & Atayan_2016 & 1 & 1 & 1 & 1 & 1 & 1 & 0.000000 & 1 & Include & JGE & 2025-05-13 00:00:00 \\
68 & 654 & 654-001 & 21.000000 & 0 & 1 & 0 & 0 & 0 & 0 & Male & 1 & 0 & 0 & 0.000000 & 0.000000 & NaN & NaN & 0.000000 & 0.000000 & 0.000000 & 1.000000 & 0.000000 & 0.000000 & 0.000000 & 0.000000 & "The patient claimed that he swallowed a toothbrush about a year ago in an attempt to commit suicide" & 0.000000 & 0.000000 & 1.000000 & 1.000000 & 0.000000 & 0.000000 & "17 cm in length (toothbrush)" & 0.000000 & 1.000000 & 0.000000 & 0.000000 & 1.000000 & 2017 & Ali, Syed Muhammad & Duodenal Perforation by Swallowed Toothbrush: Case Report and Review of Literature & Open Access Journal of Surgery & 10.19080/OAJS.2017.04.555632 & Qatar & Case Report & Ali_2017 & 1 & 1 & 1 & 1 & 1 & 1 & 1.000000 & 1 & Include & JGE & 2025-05-13 00:00:00 \\
69 & 658 & 658-001 & 44.000000 & 0 & 0 & 0 & 1 & 0 & 0 & Male & 1 & 0 & 0 & NaN & NaN & NaN & NaN & 1.000000 & NaN & NaN & 1.000000 & 0.000000 & 0.000000 & 0.000000 & 0.000000 & "Suicide Attempt", "interlectual disability" & 0.000000 & 0.000000 & 1.000000 & 1.000000 & 1.000000 & 1.000000 & "238 screws and nails" & 1.000000 & 1.000000 & 0.000000 & 0.000000 & 1.000000 & 2018 & Camacho Dorado, Cristina; S√°nchez Gallego, Alba; Miota de Llama, Jose Ignacio; Gonz√°lez Masi√°, Jose Antonio & Metallic bezoar after suicide attempt & Cirugia Espanola & 10.1016/j.ciresp.2018.02.015 & Spain & Case Report & CamachoDorado_2018 & 1 & 0 & 1 & 0 & 1 & 1 & NaN & 0 & Include & JGE & 2025-05-13 00:00:00 \\
70 & 661 & 661-001 & 44.000000 & 0 & 0 & 0 & 1 & 0 & 0 & Male & 1 & 0 & 0 & 0.000000 & 0.000000 & NaN & NaN & 0.000000 & 0.000000 & 1.000000 & 0.000000 & 0.000000 & 1.000000 & 0.000000 & 1.000000 & "In our case, the patient had a history of amphetamine abuse, and acuphagia might have occurred due to hallucinations caused by amphetamine.", "pica", "He had no history of psychological problems or psychotic behaviors and beliefs" & 0.000000 & 0.000000 & 1.000000 & 1.000000 & 1.000000 & 1.000000 & "On the autopsy, 64 bolts and metal fittings (3700 grams) were found in the esophagus, stomach, small intestine, and large intestine of the patient" & 0.000000 & 0.000000 & 1.000000 & 1.000000 & 1.000000 & 2018 & Emamhadi, Mohammad Ali; Najari, Fares; Hedayatshode, Mohammad Javad; Sharif, Shokoufeh & Sudden Death Following Oral Intake of Metal Objects (Acuphagia): a Case Report & Emergency (Tehran, Iran) & Not available. & Iran & Case Report & Emamhadi_2018 & 1 & 1 & 1 & 1 & 1 & 1 & 1.000000 & 1 & Include & JGE & 2025-05-13 00:00:00 \\
71 & 667 & 667-001 & 91.000000 & 0 & 0 & 0 & 0 & 1 & 0 & Female & 0 & 1 & 0 & 0.000000 & NaN & NaN & NaN & 1.000000 & 1.000000 & NaN & 0.000000 & 0.000000 & 1.000000 & 0.000000 & 0.000000 & "dementia", "She was unaware of how the sheet of plastic was ingested" & 0.000000 & 0.000000 & 1.000000 & 1.000000 & 0.000000 & 0.000000 & "plastic shopping bag" & 0.000000 & 1.000000 & 0.000000 & 0.000000 & 1.000000 & 2019 & Kerestes, T.; Smith, J. & Paper or Plastic? A Foreign Body Ingestion Leading to Small Bowel Obstruction. A Case Report & ARC Journal of Clinical Case Reports & 10.20431/2455-9806.0502002 & USA & Case Report & Kerestes_2019 & 1 & 1 & 1 & 1 & 1 & 1 & NaN & 1 & Include & JGE & 2025-05-13 00:00:00 \\
72 & 680 & 680-001 & 22.000000 & 0 & 1 & 0 & 0 & 0 & 0 & Male & 1 & 0 & 0 & NaN & NaN & NaN & NaN & 1.000000 & 0.000000 & NaN & 1.000000 & 0.000000 & 0.000000 & 0.000000 & 0.000000 & "He swallowed metallic nails in an attempt to kill himself" & 0.000000 & 0.000000 & 1.000000 & 1.000000 & 1.000000 & 1.000000 & "sixty curved and straight nails, needle and wires" & 0.000000 & 1.000000 & 0.000000 & 0.000000 & 1.000000 & 2022 & Mesfin, Telila; Tekalegn, Yohannes; Aman, Mudesir; Geta, Girma; Ketema, Adugna; Defere, Fekata; Girma, Dejene; Tsegaye, Mesfin; Mengistu, Takele; Seyoum, Kenbon & Ingestion of Metallic Materials Found in the Stomach and First Part of the Duodenum of a Schizophrenic Man: Case Report & International Medical Case Reports Journal & 10.2147/IMCRJ.S386883 & Ethiopia & Case Report & Mesfin_2022a & 1 & 1 & 1 & 1 & 1 & 1 & NaN & 1 & Include & JGE & 2025-05-13 00:00:00 \\
73 & 686 & 686-001 & 39.000000 & 0 & 0 & 1 & 0 & 0 & 0 & Male & 1 & 0 & 0 & 0.000000 & 0.000000 & NaN & NaN & 1.000000 & 0.000000 & 1.000000 & 0.000000 & 0.000000 & 1.000000 & 0.000000 & 0.000000 & "When questioned about the reason for swallowing a foreign object, the patient was unable to recall doing so. This unconscious eating behavior, thought to be a direct manifestation of schizophrenia when the patient is stimulated by their external environment." & 0.000000 & 0.000000 & 1.000000 & 1.000000 & 1.000000 & 1.000000 & "120 foreign objects, such as keys, nails, iron bars, needles, nail clippers, blades, and ear spoons, were successfully removed from the patient’s stomach" & 0.000000 & 1.000000 & 0.000000 & 0.000000 & 1.000000 & 2023 & Jin, Shengjian; Horiguchi, Taigo; Ma, Xiaolong; Yuan, Shichao; Liu, Qingguo & Metallic foreign bodies ingestion by schizophrenic patient: a case report & Annals of Medicine and Surgery & 10.1097/MS9.0000000000000497 & China & Case Report & Jin_2023 & 1 & 1 & 1 & 1 & 1 & 1 & NaN & 1 & Include & JGE & 2025-05-13 00:00:00 \\
74 & 692 & 692-001 & 30.000000 & 0 & 0 & 1 & 0 & 0 & 0 & Male & 1 & 0 & 0 & 1.000000 & 0.000000 & NaN & NaN & 0.000000 & 0.000000 & 0.000000 & 0.000000 & 0.000000 & 0.000000 & 1.000000 & 0.000000 & "reason for phone ingestion was to avoid detection and losing the phone to the prison authorities while being detained in prison" & 0.000000 & 0.000000 & 1.000000 & 1.000000 & 0.000000 & 0.000000 & "cell phone (with the battery in-situ) in two plastic bags before swallowing.", "71.8 mm x 23.5 mm x13.0 mm and weighed about 20 grams" & 1.000000 & 0.000000 & 0.000000 & 0.000000 & 0.000000 & 2016 & Qureshi NA, Cherian N, Ben-Hamida A, Solkar MH & Endoscopic Retrieval of an Intentionally Ingested Mobile Phone in an Adult: First Case Report of its Kind & NaN & NaN & UK & Case Report & Qureshi_2016 & 1 & 1 & 1 & 1 & 1 & 1 & NaN & 1 & Include & JGE & 2025-05-13 00:00:00 \\
\bottomrule
\end{tabular}

\caption{Export of case-level data (N = 74)}
\end{table}
\end{landscape}

%\paragraph*{Gender} 43 cases (60\%) were male \cite{Akay_2015f, Al-Faham_2020k, Alao_2006i, Ali_2017, Ali_2022g, Apikotoa_2022f, Atayan_2016, Benoist_2019e, Berry_2021e, Bhumi_2024f, CamachoDorado_2018, Csaky_1998e, Emamhadi_2018, Farhadi_2024h, Fry_2010, Gardner_2017h, Guinan_2019f, Jehangir_2019h, Jin_2023, Kobiela_2015, Kumar_2001, Kumar_2019f, Liu_2005, Losanoff_1996, Losanoff_1997e, Mesfin_2022a, Misra_2013, Qureshi_2016, Riva_2018j, Sobnach_2011f, Tammana_2012j, Tanrikulu_2015e, Tay_2004, Thapa_2019f, Trgo_2012f, Wadhwa_2015e, Yasin_2009, teWildt_2010}, 28 cases (39\%) were female \cite{AlShaaibi_2021b, Ali_2020f, Ataya_2013, Beecroft_1998, Bhasin_2014, Bhattacharjee_2008, Cauchi_2002, Chang_2017f, Cox_2007, DelgadoSalazar_2020c, DivsalarP._2023a, Goldman_1998f, Hardy_2023g, Kar_2015, Kariholu_2008, Kerestes_2019, Li_2013, Naji_2012f, Ohno_2005, Peixoto_2017f, Sakellaridis_2008f, Sultan_2024f, Tupesis_2004f, Wildhaber_2005, Wnęk_2015f, Yildiz_2016e}, 1 case (1\%) had no gender recorded \cite{fjbuilsRepeatedBehaviorDeliberate2024}. \paragraph*{Age Group} 25 cases (35\%) were between 26 and 40 years of age \cite{Alao_2006i, Ali_2022g, Apikotoa_2022f, Ataya_2013, Benoist_2019e, Bhasin_2014, Chang_2017f, Cox_2007, DelgadoSalazar_2020c, Farhadi_2024h, Fry_2010, Gardner_2017h, Guinan_2019f, Jin_2023, Kumar_2019f, Losanoff_1996, Misra_2013, Qureshi_2016, Riva_2018j, Sakellaridis_2008f, Tammana_2012j, Trgo_2012f, Wnęk_2015f, Yildiz_2016e, fjbuilsRepeatedBehaviorDeliberate2024}, 18 cases (25\%) were between 18 and 25 years of age \cite{Akay_2015f, Ali_2017, Atayan_2016, Bhattacharjee_2008, Csaky_1998e, Kar_2015, Kariholu_2008, Kobiela_2015, Losanoff_1996, Losanoff_1997e, Mesfin_2022a, Peixoto_2017f, Sobnach_2011f, Tupesis_2004f, Yasin_2009}, 13 cases (18\%) were under 18 years of age \cite{AlShaaibi_2021b, Ali_2020f, Cauchi_2002, DivsalarP._2023a, Goldman_1998f, Liu_2005, Naji_2012f, Ohno_2005, Tanrikulu_2015e, Tay_2004, Wildhaber_2005}, 11 cases (15\%) were between 41 and 60 years of age \cite{Al-Faham_2020k, Bhumi_2024f, CamachoDorado_2018, Emamhadi_2018, Hardy_2023g, Jehangir_2019h, Kumar_2001, Sultan_2024f, Thapa_2019f, Wadhwa_2015e, teWildt_2010}, 3 cases (4\%) were over 60 years of age \cite{Beecroft_1998, Kerestes_2019, Li_2013}, 2 cases (3\%) had no age documented \cite{Berry_2021e}. \paragraph*{Population} 36 cases (50\%) had a psychiatric history \cite{AlShaaibi_2021b, Alao_2006i, Ali_2020f, Apikotoa_2022f, Ataya_2013, Atayan_2016, Beecroft_1998, CamachoDorado_2018, Chang_2017f, DelgadoSalazar_2020c, DivsalarP._2023a, Farhadi_2024h, Fry_2010, Guinan_2019f, Hardy_2023g, Jehangir_2019h, Jin_2023, Kar_2015, Kerestes_2019, Kobiela_2015, Kumar_2001, Kumar_2019f, Liu_2005, Mesfin_2022a, Misra_2013, Ohno_2005, Peixoto_2017f, Sakellaridis_2008f, Sultan_2024f, Tammana_2012j, Tanrikulu_2015e, Yildiz_2016e, fjbuilsRepeatedBehaviorDeliberate2024, teWildt_2010}, 19 cases (26\%) had ingested previously \cite{Alao_2006i, Apikotoa_2022f, Berry_2021e, Bhattacharjee_2008, Csaky_1998e, DivsalarP._2023a, Emamhadi_2018, Guinan_2019f, Jehangir_2019h, Jin_2023, Liu_2005, Sakellaridis_2008f, Tanrikulu_2015e, Thapa_2019f, Yildiz_2016e, fjbuilsRepeatedBehaviorDeliberate2024, teWildt_2010}, 12 cases (17\%) were detained persons \cite{Alao_2006i, Ali_2022g, Apikotoa_2022f, Losanoff_1996, Losanoff_1997e, Qureshi_2016, Tammana_2012j, Trgo_2012f}, 7 cases (10\%) were severely disabled \cite{Atayan_2016, Kerestes_2019, Liu_2005, Ohno_2005, Peixoto_2017f, Yildiz_2016e, teWildt_2010}, 4 cases (6\%) were psychiatric inpatients \cite{DivsalarP._2023a, fjbuilsRepeatedBehaviorDeliberate2024, teWildt_2010}, 3 cases (4\%) were under the influence of alcohol \cite{Benoist_2019e, Csaky_1998e, Thapa_2019f}, 2 cases (3\%) were displaced people \cite{Akay_2015f, Gardner_2017h}. \paragraph*{Motivation} 34 cases (47\%) had a psychiatric motivation \cite{Al-Faham_2020k, Alao_2006i, Ali_2020f, Apikotoa_2022f, Ataya_2013, Atayan_2016, Bhasin_2014, Bhattacharjee_2008, DelgadoSalazar_2020c, DivsalarP._2023a, Emamhadi_2018, Farhadi_2024h, Guinan_2019f, Hardy_2023g, Jehangir_2019h, Jin_2023, Kar_2015, Kariholu_2008, Kerestes_2019, Kobiela_2015, Kumar_2001, Kumar_2019f, Li_2013, Liu_2005, Misra_2013, Ohno_2005, Sakellaridis_2008f, Sultan_2024f, Tammana_2012j, Tanrikulu_2015e, Yasin_2009, teWildt_2010}, 21 cases (29\%) were motivated by self-harm intention \cite{Al-Faham_2020k, AlShaaibi_2021b, Alao_2006i, Ali_2017, CamachoDorado_2018, Chang_2017f, Cox_2007, Csaky_1998e, Fry_2010, Li_2013, Losanoff_1996, Losanoff_1997e, Mesfin_2022a, Sakellaridis_2008f, Tammana_2012j, Tanrikulu_2015e, fjbuilsRepeatedBehaviorDeliberate2024}, 17 cases (24\%) had a psychosocial motivation \cite{Akay_2015f, Benoist_2019e, Bhattacharjee_2008, Cauchi_2002, Goldman_1998f, Hardy_2023g, Kobiela_2015, Li_2013, Naji_2012f, Qureshi_2016, Riva_2018j, Sobnach_2011f, Tay_2004, Thapa_2019f, Tupesis_2004f, Wildhaber_2005, Wnęk_2015f}, 9 cases (12\%) were motivated by protest \cite{Bhumi_2024f, Gardner_2017h, Losanoff_1996, Losanoff_1997e, Tupesis_2004f}, 9 cases (12\%) had another documented motivation \cite{Ali_2020f, Ali_2022g, Emamhadi_2018, Guinan_2019f, Peixoto_2017f, Sakellaridis_2008f, Trgo_2012f, Wadhwa_2015e, Yildiz_2016e}. \paragraph*{Object Characteristics} 51 cases (71\%) ingested a large diameter object (\textgreater{}2.5cm) \cite{Akay_2015f, Al-Faham_2020k, AlShaaibi_2021b, Alao_2006i, Ali_2017, Ali_2022g, Apikotoa_2022f, Atayan_2016, Berry_2021e, Bhasin_2014, CamachoDorado_2018, Cauchi_2002, Chang_2017f, Cox_2007, Csaky_1998e, DivsalarP._2023a, Emamhadi_2018, Gardner_2017h, Guinan_2019f, Jehangir_2019h, Jin_2023, Kariholu_2008, Kerestes_2019, Kobiela_2015, Kumar_2001, Kumar_2019f, Losanoff_1996, Losanoff_1997e, Mesfin_2022a, Misra_2013, Naji_2012f, Ohno_2005, Peixoto_2017f, Qureshi_2016, Riva_2018j, Sakellaridis_2008f, Sultan_2024f, Tanrikulu_2015e, Thapa_2019f, Trgo_2012f, Wnęk_2015f, Yildiz_2016e, fjbuilsRepeatedBehaviorDeliberate2024, teWildt_2010}, 44 cases (61\%) ingested multiple objects \cite{Ali_2020f, Apikotoa_2022f, Ataya_2013, Atayan_2016, Beecroft_1998, Bhattacharjee_2008, Bhumi_2024f, CamachoDorado_2018, Cauchi_2002, Emamhadi_2018, Farhadi_2024h, Fry_2010, Goldman_1998f, Guinan_2019f, Hardy_2023g, Jehangir_2019h, Jin_2023, Kar_2015, Kariholu_2008, Kobiela_2015, Kumar_2001, Kumar_2019f, Li_2013, Liu_2005, Losanoff_1996, Mesfin_2022a, Misra_2013, Naji_2012f, Ohno_2005, Sobnach_2011f, Sultan_2024f, Tammana_2012j, Tanrikulu_2015e, Tay_2004, Thapa_2019f, Wadhwa_2015e, Wildhaber_2005, Yasin_2009, fjbuilsRepeatedBehaviorDeliberate2024, teWildt_2010}, 34 cases (47\%) ingested a sharp object \cite{AlShaaibi_2021b, Alao_2006i, Apikotoa_2022f, Ataya_2013, Benoist_2019e, Bhasin_2014, Bhattacharjee_2008, CamachoDorado_2018, Csaky_1998e, DelgadoSalazar_2020c, DivsalarP._2023a, Emamhadi_2018, Farhadi_2024h, Fry_2010, Guinan_2019f, Hardy_2023g, Jehangir_2019h, Jin_2023, Kariholu_2008, Kobiela_2015, Kumar_2019f, Losanoff_1996, Losanoff_1997e, Mesfin_2022a, Misra_2013, Sobnach_2011f, Yasin_2009, teWildt_2010}, 32 cases (44\%) ingested a long object (\textgreater{}5cm) \cite{Al-Faham_2020k, AlShaaibi_2021b, Ali_2017, Ali_2022g, Atayan_2016, Bhasin_2014, CamachoDorado_2018, Chang_2017f, Cox_2007, Csaky_1998e, DivsalarP._2023a, Emamhadi_2018, Fry_2010, Gardner_2017h, Jin_2023, Kariholu_2008, Kerestes_2019, Kobiela_2015, Kumar_2019f, Mesfin_2022a, Misra_2013, Ohno_2005, Qureshi_2016, Sakellaridis_2008f, Sultan_2024f, Thapa_2019f, Trgo_2012f, Yasin_2009, Yildiz_2016e, teWildt_2010}, 9 cases (12\%) ingested a magnet \cite{Ali_2020f, Bhumi_2024f, Cauchi_2002, Liu_2005, Naji_2012f, Ohno_2005, Tanrikulu_2015e, Tay_2004, Wildhaber_2005}, 2 cases (3\%) ingested a button battery \cite{Berry_2021e, Bhumi_2024f}. \paragraph*{Outcomes} 48 cases (67\%) experienced a complication \cite{Ali_2017, Ali_2020f, Apikotoa_2022f, Atayan_2016, Beecroft_1998, Benoist_2019e, Berry_2021e, Bhasin_2014, Bhumi_2024f, CamachoDorado_2018, Cauchi_2002, Cox_2007, Csaky_1998e, DelgadoSalazar_2020c, DivsalarP._2023a, Emamhadi_2018, Farhadi_2024h, Fry_2010, Gardner_2017h, Goldman_1998f, Jin_2023, Kariholu_2008, Kerestes_2019, Kobiela_2015, Kumar_2001, Kumar_2019f, Liu_2005, Losanoff_1996, Mesfin_2022a, Misra_2013, Naji_2012f, Ohno_2005, Sakellaridis_2008f, Sobnach_2011f, Sultan_2024f, Tanrikulu_2015e, Tay_2004, Thapa_2019f, Trgo_2012f, Tupesis_2004f, Wildhaber_2005, Wnęk_2015f, Yasin_2009, Yildiz_2016e}, 44 cases (61\%) underwent surgery \cite{Al-Faham_2020k, AlShaaibi_2021b, Alao_2006i, Ali_2017, Ali_2020f, Atayan_2016, Beecroft_1998, Bhasin_2014, CamachoDorado_2018, Cauchi_2002, Chang_2017f, Cox_2007, Csaky_1998e, DelgadoSalazar_2020c, DivsalarP._2023a, Farhadi_2024h, Fry_2010, Gardner_2017h, Jin_2023, Kariholu_2008, Kerestes_2019, Kobiela_2015, Kumar_2019f, Liu_2005, Losanoff_1996, Losanoff_1997e, Mesfin_2022a, Misra_2013, Naji_2012f, Sobnach_2011f, Tanrikulu_2015e, Tay_2004, Thapa_2019f, Tupesis_2004f, Wildhaber_2005, Wnęk_2015f, Yasin_2009, Yildiz_2016e, fjbuilsRepeatedBehaviorDeliberate2024}, 31 cases (43\%) underwent endoscopy \cite{Akay_2015f, Ali_2022g, Apikotoa_2022f, Atayan_2016, Benoist_2019e, Berry_2021e, Bhasin_2014, Bhumi_2024f, CamachoDorado_2018, Chang_2017f, DelgadoSalazar_2020c, Gardner_2017h, Guinan_2019f, Hardy_2023g, Jehangir_2019h, Kariholu_2008, Li_2013, Liu_2005, Ohno_2005, Peixoto_2017f, Qureshi_2016, Riva_2018j, Sakellaridis_2008f, Sultan_2024f, Tammana_2012j, Tanrikulu_2015e, Trgo_2012f, Wadhwa_2015e, Wnęk_2015f, teWildt_2010}, 7 cases (10\%) were managed conservatively \cite{Ataya_2013, Bhattacharjee_2008, DivsalarP._2023a, Emamhadi_2018, Goldman_1998f, Kar_2015, Kumar_2001}, 2 cases (3\%) died \cite{Emamhadi_2018, Kumar_2001}. \paragraph*{Geographical Location}Cases were recorded in 33 countries: 13 cases from USA \cite{Alao_2006i, Ataya_2013, Bhumi_2024f, Fry_2010, Guinan_2019f, Hardy_2023g, Jehangir_2019h, Kerestes_2019, Kumar_2001, Liu_2005, Tammana_2012j, Tay_2004, Tupesis_2004f}; 7 cases from India \cite{Bhasin_2014, Bhattacharjee_2008, Kar_2015, Kariholu_2008, Kumar_2019f, Misra_2013, Wadhwa_2015e} and UK \cite{Beecroft_1998, Berry_2021e, Cauchi_2002, Cox_2007, Gardner_2017h, Qureshi_2016}; 6 cases from Bulgaria \cite{Losanoff_1996, Losanoff_1997e}; 5 cases from Iran \cite{DivsalarP._2023a, Emamhadi_2018, Farhadi_2024h}; 4 cases from Turkey \cite{Akay_2015f, Atayan_2016, Tanrikulu_2015e, Yildiz_2016e}; 2 cases from China \cite{Jin_2023, Li_2013}, Poland \cite{Kobiela_2015, Wnęk_2015f}, and Spain \cite{CamachoDorado_2018, fjbuilsRepeatedBehaviorDeliberate2024}; 1 case from Australia \cite{Apikotoa_2022f}, Bahrain \cite{Ali_2020f}, Croatia \cite{Trgo_2012f}, Ecuador \cite{DelgadoSalazar_2020c}, Egypt \cite{Ali_2022g}, Ethiopia \cite{Mesfin_2022a}, Germany \cite{teWildt_2010}, Greece \cite{Sakellaridis_2008f}, Hungary \cite{Csaky_1998e}, Iraq \cite{Al-Faham_2020k}, Israel \cite{Goldman_1998f}, Italy \cite{Riva_2018j}, Japan \cite{Ohno_2005}, Nepal \cite{Thapa_2019f}, Netherlands \cite{Benoist_2019e}, Oman \cite{AlShaaibi_2021b}, Pakistan \cite{Yasin_2009}, Portugal \cite{Peixoto_2017f}, Qatar \cite{Ali_2017}, Saudi Arabia \cite{Sultan_2024f}, South Africa \cite{Sobnach_2011f}, Sweden \cite{Naji_2012f}, Switzerland \cite{Wildhaber_2005}, and Taiwan \cite{Chang_2017f}. \paragraph*{Gender} 43 cases (60\%) were male \cite{Akay_2015f, Al-Faham_2020k, Alao_2006i, Ali_2017, Ali_2022g, Apikotoa_2022f, Atayan_2016, Benoist_2019e, Berry_2021e, Bhumi_2024f, CamachoDorado_2018, Csaky_1998e, Emamhadi_2018, Farhadi_2024h, Fry_2010, Gardner_2017h, Guinan_2019f, Jehangir_2019h, Jin_2023, Kobiela_2015, Kumar_2001, Kumar_2019f, Liu_2005, Losanoff_1996, Losanoff_1997e, Mesfin_2022a, Misra_2013, Qureshi_2016, Riva_2018j, Sobnach_2011f, Tammana_2012j, Tanrikulu_2015e, Tay_2004, Thapa_2019f, Trgo_2012f, Wadhwa_2015e, Yasin_2009, teWildt_2010}, 28 cases (39\%) were female \cite{AlShaaibi_2021b, Ali_2020f, Ataya_2013, Beecroft_1998, Bhasin_2014, Bhattacharjee_2008, Cauchi_2002, Chang_2017f, Cox_2007, DelgadoSalazar_2020c, DivsalarP._2023a, Goldman_1998f, Hardy_2023g, Kar_2015, Kariholu_2008, Kerestes_2019, Li_2013, Naji_2012f, Ohno_2005, Peixoto_2017f, Sakellaridis_2008f, Sultan_2024f, Tupesis_2004f, Wildhaber_2005, Wnęk_2015f, Yildiz_2016e}, 1 case (1\%) had no gender recorded \cite{fjbuilsRepeatedBehaviorDeliberate2024}. \paragraph*{Age Group} 25 cases (35\%) were between 26 and 40 years of age \cite{Alao_2006i, Ali_2022g, Apikotoa_2022f, Ataya_2013, Benoist_2019e, Bhasin_2014, Chang_2017f, Cox_2007, DelgadoSalazar_2020c, Farhadi_2024h, Fry_2010, Gardner_2017h, Guinan_2019f, Jin_2023, Kumar_2019f, Losanoff_1996, Misra_2013, Qureshi_2016, Riva_2018j, Sakellaridis_2008f, Tammana_2012j, Trgo_2012f, Wnęk_2015f, Yildiz_2016e, fjbuilsRepeatedBehaviorDeliberate2024}, 18 cases (25\%) were between 18 and 25 years of age \cite{Akay_2015f, Ali_2017, Atayan_2016, Bhattacharjee_2008, Csaky_1998e, Kar_2015, Kariholu_2008, Kobiela_2015, Losanoff_1996, Losanoff_1997e, Mesfin_2022a, Peixoto_2017f, Sobnach_2011f, Tupesis_2004f, Yasin_2009}, 13 cases (18\%) were under 18 years of age \cite{AlShaaibi_2021b, Ali_2020f, Cauchi_2002, DivsalarP._2023a, Goldman_1998f, Liu_2005, Naji_2012f, Ohno_2005, Tanrikulu_2015e, Tay_2004, Wildhaber_2005}, 11 cases (15\%) were between 41 and 60 years of age \cite{Al-Faham_2020k, Bhumi_2024f, CamachoDorado_2018, Emamhadi_2018, Hardy_2023g, Jehangir_2019h, Kumar_2001, Sultan_2024f, Thapa_2019f, Wadhwa_2015e, teWildt_2010}, 3 cases (4\%) were over 60 years of age \cite{Beecroft_1998, Kerestes_2019, Li_2013}, 2 cases (3\%) had no age documented \cite{Berry_2021e}. \paragraph*{Population} 36 cases (50\%) had a psychiatric history \cite{AlShaaibi_2021b, Alao_2006i, Ali_2020f, Apikotoa_2022f, Ataya_2013, Atayan_2016, Beecroft_1998, CamachoDorado_2018, Chang_2017f, DelgadoSalazar_2020c, DivsalarP._2023a, Farhadi_2024h, Fry_2010, Guinan_2019f, Hardy_2023g, Jehangir_2019h, Jin_2023, Kar_2015, Kerestes_2019, Kobiela_2015, Kumar_2001, Kumar_2019f, Liu_2005, Mesfin_2022a, Misra_2013, Ohno_2005, Peixoto_2017f, Sakellaridis_2008f, Sultan_2024f, Tammana_2012j, Tanrikulu_2015e, Yildiz_2016e, fjbuilsRepeatedBehaviorDeliberate2024, teWildt_2010}, 19 cases (26\%) had ingested previously \cite{Alao_2006i, Apikotoa_2022f, Berry_2021e, Bhattacharjee_2008, Csaky_1998e, DivsalarP._2023a, Emamhadi_2018, Guinan_2019f, Jehangir_2019h, Jin_2023, Liu_2005, Sakellaridis_2008f, Tanrikulu_2015e, Thapa_2019f, Yildiz_2016e, fjbuilsRepeatedBehaviorDeliberate2024, teWildt_2010}, 12 cases (17\%) were detained persons \cite{Alao_2006i, Ali_2022g, Apikotoa_2022f, Losanoff_1996, Losanoff_1997e, Qureshi_2016, Tammana_2012j, Trgo_2012f}, 7 cases (10\%) were severely disabled \cite{Atayan_2016, Kerestes_2019, Liu_2005, Ohno_2005, Peixoto_2017f, Yildiz_2016e, teWildt_2010}, 4 cases (6\%) were psychiatric inpatients \cite{DivsalarP._2023a, fjbuilsRepeatedBehaviorDeliberate2024, teWildt_2010}, 3 cases (4\%) were under the influence of alcohol \cite{Benoist_2019e, Csaky_1998e, Thapa_2019f}, 2 cases (3\%) were displaced people \cite{Akay_2015f, Gardner_2017h}. \paragraph*{Motivation} 34 cases (47\%) had a psychiatric motivation \cite{Al-Faham_2020k, Alao_2006i, Ali_2020f, Apikotoa_2022f, Ataya_2013, Atayan_2016, Bhasin_2014, Bhattacharjee_2008, DelgadoSalazar_2020c, DivsalarP._2023a, Emamhadi_2018, Farhadi_2024h, Guinan_2019f, Hardy_2023g, Jehangir_2019h, Jin_2023, Kar_2015, Kariholu_2008, Kerestes_2019, Kobiela_2015, Kumar_2001, Kumar_2019f, Li_2013, Liu_2005, Misra_2013, Ohno_2005, Sakellaridis_2008f, Sultan_2024f, Tammana_2012j, Tanrikulu_2015e, Yasin_2009, teWildt_2010}, 21 cases (29\%) were motivated by self-harm intention \cite{Al-Faham_2020k, AlShaaibi_2021b, Alao_2006i, Ali_2017, CamachoDorado_2018, Chang_2017f, Cox_2007, Csaky_1998e, Fry_2010, Li_2013, Losanoff_1996, Losanoff_1997e, Mesfin_2022a, Sakellaridis_2008f, Tammana_2012j, Tanrikulu_2015e, fjbuilsRepeatedBehaviorDeliberate2024}, 17 cases (24\%) had a psychosocial motivation \cite{Akay_2015f, Benoist_2019e, Bhattacharjee_2008, Cauchi_2002, Goldman_1998f, Hardy_2023g, Kobiela_2015, Li_2013, Naji_2012f, Qureshi_2016, Riva_2018j, Sobnach_2011f, Tay_2004, Thapa_2019f, Tupesis_2004f, Wildhaber_2005, Wnęk_2015f}, 9 cases (12\%) were motivated by protest \cite{Bhumi_2024f, Gardner_2017h, Losanoff_1996, Losanoff_1997e, Tupesis_2004f}, 9 cases (12\%) had another documented motivation \cite{Ali_2020f, Ali_2022g, Emamhadi_2018, Guinan_2019f, Peixoto_2017f, Sakellaridis_2008f, Trgo_2012f, Wadhwa_2015e, Yildiz_2016e}. \paragraph*{Object Characteristics} 51 cases (71\%) ingested a large diameter object (\textgreater{}2.5cm) \cite{Akay_2015f, Al-Faham_2020k, AlShaaibi_2021b, Alao_2006i, Ali_2017, Ali_2022g, Apikotoa_2022f, Atayan_2016, Berry_2021e, Bhasin_2014, CamachoDorado_2018, Cauchi_2002, Chang_2017f, Cox_2007, Csaky_1998e, DivsalarP._2023a, Emamhadi_2018, Gardner_2017h, Guinan_2019f, Jehangir_2019h, Jin_2023, Kariholu_2008, Kerestes_2019, Kobiela_2015, Kumar_2001, Kumar_2019f, Losanoff_1996, Losanoff_1997e, Mesfin_2022a, Misra_2013, Naji_2012f, Ohno_2005, Peixoto_2017f, Qureshi_2016, Riva_2018j, Sakellaridis_2008f, Sultan_2024f, Tanrikulu_2015e, Thapa_2019f, Trgo_2012f, Wnęk_2015f, Yildiz_2016e, fjbuilsRepeatedBehaviorDeliberate2024, teWildt_2010}, 44 cases (61\%) ingested multiple objects \cite{Ali_2020f, Apikotoa_2022f, Ataya_2013, Atayan_2016, Beecroft_1998, Bhattacharjee_2008, Bhumi_2024f, CamachoDorado_2018, Cauchi_2002, Emamhadi_2018, Farhadi_2024h, Fry_2010, Goldman_1998f, Guinan_2019f, Hardy_2023g, Jehangir_2019h, Jin_2023, Kar_2015, Kariholu_2008, Kobiela_2015, Kumar_2001, Kumar_2019f, Li_2013, Liu_2005, Losanoff_1996, Mesfin_2022a, Misra_2013, Naji_2012f, Ohno_2005, Sobnach_2011f, Sultan_2024f, Tammana_2012j, Tanrikulu_2015e, Tay_2004, Thapa_2019f, Wadhwa_2015e, Wildhaber_2005, Yasin_2009, fjbuilsRepeatedBehaviorDeliberate2024, teWildt_2010}, 34 cases (47\%) ingested a sharp object \cite{AlShaaibi_2021b, Alao_2006i, Apikotoa_2022f, Ataya_2013, Benoist_2019e, Bhasin_2014, Bhattacharjee_2008, CamachoDorado_2018, Csaky_1998e, DelgadoSalazar_2020c, DivsalarP._2023a, Emamhadi_2018, Farhadi_2024h, Fry_2010, Guinan_2019f, Hardy_2023g, Jehangir_2019h, Jin_2023, Kariholu_2008, Kobiela_2015, Kumar_2019f, Losanoff_1996, Losanoff_1997e, Mesfin_2022a, Misra_2013, Sobnach_2011f, Yasin_2009, teWildt_2010}, 32 cases (44\%) ingested a long object (\textgreater{}5cm) \cite{Al-Faham_2020k, AlShaaibi_2021b, Ali_2017, Ali_2022g, Atayan_2016, Bhasin_2014, CamachoDorado_2018, Chang_2017f, Cox_2007, Csaky_1998e, DivsalarP._2023a, Emamhadi_2018, Fry_2010, Gardner_2017h, Jin_2023, Kariholu_2008, Kerestes_2019, Kobiela_2015, Kumar_2019f, Mesfin_2022a, Misra_2013, Ohno_2005, Qureshi_2016, Sakellaridis_2008f, Sultan_2024f, Thapa_2019f, Trgo_2012f, Yasin_2009, Yildiz_2016e, teWildt_2010}, 9 cases (12\%) ingested a magnet \cite{Ali_2020f, Bhumi_2024f, Cauchi_2002, Liu_2005, Naji_2012f, Ohno_2005, Tanrikulu_2015e, Tay_2004, Wildhaber_2005}, 2 cases (3\%) ingested a button battery \cite{Berry_2021e, Bhumi_2024f}. \paragraph*{Outcomes} 48 cases (67\%) experienced a complication \cite{Ali_2017, Ali_2020f, Apikotoa_2022f, Atayan_2016, Beecroft_1998, Benoist_2019e, Berry_2021e, Bhasin_2014, Bhumi_2024f, CamachoDorado_2018, Cauchi_2002, Cox_2007, Csaky_1998e, DelgadoSalazar_2020c, DivsalarP._2023a, Emamhadi_2018, Farhadi_2024h, Fry_2010, Gardner_2017h, Goldman_1998f, Jin_2023, Kariholu_2008, Kerestes_2019, Kobiela_2015, Kumar_2001, Kumar_2019f, Liu_2005, Losanoff_1996, Mesfin_2022a, Misra_2013, Naji_2012f, Ohno_2005, Sakellaridis_2008f, Sobnach_2011f, Sultan_2024f, Tanrikulu_2015e, Tay_2004, Thapa_2019f, Trgo_2012f, Tupesis_2004f, Wildhaber_2005, Wnęk_2015f, Yasin_2009, Yildiz_2016e}, 44 cases (61\%) underwent surgery \cite{Al-Faham_2020k, AlShaaibi_2021b, Alao_2006i, Ali_2017, Ali_2020f, Atayan_2016, Beecroft_1998, Bhasin_2014, CamachoDorado_2018, Cauchi_2002, Chang_2017f, Cox_2007, Csaky_1998e, DelgadoSalazar_2020c, DivsalarP._2023a, Farhadi_2024h, Fry_2010, Gardner_2017h, Jin_2023, Kariholu_2008, Kerestes_2019, Kobiela_2015, Kumar_2019f, Liu_2005, Losanoff_1996, Losanoff_1997e, Mesfin_2022a, Misra_2013, Naji_2012f, Sobnach_2011f, Tanrikulu_2015e, Tay_2004, Thapa_2019f, Tupesis_2004f, Wildhaber_2005, Wnęk_2015f, Yasin_2009, Yildiz_2016e, fjbuilsRepeatedBehaviorDeliberate2024}, 31 cases (43\%) underwent endoscopy \cite{Akay_2015f, Ali_2022g, Apikotoa_2022f, Atayan_2016, Benoist_2019e, Berry_2021e, Bhasin_2014, Bhumi_2024f, CamachoDorado_2018, Chang_2017f, DelgadoSalazar_2020c, Gardner_2017h, Guinan_2019f, Hardy_2023g, Jehangir_2019h, Kariholu_2008, Li_2013, Liu_2005, Ohno_2005, Peixoto_2017f, Qureshi_2016, Riva_2018j, Sakellaridis_2008f, Sultan_2024f, Tammana_2012j, Tanrikulu_2015e, Trgo_2012f, Wadhwa_2015e, Wnęk_2015f, teWildt_2010}, 7 cases (10\%) were managed conservatively \cite{Ataya_2013, Bhattacharjee_2008, DivsalarP._2023a, Emamhadi_2018, Goldman_1998f, Kar_2015, Kumar_2001}, 2 cases (3\%) died \cite{Emamhadi_2018, Kumar_2001}. All 90 were male gender. 90 cases (100\%) were detained at the time of ingestion \cite{Elghali_2016, Karp_1991b, Lee_2007}, 88 cases (98\%) were intentional ingestions \cite{Elghali_2016, Karp_1991b, Lee_2007}, 30 cases (33\%) had a psychiatric history documented \cite{Elghali_2016, Karp_1991b, Lee_2007}, 2 cases (2\%) had a history of prior ingestion \cite{Elghali_2016}. No cases were reported for were psychiatric inpatients, were displaced people, were under the influence of alcohol at the time of ingestion, and had a severe disability history.
\paragraph*{Motivation}  70 cases (78\%) reported protest motivation \cite{Elghali_2016, Karp_1991b, Lee_2007}, 12 cases (13\%) reported psychiatric motivation \cite{Karp_1991b}, 6 cases (7\%) reported self-harm motivation \cite{Elghali_2016, Karp_1991b}. No cases were reported for psychosocial motivation and other motivation.
\paragraph*{Object Characteristics}  68 cases (76\%) involved sharp object ingestion \cite{Elghali_2016, Karp_1991b, Lee_2007}, 32 cases (36\%) involved long (\textgreater 5cm) object ingestion \cite{Lee_2007}, 25 cases (28\%) involved ingestion of multiple objects \cite{Elghali_2016, Lee_2007}. No cases were reported for button battery ingestion, magnet ingestion, and involved large diameter (\textgreater 2.5cm) object ingestion.
\paragraph*{Outcomes}  47 cases (52\%) underwent endoscopic intervention \cite{Elghali_2016, Lee_2007}, 29 cases (32\%) were managed conservatively \cite{Elghali_2016, Karp_1991b}, 15 cases (17\%) underwent surgical intervention \cite{Elghali_2016, Karp_1991b, Lee_2007}, 6 cases (7\%) reported complications \cite{Lee_2007}, 1 case (1\%) died \cite{Elghali_2016}.
\paragraph*{Geographical Location}Cases were recorded in 33 countries: 13 cases from USA \cite{Alao_2006i, Ataya_2013, Bhumi_2024f, Fry_2010, Guinan_2019f, Hardy_2023g, Jehangir_2019h, Kerestes_2019, Kumar_2001, Liu_2005, Tammana_2012j, Tay_2004, Tupesis_2004f}; 7 cases from India \cite{Bhasin_2014, Bhattacharjee_2008, Kar_2015, Kariholu_2008, Kumar_2019f, Misra_2013, Wadhwa_2015e} and UK \cite{Beecroft_1998, Berry_2021e, Cauchi_2002, Cox_2007, Gardner_2017h, Qureshi_2016}; 6 cases from Bulgaria \cite{Losanoff_1996, Losanoff_1997e}; 5 cases from Iran \cite{DivsalarP._2023a, Emamhadi_2018, Farhadi_2024h}; 4 cases from Turkey \cite{Akay_2015f, Atayan_2016, Tanrikulu_2015e, Yildiz_2016e}; 2 cases from China \cite{Jin_2023, Li_2013}, Poland \cite{Kobiela_2015, Wnęk_2015f}, and Spain \cite{CamachoDorado_2018, fjbuilsRepeatedBehaviorDeliberate2024}; 1 case from Australia \cite{Apikotoa_2022f}, Bahrain \cite{Ali_2020f}, Croatia \cite{Trgo_2012f}, Ecuador \cite{DelgadoSalazar_2020c}, Egypt \cite{Ali_2022g}, Ethiopia \cite{Mesfin_2022a}, Germany \cite{teWildt_2010}, Greece \cite{Sakellaridis_2008f}, Hungary \cite{Csaky_1998e}, Iraq \cite{Al-Faham_2020k}, Israel \cite{Goldman_1998f}, Italy \cite{Riva_2018j}, Japan \cite{Ohno_2005}, Nepal \cite{Thapa_2019f}, Netherlands \cite{Benoist_2019e}, Oman \cite{AlShaaibi_2021b}, Pakistan \cite{Yasin_2009}, Portugal \cite{Peixoto_2017f}, Qatar \cite{Ali_2017}, Saudi Arabia \cite{Sultan_2024f}, South Africa \cite{Sobnach_2011f}, Sweden \cite{Naji_2012f}, Switzerland \cite{Wildhaber_2005}, and Taiwan \cite{Chang_2017f}. \paragraph*{Gender} 43 cases (60\%) were male \cite{Akay_2015f, Al-Faham_2020k, Alao_2006i, Ali_2017, Ali_2022g, Apikotoa_2022f, Atayan_2016, Benoist_2019e, Berry_2021e, Bhumi_2024f, CamachoDorado_2018, Csaky_1998e, Emamhadi_2018, Farhadi_2024h, Fry_2010, Gardner_2017h, Guinan_2019f, Jehangir_2019h, Jin_2023, Kobiela_2015, Kumar_2001, Kumar_2019f, Liu_2005, Losanoff_1996, Losanoff_1997e, Mesfin_2022a, Misra_2013, Qureshi_2016, Riva_2018j, Sobnach_2011f, Tammana_2012j, Tanrikulu_2015e, Tay_2004, Thapa_2019f, Trgo_2012f, Wadhwa_2015e, Yasin_2009, teWildt_2010}, 28 cases (39\%) were female \cite{AlShaaibi_2021b, Ali_2020f, Ataya_2013, Beecroft_1998, Bhasin_2014, Bhattacharjee_2008, Cauchi_2002, Chang_2017f, Cox_2007, DelgadoSalazar_2020c, DivsalarP._2023a, Goldman_1998f, Hardy_2023g, Kar_2015, Kariholu_2008, Kerestes_2019, Li_2013, Naji_2012f, Ohno_2005, Peixoto_2017f, Sakellaridis_2008f, Sultan_2024f, Tupesis_2004f, Wildhaber_2005, Wnęk_2015f, Yildiz_2016e}, 1 case (1\%) had no gender recorded \cite{fjbuilsRepeatedBehaviorDeliberate2024}. \paragraph*{Age Group} 25 cases (35\%) were between 26 and 40 years of age \cite{Alao_2006i, Ali_2022g, Apikotoa_2022f, Ataya_2013, Benoist_2019e, Bhasin_2014, Chang_2017f, Cox_2007, DelgadoSalazar_2020c, Farhadi_2024h, Fry_2010, Gardner_2017h, Guinan_2019f, Jin_2023, Kumar_2019f, Losanoff_1996, Misra_2013, Qureshi_2016, Riva_2018j, Sakellaridis_2008f, Tammana_2012j, Trgo_2012f, Wnęk_2015f, Yildiz_2016e, fjbuilsRepeatedBehaviorDeliberate2024}, 18 cases (25\%) were between 18 and 25 years of age \cite{Akay_2015f, Ali_2017, Atayan_2016, Bhattacharjee_2008, Csaky_1998e, Kar_2015, Kariholu_2008, Kobiela_2015, Losanoff_1996, Losanoff_1997e, Mesfin_2022a, Peixoto_2017f, Sobnach_2011f, Tupesis_2004f, Yasin_2009}, 13 cases (18\%) were under 18 years of age \cite{AlShaaibi_2021b, Ali_2020f, Cauchi_2002, DivsalarP._2023a, Goldman_1998f, Liu_2005, Naji_2012f, Ohno_2005, Tanrikulu_2015e, Tay_2004, Wildhaber_2005}, 11 cases (15\%) were between 41 and 60 years of age \cite{Al-Faham_2020k, Bhumi_2024f, CamachoDorado_2018, Emamhadi_2018, Hardy_2023g, Jehangir_2019h, Kumar_2001, Sultan_2024f, Thapa_2019f, Wadhwa_2015e, teWildt_2010}, 3 cases (4\%) were over 60 years of age \cite{Beecroft_1998, Kerestes_2019, Li_2013}, 2 cases (3\%) had no age documented \cite{Berry_2021e}. \paragraph*{Population} 36 cases (50\%) had a psychiatric history \cite{AlShaaibi_2021b, Alao_2006i, Ali_2020f, Apikotoa_2022f, Ataya_2013, Atayan_2016, Beecroft_1998, CamachoDorado_2018, Chang_2017f, DelgadoSalazar_2020c, DivsalarP._2023a, Farhadi_2024h, Fry_2010, Guinan_2019f, Hardy_2023g, Jehangir_2019h, Jin_2023, Kar_2015, Kerestes_2019, Kobiela_2015, Kumar_2001, Kumar_2019f, Liu_2005, Mesfin_2022a, Misra_2013, Ohno_2005, Peixoto_2017f, Sakellaridis_2008f, Sultan_2024f, Tammana_2012j, Tanrikulu_2015e, Yildiz_2016e, fjbuilsRepeatedBehaviorDeliberate2024, teWildt_2010}, 19 cases (26\%) had ingested previously \cite{Alao_2006i, Apikotoa_2022f, Berry_2021e, Bhattacharjee_2008, Csaky_1998e, DivsalarP._2023a, Emamhadi_2018, Guinan_2019f, Jehangir_2019h, Jin_2023, Liu_2005, Sakellaridis_2008f, Tanrikulu_2015e, Thapa_2019f, Yildiz_2016e, fjbuilsRepeatedBehaviorDeliberate2024, teWildt_2010}, 12 cases (17\%) were detained persons \cite{Alao_2006i, Ali_2022g, Apikotoa_2022f, Losanoff_1996, Losanoff_1997e, Qureshi_2016, Tammana_2012j, Trgo_2012f}, 7 cases (10\%) were severely disabled \cite{Atayan_2016, Kerestes_2019, Liu_2005, Ohno_2005, Peixoto_2017f, Yildiz_2016e, teWildt_2010}, 4 cases (6\%) were psychiatric inpatients \cite{DivsalarP._2023a, fjbuilsRepeatedBehaviorDeliberate2024, teWildt_2010}, 3 cases (4\%) were under the influence of alcohol \cite{Benoist_2019e, Csaky_1998e, Thapa_2019f}, 2 cases (3\%) were displaced people \cite{Akay_2015f, Gardner_2017h}. \paragraph*{Motivation} 34 cases (47\%) had a psychiatric motivation \cite{Al-Faham_2020k, Alao_2006i, Ali_2020f, Apikotoa_2022f, Ataya_2013, Atayan_2016, Bhasin_2014, Bhattacharjee_2008, DelgadoSalazar_2020c, DivsalarP._2023a, Emamhadi_2018, Farhadi_2024h, Guinan_2019f, Hardy_2023g, Jehangir_2019h, Jin_2023, Kar_2015, Kariholu_2008, Kerestes_2019, Kobiela_2015, Kumar_2001, Kumar_2019f, Li_2013, Liu_2005, Misra_2013, Ohno_2005, Sakellaridis_2008f, Sultan_2024f, Tammana_2012j, Tanrikulu_2015e, Yasin_2009, teWildt_2010}, 21 cases (29\%) were motivated by self-harm intention \cite{Al-Faham_2020k, AlShaaibi_2021b, Alao_2006i, Ali_2017, CamachoDorado_2018, Chang_2017f, Cox_2007, Csaky_1998e, Fry_2010, Li_2013, Losanoff_1996, Losanoff_1997e, Mesfin_2022a, Sakellaridis_2008f, Tammana_2012j, Tanrikulu_2015e, fjbuilsRepeatedBehaviorDeliberate2024}, 17 cases (24\%) had a psychosocial motivation \cite{Akay_2015f, Benoist_2019e, Bhattacharjee_2008, Cauchi_2002, Goldman_1998f, Hardy_2023g, Kobiela_2015, Li_2013, Naji_2012f, Qureshi_2016, Riva_2018j, Sobnach_2011f, Tay_2004, Thapa_2019f, Tupesis_2004f, Wildhaber_2005, Wnęk_2015f}, 9 cases (12\%) were motivated by protest \cite{Bhumi_2024f, Gardner_2017h, Losanoff_1996, Losanoff_1997e, Tupesis_2004f}, 9 cases (12\%) had another documented motivation \cite{Ali_2020f, Ali_2022g, Emamhadi_2018, Guinan_2019f, Peixoto_2017f, Sakellaridis_2008f, Trgo_2012f, Wadhwa_2015e, Yildiz_2016e}. \paragraph*{Object Characteristics} 51 cases (71\%) ingested a large diameter object (\textgreater{}2.5cm) \cite{Akay_2015f, Al-Faham_2020k, AlShaaibi_2021b, Alao_2006i, Ali_2017, Ali_2022g, Apikotoa_2022f, Atayan_2016, Berry_2021e, Bhasin_2014, CamachoDorado_2018, Cauchi_2002, Chang_2017f, Cox_2007, Csaky_1998e, DivsalarP._2023a, Emamhadi_2018, Gardner_2017h, Guinan_2019f, Jehangir_2019h, Jin_2023, Kariholu_2008, Kerestes_2019, Kobiela_2015, Kumar_2001, Kumar_2019f, Losanoff_1996, Losanoff_1997e, Mesfin_2022a, Misra_2013, Naji_2012f, Ohno_2005, Peixoto_2017f, Qureshi_2016, Riva_2018j, Sakellaridis_2008f, Sultan_2024f, Tanrikulu_2015e, Thapa_2019f, Trgo_2012f, Wnęk_2015f, Yildiz_2016e, fjbuilsRepeatedBehaviorDeliberate2024, teWildt_2010}, 44 cases (61\%) ingested multiple objects \cite{Ali_2020f, Apikotoa_2022f, Ataya_2013, Atayan_2016, Beecroft_1998, Bhattacharjee_2008, Bhumi_2024f, CamachoDorado_2018, Cauchi_2002, Emamhadi_2018, Farhadi_2024h, Fry_2010, Goldman_1998f, Guinan_2019f, Hardy_2023g, Jehangir_2019h, Jin_2023, Kar_2015, Kariholu_2008, Kobiela_2015, Kumar_2001, Kumar_2019f, Li_2013, Liu_2005, Losanoff_1996, Mesfin_2022a, Misra_2013, Naji_2012f, Ohno_2005, Sobnach_2011f, Sultan_2024f, Tammana_2012j, Tanrikulu_2015e, Tay_2004, Thapa_2019f, Wadhwa_2015e, Wildhaber_2005, Yasin_2009, fjbuilsRepeatedBehaviorDeliberate2024, teWildt_2010}, 34 cases (47\%) ingested a sharp object \cite{AlShaaibi_2021b, Alao_2006i, Apikotoa_2022f, Ataya_2013, Benoist_2019e, Bhasin_2014, Bhattacharjee_2008, CamachoDorado_2018, Csaky_1998e, DelgadoSalazar_2020c, DivsalarP._2023a, Emamhadi_2018, Farhadi_2024h, Fry_2010, Guinan_2019f, Hardy_2023g, Jehangir_2019h, Jin_2023, Kariholu_2008, Kobiela_2015, Kumar_2019f, Losanoff_1996, Losanoff_1997e, Mesfin_2022a, Misra_2013, Sobnach_2011f, Yasin_2009, teWildt_2010}, 32 cases (44\%) ingested a long object (\textgreater{}5cm) \cite{Al-Faham_2020k, AlShaaibi_2021b, Ali_2017, Ali_2022g, Atayan_2016, Bhasin_2014, CamachoDorado_2018, Chang_2017f, Cox_2007, Csaky_1998e, DivsalarP._2023a, Emamhadi_2018, Fry_2010, Gardner_2017h, Jin_2023, Kariholu_2008, Kerestes_2019, Kobiela_2015, Kumar_2019f, Mesfin_2022a, Misra_2013, Ohno_2005, Qureshi_2016, Sakellaridis_2008f, Sultan_2024f, Thapa_2019f, Trgo_2012f, Yasin_2009, Yildiz_2016e, teWildt_2010}, 9 cases (12\%) ingested a magnet \cite{Ali_2020f, Bhumi_2024f, Cauchi_2002, Liu_2005, Naji_2012f, Ohno_2005, Tanrikulu_2015e, Tay_2004, Wildhaber_2005}, 2 cases (3\%) ingested a button battery \cite{Berry_2021e, Bhumi_2024f}. \paragraph*{Outcomes} 48 cases (67\%) experienced a complication \cite{Ali_2017, Ali_2020f, Apikotoa_2022f, Atayan_2016, Beecroft_1998, Benoist_2019e, Berry_2021e, Bhasin_2014, Bhumi_2024f, CamachoDorado_2018, Cauchi_2002, Cox_2007, Csaky_1998e, DelgadoSalazar_2020c, DivsalarP._2023a, Emamhadi_2018, Farhadi_2024h, Fry_2010, Gardner_2017h, Goldman_1998f, Jin_2023, Kariholu_2008, Kerestes_2019, Kobiela_2015, Kumar_2001, Kumar_2019f, Liu_2005, Losanoff_1996, Mesfin_2022a, Misra_2013, Naji_2012f, Ohno_2005, Sakellaridis_2008f, Sobnach_2011f, Sultan_2024f, Tanrikulu_2015e, Tay_2004, Thapa_2019f, Trgo_2012f, Tupesis_2004f, Wildhaber_2005, Wnęk_2015f, Yasin_2009, Yildiz_2016e}, 44 cases (61\%) underwent surgery \cite{Al-Faham_2020k, AlShaaibi_2021b, Alao_2006i, Ali_2017, Ali_2020f, Atayan_2016, Beecroft_1998, Bhasin_2014, CamachoDorado_2018, Cauchi_2002, Chang_2017f, Cox_2007, Csaky_1998e, DelgadoSalazar_2020c, DivsalarP._2023a, Farhadi_2024h, Fry_2010, Gardner_2017h, Jin_2023, Kariholu_2008, Kerestes_2019, Kobiela_2015, Kumar_2019f, Liu_2005, Losanoff_1996, Losanoff_1997e, Mesfin_2022a, Misra_2013, Naji_2012f, Sobnach_2011f, Tanrikulu_2015e, Tay_2004, Thapa_2019f, Tupesis_2004f, Wildhaber_2005, Wnęk_2015f, Yasin_2009, Yildiz_2016e, fjbuilsRepeatedBehaviorDeliberate2024}, 31 cases (43\%) underwent endoscopy \cite{Akay_2015f, Ali_2022g, Apikotoa_2022f, Atayan_2016, Benoist_2019e, Berry_2021e, Bhasin_2014, Bhumi_2024f, CamachoDorado_2018, Chang_2017f, DelgadoSalazar_2020c, Gardner_2017h, Guinan_2019f, Hardy_2023g, Jehangir_2019h, Kariholu_2008, Li_2013, Liu_2005, Ohno_2005, Peixoto_2017f, Qureshi_2016, Riva_2018j, Sakellaridis_2008f, Sultan_2024f, Tammana_2012j, Tanrikulu_2015e, Trgo_2012f, Wadhwa_2015e, Wnęk_2015f, teWildt_2010}, 7 cases (10\%) were managed conservatively \cite{Ataya_2013, Bhattacharjee_2008, DivsalarP._2023a, Emamhadi_2018, Goldman_1998f, Kar_2015, Kumar_2001}, 2 cases (3\%) died \cite{Emamhadi_2018, Kumar_2001}. All 90 were male gender. 90 cases (100\%) were detained at the time of ingestion \cite{Elghali_2016, Karp_1991b, Lee_2007}, 88 cases (98\%) were intentional ingestions \cite{Elghali_2016, Karp_1991b, Lee_2007}, 30 cases (33\%) had a psychiatric history documented \cite{Elghali_2016, Karp_1991b, Lee_2007}, 2 cases (2\%) had a history of prior ingestion \cite{Elghali_2016}. No cases were reported for were psychiatric inpatients, were displaced people, were under the influence of alcohol at the time of ingestion, and had a severe disability history.
\paragraph*{Motivation}  70 cases (78\%) reported protest motivation \cite{Elghali_2016, Karp_1991b, Lee_2007}, 12 cases (13\%) reported psychiatric motivation \cite{Karp_1991b}, 6 cases (7\%) reported self-harm motivation \cite{Elghali_2016, Karp_1991b}. No cases were reported for psychosocial motivation and other motivation.
\paragraph*{Object Characteristics}  68 cases (76\%) involved sharp object ingestion \cite{Elghali_2016, Karp_1991b, Lee_2007}, 32 cases (36\%) involved long (\textgreater 5cm) object ingestion \cite{Lee_2007}, 25 cases (28\%) involved ingestion of multiple objects \cite{Elghali_2016, Lee_2007}. No cases were reported for button battery ingestion, magnet ingestion, and involved large diameter (\textgreater 2.5cm) object ingestion.
\paragraph*{Outcomes}  47 cases (52\%) underwent endoscopic intervention \cite{Elghali_2016, Lee_2007}, 29 cases (32\%) were managed conservatively \cite{Elghali_2016, Karp_1991b}, 15 cases (17\%) underwent surgical intervention \cite{Elghali_2016, Karp_1991b, Lee_2007}, 6 cases (7\%) reported complications \cite{Lee_2007}, 1 case (1\%) died \cite{Elghali_2016}.
\paragraph*{Geographical Location}Cases were recorded in 33 countries: 13 cases from USA \cite{Alao_2006i, Ataya_2013, Bhumi_2024f, Fry_2010, Guinan_2019f, Hardy_2023g, Jehangir_2019h, Kerestes_2019, Kumar_2001, Liu_2005, Tammana_2012j, Tay_2004, Tupesis_2004f}; 7 cases from India \cite{Bhasin_2014, Bhattacharjee_2008, Kar_2015, Kariholu_2008, Kumar_2019f, Misra_2013, Wadhwa_2015e} and UK \cite{Beecroft_1998, Berry_2021e, Cauchi_2002, Cox_2007, Gardner_2017h, Qureshi_2016}; 6 cases from Bulgaria \cite{Losanoff_1996, Losanoff_1997e}; 5 cases from Iran \cite{DivsalarP._2023a, Emamhadi_2018, Farhadi_2024h}; 4 cases from Turkey \cite{Akay_2015f, Atayan_2016, Tanrikulu_2015e, Yildiz_2016e}; 2 cases from China \cite{Jin_2023, Li_2013}, Poland \cite{Kobiela_2015, Wnęk_2015f}, and Spain \cite{CamachoDorado_2018, fjbuilsRepeatedBehaviorDeliberate2024}; 1 case from Australia \cite{Apikotoa_2022f}, Bahrain \cite{Ali_2020f}, Croatia \cite{Trgo_2012f}, Ecuador \cite{DelgadoSalazar_2020c}, Egypt \cite{Ali_2022g}, Ethiopia \cite{Mesfin_2022a}, Germany \cite{teWildt_2010}, Greece \cite{Sakellaridis_2008f}, Hungary \cite{Csaky_1998e}, Iraq \cite{Al-Faham_2020k}, Israel \cite{Goldman_1998f}, Italy \cite{Riva_2018j}, Japan \cite{Ohno_2005}, Nepal \cite{Thapa_2019f}, Netherlands \cite{Benoist_2019e}, Oman \cite{AlShaaibi_2021b}, Pakistan \cite{Yasin_2009}, Portugal \cite{Peixoto_2017f}, Qatar \cite{Ali_2017}, Saudi Arabia \cite{Sultan_2024f}, South Africa \cite{Sobnach_2011f}, Sweden \cite{Naji_2012f}, Switzerland \cite{Wildhaber_2005}, and Taiwan \cite{Chang_2017f}. \paragraph*{Gender} 43 cases (60\%) were male \cite{Akay_2015f, Al-Faham_2020k, Alao_2006i, Ali_2017, Ali_2022g, Apikotoa_2022f, Atayan_2016, Benoist_2019e, Berry_2021e, Bhumi_2024f, CamachoDorado_2018, Csaky_1998e, Emamhadi_2018, Farhadi_2024h, Fry_2010, Gardner_2017h, Guinan_2019f, Jehangir_2019h, Jin_2023, Kobiela_2015, Kumar_2001, Kumar_2019f, Liu_2005, Losanoff_1996, Losanoff_1997e, Mesfin_2022a, Misra_2013, Qureshi_2016, Riva_2018j, Sobnach_2011f, Tammana_2012j, Tanrikulu_2015e, Tay_2004, Thapa_2019f, Trgo_2012f, Wadhwa_2015e, Yasin_2009, teWildt_2010}, 28 cases (39\%) were female \cite{AlShaaibi_2021b, Ali_2020f, Ataya_2013, Beecroft_1998, Bhasin_2014, Bhattacharjee_2008, Cauchi_2002, Chang_2017f, Cox_2007, DelgadoSalazar_2020c, DivsalarP._2023a, Goldman_1998f, Hardy_2023g, Kar_2015, Kariholu_2008, Kerestes_2019, Li_2013, Naji_2012f, Ohno_2005, Peixoto_2017f, Sakellaridis_2008f, Sultan_2024f, Tupesis_2004f, Wildhaber_2005, Wnęk_2015f, Yildiz_2016e}, 1 case (1\%) had no gender recorded \cite{fjbuilsRepeatedBehaviorDeliberate2024}. \paragraph*{Age Group} 25 cases (35\%) were between 26 and 40 years of age \cite{Alao_2006i, Ali_2022g, Apikotoa_2022f, Ataya_2013, Benoist_2019e, Bhasin_2014, Chang_2017f, Cox_2007, DelgadoSalazar_2020c, Farhadi_2024h, Fry_2010, Gardner_2017h, Guinan_2019f, Jin_2023, Kumar_2019f, Losanoff_1996, Misra_2013, Qureshi_2016, Riva_2018j, Sakellaridis_2008f, Tammana_2012j, Trgo_2012f, Wnęk_2015f, Yildiz_2016e, fjbuilsRepeatedBehaviorDeliberate2024}, 18 cases (25\%) were between 18 and 25 years of age \cite{Akay_2015f, Ali_2017, Atayan_2016, Bhattacharjee_2008, Csaky_1998e, Kar_2015, Kariholu_2008, Kobiela_2015, Losanoff_1996, Losanoff_1997e, Mesfin_2022a, Peixoto_2017f, Sobnach_2011f, Tupesis_2004f, Yasin_2009}, 13 cases (18\%) were under 18 years of age \cite{AlShaaibi_2021b, Ali_2020f, Cauchi_2002, DivsalarP._2023a, Goldman_1998f, Liu_2005, Naji_2012f, Ohno_2005, Tanrikulu_2015e, Tay_2004, Wildhaber_2005}, 11 cases (15\%) were between 41 and 60 years of age \cite{Al-Faham_2020k, Bhumi_2024f, CamachoDorado_2018, Emamhadi_2018, Hardy_2023g, Jehangir_2019h, Kumar_2001, Sultan_2024f, Thapa_2019f, Wadhwa_2015e, teWildt_2010}, 3 cases (4\%) were over 60 years of age \cite{Beecroft_1998, Kerestes_2019, Li_2013}, 2 cases (3\%) had no age documented \cite{Berry_2021e}. \paragraph*{Population} 36 cases (50\%) had a psychiatric history \cite{AlShaaibi_2021b, Alao_2006i, Ali_2020f, Apikotoa_2022f, Ataya_2013, Atayan_2016, Beecroft_1998, CamachoDorado_2018, Chang_2017f, DelgadoSalazar_2020c, DivsalarP._2023a, Farhadi_2024h, Fry_2010, Guinan_2019f, Hardy_2023g, Jehangir_2019h, Jin_2023, Kar_2015, Kerestes_2019, Kobiela_2015, Kumar_2001, Kumar_2019f, Liu_2005, Mesfin_2022a, Misra_2013, Ohno_2005, Peixoto_2017f, Sakellaridis_2008f, Sultan_2024f, Tammana_2012j, Tanrikulu_2015e, Yildiz_2016e, fjbuilsRepeatedBehaviorDeliberate2024, teWildt_2010}, 19 cases (26\%) had ingested previously \cite{Alao_2006i, Apikotoa_2022f, Berry_2021e, Bhattacharjee_2008, Csaky_1998e, DivsalarP._2023a, Emamhadi_2018, Guinan_2019f, Jehangir_2019h, Jin_2023, Liu_2005, Sakellaridis_2008f, Tanrikulu_2015e, Thapa_2019f, Yildiz_2016e, fjbuilsRepeatedBehaviorDeliberate2024, teWildt_2010}, 12 cases (17\%) were detained persons \cite{Alao_2006i, Ali_2022g, Apikotoa_2022f, Losanoff_1996, Losanoff_1997e, Qureshi_2016, Tammana_2012j, Trgo_2012f}, 7 cases (10\%) were severely disabled \cite{Atayan_2016, Kerestes_2019, Liu_2005, Ohno_2005, Peixoto_2017f, Yildiz_2016e, teWildt_2010}, 4 cases (6\%) were psychiatric inpatients \cite{DivsalarP._2023a, fjbuilsRepeatedBehaviorDeliberate2024, teWildt_2010}, 3 cases (4\%) were under the influence of alcohol \cite{Benoist_2019e, Csaky_1998e, Thapa_2019f}, 2 cases (3\%) were displaced people \cite{Akay_2015f, Gardner_2017h}. \paragraph*{Motivation} 34 cases (47\%) had a psychiatric motivation \cite{Al-Faham_2020k, Alao_2006i, Ali_2020f, Apikotoa_2022f, Ataya_2013, Atayan_2016, Bhasin_2014, Bhattacharjee_2008, DelgadoSalazar_2020c, DivsalarP._2023a, Emamhadi_2018, Farhadi_2024h, Guinan_2019f, Hardy_2023g, Jehangir_2019h, Jin_2023, Kar_2015, Kariholu_2008, Kerestes_2019, Kobiela_2015, Kumar_2001, Kumar_2019f, Li_2013, Liu_2005, Misra_2013, Ohno_2005, Sakellaridis_2008f, Sultan_2024f, Tammana_2012j, Tanrikulu_2015e, Yasin_2009, teWildt_2010}, 21 cases (29\%) were motivated by self-harm intention \cite{Al-Faham_2020k, AlShaaibi_2021b, Alao_2006i, Ali_2017, CamachoDorado_2018, Chang_2017f, Cox_2007, Csaky_1998e, Fry_2010, Li_2013, Losanoff_1996, Losanoff_1997e, Mesfin_2022a, Sakellaridis_2008f, Tammana_2012j, Tanrikulu_2015e, fjbuilsRepeatedBehaviorDeliberate2024}, 17 cases (24\%) had a psychosocial motivation \cite{Akay_2015f, Benoist_2019e, Bhattacharjee_2008, Cauchi_2002, Goldman_1998f, Hardy_2023g, Kobiela_2015, Li_2013, Naji_2012f, Qureshi_2016, Riva_2018j, Sobnach_2011f, Tay_2004, Thapa_2019f, Tupesis_2004f, Wildhaber_2005, Wnęk_2015f}, 9 cases (12\%) were motivated by protest \cite{Bhumi_2024f, Gardner_2017h, Losanoff_1996, Losanoff_1997e, Tupesis_2004f}, 9 cases (12\%) had another documented motivation \cite{Ali_2020f, Ali_2022g, Emamhadi_2018, Guinan_2019f, Peixoto_2017f, Sakellaridis_2008f, Trgo_2012f, Wadhwa_2015e, Yildiz_2016e}. \paragraph*{Object Characteristics} 51 cases (71\%) ingested a large diameter object (\textgreater{}2.5cm) \cite{Akay_2015f, Al-Faham_2020k, AlShaaibi_2021b, Alao_2006i, Ali_2017, Ali_2022g, Apikotoa_2022f, Atayan_2016, Berry_2021e, Bhasin_2014, CamachoDorado_2018, Cauchi_2002, Chang_2017f, Cox_2007, Csaky_1998e, DivsalarP._2023a, Emamhadi_2018, Gardner_2017h, Guinan_2019f, Jehangir_2019h, Jin_2023, Kariholu_2008, Kerestes_2019, Kobiela_2015, Kumar_2001, Kumar_2019f, Losanoff_1996, Losanoff_1997e, Mesfin_2022a, Misra_2013, Naji_2012f, Ohno_2005, Peixoto_2017f, Qureshi_2016, Riva_2018j, Sakellaridis_2008f, Sultan_2024f, Tanrikulu_2015e, Thapa_2019f, Trgo_2012f, Wnęk_2015f, Yildiz_2016e, fjbuilsRepeatedBehaviorDeliberate2024, teWildt_2010}, 44 cases (61\%) ingested multiple objects \cite{Ali_2020f, Apikotoa_2022f, Ataya_2013, Atayan_2016, Beecroft_1998, Bhattacharjee_2008, Bhumi_2024f, CamachoDorado_2018, Cauchi_2002, Emamhadi_2018, Farhadi_2024h, Fry_2010, Goldman_1998f, Guinan_2019f, Hardy_2023g, Jehangir_2019h, Jin_2023, Kar_2015, Kariholu_2008, Kobiela_2015, Kumar_2001, Kumar_2019f, Li_2013, Liu_2005, Losanoff_1996, Mesfin_2022a, Misra_2013, Naji_2012f, Ohno_2005, Sobnach_2011f, Sultan_2024f, Tammana_2012j, Tanrikulu_2015e, Tay_2004, Thapa_2019f, Wadhwa_2015e, Wildhaber_2005, Yasin_2009, fjbuilsRepeatedBehaviorDeliberate2024, teWildt_2010}, 34 cases (47\%) ingested a sharp object \cite{AlShaaibi_2021b, Alao_2006i, Apikotoa_2022f, Ataya_2013, Benoist_2019e, Bhasin_2014, Bhattacharjee_2008, CamachoDorado_2018, Csaky_1998e, DelgadoSalazar_2020c, DivsalarP._2023a, Emamhadi_2018, Farhadi_2024h, Fry_2010, Guinan_2019f, Hardy_2023g, Jehangir_2019h, Jin_2023, Kariholu_2008, Kobiela_2015, Kumar_2019f, Losanoff_1996, Losanoff_1997e, Mesfin_2022a, Misra_2013, Sobnach_2011f, Yasin_2009, teWildt_2010}, 32 cases (44\%) ingested a long object (\textgreater{}5cm) \cite{Al-Faham_2020k, AlShaaibi_2021b, Ali_2017, Ali_2022g, Atayan_2016, Bhasin_2014, CamachoDorado_2018, Chang_2017f, Cox_2007, Csaky_1998e, DivsalarP._2023a, Emamhadi_2018, Fry_2010, Gardner_2017h, Jin_2023, Kariholu_2008, Kerestes_2019, Kobiela_2015, Kumar_2019f, Mesfin_2022a, Misra_2013, Ohno_2005, Qureshi_2016, Sakellaridis_2008f, Sultan_2024f, Thapa_2019f, Trgo_2012f, Yasin_2009, Yildiz_2016e, teWildt_2010}, 9 cases (12\%) ingested a magnet \cite{Ali_2020f, Bhumi_2024f, Cauchi_2002, Liu_2005, Naji_2012f, Ohno_2005, Tanrikulu_2015e, Tay_2004, Wildhaber_2005}, 2 cases (3\%) ingested a button battery \cite{Berry_2021e, Bhumi_2024f}. \paragraph*{Outcomes} 48 cases (67\%) experienced a complication \cite{Ali_2017, Ali_2020f, Apikotoa_2022f, Atayan_2016, Beecroft_1998, Benoist_2019e, Berry_2021e, Bhasin_2014, Bhumi_2024f, CamachoDorado_2018, Cauchi_2002, Cox_2007, Csaky_1998e, DelgadoSalazar_2020c, DivsalarP._2023a, Emamhadi_2018, Farhadi_2024h, Fry_2010, Gardner_2017h, Goldman_1998f, Jin_2023, Kariholu_2008, Kerestes_2019, Kobiela_2015, Kumar_2001, Kumar_2019f, Liu_2005, Losanoff_1996, Mesfin_2022a, Misra_2013, Naji_2012f, Ohno_2005, Sakellaridis_2008f, Sobnach_2011f, Sultan_2024f, Tanrikulu_2015e, Tay_2004, Thapa_2019f, Trgo_2012f, Tupesis_2004f, Wildhaber_2005, Wnęk_2015f, Yasin_2009, Yildiz_2016e}, 44 cases (61\%) underwent surgery \cite{Al-Faham_2020k, AlShaaibi_2021b, Alao_2006i, Ali_2017, Ali_2020f, Atayan_2016, Beecroft_1998, Bhasin_2014, CamachoDorado_2018, Cauchi_2002, Chang_2017f, Cox_2007, Csaky_1998e, DelgadoSalazar_2020c, DivsalarP._2023a, Farhadi_2024h, Fry_2010, Gardner_2017h, Jin_2023, Kariholu_2008, Kerestes_2019, Kobiela_2015, Kumar_2019f, Liu_2005, Losanoff_1996, Losanoff_1997e, Mesfin_2022a, Misra_2013, Naji_2012f, Sobnach_2011f, Tanrikulu_2015e, Tay_2004, Thapa_2019f, Tupesis_2004f, Wildhaber_2005, Wnęk_2015f, Yasin_2009, Yildiz_2016e, fjbuilsRepeatedBehaviorDeliberate2024}, 31 cases (43\%) underwent endoscopy \cite{Akay_2015f, Ali_2022g, Apikotoa_2022f, Atayan_2016, Benoist_2019e, Berry_2021e, Bhasin_2014, Bhumi_2024f, CamachoDorado_2018, Chang_2017f, DelgadoSalazar_2020c, Gardner_2017h, Guinan_2019f, Hardy_2023g, Jehangir_2019h, Kariholu_2008, Li_2013, Liu_2005, Ohno_2005, Peixoto_2017f, Qureshi_2016, Riva_2018j, Sakellaridis_2008f, Sultan_2024f, Tammana_2012j, Tanrikulu_2015e, Trgo_2012f, Wadhwa_2015e, Wnęk_2015f, teWildt_2010}, 7 cases (10\%) were managed conservatively \cite{Ataya_2013, Bhattacharjee_2008, DivsalarP._2023a, Emamhadi_2018, Goldman_1998f, Kar_2015, Kumar_2001}, 2 cases (3\%) died \cite{Emamhadi_2018, Kumar_2001}. \paragraph*{Geographical Location}Cases were recorded in 33 countries: 13 cases from USA \cite{Alao_2006i, Ataya_2013, Bhumi_2024f, Fry_2010, Guinan_2019f, Hardy_2023g, Jehangir_2019h, Kerestes_2019, Kumar_2001, Liu_2005, Tammana_2012j, Tay_2004, Tupesis_2004f}; 7 cases from India \cite{Bhasin_2014, Bhattacharjee_2008, Kar_2015, Kariholu_2008, Kumar_2019f, Misra_2013, Wadhwa_2015e} and UK \cite{Beecroft_1998, Berry_2021e, Cauchi_2002, Cox_2007, Gardner_2017h, Qureshi_2016}; 6 cases from Bulgaria \cite{Losanoff_1996, Losanoff_1997e}; 5 cases from Iran \cite{DivsalarP._2023a, Emamhadi_2018, Farhadi_2024h}; 4 cases from Turkey \cite{Akay_2015f, Atayan_2016, Tanrikulu_2015e, Yildiz_2016e}; 2 cases from China \cite{Jin_2023, Li_2013}, Poland \cite{Kobiela_2015, Wnęk_2015f}, and Spain \cite{CamachoDorado_2018, fjbuilsRepeatedBehaviorDeliberate2024}; 1 case from Australia \cite{Apikotoa_2022f}, Bahrain \cite{Ali_2020f}, Croatia \cite{Trgo_2012f}, Ecuador \cite{DelgadoSalazar_2020c}, Egypt \cite{Ali_2022g}, Ethiopia \cite{Mesfin_2022a}, Germany \cite{teWildt_2010}, Greece \cite{Sakellaridis_2008f}, Hungary \cite{Csaky_1998e}, Iraq \cite{Al-Faham_2020k}, Israel \cite{Goldman_1998f}, Italy \cite{Riva_2018j}, Japan \cite{Ohno_2005}, Nepal \cite{Thapa_2019f}, Netherlands \cite{Benoist_2019e}, Oman \cite{AlShaaibi_2021b}, Pakistan \cite{Yasin_2009}, Portugal \cite{Peixoto_2017f}, Qatar \cite{Ali_2017}, Saudi Arabia \cite{Sultan_2024f}, South Africa \cite{Sobnach_2011f}, Sweden \cite{Naji_2012f}, Switzerland \cite{Wildhaber_2005}, and Taiwan \cite{Chang_2017f}. \paragraph*{Gender} 43 cases (60\%) were male \cite{Akay_2015f, Al-Faham_2020k, Alao_2006i, Ali_2017, Ali_2022g, Apikotoa_2022f, Atayan_2016, Benoist_2019e, Berry_2021e, Bhumi_2024f, CamachoDorado_2018, Csaky_1998e, Emamhadi_2018, Farhadi_2024h, Fry_2010, Gardner_2017h, Guinan_2019f, Jehangir_2019h, Jin_2023, Kobiela_2015, Kumar_2001, Kumar_2019f, Liu_2005, Losanoff_1996, Losanoff_1997e, Mesfin_2022a, Misra_2013, Qureshi_2016, Riva_2018j, Sobnach_2011f, Tammana_2012j, Tanrikulu_2015e, Tay_2004, Thapa_2019f, Trgo_2012f, Wadhwa_2015e, Yasin_2009, teWildt_2010}, 28 cases (39\%) were female \cite{AlShaaibi_2021b, Ali_2020f, Ataya_2013, Beecroft_1998, Bhasin_2014, Bhattacharjee_2008, Cauchi_2002, Chang_2017f, Cox_2007, DelgadoSalazar_2020c, DivsalarP._2023a, Goldman_1998f, Hardy_2023g, Kar_2015, Kariholu_2008, Kerestes_2019, Li_2013, Naji_2012f, Ohno_2005, Peixoto_2017f, Sakellaridis_2008f, Sultan_2024f, Tupesis_2004f, Wildhaber_2005, Wnęk_2015f, Yildiz_2016e}, 1 case (1\%) had no gender recorded \cite{fjbuilsRepeatedBehaviorDeliberate2024}. \paragraph*{Age Group} 25 cases (35\%) were between 26 and 40 years of age \cite{Alao_2006i, Ali_2022g, Apikotoa_2022f, Ataya_2013, Benoist_2019e, Bhasin_2014, Chang_2017f, Cox_2007, DelgadoSalazar_2020c, Farhadi_2024h, Fry_2010, Gardner_2017h, Guinan_2019f, Jin_2023, Kumar_2019f, Losanoff_1996, Misra_2013, Qureshi_2016, Riva_2018j, Sakellaridis_2008f, Tammana_2012j, Trgo_2012f, Wnęk_2015f, Yildiz_2016e, fjbuilsRepeatedBehaviorDeliberate2024}, 18 cases (25\%) were between 18 and 25 years of age \cite{Akay_2015f, Ali_2017, Atayan_2016, Bhattacharjee_2008, Csaky_1998e, Kar_2015, Kariholu_2008, Kobiela_2015, Losanoff_1996, Losanoff_1997e, Mesfin_2022a, Peixoto_2017f, Sobnach_2011f, Tupesis_2004f, Yasin_2009}, 13 cases (18\%) were under 18 years of age \cite{AlShaaibi_2021b, Ali_2020f, Cauchi_2002, DivsalarP._2023a, Goldman_1998f, Liu_2005, Naji_2012f, Ohno_2005, Tanrikulu_2015e, Tay_2004, Wildhaber_2005}, 11 cases (15\%) were between 41 and 60 years of age \cite{Al-Faham_2020k, Bhumi_2024f, CamachoDorado_2018, Emamhadi_2018, Hardy_2023g, Jehangir_2019h, Kumar_2001, Sultan_2024f, Thapa_2019f, Wadhwa_2015e, teWildt_2010}, 3 cases (4\%) were over 60 years of age \cite{Beecroft_1998, Kerestes_2019, Li_2013}, 2 cases (3\%) had no age documented \cite{Berry_2021e}. \paragraph*{Population} 36 cases (50\%) had a psychiatric history \cite{AlShaaibi_2021b, Alao_2006i, Ali_2020f, Apikotoa_2022f, Ataya_2013, Atayan_2016, Beecroft_1998, CamachoDorado_2018, Chang_2017f, DelgadoSalazar_2020c, DivsalarP._2023a, Farhadi_2024h, Fry_2010, Guinan_2019f, Hardy_2023g, Jehangir_2019h, Jin_2023, Kar_2015, Kerestes_2019, Kobiela_2015, Kumar_2001, Kumar_2019f, Liu_2005, Mesfin_2022a, Misra_2013, Ohno_2005, Peixoto_2017f, Sakellaridis_2008f, Sultan_2024f, Tammana_2012j, Tanrikulu_2015e, Yildiz_2016e, fjbuilsRepeatedBehaviorDeliberate2024, teWildt_2010}, 19 cases (26\%) had ingested previously \cite{Alao_2006i, Apikotoa_2022f, Berry_2021e, Bhattacharjee_2008, Csaky_1998e, DivsalarP._2023a, Emamhadi_2018, Guinan_2019f, Jehangir_2019h, Jin_2023, Liu_2005, Sakellaridis_2008f, Tanrikulu_2015e, Thapa_2019f, Yildiz_2016e, fjbuilsRepeatedBehaviorDeliberate2024, teWildt_2010}, 12 cases (17\%) were detained persons \cite{Alao_2006i, Ali_2022g, Apikotoa_2022f, Losanoff_1996, Losanoff_1997e, Qureshi_2016, Tammana_2012j, Trgo_2012f}, 7 cases (10\%) were severely disabled \cite{Atayan_2016, Kerestes_2019, Liu_2005, Ohno_2005, Peixoto_2017f, Yildiz_2016e, teWildt_2010}, 4 cases (6\%) were psychiatric inpatients \cite{DivsalarP._2023a, fjbuilsRepeatedBehaviorDeliberate2024, teWildt_2010}, 3 cases (4\%) were under the influence of alcohol \cite{Benoist_2019e, Csaky_1998e, Thapa_2019f}, 2 cases (3\%) were displaced people \cite{Akay_2015f, Gardner_2017h}. \paragraph*{Motivation} 34 cases (47\%) had a psychiatric motivation \cite{Al-Faham_2020k, Alao_2006i, Ali_2020f, Apikotoa_2022f, Ataya_2013, Atayan_2016, Bhasin_2014, Bhattacharjee_2008, DelgadoSalazar_2020c, DivsalarP._2023a, Emamhadi_2018, Farhadi_2024h, Guinan_2019f, Hardy_2023g, Jehangir_2019h, Jin_2023, Kar_2015, Kariholu_2008, Kerestes_2019, Kobiela_2015, Kumar_2001, Kumar_2019f, Li_2013, Liu_2005, Misra_2013, Ohno_2005, Sakellaridis_2008f, Sultan_2024f, Tammana_2012j, Tanrikulu_2015e, Yasin_2009, teWildt_2010}, 21 cases (29\%) were motivated by self-harm intention \cite{Al-Faham_2020k, AlShaaibi_2021b, Alao_2006i, Ali_2017, CamachoDorado_2018, Chang_2017f, Cox_2007, Csaky_1998e, Fry_2010, Li_2013, Losanoff_1996, Losanoff_1997e, Mesfin_2022a, Sakellaridis_2008f, Tammana_2012j, Tanrikulu_2015e, fjbuilsRepeatedBehaviorDeliberate2024}, 17 cases (24\%) had a psychosocial motivation \cite{Akay_2015f, Benoist_2019e, Bhattacharjee_2008, Cauchi_2002, Goldman_1998f, Hardy_2023g, Kobiela_2015, Li_2013, Naji_2012f, Qureshi_2016, Riva_2018j, Sobnach_2011f, Tay_2004, Thapa_2019f, Tupesis_2004f, Wildhaber_2005, Wnęk_2015f}, 9 cases (12\%) were motivated by protest \cite{Bhumi_2024f, Gardner_2017h, Losanoff_1996, Losanoff_1997e, Tupesis_2004f}, 9 cases (12\%) had another documented motivation \cite{Ali_2020f, Ali_2022g, Emamhadi_2018, Guinan_2019f, Peixoto_2017f, Sakellaridis_2008f, Trgo_2012f, Wadhwa_2015e, Yildiz_2016e}. \paragraph*{Object Characteristics} 51 cases (71\%) ingested a large diameter object (\textgreater{}2.5cm) \cite{Akay_2015f, Al-Faham_2020k, AlShaaibi_2021b, Alao_2006i, Ali_2017, Ali_2022g, Apikotoa_2022f, Atayan_2016, Berry_2021e, Bhasin_2014, CamachoDorado_2018, Cauchi_2002, Chang_2017f, Cox_2007, Csaky_1998e, DivsalarP._2023a, Emamhadi_2018, Gardner_2017h, Guinan_2019f, Jehangir_2019h, Jin_2023, Kariholu_2008, Kerestes_2019, Kobiela_2015, Kumar_2001, Kumar_2019f, Losanoff_1996, Losanoff_1997e, Mesfin_2022a, Misra_2013, Naji_2012f, Ohno_2005, Peixoto_2017f, Qureshi_2016, Riva_2018j, Sakellaridis_2008f, Sultan_2024f, Tanrikulu_2015e, Thapa_2019f, Trgo_2012f, Wnęk_2015f, Yildiz_2016e, fjbuilsRepeatedBehaviorDeliberate2024, teWildt_2010}, 44 cases (61\%) ingested multiple objects \cite{Ali_2020f, Apikotoa_2022f, Ataya_2013, Atayan_2016, Beecroft_1998, Bhattacharjee_2008, Bhumi_2024f, CamachoDorado_2018, Cauchi_2002, Emamhadi_2018, Farhadi_2024h, Fry_2010, Goldman_1998f, Guinan_2019f, Hardy_2023g, Jehangir_2019h, Jin_2023, Kar_2015, Kariholu_2008, Kobiela_2015, Kumar_2001, Kumar_2019f, Li_2013, Liu_2005, Losanoff_1996, Mesfin_2022a, Misra_2013, Naji_2012f, Ohno_2005, Sobnach_2011f, Sultan_2024f, Tammana_2012j, Tanrikulu_2015e, Tay_2004, Thapa_2019f, Wadhwa_2015e, Wildhaber_2005, Yasin_2009, fjbuilsRepeatedBehaviorDeliberate2024, teWildt_2010}, 34 cases (47\%) ingested a sharp object \cite{AlShaaibi_2021b, Alao_2006i, Apikotoa_2022f, Ataya_2013, Benoist_2019e, Bhasin_2014, Bhattacharjee_2008, CamachoDorado_2018, Csaky_1998e, DelgadoSalazar_2020c, DivsalarP._2023a, Emamhadi_2018, Farhadi_2024h, Fry_2010, Guinan_2019f, Hardy_2023g, Jehangir_2019h, Jin_2023, Kariholu_2008, Kobiela_2015, Kumar_2019f, Losanoff_1996, Losanoff_1997e, Mesfin_2022a, Misra_2013, Sobnach_2011f, Yasin_2009, teWildt_2010}, 32 cases (44\%) ingested a long object (\textgreater{}5cm) \cite{Al-Faham_2020k, AlShaaibi_2021b, Ali_2017, Ali_2022g, Atayan_2016, Bhasin_2014, CamachoDorado_2018, Chang_2017f, Cox_2007, Csaky_1998e, DivsalarP._2023a, Emamhadi_2018, Fry_2010, Gardner_2017h, Jin_2023, Kariholu_2008, Kerestes_2019, Kobiela_2015, Kumar_2019f, Mesfin_2022a, Misra_2013, Ohno_2005, Qureshi_2016, Sakellaridis_2008f, Sultan_2024f, Thapa_2019f, Trgo_2012f, Yasin_2009, Yildiz_2016e, teWildt_2010}, 9 cases (12\%) ingested a magnet \cite{Ali_2020f, Bhumi_2024f, Cauchi_2002, Liu_2005, Naji_2012f, Ohno_2005, Tanrikulu_2015e, Tay_2004, Wildhaber_2005}, 2 cases (3\%) ingested a button battery \cite{Berry_2021e, Bhumi_2024f}. \paragraph*{Outcomes} 48 cases (67\%) experienced a complication \cite{Ali_2017, Ali_2020f, Apikotoa_2022f, Atayan_2016, Beecroft_1998, Benoist_2019e, Berry_2021e, Bhasin_2014, Bhumi_2024f, CamachoDorado_2018, Cauchi_2002, Cox_2007, Csaky_1998e, DelgadoSalazar_2020c, DivsalarP._2023a, Emamhadi_2018, Farhadi_2024h, Fry_2010, Gardner_2017h, Goldman_1998f, Jin_2023, Kariholu_2008, Kerestes_2019, Kobiela_2015, Kumar_2001, Kumar_2019f, Liu_2005, Losanoff_1996, Mesfin_2022a, Misra_2013, Naji_2012f, Ohno_2005, Sakellaridis_2008f, Sobnach_2011f, Sultan_2024f, Tanrikulu_2015e, Tay_2004, Thapa_2019f, Trgo_2012f, Tupesis_2004f, Wildhaber_2005, Wnęk_2015f, Yasin_2009, Yildiz_2016e}, 44 cases (61\%) underwent surgery \cite{Al-Faham_2020k, AlShaaibi_2021b, Alao_2006i, Ali_2017, Ali_2020f, Atayan_2016, Beecroft_1998, Bhasin_2014, CamachoDorado_2018, Cauchi_2002, Chang_2017f, Cox_2007, Csaky_1998e, DelgadoSalazar_2020c, DivsalarP._2023a, Farhadi_2024h, Fry_2010, Gardner_2017h, Jin_2023, Kariholu_2008, Kerestes_2019, Kobiela_2015, Kumar_2019f, Liu_2005, Losanoff_1996, Losanoff_1997e, Mesfin_2022a, Misra_2013, Naji_2012f, Sobnach_2011f, Tanrikulu_2015e, Tay_2004, Thapa_2019f, Tupesis_2004f, Wildhaber_2005, Wnęk_2015f, Yasin_2009, Yildiz_2016e, fjbuilsRepeatedBehaviorDeliberate2024}, 31 cases (43\%) underwent endoscopy \cite{Akay_2015f, Ali_2022g, Apikotoa_2022f, Atayan_2016, Benoist_2019e, Berry_2021e, Bhasin_2014, Bhumi_2024f, CamachoDorado_2018, Chang_2017f, DelgadoSalazar_2020c, Gardner_2017h, Guinan_2019f, Hardy_2023g, Jehangir_2019h, Kariholu_2008, Li_2013, Liu_2005, Ohno_2005, Peixoto_2017f, Qureshi_2016, Riva_2018j, Sakellaridis_2008f, Sultan_2024f, Tammana_2012j, Tanrikulu_2015e, Trgo_2012f, Wadhwa_2015e, Wnęk_2015f, teWildt_2010}, 7 cases (10\%) were managed conservatively \cite{Ataya_2013, Bhattacharjee_2008, DivsalarP._2023a, Emamhadi_2018, Goldman_1998f, Kar_2015, Kumar_2001}, 2 cases (3\%) died \cite{Emamhadi_2018, Kumar_2001}. All 90 were male gender. 90 cases (100\%) were detained at the time of ingestion \cite{Elghali_2016, Karp_1991b, Lee_2007}, 88 cases (98\%) were intentional ingestions \cite{Elghali_2016, Karp_1991b, Lee_2007}, 30 cases (33\%) had a psychiatric history documented \cite{Elghali_2016, Karp_1991b, Lee_2007}, 2 cases (2\%) had a history of prior ingestion \cite{Elghali_2016}. No cases were reported for were psychiatric inpatients, were displaced people, were under the influence of alcohol at the time of ingestion, and had a severe disability history.
\paragraph*{Motivation}  70 cases (78\%) reported protest motivation \cite{Elghali_2016, Karp_1991b, Lee_2007}, 12 cases (13\%) reported psychiatric motivation \cite{Karp_1991b}, 6 cases (7\%) reported self-harm motivation \cite{Elghali_2016, Karp_1991b}. No cases were reported for psychosocial motivation and other motivation.
\paragraph*{Object Characteristics}  68 cases (76\%) involved sharp object ingestion \cite{Elghali_2016, Karp_1991b, Lee_2007}, 32 cases (36\%) involved long (\textgreater 5cm) object ingestion \cite{Lee_2007}, 25 cases (28\%) involved ingestion of multiple objects \cite{Elghali_2016, Lee_2007}. No cases were reported for button battery ingestion, magnet ingestion, and involved large diameter (\textgreater 2.5cm) object ingestion.
\paragraph*{Outcomes}  47 cases (52\%) underwent endoscopic intervention \cite{Elghali_2016, Lee_2007}, 29 cases (32\%) were managed conservatively \cite{Elghali_2016, Karp_1991b}, 15 cases (17\%) underwent surgical intervention \cite{Elghali_2016, Karp_1991b, Lee_2007}, 6 cases (7\%) reported complications \cite{Lee_2007}, 1 case (1\%) died \cite{Elghali_2016}.
\paragraph*{Geographical Location}Cases were recorded in 33 countries: 13 cases from USA \cite{Alao_2006i, Ataya_2013, Bhumi_2024f, Fry_2010, Guinan_2019f, Hardy_2023g, Jehangir_2019h, Kerestes_2019, Kumar_2001, Liu_2005, Tammana_2012j, Tay_2004, Tupesis_2004f}; 7 cases from India \cite{Bhasin_2014, Bhattacharjee_2008, Kar_2015, Kariholu_2008, Kumar_2019f, Misra_2013, Wadhwa_2015e} and UK \cite{Beecroft_1998, Berry_2021e, Cauchi_2002, Cox_2007, Gardner_2017h, Qureshi_2016}; 6 cases from Bulgaria \cite{Losanoff_1996, Losanoff_1997e}; 5 cases from Iran \cite{DivsalarP._2023a, Emamhadi_2018, Farhadi_2024h}; 4 cases from Turkey \cite{Akay_2015f, Atayan_2016, Tanrikulu_2015e, Yildiz_2016e}; 2 cases from China \cite{Jin_2023, Li_2013}, Poland \cite{Kobiela_2015, Wnęk_2015f}, and Spain \cite{CamachoDorado_2018, fjbuilsRepeatedBehaviorDeliberate2024}; 1 case from Australia \cite{Apikotoa_2022f}, Bahrain \cite{Ali_2020f}, Croatia \cite{Trgo_2012f}, Ecuador \cite{DelgadoSalazar_2020c}, Egypt \cite{Ali_2022g}, Ethiopia \cite{Mesfin_2022a}, Germany \cite{teWildt_2010}, Greece \cite{Sakellaridis_2008f}, Hungary \cite{Csaky_1998e}, Iraq \cite{Al-Faham_2020k}, Israel \cite{Goldman_1998f}, Italy \cite{Riva_2018j}, Japan \cite{Ohno_2005}, Nepal \cite{Thapa_2019f}, Netherlands \cite{Benoist_2019e}, Oman \cite{AlShaaibi_2021b}, Pakistan \cite{Yasin_2009}, Portugal \cite{Peixoto_2017f}, Qatar \cite{Ali_2017}, Saudi Arabia \cite{Sultan_2024f}, South Africa \cite{Sobnach_2011f}, Sweden \cite{Naji_2012f}, Switzerland \cite{Wildhaber_2005}, and Taiwan \cite{Chang_2017f}. \paragraph*{Gender} 43 cases (60\%) were male \cite{Akay_2015f, Al-Faham_2020k, Alao_2006i, Ali_2017, Ali_2022g, Apikotoa_2022f, Atayan_2016, Benoist_2019e, Berry_2021e, Bhumi_2024f, CamachoDorado_2018, Csaky_1998e, Emamhadi_2018, Farhadi_2024h, Fry_2010, Gardner_2017h, Guinan_2019f, Jehangir_2019h, Jin_2023, Kobiela_2015, Kumar_2001, Kumar_2019f, Liu_2005, Losanoff_1996, Losanoff_1997e, Mesfin_2022a, Misra_2013, Qureshi_2016, Riva_2018j, Sobnach_2011f, Tammana_2012j, Tanrikulu_2015e, Tay_2004, Thapa_2019f, Trgo_2012f, Wadhwa_2015e, Yasin_2009, teWildt_2010}, 28 cases (39\%) were female \cite{AlShaaibi_2021b, Ali_2020f, Ataya_2013, Beecroft_1998, Bhasin_2014, Bhattacharjee_2008, Cauchi_2002, Chang_2017f, Cox_2007, DelgadoSalazar_2020c, DivsalarP._2023a, Goldman_1998f, Hardy_2023g, Kar_2015, Kariholu_2008, Kerestes_2019, Li_2013, Naji_2012f, Ohno_2005, Peixoto_2017f, Sakellaridis_2008f, Sultan_2024f, Tupesis_2004f, Wildhaber_2005, Wnęk_2015f, Yildiz_2016e}, 1 case (1\%) had no gender recorded \cite{fjbuilsRepeatedBehaviorDeliberate2024}. \paragraph*{Age Group} 25 cases (35\%) were between 26 and 40 years of age \cite{Alao_2006i, Ali_2022g, Apikotoa_2022f, Ataya_2013, Benoist_2019e, Bhasin_2014, Chang_2017f, Cox_2007, DelgadoSalazar_2020c, Farhadi_2024h, Fry_2010, Gardner_2017h, Guinan_2019f, Jin_2023, Kumar_2019f, Losanoff_1996, Misra_2013, Qureshi_2016, Riva_2018j, Sakellaridis_2008f, Tammana_2012j, Trgo_2012f, Wnęk_2015f, Yildiz_2016e, fjbuilsRepeatedBehaviorDeliberate2024}, 18 cases (25\%) were between 18 and 25 years of age \cite{Akay_2015f, Ali_2017, Atayan_2016, Bhattacharjee_2008, Csaky_1998e, Kar_2015, Kariholu_2008, Kobiela_2015, Losanoff_1996, Losanoff_1997e, Mesfin_2022a, Peixoto_2017f, Sobnach_2011f, Tupesis_2004f, Yasin_2009}, 13 cases (18\%) were under 18 years of age \cite{AlShaaibi_2021b, Ali_2020f, Cauchi_2002, DivsalarP._2023a, Goldman_1998f, Liu_2005, Naji_2012f, Ohno_2005, Tanrikulu_2015e, Tay_2004, Wildhaber_2005}, 11 cases (15\%) were between 41 and 60 years of age \cite{Al-Faham_2020k, Bhumi_2024f, CamachoDorado_2018, Emamhadi_2018, Hardy_2023g, Jehangir_2019h, Kumar_2001, Sultan_2024f, Thapa_2019f, Wadhwa_2015e, teWildt_2010}, 3 cases (4\%) were over 60 years of age \cite{Beecroft_1998, Kerestes_2019, Li_2013}, 2 cases (3\%) had no age documented \cite{Berry_2021e}. \paragraph*{Population} 36 cases (50\%) had a psychiatric history \cite{AlShaaibi_2021b, Alao_2006i, Ali_2020f, Apikotoa_2022f, Ataya_2013, Atayan_2016, Beecroft_1998, CamachoDorado_2018, Chang_2017f, DelgadoSalazar_2020c, DivsalarP._2023a, Farhadi_2024h, Fry_2010, Guinan_2019f, Hardy_2023g, Jehangir_2019h, Jin_2023, Kar_2015, Kerestes_2019, Kobiela_2015, Kumar_2001, Kumar_2019f, Liu_2005, Mesfin_2022a, Misra_2013, Ohno_2005, Peixoto_2017f, Sakellaridis_2008f, Sultan_2024f, Tammana_2012j, Tanrikulu_2015e, Yildiz_2016e, fjbuilsRepeatedBehaviorDeliberate2024, teWildt_2010}, 19 cases (26\%) had ingested previously \cite{Alao_2006i, Apikotoa_2022f, Berry_2021e, Bhattacharjee_2008, Csaky_1998e, DivsalarP._2023a, Emamhadi_2018, Guinan_2019f, Jehangir_2019h, Jin_2023, Liu_2005, Sakellaridis_2008f, Tanrikulu_2015e, Thapa_2019f, Yildiz_2016e, fjbuilsRepeatedBehaviorDeliberate2024, teWildt_2010}, 12 cases (17\%) were detained persons \cite{Alao_2006i, Ali_2022g, Apikotoa_2022f, Losanoff_1996, Losanoff_1997e, Qureshi_2016, Tammana_2012j, Trgo_2012f}, 7 cases (10\%) were severely disabled \cite{Atayan_2016, Kerestes_2019, Liu_2005, Ohno_2005, Peixoto_2017f, Yildiz_2016e, teWildt_2010}, 4 cases (6\%) were psychiatric inpatients \cite{DivsalarP._2023a, fjbuilsRepeatedBehaviorDeliberate2024, teWildt_2010}, 3 cases (4\%) were under the influence of alcohol \cite{Benoist_2019e, Csaky_1998e, Thapa_2019f}, 2 cases (3\%) were displaced people \cite{Akay_2015f, Gardner_2017h}. \paragraph*{Motivation} 34 cases (47\%) had a psychiatric motivation \cite{Al-Faham_2020k, Alao_2006i, Ali_2020f, Apikotoa_2022f, Ataya_2013, Atayan_2016, Bhasin_2014, Bhattacharjee_2008, DelgadoSalazar_2020c, DivsalarP._2023a, Emamhadi_2018, Farhadi_2024h, Guinan_2019f, Hardy_2023g, Jehangir_2019h, Jin_2023, Kar_2015, Kariholu_2008, Kerestes_2019, Kobiela_2015, Kumar_2001, Kumar_2019f, Li_2013, Liu_2005, Misra_2013, Ohno_2005, Sakellaridis_2008f, Sultan_2024f, Tammana_2012j, Tanrikulu_2015e, Yasin_2009, teWildt_2010}, 21 cases (29\%) were motivated by self-harm intention \cite{Al-Faham_2020k, AlShaaibi_2021b, Alao_2006i, Ali_2017, CamachoDorado_2018, Chang_2017f, Cox_2007, Csaky_1998e, Fry_2010, Li_2013, Losanoff_1996, Losanoff_1997e, Mesfin_2022a, Sakellaridis_2008f, Tammana_2012j, Tanrikulu_2015e, fjbuilsRepeatedBehaviorDeliberate2024}, 17 cases (24\%) had a psychosocial motivation \cite{Akay_2015f, Benoist_2019e, Bhattacharjee_2008, Cauchi_2002, Goldman_1998f, Hardy_2023g, Kobiela_2015, Li_2013, Naji_2012f, Qureshi_2016, Riva_2018j, Sobnach_2011f, Tay_2004, Thapa_2019f, Tupesis_2004f, Wildhaber_2005, Wnęk_2015f}, 9 cases (12\%) were motivated by protest \cite{Bhumi_2024f, Gardner_2017h, Losanoff_1996, Losanoff_1997e, Tupesis_2004f}, 9 cases (12\%) had another documented motivation \cite{Ali_2020f, Ali_2022g, Emamhadi_2018, Guinan_2019f, Peixoto_2017f, Sakellaridis_2008f, Trgo_2012f, Wadhwa_2015e, Yildiz_2016e}. \paragraph*{Object Characteristics} 51 cases (71\%) ingested a large diameter object (\textgreater{}2.5cm) \cite{Akay_2015f, Al-Faham_2020k, AlShaaibi_2021b, Alao_2006i, Ali_2017, Ali_2022g, Apikotoa_2022f, Atayan_2016, Berry_2021e, Bhasin_2014, CamachoDorado_2018, Cauchi_2002, Chang_2017f, Cox_2007, Csaky_1998e, DivsalarP._2023a, Emamhadi_2018, Gardner_2017h, Guinan_2019f, Jehangir_2019h, Jin_2023, Kariholu_2008, Kerestes_2019, Kobiela_2015, Kumar_2001, Kumar_2019f, Losanoff_1996, Losanoff_1997e, Mesfin_2022a, Misra_2013, Naji_2012f, Ohno_2005, Peixoto_2017f, Qureshi_2016, Riva_2018j, Sakellaridis_2008f, Sultan_2024f, Tanrikulu_2015e, Thapa_2019f, Trgo_2012f, Wnęk_2015f, Yildiz_2016e, fjbuilsRepeatedBehaviorDeliberate2024, teWildt_2010}, 44 cases (61\%) ingested multiple objects \cite{Ali_2020f, Apikotoa_2022f, Ataya_2013, Atayan_2016, Beecroft_1998, Bhattacharjee_2008, Bhumi_2024f, CamachoDorado_2018, Cauchi_2002, Emamhadi_2018, Farhadi_2024h, Fry_2010, Goldman_1998f, Guinan_2019f, Hardy_2023g, Jehangir_2019h, Jin_2023, Kar_2015, Kariholu_2008, Kobiela_2015, Kumar_2001, Kumar_2019f, Li_2013, Liu_2005, Losanoff_1996, Mesfin_2022a, Misra_2013, Naji_2012f, Ohno_2005, Sobnach_2011f, Sultan_2024f, Tammana_2012j, Tanrikulu_2015e, Tay_2004, Thapa_2019f, Wadhwa_2015e, Wildhaber_2005, Yasin_2009, fjbuilsRepeatedBehaviorDeliberate2024, teWildt_2010}, 34 cases (47\%) ingested a sharp object \cite{AlShaaibi_2021b, Alao_2006i, Apikotoa_2022f, Ataya_2013, Benoist_2019e, Bhasin_2014, Bhattacharjee_2008, CamachoDorado_2018, Csaky_1998e, DelgadoSalazar_2020c, DivsalarP._2023a, Emamhadi_2018, Farhadi_2024h, Fry_2010, Guinan_2019f, Hardy_2023g, Jehangir_2019h, Jin_2023, Kariholu_2008, Kobiela_2015, Kumar_2019f, Losanoff_1996, Losanoff_1997e, Mesfin_2022a, Misra_2013, Sobnach_2011f, Yasin_2009, teWildt_2010}, 32 cases (44\%) ingested a long object (\textgreater{}5cm) \cite{Al-Faham_2020k, AlShaaibi_2021b, Ali_2017, Ali_2022g, Atayan_2016, Bhasin_2014, CamachoDorado_2018, Chang_2017f, Cox_2007, Csaky_1998e, DivsalarP._2023a, Emamhadi_2018, Fry_2010, Gardner_2017h, Jin_2023, Kariholu_2008, Kerestes_2019, Kobiela_2015, Kumar_2019f, Mesfin_2022a, Misra_2013, Ohno_2005, Qureshi_2016, Sakellaridis_2008f, Sultan_2024f, Thapa_2019f, Trgo_2012f, Yasin_2009, Yildiz_2016e, teWildt_2010}, 9 cases (12\%) ingested a magnet \cite{Ali_2020f, Bhumi_2024f, Cauchi_2002, Liu_2005, Naji_2012f, Ohno_2005, Tanrikulu_2015e, Tay_2004, Wildhaber_2005}, 2 cases (3\%) ingested a button battery \cite{Berry_2021e, Bhumi_2024f}. \paragraph*{Outcomes} 48 cases (67\%) experienced a complication \cite{Ali_2017, Ali_2020f, Apikotoa_2022f, Atayan_2016, Beecroft_1998, Benoist_2019e, Berry_2021e, Bhasin_2014, Bhumi_2024f, CamachoDorado_2018, Cauchi_2002, Cox_2007, Csaky_1998e, DelgadoSalazar_2020c, DivsalarP._2023a, Emamhadi_2018, Farhadi_2024h, Fry_2010, Gardner_2017h, Goldman_1998f, Jin_2023, Kariholu_2008, Kerestes_2019, Kobiela_2015, Kumar_2001, Kumar_2019f, Liu_2005, Losanoff_1996, Mesfin_2022a, Misra_2013, Naji_2012f, Ohno_2005, Sakellaridis_2008f, Sobnach_2011f, Sultan_2024f, Tanrikulu_2015e, Tay_2004, Thapa_2019f, Trgo_2012f, Tupesis_2004f, Wildhaber_2005, Wnęk_2015f, Yasin_2009, Yildiz_2016e}, 44 cases (61\%) underwent surgery \cite{Al-Faham_2020k, AlShaaibi_2021b, Alao_2006i, Ali_2017, Ali_2020f, Atayan_2016, Beecroft_1998, Bhasin_2014, CamachoDorado_2018, Cauchi_2002, Chang_2017f, Cox_2007, Csaky_1998e, DelgadoSalazar_2020c, DivsalarP._2023a, Farhadi_2024h, Fry_2010, Gardner_2017h, Jin_2023, Kariholu_2008, Kerestes_2019, Kobiela_2015, Kumar_2019f, Liu_2005, Losanoff_1996, Losanoff_1997e, Mesfin_2022a, Misra_2013, Naji_2012f, Sobnach_2011f, Tanrikulu_2015e, Tay_2004, Thapa_2019f, Tupesis_2004f, Wildhaber_2005, Wnęk_2015f, Yasin_2009, Yildiz_2016e, fjbuilsRepeatedBehaviorDeliberate2024}, 31 cases (43\%) underwent endoscopy \cite{Akay_2015f, Ali_2022g, Apikotoa_2022f, Atayan_2016, Benoist_2019e, Berry_2021e, Bhasin_2014, Bhumi_2024f, CamachoDorado_2018, Chang_2017f, DelgadoSalazar_2020c, Gardner_2017h, Guinan_2019f, Hardy_2023g, Jehangir_2019h, Kariholu_2008, Li_2013, Liu_2005, Ohno_2005, Peixoto_2017f, Qureshi_2016, Riva_2018j, Sakellaridis_2008f, Sultan_2024f, Tammana_2012j, Tanrikulu_2015e, Trgo_2012f, Wadhwa_2015e, Wnęk_2015f, teWildt_2010}, 7 cases (10\%) were managed conservatively \cite{Ataya_2013, Bhattacharjee_2008, DivsalarP._2023a, Emamhadi_2018, Goldman_1998f, Kar_2015, Kumar_2001}, 2 cases (3\%) died \cite{Emamhadi_2018, Kumar_2001}. All 90 were male gender. 90 cases (100\%) were detained at the time of ingestion \cite{Elghali_2016, Karp_1991b, Lee_2007}, 88 cases (98\%) were intentional ingestions \cite{Elghali_2016, Karp_1991b, Lee_2007}, 30 cases (33\%) had a psychiatric history documented \cite{Elghali_2016, Karp_1991b, Lee_2007}, 2 cases (2\%) had a history of prior ingestion \cite{Elghali_2016}. No cases were reported for were psychiatric inpatients, were displaced people, were under the influence of alcohol at the time of ingestion, and had a severe disability history.
\paragraph*{Motivation}  70 cases (78\%) reported protest motivation \cite{Elghali_2016, Karp_1991b, Lee_2007}, 12 cases (13\%) reported psychiatric motivation \cite{Karp_1991b}, 6 cases (7\%) reported self-harm motivation \cite{Elghali_2016, Karp_1991b}. No cases were reported for psychosocial motivation and other motivation.
\paragraph*{Object Characteristics}  68 cases (76\%) involved sharp object ingestion \cite{Elghali_2016, Karp_1991b, Lee_2007}, 32 cases (36\%) involved long (\textgreater 5cm) object ingestion \cite{Lee_2007}, 25 cases (28\%) involved ingestion of multiple objects \cite{Elghali_2016, Lee_2007}. No cases were reported for button battery ingestion, magnet ingestion, and involved large diameter (\textgreater 2.5cm) object ingestion.
\paragraph*{Outcomes}  47 cases (52\%) underwent endoscopic intervention \cite{Elghali_2016, Lee_2007}, 29 cases (32\%) were managed conservatively \cite{Elghali_2016, Karp_1991b}, 15 cases (17\%) underwent surgical intervention \cite{Elghali_2016, Karp_1991b, Lee_2007}, 6 cases (7\%) reported complications \cite{Lee_2007}, 1 case (1\%) died \cite{Elghali_2016}.
\paragraph*{Geographical Location}Cases were recorded in 33 countries: 13 cases from USA \cite{Alao_2006i, Ataya_2013, Bhumi_2024f, Fry_2010, Guinan_2019f, Hardy_2023g, Jehangir_2019h, Kerestes_2019, Kumar_2001, Liu_2005, Tammana_2012j, Tay_2004, Tupesis_2004f}; 7 cases from India \cite{Bhasin_2014, Bhattacharjee_2008, Kar_2015, Kariholu_2008, Kumar_2019f, Misra_2013, Wadhwa_2015e} and UK \cite{Beecroft_1998, Berry_2021e, Cauchi_2002, Cox_2007, Gardner_2017h, Qureshi_2016}; 6 cases from Bulgaria \cite{Losanoff_1996, Losanoff_1997e}; 5 cases from Iran \cite{DivsalarP._2023a, Emamhadi_2018, Farhadi_2024h}; 4 cases from Turkey \cite{Akay_2015f, Atayan_2016, Tanrikulu_2015e, Yildiz_2016e}; 2 cases from China \cite{Jin_2023, Li_2013}, Poland \cite{Kobiela_2015, Wnęk_2015f}, and Spain \cite{CamachoDorado_2018, fjbuilsRepeatedBehaviorDeliberate2024}; 1 case from Australia \cite{Apikotoa_2022f}, Bahrain \cite{Ali_2020f}, Croatia \cite{Trgo_2012f}, Ecuador \cite{DelgadoSalazar_2020c}, Egypt \cite{Ali_2022g}, Ethiopia \cite{Mesfin_2022a}, Germany \cite{teWildt_2010}, Greece \cite{Sakellaridis_2008f}, Hungary \cite{Csaky_1998e}, Iraq \cite{Al-Faham_2020k}, Israel \cite{Goldman_1998f}, Italy \cite{Riva_2018j}, Japan \cite{Ohno_2005}, Nepal \cite{Thapa_2019f}, Netherlands \cite{Benoist_2019e}, Oman \cite{AlShaaibi_2021b}, Pakistan \cite{Yasin_2009}, Portugal \cite{Peixoto_2017f}, Qatar \cite{Ali_2017}, Saudi Arabia \cite{Sultan_2024f}, South Africa \cite{Sobnach_2011f}, Sweden \cite{Naji_2012f}, Switzerland \cite{Wildhaber_2005}, and Taiwan \cite{Chang_2017f}. \paragraph*{Gender} 43 cases (60\%) were male \cite{Akay_2015f, Al-Faham_2020k, Alao_2006i, Ali_2017, Ali_2022g, Apikotoa_2022f, Atayan_2016, Benoist_2019e, Berry_2021e, Bhumi_2024f, CamachoDorado_2018, Csaky_1998e, Emamhadi_2018, Farhadi_2024h, Fry_2010, Gardner_2017h, Guinan_2019f, Jehangir_2019h, Jin_2023, Kobiela_2015, Kumar_2001, Kumar_2019f, Liu_2005, Losanoff_1996, Losanoff_1997e, Mesfin_2022a, Misra_2013, Qureshi_2016, Riva_2018j, Sobnach_2011f, Tammana_2012j, Tanrikulu_2015e, Tay_2004, Thapa_2019f, Trgo_2012f, Wadhwa_2015e, Yasin_2009, teWildt_2010}, 28 cases (39\%) were female \cite{AlShaaibi_2021b, Ali_2020f, Ataya_2013, Beecroft_1998, Bhasin_2014, Bhattacharjee_2008, Cauchi_2002, Chang_2017f, Cox_2007, DelgadoSalazar_2020c, DivsalarP._2023a, Goldman_1998f, Hardy_2023g, Kar_2015, Kariholu_2008, Kerestes_2019, Li_2013, Naji_2012f, Ohno_2005, Peixoto_2017f, Sakellaridis_2008f, Sultan_2024f, Tupesis_2004f, Wildhaber_2005, Wnęk_2015f, Yildiz_2016e}, 1 case (1\%) had no gender recorded \cite{fjbuilsRepeatedBehaviorDeliberate2024}. \paragraph*{Age Group} 25 cases (35\%) were between 26 and 40 years of age \cite{Alao_2006i, Ali_2022g, Apikotoa_2022f, Ataya_2013, Benoist_2019e, Bhasin_2014, Chang_2017f, Cox_2007, DelgadoSalazar_2020c, Farhadi_2024h, Fry_2010, Gardner_2017h, Guinan_2019f, Jin_2023, Kumar_2019f, Losanoff_1996, Misra_2013, Qureshi_2016, Riva_2018j, Sakellaridis_2008f, Tammana_2012j, Trgo_2012f, Wnęk_2015f, Yildiz_2016e, fjbuilsRepeatedBehaviorDeliberate2024}, 18 cases (25\%) were between 18 and 25 years of age \cite{Akay_2015f, Ali_2017, Atayan_2016, Bhattacharjee_2008, Csaky_1998e, Kar_2015, Kariholu_2008, Kobiela_2015, Losanoff_1996, Losanoff_1997e, Mesfin_2022a, Peixoto_2017f, Sobnach_2011f, Tupesis_2004f, Yasin_2009}, 13 cases (18\%) were under 18 years of age \cite{AlShaaibi_2021b, Ali_2020f, Cauchi_2002, DivsalarP._2023a, Goldman_1998f, Liu_2005, Naji_2012f, Ohno_2005, Tanrikulu_2015e, Tay_2004, Wildhaber_2005}, 11 cases (15\%) were between 41 and 60 years of age \cite{Al-Faham_2020k, Bhumi_2024f, CamachoDorado_2018, Emamhadi_2018, Hardy_2023g, Jehangir_2019h, Kumar_2001, Sultan_2024f, Thapa_2019f, Wadhwa_2015e, teWildt_2010}, 3 cases (4\%) were over 60 years of age \cite{Beecroft_1998, Kerestes_2019, Li_2013}, 2 cases (3\%) had no age documented \cite{Berry_2021e}. \paragraph*{Population} 36 cases (50\%) had a psychiatric history \cite{AlShaaibi_2021b, Alao_2006i, Ali_2020f, Apikotoa_2022f, Ataya_2013, Atayan_2016, Beecroft_1998, CamachoDorado_2018, Chang_2017f, DelgadoSalazar_2020c, DivsalarP._2023a, Farhadi_2024h, Fry_2010, Guinan_2019f, Hardy_2023g, Jehangir_2019h, Jin_2023, Kar_2015, Kerestes_2019, Kobiela_2015, Kumar_2001, Kumar_2019f, Liu_2005, Mesfin_2022a, Misra_2013, Ohno_2005, Peixoto_2017f, Sakellaridis_2008f, Sultan_2024f, Tammana_2012j, Tanrikulu_2015e, Yildiz_2016e, fjbuilsRepeatedBehaviorDeliberate2024, teWildt_2010}, 19 cases (26\%) had ingested previously \cite{Alao_2006i, Apikotoa_2022f, Berry_2021e, Bhattacharjee_2008, Csaky_1998e, DivsalarP._2023a, Emamhadi_2018, Guinan_2019f, Jehangir_2019h, Jin_2023, Liu_2005, Sakellaridis_2008f, Tanrikulu_2015e, Thapa_2019f, Yildiz_2016e, fjbuilsRepeatedBehaviorDeliberate2024, teWildt_2010}, 12 cases (17\%) were detained persons \cite{Alao_2006i, Ali_2022g, Apikotoa_2022f, Losanoff_1996, Losanoff_1997e, Qureshi_2016, Tammana_2012j, Trgo_2012f}, 7 cases (10\%) were severely disabled \cite{Atayan_2016, Kerestes_2019, Liu_2005, Ohno_2005, Peixoto_2017f, Yildiz_2016e, teWildt_2010}, 4 cases (6\%) were psychiatric inpatients \cite{DivsalarP._2023a, fjbuilsRepeatedBehaviorDeliberate2024, teWildt_2010}, 3 cases (4\%) were under the influence of alcohol \cite{Benoist_2019e, Csaky_1998e, Thapa_2019f}, 2 cases (3\%) were displaced people \cite{Akay_2015f, Gardner_2017h}. \paragraph*{Motivation} 34 cases (47\%) had a psychiatric motivation \cite{Al-Faham_2020k, Alao_2006i, Ali_2020f, Apikotoa_2022f, Ataya_2013, Atayan_2016, Bhasin_2014, Bhattacharjee_2008, DelgadoSalazar_2020c, DivsalarP._2023a, Emamhadi_2018, Farhadi_2024h, Guinan_2019f, Hardy_2023g, Jehangir_2019h, Jin_2023, Kar_2015, Kariholu_2008, Kerestes_2019, Kobiela_2015, Kumar_2001, Kumar_2019f, Li_2013, Liu_2005, Misra_2013, Ohno_2005, Sakellaridis_2008f, Sultan_2024f, Tammana_2012j, Tanrikulu_2015e, Yasin_2009, teWildt_2010}, 21 cases (29\%) were motivated by self-harm intention \cite{Al-Faham_2020k, AlShaaibi_2021b, Alao_2006i, Ali_2017, CamachoDorado_2018, Chang_2017f, Cox_2007, Csaky_1998e, Fry_2010, Li_2013, Losanoff_1996, Losanoff_1997e, Mesfin_2022a, Sakellaridis_2008f, Tammana_2012j, Tanrikulu_2015e, fjbuilsRepeatedBehaviorDeliberate2024}, 17 cases (24\%) had a psychosocial motivation \cite{Akay_2015f, Benoist_2019e, Bhattacharjee_2008, Cauchi_2002, Goldman_1998f, Hardy_2023g, Kobiela_2015, Li_2013, Naji_2012f, Qureshi_2016, Riva_2018j, Sobnach_2011f, Tay_2004, Thapa_2019f, Tupesis_2004f, Wildhaber_2005, Wnęk_2015f}, 9 cases (12\%) were motivated by protest \cite{Bhumi_2024f, Gardner_2017h, Losanoff_1996, Losanoff_1997e, Tupesis_2004f}, 9 cases (12\%) had another documented motivation \cite{Ali_2020f, Ali_2022g, Emamhadi_2018, Guinan_2019f, Peixoto_2017f, Sakellaridis_2008f, Trgo_2012f, Wadhwa_2015e, Yildiz_2016e}. \paragraph*{Object Characteristics} 51 cases (71\%) ingested a large diameter object (\textgreater{}2.5cm) \cite{Akay_2015f, Al-Faham_2020k, AlShaaibi_2021b, Alao_2006i, Ali_2017, Ali_2022g, Apikotoa_2022f, Atayan_2016, Berry_2021e, Bhasin_2014, CamachoDorado_2018, Cauchi_2002, Chang_2017f, Cox_2007, Csaky_1998e, DivsalarP._2023a, Emamhadi_2018, Gardner_2017h, Guinan_2019f, Jehangir_2019h, Jin_2023, Kariholu_2008, Kerestes_2019, Kobiela_2015, Kumar_2001, Kumar_2019f, Losanoff_1996, Losanoff_1997e, Mesfin_2022a, Misra_2013, Naji_2012f, Ohno_2005, Peixoto_2017f, Qureshi_2016, Riva_2018j, Sakellaridis_2008f, Sultan_2024f, Tanrikulu_2015e, Thapa_2019f, Trgo_2012f, Wnęk_2015f, Yildiz_2016e, fjbuilsRepeatedBehaviorDeliberate2024, teWildt_2010}, 44 cases (61\%) ingested multiple objects \cite{Ali_2020f, Apikotoa_2022f, Ataya_2013, Atayan_2016, Beecroft_1998, Bhattacharjee_2008, Bhumi_2024f, CamachoDorado_2018, Cauchi_2002, Emamhadi_2018, Farhadi_2024h, Fry_2010, Goldman_1998f, Guinan_2019f, Hardy_2023g, Jehangir_2019h, Jin_2023, Kar_2015, Kariholu_2008, Kobiela_2015, Kumar_2001, Kumar_2019f, Li_2013, Liu_2005, Losanoff_1996, Mesfin_2022a, Misra_2013, Naji_2012f, Ohno_2005, Sobnach_2011f, Sultan_2024f, Tammana_2012j, Tanrikulu_2015e, Tay_2004, Thapa_2019f, Wadhwa_2015e, Wildhaber_2005, Yasin_2009, fjbuilsRepeatedBehaviorDeliberate2024, teWildt_2010}, 34 cases (47\%) ingested a sharp object \cite{AlShaaibi_2021b, Alao_2006i, Apikotoa_2022f, Ataya_2013, Benoist_2019e, Bhasin_2014, Bhattacharjee_2008, CamachoDorado_2018, Csaky_1998e, DelgadoSalazar_2020c, DivsalarP._2023a, Emamhadi_2018, Farhadi_2024h, Fry_2010, Guinan_2019f, Hardy_2023g, Jehangir_2019h, Jin_2023, Kariholu_2008, Kobiela_2015, Kumar_2019f, Losanoff_1996, Losanoff_1997e, Mesfin_2022a, Misra_2013, Sobnach_2011f, Yasin_2009, teWildt_2010}, 32 cases (44\%) ingested a long object (\textgreater{}5cm) \cite{Al-Faham_2020k, AlShaaibi_2021b, Ali_2017, Ali_2022g, Atayan_2016, Bhasin_2014, CamachoDorado_2018, Chang_2017f, Cox_2007, Csaky_1998e, DivsalarP._2023a, Emamhadi_2018, Fry_2010, Gardner_2017h, Jin_2023, Kariholu_2008, Kerestes_2019, Kobiela_2015, Kumar_2019f, Mesfin_2022a, Misra_2013, Ohno_2005, Qureshi_2016, Sakellaridis_2008f, Sultan_2024f, Thapa_2019f, Trgo_2012f, Yasin_2009, Yildiz_2016e, teWildt_2010}, 9 cases (12\%) ingested a magnet \cite{Ali_2020f, Bhumi_2024f, Cauchi_2002, Liu_2005, Naji_2012f, Ohno_2005, Tanrikulu_2015e, Tay_2004, Wildhaber_2005}, 2 cases (3\%) ingested a button battery \cite{Berry_2021e, Bhumi_2024f}. \paragraph*{Outcomes} 48 cases (67\%) experienced a complication \cite{Ali_2017, Ali_2020f, Apikotoa_2022f, Atayan_2016, Beecroft_1998, Benoist_2019e, Berry_2021e, Bhasin_2014, Bhumi_2024f, CamachoDorado_2018, Cauchi_2002, Cox_2007, Csaky_1998e, DelgadoSalazar_2020c, DivsalarP._2023a, Emamhadi_2018, Farhadi_2024h, Fry_2010, Gardner_2017h, Goldman_1998f, Jin_2023, Kariholu_2008, Kerestes_2019, Kobiela_2015, Kumar_2001, Kumar_2019f, Liu_2005, Losanoff_1996, Mesfin_2022a, Misra_2013, Naji_2012f, Ohno_2005, Sakellaridis_2008f, Sobnach_2011f, Sultan_2024f, Tanrikulu_2015e, Tay_2004, Thapa_2019f, Trgo_2012f, Tupesis_2004f, Wildhaber_2005, Wnęk_2015f, Yasin_2009, Yildiz_2016e}, 44 cases (61\%) underwent surgery \cite{Al-Faham_2020k, AlShaaibi_2021b, Alao_2006i, Ali_2017, Ali_2020f, Atayan_2016, Beecroft_1998, Bhasin_2014, CamachoDorado_2018, Cauchi_2002, Chang_2017f, Cox_2007, Csaky_1998e, DelgadoSalazar_2020c, DivsalarP._2023a, Farhadi_2024h, Fry_2010, Gardner_2017h, Jin_2023, Kariholu_2008, Kerestes_2019, Kobiela_2015, Kumar_2019f, Liu_2005, Losanoff_1996, Losanoff_1997e, Mesfin_2022a, Misra_2013, Naji_2012f, Sobnach_2011f, Tanrikulu_2015e, Tay_2004, Thapa_2019f, Tupesis_2004f, Wildhaber_2005, Wnęk_2015f, Yasin_2009, Yildiz_2016e, fjbuilsRepeatedBehaviorDeliberate2024}, 31 cases (43\%) underwent endoscopy \cite{Akay_2015f, Ali_2022g, Apikotoa_2022f, Atayan_2016, Benoist_2019e, Berry_2021e, Bhasin_2014, Bhumi_2024f, CamachoDorado_2018, Chang_2017f, DelgadoSalazar_2020c, Gardner_2017h, Guinan_2019f, Hardy_2023g, Jehangir_2019h, Kariholu_2008, Li_2013, Liu_2005, Ohno_2005, Peixoto_2017f, Qureshi_2016, Riva_2018j, Sakellaridis_2008f, Sultan_2024f, Tammana_2012j, Tanrikulu_2015e, Trgo_2012f, Wadhwa_2015e, Wnęk_2015f, teWildt_2010}, 7 cases (10\%) were managed conservatively \cite{Ataya_2013, Bhattacharjee_2008, DivsalarP._2023a, Emamhadi_2018, Goldman_1998f, Kar_2015, Kumar_2001}, 2 cases (3\%) died \cite{Emamhadi_2018, Kumar_2001}. All 90 were male gender. 90 cases (100\%) were detained at the time of ingestion \cite{Elghali_2016, Karp_1991b, Lee_2007}, 88 cases (98\%) were intentional ingestions \cite{Elghali_2016, Karp_1991b, Lee_2007}, 30 cases (33\%) had a psychiatric history documented \cite{Elghali_2016, Karp_1991b, Lee_2007}, 2 cases (2\%) had a history of prior ingestion \cite{Elghali_2016}. No cases were reported for were psychiatric inpatients, were displaced people, were under the influence of alcohol at the time of ingestion, and had a severe disability history.
\paragraph*{Motivation}  70 cases (78\%) reported protest motivation \cite{Elghali_2016, Karp_1991b, Lee_2007}, 12 cases (13\%) reported psychiatric motivation \cite{Karp_1991b}, 6 cases (7\%) reported self-harm motivation \cite{Elghali_2016, Karp_1991b}. No cases were reported for psychosocial motivation and other motivation.
\paragraph*{Object Characteristics}  68 cases (76\%) involved sharp object ingestion \cite{Elghali_2016, Karp_1991b, Lee_2007}, 32 cases (36\%) involved long (\textgreater 5cm) object ingestion \cite{Lee_2007}, 25 cases (28\%) involved ingestion of multiple objects \cite{Elghali_2016, Lee_2007}. No cases were reported for button battery ingestion, magnet ingestion, and involved large diameter (\textgreater 2.5cm) object ingestion.
\paragraph*{Outcomes}  47 cases (52\%) underwent endoscopic intervention \cite{Elghali_2016, Lee_2007}, 29 cases (32\%) were managed conservatively \cite{Elghali_2016, Karp_1991b}, 15 cases (17\%) underwent surgical intervention \cite{Elghali_2016, Karp_1991b, Lee_2007}, 6 cases (7\%) reported complications \cite{Lee_2007}, 1 case (1\%) died \cite{Elghali_2016}.
\paragraph*{Geographical Location}Cases were recorded in 33 countries: 13 cases from USA \cite{Alao_2006i, Ataya_2013, Bhumi_2024f, Fry_2010, Guinan_2019f, Hardy_2023g, Jehangir_2019h, Kerestes_2019, Kumar_2001, Liu_2005, Tammana_2012j, Tay_2004, Tupesis_2004f}; 7 cases from India \cite{Bhasin_2014, Bhattacharjee_2008, Kar_2015, Kariholu_2008, Kumar_2019f, Misra_2013, Wadhwa_2015e} and UK \cite{Beecroft_1998, Berry_2021e, Cauchi_2002, Cox_2007, Gardner_2017h, Qureshi_2016}; 6 cases from Bulgaria \cite{Losanoff_1996, Losanoff_1997e}; 5 cases from Iran \cite{DivsalarP._2023a, Emamhadi_2018, Farhadi_2024h}; 4 cases from Turkey \cite{Akay_2015f, Atayan_2016, Tanrikulu_2015e, Yildiz_2016e}; 2 cases from China \cite{Jin_2023, Li_2013}, Poland \cite{Kobiela_2015, Wnęk_2015f}, and Spain \cite{CamachoDorado_2018, fjbuilsRepeatedBehaviorDeliberate2024}; 1 case from Australia \cite{Apikotoa_2022f}, Bahrain \cite{Ali_2020f}, Croatia \cite{Trgo_2012f}, Ecuador \cite{DelgadoSalazar_2020c}, Egypt \cite{Ali_2022g}, Ethiopia \cite{Mesfin_2022a}, Germany \cite{teWildt_2010}, Greece \cite{Sakellaridis_2008f}, Hungary \cite{Csaky_1998e}, Iraq \cite{Al-Faham_2020k}, Israel \cite{Goldman_1998f}, Italy \cite{Riva_2018j}, Japan \cite{Ohno_2005}, Nepal \cite{Thapa_2019f}, Netherlands \cite{Benoist_2019e}, Oman \cite{AlShaaibi_2021b}, Pakistan \cite{Yasin_2009}, Portugal \cite{Peixoto_2017f}, Qatar \cite{Ali_2017}, Saudi Arabia \cite{Sultan_2024f}, South Africa \cite{Sobnach_2011f}, Sweden \cite{Naji_2012f}, Switzerland \cite{Wildhaber_2005}, and Taiwan \cite{Chang_2017f}. \paragraph*{Gender} 43 cases (60\%) were male \cite{Akay_2015f, Al-Faham_2020k, Alao_2006i, Ali_2017, Ali_2022g, Apikotoa_2022f, Atayan_2016, Benoist_2019e, Berry_2021e, Bhumi_2024f, CamachoDorado_2018, Csaky_1998e, Emamhadi_2018, Farhadi_2024h, Fry_2010, Gardner_2017h, Guinan_2019f, Jehangir_2019h, Jin_2023, Kobiela_2015, Kumar_2001, Kumar_2019f, Liu_2005, Losanoff_1996, Losanoff_1997e, Mesfin_2022a, Misra_2013, Qureshi_2016, Riva_2018j, Sobnach_2011f, Tammana_2012j, Tanrikulu_2015e, Tay_2004, Thapa_2019f, Trgo_2012f, Wadhwa_2015e, Yasin_2009, teWildt_2010}, 28 cases (39\%) were female \cite{AlShaaibi_2021b, Ali_2020f, Ataya_2013, Beecroft_1998, Bhasin_2014, Bhattacharjee_2008, Cauchi_2002, Chang_2017f, Cox_2007, DelgadoSalazar_2020c, DivsalarP._2023a, Goldman_1998f, Hardy_2023g, Kar_2015, Kariholu_2008, Kerestes_2019, Li_2013, Naji_2012f, Ohno_2005, Peixoto_2017f, Sakellaridis_2008f, Sultan_2024f, Tupesis_2004f, Wildhaber_2005, Wnęk_2015f, Yildiz_2016e}, 1 case (1\%) had no gender recorded \cite{fjbuilsRepeatedBehaviorDeliberate2024}. \paragraph*{Age Group} 25 cases (35\%) were between 26 and 40 years of age \cite{Alao_2006i, Ali_2022g, Apikotoa_2022f, Ataya_2013, Benoist_2019e, Bhasin_2014, Chang_2017f, Cox_2007, DelgadoSalazar_2020c, Farhadi_2024h, Fry_2010, Gardner_2017h, Guinan_2019f, Jin_2023, Kumar_2019f, Losanoff_1996, Misra_2013, Qureshi_2016, Riva_2018j, Sakellaridis_2008f, Tammana_2012j, Trgo_2012f, Wnęk_2015f, Yildiz_2016e, fjbuilsRepeatedBehaviorDeliberate2024}, 18 cases (25\%) were between 18 and 25 years of age \cite{Akay_2015f, Ali_2017, Atayan_2016, Bhattacharjee_2008, Csaky_1998e, Kar_2015, Kariholu_2008, Kobiela_2015, Losanoff_1996, Losanoff_1997e, Mesfin_2022a, Peixoto_2017f, Sobnach_2011f, Tupesis_2004f, Yasin_2009}, 13 cases (18\%) were under 18 years of age \cite{AlShaaibi_2021b, Ali_2020f, Cauchi_2002, DivsalarP._2023a, Goldman_1998f, Liu_2005, Naji_2012f, Ohno_2005, Tanrikulu_2015e, Tay_2004, Wildhaber_2005}, 11 cases (15\%) were between 41 and 60 years of age \cite{Al-Faham_2020k, Bhumi_2024f, CamachoDorado_2018, Emamhadi_2018, Hardy_2023g, Jehangir_2019h, Kumar_2001, Sultan_2024f, Thapa_2019f, Wadhwa_2015e, teWildt_2010}, 3 cases (4\%) were over 60 years of age \cite{Beecroft_1998, Kerestes_2019, Li_2013}, 2 cases (3\%) had no age documented \cite{Berry_2021e}. \paragraph*{Population} 36 cases (50\%) had a psychiatric history \cite{AlShaaibi_2021b, Alao_2006i, Ali_2020f, Apikotoa_2022f, Ataya_2013, Atayan_2016, Beecroft_1998, CamachoDorado_2018, Chang_2017f, DelgadoSalazar_2020c, DivsalarP._2023a, Farhadi_2024h, Fry_2010, Guinan_2019f, Hardy_2023g, Jehangir_2019h, Jin_2023, Kar_2015, Kerestes_2019, Kobiela_2015, Kumar_2001, Kumar_2019f, Liu_2005, Mesfin_2022a, Misra_2013, Ohno_2005, Peixoto_2017f, Sakellaridis_2008f, Sultan_2024f, Tammana_2012j, Tanrikulu_2015e, Yildiz_2016e, fjbuilsRepeatedBehaviorDeliberate2024, teWildt_2010}, 19 cases (26\%) had ingested previously \cite{Alao_2006i, Apikotoa_2022f, Berry_2021e, Bhattacharjee_2008, Csaky_1998e, DivsalarP._2023a, Emamhadi_2018, Guinan_2019f, Jehangir_2019h, Jin_2023, Liu_2005, Sakellaridis_2008f, Tanrikulu_2015e, Thapa_2019f, Yildiz_2016e, fjbuilsRepeatedBehaviorDeliberate2024, teWildt_2010}, 12 cases (17\%) were detained persons \cite{Alao_2006i, Ali_2022g, Apikotoa_2022f, Losanoff_1996, Losanoff_1997e, Qureshi_2016, Tammana_2012j, Trgo_2012f}, 7 cases (10\%) were severely disabled \cite{Atayan_2016, Kerestes_2019, Liu_2005, Ohno_2005, Peixoto_2017f, Yildiz_2016e, teWildt_2010}, 4 cases (6\%) were psychiatric inpatients \cite{DivsalarP._2023a, fjbuilsRepeatedBehaviorDeliberate2024, teWildt_2010}, 3 cases (4\%) were under the influence of alcohol \cite{Benoist_2019e, Csaky_1998e, Thapa_2019f}, 2 cases (3\%) were displaced people \cite{Akay_2015f, Gardner_2017h}. \paragraph*{Motivation} 34 cases (47\%) had a psychiatric motivation \cite{Al-Faham_2020k, Alao_2006i, Ali_2020f, Apikotoa_2022f, Ataya_2013, Atayan_2016, Bhasin_2014, Bhattacharjee_2008, DelgadoSalazar_2020c, DivsalarP._2023a, Emamhadi_2018, Farhadi_2024h, Guinan_2019f, Hardy_2023g, Jehangir_2019h, Jin_2023, Kar_2015, Kariholu_2008, Kerestes_2019, Kobiela_2015, Kumar_2001, Kumar_2019f, Li_2013, Liu_2005, Misra_2013, Ohno_2005, Sakellaridis_2008f, Sultan_2024f, Tammana_2012j, Tanrikulu_2015e, Yasin_2009, teWildt_2010}, 21 cases (29\%) were motivated by self-harm intention \cite{Al-Faham_2020k, AlShaaibi_2021b, Alao_2006i, Ali_2017, CamachoDorado_2018, Chang_2017f, Cox_2007, Csaky_1998e, Fry_2010, Li_2013, Losanoff_1996, Losanoff_1997e, Mesfin_2022a, Sakellaridis_2008f, Tammana_2012j, Tanrikulu_2015e, fjbuilsRepeatedBehaviorDeliberate2024}, 17 cases (24\%) had a psychosocial motivation \cite{Akay_2015f, Benoist_2019e, Bhattacharjee_2008, Cauchi_2002, Goldman_1998f, Hardy_2023g, Kobiela_2015, Li_2013, Naji_2012f, Qureshi_2016, Riva_2018j, Sobnach_2011f, Tay_2004, Thapa_2019f, Tupesis_2004f, Wildhaber_2005, Wnęk_2015f}, 9 cases (12\%) were motivated by protest \cite{Bhumi_2024f, Gardner_2017h, Losanoff_1996, Losanoff_1997e, Tupesis_2004f}, 9 cases (12\%) had another documented motivation \cite{Ali_2020f, Ali_2022g, Emamhadi_2018, Guinan_2019f, Peixoto_2017f, Sakellaridis_2008f, Trgo_2012f, Wadhwa_2015e, Yildiz_2016e}. \paragraph*{Object Characteristics} 51 cases (71\%) ingested a large diameter object (\textgreater{}2.5cm) \cite{Akay_2015f, Al-Faham_2020k, AlShaaibi_2021b, Alao_2006i, Ali_2017, Ali_2022g, Apikotoa_2022f, Atayan_2016, Berry_2021e, Bhasin_2014, CamachoDorado_2018, Cauchi_2002, Chang_2017f, Cox_2007, Csaky_1998e, DivsalarP._2023a, Emamhadi_2018, Gardner_2017h, Guinan_2019f, Jehangir_2019h, Jin_2023, Kariholu_2008, Kerestes_2019, Kobiela_2015, Kumar_2001, Kumar_2019f, Losanoff_1996, Losanoff_1997e, Mesfin_2022a, Misra_2013, Naji_2012f, Ohno_2005, Peixoto_2017f, Qureshi_2016, Riva_2018j, Sakellaridis_2008f, Sultan_2024f, Tanrikulu_2015e, Thapa_2019f, Trgo_2012f, Wnęk_2015f, Yildiz_2016e, fjbuilsRepeatedBehaviorDeliberate2024, teWildt_2010}, 44 cases (61\%) ingested multiple objects \cite{Ali_2020f, Apikotoa_2022f, Ataya_2013, Atayan_2016, Beecroft_1998, Bhattacharjee_2008, Bhumi_2024f, CamachoDorado_2018, Cauchi_2002, Emamhadi_2018, Farhadi_2024h, Fry_2010, Goldman_1998f, Guinan_2019f, Hardy_2023g, Jehangir_2019h, Jin_2023, Kar_2015, Kariholu_2008, Kobiela_2015, Kumar_2001, Kumar_2019f, Li_2013, Liu_2005, Losanoff_1996, Mesfin_2022a, Misra_2013, Naji_2012f, Ohno_2005, Sobnach_2011f, Sultan_2024f, Tammana_2012j, Tanrikulu_2015e, Tay_2004, Thapa_2019f, Wadhwa_2015e, Wildhaber_2005, Yasin_2009, fjbuilsRepeatedBehaviorDeliberate2024, teWildt_2010}, 34 cases (47\%) ingested a sharp object \cite{AlShaaibi_2021b, Alao_2006i, Apikotoa_2022f, Ataya_2013, Benoist_2019e, Bhasin_2014, Bhattacharjee_2008, CamachoDorado_2018, Csaky_1998e, DelgadoSalazar_2020c, DivsalarP._2023a, Emamhadi_2018, Farhadi_2024h, Fry_2010, Guinan_2019f, Hardy_2023g, Jehangir_2019h, Jin_2023, Kariholu_2008, Kobiela_2015, Kumar_2019f, Losanoff_1996, Losanoff_1997e, Mesfin_2022a, Misra_2013, Sobnach_2011f, Yasin_2009, teWildt_2010}, 32 cases (44\%) ingested a long object (\textgreater{}5cm) \cite{Al-Faham_2020k, AlShaaibi_2021b, Ali_2017, Ali_2022g, Atayan_2016, Bhasin_2014, CamachoDorado_2018, Chang_2017f, Cox_2007, Csaky_1998e, DivsalarP._2023a, Emamhadi_2018, Fry_2010, Gardner_2017h, Jin_2023, Kariholu_2008, Kerestes_2019, Kobiela_2015, Kumar_2019f, Mesfin_2022a, Misra_2013, Ohno_2005, Qureshi_2016, Sakellaridis_2008f, Sultan_2024f, Thapa_2019f, Trgo_2012f, Yasin_2009, Yildiz_2016e, teWildt_2010}, 9 cases (12\%) ingested a magnet \cite{Ali_2020f, Bhumi_2024f, Cauchi_2002, Liu_2005, Naji_2012f, Ohno_2005, Tanrikulu_2015e, Tay_2004, Wildhaber_2005}, 2 cases (3\%) ingested a button battery \cite{Berry_2021e, Bhumi_2024f}. \paragraph*{Outcomes} 48 cases (67\%) experienced a complication \cite{Ali_2017, Ali_2020f, Apikotoa_2022f, Atayan_2016, Beecroft_1998, Benoist_2019e, Berry_2021e, Bhasin_2014, Bhumi_2024f, CamachoDorado_2018, Cauchi_2002, Cox_2007, Csaky_1998e, DelgadoSalazar_2020c, DivsalarP._2023a, Emamhadi_2018, Farhadi_2024h, Fry_2010, Gardner_2017h, Goldman_1998f, Jin_2023, Kariholu_2008, Kerestes_2019, Kobiela_2015, Kumar_2001, Kumar_2019f, Liu_2005, Losanoff_1996, Mesfin_2022a, Misra_2013, Naji_2012f, Ohno_2005, Sakellaridis_2008f, Sobnach_2011f, Sultan_2024f, Tanrikulu_2015e, Tay_2004, Thapa_2019f, Trgo_2012f, Tupesis_2004f, Wildhaber_2005, Wnęk_2015f, Yasin_2009, Yildiz_2016e}, 44 cases (61\%) underwent surgery \cite{Al-Faham_2020k, AlShaaibi_2021b, Alao_2006i, Ali_2017, Ali_2020f, Atayan_2016, Beecroft_1998, Bhasin_2014, CamachoDorado_2018, Cauchi_2002, Chang_2017f, Cox_2007, Csaky_1998e, DelgadoSalazar_2020c, DivsalarP._2023a, Farhadi_2024h, Fry_2010, Gardner_2017h, Jin_2023, Kariholu_2008, Kerestes_2019, Kobiela_2015, Kumar_2019f, Liu_2005, Losanoff_1996, Losanoff_1997e, Mesfin_2022a, Misra_2013, Naji_2012f, Sobnach_2011f, Tanrikulu_2015e, Tay_2004, Thapa_2019f, Tupesis_2004f, Wildhaber_2005, Wnęk_2015f, Yasin_2009, Yildiz_2016e, fjbuilsRepeatedBehaviorDeliberate2024}, 31 cases (43\%) underwent endoscopy \cite{Akay_2015f, Ali_2022g, Apikotoa_2022f, Atayan_2016, Benoist_2019e, Berry_2021e, Bhasin_2014, Bhumi_2024f, CamachoDorado_2018, Chang_2017f, DelgadoSalazar_2020c, Gardner_2017h, Guinan_2019f, Hardy_2023g, Jehangir_2019h, Kariholu_2008, Li_2013, Liu_2005, Ohno_2005, Peixoto_2017f, Qureshi_2016, Riva_2018j, Sakellaridis_2008f, Sultan_2024f, Tammana_2012j, Tanrikulu_2015e, Trgo_2012f, Wadhwa_2015e, Wnęk_2015f, teWildt_2010}, 7 cases (10\%) were managed conservatively \cite{Ataya_2013, Bhattacharjee_2008, DivsalarP._2023a, Emamhadi_2018, Goldman_1998f, Kar_2015, Kumar_2001}, 2 cases (3\%) died \cite{Emamhadi_2018, Kumar_2001}. All 90 were male gender. 90 cases (100\%) were detained at the time of ingestion \cite{Elghali_2016, Karp_1991b, Lee_2007}, 88 cases (98\%) were intentional ingestions \cite{Elghali_2016, Karp_1991b, Lee_2007}, 30 cases (33\%) had a psychiatric history documented \cite{Elghali_2016, Karp_1991b, Lee_2007}, 2 cases (2\%) had a history of prior ingestion \cite{Elghali_2016}. No cases were reported for were psychiatric inpatients, were displaced people, were under the influence of alcohol at the time of ingestion, and had a severe disability history.
\paragraph*{Motivation}  70 cases (78\%) reported protest motivation \cite{Elghali_2016, Karp_1991b, Lee_2007}, 12 cases (13\%) reported psychiatric motivation \cite{Karp_1991b}, 6 cases (7\%) reported self-harm motivation \cite{Elghali_2016, Karp_1991b}. No cases were reported for psychosocial motivation and other motivation.
\paragraph*{Object Characteristics}  68 cases (76\%) involved sharp object ingestion \cite{Elghali_2016, Karp_1991b, Lee_2007}, 32 cases (36\%) involved long (\textgreater 5cm) object ingestion \cite{Lee_2007}, 25 cases (28\%) involved ingestion of multiple objects \cite{Elghali_2016, Lee_2007}. No cases were reported for button battery ingestion, magnet ingestion, and involved large diameter (\textgreater 2.5cm) object ingestion.
\paragraph*{Outcomes}  47 cases (52\%) underwent endoscopic intervention \cite{Elghali_2016, Lee_2007}, 29 cases (32\%) were managed conservatively \cite{Elghali_2016, Karp_1991b}, 15 cases (17\%) underwent surgical intervention \cite{Elghali_2016, Karp_1991b, Lee_2007}, 6 cases (7\%) reported complications \cite{Lee_2007}, 1 case (1\%) died \cite{Elghali_2016}.
\paragraph*{Geographical Location}Cases were recorded in 33 countries: 13 cases from USA \cite{Alao_2006i, Ataya_2013, Bhumi_2024f, Fry_2010, Guinan_2019f, Hardy_2023g, Jehangir_2019h, Kerestes_2019, Kumar_2001, Liu_2005, Tammana_2012j, Tay_2004, Tupesis_2004f}; 7 cases from India \cite{Bhasin_2014, Bhattacharjee_2008, Kar_2015, Kariholu_2008, Kumar_2019f, Misra_2013, Wadhwa_2015e} and UK \cite{Beecroft_1998, Berry_2021e, Cauchi_2002, Cox_2007, Gardner_2017h, Qureshi_2016}; 6 cases from Bulgaria \cite{Losanoff_1996, Losanoff_1997e}; 5 cases from Iran \cite{DivsalarP._2023a, Emamhadi_2018, Farhadi_2024h}; 4 cases from Turkey \cite{Akay_2015f, Atayan_2016, Tanrikulu_2015e, Yildiz_2016e}; 2 cases from China \cite{Jin_2023, Li_2013}, Poland \cite{Kobiela_2015, Wnęk_2015f}, and Spain \cite{CamachoDorado_2018, fjbuilsRepeatedBehaviorDeliberate2024}; 1 case from Australia \cite{Apikotoa_2022f}, Bahrain \cite{Ali_2020f}, Croatia \cite{Trgo_2012f}, Ecuador \cite{DelgadoSalazar_2020c}, Egypt \cite{Ali_2022g}, Ethiopia \cite{Mesfin_2022a}, Germany \cite{teWildt_2010}, Greece \cite{Sakellaridis_2008f}, Hungary \cite{Csaky_1998e}, Iraq \cite{Al-Faham_2020k}, Israel \cite{Goldman_1998f}, Italy \cite{Riva_2018j}, Japan \cite{Ohno_2005}, Nepal \cite{Thapa_2019f}, Netherlands \cite{Benoist_2019e}, Oman \cite{AlShaaibi_2021b}, Pakistan \cite{Yasin_2009}, Portugal \cite{Peixoto_2017f}, Qatar \cite{Ali_2017}, Saudi Arabia \cite{Sultan_2024f}, South Africa \cite{Sobnach_2011f}, Sweden \cite{Naji_2012f}, Switzerland \cite{Wildhaber_2005}, and Taiwan \cite{Chang_2017f}. \paragraph*{Gender} 43 cases (60\%) were male \cite{Akay_2015f, Al-Faham_2020k, Alao_2006i, Ali_2017, Ali_2022g, Apikotoa_2022f, Atayan_2016, Benoist_2019e, Berry_2021e, Bhumi_2024f, CamachoDorado_2018, Csaky_1998e, Emamhadi_2018, Farhadi_2024h, Fry_2010, Gardner_2017h, Guinan_2019f, Jehangir_2019h, Jin_2023, Kobiela_2015, Kumar_2001, Kumar_2019f, Liu_2005, Losanoff_1996, Losanoff_1997e, Mesfin_2022a, Misra_2013, Qureshi_2016, Riva_2018j, Sobnach_2011f, Tammana_2012j, Tanrikulu_2015e, Tay_2004, Thapa_2019f, Trgo_2012f, Wadhwa_2015e, Yasin_2009, teWildt_2010}, 28 cases (39\%) were female \cite{AlShaaibi_2021b, Ali_2020f, Ataya_2013, Beecroft_1998, Bhasin_2014, Bhattacharjee_2008, Cauchi_2002, Chang_2017f, Cox_2007, DelgadoSalazar_2020c, DivsalarP._2023a, Goldman_1998f, Hardy_2023g, Kar_2015, Kariholu_2008, Kerestes_2019, Li_2013, Naji_2012f, Ohno_2005, Peixoto_2017f, Sakellaridis_2008f, Sultan_2024f, Tupesis_2004f, Wildhaber_2005, Wnęk_2015f, Yildiz_2016e}, 1 case (1\%) had no gender recorded \cite{fjbuilsRepeatedBehaviorDeliberate2024}. \paragraph*{Age Group} 25 cases (35\%) were between 26 and 40 years of age \cite{Alao_2006i, Ali_2022g, Apikotoa_2022f, Ataya_2013, Benoist_2019e, Bhasin_2014, Chang_2017f, Cox_2007, DelgadoSalazar_2020c, Farhadi_2024h, Fry_2010, Gardner_2017h, Guinan_2019f, Jin_2023, Kumar_2019f, Losanoff_1996, Misra_2013, Qureshi_2016, Riva_2018j, Sakellaridis_2008f, Tammana_2012j, Trgo_2012f, Wnęk_2015f, Yildiz_2016e, fjbuilsRepeatedBehaviorDeliberate2024}, 18 cases (25\%) were between 18 and 25 years of age \cite{Akay_2015f, Ali_2017, Atayan_2016, Bhattacharjee_2008, Csaky_1998e, Kar_2015, Kariholu_2008, Kobiela_2015, Losanoff_1996, Losanoff_1997e, Mesfin_2022a, Peixoto_2017f, Sobnach_2011f, Tupesis_2004f, Yasin_2009}, 13 cases (18\%) were under 18 years of age \cite{AlShaaibi_2021b, Ali_2020f, Cauchi_2002, DivsalarP._2023a, Goldman_1998f, Liu_2005, Naji_2012f, Ohno_2005, Tanrikulu_2015e, Tay_2004, Wildhaber_2005}, 11 cases (15\%) were between 41 and 60 years of age \cite{Al-Faham_2020k, Bhumi_2024f, CamachoDorado_2018, Emamhadi_2018, Hardy_2023g, Jehangir_2019h, Kumar_2001, Sultan_2024f, Thapa_2019f, Wadhwa_2015e, teWildt_2010}, 3 cases (4\%) were over 60 years of age \cite{Beecroft_1998, Kerestes_2019, Li_2013}, 2 cases (3\%) had no age documented \cite{Berry_2021e}. \paragraph*{Population} 36 cases (50\%) had a psychiatric history \cite{AlShaaibi_2021b, Alao_2006i, Ali_2020f, Apikotoa_2022f, Ataya_2013, Atayan_2016, Beecroft_1998, CamachoDorado_2018, Chang_2017f, DelgadoSalazar_2020c, DivsalarP._2023a, Farhadi_2024h, Fry_2010, Guinan_2019f, Hardy_2023g, Jehangir_2019h, Jin_2023, Kar_2015, Kerestes_2019, Kobiela_2015, Kumar_2001, Kumar_2019f, Liu_2005, Mesfin_2022a, Misra_2013, Ohno_2005, Peixoto_2017f, Sakellaridis_2008f, Sultan_2024f, Tammana_2012j, Tanrikulu_2015e, Yildiz_2016e, fjbuilsRepeatedBehaviorDeliberate2024, teWildt_2010}, 19 cases (26\%) had ingested previously \cite{Alao_2006i, Apikotoa_2022f, Berry_2021e, Bhattacharjee_2008, Csaky_1998e, DivsalarP._2023a, Emamhadi_2018, Guinan_2019f, Jehangir_2019h, Jin_2023, Liu_2005, Sakellaridis_2008f, Tanrikulu_2015e, Thapa_2019f, Yildiz_2016e, fjbuilsRepeatedBehaviorDeliberate2024, teWildt_2010}, 12 cases (17\%) were detained persons \cite{Alao_2006i, Ali_2022g, Apikotoa_2022f, Losanoff_1996, Losanoff_1997e, Qureshi_2016, Tammana_2012j, Trgo_2012f}, 7 cases (10\%) were severely disabled \cite{Atayan_2016, Kerestes_2019, Liu_2005, Ohno_2005, Peixoto_2017f, Yildiz_2016e, teWildt_2010}, 4 cases (6\%) were psychiatric inpatients \cite{DivsalarP._2023a, fjbuilsRepeatedBehaviorDeliberate2024, teWildt_2010}, 3 cases (4\%) were under the influence of alcohol \cite{Benoist_2019e, Csaky_1998e, Thapa_2019f}, 2 cases (3\%) were displaced people \cite{Akay_2015f, Gardner_2017h}. \paragraph*{Motivation} 34 cases (47\%) had a psychiatric motivation \cite{Al-Faham_2020k, Alao_2006i, Ali_2020f, Apikotoa_2022f, Ataya_2013, Atayan_2016, Bhasin_2014, Bhattacharjee_2008, DelgadoSalazar_2020c, DivsalarP._2023a, Emamhadi_2018, Farhadi_2024h, Guinan_2019f, Hardy_2023g, Jehangir_2019h, Jin_2023, Kar_2015, Kariholu_2008, Kerestes_2019, Kobiela_2015, Kumar_2001, Kumar_2019f, Li_2013, Liu_2005, Misra_2013, Ohno_2005, Sakellaridis_2008f, Sultan_2024f, Tammana_2012j, Tanrikulu_2015e, Yasin_2009, teWildt_2010}, 21 cases (29\%) were motivated by self-harm intention \cite{Al-Faham_2020k, AlShaaibi_2021b, Alao_2006i, Ali_2017, CamachoDorado_2018, Chang_2017f, Cox_2007, Csaky_1998e, Fry_2010, Li_2013, Losanoff_1996, Losanoff_1997e, Mesfin_2022a, Sakellaridis_2008f, Tammana_2012j, Tanrikulu_2015e, fjbuilsRepeatedBehaviorDeliberate2024}, 17 cases (24\%) had a psychosocial motivation \cite{Akay_2015f, Benoist_2019e, Bhattacharjee_2008, Cauchi_2002, Goldman_1998f, Hardy_2023g, Kobiela_2015, Li_2013, Naji_2012f, Qureshi_2016, Riva_2018j, Sobnach_2011f, Tay_2004, Thapa_2019f, Tupesis_2004f, Wildhaber_2005, Wnęk_2015f}, 9 cases (12\%) were motivated by protest \cite{Bhumi_2024f, Gardner_2017h, Losanoff_1996, Losanoff_1997e, Tupesis_2004f}, 9 cases (12\%) had another documented motivation \cite{Ali_2020f, Ali_2022g, Emamhadi_2018, Guinan_2019f, Peixoto_2017f, Sakellaridis_2008f, Trgo_2012f, Wadhwa_2015e, Yildiz_2016e}. \paragraph*{Object Characteristics} 51 cases (71\%) ingested a large diameter object (\textgreater{}2.5cm) \cite{Akay_2015f, Al-Faham_2020k, AlShaaibi_2021b, Alao_2006i, Ali_2017, Ali_2022g, Apikotoa_2022f, Atayan_2016, Berry_2021e, Bhasin_2014, CamachoDorado_2018, Cauchi_2002, Chang_2017f, Cox_2007, Csaky_1998e, DivsalarP._2023a, Emamhadi_2018, Gardner_2017h, Guinan_2019f, Jehangir_2019h, Jin_2023, Kariholu_2008, Kerestes_2019, Kobiela_2015, Kumar_2001, Kumar_2019f, Losanoff_1996, Losanoff_1997e, Mesfin_2022a, Misra_2013, Naji_2012f, Ohno_2005, Peixoto_2017f, Qureshi_2016, Riva_2018j, Sakellaridis_2008f, Sultan_2024f, Tanrikulu_2015e, Thapa_2019f, Trgo_2012f, Wnęk_2015f, Yildiz_2016e, fjbuilsRepeatedBehaviorDeliberate2024, teWildt_2010}, 44 cases (61\%) ingested multiple objects \cite{Ali_2020f, Apikotoa_2022f, Ataya_2013, Atayan_2016, Beecroft_1998, Bhattacharjee_2008, Bhumi_2024f, CamachoDorado_2018, Cauchi_2002, Emamhadi_2018, Farhadi_2024h, Fry_2010, Goldman_1998f, Guinan_2019f, Hardy_2023g, Jehangir_2019h, Jin_2023, Kar_2015, Kariholu_2008, Kobiela_2015, Kumar_2001, Kumar_2019f, Li_2013, Liu_2005, Losanoff_1996, Mesfin_2022a, Misra_2013, Naji_2012f, Ohno_2005, Sobnach_2011f, Sultan_2024f, Tammana_2012j, Tanrikulu_2015e, Tay_2004, Thapa_2019f, Wadhwa_2015e, Wildhaber_2005, Yasin_2009, fjbuilsRepeatedBehaviorDeliberate2024, teWildt_2010}, 34 cases (47\%) ingested a sharp object \cite{AlShaaibi_2021b, Alao_2006i, Apikotoa_2022f, Ataya_2013, Benoist_2019e, Bhasin_2014, Bhattacharjee_2008, CamachoDorado_2018, Csaky_1998e, DelgadoSalazar_2020c, DivsalarP._2023a, Emamhadi_2018, Farhadi_2024h, Fry_2010, Guinan_2019f, Hardy_2023g, Jehangir_2019h, Jin_2023, Kariholu_2008, Kobiela_2015, Kumar_2019f, Losanoff_1996, Losanoff_1997e, Mesfin_2022a, Misra_2013, Sobnach_2011f, Yasin_2009, teWildt_2010}, 32 cases (44\%) ingested a long object (\textgreater{}5cm) \cite{Al-Faham_2020k, AlShaaibi_2021b, Ali_2017, Ali_2022g, Atayan_2016, Bhasin_2014, CamachoDorado_2018, Chang_2017f, Cox_2007, Csaky_1998e, DivsalarP._2023a, Emamhadi_2018, Fry_2010, Gardner_2017h, Jin_2023, Kariholu_2008, Kerestes_2019, Kobiela_2015, Kumar_2019f, Mesfin_2022a, Misra_2013, Ohno_2005, Qureshi_2016, Sakellaridis_2008f, Sultan_2024f, Thapa_2019f, Trgo_2012f, Yasin_2009, Yildiz_2016e, teWildt_2010}, 9 cases (12\%) ingested a magnet \cite{Ali_2020f, Bhumi_2024f, Cauchi_2002, Liu_2005, Naji_2012f, Ohno_2005, Tanrikulu_2015e, Tay_2004, Wildhaber_2005}, 2 cases (3\%) ingested a button battery \cite{Berry_2021e, Bhumi_2024f}. \paragraph*{Outcomes} 48 cases (67\%) experienced a complication \cite{Ali_2017, Ali_2020f, Apikotoa_2022f, Atayan_2016, Beecroft_1998, Benoist_2019e, Berry_2021e, Bhasin_2014, Bhumi_2024f, CamachoDorado_2018, Cauchi_2002, Cox_2007, Csaky_1998e, DelgadoSalazar_2020c, DivsalarP._2023a, Emamhadi_2018, Farhadi_2024h, Fry_2010, Gardner_2017h, Goldman_1998f, Jin_2023, Kariholu_2008, Kerestes_2019, Kobiela_2015, Kumar_2001, Kumar_2019f, Liu_2005, Losanoff_1996, Mesfin_2022a, Misra_2013, Naji_2012f, Ohno_2005, Sakellaridis_2008f, Sobnach_2011f, Sultan_2024f, Tanrikulu_2015e, Tay_2004, Thapa_2019f, Trgo_2012f, Tupesis_2004f, Wildhaber_2005, Wnęk_2015f, Yasin_2009, Yildiz_2016e}, 44 cases (61\%) underwent surgery \cite{Al-Faham_2020k, AlShaaibi_2021b, Alao_2006i, Ali_2017, Ali_2020f, Atayan_2016, Beecroft_1998, Bhasin_2014, CamachoDorado_2018, Cauchi_2002, Chang_2017f, Cox_2007, Csaky_1998e, DelgadoSalazar_2020c, DivsalarP._2023a, Farhadi_2024h, Fry_2010, Gardner_2017h, Jin_2023, Kariholu_2008, Kerestes_2019, Kobiela_2015, Kumar_2019f, Liu_2005, Losanoff_1996, Losanoff_1997e, Mesfin_2022a, Misra_2013, Naji_2012f, Sobnach_2011f, Tanrikulu_2015e, Tay_2004, Thapa_2019f, Tupesis_2004f, Wildhaber_2005, Wnęk_2015f, Yasin_2009, Yildiz_2016e, fjbuilsRepeatedBehaviorDeliberate2024}, 31 cases (43\%) underwent endoscopy \cite{Akay_2015f, Ali_2022g, Apikotoa_2022f, Atayan_2016, Benoist_2019e, Berry_2021e, Bhasin_2014, Bhumi_2024f, CamachoDorado_2018, Chang_2017f, DelgadoSalazar_2020c, Gardner_2017h, Guinan_2019f, Hardy_2023g, Jehangir_2019h, Kariholu_2008, Li_2013, Liu_2005, Ohno_2005, Peixoto_2017f, Qureshi_2016, Riva_2018j, Sakellaridis_2008f, Sultan_2024f, Tammana_2012j, Tanrikulu_2015e, Trgo_2012f, Wadhwa_2015e, Wnęk_2015f, teWildt_2010}, 7 cases (10\%) were managed conservatively \cite{Ataya_2013, Bhattacharjee_2008, DivsalarP._2023a, Emamhadi_2018, Goldman_1998f, Kar_2015, Kumar_2001}, 2 cases (3\%) died \cite{Emamhadi_2018, Kumar_2001}. All 90 were male gender. 90 cases (100\%) were detained at the time of ingestion \cite{Elghali_2016, Karp_1991b, Lee_2007}, 88 cases (98\%) were intentional ingestions \cite{Elghali_2016, Karp_1991b, Lee_2007}, 30 cases (33\%) had a psychiatric history documented \cite{Elghali_2016, Karp_1991b, Lee_2007}, 2 cases (2\%) had a history of prior ingestion \cite{Elghali_2016}. No cases were reported for were psychiatric inpatients, were displaced people, were under the influence of alcohol at the time of ingestion, and had a severe disability history.
\paragraph*{Motivation}  70 cases (78\%) reported protest motivation \cite{Elghali_2016, Karp_1991b, Lee_2007}, 12 cases (13\%) reported psychiatric motivation \cite{Karp_1991b}, 6 cases (7\%) reported self-harm motivation \cite{Elghali_2016, Karp_1991b}. No cases were reported for psychosocial motivation and other motivation.
\paragraph*{Object Characteristics}  68 cases (76\%) involved sharp object ingestion \cite{Elghali_2016, Karp_1991b, Lee_2007}, 32 cases (36\%) involved long (\textgreater 5cm) object ingestion \cite{Lee_2007}, 25 cases (28\%) involved ingestion of multiple objects \cite{Elghali_2016, Lee_2007}. No cases were reported for button battery ingestion, magnet ingestion, and involved large diameter (\textgreater 2.5cm) object ingestion.
\paragraph*{Outcomes}  47 cases (52\%) underwent endoscopic intervention \cite{Elghali_2016, Lee_2007}, 29 cases (32\%) were managed conservatively \cite{Elghali_2016, Karp_1991b}, 15 cases (17\%) underwent surgical intervention \cite{Elghali_2016, Karp_1991b, Lee_2007}, 6 cases (7\%) reported complications \cite{Lee_2007}, 1 case (1\%) died \cite{Elghali_2016}.
\paragraph*{Geographical Location}Cases were recorded in 33 countries: 13 cases from USA \cite{Alao_2006i, Ataya_2013, Bhumi_2024f, Fry_2010, Guinan_2019f, Hardy_2023g, Jehangir_2019h, Kerestes_2019, Kumar_2001, Liu_2005, Tammana_2012j, Tay_2004, Tupesis_2004f}; 7 cases from India \cite{Bhasin_2014, Bhattacharjee_2008, Kar_2015, Kariholu_2008, Kumar_2019f, Misra_2013, Wadhwa_2015e} and UK \cite{Beecroft_1998, Berry_2021e, Cauchi_2002, Cox_2007, Gardner_2017h, Qureshi_2016}; 6 cases from Bulgaria \cite{Losanoff_1996, Losanoff_1997e}; 5 cases from Iran \cite{DivsalarP._2023a, Emamhadi_2018, Farhadi_2024h}; 4 cases from Turkey \cite{Akay_2015f, Atayan_2016, Tanrikulu_2015e, Yildiz_2016e}; 2 cases from China \cite{Jin_2023, Li_2013}, Poland \cite{Kobiela_2015, Wnęk_2015f}, and Spain \cite{CamachoDorado_2018, fjbuilsRepeatedBehaviorDeliberate2024}; 1 case from Australia \cite{Apikotoa_2022f}, Bahrain \cite{Ali_2020f}, Croatia \cite{Trgo_2012f}, Ecuador \cite{DelgadoSalazar_2020c}, Egypt \cite{Ali_2022g}, Ethiopia \cite{Mesfin_2022a}, Germany \cite{teWildt_2010}, Greece \cite{Sakellaridis_2008f}, Hungary \cite{Csaky_1998e}, Iraq \cite{Al-Faham_2020k}, Israel \cite{Goldman_1998f}, Italy \cite{Riva_2018j}, Japan \cite{Ohno_2005}, Nepal \cite{Thapa_2019f}, Netherlands \cite{Benoist_2019e}, Oman \cite{AlShaaibi_2021b}, Pakistan \cite{Yasin_2009}, Portugal \cite{Peixoto_2017f}, Qatar \cite{Ali_2017}, Saudi Arabia \cite{Sultan_2024f}, South Africa \cite{Sobnach_2011f}, Sweden \cite{Naji_2012f}, Switzerland \cite{Wildhaber_2005}, and Taiwan \cite{Chang_2017f}. \paragraph*{Gender} 43 cases (60\%) were male \cite{Akay_2015f, Al-Faham_2020k, Alao_2006i, Ali_2017, Ali_2022g, Apikotoa_2022f, Atayan_2016, Benoist_2019e, Berry_2021e, Bhumi_2024f, CamachoDorado_2018, Csaky_1998e, Emamhadi_2018, Farhadi_2024h, Fry_2010, Gardner_2017h, Guinan_2019f, Jehangir_2019h, Jin_2023, Kobiela_2015, Kumar_2001, Kumar_2019f, Liu_2005, Losanoff_1996, Losanoff_1997e, Mesfin_2022a, Misra_2013, Qureshi_2016, Riva_2018j, Sobnach_2011f, Tammana_2012j, Tanrikulu_2015e, Tay_2004, Thapa_2019f, Trgo_2012f, Wadhwa_2015e, Yasin_2009, teWildt_2010}, 28 cases (39\%) were female \cite{AlShaaibi_2021b, Ali_2020f, Ataya_2013, Beecroft_1998, Bhasin_2014, Bhattacharjee_2008, Cauchi_2002, Chang_2017f, Cox_2007, DelgadoSalazar_2020c, DivsalarP._2023a, Goldman_1998f, Hardy_2023g, Kar_2015, Kariholu_2008, Kerestes_2019, Li_2013, Naji_2012f, Ohno_2005, Peixoto_2017f, Sakellaridis_2008f, Sultan_2024f, Tupesis_2004f, Wildhaber_2005, Wnęk_2015f, Yildiz_2016e}, 1 case (1\%) had no gender recorded \cite{fjbuilsRepeatedBehaviorDeliberate2024}. \paragraph*{Age Group} 25 cases (35\%) were between 26 and 40 years of age \cite{Alao_2006i, Ali_2022g, Apikotoa_2022f, Ataya_2013, Benoist_2019e, Bhasin_2014, Chang_2017f, Cox_2007, DelgadoSalazar_2020c, Farhadi_2024h, Fry_2010, Gardner_2017h, Guinan_2019f, Jin_2023, Kumar_2019f, Losanoff_1996, Misra_2013, Qureshi_2016, Riva_2018j, Sakellaridis_2008f, Tammana_2012j, Trgo_2012f, Wnęk_2015f, Yildiz_2016e, fjbuilsRepeatedBehaviorDeliberate2024}, 18 cases (25\%) were between 18 and 25 years of age \cite{Akay_2015f, Ali_2017, Atayan_2016, Bhattacharjee_2008, Csaky_1998e, Kar_2015, Kariholu_2008, Kobiela_2015, Losanoff_1996, Losanoff_1997e, Mesfin_2022a, Peixoto_2017f, Sobnach_2011f, Tupesis_2004f, Yasin_2009}, 13 cases (18\%) were under 18 years of age \cite{AlShaaibi_2021b, Ali_2020f, Cauchi_2002, DivsalarP._2023a, Goldman_1998f, Liu_2005, Naji_2012f, Ohno_2005, Tanrikulu_2015e, Tay_2004, Wildhaber_2005}, 11 cases (15\%) were between 41 and 60 years of age \cite{Al-Faham_2020k, Bhumi_2024f, CamachoDorado_2018, Emamhadi_2018, Hardy_2023g, Jehangir_2019h, Kumar_2001, Sultan_2024f, Thapa_2019f, Wadhwa_2015e, teWildt_2010}, 3 cases (4\%) were over 60 years of age \cite{Beecroft_1998, Kerestes_2019, Li_2013}, 2 cases (3\%) had no age documented \cite{Berry_2021e}. \paragraph*{Population} 36 cases (50\%) had a psychiatric history \cite{AlShaaibi_2021b, Alao_2006i, Ali_2020f, Apikotoa_2022f, Ataya_2013, Atayan_2016, Beecroft_1998, CamachoDorado_2018, Chang_2017f, DelgadoSalazar_2020c, DivsalarP._2023a, Farhadi_2024h, Fry_2010, Guinan_2019f, Hardy_2023g, Jehangir_2019h, Jin_2023, Kar_2015, Kerestes_2019, Kobiela_2015, Kumar_2001, Kumar_2019f, Liu_2005, Mesfin_2022a, Misra_2013, Ohno_2005, Peixoto_2017f, Sakellaridis_2008f, Sultan_2024f, Tammana_2012j, Tanrikulu_2015e, Yildiz_2016e, fjbuilsRepeatedBehaviorDeliberate2024, teWildt_2010}, 19 cases (26\%) had ingested previously \cite{Alao_2006i, Apikotoa_2022f, Berry_2021e, Bhattacharjee_2008, Csaky_1998e, DivsalarP._2023a, Emamhadi_2018, Guinan_2019f, Jehangir_2019h, Jin_2023, Liu_2005, Sakellaridis_2008f, Tanrikulu_2015e, Thapa_2019f, Yildiz_2016e, fjbuilsRepeatedBehaviorDeliberate2024, teWildt_2010}, 12 cases (17\%) were detained persons \cite{Alao_2006i, Ali_2022g, Apikotoa_2022f, Losanoff_1996, Losanoff_1997e, Qureshi_2016, Tammana_2012j, Trgo_2012f}, 7 cases (10\%) were severely disabled \cite{Atayan_2016, Kerestes_2019, Liu_2005, Ohno_2005, Peixoto_2017f, Yildiz_2016e, teWildt_2010}, 4 cases (6\%) were psychiatric inpatients \cite{DivsalarP._2023a, fjbuilsRepeatedBehaviorDeliberate2024, teWildt_2010}, 3 cases (4\%) were under the influence of alcohol \cite{Benoist_2019e, Csaky_1998e, Thapa_2019f}, 2 cases (3\%) were displaced people \cite{Akay_2015f, Gardner_2017h}. \paragraph*{Motivation} 34 cases (47\%) had a psychiatric motivation \cite{Al-Faham_2020k, Alao_2006i, Ali_2020f, Apikotoa_2022f, Ataya_2013, Atayan_2016, Bhasin_2014, Bhattacharjee_2008, DelgadoSalazar_2020c, DivsalarP._2023a, Emamhadi_2018, Farhadi_2024h, Guinan_2019f, Hardy_2023g, Jehangir_2019h, Jin_2023, Kar_2015, Kariholu_2008, Kerestes_2019, Kobiela_2015, Kumar_2001, Kumar_2019f, Li_2013, Liu_2005, Misra_2013, Ohno_2005, Sakellaridis_2008f, Sultan_2024f, Tammana_2012j, Tanrikulu_2015e, Yasin_2009, teWildt_2010}, 21 cases (29\%) were motivated by self-harm intention \cite{Al-Faham_2020k, AlShaaibi_2021b, Alao_2006i, Ali_2017, CamachoDorado_2018, Chang_2017f, Cox_2007, Csaky_1998e, Fry_2010, Li_2013, Losanoff_1996, Losanoff_1997e, Mesfin_2022a, Sakellaridis_2008f, Tammana_2012j, Tanrikulu_2015e, fjbuilsRepeatedBehaviorDeliberate2024}, 17 cases (24\%) had a psychosocial motivation \cite{Akay_2015f, Benoist_2019e, Bhattacharjee_2008, Cauchi_2002, Goldman_1998f, Hardy_2023g, Kobiela_2015, Li_2013, Naji_2012f, Qureshi_2016, Riva_2018j, Sobnach_2011f, Tay_2004, Thapa_2019f, Tupesis_2004f, Wildhaber_2005, Wnęk_2015f}, 9 cases (12\%) were motivated by protest \cite{Bhumi_2024f, Gardner_2017h, Losanoff_1996, Losanoff_1997e, Tupesis_2004f}, 9 cases (12\%) had another documented motivation \cite{Ali_2020f, Ali_2022g, Emamhadi_2018, Guinan_2019f, Peixoto_2017f, Sakellaridis_2008f, Trgo_2012f, Wadhwa_2015e, Yildiz_2016e}. \paragraph*{Object Characteristics} 51 cases (71\%) ingested a large diameter object (\textgreater{}2.5cm) \cite{Akay_2015f, Al-Faham_2020k, AlShaaibi_2021b, Alao_2006i, Ali_2017, Ali_2022g, Apikotoa_2022f, Atayan_2016, Berry_2021e, Bhasin_2014, CamachoDorado_2018, Cauchi_2002, Chang_2017f, Cox_2007, Csaky_1998e, DivsalarP._2023a, Emamhadi_2018, Gardner_2017h, Guinan_2019f, Jehangir_2019h, Jin_2023, Kariholu_2008, Kerestes_2019, Kobiela_2015, Kumar_2001, Kumar_2019f, Losanoff_1996, Losanoff_1997e, Mesfin_2022a, Misra_2013, Naji_2012f, Ohno_2005, Peixoto_2017f, Qureshi_2016, Riva_2018j, Sakellaridis_2008f, Sultan_2024f, Tanrikulu_2015e, Thapa_2019f, Trgo_2012f, Wnęk_2015f, Yildiz_2016e, fjbuilsRepeatedBehaviorDeliberate2024, teWildt_2010}, 44 cases (61\%) ingested multiple objects \cite{Ali_2020f, Apikotoa_2022f, Ataya_2013, Atayan_2016, Beecroft_1998, Bhattacharjee_2008, Bhumi_2024f, CamachoDorado_2018, Cauchi_2002, Emamhadi_2018, Farhadi_2024h, Fry_2010, Goldman_1998f, Guinan_2019f, Hardy_2023g, Jehangir_2019h, Jin_2023, Kar_2015, Kariholu_2008, Kobiela_2015, Kumar_2001, Kumar_2019f, Li_2013, Liu_2005, Losanoff_1996, Mesfin_2022a, Misra_2013, Naji_2012f, Ohno_2005, Sobnach_2011f, Sultan_2024f, Tammana_2012j, Tanrikulu_2015e, Tay_2004, Thapa_2019f, Wadhwa_2015e, Wildhaber_2005, Yasin_2009, fjbuilsRepeatedBehaviorDeliberate2024, teWildt_2010}, 34 cases (47\%) ingested a sharp object \cite{AlShaaibi_2021b, Alao_2006i, Apikotoa_2022f, Ataya_2013, Benoist_2019e, Bhasin_2014, Bhattacharjee_2008, CamachoDorado_2018, Csaky_1998e, DelgadoSalazar_2020c, DivsalarP._2023a, Emamhadi_2018, Farhadi_2024h, Fry_2010, Guinan_2019f, Hardy_2023g, Jehangir_2019h, Jin_2023, Kariholu_2008, Kobiela_2015, Kumar_2019f, Losanoff_1996, Losanoff_1997e, Mesfin_2022a, Misra_2013, Sobnach_2011f, Yasin_2009, teWildt_2010}, 32 cases (44\%) ingested a long object (\textgreater{}5cm) \cite{Al-Faham_2020k, AlShaaibi_2021b, Ali_2017, Ali_2022g, Atayan_2016, Bhasin_2014, CamachoDorado_2018, Chang_2017f, Cox_2007, Csaky_1998e, DivsalarP._2023a, Emamhadi_2018, Fry_2010, Gardner_2017h, Jin_2023, Kariholu_2008, Kerestes_2019, Kobiela_2015, Kumar_2019f, Mesfin_2022a, Misra_2013, Ohno_2005, Qureshi_2016, Sakellaridis_2008f, Sultan_2024f, Thapa_2019f, Trgo_2012f, Yasin_2009, Yildiz_2016e, teWildt_2010}, 9 cases (12\%) ingested a magnet \cite{Ali_2020f, Bhumi_2024f, Cauchi_2002, Liu_2005, Naji_2012f, Ohno_2005, Tanrikulu_2015e, Tay_2004, Wildhaber_2005}, 2 cases (3\%) ingested a button battery \cite{Berry_2021e, Bhumi_2024f}. \paragraph*{Outcomes} 48 cases (67\%) experienced a complication \cite{Ali_2017, Ali_2020f, Apikotoa_2022f, Atayan_2016, Beecroft_1998, Benoist_2019e, Berry_2021e, Bhasin_2014, Bhumi_2024f, CamachoDorado_2018, Cauchi_2002, Cox_2007, Csaky_1998e, DelgadoSalazar_2020c, DivsalarP._2023a, Emamhadi_2018, Farhadi_2024h, Fry_2010, Gardner_2017h, Goldman_1998f, Jin_2023, Kariholu_2008, Kerestes_2019, Kobiela_2015, Kumar_2001, Kumar_2019f, Liu_2005, Losanoff_1996, Mesfin_2022a, Misra_2013, Naji_2012f, Ohno_2005, Sakellaridis_2008f, Sobnach_2011f, Sultan_2024f, Tanrikulu_2015e, Tay_2004, Thapa_2019f, Trgo_2012f, Tupesis_2004f, Wildhaber_2005, Wnęk_2015f, Yasin_2009, Yildiz_2016e}, 44 cases (61\%) underwent surgery \cite{Al-Faham_2020k, AlShaaibi_2021b, Alao_2006i, Ali_2017, Ali_2020f, Atayan_2016, Beecroft_1998, Bhasin_2014, CamachoDorado_2018, Cauchi_2002, Chang_2017f, Cox_2007, Csaky_1998e, DelgadoSalazar_2020c, DivsalarP._2023a, Farhadi_2024h, Fry_2010, Gardner_2017h, Jin_2023, Kariholu_2008, Kerestes_2019, Kobiela_2015, Kumar_2019f, Liu_2005, Losanoff_1996, Losanoff_1997e, Mesfin_2022a, Misra_2013, Naji_2012f, Sobnach_2011f, Tanrikulu_2015e, Tay_2004, Thapa_2019f, Tupesis_2004f, Wildhaber_2005, Wnęk_2015f, Yasin_2009, Yildiz_2016e, fjbuilsRepeatedBehaviorDeliberate2024}, 31 cases (43\%) underwent endoscopy \cite{Akay_2015f, Ali_2022g, Apikotoa_2022f, Atayan_2016, Benoist_2019e, Berry_2021e, Bhasin_2014, Bhumi_2024f, CamachoDorado_2018, Chang_2017f, DelgadoSalazar_2020c, Gardner_2017h, Guinan_2019f, Hardy_2023g, Jehangir_2019h, Kariholu_2008, Li_2013, Liu_2005, Ohno_2005, Peixoto_2017f, Qureshi_2016, Riva_2018j, Sakellaridis_2008f, Sultan_2024f, Tammana_2012j, Tanrikulu_2015e, Trgo_2012f, Wadhwa_2015e, Wnęk_2015f, teWildt_2010}, 7 cases (10\%) were managed conservatively \cite{Ataya_2013, Bhattacharjee_2008, DivsalarP._2023a, Emamhadi_2018, Goldman_1998f, Kar_2015, Kumar_2001}, 2 cases (3\%) died \cite{Emamhadi_2018, Kumar_2001}. All 90 were male gender. 90 cases (100\%) were detained at the time of ingestion \cite{Elghali_2016, Karp_1991b, Lee_2007}, 88 cases (98\%) were intentional ingestions \cite{Elghali_2016, Karp_1991b, Lee_2007}, 30 cases (33\%) had a psychiatric history documented \cite{Elghali_2016, Karp_1991b, Lee_2007}, 2 cases (2\%) had a history of prior ingestion \cite{Elghali_2016}. No cases were reported for were psychiatric inpatients, were displaced people, were under the influence of alcohol at the time of ingestion, and had a severe disability history.
\paragraph*{Motivation}  70 cases (78\%) reported protest motivation \cite{Elghali_2016, Karp_1991b, Lee_2007}, 12 cases (13\%) reported psychiatric motivation \cite{Karp_1991b}, 6 cases (7\%) reported self-harm motivation \cite{Elghali_2016, Karp_1991b}. No cases were reported for psychosocial motivation and other motivation.
\paragraph*{Object Characteristics}  68 cases (76\%) involved sharp object ingestion \cite{Elghali_2016, Karp_1991b, Lee_2007}, 32 cases (36\%) involved long (\textgreater 5cm) object ingestion \cite{Lee_2007}, 25 cases (28\%) involved ingestion of multiple objects \cite{Elghali_2016, Lee_2007}. No cases were reported for button battery ingestion, magnet ingestion, and involved large diameter (\textgreater 2.5cm) object ingestion.
\paragraph*{Outcomes}  47 cases (52\%) underwent endoscopic intervention \cite{Elghali_2016, Lee_2007}, 29 cases (32\%) were managed conservatively \cite{Elghali_2016, Karp_1991b}, 15 cases (17\%) underwent surgical intervention \cite{Elghali_2016, Karp_1991b, Lee_2007}, 6 cases (7\%) reported complications \cite{Lee_2007}, 1 case (1\%) died \cite{Elghali_2016}.
\paragraph*{Geographical Location}Cases were recorded in 33 countries: 13 cases from USA \cite{Alao_2006i, Ataya_2013, Bhumi_2024f, Fry_2010, Guinan_2019f, Hardy_2023g, Jehangir_2019h, Kerestes_2019, Kumar_2001, Liu_2005, Tammana_2012j, Tay_2004, Tupesis_2004f}; 7 cases from India \cite{Bhasin_2014, Bhattacharjee_2008, Kar_2015, Kariholu_2008, Kumar_2019f, Misra_2013, Wadhwa_2015e} and UK \cite{Beecroft_1998, Berry_2021e, Cauchi_2002, Cox_2007, Gardner_2017h, Qureshi_2016}; 6 cases from Bulgaria \cite{Losanoff_1996, Losanoff_1997e}; 5 cases from Iran \cite{DivsalarP._2023a, Emamhadi_2018, Farhadi_2024h}; 4 cases from Turkey \cite{Akay_2015f, Atayan_2016, Tanrikulu_2015e, Yildiz_2016e}; 2 cases from China \cite{Jin_2023, Li_2013}, Poland \cite{Kobiela_2015, Wnęk_2015f}, and Spain \cite{CamachoDorado_2018, fjbuilsRepeatedBehaviorDeliberate2024}; 1 case from Australia \cite{Apikotoa_2022f}, Bahrain \cite{Ali_2020f}, Croatia \cite{Trgo_2012f}, Ecuador \cite{DelgadoSalazar_2020c}, Egypt \cite{Ali_2022g}, Ethiopia \cite{Mesfin_2022a}, Germany \cite{teWildt_2010}, Greece \cite{Sakellaridis_2008f}, Hungary \cite{Csaky_1998e}, Iraq \cite{Al-Faham_2020k}, Israel \cite{Goldman_1998f}, Italy \cite{Riva_2018j}, Japan \cite{Ohno_2005}, Nepal \cite{Thapa_2019f}, Netherlands \cite{Benoist_2019e}, Oman \cite{AlShaaibi_2021b}, Pakistan \cite{Yasin_2009}, Portugal \cite{Peixoto_2017f}, Qatar \cite{Ali_2017}, Saudi Arabia \cite{Sultan_2024f}, South Africa \cite{Sobnach_2011f}, Sweden \cite{Naji_2012f}, Switzerland \cite{Wildhaber_2005}, and Taiwan \cite{Chang_2017f}. \paragraph*{Gender} 43 cases (60\%) were male \cite{Akay_2015f, Al-Faham_2020k, Alao_2006i, Ali_2017, Ali_2022g, Apikotoa_2022f, Atayan_2016, Benoist_2019e, Berry_2021e, Bhumi_2024f, CamachoDorado_2018, Csaky_1998e, Emamhadi_2018, Farhadi_2024h, Fry_2010, Gardner_2017h, Guinan_2019f, Jehangir_2019h, Jin_2023, Kobiela_2015, Kumar_2001, Kumar_2019f, Liu_2005, Losanoff_1996, Losanoff_1997e, Mesfin_2022a, Misra_2013, Qureshi_2016, Riva_2018j, Sobnach_2011f, Tammana_2012j, Tanrikulu_2015e, Tay_2004, Thapa_2019f, Trgo_2012f, Wadhwa_2015e, Yasin_2009, teWildt_2010}, 28 cases (39\%) were female \cite{AlShaaibi_2021b, Ali_2020f, Ataya_2013, Beecroft_1998, Bhasin_2014, Bhattacharjee_2008, Cauchi_2002, Chang_2017f, Cox_2007, DelgadoSalazar_2020c, DivsalarP._2023a, Goldman_1998f, Hardy_2023g, Kar_2015, Kariholu_2008, Kerestes_2019, Li_2013, Naji_2012f, Ohno_2005, Peixoto_2017f, Sakellaridis_2008f, Sultan_2024f, Tupesis_2004f, Wildhaber_2005, Wnęk_2015f, Yildiz_2016e}, 1 case (1\%) had no gender recorded \cite{fjbuilsRepeatedBehaviorDeliberate2024}. \paragraph*{Age Group} 25 cases (35\%) were between 26 and 40 years of age \cite{Alao_2006i, Ali_2022g, Apikotoa_2022f, Ataya_2013, Benoist_2019e, Bhasin_2014, Chang_2017f, Cox_2007, DelgadoSalazar_2020c, Farhadi_2024h, Fry_2010, Gardner_2017h, Guinan_2019f, Jin_2023, Kumar_2019f, Losanoff_1996, Misra_2013, Qureshi_2016, Riva_2018j, Sakellaridis_2008f, Tammana_2012j, Trgo_2012f, Wnęk_2015f, Yildiz_2016e, fjbuilsRepeatedBehaviorDeliberate2024}, 18 cases (25\%) were between 18 and 25 years of age \cite{Akay_2015f, Ali_2017, Atayan_2016, Bhattacharjee_2008, Csaky_1998e, Kar_2015, Kariholu_2008, Kobiela_2015, Losanoff_1996, Losanoff_1997e, Mesfin_2022a, Peixoto_2017f, Sobnach_2011f, Tupesis_2004f, Yasin_2009}, 13 cases (18\%) were under 18 years of age \cite{AlShaaibi_2021b, Ali_2020f, Cauchi_2002, DivsalarP._2023a, Goldman_1998f, Liu_2005, Naji_2012f, Ohno_2005, Tanrikulu_2015e, Tay_2004, Wildhaber_2005}, 11 cases (15\%) were between 41 and 60 years of age \cite{Al-Faham_2020k, Bhumi_2024f, CamachoDorado_2018, Emamhadi_2018, Hardy_2023g, Jehangir_2019h, Kumar_2001, Sultan_2024f, Thapa_2019f, Wadhwa_2015e, teWildt_2010}, 3 cases (4\%) were over 60 years of age \cite{Beecroft_1998, Kerestes_2019, Li_2013}, 2 cases (3\%) had no age documented \cite{Berry_2021e}. \paragraph*{Population} 36 cases (50\%) had a psychiatric history \cite{AlShaaibi_2021b, Alao_2006i, Ali_2020f, Apikotoa_2022f, Ataya_2013, Atayan_2016, Beecroft_1998, CamachoDorado_2018, Chang_2017f, DelgadoSalazar_2020c, DivsalarP._2023a, Farhadi_2024h, Fry_2010, Guinan_2019f, Hardy_2023g, Jehangir_2019h, Jin_2023, Kar_2015, Kerestes_2019, Kobiela_2015, Kumar_2001, Kumar_2019f, Liu_2005, Mesfin_2022a, Misra_2013, Ohno_2005, Peixoto_2017f, Sakellaridis_2008f, Sultan_2024f, Tammana_2012j, Tanrikulu_2015e, Yildiz_2016e, fjbuilsRepeatedBehaviorDeliberate2024, teWildt_2010}, 19 cases (26\%) had ingested previously \cite{Alao_2006i, Apikotoa_2022f, Berry_2021e, Bhattacharjee_2008, Csaky_1998e, DivsalarP._2023a, Emamhadi_2018, Guinan_2019f, Jehangir_2019h, Jin_2023, Liu_2005, Sakellaridis_2008f, Tanrikulu_2015e, Thapa_2019f, Yildiz_2016e, fjbuilsRepeatedBehaviorDeliberate2024, teWildt_2010}, 12 cases (17\%) were detained persons \cite{Alao_2006i, Ali_2022g, Apikotoa_2022f, Losanoff_1996, Losanoff_1997e, Qureshi_2016, Tammana_2012j, Trgo_2012f}, 7 cases (10\%) were severely disabled \cite{Atayan_2016, Kerestes_2019, Liu_2005, Ohno_2005, Peixoto_2017f, Yildiz_2016e, teWildt_2010}, 4 cases (6\%) were psychiatric inpatients \cite{DivsalarP._2023a, fjbuilsRepeatedBehaviorDeliberate2024, teWildt_2010}, 3 cases (4\%) were under the influence of alcohol \cite{Benoist_2019e, Csaky_1998e, Thapa_2019f}, 2 cases (3\%) were displaced people \cite{Akay_2015f, Gardner_2017h}. \paragraph*{Motivation} 34 cases (47\%) had a psychiatric motivation \cite{Al-Faham_2020k, Alao_2006i, Ali_2020f, Apikotoa_2022f, Ataya_2013, Atayan_2016, Bhasin_2014, Bhattacharjee_2008, DelgadoSalazar_2020c, DivsalarP._2023a, Emamhadi_2018, Farhadi_2024h, Guinan_2019f, Hardy_2023g, Jehangir_2019h, Jin_2023, Kar_2015, Kariholu_2008, Kerestes_2019, Kobiela_2015, Kumar_2001, Kumar_2019f, Li_2013, Liu_2005, Misra_2013, Ohno_2005, Sakellaridis_2008f, Sultan_2024f, Tammana_2012j, Tanrikulu_2015e, Yasin_2009, teWildt_2010}, 21 cases (29\%) were motivated by self-harm intention \cite{Al-Faham_2020k, AlShaaibi_2021b, Alao_2006i, Ali_2017, CamachoDorado_2018, Chang_2017f, Cox_2007, Csaky_1998e, Fry_2010, Li_2013, Losanoff_1996, Losanoff_1997e, Mesfin_2022a, Sakellaridis_2008f, Tammana_2012j, Tanrikulu_2015e, fjbuilsRepeatedBehaviorDeliberate2024}, 17 cases (24\%) had a psychosocial motivation \cite{Akay_2015f, Benoist_2019e, Bhattacharjee_2008, Cauchi_2002, Goldman_1998f, Hardy_2023g, Kobiela_2015, Li_2013, Naji_2012f, Qureshi_2016, Riva_2018j, Sobnach_2011f, Tay_2004, Thapa_2019f, Tupesis_2004f, Wildhaber_2005, Wnęk_2015f}, 9 cases (12\%) were motivated by protest \cite{Bhumi_2024f, Gardner_2017h, Losanoff_1996, Losanoff_1997e, Tupesis_2004f}, 9 cases (12\%) had another documented motivation \cite{Ali_2020f, Ali_2022g, Emamhadi_2018, Guinan_2019f, Peixoto_2017f, Sakellaridis_2008f, Trgo_2012f, Wadhwa_2015e, Yildiz_2016e}. \paragraph*{Object Characteristics} 51 cases (71\%) ingested a large diameter object (\textgreater{}2.5cm) \cite{Akay_2015f, Al-Faham_2020k, AlShaaibi_2021b, Alao_2006i, Ali_2017, Ali_2022g, Apikotoa_2022f, Atayan_2016, Berry_2021e, Bhasin_2014, CamachoDorado_2018, Cauchi_2002, Chang_2017f, Cox_2007, Csaky_1998e, DivsalarP._2023a, Emamhadi_2018, Gardner_2017h, Guinan_2019f, Jehangir_2019h, Jin_2023, Kariholu_2008, Kerestes_2019, Kobiela_2015, Kumar_2001, Kumar_2019f, Losanoff_1996, Losanoff_1997e, Mesfin_2022a, Misra_2013, Naji_2012f, Ohno_2005, Peixoto_2017f, Qureshi_2016, Riva_2018j, Sakellaridis_2008f, Sultan_2024f, Tanrikulu_2015e, Thapa_2019f, Trgo_2012f, Wnęk_2015f, Yildiz_2016e, fjbuilsRepeatedBehaviorDeliberate2024, teWildt_2010}, 44 cases (61\%) ingested multiple objects \cite{Ali_2020f, Apikotoa_2022f, Ataya_2013, Atayan_2016, Beecroft_1998, Bhattacharjee_2008, Bhumi_2024f, CamachoDorado_2018, Cauchi_2002, Emamhadi_2018, Farhadi_2024h, Fry_2010, Goldman_1998f, Guinan_2019f, Hardy_2023g, Jehangir_2019h, Jin_2023, Kar_2015, Kariholu_2008, Kobiela_2015, Kumar_2001, Kumar_2019f, Li_2013, Liu_2005, Losanoff_1996, Mesfin_2022a, Misra_2013, Naji_2012f, Ohno_2005, Sobnach_2011f, Sultan_2024f, Tammana_2012j, Tanrikulu_2015e, Tay_2004, Thapa_2019f, Wadhwa_2015e, Wildhaber_2005, Yasin_2009, fjbuilsRepeatedBehaviorDeliberate2024, teWildt_2010}, 34 cases (47\%) ingested a sharp object \cite{AlShaaibi_2021b, Alao_2006i, Apikotoa_2022f, Ataya_2013, Benoist_2019e, Bhasin_2014, Bhattacharjee_2008, CamachoDorado_2018, Csaky_1998e, DelgadoSalazar_2020c, DivsalarP._2023a, Emamhadi_2018, Farhadi_2024h, Fry_2010, Guinan_2019f, Hardy_2023g, Jehangir_2019h, Jin_2023, Kariholu_2008, Kobiela_2015, Kumar_2019f, Losanoff_1996, Losanoff_1997e, Mesfin_2022a, Misra_2013, Sobnach_2011f, Yasin_2009, teWildt_2010}, 32 cases (44\%) ingested a long object (\textgreater{}5cm) \cite{Al-Faham_2020k, AlShaaibi_2021b, Ali_2017, Ali_2022g, Atayan_2016, Bhasin_2014, CamachoDorado_2018, Chang_2017f, Cox_2007, Csaky_1998e, DivsalarP._2023a, Emamhadi_2018, Fry_2010, Gardner_2017h, Jin_2023, Kariholu_2008, Kerestes_2019, Kobiela_2015, Kumar_2019f, Mesfin_2022a, Misra_2013, Ohno_2005, Qureshi_2016, Sakellaridis_2008f, Sultan_2024f, Thapa_2019f, Trgo_2012f, Yasin_2009, Yildiz_2016e, teWildt_2010}, 9 cases (12\%) ingested a magnet \cite{Ali_2020f, Bhumi_2024f, Cauchi_2002, Liu_2005, Naji_2012f, Ohno_2005, Tanrikulu_2015e, Tay_2004, Wildhaber_2005}, 2 cases (3\%) ingested a button battery \cite{Berry_2021e, Bhumi_2024f}. \paragraph*{Outcomes} 48 cases (67\%) experienced a complication \cite{Ali_2017, Ali_2020f, Apikotoa_2022f, Atayan_2016, Beecroft_1998, Benoist_2019e, Berry_2021e, Bhasin_2014, Bhumi_2024f, CamachoDorado_2018, Cauchi_2002, Cox_2007, Csaky_1998e, DelgadoSalazar_2020c, DivsalarP._2023a, Emamhadi_2018, Farhadi_2024h, Fry_2010, Gardner_2017h, Goldman_1998f, Jin_2023, Kariholu_2008, Kerestes_2019, Kobiela_2015, Kumar_2001, Kumar_2019f, Liu_2005, Losanoff_1996, Mesfin_2022a, Misra_2013, Naji_2012f, Ohno_2005, Sakellaridis_2008f, Sobnach_2011f, Sultan_2024f, Tanrikulu_2015e, Tay_2004, Thapa_2019f, Trgo_2012f, Tupesis_2004f, Wildhaber_2005, Wnęk_2015f, Yasin_2009, Yildiz_2016e}, 44 cases (61\%) underwent surgery \cite{Al-Faham_2020k, AlShaaibi_2021b, Alao_2006i, Ali_2017, Ali_2020f, Atayan_2016, Beecroft_1998, Bhasin_2014, CamachoDorado_2018, Cauchi_2002, Chang_2017f, Cox_2007, Csaky_1998e, DelgadoSalazar_2020c, DivsalarP._2023a, Farhadi_2024h, Fry_2010, Gardner_2017h, Jin_2023, Kariholu_2008, Kerestes_2019, Kobiela_2015, Kumar_2019f, Liu_2005, Losanoff_1996, Losanoff_1997e, Mesfin_2022a, Misra_2013, Naji_2012f, Sobnach_2011f, Tanrikulu_2015e, Tay_2004, Thapa_2019f, Tupesis_2004f, Wildhaber_2005, Wnęk_2015f, Yasin_2009, Yildiz_2016e, fjbuilsRepeatedBehaviorDeliberate2024}, 31 cases (43\%) underwent endoscopy \cite{Akay_2015f, Ali_2022g, Apikotoa_2022f, Atayan_2016, Benoist_2019e, Berry_2021e, Bhasin_2014, Bhumi_2024f, CamachoDorado_2018, Chang_2017f, DelgadoSalazar_2020c, Gardner_2017h, Guinan_2019f, Hardy_2023g, Jehangir_2019h, Kariholu_2008, Li_2013, Liu_2005, Ohno_2005, Peixoto_2017f, Qureshi_2016, Riva_2018j, Sakellaridis_2008f, Sultan_2024f, Tammana_2012j, Tanrikulu_2015e, Trgo_2012f, Wadhwa_2015e, Wnęk_2015f, teWildt_2010}, 7 cases (10\%) were managed conservatively \cite{Ataya_2013, Bhattacharjee_2008, DivsalarP._2023a, Emamhadi_2018, Goldman_1998f, Kar_2015, Kumar_2001}, 2 cases (3\%) died \cite{Emamhadi_2018, Kumar_2001}. All 90 were male gender. 90 cases (100\%) were detained at the time of ingestion \cite{Elghali_2016, Karp_1991b, Lee_2007}, 88 cases (98\%) were intentional ingestions \cite{Elghali_2016, Karp_1991b, Lee_2007}, 30 cases (33\%) had a psychiatric history documented \cite{Elghali_2016, Karp_1991b, Lee_2007}, 2 cases (2\%) had a history of prior ingestion \cite{Elghali_2016}. No cases were reported for were psychiatric inpatients, were displaced people, were under the influence of alcohol at the time of ingestion, and had a severe disability history.
\paragraph*{Motivation}  70 cases (78\%) reported protest motivation \cite{Elghali_2016, Karp_1991b, Lee_2007}, 12 cases (13\%) reported psychiatric motivation \cite{Karp_1991b}, 6 cases (7\%) reported self-harm motivation \cite{Elghali_2016, Karp_1991b}. No cases were reported for psychosocial motivation and other motivation.
\paragraph*{Object Characteristics}  68 cases (76\%) involved sharp object ingestion \cite{Elghali_2016, Karp_1991b, Lee_2007}, 32 cases (36\%) involved long (\textgreater 5cm) object ingestion \cite{Lee_2007}, 25 cases (28\%) involved ingestion of multiple objects \cite{Elghali_2016, Lee_2007}. No cases were reported for button battery ingestion, magnet ingestion, and involved large diameter (\textgreater 2.5cm) object ingestion.
\paragraph*{Outcomes}  47 cases (52\%) underwent endoscopic intervention \cite{Elghali_2016, Lee_2007}, 29 cases (32\%) were managed conservatively \cite{Elghali_2016, Karp_1991b}, 15 cases (17\%) underwent surgical intervention \cite{Elghali_2016, Karp_1991b, Lee_2007}, 6 cases (7\%) reported complications \cite{Lee_2007}, 1 case (1\%) died \cite{Elghali_2016}.
\paragraph*{Geographical Location}Cases were recorded in 33 countries: 13 cases from USA \cite{Alao_2006i, Ataya_2013, Bhumi_2024f, Fry_2010, Guinan_2019f, Hardy_2023g, Jehangir_2019h, Kerestes_2019, Kumar_2001, Liu_2005, Tammana_2012j, Tay_2004, Tupesis_2004f}; 7 cases from India \cite{Bhasin_2014, Bhattacharjee_2008, Kar_2015, Kariholu_2008, Kumar_2019f, Misra_2013, Wadhwa_2015e} and UK \cite{Beecroft_1998, Berry_2021e, Cauchi_2002, Cox_2007, Gardner_2017h, Qureshi_2016}; 6 cases from Bulgaria \cite{Losanoff_1996, Losanoff_1997e}; 5 cases from Iran \cite{DivsalarP._2023a, Emamhadi_2018, Farhadi_2024h}; 4 cases from Turkey \cite{Akay_2015f, Atayan_2016, Tanrikulu_2015e, Yildiz_2016e}; 2 cases from China \cite{Jin_2023, Li_2013}, Poland \cite{Kobiela_2015, Wnęk_2015f}, and Spain \cite{CamachoDorado_2018, fjbuilsRepeatedBehaviorDeliberate2024}; 1 case from Australia \cite{Apikotoa_2022f}, Bahrain \cite{Ali_2020f}, Croatia \cite{Trgo_2012f}, Ecuador \cite{DelgadoSalazar_2020c}, Egypt \cite{Ali_2022g}, Ethiopia \cite{Mesfin_2022a}, Germany \cite{teWildt_2010}, Greece \cite{Sakellaridis_2008f}, Hungary \cite{Csaky_1998e}, Iraq \cite{Al-Faham_2020k}, Israel \cite{Goldman_1998f}, Italy \cite{Riva_2018j}, Japan \cite{Ohno_2005}, Nepal \cite{Thapa_2019f}, Netherlands \cite{Benoist_2019e}, Oman \cite{AlShaaibi_2021b}, Pakistan \cite{Yasin_2009}, Portugal \cite{Peixoto_2017f}, Qatar \cite{Ali_2017}, Saudi Arabia \cite{Sultan_2024f}, South Africa \cite{Sobnach_2011f}, Sweden \cite{Naji_2012f}, Switzerland \cite{Wildhaber_2005}, and Taiwan \cite{Chang_2017f}. \paragraph*{Gender} 43 cases (60\%) were male \cite{Akay_2015f, Al-Faham_2020k, Alao_2006i, Ali_2017, Ali_2022g, Apikotoa_2022f, Atayan_2016, Benoist_2019e, Berry_2021e, Bhumi_2024f, CamachoDorado_2018, Csaky_1998e, Emamhadi_2018, Farhadi_2024h, Fry_2010, Gardner_2017h, Guinan_2019f, Jehangir_2019h, Jin_2023, Kobiela_2015, Kumar_2001, Kumar_2019f, Liu_2005, Losanoff_1996, Losanoff_1997e, Mesfin_2022a, Misra_2013, Qureshi_2016, Riva_2018j, Sobnach_2011f, Tammana_2012j, Tanrikulu_2015e, Tay_2004, Thapa_2019f, Trgo_2012f, Wadhwa_2015e, Yasin_2009, teWildt_2010}, 28 cases (39\%) were female \cite{AlShaaibi_2021b, Ali_2020f, Ataya_2013, Beecroft_1998, Bhasin_2014, Bhattacharjee_2008, Cauchi_2002, Chang_2017f, Cox_2007, DelgadoSalazar_2020c, DivsalarP._2023a, Goldman_1998f, Hardy_2023g, Kar_2015, Kariholu_2008, Kerestes_2019, Li_2013, Naji_2012f, Ohno_2005, Peixoto_2017f, Sakellaridis_2008f, Sultan_2024f, Tupesis_2004f, Wildhaber_2005, Wnęk_2015f, Yildiz_2016e}, 1 case (1\%) had no gender recorded \cite{fjbuilsRepeatedBehaviorDeliberate2024}. \paragraph*{Age Group} 25 cases (35\%) were between 26 and 40 years of age \cite{Alao_2006i, Ali_2022g, Apikotoa_2022f, Ataya_2013, Benoist_2019e, Bhasin_2014, Chang_2017f, Cox_2007, DelgadoSalazar_2020c, Farhadi_2024h, Fry_2010, Gardner_2017h, Guinan_2019f, Jin_2023, Kumar_2019f, Losanoff_1996, Misra_2013, Qureshi_2016, Riva_2018j, Sakellaridis_2008f, Tammana_2012j, Trgo_2012f, Wnęk_2015f, Yildiz_2016e, fjbuilsRepeatedBehaviorDeliberate2024}, 18 cases (25\%) were between 18 and 25 years of age \cite{Akay_2015f, Ali_2017, Atayan_2016, Bhattacharjee_2008, Csaky_1998e, Kar_2015, Kariholu_2008, Kobiela_2015, Losanoff_1996, Losanoff_1997e, Mesfin_2022a, Peixoto_2017f, Sobnach_2011f, Tupesis_2004f, Yasin_2009}, 13 cases (18\%) were under 18 years of age \cite{AlShaaibi_2021b, Ali_2020f, Cauchi_2002, DivsalarP._2023a, Goldman_1998f, Liu_2005, Naji_2012f, Ohno_2005, Tanrikulu_2015e, Tay_2004, Wildhaber_2005}, 11 cases (15\%) were between 41 and 60 years of age \cite{Al-Faham_2020k, Bhumi_2024f, CamachoDorado_2018, Emamhadi_2018, Hardy_2023g, Jehangir_2019h, Kumar_2001, Sultan_2024f, Thapa_2019f, Wadhwa_2015e, teWildt_2010}, 3 cases (4\%) were over 60 years of age \cite{Beecroft_1998, Kerestes_2019, Li_2013}, 2 cases (3\%) had no age documented \cite{Berry_2021e}. \paragraph*{Population} 36 cases (50\%) had a psychiatric history \cite{AlShaaibi_2021b, Alao_2006i, Ali_2020f, Apikotoa_2022f, Ataya_2013, Atayan_2016, Beecroft_1998, CamachoDorado_2018, Chang_2017f, DelgadoSalazar_2020c, DivsalarP._2023a, Farhadi_2024h, Fry_2010, Guinan_2019f, Hardy_2023g, Jehangir_2019h, Jin_2023, Kar_2015, Kerestes_2019, Kobiela_2015, Kumar_2001, Kumar_2019f, Liu_2005, Mesfin_2022a, Misra_2013, Ohno_2005, Peixoto_2017f, Sakellaridis_2008f, Sultan_2024f, Tammana_2012j, Tanrikulu_2015e, Yildiz_2016e, fjbuilsRepeatedBehaviorDeliberate2024, teWildt_2010}, 19 cases (26\%) had ingested previously \cite{Alao_2006i, Apikotoa_2022f, Berry_2021e, Bhattacharjee_2008, Csaky_1998e, DivsalarP._2023a, Emamhadi_2018, Guinan_2019f, Jehangir_2019h, Jin_2023, Liu_2005, Sakellaridis_2008f, Tanrikulu_2015e, Thapa_2019f, Yildiz_2016e, fjbuilsRepeatedBehaviorDeliberate2024, teWildt_2010}, 12 cases (17\%) were detained persons \cite{Alao_2006i, Ali_2022g, Apikotoa_2022f, Losanoff_1996, Losanoff_1997e, Qureshi_2016, Tammana_2012j, Trgo_2012f}, 7 cases (10\%) were severely disabled \cite{Atayan_2016, Kerestes_2019, Liu_2005, Ohno_2005, Peixoto_2017f, Yildiz_2016e, teWildt_2010}, 4 cases (6\%) were psychiatric inpatients \cite{DivsalarP._2023a, fjbuilsRepeatedBehaviorDeliberate2024, teWildt_2010}, 3 cases (4\%) were under the influence of alcohol \cite{Benoist_2019e, Csaky_1998e, Thapa_2019f}, 2 cases (3\%) were displaced people \cite{Akay_2015f, Gardner_2017h}. \paragraph*{Motivation} 34 cases (47\%) had a psychiatric motivation \cite{Al-Faham_2020k, Alao_2006i, Ali_2020f, Apikotoa_2022f, Ataya_2013, Atayan_2016, Bhasin_2014, Bhattacharjee_2008, DelgadoSalazar_2020c, DivsalarP._2023a, Emamhadi_2018, Farhadi_2024h, Guinan_2019f, Hardy_2023g, Jehangir_2019h, Jin_2023, Kar_2015, Kariholu_2008, Kerestes_2019, Kobiela_2015, Kumar_2001, Kumar_2019f, Li_2013, Liu_2005, Misra_2013, Ohno_2005, Sakellaridis_2008f, Sultan_2024f, Tammana_2012j, Tanrikulu_2015e, Yasin_2009, teWildt_2010}, 21 cases (29\%) were motivated by self-harm intention \cite{Al-Faham_2020k, AlShaaibi_2021b, Alao_2006i, Ali_2017, CamachoDorado_2018, Chang_2017f, Cox_2007, Csaky_1998e, Fry_2010, Li_2013, Losanoff_1996, Losanoff_1997e, Mesfin_2022a, Sakellaridis_2008f, Tammana_2012j, Tanrikulu_2015e, fjbuilsRepeatedBehaviorDeliberate2024}, 17 cases (24\%) had a psychosocial motivation \cite{Akay_2015f, Benoist_2019e, Bhattacharjee_2008, Cauchi_2002, Goldman_1998f, Hardy_2023g, Kobiela_2015, Li_2013, Naji_2012f, Qureshi_2016, Riva_2018j, Sobnach_2011f, Tay_2004, Thapa_2019f, Tupesis_2004f, Wildhaber_2005, Wnęk_2015f}, 9 cases (12\%) were motivated by protest \cite{Bhumi_2024f, Gardner_2017h, Losanoff_1996, Losanoff_1997e, Tupesis_2004f}, 9 cases (12\%) had another documented motivation \cite{Ali_2020f, Ali_2022g, Emamhadi_2018, Guinan_2019f, Peixoto_2017f, Sakellaridis_2008f, Trgo_2012f, Wadhwa_2015e, Yildiz_2016e}. \paragraph*{Object Characteristics} 51 cases (71\%) ingested a large diameter object (\textgreater{}2.5cm) \cite{Akay_2015f, Al-Faham_2020k, AlShaaibi_2021b, Alao_2006i, Ali_2017, Ali_2022g, Apikotoa_2022f, Atayan_2016, Berry_2021e, Bhasin_2014, CamachoDorado_2018, Cauchi_2002, Chang_2017f, Cox_2007, Csaky_1998e, DivsalarP._2023a, Emamhadi_2018, Gardner_2017h, Guinan_2019f, Jehangir_2019h, Jin_2023, Kariholu_2008, Kerestes_2019, Kobiela_2015, Kumar_2001, Kumar_2019f, Losanoff_1996, Losanoff_1997e, Mesfin_2022a, Misra_2013, Naji_2012f, Ohno_2005, Peixoto_2017f, Qureshi_2016, Riva_2018j, Sakellaridis_2008f, Sultan_2024f, Tanrikulu_2015e, Thapa_2019f, Trgo_2012f, Wnęk_2015f, Yildiz_2016e, fjbuilsRepeatedBehaviorDeliberate2024, teWildt_2010}, 44 cases (61\%) ingested multiple objects \cite{Ali_2020f, Apikotoa_2022f, Ataya_2013, Atayan_2016, Beecroft_1998, Bhattacharjee_2008, Bhumi_2024f, CamachoDorado_2018, Cauchi_2002, Emamhadi_2018, Farhadi_2024h, Fry_2010, Goldman_1998f, Guinan_2019f, Hardy_2023g, Jehangir_2019h, Jin_2023, Kar_2015, Kariholu_2008, Kobiela_2015, Kumar_2001, Kumar_2019f, Li_2013, Liu_2005, Losanoff_1996, Mesfin_2022a, Misra_2013, Naji_2012f, Ohno_2005, Sobnach_2011f, Sultan_2024f, Tammana_2012j, Tanrikulu_2015e, Tay_2004, Thapa_2019f, Wadhwa_2015e, Wildhaber_2005, Yasin_2009, fjbuilsRepeatedBehaviorDeliberate2024, teWildt_2010}, 34 cases (47\%) ingested a sharp object \cite{AlShaaibi_2021b, Alao_2006i, Apikotoa_2022f, Ataya_2013, Benoist_2019e, Bhasin_2014, Bhattacharjee_2008, CamachoDorado_2018, Csaky_1998e, DelgadoSalazar_2020c, DivsalarP._2023a, Emamhadi_2018, Farhadi_2024h, Fry_2010, Guinan_2019f, Hardy_2023g, Jehangir_2019h, Jin_2023, Kariholu_2008, Kobiela_2015, Kumar_2019f, Losanoff_1996, Losanoff_1997e, Mesfin_2022a, Misra_2013, Sobnach_2011f, Yasin_2009, teWildt_2010}, 32 cases (44\%) ingested a long object (\textgreater{}5cm) \cite{Al-Faham_2020k, AlShaaibi_2021b, Ali_2017, Ali_2022g, Atayan_2016, Bhasin_2014, CamachoDorado_2018, Chang_2017f, Cox_2007, Csaky_1998e, DivsalarP._2023a, Emamhadi_2018, Fry_2010, Gardner_2017h, Jin_2023, Kariholu_2008, Kerestes_2019, Kobiela_2015, Kumar_2019f, Mesfin_2022a, Misra_2013, Ohno_2005, Qureshi_2016, Sakellaridis_2008f, Sultan_2024f, Thapa_2019f, Trgo_2012f, Yasin_2009, Yildiz_2016e, teWildt_2010}, 9 cases (12\%) ingested a magnet \cite{Ali_2020f, Bhumi_2024f, Cauchi_2002, Liu_2005, Naji_2012f, Ohno_2005, Tanrikulu_2015e, Tay_2004, Wildhaber_2005}, 2 cases (3\%) ingested a button battery \cite{Berry_2021e, Bhumi_2024f}. \paragraph*{Outcomes} 48 cases (67\%) experienced a complication \cite{Ali_2017, Ali_2020f, Apikotoa_2022f, Atayan_2016, Beecroft_1998, Benoist_2019e, Berry_2021e, Bhasin_2014, Bhumi_2024f, CamachoDorado_2018, Cauchi_2002, Cox_2007, Csaky_1998e, DelgadoSalazar_2020c, DivsalarP._2023a, Emamhadi_2018, Farhadi_2024h, Fry_2010, Gardner_2017h, Goldman_1998f, Jin_2023, Kariholu_2008, Kerestes_2019, Kobiela_2015, Kumar_2001, Kumar_2019f, Liu_2005, Losanoff_1996, Mesfin_2022a, Misra_2013, Naji_2012f, Ohno_2005, Sakellaridis_2008f, Sobnach_2011f, Sultan_2024f, Tanrikulu_2015e, Tay_2004, Thapa_2019f, Trgo_2012f, Tupesis_2004f, Wildhaber_2005, Wnęk_2015f, Yasin_2009, Yildiz_2016e}, 44 cases (61\%) underwent surgery \cite{Al-Faham_2020k, AlShaaibi_2021b, Alao_2006i, Ali_2017, Ali_2020f, Atayan_2016, Beecroft_1998, Bhasin_2014, CamachoDorado_2018, Cauchi_2002, Chang_2017f, Cox_2007, Csaky_1998e, DelgadoSalazar_2020c, DivsalarP._2023a, Farhadi_2024h, Fry_2010, Gardner_2017h, Jin_2023, Kariholu_2008, Kerestes_2019, Kobiela_2015, Kumar_2019f, Liu_2005, Losanoff_1996, Losanoff_1997e, Mesfin_2022a, Misra_2013, Naji_2012f, Sobnach_2011f, Tanrikulu_2015e, Tay_2004, Thapa_2019f, Tupesis_2004f, Wildhaber_2005, Wnęk_2015f, Yasin_2009, Yildiz_2016e, fjbuilsRepeatedBehaviorDeliberate2024}, 31 cases (43\%) underwent endoscopy \cite{Akay_2015f, Ali_2022g, Apikotoa_2022f, Atayan_2016, Benoist_2019e, Berry_2021e, Bhasin_2014, Bhumi_2024f, CamachoDorado_2018, Chang_2017f, DelgadoSalazar_2020c, Gardner_2017h, Guinan_2019f, Hardy_2023g, Jehangir_2019h, Kariholu_2008, Li_2013, Liu_2005, Ohno_2005, Peixoto_2017f, Qureshi_2016, Riva_2018j, Sakellaridis_2008f, Sultan_2024f, Tammana_2012j, Tanrikulu_2015e, Trgo_2012f, Wadhwa_2015e, Wnęk_2015f, teWildt_2010}, 7 cases (10\%) were managed conservatively \cite{Ataya_2013, Bhattacharjee_2008, DivsalarP._2023a, Emamhadi_2018, Goldman_1998f, Kar_2015, Kumar_2001}, 2 cases (3\%) died \cite{Emamhadi_2018, Kumar_2001}. All 90 were male gender. 90 cases (100\%) were detained at the time of ingestion \cite{Elghali_2016, Karp_1991b, Lee_2007}, 88 cases (98\%) were intentional ingestions \cite{Elghali_2016, Karp_1991b, Lee_2007}, 30 cases (33\%) had a psychiatric history documented \cite{Elghali_2016, Karp_1991b, Lee_2007}, 2 cases (2\%) had a history of prior ingestion \cite{Elghali_2016}. No cases were reported for were psychiatric inpatients, were displaced people, were under the influence of alcohol at the time of ingestion, and had a severe disability history.
\paragraph*{Motivation}  70 cases (78\%) reported protest motivation \cite{Elghali_2016, Karp_1991b, Lee_2007}, 12 cases (13\%) reported psychiatric motivation \cite{Karp_1991b}, 6 cases (7\%) reported self-harm motivation \cite{Elghali_2016, Karp_1991b}. No cases were reported for psychosocial motivation and other motivation.
\paragraph*{Object Characteristics}  68 cases (76\%) involved sharp object ingestion \cite{Elghali_2016, Karp_1991b, Lee_2007}, 32 cases (36\%) involved long (\textgreater 5cm) object ingestion \cite{Lee_2007}, 25 cases (28\%) involved ingestion of multiple objects \cite{Elghali_2016, Lee_2007}. No cases were reported for button battery ingestion, magnet ingestion, and involved large diameter (\textgreater 2.5cm) object ingestion.
\paragraph*{Outcomes}  47 cases (52\%) underwent endoscopic intervention \cite{Elghali_2016, Lee_2007}, 29 cases (32\%) were managed conservatively \cite{Elghali_2016, Karp_1991b}, 15 cases (17\%) underwent surgical intervention \cite{Elghali_2016, Karp_1991b, Lee_2007}, 6 cases (7\%) reported complications \cite{Lee_2007}, 1 case (1\%) died \cite{Elghali_2016}.
\paragraph*{Geographical Location}Cases were recorded in 33 countries: 13 cases from USA \cite{Alao_2006i, Ataya_2013, Bhumi_2024f, Fry_2010, Guinan_2019f, Hardy_2023g, Jehangir_2019h, Kerestes_2019, Kumar_2001, Liu_2005, Tammana_2012j, Tay_2004, Tupesis_2004f}; 7 cases from India \cite{Bhasin_2014, Bhattacharjee_2008, Kar_2015, Kariholu_2008, Kumar_2019f, Misra_2013, Wadhwa_2015e} and UK \cite{Beecroft_1998, Berry_2021e, Cauchi_2002, Cox_2007, Gardner_2017h, Qureshi_2016}; 6 cases from Bulgaria \cite{Losanoff_1996, Losanoff_1997e}; 5 cases from Iran \cite{DivsalarP._2023a, Emamhadi_2018, Farhadi_2024h}; 4 cases from Turkey \cite{Akay_2015f, Atayan_2016, Tanrikulu_2015e, Yildiz_2016e}; 2 cases from China \cite{Jin_2023, Li_2013}, Poland \cite{Kobiela_2015, Wnęk_2015f}, and Spain \cite{CamachoDorado_2018, fjbuilsRepeatedBehaviorDeliberate2024}; 1 case from Australia \cite{Apikotoa_2022f}, Bahrain \cite{Ali_2020f}, Croatia \cite{Trgo_2012f}, Ecuador \cite{DelgadoSalazar_2020c}, Egypt \cite{Ali_2022g}, Ethiopia \cite{Mesfin_2022a}, Germany \cite{teWildt_2010}, Greece \cite{Sakellaridis_2008f}, Hungary \cite{Csaky_1998e}, Iraq \cite{Al-Faham_2020k}, Israel \cite{Goldman_1998f}, Italy \cite{Riva_2018j}, Japan \cite{Ohno_2005}, Nepal \cite{Thapa_2019f}, Netherlands \cite{Benoist_2019e}, Oman \cite{AlShaaibi_2021b}, Pakistan \cite{Yasin_2009}, Portugal \cite{Peixoto_2017f}, Qatar \cite{Ali_2017}, Saudi Arabia \cite{Sultan_2024f}, South Africa \cite{Sobnach_2011f}, Sweden \cite{Naji_2012f}, Switzerland \cite{Wildhaber_2005}, and Taiwan \cite{Chang_2017f}. \paragraph*{Gender} 43 cases (60\%) were male \cite{Akay_2015f, Al-Faham_2020k, Alao_2006i, Ali_2017, Ali_2022g, Apikotoa_2022f, Atayan_2016, Benoist_2019e, Berry_2021e, Bhumi_2024f, CamachoDorado_2018, Csaky_1998e, Emamhadi_2018, Farhadi_2024h, Fry_2010, Gardner_2017h, Guinan_2019f, Jehangir_2019h, Jin_2023, Kobiela_2015, Kumar_2001, Kumar_2019f, Liu_2005, Losanoff_1996, Losanoff_1997e, Mesfin_2022a, Misra_2013, Qureshi_2016, Riva_2018j, Sobnach_2011f, Tammana_2012j, Tanrikulu_2015e, Tay_2004, Thapa_2019f, Trgo_2012f, Wadhwa_2015e, Yasin_2009, teWildt_2010}, 28 cases (39\%) were female \cite{AlShaaibi_2021b, Ali_2020f, Ataya_2013, Beecroft_1998, Bhasin_2014, Bhattacharjee_2008, Cauchi_2002, Chang_2017f, Cox_2007, DelgadoSalazar_2020c, DivsalarP._2023a, Goldman_1998f, Hardy_2023g, Kar_2015, Kariholu_2008, Kerestes_2019, Li_2013, Naji_2012f, Ohno_2005, Peixoto_2017f, Sakellaridis_2008f, Sultan_2024f, Tupesis_2004f, Wildhaber_2005, Wnęk_2015f, Yildiz_2016e}, 1 case (1\%) had no gender recorded \cite{fjbuilsRepeatedBehaviorDeliberate2024}. \paragraph*{Age Group} 25 cases (35\%) were between 26 and 40 years of age \cite{Alao_2006i, Ali_2022g, Apikotoa_2022f, Ataya_2013, Benoist_2019e, Bhasin_2014, Chang_2017f, Cox_2007, DelgadoSalazar_2020c, Farhadi_2024h, Fry_2010, Gardner_2017h, Guinan_2019f, Jin_2023, Kumar_2019f, Losanoff_1996, Misra_2013, Qureshi_2016, Riva_2018j, Sakellaridis_2008f, Tammana_2012j, Trgo_2012f, Wnęk_2015f, Yildiz_2016e, fjbuilsRepeatedBehaviorDeliberate2024}, 18 cases (25\%) were between 18 and 25 years of age \cite{Akay_2015f, Ali_2017, Atayan_2016, Bhattacharjee_2008, Csaky_1998e, Kar_2015, Kariholu_2008, Kobiela_2015, Losanoff_1996, Losanoff_1997e, Mesfin_2022a, Peixoto_2017f, Sobnach_2011f, Tupesis_2004f, Yasin_2009}, 13 cases (18\%) were under 18 years of age \cite{AlShaaibi_2021b, Ali_2020f, Cauchi_2002, DivsalarP._2023a, Goldman_1998f, Liu_2005, Naji_2012f, Ohno_2005, Tanrikulu_2015e, Tay_2004, Wildhaber_2005}, 11 cases (15\%) were between 41 and 60 years of age \cite{Al-Faham_2020k, Bhumi_2024f, CamachoDorado_2018, Emamhadi_2018, Hardy_2023g, Jehangir_2019h, Kumar_2001, Sultan_2024f, Thapa_2019f, Wadhwa_2015e, teWildt_2010}, 3 cases (4\%) were over 60 years of age \cite{Beecroft_1998, Kerestes_2019, Li_2013}, 2 cases (3\%) had no age documented \cite{Berry_2021e}. \paragraph*{Population} 36 cases (50\%) had a psychiatric history \cite{AlShaaibi_2021b, Alao_2006i, Ali_2020f, Apikotoa_2022f, Ataya_2013, Atayan_2016, Beecroft_1998, CamachoDorado_2018, Chang_2017f, DelgadoSalazar_2020c, DivsalarP._2023a, Farhadi_2024h, Fry_2010, Guinan_2019f, Hardy_2023g, Jehangir_2019h, Jin_2023, Kar_2015, Kerestes_2019, Kobiela_2015, Kumar_2001, Kumar_2019f, Liu_2005, Mesfin_2022a, Misra_2013, Ohno_2005, Peixoto_2017f, Sakellaridis_2008f, Sultan_2024f, Tammana_2012j, Tanrikulu_2015e, Yildiz_2016e, fjbuilsRepeatedBehaviorDeliberate2024, teWildt_2010}, 19 cases (26\%) had ingested previously \cite{Alao_2006i, Apikotoa_2022f, Berry_2021e, Bhattacharjee_2008, Csaky_1998e, DivsalarP._2023a, Emamhadi_2018, Guinan_2019f, Jehangir_2019h, Jin_2023, Liu_2005, Sakellaridis_2008f, Tanrikulu_2015e, Thapa_2019f, Yildiz_2016e, fjbuilsRepeatedBehaviorDeliberate2024, teWildt_2010}, 12 cases (17\%) were detained persons \cite{Alao_2006i, Ali_2022g, Apikotoa_2022f, Losanoff_1996, Losanoff_1997e, Qureshi_2016, Tammana_2012j, Trgo_2012f}, 7 cases (10\%) were severely disabled \cite{Atayan_2016, Kerestes_2019, Liu_2005, Ohno_2005, Peixoto_2017f, Yildiz_2016e, teWildt_2010}, 4 cases (6\%) were psychiatric inpatients \cite{DivsalarP._2023a, fjbuilsRepeatedBehaviorDeliberate2024, teWildt_2010}, 3 cases (4\%) were under the influence of alcohol \cite{Benoist_2019e, Csaky_1998e, Thapa_2019f}, 2 cases (3\%) were displaced people \cite{Akay_2015f, Gardner_2017h}. \paragraph*{Motivation} 34 cases (47\%) had a psychiatric motivation \cite{Al-Faham_2020k, Alao_2006i, Ali_2020f, Apikotoa_2022f, Ataya_2013, Atayan_2016, Bhasin_2014, Bhattacharjee_2008, DelgadoSalazar_2020c, DivsalarP._2023a, Emamhadi_2018, Farhadi_2024h, Guinan_2019f, Hardy_2023g, Jehangir_2019h, Jin_2023, Kar_2015, Kariholu_2008, Kerestes_2019, Kobiela_2015, Kumar_2001, Kumar_2019f, Li_2013, Liu_2005, Misra_2013, Ohno_2005, Sakellaridis_2008f, Sultan_2024f, Tammana_2012j, Tanrikulu_2015e, Yasin_2009, teWildt_2010}, 21 cases (29\%) were motivated by self-harm intention \cite{Al-Faham_2020k, AlShaaibi_2021b, Alao_2006i, Ali_2017, CamachoDorado_2018, Chang_2017f, Cox_2007, Csaky_1998e, Fry_2010, Li_2013, Losanoff_1996, Losanoff_1997e, Mesfin_2022a, Sakellaridis_2008f, Tammana_2012j, Tanrikulu_2015e, fjbuilsRepeatedBehaviorDeliberate2024}, 17 cases (24\%) had a psychosocial motivation \cite{Akay_2015f, Benoist_2019e, Bhattacharjee_2008, Cauchi_2002, Goldman_1998f, Hardy_2023g, Kobiela_2015, Li_2013, Naji_2012f, Qureshi_2016, Riva_2018j, Sobnach_2011f, Tay_2004, Thapa_2019f, Tupesis_2004f, Wildhaber_2005, Wnęk_2015f}, 9 cases (12\%) were motivated by protest \cite{Bhumi_2024f, Gardner_2017h, Losanoff_1996, Losanoff_1997e, Tupesis_2004f}, 9 cases (12\%) had another documented motivation \cite{Ali_2020f, Ali_2022g, Emamhadi_2018, Guinan_2019f, Peixoto_2017f, Sakellaridis_2008f, Trgo_2012f, Wadhwa_2015e, Yildiz_2016e}. \paragraph*{Object Characteristics} 51 cases (71\%) ingested a large diameter object (\textgreater{}2.5cm) \cite{Akay_2015f, Al-Faham_2020k, AlShaaibi_2021b, Alao_2006i, Ali_2017, Ali_2022g, Apikotoa_2022f, Atayan_2016, Berry_2021e, Bhasin_2014, CamachoDorado_2018, Cauchi_2002, Chang_2017f, Cox_2007, Csaky_1998e, DivsalarP._2023a, Emamhadi_2018, Gardner_2017h, Guinan_2019f, Jehangir_2019h, Jin_2023, Kariholu_2008, Kerestes_2019, Kobiela_2015, Kumar_2001, Kumar_2019f, Losanoff_1996, Losanoff_1997e, Mesfin_2022a, Misra_2013, Naji_2012f, Ohno_2005, Peixoto_2017f, Qureshi_2016, Riva_2018j, Sakellaridis_2008f, Sultan_2024f, Tanrikulu_2015e, Thapa_2019f, Trgo_2012f, Wnęk_2015f, Yildiz_2016e, fjbuilsRepeatedBehaviorDeliberate2024, teWildt_2010}, 44 cases (61\%) ingested multiple objects \cite{Ali_2020f, Apikotoa_2022f, Ataya_2013, Atayan_2016, Beecroft_1998, Bhattacharjee_2008, Bhumi_2024f, CamachoDorado_2018, Cauchi_2002, Emamhadi_2018, Farhadi_2024h, Fry_2010, Goldman_1998f, Guinan_2019f, Hardy_2023g, Jehangir_2019h, Jin_2023, Kar_2015, Kariholu_2008, Kobiela_2015, Kumar_2001, Kumar_2019f, Li_2013, Liu_2005, Losanoff_1996, Mesfin_2022a, Misra_2013, Naji_2012f, Ohno_2005, Sobnach_2011f, Sultan_2024f, Tammana_2012j, Tanrikulu_2015e, Tay_2004, Thapa_2019f, Wadhwa_2015e, Wildhaber_2005, Yasin_2009, fjbuilsRepeatedBehaviorDeliberate2024, teWildt_2010}, 34 cases (47\%) ingested a sharp object \cite{AlShaaibi_2021b, Alao_2006i, Apikotoa_2022f, Ataya_2013, Benoist_2019e, Bhasin_2014, Bhattacharjee_2008, CamachoDorado_2018, Csaky_1998e, DelgadoSalazar_2020c, DivsalarP._2023a, Emamhadi_2018, Farhadi_2024h, Fry_2010, Guinan_2019f, Hardy_2023g, Jehangir_2019h, Jin_2023, Kariholu_2008, Kobiela_2015, Kumar_2019f, Losanoff_1996, Losanoff_1997e, Mesfin_2022a, Misra_2013, Sobnach_2011f, Yasin_2009, teWildt_2010}, 32 cases (44\%) ingested a long object (\textgreater{}5cm) \cite{Al-Faham_2020k, AlShaaibi_2021b, Ali_2017, Ali_2022g, Atayan_2016, Bhasin_2014, CamachoDorado_2018, Chang_2017f, Cox_2007, Csaky_1998e, DivsalarP._2023a, Emamhadi_2018, Fry_2010, Gardner_2017h, Jin_2023, Kariholu_2008, Kerestes_2019, Kobiela_2015, Kumar_2019f, Mesfin_2022a, Misra_2013, Ohno_2005, Qureshi_2016, Sakellaridis_2008f, Sultan_2024f, Thapa_2019f, Trgo_2012f, Yasin_2009, Yildiz_2016e, teWildt_2010}, 9 cases (12\%) ingested a magnet \cite{Ali_2020f, Bhumi_2024f, Cauchi_2002, Liu_2005, Naji_2012f, Ohno_2005, Tanrikulu_2015e, Tay_2004, Wildhaber_2005}, 2 cases (3\%) ingested a button battery \cite{Berry_2021e, Bhumi_2024f}. \paragraph*{Outcomes} 48 cases (67\%) experienced a complication \cite{Ali_2017, Ali_2020f, Apikotoa_2022f, Atayan_2016, Beecroft_1998, Benoist_2019e, Berry_2021e, Bhasin_2014, Bhumi_2024f, CamachoDorado_2018, Cauchi_2002, Cox_2007, Csaky_1998e, DelgadoSalazar_2020c, DivsalarP._2023a, Emamhadi_2018, Farhadi_2024h, Fry_2010, Gardner_2017h, Goldman_1998f, Jin_2023, Kariholu_2008, Kerestes_2019, Kobiela_2015, Kumar_2001, Kumar_2019f, Liu_2005, Losanoff_1996, Mesfin_2022a, Misra_2013, Naji_2012f, Ohno_2005, Sakellaridis_2008f, Sobnach_2011f, Sultan_2024f, Tanrikulu_2015e, Tay_2004, Thapa_2019f, Trgo_2012f, Tupesis_2004f, Wildhaber_2005, Wnęk_2015f, Yasin_2009, Yildiz_2016e}, 44 cases (61\%) underwent surgery \cite{Al-Faham_2020k, AlShaaibi_2021b, Alao_2006i, Ali_2017, Ali_2020f, Atayan_2016, Beecroft_1998, Bhasin_2014, CamachoDorado_2018, Cauchi_2002, Chang_2017f, Cox_2007, Csaky_1998e, DelgadoSalazar_2020c, DivsalarP._2023a, Farhadi_2024h, Fry_2010, Gardner_2017h, Jin_2023, Kariholu_2008, Kerestes_2019, Kobiela_2015, Kumar_2019f, Liu_2005, Losanoff_1996, Losanoff_1997e, Mesfin_2022a, Misra_2013, Naji_2012f, Sobnach_2011f, Tanrikulu_2015e, Tay_2004, Thapa_2019f, Tupesis_2004f, Wildhaber_2005, Wnęk_2015f, Yasin_2009, Yildiz_2016e, fjbuilsRepeatedBehaviorDeliberate2024}, 31 cases (43\%) underwent endoscopy \cite{Akay_2015f, Ali_2022g, Apikotoa_2022f, Atayan_2016, Benoist_2019e, Berry_2021e, Bhasin_2014, Bhumi_2024f, CamachoDorado_2018, Chang_2017f, DelgadoSalazar_2020c, Gardner_2017h, Guinan_2019f, Hardy_2023g, Jehangir_2019h, Kariholu_2008, Li_2013, Liu_2005, Ohno_2005, Peixoto_2017f, Qureshi_2016, Riva_2018j, Sakellaridis_2008f, Sultan_2024f, Tammana_2012j, Tanrikulu_2015e, Trgo_2012f, Wadhwa_2015e, Wnęk_2015f, teWildt_2010}, 7 cases (10\%) were managed conservatively \cite{Ataya_2013, Bhattacharjee_2008, DivsalarP._2023a, Emamhadi_2018, Goldman_1998f, Kar_2015, Kumar_2001}, 2 cases (3\%) died \cite{Emamhadi_2018, Kumar_2001}. All 90 were male gender. 90 cases (100\%) were detained at the time of ingestion \cite{Elghali_2016, Karp_1991b, Lee_2007}, 88 cases (98\%) were intentional ingestions \cite{Elghali_2016, Karp_1991b, Lee_2007}, 30 cases (33\%) had a psychiatric history documented \cite{Elghali_2016, Karp_1991b, Lee_2007}, 2 cases (2\%) had a history of prior ingestion \cite{Elghali_2016}. No cases were reported for were psychiatric inpatients, were displaced people, were under the influence of alcohol at the time of ingestion, and had a severe disability history.
\paragraph*{Motivation}  70 cases (78\%) reported protest motivation \cite{Elghali_2016, Karp_1991b, Lee_2007}, 12 cases (13\%) reported psychiatric motivation \cite{Karp_1991b}, 6 cases (7\%) reported self-harm motivation \cite{Elghali_2016, Karp_1991b}. No cases were reported for psychosocial motivation and other motivation.
\paragraph*{Object Characteristics}  68 cases (76\%) involved sharp object ingestion \cite{Elghali_2016, Karp_1991b, Lee_2007}, 32 cases (36\%) involved long (\textgreater 5cm) object ingestion \cite{Lee_2007}, 25 cases (28\%) involved ingestion of multiple objects \cite{Elghali_2016, Lee_2007}. No cases were reported for button battery ingestion, magnet ingestion, and involved large diameter (\textgreater 2.5cm) object ingestion.
\paragraph*{Outcomes}  47 cases (52\%) underwent endoscopic intervention \cite{Elghali_2016, Lee_2007}, 29 cases (32\%) were managed conservatively \cite{Elghali_2016, Karp_1991b}, 15 cases (17\%) underwent surgical intervention \cite{Elghali_2016, Karp_1991b, Lee_2007}, 6 cases (7\%) reported complications \cite{Lee_2007}, 1 case (1\%) died \cite{Elghali_2016}.
\paragraph*{Geographical Location}Cases were recorded in 33 countries: 13 cases from USA \cite{Alao_2006i, Ataya_2013, Bhumi_2024f, Fry_2010, Guinan_2019f, Hardy_2023g, Jehangir_2019h, Kerestes_2019, Kumar_2001, Liu_2005, Tammana_2012j, Tay_2004, Tupesis_2004f}; 7 cases from India \cite{Bhasin_2014, Bhattacharjee_2008, Kar_2015, Kariholu_2008, Kumar_2019f, Misra_2013, Wadhwa_2015e} and UK \cite{Beecroft_1998, Berry_2021e, Cauchi_2002, Cox_2007, Gardner_2017h, Qureshi_2016}; 6 cases from Bulgaria \cite{Losanoff_1996, Losanoff_1997e}; 5 cases from Iran \cite{DivsalarP._2023a, Emamhadi_2018, Farhadi_2024h}; 4 cases from Turkey \cite{Akay_2015f, Atayan_2016, Tanrikulu_2015e, Yildiz_2016e}; 2 cases from China \cite{Jin_2023, Li_2013}, Poland \cite{Kobiela_2015, Wnęk_2015f}, and Spain \cite{CamachoDorado_2018, fjbuilsRepeatedBehaviorDeliberate2024}; 1 case from Australia \cite{Apikotoa_2022f}, Bahrain \cite{Ali_2020f}, Croatia \cite{Trgo_2012f}, Ecuador \cite{DelgadoSalazar_2020c}, Egypt \cite{Ali_2022g}, Ethiopia \cite{Mesfin_2022a}, Germany \cite{teWildt_2010}, Greece \cite{Sakellaridis_2008f}, Hungary \cite{Csaky_1998e}, Iraq \cite{Al-Faham_2020k}, Israel \cite{Goldman_1998f}, Italy \cite{Riva_2018j}, Japan \cite{Ohno_2005}, Nepal \cite{Thapa_2019f}, Netherlands \cite{Benoist_2019e}, Oman \cite{AlShaaibi_2021b}, Pakistan \cite{Yasin_2009}, Portugal \cite{Peixoto_2017f}, Qatar \cite{Ali_2017}, Saudi Arabia \cite{Sultan_2024f}, South Africa \cite{Sobnach_2011f}, Sweden \cite{Naji_2012f}, Switzerland \cite{Wildhaber_2005}, and Taiwan \cite{Chang_2017f}. \paragraph*{Gender} 43 cases (60\%) were male \cite{Akay_2015f, Al-Faham_2020k, Alao_2006i, Ali_2017, Ali_2022g, Apikotoa_2022f, Atayan_2016, Benoist_2019e, Berry_2021e, Bhumi_2024f, CamachoDorado_2018, Csaky_1998e, Emamhadi_2018, Farhadi_2024h, Fry_2010, Gardner_2017h, Guinan_2019f, Jehangir_2019h, Jin_2023, Kobiela_2015, Kumar_2001, Kumar_2019f, Liu_2005, Losanoff_1996, Losanoff_1997e, Mesfin_2022a, Misra_2013, Qureshi_2016, Riva_2018j, Sobnach_2011f, Tammana_2012j, Tanrikulu_2015e, Tay_2004, Thapa_2019f, Trgo_2012f, Wadhwa_2015e, Yasin_2009, teWildt_2010}, 28 cases (39\%) were female \cite{AlShaaibi_2021b, Ali_2020f, Ataya_2013, Beecroft_1998, Bhasin_2014, Bhattacharjee_2008, Cauchi_2002, Chang_2017f, Cox_2007, DelgadoSalazar_2020c, DivsalarP._2023a, Goldman_1998f, Hardy_2023g, Kar_2015, Kariholu_2008, Kerestes_2019, Li_2013, Naji_2012f, Ohno_2005, Peixoto_2017f, Sakellaridis_2008f, Sultan_2024f, Tupesis_2004f, Wildhaber_2005, Wnęk_2015f, Yildiz_2016e}, 1 case (1\%) had no gender recorded \cite{fjbuilsRepeatedBehaviorDeliberate2024}. \paragraph*{Age Group} 25 cases (35\%) were between 26 and 40 years of age \cite{Alao_2006i, Ali_2022g, Apikotoa_2022f, Ataya_2013, Benoist_2019e, Bhasin_2014, Chang_2017f, Cox_2007, DelgadoSalazar_2020c, Farhadi_2024h, Fry_2010, Gardner_2017h, Guinan_2019f, Jin_2023, Kumar_2019f, Losanoff_1996, Misra_2013, Qureshi_2016, Riva_2018j, Sakellaridis_2008f, Tammana_2012j, Trgo_2012f, Wnęk_2015f, Yildiz_2016e, fjbuilsRepeatedBehaviorDeliberate2024}, 18 cases (25\%) were between 18 and 25 years of age \cite{Akay_2015f, Ali_2017, Atayan_2016, Bhattacharjee_2008, Csaky_1998e, Kar_2015, Kariholu_2008, Kobiela_2015, Losanoff_1996, Losanoff_1997e, Mesfin_2022a, Peixoto_2017f, Sobnach_2011f, Tupesis_2004f, Yasin_2009}, 13 cases (18\%) were under 18 years of age \cite{AlShaaibi_2021b, Ali_2020f, Cauchi_2002, DivsalarP._2023a, Goldman_1998f, Liu_2005, Naji_2012f, Ohno_2005, Tanrikulu_2015e, Tay_2004, Wildhaber_2005}, 11 cases (15\%) were between 41 and 60 years of age \cite{Al-Faham_2020k, Bhumi_2024f, CamachoDorado_2018, Emamhadi_2018, Hardy_2023g, Jehangir_2019h, Kumar_2001, Sultan_2024f, Thapa_2019f, Wadhwa_2015e, teWildt_2010}, 3 cases (4\%) were over 60 years of age \cite{Beecroft_1998, Kerestes_2019, Li_2013}, 2 cases (3\%) had no age documented \cite{Berry_2021e}. \paragraph*{Population} 36 cases (50\%) had a psychiatric history \cite{AlShaaibi_2021b, Alao_2006i, Ali_2020f, Apikotoa_2022f, Ataya_2013, Atayan_2016, Beecroft_1998, CamachoDorado_2018, Chang_2017f, DelgadoSalazar_2020c, DivsalarP._2023a, Farhadi_2024h, Fry_2010, Guinan_2019f, Hardy_2023g, Jehangir_2019h, Jin_2023, Kar_2015, Kerestes_2019, Kobiela_2015, Kumar_2001, Kumar_2019f, Liu_2005, Mesfin_2022a, Misra_2013, Ohno_2005, Peixoto_2017f, Sakellaridis_2008f, Sultan_2024f, Tammana_2012j, Tanrikulu_2015e, Yildiz_2016e, fjbuilsRepeatedBehaviorDeliberate2024, teWildt_2010}, 19 cases (26\%) had ingested previously \cite{Alao_2006i, Apikotoa_2022f, Berry_2021e, Bhattacharjee_2008, Csaky_1998e, DivsalarP._2023a, Emamhadi_2018, Guinan_2019f, Jehangir_2019h, Jin_2023, Liu_2005, Sakellaridis_2008f, Tanrikulu_2015e, Thapa_2019f, Yildiz_2016e, fjbuilsRepeatedBehaviorDeliberate2024, teWildt_2010}, 12 cases (17\%) were detained persons \cite{Alao_2006i, Ali_2022g, Apikotoa_2022f, Losanoff_1996, Losanoff_1997e, Qureshi_2016, Tammana_2012j, Trgo_2012f}, 7 cases (10\%) were severely disabled \cite{Atayan_2016, Kerestes_2019, Liu_2005, Ohno_2005, Peixoto_2017f, Yildiz_2016e, teWildt_2010}, 4 cases (6\%) were psychiatric inpatients \cite{DivsalarP._2023a, fjbuilsRepeatedBehaviorDeliberate2024, teWildt_2010}, 3 cases (4\%) were under the influence of alcohol \cite{Benoist_2019e, Csaky_1998e, Thapa_2019f}, 2 cases (3\%) were displaced people \cite{Akay_2015f, Gardner_2017h}. \paragraph*{Motivation} 34 cases (47\%) had a psychiatric motivation \cite{Al-Faham_2020k, Alao_2006i, Ali_2020f, Apikotoa_2022f, Ataya_2013, Atayan_2016, Bhasin_2014, Bhattacharjee_2008, DelgadoSalazar_2020c, DivsalarP._2023a, Emamhadi_2018, Farhadi_2024h, Guinan_2019f, Hardy_2023g, Jehangir_2019h, Jin_2023, Kar_2015, Kariholu_2008, Kerestes_2019, Kobiela_2015, Kumar_2001, Kumar_2019f, Li_2013, Liu_2005, Misra_2013, Ohno_2005, Sakellaridis_2008f, Sultan_2024f, Tammana_2012j, Tanrikulu_2015e, Yasin_2009, teWildt_2010}, 21 cases (29\%) were motivated by self-harm intention \cite{Al-Faham_2020k, AlShaaibi_2021b, Alao_2006i, Ali_2017, CamachoDorado_2018, Chang_2017f, Cox_2007, Csaky_1998e, Fry_2010, Li_2013, Losanoff_1996, Losanoff_1997e, Mesfin_2022a, Sakellaridis_2008f, Tammana_2012j, Tanrikulu_2015e, fjbuilsRepeatedBehaviorDeliberate2024}, 17 cases (24\%) had a psychosocial motivation \cite{Akay_2015f, Benoist_2019e, Bhattacharjee_2008, Cauchi_2002, Goldman_1998f, Hardy_2023g, Kobiela_2015, Li_2013, Naji_2012f, Qureshi_2016, Riva_2018j, Sobnach_2011f, Tay_2004, Thapa_2019f, Tupesis_2004f, Wildhaber_2005, Wnęk_2015f}, 9 cases (12\%) were motivated by protest \cite{Bhumi_2024f, Gardner_2017h, Losanoff_1996, Losanoff_1997e, Tupesis_2004f}, 9 cases (12\%) had another documented motivation \cite{Ali_2020f, Ali_2022g, Emamhadi_2018, Guinan_2019f, Peixoto_2017f, Sakellaridis_2008f, Trgo_2012f, Wadhwa_2015e, Yildiz_2016e}. \paragraph*{Object Characteristics} 51 cases (71\%) ingested a large diameter object (\textgreater{}2.5cm) \cite{Akay_2015f, Al-Faham_2020k, AlShaaibi_2021b, Alao_2006i, Ali_2017, Ali_2022g, Apikotoa_2022f, Atayan_2016, Berry_2021e, Bhasin_2014, CamachoDorado_2018, Cauchi_2002, Chang_2017f, Cox_2007, Csaky_1998e, DivsalarP._2023a, Emamhadi_2018, Gardner_2017h, Guinan_2019f, Jehangir_2019h, Jin_2023, Kariholu_2008, Kerestes_2019, Kobiela_2015, Kumar_2001, Kumar_2019f, Losanoff_1996, Losanoff_1997e, Mesfin_2022a, Misra_2013, Naji_2012f, Ohno_2005, Peixoto_2017f, Qureshi_2016, Riva_2018j, Sakellaridis_2008f, Sultan_2024f, Tanrikulu_2015e, Thapa_2019f, Trgo_2012f, Wnęk_2015f, Yildiz_2016e, fjbuilsRepeatedBehaviorDeliberate2024, teWildt_2010}, 44 cases (61\%) ingested multiple objects \cite{Ali_2020f, Apikotoa_2022f, Ataya_2013, Atayan_2016, Beecroft_1998, Bhattacharjee_2008, Bhumi_2024f, CamachoDorado_2018, Cauchi_2002, Emamhadi_2018, Farhadi_2024h, Fry_2010, Goldman_1998f, Guinan_2019f, Hardy_2023g, Jehangir_2019h, Jin_2023, Kar_2015, Kariholu_2008, Kobiela_2015, Kumar_2001, Kumar_2019f, Li_2013, Liu_2005, Losanoff_1996, Mesfin_2022a, Misra_2013, Naji_2012f, Ohno_2005, Sobnach_2011f, Sultan_2024f, Tammana_2012j, Tanrikulu_2015e, Tay_2004, Thapa_2019f, Wadhwa_2015e, Wildhaber_2005, Yasin_2009, fjbuilsRepeatedBehaviorDeliberate2024, teWildt_2010}, 34 cases (47\%) ingested a sharp object \cite{AlShaaibi_2021b, Alao_2006i, Apikotoa_2022f, Ataya_2013, Benoist_2019e, Bhasin_2014, Bhattacharjee_2008, CamachoDorado_2018, Csaky_1998e, DelgadoSalazar_2020c, DivsalarP._2023a, Emamhadi_2018, Farhadi_2024h, Fry_2010, Guinan_2019f, Hardy_2023g, Jehangir_2019h, Jin_2023, Kariholu_2008, Kobiela_2015, Kumar_2019f, Losanoff_1996, Losanoff_1997e, Mesfin_2022a, Misra_2013, Sobnach_2011f, Yasin_2009, teWildt_2010}, 32 cases (44\%) ingested a long object (\textgreater{}5cm) \cite{Al-Faham_2020k, AlShaaibi_2021b, Ali_2017, Ali_2022g, Atayan_2016, Bhasin_2014, CamachoDorado_2018, Chang_2017f, Cox_2007, Csaky_1998e, DivsalarP._2023a, Emamhadi_2018, Fry_2010, Gardner_2017h, Jin_2023, Kariholu_2008, Kerestes_2019, Kobiela_2015, Kumar_2019f, Mesfin_2022a, Misra_2013, Ohno_2005, Qureshi_2016, Sakellaridis_2008f, Sultan_2024f, Thapa_2019f, Trgo_2012f, Yasin_2009, Yildiz_2016e, teWildt_2010}, 9 cases (12\%) ingested a magnet \cite{Ali_2020f, Bhumi_2024f, Cauchi_2002, Liu_2005, Naji_2012f, Ohno_2005, Tanrikulu_2015e, Tay_2004, Wildhaber_2005}, 2 cases (3\%) ingested a button battery \cite{Berry_2021e, Bhumi_2024f}. \paragraph*{Outcomes} 48 cases (67\%) experienced a complication \cite{Ali_2017, Ali_2020f, Apikotoa_2022f, Atayan_2016, Beecroft_1998, Benoist_2019e, Berry_2021e, Bhasin_2014, Bhumi_2024f, CamachoDorado_2018, Cauchi_2002, Cox_2007, Csaky_1998e, DelgadoSalazar_2020c, DivsalarP._2023a, Emamhadi_2018, Farhadi_2024h, Fry_2010, Gardner_2017h, Goldman_1998f, Jin_2023, Kariholu_2008, Kerestes_2019, Kobiela_2015, Kumar_2001, Kumar_2019f, Liu_2005, Losanoff_1996, Mesfin_2022a, Misra_2013, Naji_2012f, Ohno_2005, Sakellaridis_2008f, Sobnach_2011f, Sultan_2024f, Tanrikulu_2015e, Tay_2004, Thapa_2019f, Trgo_2012f, Tupesis_2004f, Wildhaber_2005, Wnęk_2015f, Yasin_2009, Yildiz_2016e}, 44 cases (61\%) underwent surgery \cite{Al-Faham_2020k, AlShaaibi_2021b, Alao_2006i, Ali_2017, Ali_2020f, Atayan_2016, Beecroft_1998, Bhasin_2014, CamachoDorado_2018, Cauchi_2002, Chang_2017f, Cox_2007, Csaky_1998e, DelgadoSalazar_2020c, DivsalarP._2023a, Farhadi_2024h, Fry_2010, Gardner_2017h, Jin_2023, Kariholu_2008, Kerestes_2019, Kobiela_2015, Kumar_2019f, Liu_2005, Losanoff_1996, Losanoff_1997e, Mesfin_2022a, Misra_2013, Naji_2012f, Sobnach_2011f, Tanrikulu_2015e, Tay_2004, Thapa_2019f, Tupesis_2004f, Wildhaber_2005, Wnęk_2015f, Yasin_2009, Yildiz_2016e, fjbuilsRepeatedBehaviorDeliberate2024}, 31 cases (43\%) underwent endoscopy \cite{Akay_2015f, Ali_2022g, Apikotoa_2022f, Atayan_2016, Benoist_2019e, Berry_2021e, Bhasin_2014, Bhumi_2024f, CamachoDorado_2018, Chang_2017f, DelgadoSalazar_2020c, Gardner_2017h, Guinan_2019f, Hardy_2023g, Jehangir_2019h, Kariholu_2008, Li_2013, Liu_2005, Ohno_2005, Peixoto_2017f, Qureshi_2016, Riva_2018j, Sakellaridis_2008f, Sultan_2024f, Tammana_2012j, Tanrikulu_2015e, Trgo_2012f, Wadhwa_2015e, Wnęk_2015f, teWildt_2010}, 7 cases (10\%) were managed conservatively \cite{Ataya_2013, Bhattacharjee_2008, DivsalarP._2023a, Emamhadi_2018, Goldman_1998f, Kar_2015, Kumar_2001}, 2 cases (3\%) died \cite{Emamhadi_2018, Kumar_2001}. All 90 were male gender. 90 cases (100\%) were detained at the time of ingestion \cite{Elghali_2016, Karp_1991b, Lee_2007}, 88 cases (98\%) were intentional ingestions \cite{Elghali_2016, Karp_1991b, Lee_2007}, 30 cases (33\%) had a psychiatric history documented \cite{Elghali_2016, Karp_1991b, Lee_2007}, 2 cases (2\%) had a history of prior ingestion \cite{Elghali_2016}. No cases were reported for were psychiatric inpatients, were displaced people, were under the influence of alcohol at the time of ingestion, and had a severe disability history.
\paragraph*{Motivation}  70 cases (78\%) reported protest motivation \cite{Elghali_2016, Karp_1991b, Lee_2007}, 12 cases (13\%) reported psychiatric motivation \cite{Karp_1991b}, 6 cases (7\%) reported self-harm motivation \cite{Elghali_2016, Karp_1991b}. No cases were reported for psychosocial motivation and other motivation.
\paragraph*{Object Characteristics}  68 cases (76\%) involved sharp object ingestion \cite{Elghali_2016, Karp_1991b, Lee_2007}, 32 cases (36\%) involved long (\textgreater 5cm) object ingestion \cite{Lee_2007}, 25 cases (28\%) involved ingestion of multiple objects \cite{Elghali_2016, Lee_2007}. No cases were reported for button battery ingestion, magnet ingestion, and involved large diameter (\textgreater 2.5cm) object ingestion.
\paragraph*{Outcomes}  47 cases (52\%) underwent endoscopic intervention \cite{Elghali_2016, Lee_2007}, 29 cases (32\%) were managed conservatively \cite{Elghali_2016, Karp_1991b}, 15 cases (17\%) underwent surgical intervention \cite{Elghali_2016, Karp_1991b, Lee_2007}, 6 cases (7\%) reported complications \cite{Lee_2007}, 1 case (1\%) died \cite{Elghali_2016}.
\paragraph*{Geographical Location}Cases were recorded in 33 countries: 13 cases from USA \cite{Alao_2006i, Ataya_2013, Bhumi_2024f, Fry_2010, Guinan_2019f, Hardy_2023g, Jehangir_2019h, Kerestes_2019, Kumar_2001, Liu_2005, Tammana_2012j, Tay_2004, Tupesis_2004f}; 7 cases from India \cite{Bhasin_2014, Bhattacharjee_2008, Kar_2015, Kariholu_2008, Kumar_2019f, Misra_2013, Wadhwa_2015e} and UK \cite{Beecroft_1998, Berry_2021e, Cauchi_2002, Cox_2007, Gardner_2017h, Qureshi_2016}; 6 cases from Bulgaria \cite{Losanoff_1996, Losanoff_1997e}; 5 cases from Iran \cite{DivsalarP._2023a, Emamhadi_2018, Farhadi_2024h}; 4 cases from Turkey \cite{Akay_2015f, Atayan_2016, Tanrikulu_2015e, Yildiz_2016e}; 2 cases from China \cite{Jin_2023, Li_2013}, Poland \cite{Kobiela_2015, Wnęk_2015f}, and Spain \cite{CamachoDorado_2018, fjbuilsRepeatedBehaviorDeliberate2024}; 1 case from Australia \cite{Apikotoa_2022f}, Bahrain \cite{Ali_2020f}, Croatia \cite{Trgo_2012f}, Ecuador \cite{DelgadoSalazar_2020c}, Egypt \cite{Ali_2022g}, Ethiopia \cite{Mesfin_2022a}, Germany \cite{teWildt_2010}, Greece \cite{Sakellaridis_2008f}, Hungary \cite{Csaky_1998e}, Iraq \cite{Al-Faham_2020k}, Israel \cite{Goldman_1998f}, Italy \cite{Riva_2018j}, Japan \cite{Ohno_2005}, Nepal \cite{Thapa_2019f}, Netherlands \cite{Benoist_2019e}, Oman \cite{AlShaaibi_2021b}, Pakistan \cite{Yasin_2009}, Portugal \cite{Peixoto_2017f}, Qatar \cite{Ali_2017}, Saudi Arabia \cite{Sultan_2024f}, South Africa \cite{Sobnach_2011f}, Sweden \cite{Naji_2012f}, Switzerland \cite{Wildhaber_2005}, and Taiwan \cite{Chang_2017f}. \paragraph*{Gender} 43 cases (60\%) were male \cite{Akay_2015f, Al-Faham_2020k, Alao_2006i, Ali_2017, Ali_2022g, Apikotoa_2022f, Atayan_2016, Benoist_2019e, Berry_2021e, Bhumi_2024f, CamachoDorado_2018, Csaky_1998e, Emamhadi_2018, Farhadi_2024h, Fry_2010, Gardner_2017h, Guinan_2019f, Jehangir_2019h, Jin_2023, Kobiela_2015, Kumar_2001, Kumar_2019f, Liu_2005, Losanoff_1996, Losanoff_1997e, Mesfin_2022a, Misra_2013, Qureshi_2016, Riva_2018j, Sobnach_2011f, Tammana_2012j, Tanrikulu_2015e, Tay_2004, Thapa_2019f, Trgo_2012f, Wadhwa_2015e, Yasin_2009, teWildt_2010}, 28 cases (39\%) were female \cite{AlShaaibi_2021b, Ali_2020f, Ataya_2013, Beecroft_1998, Bhasin_2014, Bhattacharjee_2008, Cauchi_2002, Chang_2017f, Cox_2007, DelgadoSalazar_2020c, DivsalarP._2023a, Goldman_1998f, Hardy_2023g, Kar_2015, Kariholu_2008, Kerestes_2019, Li_2013, Naji_2012f, Ohno_2005, Peixoto_2017f, Sakellaridis_2008f, Sultan_2024f, Tupesis_2004f, Wildhaber_2005, Wnęk_2015f, Yildiz_2016e}, 1 case (1\%) had no gender recorded \cite{fjbuilsRepeatedBehaviorDeliberate2024}. \paragraph*{Age Group} 25 cases (35\%) were between 26 and 40 years of age \cite{Alao_2006i, Ali_2022g, Apikotoa_2022f, Ataya_2013, Benoist_2019e, Bhasin_2014, Chang_2017f, Cox_2007, DelgadoSalazar_2020c, Farhadi_2024h, Fry_2010, Gardner_2017h, Guinan_2019f, Jin_2023, Kumar_2019f, Losanoff_1996, Misra_2013, Qureshi_2016, Riva_2018j, Sakellaridis_2008f, Tammana_2012j, Trgo_2012f, Wnęk_2015f, Yildiz_2016e, fjbuilsRepeatedBehaviorDeliberate2024}, 18 cases (25\%) were between 18 and 25 years of age \cite{Akay_2015f, Ali_2017, Atayan_2016, Bhattacharjee_2008, Csaky_1998e, Kar_2015, Kariholu_2008, Kobiela_2015, Losanoff_1996, Losanoff_1997e, Mesfin_2022a, Peixoto_2017f, Sobnach_2011f, Tupesis_2004f, Yasin_2009}, 13 cases (18\%) were under 18 years of age \cite{AlShaaibi_2021b, Ali_2020f, Cauchi_2002, DivsalarP._2023a, Goldman_1998f, Liu_2005, Naji_2012f, Ohno_2005, Tanrikulu_2015e, Tay_2004, Wildhaber_2005}, 11 cases (15\%) were between 41 and 60 years of age \cite{Al-Faham_2020k, Bhumi_2024f, CamachoDorado_2018, Emamhadi_2018, Hardy_2023g, Jehangir_2019h, Kumar_2001, Sultan_2024f, Thapa_2019f, Wadhwa_2015e, teWildt_2010}, 3 cases (4\%) were over 60 years of age \cite{Beecroft_1998, Kerestes_2019, Li_2013}, 2 cases (3\%) had no age documented \cite{Berry_2021e}. \paragraph*{Population} 36 cases (50\%) had a psychiatric history \cite{AlShaaibi_2021b, Alao_2006i, Ali_2020f, Apikotoa_2022f, Ataya_2013, Atayan_2016, Beecroft_1998, CamachoDorado_2018, Chang_2017f, DelgadoSalazar_2020c, DivsalarP._2023a, Farhadi_2024h, Fry_2010, Guinan_2019f, Hardy_2023g, Jehangir_2019h, Jin_2023, Kar_2015, Kerestes_2019, Kobiela_2015, Kumar_2001, Kumar_2019f, Liu_2005, Mesfin_2022a, Misra_2013, Ohno_2005, Peixoto_2017f, Sakellaridis_2008f, Sultan_2024f, Tammana_2012j, Tanrikulu_2015e, Yildiz_2016e, fjbuilsRepeatedBehaviorDeliberate2024, teWildt_2010}, 19 cases (26\%) had ingested previously \cite{Alao_2006i, Apikotoa_2022f, Berry_2021e, Bhattacharjee_2008, Csaky_1998e, DivsalarP._2023a, Emamhadi_2018, Guinan_2019f, Jehangir_2019h, Jin_2023, Liu_2005, Sakellaridis_2008f, Tanrikulu_2015e, Thapa_2019f, Yildiz_2016e, fjbuilsRepeatedBehaviorDeliberate2024, teWildt_2010}, 12 cases (17\%) were detained persons \cite{Alao_2006i, Ali_2022g, Apikotoa_2022f, Losanoff_1996, Losanoff_1997e, Qureshi_2016, Tammana_2012j, Trgo_2012f}, 7 cases (10\%) were severely disabled \cite{Atayan_2016, Kerestes_2019, Liu_2005, Ohno_2005, Peixoto_2017f, Yildiz_2016e, teWildt_2010}, 4 cases (6\%) were psychiatric inpatients \cite{DivsalarP._2023a, fjbuilsRepeatedBehaviorDeliberate2024, teWildt_2010}, 3 cases (4\%) were under the influence of alcohol \cite{Benoist_2019e, Csaky_1998e, Thapa_2019f}, 2 cases (3\%) were displaced people \cite{Akay_2015f, Gardner_2017h}. \paragraph*{Motivation} 34 cases (47\%) had a psychiatric motivation \cite{Al-Faham_2020k, Alao_2006i, Ali_2020f, Apikotoa_2022f, Ataya_2013, Atayan_2016, Bhasin_2014, Bhattacharjee_2008, DelgadoSalazar_2020c, DivsalarP._2023a, Emamhadi_2018, Farhadi_2024h, Guinan_2019f, Hardy_2023g, Jehangir_2019h, Jin_2023, Kar_2015, Kariholu_2008, Kerestes_2019, Kobiela_2015, Kumar_2001, Kumar_2019f, Li_2013, Liu_2005, Misra_2013, Ohno_2005, Sakellaridis_2008f, Sultan_2024f, Tammana_2012j, Tanrikulu_2015e, Yasin_2009, teWildt_2010}, 21 cases (29\%) were motivated by self-harm intention \cite{Al-Faham_2020k, AlShaaibi_2021b, Alao_2006i, Ali_2017, CamachoDorado_2018, Chang_2017f, Cox_2007, Csaky_1998e, Fry_2010, Li_2013, Losanoff_1996, Losanoff_1997e, Mesfin_2022a, Sakellaridis_2008f, Tammana_2012j, Tanrikulu_2015e, fjbuilsRepeatedBehaviorDeliberate2024}, 17 cases (24\%) had a psychosocial motivation \cite{Akay_2015f, Benoist_2019e, Bhattacharjee_2008, Cauchi_2002, Goldman_1998f, Hardy_2023g, Kobiela_2015, Li_2013, Naji_2012f, Qureshi_2016, Riva_2018j, Sobnach_2011f, Tay_2004, Thapa_2019f, Tupesis_2004f, Wildhaber_2005, Wnęk_2015f}, 9 cases (12\%) were motivated by protest \cite{Bhumi_2024f, Gardner_2017h, Losanoff_1996, Losanoff_1997e, Tupesis_2004f}, 9 cases (12\%) had another documented motivation \cite{Ali_2020f, Ali_2022g, Emamhadi_2018, Guinan_2019f, Peixoto_2017f, Sakellaridis_2008f, Trgo_2012f, Wadhwa_2015e, Yildiz_2016e}. \paragraph*{Object Characteristics} 51 cases (71\%) ingested a large diameter object (\textgreater{}2.5cm) \cite{Akay_2015f, Al-Faham_2020k, AlShaaibi_2021b, Alao_2006i, Ali_2017, Ali_2022g, Apikotoa_2022f, Atayan_2016, Berry_2021e, Bhasin_2014, CamachoDorado_2018, Cauchi_2002, Chang_2017f, Cox_2007, Csaky_1998e, DivsalarP._2023a, Emamhadi_2018, Gardner_2017h, Guinan_2019f, Jehangir_2019h, Jin_2023, Kariholu_2008, Kerestes_2019, Kobiela_2015, Kumar_2001, Kumar_2019f, Losanoff_1996, Losanoff_1997e, Mesfin_2022a, Misra_2013, Naji_2012f, Ohno_2005, Peixoto_2017f, Qureshi_2016, Riva_2018j, Sakellaridis_2008f, Sultan_2024f, Tanrikulu_2015e, Thapa_2019f, Trgo_2012f, Wnęk_2015f, Yildiz_2016e, fjbuilsRepeatedBehaviorDeliberate2024, teWildt_2010}, 44 cases (61\%) ingested multiple objects \cite{Ali_2020f, Apikotoa_2022f, Ataya_2013, Atayan_2016, Beecroft_1998, Bhattacharjee_2008, Bhumi_2024f, CamachoDorado_2018, Cauchi_2002, Emamhadi_2018, Farhadi_2024h, Fry_2010, Goldman_1998f, Guinan_2019f, Hardy_2023g, Jehangir_2019h, Jin_2023, Kar_2015, Kariholu_2008, Kobiela_2015, Kumar_2001, Kumar_2019f, Li_2013, Liu_2005, Losanoff_1996, Mesfin_2022a, Misra_2013, Naji_2012f, Ohno_2005, Sobnach_2011f, Sultan_2024f, Tammana_2012j, Tanrikulu_2015e, Tay_2004, Thapa_2019f, Wadhwa_2015e, Wildhaber_2005, Yasin_2009, fjbuilsRepeatedBehaviorDeliberate2024, teWildt_2010}, 34 cases (47\%) ingested a sharp object \cite{AlShaaibi_2021b, Alao_2006i, Apikotoa_2022f, Ataya_2013, Benoist_2019e, Bhasin_2014, Bhattacharjee_2008, CamachoDorado_2018, Csaky_1998e, DelgadoSalazar_2020c, DivsalarP._2023a, Emamhadi_2018, Farhadi_2024h, Fry_2010, Guinan_2019f, Hardy_2023g, Jehangir_2019h, Jin_2023, Kariholu_2008, Kobiela_2015, Kumar_2019f, Losanoff_1996, Losanoff_1997e, Mesfin_2022a, Misra_2013, Sobnach_2011f, Yasin_2009, teWildt_2010}, 32 cases (44\%) ingested a long object (\textgreater{}5cm) \cite{Al-Faham_2020k, AlShaaibi_2021b, Ali_2017, Ali_2022g, Atayan_2016, Bhasin_2014, CamachoDorado_2018, Chang_2017f, Cox_2007, Csaky_1998e, DivsalarP._2023a, Emamhadi_2018, Fry_2010, Gardner_2017h, Jin_2023, Kariholu_2008, Kerestes_2019, Kobiela_2015, Kumar_2019f, Mesfin_2022a, Misra_2013, Ohno_2005, Qureshi_2016, Sakellaridis_2008f, Sultan_2024f, Thapa_2019f, Trgo_2012f, Yasin_2009, Yildiz_2016e, teWildt_2010}, 9 cases (12\%) ingested a magnet \cite{Ali_2020f, Bhumi_2024f, Cauchi_2002, Liu_2005, Naji_2012f, Ohno_2005, Tanrikulu_2015e, Tay_2004, Wildhaber_2005}, 2 cases (3\%) ingested a button battery \cite{Berry_2021e, Bhumi_2024f}. \paragraph*{Outcomes} 48 cases (67\%) experienced a complication \cite{Ali_2017, Ali_2020f, Apikotoa_2022f, Atayan_2016, Beecroft_1998, Benoist_2019e, Berry_2021e, Bhasin_2014, Bhumi_2024f, CamachoDorado_2018, Cauchi_2002, Cox_2007, Csaky_1998e, DelgadoSalazar_2020c, DivsalarP._2023a, Emamhadi_2018, Farhadi_2024h, Fry_2010, Gardner_2017h, Goldman_1998f, Jin_2023, Kariholu_2008, Kerestes_2019, Kobiela_2015, Kumar_2001, Kumar_2019f, Liu_2005, Losanoff_1996, Mesfin_2022a, Misra_2013, Naji_2012f, Ohno_2005, Sakellaridis_2008f, Sobnach_2011f, Sultan_2024f, Tanrikulu_2015e, Tay_2004, Thapa_2019f, Trgo_2012f, Tupesis_2004f, Wildhaber_2005, Wnęk_2015f, Yasin_2009, Yildiz_2016e}, 44 cases (61\%) underwent surgery \cite{Al-Faham_2020k, AlShaaibi_2021b, Alao_2006i, Ali_2017, Ali_2020f, Atayan_2016, Beecroft_1998, Bhasin_2014, CamachoDorado_2018, Cauchi_2002, Chang_2017f, Cox_2007, Csaky_1998e, DelgadoSalazar_2020c, DivsalarP._2023a, Farhadi_2024h, Fry_2010, Gardner_2017h, Jin_2023, Kariholu_2008, Kerestes_2019, Kobiela_2015, Kumar_2019f, Liu_2005, Losanoff_1996, Losanoff_1997e, Mesfin_2022a, Misra_2013, Naji_2012f, Sobnach_2011f, Tanrikulu_2015e, Tay_2004, Thapa_2019f, Tupesis_2004f, Wildhaber_2005, Wnęk_2015f, Yasin_2009, Yildiz_2016e, fjbuilsRepeatedBehaviorDeliberate2024}, 31 cases (43\%) underwent endoscopy \cite{Akay_2015f, Ali_2022g, Apikotoa_2022f, Atayan_2016, Benoist_2019e, Berry_2021e, Bhasin_2014, Bhumi_2024f, CamachoDorado_2018, Chang_2017f, DelgadoSalazar_2020c, Gardner_2017h, Guinan_2019f, Hardy_2023g, Jehangir_2019h, Kariholu_2008, Li_2013, Liu_2005, Ohno_2005, Peixoto_2017f, Qureshi_2016, Riva_2018j, Sakellaridis_2008f, Sultan_2024f, Tammana_2012j, Tanrikulu_2015e, Trgo_2012f, Wadhwa_2015e, Wnęk_2015f, teWildt_2010}, 7 cases (10\%) were managed conservatively \cite{Ataya_2013, Bhattacharjee_2008, DivsalarP._2023a, Emamhadi_2018, Goldman_1998f, Kar_2015, Kumar_2001}, 2 cases (3\%) died \cite{Emamhadi_2018, Kumar_2001}. All 90 were male gender. 90 cases (100\%) were detained at the time of ingestion \cite{Elghali_2016, Karp_1991b, Lee_2007}, 88 cases (98\%) were intentional ingestions \cite{Elghali_2016, Karp_1991b, Lee_2007}, 30 cases (33\%) had a psychiatric history documented \cite{Elghali_2016, Karp_1991b, Lee_2007}, 2 cases (2\%) had a history of prior ingestion \cite{Elghali_2016}. No cases were reported for were psychiatric inpatients, were displaced people, were under the influence of alcohol at the time of ingestion, and had a severe disability history.
\paragraph*{Motivation}  70 cases (78\%) reported protest motivation \cite{Elghali_2016, Karp_1991b, Lee_2007}, 12 cases (13\%) reported psychiatric motivation \cite{Karp_1991b}, 6 cases (7\%) reported self-harm motivation \cite{Elghali_2016, Karp_1991b}. No cases were reported for psychosocial motivation and other motivation.
\paragraph*{Object Characteristics}  68 cases (76\%) involved sharp object ingestion \cite{Elghali_2016, Karp_1991b, Lee_2007}, 32 cases (36\%) involved long (\textgreater 5cm) object ingestion \cite{Lee_2007}, 25 cases (28\%) involved ingestion of multiple objects \cite{Elghali_2016, Lee_2007}. No cases were reported for button battery ingestion, magnet ingestion, and involved large diameter (\textgreater 2.5cm) object ingestion.
\paragraph*{Outcomes}  47 cases (52\%) underwent endoscopic intervention \cite{Elghali_2016, Lee_2007}, 29 cases (32\%) were managed conservatively \cite{Elghali_2016, Karp_1991b}, 15 cases (17\%) underwent surgical intervention \cite{Elghali_2016, Karp_1991b, Lee_2007}, 6 cases (7\%) reported complications \cite{Lee_2007}, 1 case (1\%) died \cite{Elghali_2016}.
\paragraph*{Geographical Location}Cases were recorded in 33 countries: 13 cases from USA \cite{Alao_2006i, Ataya_2013, Bhumi_2024f, Fry_2010, Guinan_2019f, Hardy_2023g, Jehangir_2019h, Kerestes_2019, Kumar_2001, Liu_2005, Tammana_2012j, Tay_2004, Tupesis_2004f}; 7 cases from India \cite{Bhasin_2014, Bhattacharjee_2008, Kar_2015, Kariholu_2008, Kumar_2019f, Misra_2013, Wadhwa_2015e} and UK \cite{Beecroft_1998, Berry_2021e, Cauchi_2002, Cox_2007, Gardner_2017h, Qureshi_2016}; 6 cases from Bulgaria \cite{Losanoff_1996, Losanoff_1997e}; 5 cases from Iran \cite{DivsalarP._2023a, Emamhadi_2018, Farhadi_2024h}; 4 cases from Turkey \cite{Akay_2015f, Atayan_2016, Tanrikulu_2015e, Yildiz_2016e}; 2 cases from China \cite{Jin_2023, Li_2013}, Poland \cite{Kobiela_2015, Wnęk_2015f}, and Spain \cite{CamachoDorado_2018, fjbuilsRepeatedBehaviorDeliberate2024}; 1 case from Australia \cite{Apikotoa_2022f}, Bahrain \cite{Ali_2020f}, Croatia \cite{Trgo_2012f}, Ecuador \cite{DelgadoSalazar_2020c}, Egypt \cite{Ali_2022g}, Ethiopia \cite{Mesfin_2022a}, Germany \cite{teWildt_2010}, Greece \cite{Sakellaridis_2008f}, Hungary \cite{Csaky_1998e}, Iraq \cite{Al-Faham_2020k}, Israel \cite{Goldman_1998f}, Italy \cite{Riva_2018j}, Japan \cite{Ohno_2005}, Nepal \cite{Thapa_2019f}, Netherlands \cite{Benoist_2019e}, Oman \cite{AlShaaibi_2021b}, Pakistan \cite{Yasin_2009}, Portugal \cite{Peixoto_2017f}, Qatar \cite{Ali_2017}, Saudi Arabia \cite{Sultan_2024f}, South Africa \cite{Sobnach_2011f}, Sweden \cite{Naji_2012f}, Switzerland \cite{Wildhaber_2005}, and Taiwan \cite{Chang_2017f}. \paragraph*{Gender} 43 cases (60\%) were male \cite{Akay_2015f, Al-Faham_2020k, Alao_2006i, Ali_2017, Ali_2022g, Apikotoa_2022f, Atayan_2016, Benoist_2019e, Berry_2021e, Bhumi_2024f, CamachoDorado_2018, Csaky_1998e, Emamhadi_2018, Farhadi_2024h, Fry_2010, Gardner_2017h, Guinan_2019f, Jehangir_2019h, Jin_2023, Kobiela_2015, Kumar_2001, Kumar_2019f, Liu_2005, Losanoff_1996, Losanoff_1997e, Mesfin_2022a, Misra_2013, Qureshi_2016, Riva_2018j, Sobnach_2011f, Tammana_2012j, Tanrikulu_2015e, Tay_2004, Thapa_2019f, Trgo_2012f, Wadhwa_2015e, Yasin_2009, teWildt_2010}, 28 cases (39\%) were female \cite{AlShaaibi_2021b, Ali_2020f, Ataya_2013, Beecroft_1998, Bhasin_2014, Bhattacharjee_2008, Cauchi_2002, Chang_2017f, Cox_2007, DelgadoSalazar_2020c, DivsalarP._2023a, Goldman_1998f, Hardy_2023g, Kar_2015, Kariholu_2008, Kerestes_2019, Li_2013, Naji_2012f, Ohno_2005, Peixoto_2017f, Sakellaridis_2008f, Sultan_2024f, Tupesis_2004f, Wildhaber_2005, Wnęk_2015f, Yildiz_2016e}, 1 case (1\%) had no gender recorded \cite{fjbuilsRepeatedBehaviorDeliberate2024}. \paragraph*{Age Group} 25 cases (35\%) were between 26 and 40 years of age \cite{Alao_2006i, Ali_2022g, Apikotoa_2022f, Ataya_2013, Benoist_2019e, Bhasin_2014, Chang_2017f, Cox_2007, DelgadoSalazar_2020c, Farhadi_2024h, Fry_2010, Gardner_2017h, Guinan_2019f, Jin_2023, Kumar_2019f, Losanoff_1996, Misra_2013, Qureshi_2016, Riva_2018j, Sakellaridis_2008f, Tammana_2012j, Trgo_2012f, Wnęk_2015f, Yildiz_2016e, fjbuilsRepeatedBehaviorDeliberate2024}, 18 cases (25\%) were between 18 and 25 years of age \cite{Akay_2015f, Ali_2017, Atayan_2016, Bhattacharjee_2008, Csaky_1998e, Kar_2015, Kariholu_2008, Kobiela_2015, Losanoff_1996, Losanoff_1997e, Mesfin_2022a, Peixoto_2017f, Sobnach_2011f, Tupesis_2004f, Yasin_2009}, 13 cases (18\%) were under 18 years of age \cite{AlShaaibi_2021b, Ali_2020f, Cauchi_2002, DivsalarP._2023a, Goldman_1998f, Liu_2005, Naji_2012f, Ohno_2005, Tanrikulu_2015e, Tay_2004, Wildhaber_2005}, 11 cases (15\%) were between 41 and 60 years of age \cite{Al-Faham_2020k, Bhumi_2024f, CamachoDorado_2018, Emamhadi_2018, Hardy_2023g, Jehangir_2019h, Kumar_2001, Sultan_2024f, Thapa_2019f, Wadhwa_2015e, teWildt_2010}, 3 cases (4\%) were over 60 years of age \cite{Beecroft_1998, Kerestes_2019, Li_2013}, 2 cases (3\%) had no age documented \cite{Berry_2021e}. \paragraph*{Population} 36 cases (50\%) had a psychiatric history \cite{AlShaaibi_2021b, Alao_2006i, Ali_2020f, Apikotoa_2022f, Ataya_2013, Atayan_2016, Beecroft_1998, CamachoDorado_2018, Chang_2017f, DelgadoSalazar_2020c, DivsalarP._2023a, Farhadi_2024h, Fry_2010, Guinan_2019f, Hardy_2023g, Jehangir_2019h, Jin_2023, Kar_2015, Kerestes_2019, Kobiela_2015, Kumar_2001, Kumar_2019f, Liu_2005, Mesfin_2022a, Misra_2013, Ohno_2005, Peixoto_2017f, Sakellaridis_2008f, Sultan_2024f, Tammana_2012j, Tanrikulu_2015e, Yildiz_2016e, fjbuilsRepeatedBehaviorDeliberate2024, teWildt_2010}, 19 cases (26\%) had ingested previously \cite{Alao_2006i, Apikotoa_2022f, Berry_2021e, Bhattacharjee_2008, Csaky_1998e, DivsalarP._2023a, Emamhadi_2018, Guinan_2019f, Jehangir_2019h, Jin_2023, Liu_2005, Sakellaridis_2008f, Tanrikulu_2015e, Thapa_2019f, Yildiz_2016e, fjbuilsRepeatedBehaviorDeliberate2024, teWildt_2010}, 12 cases (17\%) were detained persons \cite{Alao_2006i, Ali_2022g, Apikotoa_2022f, Losanoff_1996, Losanoff_1997e, Qureshi_2016, Tammana_2012j, Trgo_2012f}, 7 cases (10\%) were severely disabled \cite{Atayan_2016, Kerestes_2019, Liu_2005, Ohno_2005, Peixoto_2017f, Yildiz_2016e, teWildt_2010}, 4 cases (6\%) were psychiatric inpatients \cite{DivsalarP._2023a, fjbuilsRepeatedBehaviorDeliberate2024, teWildt_2010}, 3 cases (4\%) were under the influence of alcohol \cite{Benoist_2019e, Csaky_1998e, Thapa_2019f}, 2 cases (3\%) were displaced people \cite{Akay_2015f, Gardner_2017h}. \paragraph*{Motivation} 34 cases (47\%) had a psychiatric motivation \cite{Al-Faham_2020k, Alao_2006i, Ali_2020f, Apikotoa_2022f, Ataya_2013, Atayan_2016, Bhasin_2014, Bhattacharjee_2008, DelgadoSalazar_2020c, DivsalarP._2023a, Emamhadi_2018, Farhadi_2024h, Guinan_2019f, Hardy_2023g, Jehangir_2019h, Jin_2023, Kar_2015, Kariholu_2008, Kerestes_2019, Kobiela_2015, Kumar_2001, Kumar_2019f, Li_2013, Liu_2005, Misra_2013, Ohno_2005, Sakellaridis_2008f, Sultan_2024f, Tammana_2012j, Tanrikulu_2015e, Yasin_2009, teWildt_2010}, 21 cases (29\%) were motivated by self-harm intention \cite{Al-Faham_2020k, AlShaaibi_2021b, Alao_2006i, Ali_2017, CamachoDorado_2018, Chang_2017f, Cox_2007, Csaky_1998e, Fry_2010, Li_2013, Losanoff_1996, Losanoff_1997e, Mesfin_2022a, Sakellaridis_2008f, Tammana_2012j, Tanrikulu_2015e, fjbuilsRepeatedBehaviorDeliberate2024}, 17 cases (24\%) had a psychosocial motivation \cite{Akay_2015f, Benoist_2019e, Bhattacharjee_2008, Cauchi_2002, Goldman_1998f, Hardy_2023g, Kobiela_2015, Li_2013, Naji_2012f, Qureshi_2016, Riva_2018j, Sobnach_2011f, Tay_2004, Thapa_2019f, Tupesis_2004f, Wildhaber_2005, Wnęk_2015f}, 9 cases (12\%) were motivated by protest \cite{Bhumi_2024f, Gardner_2017h, Losanoff_1996, Losanoff_1997e, Tupesis_2004f}, 9 cases (12\%) had another documented motivation \cite{Ali_2020f, Ali_2022g, Emamhadi_2018, Guinan_2019f, Peixoto_2017f, Sakellaridis_2008f, Trgo_2012f, Wadhwa_2015e, Yildiz_2016e}. \paragraph*{Object Characteristics} 51 cases (71\%) ingested a large diameter object (\textgreater{}2.5cm) \cite{Akay_2015f, Al-Faham_2020k, AlShaaibi_2021b, Alao_2006i, Ali_2017, Ali_2022g, Apikotoa_2022f, Atayan_2016, Berry_2021e, Bhasin_2014, CamachoDorado_2018, Cauchi_2002, Chang_2017f, Cox_2007, Csaky_1998e, DivsalarP._2023a, Emamhadi_2018, Gardner_2017h, Guinan_2019f, Jehangir_2019h, Jin_2023, Kariholu_2008, Kerestes_2019, Kobiela_2015, Kumar_2001, Kumar_2019f, Losanoff_1996, Losanoff_1997e, Mesfin_2022a, Misra_2013, Naji_2012f, Ohno_2005, Peixoto_2017f, Qureshi_2016, Riva_2018j, Sakellaridis_2008f, Sultan_2024f, Tanrikulu_2015e, Thapa_2019f, Trgo_2012f, Wnęk_2015f, Yildiz_2016e, fjbuilsRepeatedBehaviorDeliberate2024, teWildt_2010}, 44 cases (61\%) ingested multiple objects \cite{Ali_2020f, Apikotoa_2022f, Ataya_2013, Atayan_2016, Beecroft_1998, Bhattacharjee_2008, Bhumi_2024f, CamachoDorado_2018, Cauchi_2002, Emamhadi_2018, Farhadi_2024h, Fry_2010, Goldman_1998f, Guinan_2019f, Hardy_2023g, Jehangir_2019h, Jin_2023, Kar_2015, Kariholu_2008, Kobiela_2015, Kumar_2001, Kumar_2019f, Li_2013, Liu_2005, Losanoff_1996, Mesfin_2022a, Misra_2013, Naji_2012f, Ohno_2005, Sobnach_2011f, Sultan_2024f, Tammana_2012j, Tanrikulu_2015e, Tay_2004, Thapa_2019f, Wadhwa_2015e, Wildhaber_2005, Yasin_2009, fjbuilsRepeatedBehaviorDeliberate2024, teWildt_2010}, 34 cases (47\%) ingested a sharp object \cite{AlShaaibi_2021b, Alao_2006i, Apikotoa_2022f, Ataya_2013, Benoist_2019e, Bhasin_2014, Bhattacharjee_2008, CamachoDorado_2018, Csaky_1998e, DelgadoSalazar_2020c, DivsalarP._2023a, Emamhadi_2018, Farhadi_2024h, Fry_2010, Guinan_2019f, Hardy_2023g, Jehangir_2019h, Jin_2023, Kariholu_2008, Kobiela_2015, Kumar_2019f, Losanoff_1996, Losanoff_1997e, Mesfin_2022a, Misra_2013, Sobnach_2011f, Yasin_2009, teWildt_2010}, 32 cases (44\%) ingested a long object (\textgreater{}5cm) \cite{Al-Faham_2020k, AlShaaibi_2021b, Ali_2017, Ali_2022g, Atayan_2016, Bhasin_2014, CamachoDorado_2018, Chang_2017f, Cox_2007, Csaky_1998e, DivsalarP._2023a, Emamhadi_2018, Fry_2010, Gardner_2017h, Jin_2023, Kariholu_2008, Kerestes_2019, Kobiela_2015, Kumar_2019f, Mesfin_2022a, Misra_2013, Ohno_2005, Qureshi_2016, Sakellaridis_2008f, Sultan_2024f, Thapa_2019f, Trgo_2012f, Yasin_2009, Yildiz_2016e, teWildt_2010}, 9 cases (12\%) ingested a magnet \cite{Ali_2020f, Bhumi_2024f, Cauchi_2002, Liu_2005, Naji_2012f, Ohno_2005, Tanrikulu_2015e, Tay_2004, Wildhaber_2005}, 2 cases (3\%) ingested a button battery \cite{Berry_2021e, Bhumi_2024f}. \paragraph*{Outcomes} 48 cases (67\%) experienced a complication \cite{Ali_2017, Ali_2020f, Apikotoa_2022f, Atayan_2016, Beecroft_1998, Benoist_2019e, Berry_2021e, Bhasin_2014, Bhumi_2024f, CamachoDorado_2018, Cauchi_2002, Cox_2007, Csaky_1998e, DelgadoSalazar_2020c, DivsalarP._2023a, Emamhadi_2018, Farhadi_2024h, Fry_2010, Gardner_2017h, Goldman_1998f, Jin_2023, Kariholu_2008, Kerestes_2019, Kobiela_2015, Kumar_2001, Kumar_2019f, Liu_2005, Losanoff_1996, Mesfin_2022a, Misra_2013, Naji_2012f, Ohno_2005, Sakellaridis_2008f, Sobnach_2011f, Sultan_2024f, Tanrikulu_2015e, Tay_2004, Thapa_2019f, Trgo_2012f, Tupesis_2004f, Wildhaber_2005, Wnęk_2015f, Yasin_2009, Yildiz_2016e}, 44 cases (61\%) underwent surgery \cite{Al-Faham_2020k, AlShaaibi_2021b, Alao_2006i, Ali_2017, Ali_2020f, Atayan_2016, Beecroft_1998, Bhasin_2014, CamachoDorado_2018, Cauchi_2002, Chang_2017f, Cox_2007, Csaky_1998e, DelgadoSalazar_2020c, DivsalarP._2023a, Farhadi_2024h, Fry_2010, Gardner_2017h, Jin_2023, Kariholu_2008, Kerestes_2019, Kobiela_2015, Kumar_2019f, Liu_2005, Losanoff_1996, Losanoff_1997e, Mesfin_2022a, Misra_2013, Naji_2012f, Sobnach_2011f, Tanrikulu_2015e, Tay_2004, Thapa_2019f, Tupesis_2004f, Wildhaber_2005, Wnęk_2015f, Yasin_2009, Yildiz_2016e, fjbuilsRepeatedBehaviorDeliberate2024}, 31 cases (43\%) underwent endoscopy \cite{Akay_2015f, Ali_2022g, Apikotoa_2022f, Atayan_2016, Benoist_2019e, Berry_2021e, Bhasin_2014, Bhumi_2024f, CamachoDorado_2018, Chang_2017f, DelgadoSalazar_2020c, Gardner_2017h, Guinan_2019f, Hardy_2023g, Jehangir_2019h, Kariholu_2008, Li_2013, Liu_2005, Ohno_2005, Peixoto_2017f, Qureshi_2016, Riva_2018j, Sakellaridis_2008f, Sultan_2024f, Tammana_2012j, Tanrikulu_2015e, Trgo_2012f, Wadhwa_2015e, Wnęk_2015f, teWildt_2010}, 7 cases (10\%) were managed conservatively \cite{Ataya_2013, Bhattacharjee_2008, DivsalarP._2023a, Emamhadi_2018, Goldman_1998f, Kar_2015, Kumar_2001}, 2 cases (3\%) died \cite{Emamhadi_2018, Kumar_2001}. All 90 were male gender. 90 cases (100\%) were detained at the time of ingestion \cite{Elghali_2016, Karp_1991b, Lee_2007}, 88 cases (98\%) were intentional ingestions \cite{Elghali_2016, Karp_1991b, Lee_2007}, 30 cases (33\%) had a psychiatric history documented \cite{Elghali_2016, Karp_1991b, Lee_2007}, 2 cases (2\%) had a history of prior ingestion \cite{Elghali_2016}. No cases were reported for were psychiatric inpatients, were displaced people, were under the influence of alcohol at the time of ingestion, and had a severe disability history.
\paragraph*{Motivation}  70 cases (78\%) reported protest motivation \cite{Elghali_2016, Karp_1991b, Lee_2007}, 12 cases (13\%) reported psychiatric motivation \cite{Karp_1991b}, 6 cases (7\%) reported self-harm motivation \cite{Elghali_2016, Karp_1991b}. No cases were reported for psychosocial motivation and other motivation.
\paragraph*{Object Characteristics}  68 cases (76\%) involved sharp object ingestion \cite{Elghali_2016, Karp_1991b, Lee_2007}, 32 cases (36\%) involved long (\textgreater 5cm) object ingestion \cite{Lee_2007}, 25 cases (28\%) involved ingestion of multiple objects \cite{Elghali_2016, Lee_2007}. No cases were reported for button battery ingestion, magnet ingestion, and involved large diameter (\textgreater 2.5cm) object ingestion.
\paragraph*{Outcomes}  47 cases (52\%) underwent endoscopic intervention \cite{Elghali_2016, Lee_2007}, 29 cases (32\%) were managed conservatively \cite{Elghali_2016, Karp_1991b}, 15 cases (17\%) underwent surgical intervention \cite{Elghali_2016, Karp_1991b, Lee_2007}, 6 cases (7\%) reported complications \cite{Lee_2007}, 1 case (1\%) died \cite{Elghali_2016}.
\paragraph*{Geographical Location}Cases were recorded in 33 countries: 13 cases from USA \cite{Alao_2006i, Ataya_2013, Bhumi_2024f, Fry_2010, Guinan_2019f, Hardy_2023g, Jehangir_2019h, Kerestes_2019, Kumar_2001, Liu_2005, Tammana_2012j, Tay_2004, Tupesis_2004f}; 7 cases from India \cite{Bhasin_2014, Bhattacharjee_2008, Kar_2015, Kariholu_2008, Kumar_2019f, Misra_2013, Wadhwa_2015e} and UK \cite{Beecroft_1998, Berry_2021e, Cauchi_2002, Cox_2007, Gardner_2017h, Qureshi_2016}; 6 cases from Bulgaria \cite{Losanoff_1996, Losanoff_1997e}; 5 cases from Iran \cite{DivsalarP._2023a, Emamhadi_2018, Farhadi_2024h}; 4 cases from Turkey \cite{Akay_2015f, Atayan_2016, Tanrikulu_2015e, Yildiz_2016e}; 2 cases from China \cite{Jin_2023, Li_2013}, Poland \cite{Kobiela_2015, Wnęk_2015f}, and Spain \cite{CamachoDorado_2018, fjbuilsRepeatedBehaviorDeliberate2024}; 1 case from Australia \cite{Apikotoa_2022f}, Bahrain \cite{Ali_2020f}, Croatia \cite{Trgo_2012f}, Ecuador \cite{DelgadoSalazar_2020c}, Egypt \cite{Ali_2022g}, Ethiopia \cite{Mesfin_2022a}, Germany \cite{teWildt_2010}, Greece \cite{Sakellaridis_2008f}, Hungary \cite{Csaky_1998e}, Iraq \cite{Al-Faham_2020k}, Israel \cite{Goldman_1998f}, Italy \cite{Riva_2018j}, Japan \cite{Ohno_2005}, Nepal \cite{Thapa_2019f}, Netherlands \cite{Benoist_2019e}, Oman \cite{AlShaaibi_2021b}, Pakistan \cite{Yasin_2009}, Portugal \cite{Peixoto_2017f}, Qatar \cite{Ali_2017}, Saudi Arabia \cite{Sultan_2024f}, South Africa \cite{Sobnach_2011f}, Sweden \cite{Naji_2012f}, Switzerland \cite{Wildhaber_2005}, and Taiwan \cite{Chang_2017f}. \paragraph*{Gender} 43 cases (60\%) were male \cite{Akay_2015f, Al-Faham_2020k, Alao_2006i, Ali_2017, Ali_2022g, Apikotoa_2022f, Atayan_2016, Benoist_2019e, Berry_2021e, Bhumi_2024f, CamachoDorado_2018, Csaky_1998e, Emamhadi_2018, Farhadi_2024h, Fry_2010, Gardner_2017h, Guinan_2019f, Jehangir_2019h, Jin_2023, Kobiela_2015, Kumar_2001, Kumar_2019f, Liu_2005, Losanoff_1996, Losanoff_1997e, Mesfin_2022a, Misra_2013, Qureshi_2016, Riva_2018j, Sobnach_2011f, Tammana_2012j, Tanrikulu_2015e, Tay_2004, Thapa_2019f, Trgo_2012f, Wadhwa_2015e, Yasin_2009, teWildt_2010}, 28 cases (39\%) were female \cite{AlShaaibi_2021b, Ali_2020f, Ataya_2013, Beecroft_1998, Bhasin_2014, Bhattacharjee_2008, Cauchi_2002, Chang_2017f, Cox_2007, DelgadoSalazar_2020c, DivsalarP._2023a, Goldman_1998f, Hardy_2023g, Kar_2015, Kariholu_2008, Kerestes_2019, Li_2013, Naji_2012f, Ohno_2005, Peixoto_2017f, Sakellaridis_2008f, Sultan_2024f, Tupesis_2004f, Wildhaber_2005, Wnęk_2015f, Yildiz_2016e}, 1 case (1\%) had no gender recorded \cite{fjbuilsRepeatedBehaviorDeliberate2024}. \paragraph*{Age Group} 25 cases (35\%) were between 26 and 40 years of age \cite{Alao_2006i, Ali_2022g, Apikotoa_2022f, Ataya_2013, Benoist_2019e, Bhasin_2014, Chang_2017f, Cox_2007, DelgadoSalazar_2020c, Farhadi_2024h, Fry_2010, Gardner_2017h, Guinan_2019f, Jin_2023, Kumar_2019f, Losanoff_1996, Misra_2013, Qureshi_2016, Riva_2018j, Sakellaridis_2008f, Tammana_2012j, Trgo_2012f, Wnęk_2015f, Yildiz_2016e, fjbuilsRepeatedBehaviorDeliberate2024}, 18 cases (25\%) were between 18 and 25 years of age \cite{Akay_2015f, Ali_2017, Atayan_2016, Bhattacharjee_2008, Csaky_1998e, Kar_2015, Kariholu_2008, Kobiela_2015, Losanoff_1996, Losanoff_1997e, Mesfin_2022a, Peixoto_2017f, Sobnach_2011f, Tupesis_2004f, Yasin_2009}, 13 cases (18\%) were under 18 years of age \cite{AlShaaibi_2021b, Ali_2020f, Cauchi_2002, DivsalarP._2023a, Goldman_1998f, Liu_2005, Naji_2012f, Ohno_2005, Tanrikulu_2015e, Tay_2004, Wildhaber_2005}, 11 cases (15\%) were between 41 and 60 years of age \cite{Al-Faham_2020k, Bhumi_2024f, CamachoDorado_2018, Emamhadi_2018, Hardy_2023g, Jehangir_2019h, Kumar_2001, Sultan_2024f, Thapa_2019f, Wadhwa_2015e, teWildt_2010}, 3 cases (4\%) were over 60 years of age \cite{Beecroft_1998, Kerestes_2019, Li_2013}, 2 cases (3\%) had no age documented \cite{Berry_2021e}. \paragraph*{Population} 36 cases (50\%) had a psychiatric history \cite{AlShaaibi_2021b, Alao_2006i, Ali_2020f, Apikotoa_2022f, Ataya_2013, Atayan_2016, Beecroft_1998, CamachoDorado_2018, Chang_2017f, DelgadoSalazar_2020c, DivsalarP._2023a, Farhadi_2024h, Fry_2010, Guinan_2019f, Hardy_2023g, Jehangir_2019h, Jin_2023, Kar_2015, Kerestes_2019, Kobiela_2015, Kumar_2001, Kumar_2019f, Liu_2005, Mesfin_2022a, Misra_2013, Ohno_2005, Peixoto_2017f, Sakellaridis_2008f, Sultan_2024f, Tammana_2012j, Tanrikulu_2015e, Yildiz_2016e, fjbuilsRepeatedBehaviorDeliberate2024, teWildt_2010}, 19 cases (26\%) had ingested previously \cite{Alao_2006i, Apikotoa_2022f, Berry_2021e, Bhattacharjee_2008, Csaky_1998e, DivsalarP._2023a, Emamhadi_2018, Guinan_2019f, Jehangir_2019h, Jin_2023, Liu_2005, Sakellaridis_2008f, Tanrikulu_2015e, Thapa_2019f, Yildiz_2016e, fjbuilsRepeatedBehaviorDeliberate2024, teWildt_2010}, 12 cases (17\%) were detained persons \cite{Alao_2006i, Ali_2022g, Apikotoa_2022f, Losanoff_1996, Losanoff_1997e, Qureshi_2016, Tammana_2012j, Trgo_2012f}, 7 cases (10\%) were severely disabled \cite{Atayan_2016, Kerestes_2019, Liu_2005, Ohno_2005, Peixoto_2017f, Yildiz_2016e, teWildt_2010}, 4 cases (6\%) were psychiatric inpatients \cite{DivsalarP._2023a, fjbuilsRepeatedBehaviorDeliberate2024, teWildt_2010}, 3 cases (4\%) were under the influence of alcohol \cite{Benoist_2019e, Csaky_1998e, Thapa_2019f}, 2 cases (3\%) were displaced people \cite{Akay_2015f, Gardner_2017h}. \paragraph*{Motivation} 34 cases (47\%) had a psychiatric motivation \cite{Al-Faham_2020k, Alao_2006i, Ali_2020f, Apikotoa_2022f, Ataya_2013, Atayan_2016, Bhasin_2014, Bhattacharjee_2008, DelgadoSalazar_2020c, DivsalarP._2023a, Emamhadi_2018, Farhadi_2024h, Guinan_2019f, Hardy_2023g, Jehangir_2019h, Jin_2023, Kar_2015, Kariholu_2008, Kerestes_2019, Kobiela_2015, Kumar_2001, Kumar_2019f, Li_2013, Liu_2005, Misra_2013, Ohno_2005, Sakellaridis_2008f, Sultan_2024f, Tammana_2012j, Tanrikulu_2015e, Yasin_2009, teWildt_2010}, 21 cases (29\%) were motivated by self-harm intention \cite{Al-Faham_2020k, AlShaaibi_2021b, Alao_2006i, Ali_2017, CamachoDorado_2018, Chang_2017f, Cox_2007, Csaky_1998e, Fry_2010, Li_2013, Losanoff_1996, Losanoff_1997e, Mesfin_2022a, Sakellaridis_2008f, Tammana_2012j, Tanrikulu_2015e, fjbuilsRepeatedBehaviorDeliberate2024}, 17 cases (24\%) had a psychosocial motivation \cite{Akay_2015f, Benoist_2019e, Bhattacharjee_2008, Cauchi_2002, Goldman_1998f, Hardy_2023g, Kobiela_2015, Li_2013, Naji_2012f, Qureshi_2016, Riva_2018j, Sobnach_2011f, Tay_2004, Thapa_2019f, Tupesis_2004f, Wildhaber_2005, Wnęk_2015f}, 9 cases (12\%) were motivated by protest \cite{Bhumi_2024f, Gardner_2017h, Losanoff_1996, Losanoff_1997e, Tupesis_2004f}, 9 cases (12\%) had another documented motivation \cite{Ali_2020f, Ali_2022g, Emamhadi_2018, Guinan_2019f, Peixoto_2017f, Sakellaridis_2008f, Trgo_2012f, Wadhwa_2015e, Yildiz_2016e}. \paragraph*{Object Characteristics} 51 cases (71\%) ingested a large diameter object (\textgreater{}2.5cm) \cite{Akay_2015f, Al-Faham_2020k, AlShaaibi_2021b, Alao_2006i, Ali_2017, Ali_2022g, Apikotoa_2022f, Atayan_2016, Berry_2021e, Bhasin_2014, CamachoDorado_2018, Cauchi_2002, Chang_2017f, Cox_2007, Csaky_1998e, DivsalarP._2023a, Emamhadi_2018, Gardner_2017h, Guinan_2019f, Jehangir_2019h, Jin_2023, Kariholu_2008, Kerestes_2019, Kobiela_2015, Kumar_2001, Kumar_2019f, Losanoff_1996, Losanoff_1997e, Mesfin_2022a, Misra_2013, Naji_2012f, Ohno_2005, Peixoto_2017f, Qureshi_2016, Riva_2018j, Sakellaridis_2008f, Sultan_2024f, Tanrikulu_2015e, Thapa_2019f, Trgo_2012f, Wnęk_2015f, Yildiz_2016e, fjbuilsRepeatedBehaviorDeliberate2024, teWildt_2010}, 44 cases (61\%) ingested multiple objects \cite{Ali_2020f, Apikotoa_2022f, Ataya_2013, Atayan_2016, Beecroft_1998, Bhattacharjee_2008, Bhumi_2024f, CamachoDorado_2018, Cauchi_2002, Emamhadi_2018, Farhadi_2024h, Fry_2010, Goldman_1998f, Guinan_2019f, Hardy_2023g, Jehangir_2019h, Jin_2023, Kar_2015, Kariholu_2008, Kobiela_2015, Kumar_2001, Kumar_2019f, Li_2013, Liu_2005, Losanoff_1996, Mesfin_2022a, Misra_2013, Naji_2012f, Ohno_2005, Sobnach_2011f, Sultan_2024f, Tammana_2012j, Tanrikulu_2015e, Tay_2004, Thapa_2019f, Wadhwa_2015e, Wildhaber_2005, Yasin_2009, fjbuilsRepeatedBehaviorDeliberate2024, teWildt_2010}, 34 cases (47\%) ingested a sharp object \cite{AlShaaibi_2021b, Alao_2006i, Apikotoa_2022f, Ataya_2013, Benoist_2019e, Bhasin_2014, Bhattacharjee_2008, CamachoDorado_2018, Csaky_1998e, DelgadoSalazar_2020c, DivsalarP._2023a, Emamhadi_2018, Farhadi_2024h, Fry_2010, Guinan_2019f, Hardy_2023g, Jehangir_2019h, Jin_2023, Kariholu_2008, Kobiela_2015, Kumar_2019f, Losanoff_1996, Losanoff_1997e, Mesfin_2022a, Misra_2013, Sobnach_2011f, Yasin_2009, teWildt_2010}, 32 cases (44\%) ingested a long object (\textgreater{}5cm) \cite{Al-Faham_2020k, AlShaaibi_2021b, Ali_2017, Ali_2022g, Atayan_2016, Bhasin_2014, CamachoDorado_2018, Chang_2017f, Cox_2007, Csaky_1998e, DivsalarP._2023a, Emamhadi_2018, Fry_2010, Gardner_2017h, Jin_2023, Kariholu_2008, Kerestes_2019, Kobiela_2015, Kumar_2019f, Mesfin_2022a, Misra_2013, Ohno_2005, Qureshi_2016, Sakellaridis_2008f, Sultan_2024f, Thapa_2019f, Trgo_2012f, Yasin_2009, Yildiz_2016e, teWildt_2010}, 9 cases (12\%) ingested a magnet \cite{Ali_2020f, Bhumi_2024f, Cauchi_2002, Liu_2005, Naji_2012f, Ohno_2005, Tanrikulu_2015e, Tay_2004, Wildhaber_2005}, 2 cases (3\%) ingested a button battery \cite{Berry_2021e, Bhumi_2024f}. \paragraph*{Outcomes} 48 cases (67\%) experienced a complication \cite{Ali_2017, Ali_2020f, Apikotoa_2022f, Atayan_2016, Beecroft_1998, Benoist_2019e, Berry_2021e, Bhasin_2014, Bhumi_2024f, CamachoDorado_2018, Cauchi_2002, Cox_2007, Csaky_1998e, DelgadoSalazar_2020c, DivsalarP._2023a, Emamhadi_2018, Farhadi_2024h, Fry_2010, Gardner_2017h, Goldman_1998f, Jin_2023, Kariholu_2008, Kerestes_2019, Kobiela_2015, Kumar_2001, Kumar_2019f, Liu_2005, Losanoff_1996, Mesfin_2022a, Misra_2013, Naji_2012f, Ohno_2005, Sakellaridis_2008f, Sobnach_2011f, Sultan_2024f, Tanrikulu_2015e, Tay_2004, Thapa_2019f, Trgo_2012f, Tupesis_2004f, Wildhaber_2005, Wnęk_2015f, Yasin_2009, Yildiz_2016e}, 44 cases (61\%) underwent surgery \cite{Al-Faham_2020k, AlShaaibi_2021b, Alao_2006i, Ali_2017, Ali_2020f, Atayan_2016, Beecroft_1998, Bhasin_2014, CamachoDorado_2018, Cauchi_2002, Chang_2017f, Cox_2007, Csaky_1998e, DelgadoSalazar_2020c, DivsalarP._2023a, Farhadi_2024h, Fry_2010, Gardner_2017h, Jin_2023, Kariholu_2008, Kerestes_2019, Kobiela_2015, Kumar_2019f, Liu_2005, Losanoff_1996, Losanoff_1997e, Mesfin_2022a, Misra_2013, Naji_2012f, Sobnach_2011f, Tanrikulu_2015e, Tay_2004, Thapa_2019f, Tupesis_2004f, Wildhaber_2005, Wnęk_2015f, Yasin_2009, Yildiz_2016e, fjbuilsRepeatedBehaviorDeliberate2024}, 31 cases (43\%) underwent endoscopy \cite{Akay_2015f, Ali_2022g, Apikotoa_2022f, Atayan_2016, Benoist_2019e, Berry_2021e, Bhasin_2014, Bhumi_2024f, CamachoDorado_2018, Chang_2017f, DelgadoSalazar_2020c, Gardner_2017h, Guinan_2019f, Hardy_2023g, Jehangir_2019h, Kariholu_2008, Li_2013, Liu_2005, Ohno_2005, Peixoto_2017f, Qureshi_2016, Riva_2018j, Sakellaridis_2008f, Sultan_2024f, Tammana_2012j, Tanrikulu_2015e, Trgo_2012f, Wadhwa_2015e, Wnęk_2015f, teWildt_2010}, 7 cases (10\%) were managed conservatively \cite{Ataya_2013, Bhattacharjee_2008, DivsalarP._2023a, Emamhadi_2018, Goldman_1998f, Kar_2015, Kumar_2001}, 2 cases (3\%) died \cite{Emamhadi_2018, Kumar_2001}. All 90 were male gender. 90 cases (100\%) were detained at the time of ingestion \cite{Elghali_2016, Karp_1991b, Lee_2007}, 88 cases (98\%) were intentional ingestions \cite{Elghali_2016, Karp_1991b, Lee_2007}, 30 cases (33\%) had a psychiatric history documented \cite{Elghali_2016, Karp_1991b, Lee_2007}, 2 cases (2\%) had a history of prior ingestion \cite{Elghali_2016}. No cases were reported for were psychiatric inpatients, were displaced people, were under the influence of alcohol at the time of ingestion, and had a severe disability history.
\paragraph*{Motivation}  70 cases (78\%) reported protest motivation \cite{Elghali_2016, Karp_1991b, Lee_2007}, 12 cases (13\%) reported psychiatric motivation \cite{Karp_1991b}, 6 cases (7\%) reported self-harm motivation \cite{Elghali_2016, Karp_1991b}. No cases were reported for psychosocial motivation and other motivation.
\paragraph*{Object Characteristics}  68 cases (76\%) involved sharp object ingestion \cite{Elghali_2016, Karp_1991b, Lee_2007}, 32 cases (36\%) involved long (\textgreater 5cm) object ingestion \cite{Lee_2007}, 25 cases (28\%) involved ingestion of multiple objects \cite{Elghali_2016, Lee_2007}. No cases were reported for button battery ingestion, magnet ingestion, and involved large diameter (\textgreater 2.5cm) object ingestion.
\paragraph*{Outcomes}  47 cases (52\%) underwent endoscopic intervention \cite{Elghali_2016, Lee_2007}, 29 cases (32\%) were managed conservatively \cite{Elghali_2016, Karp_1991b}, 15 cases (17\%) underwent surgical intervention \cite{Elghali_2016, Karp_1991b, Lee_2007}, 6 cases (7\%) reported complications \cite{Lee_2007}, 1 case (1\%) died \cite{Elghali_2016}.
\paragraph*{Geographical Location}Cases were recorded in 33 countries: 13 cases from USA \cite{Alao_2006i, Ataya_2013, Bhumi_2024f, Fry_2010, Guinan_2019f, Hardy_2023g, Jehangir_2019h, Kerestes_2019, Kumar_2001, Liu_2005, Tammana_2012j, Tay_2004, Tupesis_2004f}; 7 cases from India \cite{Bhasin_2014, Bhattacharjee_2008, Kar_2015, Kariholu_2008, Kumar_2019f, Misra_2013, Wadhwa_2015e} and UK \cite{Beecroft_1998, Berry_2021e, Cauchi_2002, Cox_2007, Gardner_2017h, Qureshi_2016}; 6 cases from Bulgaria \cite{Losanoff_1996, Losanoff_1997e}; 5 cases from Iran \cite{DivsalarP._2023a, Emamhadi_2018, Farhadi_2024h}; 4 cases from Turkey \cite{Akay_2015f, Atayan_2016, Tanrikulu_2015e, Yildiz_2016e}; 2 cases from China \cite{Jin_2023, Li_2013}, Poland \cite{Kobiela_2015, Wnęk_2015f}, and Spain \cite{CamachoDorado_2018, fjbuilsRepeatedBehaviorDeliberate2024}; 1 case from Australia \cite{Apikotoa_2022f}, Bahrain \cite{Ali_2020f}, Croatia \cite{Trgo_2012f}, Ecuador \cite{DelgadoSalazar_2020c}, Egypt \cite{Ali_2022g}, Ethiopia \cite{Mesfin_2022a}, Germany \cite{teWildt_2010}, Greece \cite{Sakellaridis_2008f}, Hungary \cite{Csaky_1998e}, Iraq \cite{Al-Faham_2020k}, Israel \cite{Goldman_1998f}, Italy \cite{Riva_2018j}, Japan \cite{Ohno_2005}, Nepal \cite{Thapa_2019f}, Netherlands \cite{Benoist_2019e}, Oman \cite{AlShaaibi_2021b}, Pakistan \cite{Yasin_2009}, Portugal \cite{Peixoto_2017f}, Qatar \cite{Ali_2017}, Saudi Arabia \cite{Sultan_2024f}, South Africa \cite{Sobnach_2011f}, Sweden \cite{Naji_2012f}, Switzerland \cite{Wildhaber_2005}, and Taiwan \cite{Chang_2017f}. \paragraph*{Gender} 43 cases (60\%) were male \cite{Akay_2015f, Al-Faham_2020k, Alao_2006i, Ali_2017, Ali_2022g, Apikotoa_2022f, Atayan_2016, Benoist_2019e, Berry_2021e, Bhumi_2024f, CamachoDorado_2018, Csaky_1998e, Emamhadi_2018, Farhadi_2024h, Fry_2010, Gardner_2017h, Guinan_2019f, Jehangir_2019h, Jin_2023, Kobiela_2015, Kumar_2001, Kumar_2019f, Liu_2005, Losanoff_1996, Losanoff_1997e, Mesfin_2022a, Misra_2013, Qureshi_2016, Riva_2018j, Sobnach_2011f, Tammana_2012j, Tanrikulu_2015e, Tay_2004, Thapa_2019f, Trgo_2012f, Wadhwa_2015e, Yasin_2009, teWildt_2010}, 28 cases (39\%) were female \cite{AlShaaibi_2021b, Ali_2020f, Ataya_2013, Beecroft_1998, Bhasin_2014, Bhattacharjee_2008, Cauchi_2002, Chang_2017f, Cox_2007, DelgadoSalazar_2020c, DivsalarP._2023a, Goldman_1998f, Hardy_2023g, Kar_2015, Kariholu_2008, Kerestes_2019, Li_2013, Naji_2012f, Ohno_2005, Peixoto_2017f, Sakellaridis_2008f, Sultan_2024f, Tupesis_2004f, Wildhaber_2005, Wnęk_2015f, Yildiz_2016e}, 1 case (1\%) had no gender recorded \cite{fjbuilsRepeatedBehaviorDeliberate2024}. \paragraph*{Age Group} 25 cases (35\%) were between 26 and 40 years of age \cite{Alao_2006i, Ali_2022g, Apikotoa_2022f, Ataya_2013, Benoist_2019e, Bhasin_2014, Chang_2017f, Cox_2007, DelgadoSalazar_2020c, Farhadi_2024h, Fry_2010, Gardner_2017h, Guinan_2019f, Jin_2023, Kumar_2019f, Losanoff_1996, Misra_2013, Qureshi_2016, Riva_2018j, Sakellaridis_2008f, Tammana_2012j, Trgo_2012f, Wnęk_2015f, Yildiz_2016e, fjbuilsRepeatedBehaviorDeliberate2024}, 18 cases (25\%) were between 18 and 25 years of age \cite{Akay_2015f, Ali_2017, Atayan_2016, Bhattacharjee_2008, Csaky_1998e, Kar_2015, Kariholu_2008, Kobiela_2015, Losanoff_1996, Losanoff_1997e, Mesfin_2022a, Peixoto_2017f, Sobnach_2011f, Tupesis_2004f, Yasin_2009}, 13 cases (18\%) were under 18 years of age \cite{AlShaaibi_2021b, Ali_2020f, Cauchi_2002, DivsalarP._2023a, Goldman_1998f, Liu_2005, Naji_2012f, Ohno_2005, Tanrikulu_2015e, Tay_2004, Wildhaber_2005}, 11 cases (15\%) were between 41 and 60 years of age \cite{Al-Faham_2020k, Bhumi_2024f, CamachoDorado_2018, Emamhadi_2018, Hardy_2023g, Jehangir_2019h, Kumar_2001, Sultan_2024f, Thapa_2019f, Wadhwa_2015e, teWildt_2010}, 3 cases (4\%) were over 60 years of age \cite{Beecroft_1998, Kerestes_2019, Li_2013}, 2 cases (3\%) had no age documented \cite{Berry_2021e}. \paragraph*{Population} 36 cases (50\%) had a psychiatric history \cite{AlShaaibi_2021b, Alao_2006i, Ali_2020f, Apikotoa_2022f, Ataya_2013, Atayan_2016, Beecroft_1998, CamachoDorado_2018, Chang_2017f, DelgadoSalazar_2020c, DivsalarP._2023a, Farhadi_2024h, Fry_2010, Guinan_2019f, Hardy_2023g, Jehangir_2019h, Jin_2023, Kar_2015, Kerestes_2019, Kobiela_2015, Kumar_2001, Kumar_2019f, Liu_2005, Mesfin_2022a, Misra_2013, Ohno_2005, Peixoto_2017f, Sakellaridis_2008f, Sultan_2024f, Tammana_2012j, Tanrikulu_2015e, Yildiz_2016e, fjbuilsRepeatedBehaviorDeliberate2024, teWildt_2010}, 19 cases (26\%) had ingested previously \cite{Alao_2006i, Apikotoa_2022f, Berry_2021e, Bhattacharjee_2008, Csaky_1998e, DivsalarP._2023a, Emamhadi_2018, Guinan_2019f, Jehangir_2019h, Jin_2023, Liu_2005, Sakellaridis_2008f, Tanrikulu_2015e, Thapa_2019f, Yildiz_2016e, fjbuilsRepeatedBehaviorDeliberate2024, teWildt_2010}, 12 cases (17\%) were detained persons \cite{Alao_2006i, Ali_2022g, Apikotoa_2022f, Losanoff_1996, Losanoff_1997e, Qureshi_2016, Tammana_2012j, Trgo_2012f}, 7 cases (10\%) were severely disabled \cite{Atayan_2016, Kerestes_2019, Liu_2005, Ohno_2005, Peixoto_2017f, Yildiz_2016e, teWildt_2010}, 4 cases (6\%) were psychiatric inpatients \cite{DivsalarP._2023a, fjbuilsRepeatedBehaviorDeliberate2024, teWildt_2010}, 3 cases (4\%) were under the influence of alcohol \cite{Benoist_2019e, Csaky_1998e, Thapa_2019f}, 2 cases (3\%) were displaced people \cite{Akay_2015f, Gardner_2017h}. \paragraph*{Motivation} 34 cases (47\%) had a psychiatric motivation \cite{Al-Faham_2020k, Alao_2006i, Ali_2020f, Apikotoa_2022f, Ataya_2013, Atayan_2016, Bhasin_2014, Bhattacharjee_2008, DelgadoSalazar_2020c, DivsalarP._2023a, Emamhadi_2018, Farhadi_2024h, Guinan_2019f, Hardy_2023g, Jehangir_2019h, Jin_2023, Kar_2015, Kariholu_2008, Kerestes_2019, Kobiela_2015, Kumar_2001, Kumar_2019f, Li_2013, Liu_2005, Misra_2013, Ohno_2005, Sakellaridis_2008f, Sultan_2024f, Tammana_2012j, Tanrikulu_2015e, Yasin_2009, teWildt_2010}, 21 cases (29\%) were motivated by self-harm intention \cite{Al-Faham_2020k, AlShaaibi_2021b, Alao_2006i, Ali_2017, CamachoDorado_2018, Chang_2017f, Cox_2007, Csaky_1998e, Fry_2010, Li_2013, Losanoff_1996, Losanoff_1997e, Mesfin_2022a, Sakellaridis_2008f, Tammana_2012j, Tanrikulu_2015e, fjbuilsRepeatedBehaviorDeliberate2024}, 17 cases (24\%) had a psychosocial motivation \cite{Akay_2015f, Benoist_2019e, Bhattacharjee_2008, Cauchi_2002, Goldman_1998f, Hardy_2023g, Kobiela_2015, Li_2013, Naji_2012f, Qureshi_2016, Riva_2018j, Sobnach_2011f, Tay_2004, Thapa_2019f, Tupesis_2004f, Wildhaber_2005, Wnęk_2015f}, 9 cases (12\%) were motivated by protest \cite{Bhumi_2024f, Gardner_2017h, Losanoff_1996, Losanoff_1997e, Tupesis_2004f}, 9 cases (12\%) had another documented motivation \cite{Ali_2020f, Ali_2022g, Emamhadi_2018, Guinan_2019f, Peixoto_2017f, Sakellaridis_2008f, Trgo_2012f, Wadhwa_2015e, Yildiz_2016e}. \paragraph*{Object Characteristics} 51 cases (71\%) ingested a large diameter object (\textgreater{}2.5cm) \cite{Akay_2015f, Al-Faham_2020k, AlShaaibi_2021b, Alao_2006i, Ali_2017, Ali_2022g, Apikotoa_2022f, Atayan_2016, Berry_2021e, Bhasin_2014, CamachoDorado_2018, Cauchi_2002, Chang_2017f, Cox_2007, Csaky_1998e, DivsalarP._2023a, Emamhadi_2018, Gardner_2017h, Guinan_2019f, Jehangir_2019h, Jin_2023, Kariholu_2008, Kerestes_2019, Kobiela_2015, Kumar_2001, Kumar_2019f, Losanoff_1996, Losanoff_1997e, Mesfin_2022a, Misra_2013, Naji_2012f, Ohno_2005, Peixoto_2017f, Qureshi_2016, Riva_2018j, Sakellaridis_2008f, Sultan_2024f, Tanrikulu_2015e, Thapa_2019f, Trgo_2012f, Wnęk_2015f, Yildiz_2016e, fjbuilsRepeatedBehaviorDeliberate2024, teWildt_2010}, 44 cases (61\%) ingested multiple objects \cite{Ali_2020f, Apikotoa_2022f, Ataya_2013, Atayan_2016, Beecroft_1998, Bhattacharjee_2008, Bhumi_2024f, CamachoDorado_2018, Cauchi_2002, Emamhadi_2018, Farhadi_2024h, Fry_2010, Goldman_1998f, Guinan_2019f, Hardy_2023g, Jehangir_2019h, Jin_2023, Kar_2015, Kariholu_2008, Kobiela_2015, Kumar_2001, Kumar_2019f, Li_2013, Liu_2005, Losanoff_1996, Mesfin_2022a, Misra_2013, Naji_2012f, Ohno_2005, Sobnach_2011f, Sultan_2024f, Tammana_2012j, Tanrikulu_2015e, Tay_2004, Thapa_2019f, Wadhwa_2015e, Wildhaber_2005, Yasin_2009, fjbuilsRepeatedBehaviorDeliberate2024, teWildt_2010}, 34 cases (47\%) ingested a sharp object \cite{AlShaaibi_2021b, Alao_2006i, Apikotoa_2022f, Ataya_2013, Benoist_2019e, Bhasin_2014, Bhattacharjee_2008, CamachoDorado_2018, Csaky_1998e, DelgadoSalazar_2020c, DivsalarP._2023a, Emamhadi_2018, Farhadi_2024h, Fry_2010, Guinan_2019f, Hardy_2023g, Jehangir_2019h, Jin_2023, Kariholu_2008, Kobiela_2015, Kumar_2019f, Losanoff_1996, Losanoff_1997e, Mesfin_2022a, Misra_2013, Sobnach_2011f, Yasin_2009, teWildt_2010}, 32 cases (44\%) ingested a long object (\textgreater{}5cm) \cite{Al-Faham_2020k, AlShaaibi_2021b, Ali_2017, Ali_2022g, Atayan_2016, Bhasin_2014, CamachoDorado_2018, Chang_2017f, Cox_2007, Csaky_1998e, DivsalarP._2023a, Emamhadi_2018, Fry_2010, Gardner_2017h, Jin_2023, Kariholu_2008, Kerestes_2019, Kobiela_2015, Kumar_2019f, Mesfin_2022a, Misra_2013, Ohno_2005, Qureshi_2016, Sakellaridis_2008f, Sultan_2024f, Thapa_2019f, Trgo_2012f, Yasin_2009, Yildiz_2016e, teWildt_2010}, 9 cases (12\%) ingested a magnet \cite{Ali_2020f, Bhumi_2024f, Cauchi_2002, Liu_2005, Naji_2012f, Ohno_2005, Tanrikulu_2015e, Tay_2004, Wildhaber_2005}, 2 cases (3\%) ingested a button battery \cite{Berry_2021e, Bhumi_2024f}. \paragraph*{Outcomes} 48 cases (67\%) experienced a complication \cite{Ali_2017, Ali_2020f, Apikotoa_2022f, Atayan_2016, Beecroft_1998, Benoist_2019e, Berry_2021e, Bhasin_2014, Bhumi_2024f, CamachoDorado_2018, Cauchi_2002, Cox_2007, Csaky_1998e, DelgadoSalazar_2020c, DivsalarP._2023a, Emamhadi_2018, Farhadi_2024h, Fry_2010, Gardner_2017h, Goldman_1998f, Jin_2023, Kariholu_2008, Kerestes_2019, Kobiela_2015, Kumar_2001, Kumar_2019f, Liu_2005, Losanoff_1996, Mesfin_2022a, Misra_2013, Naji_2012f, Ohno_2005, Sakellaridis_2008f, Sobnach_2011f, Sultan_2024f, Tanrikulu_2015e, Tay_2004, Thapa_2019f, Trgo_2012f, Tupesis_2004f, Wildhaber_2005, Wnęk_2015f, Yasin_2009, Yildiz_2016e}, 44 cases (61\%) underwent surgery \cite{Al-Faham_2020k, AlShaaibi_2021b, Alao_2006i, Ali_2017, Ali_2020f, Atayan_2016, Beecroft_1998, Bhasin_2014, CamachoDorado_2018, Cauchi_2002, Chang_2017f, Cox_2007, Csaky_1998e, DelgadoSalazar_2020c, DivsalarP._2023a, Farhadi_2024h, Fry_2010, Gardner_2017h, Jin_2023, Kariholu_2008, Kerestes_2019, Kobiela_2015, Kumar_2019f, Liu_2005, Losanoff_1996, Losanoff_1997e, Mesfin_2022a, Misra_2013, Naji_2012f, Sobnach_2011f, Tanrikulu_2015e, Tay_2004, Thapa_2019f, Tupesis_2004f, Wildhaber_2005, Wnęk_2015f, Yasin_2009, Yildiz_2016e, fjbuilsRepeatedBehaviorDeliberate2024}, 31 cases (43\%) underwent endoscopy \cite{Akay_2015f, Ali_2022g, Apikotoa_2022f, Atayan_2016, Benoist_2019e, Berry_2021e, Bhasin_2014, Bhumi_2024f, CamachoDorado_2018, Chang_2017f, DelgadoSalazar_2020c, Gardner_2017h, Guinan_2019f, Hardy_2023g, Jehangir_2019h, Kariholu_2008, Li_2013, Liu_2005, Ohno_2005, Peixoto_2017f, Qureshi_2016, Riva_2018j, Sakellaridis_2008f, Sultan_2024f, Tammana_2012j, Tanrikulu_2015e, Trgo_2012f, Wadhwa_2015e, Wnęk_2015f, teWildt_2010}, 7 cases (10\%) were managed conservatively \cite{Ataya_2013, Bhattacharjee_2008, DivsalarP._2023a, Emamhadi_2018, Goldman_1998f, Kar_2015, Kumar_2001}, 2 cases (3\%) died \cite{Emamhadi_2018, Kumar_2001}. All 90 were male gender. 90 cases (100\%) were detained at the time of ingestion \cite{Elghali_2016, Karp_1991b, Lee_2007}, 88 cases (98\%) were intentional ingestions \cite{Elghali_2016, Karp_1991b, Lee_2007}, 30 cases (33\%) had a psychiatric history documented \cite{Elghali_2016, Karp_1991b, Lee_2007}, 2 cases (2\%) had a history of prior ingestion \cite{Elghali_2016}. No cases were reported for were psychiatric inpatients, were displaced people, were under the influence of alcohol at the time of ingestion, and had a severe disability history.
\paragraph*{Motivation}  70 cases (78\%) reported protest motivation \cite{Elghali_2016, Karp_1991b, Lee_2007}, 12 cases (13\%) reported psychiatric motivation \cite{Karp_1991b}, 6 cases (7\%) reported self-harm motivation \cite{Elghali_2016, Karp_1991b}. No cases were reported for psychosocial motivation and other motivation.
\paragraph*{Object Characteristics}  68 cases (76\%) involved sharp object ingestion \cite{Elghali_2016, Karp_1991b, Lee_2007}, 32 cases (36\%) involved long (\textgreater 5cm) object ingestion \cite{Lee_2007}, 25 cases (28\%) involved ingestion of multiple objects \cite{Elghali_2016, Lee_2007}. No cases were reported for button battery ingestion, magnet ingestion, and involved large diameter (\textgreater 2.5cm) object ingestion.
\paragraph*{Outcomes}  47 cases (52\%) underwent endoscopic intervention \cite{Elghali_2016, Lee_2007}, 29 cases (32\%) were managed conservatively \cite{Elghali_2016, Karp_1991b}, 15 cases (17\%) underwent surgical intervention \cite{Elghali_2016, Karp_1991b, Lee_2007}, 6 cases (7\%) reported complications \cite{Lee_2007}, 1 case (1\%) died \cite{Elghali_2016}.
\paragraph*{Geographical Location}Cases were recorded in 33 countries: 13 cases from USA \cite{Alao_2006i, Ataya_2013, Bhumi_2024f, Fry_2010, Guinan_2019f, Hardy_2023g, Jehangir_2019h, Kerestes_2019, Kumar_2001, Liu_2005, Tammana_2012j, Tay_2004, Tupesis_2004f}; 7 cases from India \cite{Bhasin_2014, Bhattacharjee_2008, Kar_2015, Kariholu_2008, Kumar_2019f, Misra_2013, Wadhwa_2015e} and UK \cite{Beecroft_1998, Berry_2021e, Cauchi_2002, Cox_2007, Gardner_2017h, Qureshi_2016}; 6 cases from Bulgaria \cite{Losanoff_1996, Losanoff_1997e}; 5 cases from Iran \cite{DivsalarP._2023a, Emamhadi_2018, Farhadi_2024h}; 4 cases from Turkey \cite{Akay_2015f, Atayan_2016, Tanrikulu_2015e, Yildiz_2016e}; 2 cases from China \cite{Jin_2023, Li_2013}, Poland \cite{Kobiela_2015, Wnęk_2015f}, and Spain \cite{CamachoDorado_2018, fjbuilsRepeatedBehaviorDeliberate2024}; 1 case from Australia \cite{Apikotoa_2022f}, Bahrain \cite{Ali_2020f}, Croatia \cite{Trgo_2012f}, Ecuador \cite{DelgadoSalazar_2020c}, Egypt \cite{Ali_2022g}, Ethiopia \cite{Mesfin_2022a}, Germany \cite{teWildt_2010}, Greece \cite{Sakellaridis_2008f}, Hungary \cite{Csaky_1998e}, Iraq \cite{Al-Faham_2020k}, Israel \cite{Goldman_1998f}, Italy \cite{Riva_2018j}, Japan \cite{Ohno_2005}, Nepal \cite{Thapa_2019f}, Netherlands \cite{Benoist_2019e}, Oman \cite{AlShaaibi_2021b}, Pakistan \cite{Yasin_2009}, Portugal \cite{Peixoto_2017f}, Qatar \cite{Ali_2017}, Saudi Arabia \cite{Sultan_2024f}, South Africa \cite{Sobnach_2011f}, Sweden \cite{Naji_2012f}, Switzerland \cite{Wildhaber_2005}, and Taiwan \cite{Chang_2017f}. \paragraph*{Gender} 43 cases (60\%) were male \cite{Akay_2015f, Al-Faham_2020k, Alao_2006i, Ali_2017, Ali_2022g, Apikotoa_2022f, Atayan_2016, Benoist_2019e, Berry_2021e, Bhumi_2024f, CamachoDorado_2018, Csaky_1998e, Emamhadi_2018, Farhadi_2024h, Fry_2010, Gardner_2017h, Guinan_2019f, Jehangir_2019h, Jin_2023, Kobiela_2015, Kumar_2001, Kumar_2019f, Liu_2005, Losanoff_1996, Losanoff_1997e, Mesfin_2022a, Misra_2013, Qureshi_2016, Riva_2018j, Sobnach_2011f, Tammana_2012j, Tanrikulu_2015e, Tay_2004, Thapa_2019f, Trgo_2012f, Wadhwa_2015e, Yasin_2009, teWildt_2010}, 28 cases (39\%) were female \cite{AlShaaibi_2021b, Ali_2020f, Ataya_2013, Beecroft_1998, Bhasin_2014, Bhattacharjee_2008, Cauchi_2002, Chang_2017f, Cox_2007, DelgadoSalazar_2020c, DivsalarP._2023a, Goldman_1998f, Hardy_2023g, Kar_2015, Kariholu_2008, Kerestes_2019, Li_2013, Naji_2012f, Ohno_2005, Peixoto_2017f, Sakellaridis_2008f, Sultan_2024f, Tupesis_2004f, Wildhaber_2005, Wnęk_2015f, Yildiz_2016e}, 1 case (1\%) had no gender recorded \cite{fjbuilsRepeatedBehaviorDeliberate2024}. \paragraph*{Age Group} 25 cases (35\%) were between 26 and 40 years of age \cite{Alao_2006i, Ali_2022g, Apikotoa_2022f, Ataya_2013, Benoist_2019e, Bhasin_2014, Chang_2017f, Cox_2007, DelgadoSalazar_2020c, Farhadi_2024h, Fry_2010, Gardner_2017h, Guinan_2019f, Jin_2023, Kumar_2019f, Losanoff_1996, Misra_2013, Qureshi_2016, Riva_2018j, Sakellaridis_2008f, Tammana_2012j, Trgo_2012f, Wnęk_2015f, Yildiz_2016e, fjbuilsRepeatedBehaviorDeliberate2024}, 18 cases (25\%) were between 18 and 25 years of age \cite{Akay_2015f, Ali_2017, Atayan_2016, Bhattacharjee_2008, Csaky_1998e, Kar_2015, Kariholu_2008, Kobiela_2015, Losanoff_1996, Losanoff_1997e, Mesfin_2022a, Peixoto_2017f, Sobnach_2011f, Tupesis_2004f, Yasin_2009}, 13 cases (18\%) were under 18 years of age \cite{AlShaaibi_2021b, Ali_2020f, Cauchi_2002, DivsalarP._2023a, Goldman_1998f, Liu_2005, Naji_2012f, Ohno_2005, Tanrikulu_2015e, Tay_2004, Wildhaber_2005}, 11 cases (15\%) were between 41 and 60 years of age \cite{Al-Faham_2020k, Bhumi_2024f, CamachoDorado_2018, Emamhadi_2018, Hardy_2023g, Jehangir_2019h, Kumar_2001, Sultan_2024f, Thapa_2019f, Wadhwa_2015e, teWildt_2010}, 3 cases (4\%) were over 60 years of age \cite{Beecroft_1998, Kerestes_2019, Li_2013}, 2 cases (3\%) had no age documented \cite{Berry_2021e}. \paragraph*{Population} 36 cases (50\%) had a psychiatric history \cite{AlShaaibi_2021b, Alao_2006i, Ali_2020f, Apikotoa_2022f, Ataya_2013, Atayan_2016, Beecroft_1998, CamachoDorado_2018, Chang_2017f, DelgadoSalazar_2020c, DivsalarP._2023a, Farhadi_2024h, Fry_2010, Guinan_2019f, Hardy_2023g, Jehangir_2019h, Jin_2023, Kar_2015, Kerestes_2019, Kobiela_2015, Kumar_2001, Kumar_2019f, Liu_2005, Mesfin_2022a, Misra_2013, Ohno_2005, Peixoto_2017f, Sakellaridis_2008f, Sultan_2024f, Tammana_2012j, Tanrikulu_2015e, Yildiz_2016e, fjbuilsRepeatedBehaviorDeliberate2024, teWildt_2010}, 19 cases (26\%) had ingested previously \cite{Alao_2006i, Apikotoa_2022f, Berry_2021e, Bhattacharjee_2008, Csaky_1998e, DivsalarP._2023a, Emamhadi_2018, Guinan_2019f, Jehangir_2019h, Jin_2023, Liu_2005, Sakellaridis_2008f, Tanrikulu_2015e, Thapa_2019f, Yildiz_2016e, fjbuilsRepeatedBehaviorDeliberate2024, teWildt_2010}, 12 cases (17\%) were detained persons \cite{Alao_2006i, Ali_2022g, Apikotoa_2022f, Losanoff_1996, Losanoff_1997e, Qureshi_2016, Tammana_2012j, Trgo_2012f}, 7 cases (10\%) were severely disabled \cite{Atayan_2016, Kerestes_2019, Liu_2005, Ohno_2005, Peixoto_2017f, Yildiz_2016e, teWildt_2010}, 4 cases (6\%) were psychiatric inpatients \cite{DivsalarP._2023a, fjbuilsRepeatedBehaviorDeliberate2024, teWildt_2010}, 3 cases (4\%) were under the influence of alcohol \cite{Benoist_2019e, Csaky_1998e, Thapa_2019f}, 2 cases (3\%) were displaced people \cite{Akay_2015f, Gardner_2017h}. \paragraph*{Motivation} 34 cases (47\%) had a psychiatric motivation \cite{Al-Faham_2020k, Alao_2006i, Ali_2020f, Apikotoa_2022f, Ataya_2013, Atayan_2016, Bhasin_2014, Bhattacharjee_2008, DelgadoSalazar_2020c, DivsalarP._2023a, Emamhadi_2018, Farhadi_2024h, Guinan_2019f, Hardy_2023g, Jehangir_2019h, Jin_2023, Kar_2015, Kariholu_2008, Kerestes_2019, Kobiela_2015, Kumar_2001, Kumar_2019f, Li_2013, Liu_2005, Misra_2013, Ohno_2005, Sakellaridis_2008f, Sultan_2024f, Tammana_2012j, Tanrikulu_2015e, Yasin_2009, teWildt_2010}, 21 cases (29\%) were motivated by self-harm intention \cite{Al-Faham_2020k, AlShaaibi_2021b, Alao_2006i, Ali_2017, CamachoDorado_2018, Chang_2017f, Cox_2007, Csaky_1998e, Fry_2010, Li_2013, Losanoff_1996, Losanoff_1997e, Mesfin_2022a, Sakellaridis_2008f, Tammana_2012j, Tanrikulu_2015e, fjbuilsRepeatedBehaviorDeliberate2024}, 17 cases (24\%) had a psychosocial motivation \cite{Akay_2015f, Benoist_2019e, Bhattacharjee_2008, Cauchi_2002, Goldman_1998f, Hardy_2023g, Kobiela_2015, Li_2013, Naji_2012f, Qureshi_2016, Riva_2018j, Sobnach_2011f, Tay_2004, Thapa_2019f, Tupesis_2004f, Wildhaber_2005, Wnęk_2015f}, 9 cases (12\%) were motivated by protest \cite{Bhumi_2024f, Gardner_2017h, Losanoff_1996, Losanoff_1997e, Tupesis_2004f}, 9 cases (12\%) had another documented motivation \cite{Ali_2020f, Ali_2022g, Emamhadi_2018, Guinan_2019f, Peixoto_2017f, Sakellaridis_2008f, Trgo_2012f, Wadhwa_2015e, Yildiz_2016e}. \paragraph*{Object Characteristics} 51 cases (71\%) ingested a large diameter object (\textgreater{}2.5cm) \cite{Akay_2015f, Al-Faham_2020k, AlShaaibi_2021b, Alao_2006i, Ali_2017, Ali_2022g, Apikotoa_2022f, Atayan_2016, Berry_2021e, Bhasin_2014, CamachoDorado_2018, Cauchi_2002, Chang_2017f, Cox_2007, Csaky_1998e, DivsalarP._2023a, Emamhadi_2018, Gardner_2017h, Guinan_2019f, Jehangir_2019h, Jin_2023, Kariholu_2008, Kerestes_2019, Kobiela_2015, Kumar_2001, Kumar_2019f, Losanoff_1996, Losanoff_1997e, Mesfin_2022a, Misra_2013, Naji_2012f, Ohno_2005, Peixoto_2017f, Qureshi_2016, Riva_2018j, Sakellaridis_2008f, Sultan_2024f, Tanrikulu_2015e, Thapa_2019f, Trgo_2012f, Wnęk_2015f, Yildiz_2016e, fjbuilsRepeatedBehaviorDeliberate2024, teWildt_2010}, 44 cases (61\%) ingested multiple objects \cite{Ali_2020f, Apikotoa_2022f, Ataya_2013, Atayan_2016, Beecroft_1998, Bhattacharjee_2008, Bhumi_2024f, CamachoDorado_2018, Cauchi_2002, Emamhadi_2018, Farhadi_2024h, Fry_2010, Goldman_1998f, Guinan_2019f, Hardy_2023g, Jehangir_2019h, Jin_2023, Kar_2015, Kariholu_2008, Kobiela_2015, Kumar_2001, Kumar_2019f, Li_2013, Liu_2005, Losanoff_1996, Mesfin_2022a, Misra_2013, Naji_2012f, Ohno_2005, Sobnach_2011f, Sultan_2024f, Tammana_2012j, Tanrikulu_2015e, Tay_2004, Thapa_2019f, Wadhwa_2015e, Wildhaber_2005, Yasin_2009, fjbuilsRepeatedBehaviorDeliberate2024, teWildt_2010}, 34 cases (47\%) ingested a sharp object \cite{AlShaaibi_2021b, Alao_2006i, Apikotoa_2022f, Ataya_2013, Benoist_2019e, Bhasin_2014, Bhattacharjee_2008, CamachoDorado_2018, Csaky_1998e, DelgadoSalazar_2020c, DivsalarP._2023a, Emamhadi_2018, Farhadi_2024h, Fry_2010, Guinan_2019f, Hardy_2023g, Jehangir_2019h, Jin_2023, Kariholu_2008, Kobiela_2015, Kumar_2019f, Losanoff_1996, Losanoff_1997e, Mesfin_2022a, Misra_2013, Sobnach_2011f, Yasin_2009, teWildt_2010}, 32 cases (44\%) ingested a long object (\textgreater{}5cm) \cite{Al-Faham_2020k, AlShaaibi_2021b, Ali_2017, Ali_2022g, Atayan_2016, Bhasin_2014, CamachoDorado_2018, Chang_2017f, Cox_2007, Csaky_1998e, DivsalarP._2023a, Emamhadi_2018, Fry_2010, Gardner_2017h, Jin_2023, Kariholu_2008, Kerestes_2019, Kobiela_2015, Kumar_2019f, Mesfin_2022a, Misra_2013, Ohno_2005, Qureshi_2016, Sakellaridis_2008f, Sultan_2024f, Thapa_2019f, Trgo_2012f, Yasin_2009, Yildiz_2016e, teWildt_2010}, 9 cases (12\%) ingested a magnet \cite{Ali_2020f, Bhumi_2024f, Cauchi_2002, Liu_2005, Naji_2012f, Ohno_2005, Tanrikulu_2015e, Tay_2004, Wildhaber_2005}, 2 cases (3\%) ingested a button battery \cite{Berry_2021e, Bhumi_2024f}. \paragraph*{Outcomes} 48 cases (67\%) experienced a complication \cite{Ali_2017, Ali_2020f, Apikotoa_2022f, Atayan_2016, Beecroft_1998, Benoist_2019e, Berry_2021e, Bhasin_2014, Bhumi_2024f, CamachoDorado_2018, Cauchi_2002, Cox_2007, Csaky_1998e, DelgadoSalazar_2020c, DivsalarP._2023a, Emamhadi_2018, Farhadi_2024h, Fry_2010, Gardner_2017h, Goldman_1998f, Jin_2023, Kariholu_2008, Kerestes_2019, Kobiela_2015, Kumar_2001, Kumar_2019f, Liu_2005, Losanoff_1996, Mesfin_2022a, Misra_2013, Naji_2012f, Ohno_2005, Sakellaridis_2008f, Sobnach_2011f, Sultan_2024f, Tanrikulu_2015e, Tay_2004, Thapa_2019f, Trgo_2012f, Tupesis_2004f, Wildhaber_2005, Wnęk_2015f, Yasin_2009, Yildiz_2016e}, 44 cases (61\%) underwent surgery \cite{Al-Faham_2020k, AlShaaibi_2021b, Alao_2006i, Ali_2017, Ali_2020f, Atayan_2016, Beecroft_1998, Bhasin_2014, CamachoDorado_2018, Cauchi_2002, Chang_2017f, Cox_2007, Csaky_1998e, DelgadoSalazar_2020c, DivsalarP._2023a, Farhadi_2024h, Fry_2010, Gardner_2017h, Jin_2023, Kariholu_2008, Kerestes_2019, Kobiela_2015, Kumar_2019f, Liu_2005, Losanoff_1996, Losanoff_1997e, Mesfin_2022a, Misra_2013, Naji_2012f, Sobnach_2011f, Tanrikulu_2015e, Tay_2004, Thapa_2019f, Tupesis_2004f, Wildhaber_2005, Wnęk_2015f, Yasin_2009, Yildiz_2016e, fjbuilsRepeatedBehaviorDeliberate2024}, 31 cases (43\%) underwent endoscopy \cite{Akay_2015f, Ali_2022g, Apikotoa_2022f, Atayan_2016, Benoist_2019e, Berry_2021e, Bhasin_2014, Bhumi_2024f, CamachoDorado_2018, Chang_2017f, DelgadoSalazar_2020c, Gardner_2017h, Guinan_2019f, Hardy_2023g, Jehangir_2019h, Kariholu_2008, Li_2013, Liu_2005, Ohno_2005, Peixoto_2017f, Qureshi_2016, Riva_2018j, Sakellaridis_2008f, Sultan_2024f, Tammana_2012j, Tanrikulu_2015e, Trgo_2012f, Wadhwa_2015e, Wnęk_2015f, teWildt_2010}, 7 cases (10\%) were managed conservatively \cite{Ataya_2013, Bhattacharjee_2008, DivsalarP._2023a, Emamhadi_2018, Goldman_1998f, Kar_2015, Kumar_2001}, 2 cases (3\%) died \cite{Emamhadi_2018, Kumar_2001}. All 90 were male gender. 90 cases (100\%) were detained at the time of ingestion \cite{Elghali_2016, Karp_1991b, Lee_2007}, 88 cases (98\%) were intentional ingestions \cite{Elghali_2016, Karp_1991b, Lee_2007}, 30 cases (33\%) had a psychiatric history documented \cite{Elghali_2016, Karp_1991b, Lee_2007}, 2 cases (2\%) had a history of prior ingestion \cite{Elghali_2016}. No cases were reported for were psychiatric inpatients, were displaced people, were under the influence of alcohol at the time of ingestion, and had a severe disability history.
\paragraph*{Motivation}  70 cases (78\%) reported protest motivation \cite{Elghali_2016, Karp_1991b, Lee_2007}, 12 cases (13\%) reported psychiatric motivation \cite{Karp_1991b}, 6 cases (7\%) reported self-harm motivation \cite{Elghali_2016, Karp_1991b}. No cases were reported for psychosocial motivation and other motivation.
\paragraph*{Object Characteristics}  68 cases (76\%) involved sharp object ingestion \cite{Elghali_2016, Karp_1991b, Lee_2007}, 32 cases (36\%) involved long (\textgreater 5cm) object ingestion \cite{Lee_2007}, 25 cases (28\%) involved ingestion of multiple objects \cite{Elghali_2016, Lee_2007}. No cases were reported for button battery ingestion, magnet ingestion, and involved large diameter (\textgreater 2.5cm) object ingestion.
\paragraph*{Outcomes}  47 cases (52\%) underwent endoscopic intervention \cite{Elghali_2016, Lee_2007}, 29 cases (32\%) were managed conservatively \cite{Elghali_2016, Karp_1991b}, 15 cases (17\%) underwent surgical intervention \cite{Elghali_2016, Karp_1991b, Lee_2007}, 6 cases (7\%) reported complications \cite{Lee_2007}, 1 case (1\%) died \cite{Elghali_2016}.
\paragraph*{Geographical Location}Cases were recorded in 33 countries: 13 cases from USA \cite{Alao_2006i, Ataya_2013, Bhumi_2024f, Fry_2010, Guinan_2019f, Hardy_2023g, Jehangir_2019h, Kerestes_2019, Kumar_2001, Liu_2005, Tammana_2012j, Tay_2004, Tupesis_2004f}; 7 cases from India \cite{Bhasin_2014, Bhattacharjee_2008, Kar_2015, Kariholu_2008, Kumar_2019f, Misra_2013, Wadhwa_2015e} and UK \cite{Beecroft_1998, Berry_2021e, Cauchi_2002, Cox_2007, Gardner_2017h, Qureshi_2016}; 6 cases from Bulgaria \cite{Losanoff_1996, Losanoff_1997e}; 5 cases from Iran \cite{DivsalarP._2023a, Emamhadi_2018, Farhadi_2024h}; 4 cases from Turkey \cite{Akay_2015f, Atayan_2016, Tanrikulu_2015e, Yildiz_2016e}; 2 cases from China \cite{Jin_2023, Li_2013}, Poland \cite{Kobiela_2015, Wnęk_2015f}, and Spain \cite{CamachoDorado_2018, fjbuilsRepeatedBehaviorDeliberate2024}; 1 case from Australia \cite{Apikotoa_2022f}, Bahrain \cite{Ali_2020f}, Croatia \cite{Trgo_2012f}, Ecuador \cite{DelgadoSalazar_2020c}, Egypt \cite{Ali_2022g}, Ethiopia \cite{Mesfin_2022a}, Germany \cite{teWildt_2010}, Greece \cite{Sakellaridis_2008f}, Hungary \cite{Csaky_1998e}, Iraq \cite{Al-Faham_2020k}, Israel \cite{Goldman_1998f}, Italy \cite{Riva_2018j}, Japan \cite{Ohno_2005}, Nepal \cite{Thapa_2019f}, Netherlands \cite{Benoist_2019e}, Oman \cite{AlShaaibi_2021b}, Pakistan \cite{Yasin_2009}, Portugal \cite{Peixoto_2017f}, Qatar \cite{Ali_2017}, Saudi Arabia \cite{Sultan_2024f}, South Africa \cite{Sobnach_2011f}, Sweden \cite{Naji_2012f}, Switzerland \cite{Wildhaber_2005}, and Taiwan \cite{Chang_2017f}. \paragraph*{Gender} 43 cases (60\%) were male \cite{Akay_2015f, Al-Faham_2020k, Alao_2006i, Ali_2017, Ali_2022g, Apikotoa_2022f, Atayan_2016, Benoist_2019e, Berry_2021e, Bhumi_2024f, CamachoDorado_2018, Csaky_1998e, Emamhadi_2018, Farhadi_2024h, Fry_2010, Gardner_2017h, Guinan_2019f, Jehangir_2019h, Jin_2023, Kobiela_2015, Kumar_2001, Kumar_2019f, Liu_2005, Losanoff_1996, Losanoff_1997e, Mesfin_2022a, Misra_2013, Qureshi_2016, Riva_2018j, Sobnach_2011f, Tammana_2012j, Tanrikulu_2015e, Tay_2004, Thapa_2019f, Trgo_2012f, Wadhwa_2015e, Yasin_2009, teWildt_2010}, 28 cases (39\%) were female \cite{AlShaaibi_2021b, Ali_2020f, Ataya_2013, Beecroft_1998, Bhasin_2014, Bhattacharjee_2008, Cauchi_2002, Chang_2017f, Cox_2007, DelgadoSalazar_2020c, DivsalarP._2023a, Goldman_1998f, Hardy_2023g, Kar_2015, Kariholu_2008, Kerestes_2019, Li_2013, Naji_2012f, Ohno_2005, Peixoto_2017f, Sakellaridis_2008f, Sultan_2024f, Tupesis_2004f, Wildhaber_2005, Wnęk_2015f, Yildiz_2016e}, 1 case (1\%) had no gender recorded \cite{fjbuilsRepeatedBehaviorDeliberate2024}. \paragraph*{Age Group} 25 cases (35\%) were between 26 and 40 years of age \cite{Alao_2006i, Ali_2022g, Apikotoa_2022f, Ataya_2013, Benoist_2019e, Bhasin_2014, Chang_2017f, Cox_2007, DelgadoSalazar_2020c, Farhadi_2024h, Fry_2010, Gardner_2017h, Guinan_2019f, Jin_2023, Kumar_2019f, Losanoff_1996, Misra_2013, Qureshi_2016, Riva_2018j, Sakellaridis_2008f, Tammana_2012j, Trgo_2012f, Wnęk_2015f, Yildiz_2016e, fjbuilsRepeatedBehaviorDeliberate2024}, 18 cases (25\%) were between 18 and 25 years of age \cite{Akay_2015f, Ali_2017, Atayan_2016, Bhattacharjee_2008, Csaky_1998e, Kar_2015, Kariholu_2008, Kobiela_2015, Losanoff_1996, Losanoff_1997e, Mesfin_2022a, Peixoto_2017f, Sobnach_2011f, Tupesis_2004f, Yasin_2009}, 13 cases (18\%) were under 18 years of age \cite{AlShaaibi_2021b, Ali_2020f, Cauchi_2002, DivsalarP._2023a, Goldman_1998f, Liu_2005, Naji_2012f, Ohno_2005, Tanrikulu_2015e, Tay_2004, Wildhaber_2005}, 11 cases (15\%) were between 41 and 60 years of age \cite{Al-Faham_2020k, Bhumi_2024f, CamachoDorado_2018, Emamhadi_2018, Hardy_2023g, Jehangir_2019h, Kumar_2001, Sultan_2024f, Thapa_2019f, Wadhwa_2015e, teWildt_2010}, 3 cases (4\%) were over 60 years of age \cite{Beecroft_1998, Kerestes_2019, Li_2013}, 2 cases (3\%) had no age documented \cite{Berry_2021e}. \paragraph*{Population} 36 cases (50\%) had a psychiatric history \cite{AlShaaibi_2021b, Alao_2006i, Ali_2020f, Apikotoa_2022f, Ataya_2013, Atayan_2016, Beecroft_1998, CamachoDorado_2018, Chang_2017f, DelgadoSalazar_2020c, DivsalarP._2023a, Farhadi_2024h, Fry_2010, Guinan_2019f, Hardy_2023g, Jehangir_2019h, Jin_2023, Kar_2015, Kerestes_2019, Kobiela_2015, Kumar_2001, Kumar_2019f, Liu_2005, Mesfin_2022a, Misra_2013, Ohno_2005, Peixoto_2017f, Sakellaridis_2008f, Sultan_2024f, Tammana_2012j, Tanrikulu_2015e, Yildiz_2016e, fjbuilsRepeatedBehaviorDeliberate2024, teWildt_2010}, 19 cases (26\%) had ingested previously \cite{Alao_2006i, Apikotoa_2022f, Berry_2021e, Bhattacharjee_2008, Csaky_1998e, DivsalarP._2023a, Emamhadi_2018, Guinan_2019f, Jehangir_2019h, Jin_2023, Liu_2005, Sakellaridis_2008f, Tanrikulu_2015e, Thapa_2019f, Yildiz_2016e, fjbuilsRepeatedBehaviorDeliberate2024, teWildt_2010}, 12 cases (17\%) were detained persons \cite{Alao_2006i, Ali_2022g, Apikotoa_2022f, Losanoff_1996, Losanoff_1997e, Qureshi_2016, Tammana_2012j, Trgo_2012f}, 7 cases (10\%) were severely disabled \cite{Atayan_2016, Kerestes_2019, Liu_2005, Ohno_2005, Peixoto_2017f, Yildiz_2016e, teWildt_2010}, 4 cases (6\%) were psychiatric inpatients \cite{DivsalarP._2023a, fjbuilsRepeatedBehaviorDeliberate2024, teWildt_2010}, 3 cases (4\%) were under the influence of alcohol \cite{Benoist_2019e, Csaky_1998e, Thapa_2019f}, 2 cases (3\%) were displaced people \cite{Akay_2015f, Gardner_2017h}. \paragraph*{Motivation} 34 cases (47\%) had a psychiatric motivation \cite{Al-Faham_2020k, Alao_2006i, Ali_2020f, Apikotoa_2022f, Ataya_2013, Atayan_2016, Bhasin_2014, Bhattacharjee_2008, DelgadoSalazar_2020c, DivsalarP._2023a, Emamhadi_2018, Farhadi_2024h, Guinan_2019f, Hardy_2023g, Jehangir_2019h, Jin_2023, Kar_2015, Kariholu_2008, Kerestes_2019, Kobiela_2015, Kumar_2001, Kumar_2019f, Li_2013, Liu_2005, Misra_2013, Ohno_2005, Sakellaridis_2008f, Sultan_2024f, Tammana_2012j, Tanrikulu_2015e, Yasin_2009, teWildt_2010}, 21 cases (29\%) were motivated by self-harm intention \cite{Al-Faham_2020k, AlShaaibi_2021b, Alao_2006i, Ali_2017, CamachoDorado_2018, Chang_2017f, Cox_2007, Csaky_1998e, Fry_2010, Li_2013, Losanoff_1996, Losanoff_1997e, Mesfin_2022a, Sakellaridis_2008f, Tammana_2012j, Tanrikulu_2015e, fjbuilsRepeatedBehaviorDeliberate2024}, 17 cases (24\%) had a psychosocial motivation \cite{Akay_2015f, Benoist_2019e, Bhattacharjee_2008, Cauchi_2002, Goldman_1998f, Hardy_2023g, Kobiela_2015, Li_2013, Naji_2012f, Qureshi_2016, Riva_2018j, Sobnach_2011f, Tay_2004, Thapa_2019f, Tupesis_2004f, Wildhaber_2005, Wnęk_2015f}, 9 cases (12\%) were motivated by protest \cite{Bhumi_2024f, Gardner_2017h, Losanoff_1996, Losanoff_1997e, Tupesis_2004f}, 9 cases (12\%) had another documented motivation \cite{Ali_2020f, Ali_2022g, Emamhadi_2018, Guinan_2019f, Peixoto_2017f, Sakellaridis_2008f, Trgo_2012f, Wadhwa_2015e, Yildiz_2016e}. \paragraph*{Object Characteristics} 51 cases (71\%) ingested a large diameter object (\textgreater{}2.5cm) \cite{Akay_2015f, Al-Faham_2020k, AlShaaibi_2021b, Alao_2006i, Ali_2017, Ali_2022g, Apikotoa_2022f, Atayan_2016, Berry_2021e, Bhasin_2014, CamachoDorado_2018, Cauchi_2002, Chang_2017f, Cox_2007, Csaky_1998e, DivsalarP._2023a, Emamhadi_2018, Gardner_2017h, Guinan_2019f, Jehangir_2019h, Jin_2023, Kariholu_2008, Kerestes_2019, Kobiela_2015, Kumar_2001, Kumar_2019f, Losanoff_1996, Losanoff_1997e, Mesfin_2022a, Misra_2013, Naji_2012f, Ohno_2005, Peixoto_2017f, Qureshi_2016, Riva_2018j, Sakellaridis_2008f, Sultan_2024f, Tanrikulu_2015e, Thapa_2019f, Trgo_2012f, Wnęk_2015f, Yildiz_2016e, fjbuilsRepeatedBehaviorDeliberate2024, teWildt_2010}, 44 cases (61\%) ingested multiple objects \cite{Ali_2020f, Apikotoa_2022f, Ataya_2013, Atayan_2016, Beecroft_1998, Bhattacharjee_2008, Bhumi_2024f, CamachoDorado_2018, Cauchi_2002, Emamhadi_2018, Farhadi_2024h, Fry_2010, Goldman_1998f, Guinan_2019f, Hardy_2023g, Jehangir_2019h, Jin_2023, Kar_2015, Kariholu_2008, Kobiela_2015, Kumar_2001, Kumar_2019f, Li_2013, Liu_2005, Losanoff_1996, Mesfin_2022a, Misra_2013, Naji_2012f, Ohno_2005, Sobnach_2011f, Sultan_2024f, Tammana_2012j, Tanrikulu_2015e, Tay_2004, Thapa_2019f, Wadhwa_2015e, Wildhaber_2005, Yasin_2009, fjbuilsRepeatedBehaviorDeliberate2024, teWildt_2010}, 34 cases (47\%) ingested a sharp object \cite{AlShaaibi_2021b, Alao_2006i, Apikotoa_2022f, Ataya_2013, Benoist_2019e, Bhasin_2014, Bhattacharjee_2008, CamachoDorado_2018, Csaky_1998e, DelgadoSalazar_2020c, DivsalarP._2023a, Emamhadi_2018, Farhadi_2024h, Fry_2010, Guinan_2019f, Hardy_2023g, Jehangir_2019h, Jin_2023, Kariholu_2008, Kobiela_2015, Kumar_2019f, Losanoff_1996, Losanoff_1997e, Mesfin_2022a, Misra_2013, Sobnach_2011f, Yasin_2009, teWildt_2010}, 32 cases (44\%) ingested a long object (\textgreater{}5cm) \cite{Al-Faham_2020k, AlShaaibi_2021b, Ali_2017, Ali_2022g, Atayan_2016, Bhasin_2014, CamachoDorado_2018, Chang_2017f, Cox_2007, Csaky_1998e, DivsalarP._2023a, Emamhadi_2018, Fry_2010, Gardner_2017h, Jin_2023, Kariholu_2008, Kerestes_2019, Kobiela_2015, Kumar_2019f, Mesfin_2022a, Misra_2013, Ohno_2005, Qureshi_2016, Sakellaridis_2008f, Sultan_2024f, Thapa_2019f, Trgo_2012f, Yasin_2009, Yildiz_2016e, teWildt_2010}, 9 cases (12\%) ingested a magnet \cite{Ali_2020f, Bhumi_2024f, Cauchi_2002, Liu_2005, Naji_2012f, Ohno_2005, Tanrikulu_2015e, Tay_2004, Wildhaber_2005}, 2 cases (3\%) ingested a button battery \cite{Berry_2021e, Bhumi_2024f}. \paragraph*{Outcomes} 48 cases (67\%) experienced a complication \cite{Ali_2017, Ali_2020f, Apikotoa_2022f, Atayan_2016, Beecroft_1998, Benoist_2019e, Berry_2021e, Bhasin_2014, Bhumi_2024f, CamachoDorado_2018, Cauchi_2002, Cox_2007, Csaky_1998e, DelgadoSalazar_2020c, DivsalarP._2023a, Emamhadi_2018, Farhadi_2024h, Fry_2010, Gardner_2017h, Goldman_1998f, Jin_2023, Kariholu_2008, Kerestes_2019, Kobiela_2015, Kumar_2001, Kumar_2019f, Liu_2005, Losanoff_1996, Mesfin_2022a, Misra_2013, Naji_2012f, Ohno_2005, Sakellaridis_2008f, Sobnach_2011f, Sultan_2024f, Tanrikulu_2015e, Tay_2004, Thapa_2019f, Trgo_2012f, Tupesis_2004f, Wildhaber_2005, Wnęk_2015f, Yasin_2009, Yildiz_2016e}, 44 cases (61\%) underwent surgery \cite{Al-Faham_2020k, AlShaaibi_2021b, Alao_2006i, Ali_2017, Ali_2020f, Atayan_2016, Beecroft_1998, Bhasin_2014, CamachoDorado_2018, Cauchi_2002, Chang_2017f, Cox_2007, Csaky_1998e, DelgadoSalazar_2020c, DivsalarP._2023a, Farhadi_2024h, Fry_2010, Gardner_2017h, Jin_2023, Kariholu_2008, Kerestes_2019, Kobiela_2015, Kumar_2019f, Liu_2005, Losanoff_1996, Losanoff_1997e, Mesfin_2022a, Misra_2013, Naji_2012f, Sobnach_2011f, Tanrikulu_2015e, Tay_2004, Thapa_2019f, Tupesis_2004f, Wildhaber_2005, Wnęk_2015f, Yasin_2009, Yildiz_2016e, fjbuilsRepeatedBehaviorDeliberate2024}, 31 cases (43\%) underwent endoscopy \cite{Akay_2015f, Ali_2022g, Apikotoa_2022f, Atayan_2016, Benoist_2019e, Berry_2021e, Bhasin_2014, Bhumi_2024f, CamachoDorado_2018, Chang_2017f, DelgadoSalazar_2020c, Gardner_2017h, Guinan_2019f, Hardy_2023g, Jehangir_2019h, Kariholu_2008, Li_2013, Liu_2005, Ohno_2005, Peixoto_2017f, Qureshi_2016, Riva_2018j, Sakellaridis_2008f, Sultan_2024f, Tammana_2012j, Tanrikulu_2015e, Trgo_2012f, Wadhwa_2015e, Wnęk_2015f, teWildt_2010}, 7 cases (10\%) were managed conservatively \cite{Ataya_2013, Bhattacharjee_2008, DivsalarP._2023a, Emamhadi_2018, Goldman_1998f, Kar_2015, Kumar_2001}, 2 cases (3\%) died \cite{Emamhadi_2018, Kumar_2001}. All 90 were male gender. 90 cases (100\%) were detained at the time of ingestion \cite{Elghali_2016, Karp_1991b, Lee_2007}, 88 cases (98\%) were intentional ingestions \cite{Elghali_2016, Karp_1991b, Lee_2007}, 30 cases (33\%) had a psychiatric history documented \cite{Elghali_2016, Karp_1991b, Lee_2007}, 2 cases (2\%) had a history of prior ingestion \cite{Elghali_2016}. No cases were reported for were psychiatric inpatients, were displaced people, were under the influence of alcohol at the time of ingestion, and had a severe disability history.
\paragraph*{Motivation}  70 cases (78\%) reported protest motivation \cite{Elghali_2016, Karp_1991b, Lee_2007}, 12 cases (13\%) reported psychiatric motivation \cite{Karp_1991b}, 6 cases (7\%) reported self-harm motivation \cite{Elghali_2016, Karp_1991b}. No cases were reported for psychosocial motivation and other motivation.
\paragraph*{Object Characteristics}  68 cases (76\%) involved sharp object ingestion \cite{Elghali_2016, Karp_1991b, Lee_2007}, 32 cases (36\%) involved long (\textgreater 5cm) object ingestion \cite{Lee_2007}, 25 cases (28\%) involved ingestion of multiple objects \cite{Elghali_2016, Lee_2007}. No cases were reported for button battery ingestion, magnet ingestion, and involved large diameter (\textgreater 2.5cm) object ingestion.
\paragraph*{Outcomes}  47 cases (52\%) underwent endoscopic intervention \cite{Elghali_2016, Lee_2007}, 29 cases (32\%) were managed conservatively \cite{Elghali_2016, Karp_1991b}, 15 cases (17\%) underwent surgical intervention \cite{Elghali_2016, Karp_1991b, Lee_2007}, 6 cases (7\%) reported complications \cite{Lee_2007}, 1 case (1\%) died \cite{Elghali_2016}.
\paragraph*{Geographical Location}Cases were recorded in 33 countries: 13 cases from USA \cite{Alao_2006i, Ataya_2013, Bhumi_2024f, Fry_2010, Guinan_2019f, Hardy_2023g, Jehangir_2019h, Kerestes_2019, Kumar_2001, Liu_2005, Tammana_2012j, Tay_2004, Tupesis_2004f}; 7 cases from India \cite{Bhasin_2014, Bhattacharjee_2008, Kar_2015, Kariholu_2008, Kumar_2019f, Misra_2013, Wadhwa_2015e} and UK \cite{Beecroft_1998, Berry_2021e, Cauchi_2002, Cox_2007, Gardner_2017h, Qureshi_2016}; 6 cases from Bulgaria \cite{Losanoff_1996, Losanoff_1997e}; 5 cases from Iran \cite{DivsalarP._2023a, Emamhadi_2018, Farhadi_2024h}; 4 cases from Turkey \cite{Akay_2015f, Atayan_2016, Tanrikulu_2015e, Yildiz_2016e}; 2 cases from China \cite{Jin_2023, Li_2013}, Poland \cite{Kobiela_2015, Wnęk_2015f}, and Spain \cite{CamachoDorado_2018, fjbuilsRepeatedBehaviorDeliberate2024}; 1 case from Australia \cite{Apikotoa_2022f}, Bahrain \cite{Ali_2020f}, Croatia \cite{Trgo_2012f}, Ecuador \cite{DelgadoSalazar_2020c}, Egypt \cite{Ali_2022g}, Ethiopia \cite{Mesfin_2022a}, Germany \cite{teWildt_2010}, Greece \cite{Sakellaridis_2008f}, Hungary \cite{Csaky_1998e}, Iraq \cite{Al-Faham_2020k}, Israel \cite{Goldman_1998f}, Italy \cite{Riva_2018j}, Japan \cite{Ohno_2005}, Nepal \cite{Thapa_2019f}, Netherlands \cite{Benoist_2019e}, Oman \cite{AlShaaibi_2021b}, Pakistan \cite{Yasin_2009}, Portugal \cite{Peixoto_2017f}, Qatar \cite{Ali_2017}, Saudi Arabia \cite{Sultan_2024f}, South Africa \cite{Sobnach_2011f}, Sweden \cite{Naji_2012f}, Switzerland \cite{Wildhaber_2005}, and Taiwan \cite{Chang_2017f}. \paragraph*{Gender} 43 cases (60\%) were male \cite{Akay_2015f, Al-Faham_2020k, Alao_2006i, Ali_2017, Ali_2022g, Apikotoa_2022f, Atayan_2016, Benoist_2019e, Berry_2021e, Bhumi_2024f, CamachoDorado_2018, Csaky_1998e, Emamhadi_2018, Farhadi_2024h, Fry_2010, Gardner_2017h, Guinan_2019f, Jehangir_2019h, Jin_2023, Kobiela_2015, Kumar_2001, Kumar_2019f, Liu_2005, Losanoff_1996, Losanoff_1997e, Mesfin_2022a, Misra_2013, Qureshi_2016, Riva_2018j, Sobnach_2011f, Tammana_2012j, Tanrikulu_2015e, Tay_2004, Thapa_2019f, Trgo_2012f, Wadhwa_2015e, Yasin_2009, teWildt_2010}, 28 cases (39\%) were female \cite{AlShaaibi_2021b, Ali_2020f, Ataya_2013, Beecroft_1998, Bhasin_2014, Bhattacharjee_2008, Cauchi_2002, Chang_2017f, Cox_2007, DelgadoSalazar_2020c, DivsalarP._2023a, Goldman_1998f, Hardy_2023g, Kar_2015, Kariholu_2008, Kerestes_2019, Li_2013, Naji_2012f, Ohno_2005, Peixoto_2017f, Sakellaridis_2008f, Sultan_2024f, Tupesis_2004f, Wildhaber_2005, Wnęk_2015f, Yildiz_2016e}, 1 case (1\%) had no gender recorded \cite{fjbuilsRepeatedBehaviorDeliberate2024}. \paragraph*{Age Group} 25 cases (35\%) were between 26 and 40 years of age \cite{Alao_2006i, Ali_2022g, Apikotoa_2022f, Ataya_2013, Benoist_2019e, Bhasin_2014, Chang_2017f, Cox_2007, DelgadoSalazar_2020c, Farhadi_2024h, Fry_2010, Gardner_2017h, Guinan_2019f, Jin_2023, Kumar_2019f, Losanoff_1996, Misra_2013, Qureshi_2016, Riva_2018j, Sakellaridis_2008f, Tammana_2012j, Trgo_2012f, Wnęk_2015f, Yildiz_2016e, fjbuilsRepeatedBehaviorDeliberate2024}, 18 cases (25\%) were between 18 and 25 years of age \cite{Akay_2015f, Ali_2017, Atayan_2016, Bhattacharjee_2008, Csaky_1998e, Kar_2015, Kariholu_2008, Kobiela_2015, Losanoff_1996, Losanoff_1997e, Mesfin_2022a, Peixoto_2017f, Sobnach_2011f, Tupesis_2004f, Yasin_2009}, 13 cases (18\%) were under 18 years of age \cite{AlShaaibi_2021b, Ali_2020f, Cauchi_2002, DivsalarP._2023a, Goldman_1998f, Liu_2005, Naji_2012f, Ohno_2005, Tanrikulu_2015e, Tay_2004, Wildhaber_2005}, 11 cases (15\%) were between 41 and 60 years of age \cite{Al-Faham_2020k, Bhumi_2024f, CamachoDorado_2018, Emamhadi_2018, Hardy_2023g, Jehangir_2019h, Kumar_2001, Sultan_2024f, Thapa_2019f, Wadhwa_2015e, teWildt_2010}, 3 cases (4\%) were over 60 years of age \cite{Beecroft_1998, Kerestes_2019, Li_2013}, 2 cases (3\%) had no age documented \cite{Berry_2021e}. \paragraph*{Population} 36 cases (50\%) had a psychiatric history \cite{AlShaaibi_2021b, Alao_2006i, Ali_2020f, Apikotoa_2022f, Ataya_2013, Atayan_2016, Beecroft_1998, CamachoDorado_2018, Chang_2017f, DelgadoSalazar_2020c, DivsalarP._2023a, Farhadi_2024h, Fry_2010, Guinan_2019f, Hardy_2023g, Jehangir_2019h, Jin_2023, Kar_2015, Kerestes_2019, Kobiela_2015, Kumar_2001, Kumar_2019f, Liu_2005, Mesfin_2022a, Misra_2013, Ohno_2005, Peixoto_2017f, Sakellaridis_2008f, Sultan_2024f, Tammana_2012j, Tanrikulu_2015e, Yildiz_2016e, fjbuilsRepeatedBehaviorDeliberate2024, teWildt_2010}, 19 cases (26\%) had ingested previously \cite{Alao_2006i, Apikotoa_2022f, Berry_2021e, Bhattacharjee_2008, Csaky_1998e, DivsalarP._2023a, Emamhadi_2018, Guinan_2019f, Jehangir_2019h, Jin_2023, Liu_2005, Sakellaridis_2008f, Tanrikulu_2015e, Thapa_2019f, Yildiz_2016e, fjbuilsRepeatedBehaviorDeliberate2024, teWildt_2010}, 12 cases (17\%) were detained persons \cite{Alao_2006i, Ali_2022g, Apikotoa_2022f, Losanoff_1996, Losanoff_1997e, Qureshi_2016, Tammana_2012j, Trgo_2012f}, 7 cases (10\%) were severely disabled \cite{Atayan_2016, Kerestes_2019, Liu_2005, Ohno_2005, Peixoto_2017f, Yildiz_2016e, teWildt_2010}, 4 cases (6\%) were psychiatric inpatients \cite{DivsalarP._2023a, fjbuilsRepeatedBehaviorDeliberate2024, teWildt_2010}, 3 cases (4\%) were under the influence of alcohol \cite{Benoist_2019e, Csaky_1998e, Thapa_2019f}, 2 cases (3\%) were displaced people \cite{Akay_2015f, Gardner_2017h}. \paragraph*{Motivation} 34 cases (47\%) had a psychiatric motivation \cite{Al-Faham_2020k, Alao_2006i, Ali_2020f, Apikotoa_2022f, Ataya_2013, Atayan_2016, Bhasin_2014, Bhattacharjee_2008, DelgadoSalazar_2020c, DivsalarP._2023a, Emamhadi_2018, Farhadi_2024h, Guinan_2019f, Hardy_2023g, Jehangir_2019h, Jin_2023, Kar_2015, Kariholu_2008, Kerestes_2019, Kobiela_2015, Kumar_2001, Kumar_2019f, Li_2013, Liu_2005, Misra_2013, Ohno_2005, Sakellaridis_2008f, Sultan_2024f, Tammana_2012j, Tanrikulu_2015e, Yasin_2009, teWildt_2010}, 21 cases (29\%) were motivated by self-harm intention \cite{Al-Faham_2020k, AlShaaibi_2021b, Alao_2006i, Ali_2017, CamachoDorado_2018, Chang_2017f, Cox_2007, Csaky_1998e, Fry_2010, Li_2013, Losanoff_1996, Losanoff_1997e, Mesfin_2022a, Sakellaridis_2008f, Tammana_2012j, Tanrikulu_2015e, fjbuilsRepeatedBehaviorDeliberate2024}, 17 cases (24\%) had a psychosocial motivation \cite{Akay_2015f, Benoist_2019e, Bhattacharjee_2008, Cauchi_2002, Goldman_1998f, Hardy_2023g, Kobiela_2015, Li_2013, Naji_2012f, Qureshi_2016, Riva_2018j, Sobnach_2011f, Tay_2004, Thapa_2019f, Tupesis_2004f, Wildhaber_2005, Wnęk_2015f}, 9 cases (12\%) were motivated by protest \cite{Bhumi_2024f, Gardner_2017h, Losanoff_1996, Losanoff_1997e, Tupesis_2004f}, 9 cases (12\%) had another documented motivation \cite{Ali_2020f, Ali_2022g, Emamhadi_2018, Guinan_2019f, Peixoto_2017f, Sakellaridis_2008f, Trgo_2012f, Wadhwa_2015e, Yildiz_2016e}. \paragraph*{Object Characteristics} 51 cases (71\%) ingested a large diameter object (\textgreater{}2.5cm) \cite{Akay_2015f, Al-Faham_2020k, AlShaaibi_2021b, Alao_2006i, Ali_2017, Ali_2022g, Apikotoa_2022f, Atayan_2016, Berry_2021e, Bhasin_2014, CamachoDorado_2018, Cauchi_2002, Chang_2017f, Cox_2007, Csaky_1998e, DivsalarP._2023a, Emamhadi_2018, Gardner_2017h, Guinan_2019f, Jehangir_2019h, Jin_2023, Kariholu_2008, Kerestes_2019, Kobiela_2015, Kumar_2001, Kumar_2019f, Losanoff_1996, Losanoff_1997e, Mesfin_2022a, Misra_2013, Naji_2012f, Ohno_2005, Peixoto_2017f, Qureshi_2016, Riva_2018j, Sakellaridis_2008f, Sultan_2024f, Tanrikulu_2015e, Thapa_2019f, Trgo_2012f, Wnęk_2015f, Yildiz_2016e, fjbuilsRepeatedBehaviorDeliberate2024, teWildt_2010}, 44 cases (61\%) ingested multiple objects \cite{Ali_2020f, Apikotoa_2022f, Ataya_2013, Atayan_2016, Beecroft_1998, Bhattacharjee_2008, Bhumi_2024f, CamachoDorado_2018, Cauchi_2002, Emamhadi_2018, Farhadi_2024h, Fry_2010, Goldman_1998f, Guinan_2019f, Hardy_2023g, Jehangir_2019h, Jin_2023, Kar_2015, Kariholu_2008, Kobiela_2015, Kumar_2001, Kumar_2019f, Li_2013, Liu_2005, Losanoff_1996, Mesfin_2022a, Misra_2013, Naji_2012f, Ohno_2005, Sobnach_2011f, Sultan_2024f, Tammana_2012j, Tanrikulu_2015e, Tay_2004, Thapa_2019f, Wadhwa_2015e, Wildhaber_2005, Yasin_2009, fjbuilsRepeatedBehaviorDeliberate2024, teWildt_2010}, 34 cases (47\%) ingested a sharp object \cite{AlShaaibi_2021b, Alao_2006i, Apikotoa_2022f, Ataya_2013, Benoist_2019e, Bhasin_2014, Bhattacharjee_2008, CamachoDorado_2018, Csaky_1998e, DelgadoSalazar_2020c, DivsalarP._2023a, Emamhadi_2018, Farhadi_2024h, Fry_2010, Guinan_2019f, Hardy_2023g, Jehangir_2019h, Jin_2023, Kariholu_2008, Kobiela_2015, Kumar_2019f, Losanoff_1996, Losanoff_1997e, Mesfin_2022a, Misra_2013, Sobnach_2011f, Yasin_2009, teWildt_2010}, 32 cases (44\%) ingested a long object (\textgreater{}5cm) \cite{Al-Faham_2020k, AlShaaibi_2021b, Ali_2017, Ali_2022g, Atayan_2016, Bhasin_2014, CamachoDorado_2018, Chang_2017f, Cox_2007, Csaky_1998e, DivsalarP._2023a, Emamhadi_2018, Fry_2010, Gardner_2017h, Jin_2023, Kariholu_2008, Kerestes_2019, Kobiela_2015, Kumar_2019f, Mesfin_2022a, Misra_2013, Ohno_2005, Qureshi_2016, Sakellaridis_2008f, Sultan_2024f, Thapa_2019f, Trgo_2012f, Yasin_2009, Yildiz_2016e, teWildt_2010}, 9 cases (12\%) ingested a magnet \cite{Ali_2020f, Bhumi_2024f, Cauchi_2002, Liu_2005, Naji_2012f, Ohno_2005, Tanrikulu_2015e, Tay_2004, Wildhaber_2005}, 2 cases (3\%) ingested a button battery \cite{Berry_2021e, Bhumi_2024f}. \paragraph*{Outcomes} 48 cases (67\%) experienced a complication \cite{Ali_2017, Ali_2020f, Apikotoa_2022f, Atayan_2016, Beecroft_1998, Benoist_2019e, Berry_2021e, Bhasin_2014, Bhumi_2024f, CamachoDorado_2018, Cauchi_2002, Cox_2007, Csaky_1998e, DelgadoSalazar_2020c, DivsalarP._2023a, Emamhadi_2018, Farhadi_2024h, Fry_2010, Gardner_2017h, Goldman_1998f, Jin_2023, Kariholu_2008, Kerestes_2019, Kobiela_2015, Kumar_2001, Kumar_2019f, Liu_2005, Losanoff_1996, Mesfin_2022a, Misra_2013, Naji_2012f, Ohno_2005, Sakellaridis_2008f, Sobnach_2011f, Sultan_2024f, Tanrikulu_2015e, Tay_2004, Thapa_2019f, Trgo_2012f, Tupesis_2004f, Wildhaber_2005, Wnęk_2015f, Yasin_2009, Yildiz_2016e}, 44 cases (61\%) underwent surgery \cite{Al-Faham_2020k, AlShaaibi_2021b, Alao_2006i, Ali_2017, Ali_2020f, Atayan_2016, Beecroft_1998, Bhasin_2014, CamachoDorado_2018, Cauchi_2002, Chang_2017f, Cox_2007, Csaky_1998e, DelgadoSalazar_2020c, DivsalarP._2023a, Farhadi_2024h, Fry_2010, Gardner_2017h, Jin_2023, Kariholu_2008, Kerestes_2019, Kobiela_2015, Kumar_2019f, Liu_2005, Losanoff_1996, Losanoff_1997e, Mesfin_2022a, Misra_2013, Naji_2012f, Sobnach_2011f, Tanrikulu_2015e, Tay_2004, Thapa_2019f, Tupesis_2004f, Wildhaber_2005, Wnęk_2015f, Yasin_2009, Yildiz_2016e, fjbuilsRepeatedBehaviorDeliberate2024}, 31 cases (43\%) underwent endoscopy \cite{Akay_2015f, Ali_2022g, Apikotoa_2022f, Atayan_2016, Benoist_2019e, Berry_2021e, Bhasin_2014, Bhumi_2024f, CamachoDorado_2018, Chang_2017f, DelgadoSalazar_2020c, Gardner_2017h, Guinan_2019f, Hardy_2023g, Jehangir_2019h, Kariholu_2008, Li_2013, Liu_2005, Ohno_2005, Peixoto_2017f, Qureshi_2016, Riva_2018j, Sakellaridis_2008f, Sultan_2024f, Tammana_2012j, Tanrikulu_2015e, Trgo_2012f, Wadhwa_2015e, Wnęk_2015f, teWildt_2010}, 7 cases (10\%) were managed conservatively \cite{Ataya_2013, Bhattacharjee_2008, DivsalarP._2023a, Emamhadi_2018, Goldman_1998f, Kar_2015, Kumar_2001}, 2 cases (3\%) died \cite{Emamhadi_2018, Kumar_2001}. All 90 were male gender. 90 cases (100\%) were detained at the time of ingestion \cite{Elghali_2016, Karp_1991b, Lee_2007}, 88 cases (98\%) were intentional ingestions \cite{Elghali_2016, Karp_1991b, Lee_2007}, 30 cases (33\%) had a psychiatric history documented \cite{Elghali_2016, Karp_1991b, Lee_2007}, 2 cases (2\%) had a history of prior ingestion \cite{Elghali_2016}. No cases were reported for were psychiatric inpatients, were displaced people, were under the influence of alcohol at the time of ingestion, and had a severe disability history.
\paragraph*{Motivation}  70 cases (78\%) reported protest motivation \cite{Elghali_2016, Karp_1991b, Lee_2007}, 12 cases (13\%) reported psychiatric motivation \cite{Karp_1991b}, 6 cases (7\%) reported self-harm motivation \cite{Elghali_2016, Karp_1991b}. No cases were reported for psychosocial motivation and other motivation.
\paragraph*{Object Characteristics}  68 cases (76\%) involved sharp object ingestion \cite{Elghali_2016, Karp_1991b, Lee_2007}, 32 cases (36\%) involved long (\textgreater 5cm) object ingestion \cite{Lee_2007}, 25 cases (28\%) involved ingestion of multiple objects \cite{Elghali_2016, Lee_2007}. No cases were reported for button battery ingestion, magnet ingestion, and involved large diameter (\textgreater 2.5cm) object ingestion.
\paragraph*{Outcomes}  47 cases (52\%) underwent endoscopic intervention \cite{Elghali_2016, Lee_2007}, 29 cases (32\%) were managed conservatively \cite{Elghali_2016, Karp_1991b}, 15 cases (17\%) underwent surgical intervention \cite{Elghali_2016, Karp_1991b, Lee_2007}, 6 cases (7\%) reported complications \cite{Lee_2007}, 1 case (1\%) died \cite{Elghali_2016}.
\paragraph*{Geographical Location}Cases were recorded in 33 countries: 13 cases from USA \cite{Alao_2006i, Ataya_2013, Bhumi_2024f, Fry_2010, Guinan_2019f, Hardy_2023g, Jehangir_2019h, Kerestes_2019, Kumar_2001, Liu_2005, Tammana_2012j, Tay_2004, Tupesis_2004f}; 7 cases from India \cite{Bhasin_2014, Bhattacharjee_2008, Kar_2015, Kariholu_2008, Kumar_2019f, Misra_2013, Wadhwa_2015e} and UK \cite{Beecroft_1998, Berry_2021e, Cauchi_2002, Cox_2007, Gardner_2017h, Qureshi_2016}; 6 cases from Bulgaria \cite{Losanoff_1996, Losanoff_1997e}; 5 cases from Iran \cite{DivsalarP._2023a, Emamhadi_2018, Farhadi_2024h}; 4 cases from Turkey \cite{Akay_2015f, Atayan_2016, Tanrikulu_2015e, Yildiz_2016e}; 2 cases from China \cite{Jin_2023, Li_2013}, Poland \cite{Kobiela_2015, Wnęk_2015f}, and Spain \cite{CamachoDorado_2018, fjbuilsRepeatedBehaviorDeliberate2024}; 1 case from Australia \cite{Apikotoa_2022f}, Bahrain \cite{Ali_2020f}, Croatia \cite{Trgo_2012f}, Ecuador \cite{DelgadoSalazar_2020c}, Egypt \cite{Ali_2022g}, Ethiopia \cite{Mesfin_2022a}, Germany \cite{teWildt_2010}, Greece \cite{Sakellaridis_2008f}, Hungary \cite{Csaky_1998e}, Iraq \cite{Al-Faham_2020k}, Israel \cite{Goldman_1998f}, Italy \cite{Riva_2018j}, Japan \cite{Ohno_2005}, Nepal \cite{Thapa_2019f}, Netherlands \cite{Benoist_2019e}, Oman \cite{AlShaaibi_2021b}, Pakistan \cite{Yasin_2009}, Portugal \cite{Peixoto_2017f}, Qatar \cite{Ali_2017}, Saudi Arabia \cite{Sultan_2024f}, South Africa \cite{Sobnach_2011f}, Sweden \cite{Naji_2012f}, Switzerland \cite{Wildhaber_2005}, and Taiwan \cite{Chang_2017f}. \paragraph*{Gender} 43 cases (60\%) were male \cite{Akay_2015f, Al-Faham_2020k, Alao_2006i, Ali_2017, Ali_2022g, Apikotoa_2022f, Atayan_2016, Benoist_2019e, Berry_2021e, Bhumi_2024f, CamachoDorado_2018, Csaky_1998e, Emamhadi_2018, Farhadi_2024h, Fry_2010, Gardner_2017h, Guinan_2019f, Jehangir_2019h, Jin_2023, Kobiela_2015, Kumar_2001, Kumar_2019f, Liu_2005, Losanoff_1996, Losanoff_1997e, Mesfin_2022a, Misra_2013, Qureshi_2016, Riva_2018j, Sobnach_2011f, Tammana_2012j, Tanrikulu_2015e, Tay_2004, Thapa_2019f, Trgo_2012f, Wadhwa_2015e, Yasin_2009, teWildt_2010}, 28 cases (39\%) were female \cite{AlShaaibi_2021b, Ali_2020f, Ataya_2013, Beecroft_1998, Bhasin_2014, Bhattacharjee_2008, Cauchi_2002, Chang_2017f, Cox_2007, DelgadoSalazar_2020c, DivsalarP._2023a, Goldman_1998f, Hardy_2023g, Kar_2015, Kariholu_2008, Kerestes_2019, Li_2013, Naji_2012f, Ohno_2005, Peixoto_2017f, Sakellaridis_2008f, Sultan_2024f, Tupesis_2004f, Wildhaber_2005, Wnęk_2015f, Yildiz_2016e}, 1 case (1\%) had no gender recorded \cite{fjbuilsRepeatedBehaviorDeliberate2024}. \paragraph*{Age Group} 25 cases (35\%) were between 26 and 40 years of age \cite{Alao_2006i, Ali_2022g, Apikotoa_2022f, Ataya_2013, Benoist_2019e, Bhasin_2014, Chang_2017f, Cox_2007, DelgadoSalazar_2020c, Farhadi_2024h, Fry_2010, Gardner_2017h, Guinan_2019f, Jin_2023, Kumar_2019f, Losanoff_1996, Misra_2013, Qureshi_2016, Riva_2018j, Sakellaridis_2008f, Tammana_2012j, Trgo_2012f, Wnęk_2015f, Yildiz_2016e, fjbuilsRepeatedBehaviorDeliberate2024}, 18 cases (25\%) were between 18 and 25 years of age \cite{Akay_2015f, Ali_2017, Atayan_2016, Bhattacharjee_2008, Csaky_1998e, Kar_2015, Kariholu_2008, Kobiela_2015, Losanoff_1996, Losanoff_1997e, Mesfin_2022a, Peixoto_2017f, Sobnach_2011f, Tupesis_2004f, Yasin_2009}, 13 cases (18\%) were under 18 years of age \cite{AlShaaibi_2021b, Ali_2020f, Cauchi_2002, DivsalarP._2023a, Goldman_1998f, Liu_2005, Naji_2012f, Ohno_2005, Tanrikulu_2015e, Tay_2004, Wildhaber_2005}, 11 cases (15\%) were between 41 and 60 years of age \cite{Al-Faham_2020k, Bhumi_2024f, CamachoDorado_2018, Emamhadi_2018, Hardy_2023g, Jehangir_2019h, Kumar_2001, Sultan_2024f, Thapa_2019f, Wadhwa_2015e, teWildt_2010}, 3 cases (4\%) were over 60 years of age \cite{Beecroft_1998, Kerestes_2019, Li_2013}, 2 cases (3\%) had no age documented \cite{Berry_2021e}. \paragraph*{Population} 36 cases (50\%) had a psychiatric history \cite{AlShaaibi_2021b, Alao_2006i, Ali_2020f, Apikotoa_2022f, Ataya_2013, Atayan_2016, Beecroft_1998, CamachoDorado_2018, Chang_2017f, DelgadoSalazar_2020c, DivsalarP._2023a, Farhadi_2024h, Fry_2010, Guinan_2019f, Hardy_2023g, Jehangir_2019h, Jin_2023, Kar_2015, Kerestes_2019, Kobiela_2015, Kumar_2001, Kumar_2019f, Liu_2005, Mesfin_2022a, Misra_2013, Ohno_2005, Peixoto_2017f, Sakellaridis_2008f, Sultan_2024f, Tammana_2012j, Tanrikulu_2015e, Yildiz_2016e, fjbuilsRepeatedBehaviorDeliberate2024, teWildt_2010}, 19 cases (26\%) had ingested previously \cite{Alao_2006i, Apikotoa_2022f, Berry_2021e, Bhattacharjee_2008, Csaky_1998e, DivsalarP._2023a, Emamhadi_2018, Guinan_2019f, Jehangir_2019h, Jin_2023, Liu_2005, Sakellaridis_2008f, Tanrikulu_2015e, Thapa_2019f, Yildiz_2016e, fjbuilsRepeatedBehaviorDeliberate2024, teWildt_2010}, 12 cases (17\%) were detained persons \cite{Alao_2006i, Ali_2022g, Apikotoa_2022f, Losanoff_1996, Losanoff_1997e, Qureshi_2016, Tammana_2012j, Trgo_2012f}, 7 cases (10\%) were severely disabled \cite{Atayan_2016, Kerestes_2019, Liu_2005, Ohno_2005, Peixoto_2017f, Yildiz_2016e, teWildt_2010}, 4 cases (6\%) were psychiatric inpatients \cite{DivsalarP._2023a, fjbuilsRepeatedBehaviorDeliberate2024, teWildt_2010}, 3 cases (4\%) were under the influence of alcohol \cite{Benoist_2019e, Csaky_1998e, Thapa_2019f}, 2 cases (3\%) were displaced people \cite{Akay_2015f, Gardner_2017h}. \paragraph*{Motivation} 34 cases (47\%) had a psychiatric motivation \cite{Al-Faham_2020k, Alao_2006i, Ali_2020f, Apikotoa_2022f, Ataya_2013, Atayan_2016, Bhasin_2014, Bhattacharjee_2008, DelgadoSalazar_2020c, DivsalarP._2023a, Emamhadi_2018, Farhadi_2024h, Guinan_2019f, Hardy_2023g, Jehangir_2019h, Jin_2023, Kar_2015, Kariholu_2008, Kerestes_2019, Kobiela_2015, Kumar_2001, Kumar_2019f, Li_2013, Liu_2005, Misra_2013, Ohno_2005, Sakellaridis_2008f, Sultan_2024f, Tammana_2012j, Tanrikulu_2015e, Yasin_2009, teWildt_2010}, 21 cases (29\%) were motivated by self-harm intention \cite{Al-Faham_2020k, AlShaaibi_2021b, Alao_2006i, Ali_2017, CamachoDorado_2018, Chang_2017f, Cox_2007, Csaky_1998e, Fry_2010, Li_2013, Losanoff_1996, Losanoff_1997e, Mesfin_2022a, Sakellaridis_2008f, Tammana_2012j, Tanrikulu_2015e, fjbuilsRepeatedBehaviorDeliberate2024}, 17 cases (24\%) had a psychosocial motivation \cite{Akay_2015f, Benoist_2019e, Bhattacharjee_2008, Cauchi_2002, Goldman_1998f, Hardy_2023g, Kobiela_2015, Li_2013, Naji_2012f, Qureshi_2016, Riva_2018j, Sobnach_2011f, Tay_2004, Thapa_2019f, Tupesis_2004f, Wildhaber_2005, Wnęk_2015f}, 9 cases (12\%) were motivated by protest \cite{Bhumi_2024f, Gardner_2017h, Losanoff_1996, Losanoff_1997e, Tupesis_2004f}, 9 cases (12\%) had another documented motivation \cite{Ali_2020f, Ali_2022g, Emamhadi_2018, Guinan_2019f, Peixoto_2017f, Sakellaridis_2008f, Trgo_2012f, Wadhwa_2015e, Yildiz_2016e}. \paragraph*{Object Characteristics} 51 cases (71\%) ingested a large diameter object (\textgreater{}2.5cm) \cite{Akay_2015f, Al-Faham_2020k, AlShaaibi_2021b, Alao_2006i, Ali_2017, Ali_2022g, Apikotoa_2022f, Atayan_2016, Berry_2021e, Bhasin_2014, CamachoDorado_2018, Cauchi_2002, Chang_2017f, Cox_2007, Csaky_1998e, DivsalarP._2023a, Emamhadi_2018, Gardner_2017h, Guinan_2019f, Jehangir_2019h, Jin_2023, Kariholu_2008, Kerestes_2019, Kobiela_2015, Kumar_2001, Kumar_2019f, Losanoff_1996, Losanoff_1997e, Mesfin_2022a, Misra_2013, Naji_2012f, Ohno_2005, Peixoto_2017f, Qureshi_2016, Riva_2018j, Sakellaridis_2008f, Sultan_2024f, Tanrikulu_2015e, Thapa_2019f, Trgo_2012f, Wnęk_2015f, Yildiz_2016e, fjbuilsRepeatedBehaviorDeliberate2024, teWildt_2010}, 44 cases (61\%) ingested multiple objects \cite{Ali_2020f, Apikotoa_2022f, Ataya_2013, Atayan_2016, Beecroft_1998, Bhattacharjee_2008, Bhumi_2024f, CamachoDorado_2018, Cauchi_2002, Emamhadi_2018, Farhadi_2024h, Fry_2010, Goldman_1998f, Guinan_2019f, Hardy_2023g, Jehangir_2019h, Jin_2023, Kar_2015, Kariholu_2008, Kobiela_2015, Kumar_2001, Kumar_2019f, Li_2013, Liu_2005, Losanoff_1996, Mesfin_2022a, Misra_2013, Naji_2012f, Ohno_2005, Sobnach_2011f, Sultan_2024f, Tammana_2012j, Tanrikulu_2015e, Tay_2004, Thapa_2019f, Wadhwa_2015e, Wildhaber_2005, Yasin_2009, fjbuilsRepeatedBehaviorDeliberate2024, teWildt_2010}, 34 cases (47\%) ingested a sharp object \cite{AlShaaibi_2021b, Alao_2006i, Apikotoa_2022f, Ataya_2013, Benoist_2019e, Bhasin_2014, Bhattacharjee_2008, CamachoDorado_2018, Csaky_1998e, DelgadoSalazar_2020c, DivsalarP._2023a, Emamhadi_2018, Farhadi_2024h, Fry_2010, Guinan_2019f, Hardy_2023g, Jehangir_2019h, Jin_2023, Kariholu_2008, Kobiela_2015, Kumar_2019f, Losanoff_1996, Losanoff_1997e, Mesfin_2022a, Misra_2013, Sobnach_2011f, Yasin_2009, teWildt_2010}, 32 cases (44\%) ingested a long object (\textgreater{}5cm) \cite{Al-Faham_2020k, AlShaaibi_2021b, Ali_2017, Ali_2022g, Atayan_2016, Bhasin_2014, CamachoDorado_2018, Chang_2017f, Cox_2007, Csaky_1998e, DivsalarP._2023a, Emamhadi_2018, Fry_2010, Gardner_2017h, Jin_2023, Kariholu_2008, Kerestes_2019, Kobiela_2015, Kumar_2019f, Mesfin_2022a, Misra_2013, Ohno_2005, Qureshi_2016, Sakellaridis_2008f, Sultan_2024f, Thapa_2019f, Trgo_2012f, Yasin_2009, Yildiz_2016e, teWildt_2010}, 9 cases (12\%) ingested a magnet \cite{Ali_2020f, Bhumi_2024f, Cauchi_2002, Liu_2005, Naji_2012f, Ohno_2005, Tanrikulu_2015e, Tay_2004, Wildhaber_2005}, 2 cases (3\%) ingested a button battery \cite{Berry_2021e, Bhumi_2024f}. \paragraph*{Outcomes} 48 cases (67\%) experienced a complication \cite{Ali_2017, Ali_2020f, Apikotoa_2022f, Atayan_2016, Beecroft_1998, Benoist_2019e, Berry_2021e, Bhasin_2014, Bhumi_2024f, CamachoDorado_2018, Cauchi_2002, Cox_2007, Csaky_1998e, DelgadoSalazar_2020c, DivsalarP._2023a, Emamhadi_2018, Farhadi_2024h, Fry_2010, Gardner_2017h, Goldman_1998f, Jin_2023, Kariholu_2008, Kerestes_2019, Kobiela_2015, Kumar_2001, Kumar_2019f, Liu_2005, Losanoff_1996, Mesfin_2022a, Misra_2013, Naji_2012f, Ohno_2005, Sakellaridis_2008f, Sobnach_2011f, Sultan_2024f, Tanrikulu_2015e, Tay_2004, Thapa_2019f, Trgo_2012f, Tupesis_2004f, Wildhaber_2005, Wnęk_2015f, Yasin_2009, Yildiz_2016e}, 44 cases (61\%) underwent surgery \cite{Al-Faham_2020k, AlShaaibi_2021b, Alao_2006i, Ali_2017, Ali_2020f, Atayan_2016, Beecroft_1998, Bhasin_2014, CamachoDorado_2018, Cauchi_2002, Chang_2017f, Cox_2007, Csaky_1998e, DelgadoSalazar_2020c, DivsalarP._2023a, Farhadi_2024h, Fry_2010, Gardner_2017h, Jin_2023, Kariholu_2008, Kerestes_2019, Kobiela_2015, Kumar_2019f, Liu_2005, Losanoff_1996, Losanoff_1997e, Mesfin_2022a, Misra_2013, Naji_2012f, Sobnach_2011f, Tanrikulu_2015e, Tay_2004, Thapa_2019f, Tupesis_2004f, Wildhaber_2005, Wnęk_2015f, Yasin_2009, Yildiz_2016e, fjbuilsRepeatedBehaviorDeliberate2024}, 31 cases (43\%) underwent endoscopy \cite{Akay_2015f, Ali_2022g, Apikotoa_2022f, Atayan_2016, Benoist_2019e, Berry_2021e, Bhasin_2014, Bhumi_2024f, CamachoDorado_2018, Chang_2017f, DelgadoSalazar_2020c, Gardner_2017h, Guinan_2019f, Hardy_2023g, Jehangir_2019h, Kariholu_2008, Li_2013, Liu_2005, Ohno_2005, Peixoto_2017f, Qureshi_2016, Riva_2018j, Sakellaridis_2008f, Sultan_2024f, Tammana_2012j, Tanrikulu_2015e, Trgo_2012f, Wadhwa_2015e, Wnęk_2015f, teWildt_2010}, 7 cases (10\%) were managed conservatively \cite{Ataya_2013, Bhattacharjee_2008, DivsalarP._2023a, Emamhadi_2018, Goldman_1998f, Kar_2015, Kumar_2001}, 2 cases (3\%) died \cite{Emamhadi_2018, Kumar_2001}. All 90 were male gender. 90 cases (100\%) were detained at the time of ingestion \cite{Elghali_2016, Karp_1991b, Lee_2007}, 88 cases (98\%) were intentional ingestions \cite{Elghali_2016, Karp_1991b, Lee_2007}, 30 cases (33\%) had a psychiatric history documented \cite{Elghali_2016, Karp_1991b, Lee_2007}, 2 cases (2\%) had a history of prior ingestion \cite{Elghali_2016}. No cases were reported for were psychiatric inpatients, were displaced people, were under the influence of alcohol at the time of ingestion, and had a severe disability history.
\paragraph*{Motivation}  70 cases (78\%) reported protest motivation \cite{Elghali_2016, Karp_1991b, Lee_2007}, 12 cases (13\%) reported psychiatric motivation \cite{Karp_1991b}, 6 cases (7\%) reported self-harm motivation \cite{Elghali_2016, Karp_1991b}. No cases were reported for psychosocial motivation and other motivation.
\paragraph*{Object Characteristics}  68 cases (76\%) involved sharp object ingestion \cite{Elghali_2016, Karp_1991b, Lee_2007}, 32 cases (36\%) involved long (\textgreater 5cm) object ingestion \cite{Lee_2007}, 25 cases (28\%) involved ingestion of multiple objects \cite{Elghali_2016, Lee_2007}. No cases were reported for button battery ingestion, magnet ingestion, and involved large diameter (\textgreater 2.5cm) object ingestion.
\paragraph*{Outcomes}  47 cases (52\%) underwent endoscopic intervention \cite{Elghali_2016, Lee_2007}, 29 cases (32\%) were managed conservatively \cite{Elghali_2016, Karp_1991b}, 15 cases (17\%) underwent surgical intervention \cite{Elghali_2016, Karp_1991b, Lee_2007}, 6 cases (7\%) reported complications \cite{Lee_2007}, 1 case (1\%) died \cite{Elghali_2016}.
\paragraph*{Geographical Location}Cases were recorded in 33 countries: 13 cases from USA \cite{Alao_2006i, Ataya_2013, Bhumi_2024f, Fry_2010, Guinan_2019f, Hardy_2023g, Jehangir_2019h, Kerestes_2019, Kumar_2001, Liu_2005, Tammana_2012j, Tay_2004, Tupesis_2004f}; 7 cases from India \cite{Bhasin_2014, Bhattacharjee_2008, Kar_2015, Kariholu_2008, Kumar_2019f, Misra_2013, Wadhwa_2015e} and UK \cite{Beecroft_1998, Berry_2021e, Cauchi_2002, Cox_2007, Gardner_2017h, Qureshi_2016}; 6 cases from Bulgaria \cite{Losanoff_1996, Losanoff_1997e}; 5 cases from Iran \cite{DivsalarP._2023a, Emamhadi_2018, Farhadi_2024h}; 4 cases from Turkey \cite{Akay_2015f, Atayan_2016, Tanrikulu_2015e, Yildiz_2016e}; 2 cases from China \cite{Jin_2023, Li_2013}, Poland \cite{Kobiela_2015, Wnęk_2015f}, and Spain \cite{CamachoDorado_2018, fjbuilsRepeatedBehaviorDeliberate2024}; 1 case from Australia \cite{Apikotoa_2022f}, Bahrain \cite{Ali_2020f}, Croatia \cite{Trgo_2012f}, Ecuador \cite{DelgadoSalazar_2020c}, Egypt \cite{Ali_2022g}, Ethiopia \cite{Mesfin_2022a}, Germany \cite{teWildt_2010}, Greece \cite{Sakellaridis_2008f}, Hungary \cite{Csaky_1998e}, Iraq \cite{Al-Faham_2020k}, Israel \cite{Goldman_1998f}, Italy \cite{Riva_2018j}, Japan \cite{Ohno_2005}, Nepal \cite{Thapa_2019f}, Netherlands \cite{Benoist_2019e}, Oman \cite{AlShaaibi_2021b}, Pakistan \cite{Yasin_2009}, Portugal \cite{Peixoto_2017f}, Qatar \cite{Ali_2017}, Saudi Arabia \cite{Sultan_2024f}, South Africa \cite{Sobnach_2011f}, Sweden \cite{Naji_2012f}, Switzerland \cite{Wildhaber_2005}, and Taiwan \cite{Chang_2017f}. \paragraph*{Gender} 43 cases (60\%) were male \cite{Akay_2015f, Al-Faham_2020k, Alao_2006i, Ali_2017, Ali_2022g, Apikotoa_2022f, Atayan_2016, Benoist_2019e, Berry_2021e, Bhumi_2024f, CamachoDorado_2018, Csaky_1998e, Emamhadi_2018, Farhadi_2024h, Fry_2010, Gardner_2017h, Guinan_2019f, Jehangir_2019h, Jin_2023, Kobiela_2015, Kumar_2001, Kumar_2019f, Liu_2005, Losanoff_1996, Losanoff_1997e, Mesfin_2022a, Misra_2013, Qureshi_2016, Riva_2018j, Sobnach_2011f, Tammana_2012j, Tanrikulu_2015e, Tay_2004, Thapa_2019f, Trgo_2012f, Wadhwa_2015e, Yasin_2009, teWildt_2010}, 28 cases (39\%) were female \cite{AlShaaibi_2021b, Ali_2020f, Ataya_2013, Beecroft_1998, Bhasin_2014, Bhattacharjee_2008, Cauchi_2002, Chang_2017f, Cox_2007, DelgadoSalazar_2020c, DivsalarP._2023a, Goldman_1998f, Hardy_2023g, Kar_2015, Kariholu_2008, Kerestes_2019, Li_2013, Naji_2012f, Ohno_2005, Peixoto_2017f, Sakellaridis_2008f, Sultan_2024f, Tupesis_2004f, Wildhaber_2005, Wnęk_2015f, Yildiz_2016e}, 1 case (1\%) had no gender recorded \cite{fjbuilsRepeatedBehaviorDeliberate2024}. \paragraph*{Age Group} 25 cases (35\%) were between 26 and 40 years of age \cite{Alao_2006i, Ali_2022g, Apikotoa_2022f, Ataya_2013, Benoist_2019e, Bhasin_2014, Chang_2017f, Cox_2007, DelgadoSalazar_2020c, Farhadi_2024h, Fry_2010, Gardner_2017h, Guinan_2019f, Jin_2023, Kumar_2019f, Losanoff_1996, Misra_2013, Qureshi_2016, Riva_2018j, Sakellaridis_2008f, Tammana_2012j, Trgo_2012f, Wnęk_2015f, Yildiz_2016e, fjbuilsRepeatedBehaviorDeliberate2024}, 18 cases (25\%) were between 18 and 25 years of age \cite{Akay_2015f, Ali_2017, Atayan_2016, Bhattacharjee_2008, Csaky_1998e, Kar_2015, Kariholu_2008, Kobiela_2015, Losanoff_1996, Losanoff_1997e, Mesfin_2022a, Peixoto_2017f, Sobnach_2011f, Tupesis_2004f, Yasin_2009}, 13 cases (18\%) were under 18 years of age \cite{AlShaaibi_2021b, Ali_2020f, Cauchi_2002, DivsalarP._2023a, Goldman_1998f, Liu_2005, Naji_2012f, Ohno_2005, Tanrikulu_2015e, Tay_2004, Wildhaber_2005}, 11 cases (15\%) were between 41 and 60 years of age \cite{Al-Faham_2020k, Bhumi_2024f, CamachoDorado_2018, Emamhadi_2018, Hardy_2023g, Jehangir_2019h, Kumar_2001, Sultan_2024f, Thapa_2019f, Wadhwa_2015e, teWildt_2010}, 3 cases (4\%) were over 60 years of age \cite{Beecroft_1998, Kerestes_2019, Li_2013}, 2 cases (3\%) had no age documented \cite{Berry_2021e}. \paragraph*{Population} 36 cases (50\%) had a psychiatric history \cite{AlShaaibi_2021b, Alao_2006i, Ali_2020f, Apikotoa_2022f, Ataya_2013, Atayan_2016, Beecroft_1998, CamachoDorado_2018, Chang_2017f, DelgadoSalazar_2020c, DivsalarP._2023a, Farhadi_2024h, Fry_2010, Guinan_2019f, Hardy_2023g, Jehangir_2019h, Jin_2023, Kar_2015, Kerestes_2019, Kobiela_2015, Kumar_2001, Kumar_2019f, Liu_2005, Mesfin_2022a, Misra_2013, Ohno_2005, Peixoto_2017f, Sakellaridis_2008f, Sultan_2024f, Tammana_2012j, Tanrikulu_2015e, Yildiz_2016e, fjbuilsRepeatedBehaviorDeliberate2024, teWildt_2010}, 19 cases (26\%) had ingested previously \cite{Alao_2006i, Apikotoa_2022f, Berry_2021e, Bhattacharjee_2008, Csaky_1998e, DivsalarP._2023a, Emamhadi_2018, Guinan_2019f, Jehangir_2019h, Jin_2023, Liu_2005, Sakellaridis_2008f, Tanrikulu_2015e, Thapa_2019f, Yildiz_2016e, fjbuilsRepeatedBehaviorDeliberate2024, teWildt_2010}, 12 cases (17\%) were detained persons \cite{Alao_2006i, Ali_2022g, Apikotoa_2022f, Losanoff_1996, Losanoff_1997e, Qureshi_2016, Tammana_2012j, Trgo_2012f}, 7 cases (10\%) were severely disabled \cite{Atayan_2016, Kerestes_2019, Liu_2005, Ohno_2005, Peixoto_2017f, Yildiz_2016e, teWildt_2010}, 4 cases (6\%) were psychiatric inpatients \cite{DivsalarP._2023a, fjbuilsRepeatedBehaviorDeliberate2024, teWildt_2010}, 3 cases (4\%) were under the influence of alcohol \cite{Benoist_2019e, Csaky_1998e, Thapa_2019f}, 2 cases (3\%) were displaced people \cite{Akay_2015f, Gardner_2017h}. \paragraph*{Motivation} 34 cases (47\%) had a psychiatric motivation \cite{Al-Faham_2020k, Alao_2006i, Ali_2020f, Apikotoa_2022f, Ataya_2013, Atayan_2016, Bhasin_2014, Bhattacharjee_2008, DelgadoSalazar_2020c, DivsalarP._2023a, Emamhadi_2018, Farhadi_2024h, Guinan_2019f, Hardy_2023g, Jehangir_2019h, Jin_2023, Kar_2015, Kariholu_2008, Kerestes_2019, Kobiela_2015, Kumar_2001, Kumar_2019f, Li_2013, Liu_2005, Misra_2013, Ohno_2005, Sakellaridis_2008f, Sultan_2024f, Tammana_2012j, Tanrikulu_2015e, Yasin_2009, teWildt_2010}, 21 cases (29\%) were motivated by self-harm intention \cite{Al-Faham_2020k, AlShaaibi_2021b, Alao_2006i, Ali_2017, CamachoDorado_2018, Chang_2017f, Cox_2007, Csaky_1998e, Fry_2010, Li_2013, Losanoff_1996, Losanoff_1997e, Mesfin_2022a, Sakellaridis_2008f, Tammana_2012j, Tanrikulu_2015e, fjbuilsRepeatedBehaviorDeliberate2024}, 17 cases (24\%) had a psychosocial motivation \cite{Akay_2015f, Benoist_2019e, Bhattacharjee_2008, Cauchi_2002, Goldman_1998f, Hardy_2023g, Kobiela_2015, Li_2013, Naji_2012f, Qureshi_2016, Riva_2018j, Sobnach_2011f, Tay_2004, Thapa_2019f, Tupesis_2004f, Wildhaber_2005, Wnęk_2015f}, 9 cases (12\%) were motivated by protest \cite{Bhumi_2024f, Gardner_2017h, Losanoff_1996, Losanoff_1997e, Tupesis_2004f}, 9 cases (12\%) had another documented motivation \cite{Ali_2020f, Ali_2022g, Emamhadi_2018, Guinan_2019f, Peixoto_2017f, Sakellaridis_2008f, Trgo_2012f, Wadhwa_2015e, Yildiz_2016e}. \paragraph*{Object Characteristics} 51 cases (71\%) ingested a large diameter object (\textgreater{}2.5cm) \cite{Akay_2015f, Al-Faham_2020k, AlShaaibi_2021b, Alao_2006i, Ali_2017, Ali_2022g, Apikotoa_2022f, Atayan_2016, Berry_2021e, Bhasin_2014, CamachoDorado_2018, Cauchi_2002, Chang_2017f, Cox_2007, Csaky_1998e, DivsalarP._2023a, Emamhadi_2018, Gardner_2017h, Guinan_2019f, Jehangir_2019h, Jin_2023, Kariholu_2008, Kerestes_2019, Kobiela_2015, Kumar_2001, Kumar_2019f, Losanoff_1996, Losanoff_1997e, Mesfin_2022a, Misra_2013, Naji_2012f, Ohno_2005, Peixoto_2017f, Qureshi_2016, Riva_2018j, Sakellaridis_2008f, Sultan_2024f, Tanrikulu_2015e, Thapa_2019f, Trgo_2012f, Wnęk_2015f, Yildiz_2016e, fjbuilsRepeatedBehaviorDeliberate2024, teWildt_2010}, 44 cases (61\%) ingested multiple objects \cite{Ali_2020f, Apikotoa_2022f, Ataya_2013, Atayan_2016, Beecroft_1998, Bhattacharjee_2008, Bhumi_2024f, CamachoDorado_2018, Cauchi_2002, Emamhadi_2018, Farhadi_2024h, Fry_2010, Goldman_1998f, Guinan_2019f, Hardy_2023g, Jehangir_2019h, Jin_2023, Kar_2015, Kariholu_2008, Kobiela_2015, Kumar_2001, Kumar_2019f, Li_2013, Liu_2005, Losanoff_1996, Mesfin_2022a, Misra_2013, Naji_2012f, Ohno_2005, Sobnach_2011f, Sultan_2024f, Tammana_2012j, Tanrikulu_2015e, Tay_2004, Thapa_2019f, Wadhwa_2015e, Wildhaber_2005, Yasin_2009, fjbuilsRepeatedBehaviorDeliberate2024, teWildt_2010}, 34 cases (47\%) ingested a sharp object \cite{AlShaaibi_2021b, Alao_2006i, Apikotoa_2022f, Ataya_2013, Benoist_2019e, Bhasin_2014, Bhattacharjee_2008, CamachoDorado_2018, Csaky_1998e, DelgadoSalazar_2020c, DivsalarP._2023a, Emamhadi_2018, Farhadi_2024h, Fry_2010, Guinan_2019f, Hardy_2023g, Jehangir_2019h, Jin_2023, Kariholu_2008, Kobiela_2015, Kumar_2019f, Losanoff_1996, Losanoff_1997e, Mesfin_2022a, Misra_2013, Sobnach_2011f, Yasin_2009, teWildt_2010}, 32 cases (44\%) ingested a long object (\textgreater{}5cm) \cite{Al-Faham_2020k, AlShaaibi_2021b, Ali_2017, Ali_2022g, Atayan_2016, Bhasin_2014, CamachoDorado_2018, Chang_2017f, Cox_2007, Csaky_1998e, DivsalarP._2023a, Emamhadi_2018, Fry_2010, Gardner_2017h, Jin_2023, Kariholu_2008, Kerestes_2019, Kobiela_2015, Kumar_2019f, Mesfin_2022a, Misra_2013, Ohno_2005, Qureshi_2016, Sakellaridis_2008f, Sultan_2024f, Thapa_2019f, Trgo_2012f, Yasin_2009, Yildiz_2016e, teWildt_2010}, 9 cases (12\%) ingested a magnet \cite{Ali_2020f, Bhumi_2024f, Cauchi_2002, Liu_2005, Naji_2012f, Ohno_2005, Tanrikulu_2015e, Tay_2004, Wildhaber_2005}, 2 cases (3\%) ingested a button battery \cite{Berry_2021e, Bhumi_2024f}. \paragraph*{Outcomes} 48 cases (67\%) experienced a complication \cite{Ali_2017, Ali_2020f, Apikotoa_2022f, Atayan_2016, Beecroft_1998, Benoist_2019e, Berry_2021e, Bhasin_2014, Bhumi_2024f, CamachoDorado_2018, Cauchi_2002, Cox_2007, Csaky_1998e, DelgadoSalazar_2020c, DivsalarP._2023a, Emamhadi_2018, Farhadi_2024h, Fry_2010, Gardner_2017h, Goldman_1998f, Jin_2023, Kariholu_2008, Kerestes_2019, Kobiela_2015, Kumar_2001, Kumar_2019f, Liu_2005, Losanoff_1996, Mesfin_2022a, Misra_2013, Naji_2012f, Ohno_2005, Sakellaridis_2008f, Sobnach_2011f, Sultan_2024f, Tanrikulu_2015e, Tay_2004, Thapa_2019f, Trgo_2012f, Tupesis_2004f, Wildhaber_2005, Wnęk_2015f, Yasin_2009, Yildiz_2016e}, 44 cases (61\%) underwent surgery \cite{Al-Faham_2020k, AlShaaibi_2021b, Alao_2006i, Ali_2017, Ali_2020f, Atayan_2016, Beecroft_1998, Bhasin_2014, CamachoDorado_2018, Cauchi_2002, Chang_2017f, Cox_2007, Csaky_1998e, DelgadoSalazar_2020c, DivsalarP._2023a, Farhadi_2024h, Fry_2010, Gardner_2017h, Jin_2023, Kariholu_2008, Kerestes_2019, Kobiela_2015, Kumar_2019f, Liu_2005, Losanoff_1996, Losanoff_1997e, Mesfin_2022a, Misra_2013, Naji_2012f, Sobnach_2011f, Tanrikulu_2015e, Tay_2004, Thapa_2019f, Tupesis_2004f, Wildhaber_2005, Wnęk_2015f, Yasin_2009, Yildiz_2016e, fjbuilsRepeatedBehaviorDeliberate2024}, 31 cases (43\%) underwent endoscopy \cite{Akay_2015f, Ali_2022g, Apikotoa_2022f, Atayan_2016, Benoist_2019e, Berry_2021e, Bhasin_2014, Bhumi_2024f, CamachoDorado_2018, Chang_2017f, DelgadoSalazar_2020c, Gardner_2017h, Guinan_2019f, Hardy_2023g, Jehangir_2019h, Kariholu_2008, Li_2013, Liu_2005, Ohno_2005, Peixoto_2017f, Qureshi_2016, Riva_2018j, Sakellaridis_2008f, Sultan_2024f, Tammana_2012j, Tanrikulu_2015e, Trgo_2012f, Wadhwa_2015e, Wnęk_2015f, teWildt_2010}, 7 cases (10\%) were managed conservatively \cite{Ataya_2013, Bhattacharjee_2008, DivsalarP._2023a, Emamhadi_2018, Goldman_1998f, Kar_2015, Kumar_2001}, 2 cases (3\%) died \cite{Emamhadi_2018, Kumar_2001}. All 90 were male gender. 90 cases (100\%) were detained at the time of ingestion \cite{Elghali_2016, Karp_1991b, Lee_2007}, 88 cases (98\%) were intentional ingestions \cite{Elghali_2016, Karp_1991b, Lee_2007}, 30 cases (33\%) had a psychiatric history documented \cite{Elghali_2016, Karp_1991b, Lee_2007}, 2 cases (2\%) had a history of prior ingestion \cite{Elghali_2016}. No cases were reported for were psychiatric inpatients, were displaced people, were under the influence of alcohol at the time of ingestion, and had a severe disability history.
\paragraph*{Motivation}  70 cases (78\%) reported protest motivation \cite{Elghali_2016, Karp_1991b, Lee_2007}, 12 cases (13\%) reported psychiatric motivation \cite{Karp_1991b}, 6 cases (7\%) reported self-harm motivation \cite{Elghali_2016, Karp_1991b}. No cases were reported for psychosocial motivation and other motivation.
\paragraph*{Object Characteristics}  68 cases (76\%) involved sharp object ingestion \cite{Elghali_2016, Karp_1991b, Lee_2007}, 32 cases (36\%) involved long (\textgreater 5cm) object ingestion \cite{Lee_2007}, 25 cases (28\%) involved ingestion of multiple objects \cite{Elghali_2016, Lee_2007}. No cases were reported for button battery ingestion, magnet ingestion, and involved large diameter (\textgreater 2.5cm) object ingestion.
\paragraph*{Outcomes}  47 cases (52\%) underwent endoscopic intervention \cite{Elghali_2016, Lee_2007}, 29 cases (32\%) were managed conservatively \cite{Elghali_2016, Karp_1991b}, 15 cases (17\%) underwent surgical intervention \cite{Elghali_2016, Karp_1991b, Lee_2007}, 6 cases (7\%) reported complications \cite{Lee_2007}, 1 case (1\%) died \cite{Elghali_2016}.
\paragraph*{Geographical Location}Cases were recorded in 33 countries: 13 cases from USA \cite{Alao_2006i, Ataya_2013, Bhumi_2024f, Fry_2010, Guinan_2019f, Hardy_2023g, Jehangir_2019h, Kerestes_2019, Kumar_2001, Liu_2005, Tammana_2012j, Tay_2004, Tupesis_2004f}; 7 cases from India \cite{Bhasin_2014, Bhattacharjee_2008, Kar_2015, Kariholu_2008, Kumar_2019f, Misra_2013, Wadhwa_2015e} and UK \cite{Beecroft_1998, Berry_2021e, Cauchi_2002, Cox_2007, Gardner_2017h, Qureshi_2016}; 6 cases from Bulgaria \cite{Losanoff_1996, Losanoff_1997e}; 5 cases from Iran \cite{DivsalarP._2023a, Emamhadi_2018, Farhadi_2024h}; 4 cases from Turkey \cite{Akay_2015f, Atayan_2016, Tanrikulu_2015e, Yildiz_2016e}; 2 cases from China \cite{Jin_2023, Li_2013}, Poland \cite{Kobiela_2015, Wnęk_2015f}, and Spain \cite{CamachoDorado_2018, fjbuilsRepeatedBehaviorDeliberate2024}; 1 case from Australia \cite{Apikotoa_2022f}, Bahrain \cite{Ali_2020f}, Croatia \cite{Trgo_2012f}, Ecuador \cite{DelgadoSalazar_2020c}, Egypt \cite{Ali_2022g}, Ethiopia \cite{Mesfin_2022a}, Germany \cite{teWildt_2010}, Greece \cite{Sakellaridis_2008f}, Hungary \cite{Csaky_1998e}, Iraq \cite{Al-Faham_2020k}, Israel \cite{Goldman_1998f}, Italy \cite{Riva_2018j}, Japan \cite{Ohno_2005}, Nepal \cite{Thapa_2019f}, Netherlands \cite{Benoist_2019e}, Oman \cite{AlShaaibi_2021b}, Pakistan \cite{Yasin_2009}, Portugal \cite{Peixoto_2017f}, Qatar \cite{Ali_2017}, Saudi Arabia \cite{Sultan_2024f}, South Africa \cite{Sobnach_2011f}, Sweden \cite{Naji_2012f}, Switzerland \cite{Wildhaber_2005}, and Taiwan \cite{Chang_2017f}. \paragraph*{Gender} 43 cases (60\%) were male \cite{Akay_2015f, Al-Faham_2020k, Alao_2006i, Ali_2017, Ali_2022g, Apikotoa_2022f, Atayan_2016, Benoist_2019e, Berry_2021e, Bhumi_2024f, CamachoDorado_2018, Csaky_1998e, Emamhadi_2018, Farhadi_2024h, Fry_2010, Gardner_2017h, Guinan_2019f, Jehangir_2019h, Jin_2023, Kobiela_2015, Kumar_2001, Kumar_2019f, Liu_2005, Losanoff_1996, Losanoff_1997e, Mesfin_2022a, Misra_2013, Qureshi_2016, Riva_2018j, Sobnach_2011f, Tammana_2012j, Tanrikulu_2015e, Tay_2004, Thapa_2019f, Trgo_2012f, Wadhwa_2015e, Yasin_2009, teWildt_2010}, 28 cases (39\%) were female \cite{AlShaaibi_2021b, Ali_2020f, Ataya_2013, Beecroft_1998, Bhasin_2014, Bhattacharjee_2008, Cauchi_2002, Chang_2017f, Cox_2007, DelgadoSalazar_2020c, DivsalarP._2023a, Goldman_1998f, Hardy_2023g, Kar_2015, Kariholu_2008, Kerestes_2019, Li_2013, Naji_2012f, Ohno_2005, Peixoto_2017f, Sakellaridis_2008f, Sultan_2024f, Tupesis_2004f, Wildhaber_2005, Wnęk_2015f, Yildiz_2016e}, 1 case (1\%) had no gender recorded \cite{fjbuilsRepeatedBehaviorDeliberate2024}. \paragraph*{Age Group} 25 cases (35\%) were between 26 and 40 years of age \cite{Alao_2006i, Ali_2022g, Apikotoa_2022f, Ataya_2013, Benoist_2019e, Bhasin_2014, Chang_2017f, Cox_2007, DelgadoSalazar_2020c, Farhadi_2024h, Fry_2010, Gardner_2017h, Guinan_2019f, Jin_2023, Kumar_2019f, Losanoff_1996, Misra_2013, Qureshi_2016, Riva_2018j, Sakellaridis_2008f, Tammana_2012j, Trgo_2012f, Wnęk_2015f, Yildiz_2016e, fjbuilsRepeatedBehaviorDeliberate2024}, 18 cases (25\%) were between 18 and 25 years of age \cite{Akay_2015f, Ali_2017, Atayan_2016, Bhattacharjee_2008, Csaky_1998e, Kar_2015, Kariholu_2008, Kobiela_2015, Losanoff_1996, Losanoff_1997e, Mesfin_2022a, Peixoto_2017f, Sobnach_2011f, Tupesis_2004f, Yasin_2009}, 13 cases (18\%) were under 18 years of age \cite{AlShaaibi_2021b, Ali_2020f, Cauchi_2002, DivsalarP._2023a, Goldman_1998f, Liu_2005, Naji_2012f, Ohno_2005, Tanrikulu_2015e, Tay_2004, Wildhaber_2005}, 11 cases (15\%) were between 41 and 60 years of age \cite{Al-Faham_2020k, Bhumi_2024f, CamachoDorado_2018, Emamhadi_2018, Hardy_2023g, Jehangir_2019h, Kumar_2001, Sultan_2024f, Thapa_2019f, Wadhwa_2015e, teWildt_2010}, 3 cases (4\%) were over 60 years of age \cite{Beecroft_1998, Kerestes_2019, Li_2013}, 2 cases (3\%) had no age documented \cite{Berry_2021e}. \paragraph*{Population} 36 cases (50\%) had a psychiatric history \cite{AlShaaibi_2021b, Alao_2006i, Ali_2020f, Apikotoa_2022f, Ataya_2013, Atayan_2016, Beecroft_1998, CamachoDorado_2018, Chang_2017f, DelgadoSalazar_2020c, DivsalarP._2023a, Farhadi_2024h, Fry_2010, Guinan_2019f, Hardy_2023g, Jehangir_2019h, Jin_2023, Kar_2015, Kerestes_2019, Kobiela_2015, Kumar_2001, Kumar_2019f, Liu_2005, Mesfin_2022a, Misra_2013, Ohno_2005, Peixoto_2017f, Sakellaridis_2008f, Sultan_2024f, Tammana_2012j, Tanrikulu_2015e, Yildiz_2016e, fjbuilsRepeatedBehaviorDeliberate2024, teWildt_2010}, 19 cases (26\%) had ingested previously \cite{Alao_2006i, Apikotoa_2022f, Berry_2021e, Bhattacharjee_2008, Csaky_1998e, DivsalarP._2023a, Emamhadi_2018, Guinan_2019f, Jehangir_2019h, Jin_2023, Liu_2005, Sakellaridis_2008f, Tanrikulu_2015e, Thapa_2019f, Yildiz_2016e, fjbuilsRepeatedBehaviorDeliberate2024, teWildt_2010}, 12 cases (17\%) were detained persons \cite{Alao_2006i, Ali_2022g, Apikotoa_2022f, Losanoff_1996, Losanoff_1997e, Qureshi_2016, Tammana_2012j, Trgo_2012f}, 7 cases (10\%) were severely disabled \cite{Atayan_2016, Kerestes_2019, Liu_2005, Ohno_2005, Peixoto_2017f, Yildiz_2016e, teWildt_2010}, 4 cases (6\%) were psychiatric inpatients \cite{DivsalarP._2023a, fjbuilsRepeatedBehaviorDeliberate2024, teWildt_2010}, 3 cases (4\%) were under the influence of alcohol \cite{Benoist_2019e, Csaky_1998e, Thapa_2019f}, 2 cases (3\%) were displaced people \cite{Akay_2015f, Gardner_2017h}. \paragraph*{Motivation} 34 cases (47\%) had a psychiatric motivation \cite{Al-Faham_2020k, Alao_2006i, Ali_2020f, Apikotoa_2022f, Ataya_2013, Atayan_2016, Bhasin_2014, Bhattacharjee_2008, DelgadoSalazar_2020c, DivsalarP._2023a, Emamhadi_2018, Farhadi_2024h, Guinan_2019f, Hardy_2023g, Jehangir_2019h, Jin_2023, Kar_2015, Kariholu_2008, Kerestes_2019, Kobiela_2015, Kumar_2001, Kumar_2019f, Li_2013, Liu_2005, Misra_2013, Ohno_2005, Sakellaridis_2008f, Sultan_2024f, Tammana_2012j, Tanrikulu_2015e, Yasin_2009, teWildt_2010}, 21 cases (29\%) were motivated by self-harm intention \cite{Al-Faham_2020k, AlShaaibi_2021b, Alao_2006i, Ali_2017, CamachoDorado_2018, Chang_2017f, Cox_2007, Csaky_1998e, Fry_2010, Li_2013, Losanoff_1996, Losanoff_1997e, Mesfin_2022a, Sakellaridis_2008f, Tammana_2012j, Tanrikulu_2015e, fjbuilsRepeatedBehaviorDeliberate2024}, 17 cases (24\%) had a psychosocial motivation \cite{Akay_2015f, Benoist_2019e, Bhattacharjee_2008, Cauchi_2002, Goldman_1998f, Hardy_2023g, Kobiela_2015, Li_2013, Naji_2012f, Qureshi_2016, Riva_2018j, Sobnach_2011f, Tay_2004, Thapa_2019f, Tupesis_2004f, Wildhaber_2005, Wnęk_2015f}, 9 cases (12\%) were motivated by protest \cite{Bhumi_2024f, Gardner_2017h, Losanoff_1996, Losanoff_1997e, Tupesis_2004f}, 9 cases (12\%) had another documented motivation \cite{Ali_2020f, Ali_2022g, Emamhadi_2018, Guinan_2019f, Peixoto_2017f, Sakellaridis_2008f, Trgo_2012f, Wadhwa_2015e, Yildiz_2016e}. \paragraph*{Object Characteristics} 51 cases (71\%) ingested a large diameter object (\textgreater{}2.5cm) \cite{Akay_2015f, Al-Faham_2020k, AlShaaibi_2021b, Alao_2006i, Ali_2017, Ali_2022g, Apikotoa_2022f, Atayan_2016, Berry_2021e, Bhasin_2014, CamachoDorado_2018, Cauchi_2002, Chang_2017f, Cox_2007, Csaky_1998e, DivsalarP._2023a, Emamhadi_2018, Gardner_2017h, Guinan_2019f, Jehangir_2019h, Jin_2023, Kariholu_2008, Kerestes_2019, Kobiela_2015, Kumar_2001, Kumar_2019f, Losanoff_1996, Losanoff_1997e, Mesfin_2022a, Misra_2013, Naji_2012f, Ohno_2005, Peixoto_2017f, Qureshi_2016, Riva_2018j, Sakellaridis_2008f, Sultan_2024f, Tanrikulu_2015e, Thapa_2019f, Trgo_2012f, Wnęk_2015f, Yildiz_2016e, fjbuilsRepeatedBehaviorDeliberate2024, teWildt_2010}, 44 cases (61\%) ingested multiple objects \cite{Ali_2020f, Apikotoa_2022f, Ataya_2013, Atayan_2016, Beecroft_1998, Bhattacharjee_2008, Bhumi_2024f, CamachoDorado_2018, Cauchi_2002, Emamhadi_2018, Farhadi_2024h, Fry_2010, Goldman_1998f, Guinan_2019f, Hardy_2023g, Jehangir_2019h, Jin_2023, Kar_2015, Kariholu_2008, Kobiela_2015, Kumar_2001, Kumar_2019f, Li_2013, Liu_2005, Losanoff_1996, Mesfin_2022a, Misra_2013, Naji_2012f, Ohno_2005, Sobnach_2011f, Sultan_2024f, Tammana_2012j, Tanrikulu_2015e, Tay_2004, Thapa_2019f, Wadhwa_2015e, Wildhaber_2005, Yasin_2009, fjbuilsRepeatedBehaviorDeliberate2024, teWildt_2010}, 34 cases (47\%) ingested a sharp object \cite{AlShaaibi_2021b, Alao_2006i, Apikotoa_2022f, Ataya_2013, Benoist_2019e, Bhasin_2014, Bhattacharjee_2008, CamachoDorado_2018, Csaky_1998e, DelgadoSalazar_2020c, DivsalarP._2023a, Emamhadi_2018, Farhadi_2024h, Fry_2010, Guinan_2019f, Hardy_2023g, Jehangir_2019h, Jin_2023, Kariholu_2008, Kobiela_2015, Kumar_2019f, Losanoff_1996, Losanoff_1997e, Mesfin_2022a, Misra_2013, Sobnach_2011f, Yasin_2009, teWildt_2010}, 32 cases (44\%) ingested a long object (\textgreater{}5cm) \cite{Al-Faham_2020k, AlShaaibi_2021b, Ali_2017, Ali_2022g, Atayan_2016, Bhasin_2014, CamachoDorado_2018, Chang_2017f, Cox_2007, Csaky_1998e, DivsalarP._2023a, Emamhadi_2018, Fry_2010, Gardner_2017h, Jin_2023, Kariholu_2008, Kerestes_2019, Kobiela_2015, Kumar_2019f, Mesfin_2022a, Misra_2013, Ohno_2005, Qureshi_2016, Sakellaridis_2008f, Sultan_2024f, Thapa_2019f, Trgo_2012f, Yasin_2009, Yildiz_2016e, teWildt_2010}, 9 cases (12\%) ingested a magnet \cite{Ali_2020f, Bhumi_2024f, Cauchi_2002, Liu_2005, Naji_2012f, Ohno_2005, Tanrikulu_2015e, Tay_2004, Wildhaber_2005}, 2 cases (3\%) ingested a button battery \cite{Berry_2021e, Bhumi_2024f}. \paragraph*{Outcomes} 48 cases (67\%) experienced a complication \cite{Ali_2017, Ali_2020f, Apikotoa_2022f, Atayan_2016, Beecroft_1998, Benoist_2019e, Berry_2021e, Bhasin_2014, Bhumi_2024f, CamachoDorado_2018, Cauchi_2002, Cox_2007, Csaky_1998e, DelgadoSalazar_2020c, DivsalarP._2023a, Emamhadi_2018, Farhadi_2024h, Fry_2010, Gardner_2017h, Goldman_1998f, Jin_2023, Kariholu_2008, Kerestes_2019, Kobiela_2015, Kumar_2001, Kumar_2019f, Liu_2005, Losanoff_1996, Mesfin_2022a, Misra_2013, Naji_2012f, Ohno_2005, Sakellaridis_2008f, Sobnach_2011f, Sultan_2024f, Tanrikulu_2015e, Tay_2004, Thapa_2019f, Trgo_2012f, Tupesis_2004f, Wildhaber_2005, Wnęk_2015f, Yasin_2009, Yildiz_2016e}, 44 cases (61\%) underwent surgery \cite{Al-Faham_2020k, AlShaaibi_2021b, Alao_2006i, Ali_2017, Ali_2020f, Atayan_2016, Beecroft_1998, Bhasin_2014, CamachoDorado_2018, Cauchi_2002, Chang_2017f, Cox_2007, Csaky_1998e, DelgadoSalazar_2020c, DivsalarP._2023a, Farhadi_2024h, Fry_2010, Gardner_2017h, Jin_2023, Kariholu_2008, Kerestes_2019, Kobiela_2015, Kumar_2019f, Liu_2005, Losanoff_1996, Losanoff_1997e, Mesfin_2022a, Misra_2013, Naji_2012f, Sobnach_2011f, Tanrikulu_2015e, Tay_2004, Thapa_2019f, Tupesis_2004f, Wildhaber_2005, Wnęk_2015f, Yasin_2009, Yildiz_2016e, fjbuilsRepeatedBehaviorDeliberate2024}, 31 cases (43\%) underwent endoscopy \cite{Akay_2015f, Ali_2022g, Apikotoa_2022f, Atayan_2016, Benoist_2019e, Berry_2021e, Bhasin_2014, Bhumi_2024f, CamachoDorado_2018, Chang_2017f, DelgadoSalazar_2020c, Gardner_2017h, Guinan_2019f, Hardy_2023g, Jehangir_2019h, Kariholu_2008, Li_2013, Liu_2005, Ohno_2005, Peixoto_2017f, Qureshi_2016, Riva_2018j, Sakellaridis_2008f, Sultan_2024f, Tammana_2012j, Tanrikulu_2015e, Trgo_2012f, Wadhwa_2015e, Wnęk_2015f, teWildt_2010}, 7 cases (10\%) were managed conservatively \cite{Ataya_2013, Bhattacharjee_2008, DivsalarP._2023a, Emamhadi_2018, Goldman_1998f, Kar_2015, Kumar_2001}, 2 cases (3\%) died \cite{Emamhadi_2018, Kumar_2001}. All 90 were male gender. 90 cases (100\%) were detained at the time of ingestion \cite{Elghali_2016, Karp_1991b, Lee_2007}, 88 cases (98\%) were intentional ingestions \cite{Elghali_2016, Karp_1991b, Lee_2007}, 30 cases (33\%) had a psychiatric history documented \cite{Elghali_2016, Karp_1991b, Lee_2007}, 2 cases (2\%) had a history of prior ingestion \cite{Elghali_2016}. No cases were reported for were psychiatric inpatients, were displaced people, were under the influence of alcohol at the time of ingestion, and had a severe disability history.
\paragraph*{Motivation}  70 cases (78\%) reported protest motivation \cite{Elghali_2016, Karp_1991b, Lee_2007}, 12 cases (13\%) reported psychiatric motivation \cite{Karp_1991b}, 6 cases (7\%) reported self-harm motivation \cite{Elghali_2016, Karp_1991b}. No cases were reported for psychosocial motivation and other motivation.
\paragraph*{Object Characteristics}  68 cases (76\%) involved sharp object ingestion \cite{Elghali_2016, Karp_1991b, Lee_2007}, 32 cases (36\%) involved long (\textgreater 5cm) object ingestion \cite{Lee_2007}, 25 cases (28\%) involved ingestion of multiple objects \cite{Elghali_2016, Lee_2007}. No cases were reported for button battery ingestion, magnet ingestion, and involved large diameter (\textgreater 2.5cm) object ingestion.
\paragraph*{Outcomes}  47 cases (52\%) underwent endoscopic intervention \cite{Elghali_2016, Lee_2007}, 29 cases (32\%) were managed conservatively \cite{Elghali_2016, Karp_1991b}, 15 cases (17\%) underwent surgical intervention \cite{Elghali_2016, Karp_1991b, Lee_2007}, 6 cases (7\%) reported complications \cite{Lee_2007}, 1 case (1\%) died \cite{Elghali_2016}.
\paragraph*{Geographical Location}Cases were recorded in 33 countries: 13 cases from USA \cite{Alao_2006i, Ataya_2013, Bhumi_2024f, Fry_2010, Guinan_2019f, Hardy_2023g, Jehangir_2019h, Kerestes_2019, Kumar_2001, Liu_2005, Tammana_2012j, Tay_2004, Tupesis_2004f}; 7 cases from India \cite{Bhasin_2014, Bhattacharjee_2008, Kar_2015, Kariholu_2008, Kumar_2019f, Misra_2013, Wadhwa_2015e} and UK \cite{Beecroft_1998, Berry_2021e, Cauchi_2002, Cox_2007, Gardner_2017h, Qureshi_2016}; 6 cases from Bulgaria \cite{Losanoff_1996, Losanoff_1997e}; 5 cases from Iran \cite{DivsalarP._2023a, Emamhadi_2018, Farhadi_2024h}; 4 cases from Turkey \cite{Akay_2015f, Atayan_2016, Tanrikulu_2015e, Yildiz_2016e}; 2 cases from China \cite{Jin_2023, Li_2013}, Poland \cite{Kobiela_2015, Wnęk_2015f}, and Spain \cite{CamachoDorado_2018, fjbuilsRepeatedBehaviorDeliberate2024}; 1 case from Australia \cite{Apikotoa_2022f}, Bahrain \cite{Ali_2020f}, Croatia \cite{Trgo_2012f}, Ecuador \cite{DelgadoSalazar_2020c}, Egypt \cite{Ali_2022g}, Ethiopia \cite{Mesfin_2022a}, Germany \cite{teWildt_2010}, Greece \cite{Sakellaridis_2008f}, Hungary \cite{Csaky_1998e}, Iraq \cite{Al-Faham_2020k}, Israel \cite{Goldman_1998f}, Italy \cite{Riva_2018j}, Japan \cite{Ohno_2005}, Nepal \cite{Thapa_2019f}, Netherlands \cite{Benoist_2019e}, Oman \cite{AlShaaibi_2021b}, Pakistan \cite{Yasin_2009}, Portugal \cite{Peixoto_2017f}, Qatar \cite{Ali_2017}, Saudi Arabia \cite{Sultan_2024f}, South Africa \cite{Sobnach_2011f}, Sweden \cite{Naji_2012f}, Switzerland \cite{Wildhaber_2005}, and Taiwan \cite{Chang_2017f}. \paragraph*{Gender} 43 cases (60\%) were male \cite{Akay_2015f, Al-Faham_2020k, Alao_2006i, Ali_2017, Ali_2022g, Apikotoa_2022f, Atayan_2016, Benoist_2019e, Berry_2021e, Bhumi_2024f, CamachoDorado_2018, Csaky_1998e, Emamhadi_2018, Farhadi_2024h, Fry_2010, Gardner_2017h, Guinan_2019f, Jehangir_2019h, Jin_2023, Kobiela_2015, Kumar_2001, Kumar_2019f, Liu_2005, Losanoff_1996, Losanoff_1997e, Mesfin_2022a, Misra_2013, Qureshi_2016, Riva_2018j, Sobnach_2011f, Tammana_2012j, Tanrikulu_2015e, Tay_2004, Thapa_2019f, Trgo_2012f, Wadhwa_2015e, Yasin_2009, teWildt_2010}, 28 cases (39\%) were female \cite{AlShaaibi_2021b, Ali_2020f, Ataya_2013, Beecroft_1998, Bhasin_2014, Bhattacharjee_2008, Cauchi_2002, Chang_2017f, Cox_2007, DelgadoSalazar_2020c, DivsalarP._2023a, Goldman_1998f, Hardy_2023g, Kar_2015, Kariholu_2008, Kerestes_2019, Li_2013, Naji_2012f, Ohno_2005, Peixoto_2017f, Sakellaridis_2008f, Sultan_2024f, Tupesis_2004f, Wildhaber_2005, Wnęk_2015f, Yildiz_2016e}, 1 case (1\%) had no gender recorded \cite{fjbuilsRepeatedBehaviorDeliberate2024}. \paragraph*{Age Group} 25 cases (35\%) were between 26 and 40 years of age \cite{Alao_2006i, Ali_2022g, Apikotoa_2022f, Ataya_2013, Benoist_2019e, Bhasin_2014, Chang_2017f, Cox_2007, DelgadoSalazar_2020c, Farhadi_2024h, Fry_2010, Gardner_2017h, Guinan_2019f, Jin_2023, Kumar_2019f, Losanoff_1996, Misra_2013, Qureshi_2016, Riva_2018j, Sakellaridis_2008f, Tammana_2012j, Trgo_2012f, Wnęk_2015f, Yildiz_2016e, fjbuilsRepeatedBehaviorDeliberate2024}, 18 cases (25\%) were between 18 and 25 years of age \cite{Akay_2015f, Ali_2017, Atayan_2016, Bhattacharjee_2008, Csaky_1998e, Kar_2015, Kariholu_2008, Kobiela_2015, Losanoff_1996, Losanoff_1997e, Mesfin_2022a, Peixoto_2017f, Sobnach_2011f, Tupesis_2004f, Yasin_2009}, 13 cases (18\%) were under 18 years of age \cite{AlShaaibi_2021b, Ali_2020f, Cauchi_2002, DivsalarP._2023a, Goldman_1998f, Liu_2005, Naji_2012f, Ohno_2005, Tanrikulu_2015e, Tay_2004, Wildhaber_2005}, 11 cases (15\%) were between 41 and 60 years of age \cite{Al-Faham_2020k, Bhumi_2024f, CamachoDorado_2018, Emamhadi_2018, Hardy_2023g, Jehangir_2019h, Kumar_2001, Sultan_2024f, Thapa_2019f, Wadhwa_2015e, teWildt_2010}, 3 cases (4\%) were over 60 years of age \cite{Beecroft_1998, Kerestes_2019, Li_2013}, 2 cases (3\%) had no age documented \cite{Berry_2021e}. \paragraph*{Population} 36 cases (50\%) had a psychiatric history \cite{AlShaaibi_2021b, Alao_2006i, Ali_2020f, Apikotoa_2022f, Ataya_2013, Atayan_2016, Beecroft_1998, CamachoDorado_2018, Chang_2017f, DelgadoSalazar_2020c, DivsalarP._2023a, Farhadi_2024h, Fry_2010, Guinan_2019f, Hardy_2023g, Jehangir_2019h, Jin_2023, Kar_2015, Kerestes_2019, Kobiela_2015, Kumar_2001, Kumar_2019f, Liu_2005, Mesfin_2022a, Misra_2013, Ohno_2005, Peixoto_2017f, Sakellaridis_2008f, Sultan_2024f, Tammana_2012j, Tanrikulu_2015e, Yildiz_2016e, fjbuilsRepeatedBehaviorDeliberate2024, teWildt_2010}, 19 cases (26\%) had ingested previously \cite{Alao_2006i, Apikotoa_2022f, Berry_2021e, Bhattacharjee_2008, Csaky_1998e, DivsalarP._2023a, Emamhadi_2018, Guinan_2019f, Jehangir_2019h, Jin_2023, Liu_2005, Sakellaridis_2008f, Tanrikulu_2015e, Thapa_2019f, Yildiz_2016e, fjbuilsRepeatedBehaviorDeliberate2024, teWildt_2010}, 12 cases (17\%) were detained persons \cite{Alao_2006i, Ali_2022g, Apikotoa_2022f, Losanoff_1996, Losanoff_1997e, Qureshi_2016, Tammana_2012j, Trgo_2012f}, 7 cases (10\%) were severely disabled \cite{Atayan_2016, Kerestes_2019, Liu_2005, Ohno_2005, Peixoto_2017f, Yildiz_2016e, teWildt_2010}, 4 cases (6\%) were psychiatric inpatients \cite{DivsalarP._2023a, fjbuilsRepeatedBehaviorDeliberate2024, teWildt_2010}, 3 cases (4\%) were under the influence of alcohol \cite{Benoist_2019e, Csaky_1998e, Thapa_2019f}, 2 cases (3\%) were displaced people \cite{Akay_2015f, Gardner_2017h}. \paragraph*{Motivation} 34 cases (47\%) had a psychiatric motivation \cite{Al-Faham_2020k, Alao_2006i, Ali_2020f, Apikotoa_2022f, Ataya_2013, Atayan_2016, Bhasin_2014, Bhattacharjee_2008, DelgadoSalazar_2020c, DivsalarP._2023a, Emamhadi_2018, Farhadi_2024h, Guinan_2019f, Hardy_2023g, Jehangir_2019h, Jin_2023, Kar_2015, Kariholu_2008, Kerestes_2019, Kobiela_2015, Kumar_2001, Kumar_2019f, Li_2013, Liu_2005, Misra_2013, Ohno_2005, Sakellaridis_2008f, Sultan_2024f, Tammana_2012j, Tanrikulu_2015e, Yasin_2009, teWildt_2010}, 21 cases (29\%) were motivated by self-harm intention \cite{Al-Faham_2020k, AlShaaibi_2021b, Alao_2006i, Ali_2017, CamachoDorado_2018, Chang_2017f, Cox_2007, Csaky_1998e, Fry_2010, Li_2013, Losanoff_1996, Losanoff_1997e, Mesfin_2022a, Sakellaridis_2008f, Tammana_2012j, Tanrikulu_2015e, fjbuilsRepeatedBehaviorDeliberate2024}, 17 cases (24\%) had a psychosocial motivation \cite{Akay_2015f, Benoist_2019e, Bhattacharjee_2008, Cauchi_2002, Goldman_1998f, Hardy_2023g, Kobiela_2015, Li_2013, Naji_2012f, Qureshi_2016, Riva_2018j, Sobnach_2011f, Tay_2004, Thapa_2019f, Tupesis_2004f, Wildhaber_2005, Wnęk_2015f}, 9 cases (12\%) were motivated by protest \cite{Bhumi_2024f, Gardner_2017h, Losanoff_1996, Losanoff_1997e, Tupesis_2004f}, 9 cases (12\%) had another documented motivation \cite{Ali_2020f, Ali_2022g, Emamhadi_2018, Guinan_2019f, Peixoto_2017f, Sakellaridis_2008f, Trgo_2012f, Wadhwa_2015e, Yildiz_2016e}. \paragraph*{Object Characteristics} 51 cases (71\%) ingested a large diameter object (\textgreater{}2.5cm) \cite{Akay_2015f, Al-Faham_2020k, AlShaaibi_2021b, Alao_2006i, Ali_2017, Ali_2022g, Apikotoa_2022f, Atayan_2016, Berry_2021e, Bhasin_2014, CamachoDorado_2018, Cauchi_2002, Chang_2017f, Cox_2007, Csaky_1998e, DivsalarP._2023a, Emamhadi_2018, Gardner_2017h, Guinan_2019f, Jehangir_2019h, Jin_2023, Kariholu_2008, Kerestes_2019, Kobiela_2015, Kumar_2001, Kumar_2019f, Losanoff_1996, Losanoff_1997e, Mesfin_2022a, Misra_2013, Naji_2012f, Ohno_2005, Peixoto_2017f, Qureshi_2016, Riva_2018j, Sakellaridis_2008f, Sultan_2024f, Tanrikulu_2015e, Thapa_2019f, Trgo_2012f, Wnęk_2015f, Yildiz_2016e, fjbuilsRepeatedBehaviorDeliberate2024, teWildt_2010}, 44 cases (61\%) ingested multiple objects \cite{Ali_2020f, Apikotoa_2022f, Ataya_2013, Atayan_2016, Beecroft_1998, Bhattacharjee_2008, Bhumi_2024f, CamachoDorado_2018, Cauchi_2002, Emamhadi_2018, Farhadi_2024h, Fry_2010, Goldman_1998f, Guinan_2019f, Hardy_2023g, Jehangir_2019h, Jin_2023, Kar_2015, Kariholu_2008, Kobiela_2015, Kumar_2001, Kumar_2019f, Li_2013, Liu_2005, Losanoff_1996, Mesfin_2022a, Misra_2013, Naji_2012f, Ohno_2005, Sobnach_2011f, Sultan_2024f, Tammana_2012j, Tanrikulu_2015e, Tay_2004, Thapa_2019f, Wadhwa_2015e, Wildhaber_2005, Yasin_2009, fjbuilsRepeatedBehaviorDeliberate2024, teWildt_2010}, 34 cases (47\%) ingested a sharp object \cite{AlShaaibi_2021b, Alao_2006i, Apikotoa_2022f, Ataya_2013, Benoist_2019e, Bhasin_2014, Bhattacharjee_2008, CamachoDorado_2018, Csaky_1998e, DelgadoSalazar_2020c, DivsalarP._2023a, Emamhadi_2018, Farhadi_2024h, Fry_2010, Guinan_2019f, Hardy_2023g, Jehangir_2019h, Jin_2023, Kariholu_2008, Kobiela_2015, Kumar_2019f, Losanoff_1996, Losanoff_1997e, Mesfin_2022a, Misra_2013, Sobnach_2011f, Yasin_2009, teWildt_2010}, 32 cases (44\%) ingested a long object (\textgreater{}5cm) \cite{Al-Faham_2020k, AlShaaibi_2021b, Ali_2017, Ali_2022g, Atayan_2016, Bhasin_2014, CamachoDorado_2018, Chang_2017f, Cox_2007, Csaky_1998e, DivsalarP._2023a, Emamhadi_2018, Fry_2010, Gardner_2017h, Jin_2023, Kariholu_2008, Kerestes_2019, Kobiela_2015, Kumar_2019f, Mesfin_2022a, Misra_2013, Ohno_2005, Qureshi_2016, Sakellaridis_2008f, Sultan_2024f, Thapa_2019f, Trgo_2012f, Yasin_2009, Yildiz_2016e, teWildt_2010}, 9 cases (12\%) ingested a magnet \cite{Ali_2020f, Bhumi_2024f, Cauchi_2002, Liu_2005, Naji_2012f, Ohno_2005, Tanrikulu_2015e, Tay_2004, Wildhaber_2005}, 2 cases (3\%) ingested a button battery \cite{Berry_2021e, Bhumi_2024f}. \paragraph*{Outcomes} 48 cases (67\%) experienced a complication \cite{Ali_2017, Ali_2020f, Apikotoa_2022f, Atayan_2016, Beecroft_1998, Benoist_2019e, Berry_2021e, Bhasin_2014, Bhumi_2024f, CamachoDorado_2018, Cauchi_2002, Cox_2007, Csaky_1998e, DelgadoSalazar_2020c, DivsalarP._2023a, Emamhadi_2018, Farhadi_2024h, Fry_2010, Gardner_2017h, Goldman_1998f, Jin_2023, Kariholu_2008, Kerestes_2019, Kobiela_2015, Kumar_2001, Kumar_2019f, Liu_2005, Losanoff_1996, Mesfin_2022a, Misra_2013, Naji_2012f, Ohno_2005, Sakellaridis_2008f, Sobnach_2011f, Sultan_2024f, Tanrikulu_2015e, Tay_2004, Thapa_2019f, Trgo_2012f, Tupesis_2004f, Wildhaber_2005, Wnęk_2015f, Yasin_2009, Yildiz_2016e}, 44 cases (61\%) underwent surgery \cite{Al-Faham_2020k, AlShaaibi_2021b, Alao_2006i, Ali_2017, Ali_2020f, Atayan_2016, Beecroft_1998, Bhasin_2014, CamachoDorado_2018, Cauchi_2002, Chang_2017f, Cox_2007, Csaky_1998e, DelgadoSalazar_2020c, DivsalarP._2023a, Farhadi_2024h, Fry_2010, Gardner_2017h, Jin_2023, Kariholu_2008, Kerestes_2019, Kobiela_2015, Kumar_2019f, Liu_2005, Losanoff_1996, Losanoff_1997e, Mesfin_2022a, Misra_2013, Naji_2012f, Sobnach_2011f, Tanrikulu_2015e, Tay_2004, Thapa_2019f, Tupesis_2004f, Wildhaber_2005, Wnęk_2015f, Yasin_2009, Yildiz_2016e, fjbuilsRepeatedBehaviorDeliberate2024}, 31 cases (43\%) underwent endoscopy \cite{Akay_2015f, Ali_2022g, Apikotoa_2022f, Atayan_2016, Benoist_2019e, Berry_2021e, Bhasin_2014, Bhumi_2024f, CamachoDorado_2018, Chang_2017f, DelgadoSalazar_2020c, Gardner_2017h, Guinan_2019f, Hardy_2023g, Jehangir_2019h, Kariholu_2008, Li_2013, Liu_2005, Ohno_2005, Peixoto_2017f, Qureshi_2016, Riva_2018j, Sakellaridis_2008f, Sultan_2024f, Tammana_2012j, Tanrikulu_2015e, Trgo_2012f, Wadhwa_2015e, Wnęk_2015f, teWildt_2010}, 7 cases (10\%) were managed conservatively \cite{Ataya_2013, Bhattacharjee_2008, DivsalarP._2023a, Emamhadi_2018, Goldman_1998f, Kar_2015, Kumar_2001}, 2 cases (3\%) died \cite{Emamhadi_2018, Kumar_2001}. All 90 were male gender. 90 cases (100\%) were detained at the time of ingestion \cite{Elghali_2016, Karp_1991b, Lee_2007}, 88 cases (98\%) were intentional ingestions \cite{Elghali_2016, Karp_1991b, Lee_2007}, 30 cases (33\%) had a psychiatric history documented \cite{Elghali_2016, Karp_1991b, Lee_2007}, 2 cases (2\%) had a history of prior ingestion \cite{Elghali_2016}. No cases were reported for were psychiatric inpatients, were displaced people, were under the influence of alcohol at the time of ingestion, and had a severe disability history.
\paragraph*{Motivation}  70 cases (78\%) reported protest motivation \cite{Elghali_2016, Karp_1991b, Lee_2007}, 12 cases (13\%) reported psychiatric motivation \cite{Karp_1991b}, 6 cases (7\%) reported self-harm motivation \cite{Elghali_2016, Karp_1991b}. No cases were reported for psychosocial motivation and other motivation.
\paragraph*{Object Characteristics}  68 cases (76\%) involved sharp object ingestion \cite{Elghali_2016, Karp_1991b, Lee_2007}, 32 cases (36\%) involved long (\textgreater 5cm) object ingestion \cite{Lee_2007}, 25 cases (28\%) involved ingestion of multiple objects \cite{Elghali_2016, Lee_2007}. No cases were reported for button battery ingestion, magnet ingestion, and involved large diameter (\textgreater 2.5cm) object ingestion.
\paragraph*{Outcomes}  47 cases (52\%) underwent endoscopic intervention \cite{Elghali_2016, Lee_2007}, 29 cases (32\%) were managed conservatively \cite{Elghali_2016, Karp_1991b}, 15 cases (17\%) underwent surgical intervention \cite{Elghali_2016, Karp_1991b, Lee_2007}, 6 cases (7\%) reported complications \cite{Lee_2007}, 1 case (1\%) died \cite{Elghali_2016}.
\paragraph*{Geographical Location}Cases were recorded in 33 countries: 13 cases from USA \cite{Alao_2006i, Ataya_2013, Bhumi_2024f, Fry_2010, Guinan_2019f, Hardy_2023g, Jehangir_2019h, Kerestes_2019, Kumar_2001, Liu_2005, Tammana_2012j, Tay_2004, Tupesis_2004f}; 7 cases from India \cite{Bhasin_2014, Bhattacharjee_2008, Kar_2015, Kariholu_2008, Kumar_2019f, Misra_2013, Wadhwa_2015e} and UK \cite{Beecroft_1998, Berry_2021e, Cauchi_2002, Cox_2007, Gardner_2017h, Qureshi_2016}; 6 cases from Bulgaria \cite{Losanoff_1996, Losanoff_1997e}; 5 cases from Iran \cite{DivsalarP._2023a, Emamhadi_2018, Farhadi_2024h}; 4 cases from Turkey \cite{Akay_2015f, Atayan_2016, Tanrikulu_2015e, Yildiz_2016e}; 2 cases from China \cite{Jin_2023, Li_2013}, Poland \cite{Kobiela_2015, Wnęk_2015f}, and Spain \cite{CamachoDorado_2018, fjbuilsRepeatedBehaviorDeliberate2024}; 1 case from Australia \cite{Apikotoa_2022f}, Bahrain \cite{Ali_2020f}, Croatia \cite{Trgo_2012f}, Ecuador \cite{DelgadoSalazar_2020c}, Egypt \cite{Ali_2022g}, Ethiopia \cite{Mesfin_2022a}, Germany \cite{teWildt_2010}, Greece \cite{Sakellaridis_2008f}, Hungary \cite{Csaky_1998e}, Iraq \cite{Al-Faham_2020k}, Israel \cite{Goldman_1998f}, Italy \cite{Riva_2018j}, Japan \cite{Ohno_2005}, Nepal \cite{Thapa_2019f}, Netherlands \cite{Benoist_2019e}, Oman \cite{AlShaaibi_2021b}, Pakistan \cite{Yasin_2009}, Portugal \cite{Peixoto_2017f}, Qatar \cite{Ali_2017}, Saudi Arabia \cite{Sultan_2024f}, South Africa \cite{Sobnach_2011f}, Sweden \cite{Naji_2012f}, Switzerland \cite{Wildhaber_2005}, and Taiwan \cite{Chang_2017f}. \paragraph*{Gender} 43 cases (60\%) were male \cite{Akay_2015f, Al-Faham_2020k, Alao_2006i, Ali_2017, Ali_2022g, Apikotoa_2022f, Atayan_2016, Benoist_2019e, Berry_2021e, Bhumi_2024f, CamachoDorado_2018, Csaky_1998e, Emamhadi_2018, Farhadi_2024h, Fry_2010, Gardner_2017h, Guinan_2019f, Jehangir_2019h, Jin_2023, Kobiela_2015, Kumar_2001, Kumar_2019f, Liu_2005, Losanoff_1996, Losanoff_1997e, Mesfin_2022a, Misra_2013, Qureshi_2016, Riva_2018j, Sobnach_2011f, Tammana_2012j, Tanrikulu_2015e, Tay_2004, Thapa_2019f, Trgo_2012f, Wadhwa_2015e, Yasin_2009, teWildt_2010}, 28 cases (39\%) were female \cite{AlShaaibi_2021b, Ali_2020f, Ataya_2013, Beecroft_1998, Bhasin_2014, Bhattacharjee_2008, Cauchi_2002, Chang_2017f, Cox_2007, DelgadoSalazar_2020c, DivsalarP._2023a, Goldman_1998f, Hardy_2023g, Kar_2015, Kariholu_2008, Kerestes_2019, Li_2013, Naji_2012f, Ohno_2005, Peixoto_2017f, Sakellaridis_2008f, Sultan_2024f, Tupesis_2004f, Wildhaber_2005, Wnęk_2015f, Yildiz_2016e}, 1 case (1\%) had no gender recorded \cite{fjbuilsRepeatedBehaviorDeliberate2024}. \paragraph*{Age Group} 25 cases (35\%) were between 26 and 40 years of age \cite{Alao_2006i, Ali_2022g, Apikotoa_2022f, Ataya_2013, Benoist_2019e, Bhasin_2014, Chang_2017f, Cox_2007, DelgadoSalazar_2020c, Farhadi_2024h, Fry_2010, Gardner_2017h, Guinan_2019f, Jin_2023, Kumar_2019f, Losanoff_1996, Misra_2013, Qureshi_2016, Riva_2018j, Sakellaridis_2008f, Tammana_2012j, Trgo_2012f, Wnęk_2015f, Yildiz_2016e, fjbuilsRepeatedBehaviorDeliberate2024}, 18 cases (25\%) were between 18 and 25 years of age \cite{Akay_2015f, Ali_2017, Atayan_2016, Bhattacharjee_2008, Csaky_1998e, Kar_2015, Kariholu_2008, Kobiela_2015, Losanoff_1996, Losanoff_1997e, Mesfin_2022a, Peixoto_2017f, Sobnach_2011f, Tupesis_2004f, Yasin_2009}, 13 cases (18\%) were under 18 years of age \cite{AlShaaibi_2021b, Ali_2020f, Cauchi_2002, DivsalarP._2023a, Goldman_1998f, Liu_2005, Naji_2012f, Ohno_2005, Tanrikulu_2015e, Tay_2004, Wildhaber_2005}, 11 cases (15\%) were between 41 and 60 years of age \cite{Al-Faham_2020k, Bhumi_2024f, CamachoDorado_2018, Emamhadi_2018, Hardy_2023g, Jehangir_2019h, Kumar_2001, Sultan_2024f, Thapa_2019f, Wadhwa_2015e, teWildt_2010}, 3 cases (4\%) were over 60 years of age \cite{Beecroft_1998, Kerestes_2019, Li_2013}, 2 cases (3\%) had no age documented \cite{Berry_2021e}. \paragraph*{Population} 36 cases (50\%) had a psychiatric history \cite{AlShaaibi_2021b, Alao_2006i, Ali_2020f, Apikotoa_2022f, Ataya_2013, Atayan_2016, Beecroft_1998, CamachoDorado_2018, Chang_2017f, DelgadoSalazar_2020c, DivsalarP._2023a, Farhadi_2024h, Fry_2010, Guinan_2019f, Hardy_2023g, Jehangir_2019h, Jin_2023, Kar_2015, Kerestes_2019, Kobiela_2015, Kumar_2001, Kumar_2019f, Liu_2005, Mesfin_2022a, Misra_2013, Ohno_2005, Peixoto_2017f, Sakellaridis_2008f, Sultan_2024f, Tammana_2012j, Tanrikulu_2015e, Yildiz_2016e, fjbuilsRepeatedBehaviorDeliberate2024, teWildt_2010}, 19 cases (26\%) had ingested previously \cite{Alao_2006i, Apikotoa_2022f, Berry_2021e, Bhattacharjee_2008, Csaky_1998e, DivsalarP._2023a, Emamhadi_2018, Guinan_2019f, Jehangir_2019h, Jin_2023, Liu_2005, Sakellaridis_2008f, Tanrikulu_2015e, Thapa_2019f, Yildiz_2016e, fjbuilsRepeatedBehaviorDeliberate2024, teWildt_2010}, 12 cases (17\%) were detained persons \cite{Alao_2006i, Ali_2022g, Apikotoa_2022f, Losanoff_1996, Losanoff_1997e, Qureshi_2016, Tammana_2012j, Trgo_2012f}, 7 cases (10\%) were severely disabled \cite{Atayan_2016, Kerestes_2019, Liu_2005, Ohno_2005, Peixoto_2017f, Yildiz_2016e, teWildt_2010}, 4 cases (6\%) were psychiatric inpatients \cite{DivsalarP._2023a, fjbuilsRepeatedBehaviorDeliberate2024, teWildt_2010}, 3 cases (4\%) were under the influence of alcohol \cite{Benoist_2019e, Csaky_1998e, Thapa_2019f}, 2 cases (3\%) were displaced people \cite{Akay_2015f, Gardner_2017h}. \paragraph*{Motivation} 34 cases (47\%) had a psychiatric motivation \cite{Al-Faham_2020k, Alao_2006i, Ali_2020f, Apikotoa_2022f, Ataya_2013, Atayan_2016, Bhasin_2014, Bhattacharjee_2008, DelgadoSalazar_2020c, DivsalarP._2023a, Emamhadi_2018, Farhadi_2024h, Guinan_2019f, Hardy_2023g, Jehangir_2019h, Jin_2023, Kar_2015, Kariholu_2008, Kerestes_2019, Kobiela_2015, Kumar_2001, Kumar_2019f, Li_2013, Liu_2005, Misra_2013, Ohno_2005, Sakellaridis_2008f, Sultan_2024f, Tammana_2012j, Tanrikulu_2015e, Yasin_2009, teWildt_2010}, 21 cases (29\%) were motivated by self-harm intention \cite{Al-Faham_2020k, AlShaaibi_2021b, Alao_2006i, Ali_2017, CamachoDorado_2018, Chang_2017f, Cox_2007, Csaky_1998e, Fry_2010, Li_2013, Losanoff_1996, Losanoff_1997e, Mesfin_2022a, Sakellaridis_2008f, Tammana_2012j, Tanrikulu_2015e, fjbuilsRepeatedBehaviorDeliberate2024}, 17 cases (24\%) had a psychosocial motivation \cite{Akay_2015f, Benoist_2019e, Bhattacharjee_2008, Cauchi_2002, Goldman_1998f, Hardy_2023g, Kobiela_2015, Li_2013, Naji_2012f, Qureshi_2016, Riva_2018j, Sobnach_2011f, Tay_2004, Thapa_2019f, Tupesis_2004f, Wildhaber_2005, Wnęk_2015f}, 9 cases (12\%) were motivated by protest \cite{Bhumi_2024f, Gardner_2017h, Losanoff_1996, Losanoff_1997e, Tupesis_2004f}, 9 cases (12\%) had another documented motivation \cite{Ali_2020f, Ali_2022g, Emamhadi_2018, Guinan_2019f, Peixoto_2017f, Sakellaridis_2008f, Trgo_2012f, Wadhwa_2015e, Yildiz_2016e}. \paragraph*{Object Characteristics} 51 cases (71\%) ingested a large diameter object (\textgreater{}2.5cm) \cite{Akay_2015f, Al-Faham_2020k, AlShaaibi_2021b, Alao_2006i, Ali_2017, Ali_2022g, Apikotoa_2022f, Atayan_2016, Berry_2021e, Bhasin_2014, CamachoDorado_2018, Cauchi_2002, Chang_2017f, Cox_2007, Csaky_1998e, DivsalarP._2023a, Emamhadi_2018, Gardner_2017h, Guinan_2019f, Jehangir_2019h, Jin_2023, Kariholu_2008, Kerestes_2019, Kobiela_2015, Kumar_2001, Kumar_2019f, Losanoff_1996, Losanoff_1997e, Mesfin_2022a, Misra_2013, Naji_2012f, Ohno_2005, Peixoto_2017f, Qureshi_2016, Riva_2018j, Sakellaridis_2008f, Sultan_2024f, Tanrikulu_2015e, Thapa_2019f, Trgo_2012f, Wnęk_2015f, Yildiz_2016e, fjbuilsRepeatedBehaviorDeliberate2024, teWildt_2010}, 44 cases (61\%) ingested multiple objects \cite{Ali_2020f, Apikotoa_2022f, Ataya_2013, Atayan_2016, Beecroft_1998, Bhattacharjee_2008, Bhumi_2024f, CamachoDorado_2018, Cauchi_2002, Emamhadi_2018, Farhadi_2024h, Fry_2010, Goldman_1998f, Guinan_2019f, Hardy_2023g, Jehangir_2019h, Jin_2023, Kar_2015, Kariholu_2008, Kobiela_2015, Kumar_2001, Kumar_2019f, Li_2013, Liu_2005, Losanoff_1996, Mesfin_2022a, Misra_2013, Naji_2012f, Ohno_2005, Sobnach_2011f, Sultan_2024f, Tammana_2012j, Tanrikulu_2015e, Tay_2004, Thapa_2019f, Wadhwa_2015e, Wildhaber_2005, Yasin_2009, fjbuilsRepeatedBehaviorDeliberate2024, teWildt_2010}, 34 cases (47\%) ingested a sharp object \cite{AlShaaibi_2021b, Alao_2006i, Apikotoa_2022f, Ataya_2013, Benoist_2019e, Bhasin_2014, Bhattacharjee_2008, CamachoDorado_2018, Csaky_1998e, DelgadoSalazar_2020c, DivsalarP._2023a, Emamhadi_2018, Farhadi_2024h, Fry_2010, Guinan_2019f, Hardy_2023g, Jehangir_2019h, Jin_2023, Kariholu_2008, Kobiela_2015, Kumar_2019f, Losanoff_1996, Losanoff_1997e, Mesfin_2022a, Misra_2013, Sobnach_2011f, Yasin_2009, teWildt_2010}, 32 cases (44\%) ingested a long object (\textgreater{}5cm) \cite{Al-Faham_2020k, AlShaaibi_2021b, Ali_2017, Ali_2022g, Atayan_2016, Bhasin_2014, CamachoDorado_2018, Chang_2017f, Cox_2007, Csaky_1998e, DivsalarP._2023a, Emamhadi_2018, Fry_2010, Gardner_2017h, Jin_2023, Kariholu_2008, Kerestes_2019, Kobiela_2015, Kumar_2019f, Mesfin_2022a, Misra_2013, Ohno_2005, Qureshi_2016, Sakellaridis_2008f, Sultan_2024f, Thapa_2019f, Trgo_2012f, Yasin_2009, Yildiz_2016e, teWildt_2010}, 9 cases (12\%) ingested a magnet \cite{Ali_2020f, Bhumi_2024f, Cauchi_2002, Liu_2005, Naji_2012f, Ohno_2005, Tanrikulu_2015e, Tay_2004, Wildhaber_2005}, 2 cases (3\%) ingested a button battery \cite{Berry_2021e, Bhumi_2024f}. \paragraph*{Outcomes} 48 cases (67\%) experienced a complication \cite{Ali_2017, Ali_2020f, Apikotoa_2022f, Atayan_2016, Beecroft_1998, Benoist_2019e, Berry_2021e, Bhasin_2014, Bhumi_2024f, CamachoDorado_2018, Cauchi_2002, Cox_2007, Csaky_1998e, DelgadoSalazar_2020c, DivsalarP._2023a, Emamhadi_2018, Farhadi_2024h, Fry_2010, Gardner_2017h, Goldman_1998f, Jin_2023, Kariholu_2008, Kerestes_2019, Kobiela_2015, Kumar_2001, Kumar_2019f, Liu_2005, Losanoff_1996, Mesfin_2022a, Misra_2013, Naji_2012f, Ohno_2005, Sakellaridis_2008f, Sobnach_2011f, Sultan_2024f, Tanrikulu_2015e, Tay_2004, Thapa_2019f, Trgo_2012f, Tupesis_2004f, Wildhaber_2005, Wnęk_2015f, Yasin_2009, Yildiz_2016e}, 44 cases (61\%) underwent surgery \cite{Al-Faham_2020k, AlShaaibi_2021b, Alao_2006i, Ali_2017, Ali_2020f, Atayan_2016, Beecroft_1998, Bhasin_2014, CamachoDorado_2018, Cauchi_2002, Chang_2017f, Cox_2007, Csaky_1998e, DelgadoSalazar_2020c, DivsalarP._2023a, Farhadi_2024h, Fry_2010, Gardner_2017h, Jin_2023, Kariholu_2008, Kerestes_2019, Kobiela_2015, Kumar_2019f, Liu_2005, Losanoff_1996, Losanoff_1997e, Mesfin_2022a, Misra_2013, Naji_2012f, Sobnach_2011f, Tanrikulu_2015e, Tay_2004, Thapa_2019f, Tupesis_2004f, Wildhaber_2005, Wnęk_2015f, Yasin_2009, Yildiz_2016e, fjbuilsRepeatedBehaviorDeliberate2024}, 31 cases (43\%) underwent endoscopy \cite{Akay_2015f, Ali_2022g, Apikotoa_2022f, Atayan_2016, Benoist_2019e, Berry_2021e, Bhasin_2014, Bhumi_2024f, CamachoDorado_2018, Chang_2017f, DelgadoSalazar_2020c, Gardner_2017h, Guinan_2019f, Hardy_2023g, Jehangir_2019h, Kariholu_2008, Li_2013, Liu_2005, Ohno_2005, Peixoto_2017f, Qureshi_2016, Riva_2018j, Sakellaridis_2008f, Sultan_2024f, Tammana_2012j, Tanrikulu_2015e, Trgo_2012f, Wadhwa_2015e, Wnęk_2015f, teWildt_2010}, 7 cases (10\%) were managed conservatively \cite{Ataya_2013, Bhattacharjee_2008, DivsalarP._2023a, Emamhadi_2018, Goldman_1998f, Kar_2015, Kumar_2001}, 2 cases (3\%) died \cite{Emamhadi_2018, Kumar_2001}. All 90 were male gender. 90 cases (100\%) were detained at the time of ingestion \cite{Elghali_2016, Karp_1991b, Lee_2007}, 88 cases (98\%) were intentional ingestions \cite{Elghali_2016, Karp_1991b, Lee_2007}, 30 cases (33\%) had a psychiatric history documented \cite{Elghali_2016, Karp_1991b, Lee_2007}, 2 cases (2\%) had a history of prior ingestion \cite{Elghali_2016}. No cases were reported for were psychiatric inpatients, were displaced people, were under the influence of alcohol at the time of ingestion, and had a severe disability history.
\paragraph*{Motivation}  70 cases (78\%) reported protest motivation \cite{Elghali_2016, Karp_1991b, Lee_2007}, 12 cases (13\%) reported psychiatric motivation \cite{Karp_1991b}, 6 cases (7\%) reported self-harm motivation \cite{Elghali_2016, Karp_1991b}. No cases were reported for psychosocial motivation and other motivation.
\paragraph*{Object Characteristics}  68 cases (76\%) involved sharp object ingestion \cite{Elghali_2016, Karp_1991b, Lee_2007}, 32 cases (36\%) involved long (\textgreater 5cm) object ingestion \cite{Lee_2007}, 25 cases (28\%) involved ingestion of multiple objects \cite{Elghali_2016, Lee_2007}. No cases were reported for button battery ingestion, magnet ingestion, and involved large diameter (\textgreater 2.5cm) object ingestion.
\paragraph*{Outcomes}  47 cases (52\%) underwent endoscopic intervention \cite{Elghali_2016, Lee_2007}, 29 cases (32\%) were managed conservatively \cite{Elghali_2016, Karp_1991b}, 15 cases (17\%) underwent surgical intervention \cite{Elghali_2016, Karp_1991b, Lee_2007}, 6 cases (7\%) reported complications \cite{Lee_2007}, 1 case (1\%) died \cite{Elghali_2016}.
\paragraph*{Geographical Location}Cases were recorded in 33 countries: 13 cases from USA \cite{Alao_2006i, Ataya_2013, Bhumi_2024f, Fry_2010, Guinan_2019f, Hardy_2023g, Jehangir_2019h, Kerestes_2019, Kumar_2001, Liu_2005, Tammana_2012j, Tay_2004, Tupesis_2004f}; 7 cases from India \cite{Bhasin_2014, Bhattacharjee_2008, Kar_2015, Kariholu_2008, Kumar_2019f, Misra_2013, Wadhwa_2015e} and UK \cite{Beecroft_1998, Berry_2021e, Cauchi_2002, Cox_2007, Gardner_2017h, Qureshi_2016}; 6 cases from Bulgaria \cite{Losanoff_1996, Losanoff_1997e}; 5 cases from Iran \cite{DivsalarP._2023a, Emamhadi_2018, Farhadi_2024h}; 4 cases from Turkey \cite{Akay_2015f, Atayan_2016, Tanrikulu_2015e, Yildiz_2016e}; 2 cases from China \cite{Jin_2023, Li_2013}, Poland \cite{Kobiela_2015, Wnęk_2015f}, and Spain \cite{CamachoDorado_2018, fjbuilsRepeatedBehaviorDeliberate2024}; 1 case from Australia \cite{Apikotoa_2022f}, Bahrain \cite{Ali_2020f}, Croatia \cite{Trgo_2012f}, Ecuador \cite{DelgadoSalazar_2020c}, Egypt \cite{Ali_2022g}, Ethiopia \cite{Mesfin_2022a}, Germany \cite{teWildt_2010}, Greece \cite{Sakellaridis_2008f}, Hungary \cite{Csaky_1998e}, Iraq \cite{Al-Faham_2020k}, Israel \cite{Goldman_1998f}, Italy \cite{Riva_2018j}, Japan \cite{Ohno_2005}, Nepal \cite{Thapa_2019f}, Netherlands \cite{Benoist_2019e}, Oman \cite{AlShaaibi_2021b}, Pakistan \cite{Yasin_2009}, Portugal \cite{Peixoto_2017f}, Qatar \cite{Ali_2017}, Saudi Arabia \cite{Sultan_2024f}, South Africa \cite{Sobnach_2011f}, Sweden \cite{Naji_2012f}, Switzerland \cite{Wildhaber_2005}, and Taiwan \cite{Chang_2017f}. \paragraph*{Gender} 43 cases (60\%) were male \cite{Akay_2015f, Al-Faham_2020k, Alao_2006i, Ali_2017, Ali_2022g, Apikotoa_2022f, Atayan_2016, Benoist_2019e, Berry_2021e, Bhumi_2024f, CamachoDorado_2018, Csaky_1998e, Emamhadi_2018, Farhadi_2024h, Fry_2010, Gardner_2017h, Guinan_2019f, Jehangir_2019h, Jin_2023, Kobiela_2015, Kumar_2001, Kumar_2019f, Liu_2005, Losanoff_1996, Losanoff_1997e, Mesfin_2022a, Misra_2013, Qureshi_2016, Riva_2018j, Sobnach_2011f, Tammana_2012j, Tanrikulu_2015e, Tay_2004, Thapa_2019f, Trgo_2012f, Wadhwa_2015e, Yasin_2009, teWildt_2010}, 28 cases (39\%) were female \cite{AlShaaibi_2021b, Ali_2020f, Ataya_2013, Beecroft_1998, Bhasin_2014, Bhattacharjee_2008, Cauchi_2002, Chang_2017f, Cox_2007, DelgadoSalazar_2020c, DivsalarP._2023a, Goldman_1998f, Hardy_2023g, Kar_2015, Kariholu_2008, Kerestes_2019, Li_2013, Naji_2012f, Ohno_2005, Peixoto_2017f, Sakellaridis_2008f, Sultan_2024f, Tupesis_2004f, Wildhaber_2005, Wnęk_2015f, Yildiz_2016e}, 1 case (1\%) had no gender recorded \cite{fjbuilsRepeatedBehaviorDeliberate2024}. \paragraph*{Age Group} 25 cases (35\%) were between 26 and 40 years of age \cite{Alao_2006i, Ali_2022g, Apikotoa_2022f, Ataya_2013, Benoist_2019e, Bhasin_2014, Chang_2017f, Cox_2007, DelgadoSalazar_2020c, Farhadi_2024h, Fry_2010, Gardner_2017h, Guinan_2019f, Jin_2023, Kumar_2019f, Losanoff_1996, Misra_2013, Qureshi_2016, Riva_2018j, Sakellaridis_2008f, Tammana_2012j, Trgo_2012f, Wnęk_2015f, Yildiz_2016e, fjbuilsRepeatedBehaviorDeliberate2024}, 18 cases (25\%) were between 18 and 25 years of age \cite{Akay_2015f, Ali_2017, Atayan_2016, Bhattacharjee_2008, Csaky_1998e, Kar_2015, Kariholu_2008, Kobiela_2015, Losanoff_1996, Losanoff_1997e, Mesfin_2022a, Peixoto_2017f, Sobnach_2011f, Tupesis_2004f, Yasin_2009}, 13 cases (18\%) were under 18 years of age \cite{AlShaaibi_2021b, Ali_2020f, Cauchi_2002, DivsalarP._2023a, Goldman_1998f, Liu_2005, Naji_2012f, Ohno_2005, Tanrikulu_2015e, Tay_2004, Wildhaber_2005}, 11 cases (15\%) were between 41 and 60 years of age \cite{Al-Faham_2020k, Bhumi_2024f, CamachoDorado_2018, Emamhadi_2018, Hardy_2023g, Jehangir_2019h, Kumar_2001, Sultan_2024f, Thapa_2019f, Wadhwa_2015e, teWildt_2010}, 3 cases (4\%) were over 60 years of age \cite{Beecroft_1998, Kerestes_2019, Li_2013}, 2 cases (3\%) had no age documented \cite{Berry_2021e}. \paragraph*{Population} 36 cases (50\%) had a psychiatric history \cite{AlShaaibi_2021b, Alao_2006i, Ali_2020f, Apikotoa_2022f, Ataya_2013, Atayan_2016, Beecroft_1998, CamachoDorado_2018, Chang_2017f, DelgadoSalazar_2020c, DivsalarP._2023a, Farhadi_2024h, Fry_2010, Guinan_2019f, Hardy_2023g, Jehangir_2019h, Jin_2023, Kar_2015, Kerestes_2019, Kobiela_2015, Kumar_2001, Kumar_2019f, Liu_2005, Mesfin_2022a, Misra_2013, Ohno_2005, Peixoto_2017f, Sakellaridis_2008f, Sultan_2024f, Tammana_2012j, Tanrikulu_2015e, Yildiz_2016e, fjbuilsRepeatedBehaviorDeliberate2024, teWildt_2010}, 19 cases (26\%) had ingested previously \cite{Alao_2006i, Apikotoa_2022f, Berry_2021e, Bhattacharjee_2008, Csaky_1998e, DivsalarP._2023a, Emamhadi_2018, Guinan_2019f, Jehangir_2019h, Jin_2023, Liu_2005, Sakellaridis_2008f, Tanrikulu_2015e, Thapa_2019f, Yildiz_2016e, fjbuilsRepeatedBehaviorDeliberate2024, teWildt_2010}, 12 cases (17\%) were detained persons \cite{Alao_2006i, Ali_2022g, Apikotoa_2022f, Losanoff_1996, Losanoff_1997e, Qureshi_2016, Tammana_2012j, Trgo_2012f}, 7 cases (10\%) were severely disabled \cite{Atayan_2016, Kerestes_2019, Liu_2005, Ohno_2005, Peixoto_2017f, Yildiz_2016e, teWildt_2010}, 4 cases (6\%) were psychiatric inpatients \cite{DivsalarP._2023a, fjbuilsRepeatedBehaviorDeliberate2024, teWildt_2010}, 3 cases (4\%) were under the influence of alcohol \cite{Benoist_2019e, Csaky_1998e, Thapa_2019f}, 2 cases (3\%) were displaced people \cite{Akay_2015f, Gardner_2017h}. \paragraph*{Motivation} 34 cases (47\%) had a psychiatric motivation \cite{Al-Faham_2020k, Alao_2006i, Ali_2020f, Apikotoa_2022f, Ataya_2013, Atayan_2016, Bhasin_2014, Bhattacharjee_2008, DelgadoSalazar_2020c, DivsalarP._2023a, Emamhadi_2018, Farhadi_2024h, Guinan_2019f, Hardy_2023g, Jehangir_2019h, Jin_2023, Kar_2015, Kariholu_2008, Kerestes_2019, Kobiela_2015, Kumar_2001, Kumar_2019f, Li_2013, Liu_2005, Misra_2013, Ohno_2005, Sakellaridis_2008f, Sultan_2024f, Tammana_2012j, Tanrikulu_2015e, Yasin_2009, teWildt_2010}, 21 cases (29\%) were motivated by self-harm intention \cite{Al-Faham_2020k, AlShaaibi_2021b, Alao_2006i, Ali_2017, CamachoDorado_2018, Chang_2017f, Cox_2007, Csaky_1998e, Fry_2010, Li_2013, Losanoff_1996, Losanoff_1997e, Mesfin_2022a, Sakellaridis_2008f, Tammana_2012j, Tanrikulu_2015e, fjbuilsRepeatedBehaviorDeliberate2024}, 17 cases (24\%) had a psychosocial motivation \cite{Akay_2015f, Benoist_2019e, Bhattacharjee_2008, Cauchi_2002, Goldman_1998f, Hardy_2023g, Kobiela_2015, Li_2013, Naji_2012f, Qureshi_2016, Riva_2018j, Sobnach_2011f, Tay_2004, Thapa_2019f, Tupesis_2004f, Wildhaber_2005, Wnęk_2015f}, 9 cases (12\%) were motivated by protest \cite{Bhumi_2024f, Gardner_2017h, Losanoff_1996, Losanoff_1997e, Tupesis_2004f}, 9 cases (12\%) had another documented motivation \cite{Ali_2020f, Ali_2022g, Emamhadi_2018, Guinan_2019f, Peixoto_2017f, Sakellaridis_2008f, Trgo_2012f, Wadhwa_2015e, Yildiz_2016e}. \paragraph*{Object Characteristics} 51 cases (71\%) ingested a large diameter object (\textgreater{}2.5cm) \cite{Akay_2015f, Al-Faham_2020k, AlShaaibi_2021b, Alao_2006i, Ali_2017, Ali_2022g, Apikotoa_2022f, Atayan_2016, Berry_2021e, Bhasin_2014, CamachoDorado_2018, Cauchi_2002, Chang_2017f, Cox_2007, Csaky_1998e, DivsalarP._2023a, Emamhadi_2018, Gardner_2017h, Guinan_2019f, Jehangir_2019h, Jin_2023, Kariholu_2008, Kerestes_2019, Kobiela_2015, Kumar_2001, Kumar_2019f, Losanoff_1996, Losanoff_1997e, Mesfin_2022a, Misra_2013, Naji_2012f, Ohno_2005, Peixoto_2017f, Qureshi_2016, Riva_2018j, Sakellaridis_2008f, Sultan_2024f, Tanrikulu_2015e, Thapa_2019f, Trgo_2012f, Wnęk_2015f, Yildiz_2016e, fjbuilsRepeatedBehaviorDeliberate2024, teWildt_2010}, 44 cases (61\%) ingested multiple objects \cite{Ali_2020f, Apikotoa_2022f, Ataya_2013, Atayan_2016, Beecroft_1998, Bhattacharjee_2008, Bhumi_2024f, CamachoDorado_2018, Cauchi_2002, Emamhadi_2018, Farhadi_2024h, Fry_2010, Goldman_1998f, Guinan_2019f, Hardy_2023g, Jehangir_2019h, Jin_2023, Kar_2015, Kariholu_2008, Kobiela_2015, Kumar_2001, Kumar_2019f, Li_2013, Liu_2005, Losanoff_1996, Mesfin_2022a, Misra_2013, Naji_2012f, Ohno_2005, Sobnach_2011f, Sultan_2024f, Tammana_2012j, Tanrikulu_2015e, Tay_2004, Thapa_2019f, Wadhwa_2015e, Wildhaber_2005, Yasin_2009, fjbuilsRepeatedBehaviorDeliberate2024, teWildt_2010}, 34 cases (47\%) ingested a sharp object \cite{AlShaaibi_2021b, Alao_2006i, Apikotoa_2022f, Ataya_2013, Benoist_2019e, Bhasin_2014, Bhattacharjee_2008, CamachoDorado_2018, Csaky_1998e, DelgadoSalazar_2020c, DivsalarP._2023a, Emamhadi_2018, Farhadi_2024h, Fry_2010, Guinan_2019f, Hardy_2023g, Jehangir_2019h, Jin_2023, Kariholu_2008, Kobiela_2015, Kumar_2019f, Losanoff_1996, Losanoff_1997e, Mesfin_2022a, Misra_2013, Sobnach_2011f, Yasin_2009, teWildt_2010}, 32 cases (44\%) ingested a long object (\textgreater{}5cm) \cite{Al-Faham_2020k, AlShaaibi_2021b, Ali_2017, Ali_2022g, Atayan_2016, Bhasin_2014, CamachoDorado_2018, Chang_2017f, Cox_2007, Csaky_1998e, DivsalarP._2023a, Emamhadi_2018, Fry_2010, Gardner_2017h, Jin_2023, Kariholu_2008, Kerestes_2019, Kobiela_2015, Kumar_2019f, Mesfin_2022a, Misra_2013, Ohno_2005, Qureshi_2016, Sakellaridis_2008f, Sultan_2024f, Thapa_2019f, Trgo_2012f, Yasin_2009, Yildiz_2016e, teWildt_2010}, 9 cases (12\%) ingested a magnet \cite{Ali_2020f, Bhumi_2024f, Cauchi_2002, Liu_2005, Naji_2012f, Ohno_2005, Tanrikulu_2015e, Tay_2004, Wildhaber_2005}, 2 cases (3\%) ingested a button battery \cite{Berry_2021e, Bhumi_2024f}. \paragraph*{Outcomes} 48 cases (67\%) experienced a complication \cite{Ali_2017, Ali_2020f, Apikotoa_2022f, Atayan_2016, Beecroft_1998, Benoist_2019e, Berry_2021e, Bhasin_2014, Bhumi_2024f, CamachoDorado_2018, Cauchi_2002, Cox_2007, Csaky_1998e, DelgadoSalazar_2020c, DivsalarP._2023a, Emamhadi_2018, Farhadi_2024h, Fry_2010, Gardner_2017h, Goldman_1998f, Jin_2023, Kariholu_2008, Kerestes_2019, Kobiela_2015, Kumar_2001, Kumar_2019f, Liu_2005, Losanoff_1996, Mesfin_2022a, Misra_2013, Naji_2012f, Ohno_2005, Sakellaridis_2008f, Sobnach_2011f, Sultan_2024f, Tanrikulu_2015e, Tay_2004, Thapa_2019f, Trgo_2012f, Tupesis_2004f, Wildhaber_2005, Wnęk_2015f, Yasin_2009, Yildiz_2016e}, 44 cases (61\%) underwent surgery \cite{Al-Faham_2020k, AlShaaibi_2021b, Alao_2006i, Ali_2017, Ali_2020f, Atayan_2016, Beecroft_1998, Bhasin_2014, CamachoDorado_2018, Cauchi_2002, Chang_2017f, Cox_2007, Csaky_1998e, DelgadoSalazar_2020c, DivsalarP._2023a, Farhadi_2024h, Fry_2010, Gardner_2017h, Jin_2023, Kariholu_2008, Kerestes_2019, Kobiela_2015, Kumar_2019f, Liu_2005, Losanoff_1996, Losanoff_1997e, Mesfin_2022a, Misra_2013, Naji_2012f, Sobnach_2011f, Tanrikulu_2015e, Tay_2004, Thapa_2019f, Tupesis_2004f, Wildhaber_2005, Wnęk_2015f, Yasin_2009, Yildiz_2016e, fjbuilsRepeatedBehaviorDeliberate2024}, 31 cases (43\%) underwent endoscopy \cite{Akay_2015f, Ali_2022g, Apikotoa_2022f, Atayan_2016, Benoist_2019e, Berry_2021e, Bhasin_2014, Bhumi_2024f, CamachoDorado_2018, Chang_2017f, DelgadoSalazar_2020c, Gardner_2017h, Guinan_2019f, Hardy_2023g, Jehangir_2019h, Kariholu_2008, Li_2013, Liu_2005, Ohno_2005, Peixoto_2017f, Qureshi_2016, Riva_2018j, Sakellaridis_2008f, Sultan_2024f, Tammana_2012j, Tanrikulu_2015e, Trgo_2012f, Wadhwa_2015e, Wnęk_2015f, teWildt_2010}, 7 cases (10\%) were managed conservatively \cite{Ataya_2013, Bhattacharjee_2008, DivsalarP._2023a, Emamhadi_2018, Goldman_1998f, Kar_2015, Kumar_2001}, 2 cases (3\%) died \cite{Emamhadi_2018, Kumar_2001}. All 90 were male gender. 90 cases (100\%) were detained at the time of ingestion \cite{Elghali_2016, Karp_1991b, Lee_2007}, 88 cases (98\%) were intentional ingestions \cite{Elghali_2016, Karp_1991b, Lee_2007}, 30 cases (33\%) had a psychiatric history documented \cite{Elghali_2016, Karp_1991b, Lee_2007}, 2 cases (2\%) had a history of prior ingestion \cite{Elghali_2016}. No cases were reported for were psychiatric inpatients, were displaced people, were under the influence of alcohol at the time of ingestion, and had a severe disability history.
\paragraph*{Motivation}  70 cases (78\%) reported protest motivation \cite{Elghali_2016, Karp_1991b, Lee_2007}, 12 cases (13\%) reported psychiatric motivation \cite{Karp_1991b}, 6 cases (7\%) reported self-harm motivation \cite{Elghali_2016, Karp_1991b}. No cases were reported for psychosocial motivation and other motivation.
\paragraph*{Object Characteristics}  68 cases (76\%) involved sharp object ingestion \cite{Elghali_2016, Karp_1991b, Lee_2007}, 32 cases (36\%) involved long (\textgreater 5cm) object ingestion \cite{Lee_2007}, 25 cases (28\%) involved ingestion of multiple objects \cite{Elghali_2016, Lee_2007}. No cases were reported for button battery ingestion, magnet ingestion, and involved large diameter (\textgreater 2.5cm) object ingestion.
\paragraph*{Outcomes}  47 cases (52\%) underwent endoscopic intervention \cite{Elghali_2016, Lee_2007}, 29 cases (32\%) were managed conservatively \cite{Elghali_2016, Karp_1991b}, 15 cases (17\%) underwent surgical intervention \cite{Elghali_2016, Karp_1991b, Lee_2007}, 6 cases (7\%) reported complications \cite{Lee_2007}, 1 case (1\%) died \cite{Elghali_2016}.
\paragraph*{Geographical Location}Cases were recorded in 33 countries: 13 cases from USA \cite{Alao_2006i, Ataya_2013, Bhumi_2024f, Fry_2010, Guinan_2019f, Hardy_2023g, Jehangir_2019h, Kerestes_2019, Kumar_2001, Liu_2005, Tammana_2012j, Tay_2004, Tupesis_2004f}; 7 cases from India \cite{Bhasin_2014, Bhattacharjee_2008, Kar_2015, Kariholu_2008, Kumar_2019f, Misra_2013, Wadhwa_2015e} and UK \cite{Beecroft_1998, Berry_2021e, Cauchi_2002, Cox_2007, Gardner_2017h, Qureshi_2016}; 6 cases from Bulgaria \cite{Losanoff_1996, Losanoff_1997e}; 5 cases from Iran \cite{DivsalarP._2023a, Emamhadi_2018, Farhadi_2024h}; 4 cases from Turkey \cite{Akay_2015f, Atayan_2016, Tanrikulu_2015e, Yildiz_2016e}; 2 cases from China \cite{Jin_2023, Li_2013}, Poland \cite{Kobiela_2015, Wnęk_2015f}, and Spain \cite{CamachoDorado_2018, fjbuilsRepeatedBehaviorDeliberate2024}; 1 case from Australia \cite{Apikotoa_2022f}, Bahrain \cite{Ali_2020f}, Croatia \cite{Trgo_2012f}, Ecuador \cite{DelgadoSalazar_2020c}, Egypt \cite{Ali_2022g}, Ethiopia \cite{Mesfin_2022a}, Germany \cite{teWildt_2010}, Greece \cite{Sakellaridis_2008f}, Hungary \cite{Csaky_1998e}, Iraq \cite{Al-Faham_2020k}, Israel \cite{Goldman_1998f}, Italy \cite{Riva_2018j}, Japan \cite{Ohno_2005}, Nepal \cite{Thapa_2019f}, Netherlands \cite{Benoist_2019e}, Oman \cite{AlShaaibi_2021b}, Pakistan \cite{Yasin_2009}, Portugal \cite{Peixoto_2017f}, Qatar \cite{Ali_2017}, Saudi Arabia \cite{Sultan_2024f}, South Africa \cite{Sobnach_2011f}, Sweden \cite{Naji_2012f}, Switzerland \cite{Wildhaber_2005}, and Taiwan \cite{Chang_2017f}. \paragraph*{Gender} 43 cases (60\%) were male \cite{Akay_2015f, Al-Faham_2020k, Alao_2006i, Ali_2017, Ali_2022g, Apikotoa_2022f, Atayan_2016, Benoist_2019e, Berry_2021e, Bhumi_2024f, CamachoDorado_2018, Csaky_1998e, Emamhadi_2018, Farhadi_2024h, Fry_2010, Gardner_2017h, Guinan_2019f, Jehangir_2019h, Jin_2023, Kobiela_2015, Kumar_2001, Kumar_2019f, Liu_2005, Losanoff_1996, Losanoff_1997e, Mesfin_2022a, Misra_2013, Qureshi_2016, Riva_2018j, Sobnach_2011f, Tammana_2012j, Tanrikulu_2015e, Tay_2004, Thapa_2019f, Trgo_2012f, Wadhwa_2015e, Yasin_2009, teWildt_2010}, 28 cases (39\%) were female \cite{AlShaaibi_2021b, Ali_2020f, Ataya_2013, Beecroft_1998, Bhasin_2014, Bhattacharjee_2008, Cauchi_2002, Chang_2017f, Cox_2007, DelgadoSalazar_2020c, DivsalarP._2023a, Goldman_1998f, Hardy_2023g, Kar_2015, Kariholu_2008, Kerestes_2019, Li_2013, Naji_2012f, Ohno_2005, Peixoto_2017f, Sakellaridis_2008f, Sultan_2024f, Tupesis_2004f, Wildhaber_2005, Wnęk_2015f, Yildiz_2016e}, 1 case (1\%) had no gender recorded \cite{fjbuilsRepeatedBehaviorDeliberate2024}. \paragraph*{Age Group} 25 cases (35\%) were between 26 and 40 years of age \cite{Alao_2006i, Ali_2022g, Apikotoa_2022f, Ataya_2013, Benoist_2019e, Bhasin_2014, Chang_2017f, Cox_2007, DelgadoSalazar_2020c, Farhadi_2024h, Fry_2010, Gardner_2017h, Guinan_2019f, Jin_2023, Kumar_2019f, Losanoff_1996, Misra_2013, Qureshi_2016, Riva_2018j, Sakellaridis_2008f, Tammana_2012j, Trgo_2012f, Wnęk_2015f, Yildiz_2016e, fjbuilsRepeatedBehaviorDeliberate2024}, 18 cases (25\%) were between 18 and 25 years of age \cite{Akay_2015f, Ali_2017, Atayan_2016, Bhattacharjee_2008, Csaky_1998e, Kar_2015, Kariholu_2008, Kobiela_2015, Losanoff_1996, Losanoff_1997e, Mesfin_2022a, Peixoto_2017f, Sobnach_2011f, Tupesis_2004f, Yasin_2009}, 13 cases (18\%) were under 18 years of age \cite{AlShaaibi_2021b, Ali_2020f, Cauchi_2002, DivsalarP._2023a, Goldman_1998f, Liu_2005, Naji_2012f, Ohno_2005, Tanrikulu_2015e, Tay_2004, Wildhaber_2005}, 11 cases (15\%) were between 41 and 60 years of age \cite{Al-Faham_2020k, Bhumi_2024f, CamachoDorado_2018, Emamhadi_2018, Hardy_2023g, Jehangir_2019h, Kumar_2001, Sultan_2024f, Thapa_2019f, Wadhwa_2015e, teWildt_2010}, 3 cases (4\%) were over 60 years of age \cite{Beecroft_1998, Kerestes_2019, Li_2013}, 2 cases (3\%) had no age documented \cite{Berry_2021e}. \paragraph*{Population} 36 cases (50\%) had a psychiatric history \cite{AlShaaibi_2021b, Alao_2006i, Ali_2020f, Apikotoa_2022f, Ataya_2013, Atayan_2016, Beecroft_1998, CamachoDorado_2018, Chang_2017f, DelgadoSalazar_2020c, DivsalarP._2023a, Farhadi_2024h, Fry_2010, Guinan_2019f, Hardy_2023g, Jehangir_2019h, Jin_2023, Kar_2015, Kerestes_2019, Kobiela_2015, Kumar_2001, Kumar_2019f, Liu_2005, Mesfin_2022a, Misra_2013, Ohno_2005, Peixoto_2017f, Sakellaridis_2008f, Sultan_2024f, Tammana_2012j, Tanrikulu_2015e, Yildiz_2016e, fjbuilsRepeatedBehaviorDeliberate2024, teWildt_2010}, 19 cases (26\%) had ingested previously \cite{Alao_2006i, Apikotoa_2022f, Berry_2021e, Bhattacharjee_2008, Csaky_1998e, DivsalarP._2023a, Emamhadi_2018, Guinan_2019f, Jehangir_2019h, Jin_2023, Liu_2005, Sakellaridis_2008f, Tanrikulu_2015e, Thapa_2019f, Yildiz_2016e, fjbuilsRepeatedBehaviorDeliberate2024, teWildt_2010}, 12 cases (17\%) were detained persons \cite{Alao_2006i, Ali_2022g, Apikotoa_2022f, Losanoff_1996, Losanoff_1997e, Qureshi_2016, Tammana_2012j, Trgo_2012f}, 7 cases (10\%) were severely disabled \cite{Atayan_2016, Kerestes_2019, Liu_2005, Ohno_2005, Peixoto_2017f, Yildiz_2016e, teWildt_2010}, 4 cases (6\%) were psychiatric inpatients \cite{DivsalarP._2023a, fjbuilsRepeatedBehaviorDeliberate2024, teWildt_2010}, 3 cases (4\%) were under the influence of alcohol \cite{Benoist_2019e, Csaky_1998e, Thapa_2019f}, 2 cases (3\%) were displaced people \cite{Akay_2015f, Gardner_2017h}. \paragraph*{Motivation} 34 cases (47\%) had a psychiatric motivation \cite{Al-Faham_2020k, Alao_2006i, Ali_2020f, Apikotoa_2022f, Ataya_2013, Atayan_2016, Bhasin_2014, Bhattacharjee_2008, DelgadoSalazar_2020c, DivsalarP._2023a, Emamhadi_2018, Farhadi_2024h, Guinan_2019f, Hardy_2023g, Jehangir_2019h, Jin_2023, Kar_2015, Kariholu_2008, Kerestes_2019, Kobiela_2015, Kumar_2001, Kumar_2019f, Li_2013, Liu_2005, Misra_2013, Ohno_2005, Sakellaridis_2008f, Sultan_2024f, Tammana_2012j, Tanrikulu_2015e, Yasin_2009, teWildt_2010}, 21 cases (29\%) were motivated by self-harm intention \cite{Al-Faham_2020k, AlShaaibi_2021b, Alao_2006i, Ali_2017, CamachoDorado_2018, Chang_2017f, Cox_2007, Csaky_1998e, Fry_2010, Li_2013, Losanoff_1996, Losanoff_1997e, Mesfin_2022a, Sakellaridis_2008f, Tammana_2012j, Tanrikulu_2015e, fjbuilsRepeatedBehaviorDeliberate2024}, 17 cases (24\%) had a psychosocial motivation \cite{Akay_2015f, Benoist_2019e, Bhattacharjee_2008, Cauchi_2002, Goldman_1998f, Hardy_2023g, Kobiela_2015, Li_2013, Naji_2012f, Qureshi_2016, Riva_2018j, Sobnach_2011f, Tay_2004, Thapa_2019f, Tupesis_2004f, Wildhaber_2005, Wnęk_2015f}, 9 cases (12\%) were motivated by protest \cite{Bhumi_2024f, Gardner_2017h, Losanoff_1996, Losanoff_1997e, Tupesis_2004f}, 9 cases (12\%) had another documented motivation \cite{Ali_2020f, Ali_2022g, Emamhadi_2018, Guinan_2019f, Peixoto_2017f, Sakellaridis_2008f, Trgo_2012f, Wadhwa_2015e, Yildiz_2016e}. \paragraph*{Object Characteristics} 51 cases (71\%) ingested a large diameter object (\textgreater{}2.5cm) \cite{Akay_2015f, Al-Faham_2020k, AlShaaibi_2021b, Alao_2006i, Ali_2017, Ali_2022g, Apikotoa_2022f, Atayan_2016, Berry_2021e, Bhasin_2014, CamachoDorado_2018, Cauchi_2002, Chang_2017f, Cox_2007, Csaky_1998e, DivsalarP._2023a, Emamhadi_2018, Gardner_2017h, Guinan_2019f, Jehangir_2019h, Jin_2023, Kariholu_2008, Kerestes_2019, Kobiela_2015, Kumar_2001, Kumar_2019f, Losanoff_1996, Losanoff_1997e, Mesfin_2022a, Misra_2013, Naji_2012f, Ohno_2005, Peixoto_2017f, Qureshi_2016, Riva_2018j, Sakellaridis_2008f, Sultan_2024f, Tanrikulu_2015e, Thapa_2019f, Trgo_2012f, Wnęk_2015f, Yildiz_2016e, fjbuilsRepeatedBehaviorDeliberate2024, teWildt_2010}, 44 cases (61\%) ingested multiple objects \cite{Ali_2020f, Apikotoa_2022f, Ataya_2013, Atayan_2016, Beecroft_1998, Bhattacharjee_2008, Bhumi_2024f, CamachoDorado_2018, Cauchi_2002, Emamhadi_2018, Farhadi_2024h, Fry_2010, Goldman_1998f, Guinan_2019f, Hardy_2023g, Jehangir_2019h, Jin_2023, Kar_2015, Kariholu_2008, Kobiela_2015, Kumar_2001, Kumar_2019f, Li_2013, Liu_2005, Losanoff_1996, Mesfin_2022a, Misra_2013, Naji_2012f, Ohno_2005, Sobnach_2011f, Sultan_2024f, Tammana_2012j, Tanrikulu_2015e, Tay_2004, Thapa_2019f, Wadhwa_2015e, Wildhaber_2005, Yasin_2009, fjbuilsRepeatedBehaviorDeliberate2024, teWildt_2010}, 34 cases (47\%) ingested a sharp object \cite{AlShaaibi_2021b, Alao_2006i, Apikotoa_2022f, Ataya_2013, Benoist_2019e, Bhasin_2014, Bhattacharjee_2008, CamachoDorado_2018, Csaky_1998e, DelgadoSalazar_2020c, DivsalarP._2023a, Emamhadi_2018, Farhadi_2024h, Fry_2010, Guinan_2019f, Hardy_2023g, Jehangir_2019h, Jin_2023, Kariholu_2008, Kobiela_2015, Kumar_2019f, Losanoff_1996, Losanoff_1997e, Mesfin_2022a, Misra_2013, Sobnach_2011f, Yasin_2009, teWildt_2010}, 32 cases (44\%) ingested a long object (\textgreater{}5cm) \cite{Al-Faham_2020k, AlShaaibi_2021b, Ali_2017, Ali_2022g, Atayan_2016, Bhasin_2014, CamachoDorado_2018, Chang_2017f, Cox_2007, Csaky_1998e, DivsalarP._2023a, Emamhadi_2018, Fry_2010, Gardner_2017h, Jin_2023, Kariholu_2008, Kerestes_2019, Kobiela_2015, Kumar_2019f, Mesfin_2022a, Misra_2013, Ohno_2005, Qureshi_2016, Sakellaridis_2008f, Sultan_2024f, Thapa_2019f, Trgo_2012f, Yasin_2009, Yildiz_2016e, teWildt_2010}, 9 cases (12\%) ingested a magnet \cite{Ali_2020f, Bhumi_2024f, Cauchi_2002, Liu_2005, Naji_2012f, Ohno_2005, Tanrikulu_2015e, Tay_2004, Wildhaber_2005}, 2 cases (3\%) ingested a button battery \cite{Berry_2021e, Bhumi_2024f}. \paragraph*{Outcomes} 48 cases (67\%) experienced a complication \cite{Ali_2017, Ali_2020f, Apikotoa_2022f, Atayan_2016, Beecroft_1998, Benoist_2019e, Berry_2021e, Bhasin_2014, Bhumi_2024f, CamachoDorado_2018, Cauchi_2002, Cox_2007, Csaky_1998e, DelgadoSalazar_2020c, DivsalarP._2023a, Emamhadi_2018, Farhadi_2024h, Fry_2010, Gardner_2017h, Goldman_1998f, Jin_2023, Kariholu_2008, Kerestes_2019, Kobiela_2015, Kumar_2001, Kumar_2019f, Liu_2005, Losanoff_1996, Mesfin_2022a, Misra_2013, Naji_2012f, Ohno_2005, Sakellaridis_2008f, Sobnach_2011f, Sultan_2024f, Tanrikulu_2015e, Tay_2004, Thapa_2019f, Trgo_2012f, Tupesis_2004f, Wildhaber_2005, Wnęk_2015f, Yasin_2009, Yildiz_2016e}, 44 cases (61\%) underwent surgery \cite{Al-Faham_2020k, AlShaaibi_2021b, Alao_2006i, Ali_2017, Ali_2020f, Atayan_2016, Beecroft_1998, Bhasin_2014, CamachoDorado_2018, Cauchi_2002, Chang_2017f, Cox_2007, Csaky_1998e, DelgadoSalazar_2020c, DivsalarP._2023a, Farhadi_2024h, Fry_2010, Gardner_2017h, Jin_2023, Kariholu_2008, Kerestes_2019, Kobiela_2015, Kumar_2019f, Liu_2005, Losanoff_1996, Losanoff_1997e, Mesfin_2022a, Misra_2013, Naji_2012f, Sobnach_2011f, Tanrikulu_2015e, Tay_2004, Thapa_2019f, Tupesis_2004f, Wildhaber_2005, Wnęk_2015f, Yasin_2009, Yildiz_2016e, fjbuilsRepeatedBehaviorDeliberate2024}, 31 cases (43\%) underwent endoscopy \cite{Akay_2015f, Ali_2022g, Apikotoa_2022f, Atayan_2016, Benoist_2019e, Berry_2021e, Bhasin_2014, Bhumi_2024f, CamachoDorado_2018, Chang_2017f, DelgadoSalazar_2020c, Gardner_2017h, Guinan_2019f, Hardy_2023g, Jehangir_2019h, Kariholu_2008, Li_2013, Liu_2005, Ohno_2005, Peixoto_2017f, Qureshi_2016, Riva_2018j, Sakellaridis_2008f, Sultan_2024f, Tammana_2012j, Tanrikulu_2015e, Trgo_2012f, Wadhwa_2015e, Wnęk_2015f, teWildt_2010}, 7 cases (10\%) were managed conservatively \cite{Ataya_2013, Bhattacharjee_2008, DivsalarP._2023a, Emamhadi_2018, Goldman_1998f, Kar_2015, Kumar_2001}, 2 cases (3\%) died \cite{Emamhadi_2018, Kumar_2001}. All 90 were male gender. 90 cases (100\%) were detained at the time of ingestion \cite{Elghali_2016, Karp_1991b, Lee_2007}, 88 cases (98\%) were intentional ingestions \cite{Elghali_2016, Karp_1991b, Lee_2007}, 30 cases (33\%) had a psychiatric history documented \cite{Elghali_2016, Karp_1991b, Lee_2007}, 2 cases (2\%) had a history of prior ingestion \cite{Elghali_2016}. No cases were reported for were psychiatric inpatients, were displaced people, were under the influence of alcohol at the time of ingestion, and had a severe disability history.
\paragraph*{Motivation}  70 cases (78\%) reported protest motivation \cite{Elghali_2016, Karp_1991b, Lee_2007}, 12 cases (13\%) reported psychiatric motivation \cite{Karp_1991b}, 6 cases (7\%) reported self-harm motivation \cite{Elghali_2016, Karp_1991b}. No cases were reported for psychosocial motivation and other motivation.
\paragraph*{Object Characteristics}  68 cases (76\%) involved sharp object ingestion \cite{Elghali_2016, Karp_1991b, Lee_2007}, 32 cases (36\%) involved long (\textgreater 5cm) object ingestion \cite{Lee_2007}, 25 cases (28\%) involved ingestion of multiple objects \cite{Elghali_2016, Lee_2007}. No cases were reported for button battery ingestion, magnet ingestion, and involved large diameter (\textgreater 2.5cm) object ingestion.
\paragraph*{Outcomes}  47 cases (52\%) underwent endoscopic intervention \cite{Elghali_2016, Lee_2007}, 29 cases (32\%) were managed conservatively \cite{Elghali_2016, Karp_1991b}, 15 cases (17\%) underwent surgical intervention \cite{Elghali_2016, Karp_1991b, Lee_2007}, 6 cases (7\%) reported complications \cite{Lee_2007}, 1 case (1\%) died \cite{Elghali_2016}.
\paragraph*{Geographical Location}Cases were recorded in 33 countries: 13 cases from USA \cite{Alao_2006i, Ataya_2013, Bhumi_2024f, Fry_2010, Guinan_2019f, Hardy_2023g, Jehangir_2019h, Kerestes_2019, Kumar_2001, Liu_2005, Tammana_2012j, Tay_2004, Tupesis_2004f}; 7 cases from India \cite{Bhasin_2014, Bhattacharjee_2008, Kar_2015, Kariholu_2008, Kumar_2019f, Misra_2013, Wadhwa_2015e} and UK \cite{Beecroft_1998, Berry_2021e, Cauchi_2002, Cox_2007, Gardner_2017h, Qureshi_2016}; 6 cases from Bulgaria \cite{Losanoff_1996, Losanoff_1997e}; 5 cases from Iran \cite{DivsalarP._2023a, Emamhadi_2018, Farhadi_2024h}; 4 cases from Turkey \cite{Akay_2015f, Atayan_2016, Tanrikulu_2015e, Yildiz_2016e}; 2 cases from China \cite{Jin_2023, Li_2013}, Poland \cite{Kobiela_2015, Wnęk_2015f}, and Spain \cite{CamachoDorado_2018, fjbuilsRepeatedBehaviorDeliberate2024}; 1 case from Australia \cite{Apikotoa_2022f}, Bahrain \cite{Ali_2020f}, Croatia \cite{Trgo_2012f}, Ecuador \cite{DelgadoSalazar_2020c}, Egypt \cite{Ali_2022g}, Ethiopia \cite{Mesfin_2022a}, Germany \cite{teWildt_2010}, Greece \cite{Sakellaridis_2008f}, Hungary \cite{Csaky_1998e}, Iraq \cite{Al-Faham_2020k}, Israel \cite{Goldman_1998f}, Italy \cite{Riva_2018j}, Japan \cite{Ohno_2005}, Nepal \cite{Thapa_2019f}, Netherlands \cite{Benoist_2019e}, Oman \cite{AlShaaibi_2021b}, Pakistan \cite{Yasin_2009}, Portugal \cite{Peixoto_2017f}, Qatar \cite{Ali_2017}, Saudi Arabia \cite{Sultan_2024f}, South Africa \cite{Sobnach_2011f}, Sweden \cite{Naji_2012f}, Switzerland \cite{Wildhaber_2005}, and Taiwan \cite{Chang_2017f}. \paragraph*{Gender} 43 cases (60\%) were male \cite{Akay_2015f, Al-Faham_2020k, Alao_2006i, Ali_2017, Ali_2022g, Apikotoa_2022f, Atayan_2016, Benoist_2019e, Berry_2021e, Bhumi_2024f, CamachoDorado_2018, Csaky_1998e, Emamhadi_2018, Farhadi_2024h, Fry_2010, Gardner_2017h, Guinan_2019f, Jehangir_2019h, Jin_2023, Kobiela_2015, Kumar_2001, Kumar_2019f, Liu_2005, Losanoff_1996, Losanoff_1997e, Mesfin_2022a, Misra_2013, Qureshi_2016, Riva_2018j, Sobnach_2011f, Tammana_2012j, Tanrikulu_2015e, Tay_2004, Thapa_2019f, Trgo_2012f, Wadhwa_2015e, Yasin_2009, teWildt_2010}, 28 cases (39\%) were female \cite{AlShaaibi_2021b, Ali_2020f, Ataya_2013, Beecroft_1998, Bhasin_2014, Bhattacharjee_2008, Cauchi_2002, Chang_2017f, Cox_2007, DelgadoSalazar_2020c, DivsalarP._2023a, Goldman_1998f, Hardy_2023g, Kar_2015, Kariholu_2008, Kerestes_2019, Li_2013, Naji_2012f, Ohno_2005, Peixoto_2017f, Sakellaridis_2008f, Sultan_2024f, Tupesis_2004f, Wildhaber_2005, Wnęk_2015f, Yildiz_2016e}, 1 case (1\%) had no gender recorded \cite{fjbuilsRepeatedBehaviorDeliberate2024}. \paragraph*{Age Group} 25 cases (35\%) were between 26 and 40 years of age \cite{Alao_2006i, Ali_2022g, Apikotoa_2022f, Ataya_2013, Benoist_2019e, Bhasin_2014, Chang_2017f, Cox_2007, DelgadoSalazar_2020c, Farhadi_2024h, Fry_2010, Gardner_2017h, Guinan_2019f, Jin_2023, Kumar_2019f, Losanoff_1996, Misra_2013, Qureshi_2016, Riva_2018j, Sakellaridis_2008f, Tammana_2012j, Trgo_2012f, Wnęk_2015f, Yildiz_2016e, fjbuilsRepeatedBehaviorDeliberate2024}, 18 cases (25\%) were between 18 and 25 years of age \cite{Akay_2015f, Ali_2017, Atayan_2016, Bhattacharjee_2008, Csaky_1998e, Kar_2015, Kariholu_2008, Kobiela_2015, Losanoff_1996, Losanoff_1997e, Mesfin_2022a, Peixoto_2017f, Sobnach_2011f, Tupesis_2004f, Yasin_2009}, 13 cases (18\%) were under 18 years of age \cite{AlShaaibi_2021b, Ali_2020f, Cauchi_2002, DivsalarP._2023a, Goldman_1998f, Liu_2005, Naji_2012f, Ohno_2005, Tanrikulu_2015e, Tay_2004, Wildhaber_2005}, 11 cases (15\%) were between 41 and 60 years of age \cite{Al-Faham_2020k, Bhumi_2024f, CamachoDorado_2018, Emamhadi_2018, Hardy_2023g, Jehangir_2019h, Kumar_2001, Sultan_2024f, Thapa_2019f, Wadhwa_2015e, teWildt_2010}, 3 cases (4\%) were over 60 years of age \cite{Beecroft_1998, Kerestes_2019, Li_2013}, 2 cases (3\%) had no age documented \cite{Berry_2021e}. \paragraph*{Population} 36 cases (50\%) had a psychiatric history \cite{AlShaaibi_2021b, Alao_2006i, Ali_2020f, Apikotoa_2022f, Ataya_2013, Atayan_2016, Beecroft_1998, CamachoDorado_2018, Chang_2017f, DelgadoSalazar_2020c, DivsalarP._2023a, Farhadi_2024h, Fry_2010, Guinan_2019f, Hardy_2023g, Jehangir_2019h, Jin_2023, Kar_2015, Kerestes_2019, Kobiela_2015, Kumar_2001, Kumar_2019f, Liu_2005, Mesfin_2022a, Misra_2013, Ohno_2005, Peixoto_2017f, Sakellaridis_2008f, Sultan_2024f, Tammana_2012j, Tanrikulu_2015e, Yildiz_2016e, fjbuilsRepeatedBehaviorDeliberate2024, teWildt_2010}, 19 cases (26\%) had ingested previously \cite{Alao_2006i, Apikotoa_2022f, Berry_2021e, Bhattacharjee_2008, Csaky_1998e, DivsalarP._2023a, Emamhadi_2018, Guinan_2019f, Jehangir_2019h, Jin_2023, Liu_2005, Sakellaridis_2008f, Tanrikulu_2015e, Thapa_2019f, Yildiz_2016e, fjbuilsRepeatedBehaviorDeliberate2024, teWildt_2010}, 12 cases (17\%) were detained persons \cite{Alao_2006i, Ali_2022g, Apikotoa_2022f, Losanoff_1996, Losanoff_1997e, Qureshi_2016, Tammana_2012j, Trgo_2012f}, 7 cases (10\%) were severely disabled \cite{Atayan_2016, Kerestes_2019, Liu_2005, Ohno_2005, Peixoto_2017f, Yildiz_2016e, teWildt_2010}, 4 cases (6\%) were psychiatric inpatients \cite{DivsalarP._2023a, fjbuilsRepeatedBehaviorDeliberate2024, teWildt_2010}, 3 cases (4\%) were under the influence of alcohol \cite{Benoist_2019e, Csaky_1998e, Thapa_2019f}, 2 cases (3\%) were displaced people \cite{Akay_2015f, Gardner_2017h}. \paragraph*{Motivation} 34 cases (47\%) had a psychiatric motivation \cite{Al-Faham_2020k, Alao_2006i, Ali_2020f, Apikotoa_2022f, Ataya_2013, Atayan_2016, Bhasin_2014, Bhattacharjee_2008, DelgadoSalazar_2020c, DivsalarP._2023a, Emamhadi_2018, Farhadi_2024h, Guinan_2019f, Hardy_2023g, Jehangir_2019h, Jin_2023, Kar_2015, Kariholu_2008, Kerestes_2019, Kobiela_2015, Kumar_2001, Kumar_2019f, Li_2013, Liu_2005, Misra_2013, Ohno_2005, Sakellaridis_2008f, Sultan_2024f, Tammana_2012j, Tanrikulu_2015e, Yasin_2009, teWildt_2010}, 21 cases (29\%) were motivated by self-harm intention \cite{Al-Faham_2020k, AlShaaibi_2021b, Alao_2006i, Ali_2017, CamachoDorado_2018, Chang_2017f, Cox_2007, Csaky_1998e, Fry_2010, Li_2013, Losanoff_1996, Losanoff_1997e, Mesfin_2022a, Sakellaridis_2008f, Tammana_2012j, Tanrikulu_2015e, fjbuilsRepeatedBehaviorDeliberate2024}, 17 cases (24\%) had a psychosocial motivation \cite{Akay_2015f, Benoist_2019e, Bhattacharjee_2008, Cauchi_2002, Goldman_1998f, Hardy_2023g, Kobiela_2015, Li_2013, Naji_2012f, Qureshi_2016, Riva_2018j, Sobnach_2011f, Tay_2004, Thapa_2019f, Tupesis_2004f, Wildhaber_2005, Wnęk_2015f}, 9 cases (12\%) were motivated by protest \cite{Bhumi_2024f, Gardner_2017h, Losanoff_1996, Losanoff_1997e, Tupesis_2004f}, 9 cases (12\%) had another documented motivation \cite{Ali_2020f, Ali_2022g, Emamhadi_2018, Guinan_2019f, Peixoto_2017f, Sakellaridis_2008f, Trgo_2012f, Wadhwa_2015e, Yildiz_2016e}. \paragraph*{Object Characteristics} 51 cases (71\%) ingested a large diameter object (\textgreater{}2.5cm) \cite{Akay_2015f, Al-Faham_2020k, AlShaaibi_2021b, Alao_2006i, Ali_2017, Ali_2022g, Apikotoa_2022f, Atayan_2016, Berry_2021e, Bhasin_2014, CamachoDorado_2018, Cauchi_2002, Chang_2017f, Cox_2007, Csaky_1998e, DivsalarP._2023a, Emamhadi_2018, Gardner_2017h, Guinan_2019f, Jehangir_2019h, Jin_2023, Kariholu_2008, Kerestes_2019, Kobiela_2015, Kumar_2001, Kumar_2019f, Losanoff_1996, Losanoff_1997e, Mesfin_2022a, Misra_2013, Naji_2012f, Ohno_2005, Peixoto_2017f, Qureshi_2016, Riva_2018j, Sakellaridis_2008f, Sultan_2024f, Tanrikulu_2015e, Thapa_2019f, Trgo_2012f, Wnęk_2015f, Yildiz_2016e, fjbuilsRepeatedBehaviorDeliberate2024, teWildt_2010}, 44 cases (61\%) ingested multiple objects \cite{Ali_2020f, Apikotoa_2022f, Ataya_2013, Atayan_2016, Beecroft_1998, Bhattacharjee_2008, Bhumi_2024f, CamachoDorado_2018, Cauchi_2002, Emamhadi_2018, Farhadi_2024h, Fry_2010, Goldman_1998f, Guinan_2019f, Hardy_2023g, Jehangir_2019h, Jin_2023, Kar_2015, Kariholu_2008, Kobiela_2015, Kumar_2001, Kumar_2019f, Li_2013, Liu_2005, Losanoff_1996, Mesfin_2022a, Misra_2013, Naji_2012f, Ohno_2005, Sobnach_2011f, Sultan_2024f, Tammana_2012j, Tanrikulu_2015e, Tay_2004, Thapa_2019f, Wadhwa_2015e, Wildhaber_2005, Yasin_2009, fjbuilsRepeatedBehaviorDeliberate2024, teWildt_2010}, 34 cases (47\%) ingested a sharp object \cite{AlShaaibi_2021b, Alao_2006i, Apikotoa_2022f, Ataya_2013, Benoist_2019e, Bhasin_2014, Bhattacharjee_2008, CamachoDorado_2018, Csaky_1998e, DelgadoSalazar_2020c, DivsalarP._2023a, Emamhadi_2018, Farhadi_2024h, Fry_2010, Guinan_2019f, Hardy_2023g, Jehangir_2019h, Jin_2023, Kariholu_2008, Kobiela_2015, Kumar_2019f, Losanoff_1996, Losanoff_1997e, Mesfin_2022a, Misra_2013, Sobnach_2011f, Yasin_2009, teWildt_2010}, 32 cases (44\%) ingested a long object (\textgreater{}5cm) \cite{Al-Faham_2020k, AlShaaibi_2021b, Ali_2017, Ali_2022g, Atayan_2016, Bhasin_2014, CamachoDorado_2018, Chang_2017f, Cox_2007, Csaky_1998e, DivsalarP._2023a, Emamhadi_2018, Fry_2010, Gardner_2017h, Jin_2023, Kariholu_2008, Kerestes_2019, Kobiela_2015, Kumar_2019f, Mesfin_2022a, Misra_2013, Ohno_2005, Qureshi_2016, Sakellaridis_2008f, Sultan_2024f, Thapa_2019f, Trgo_2012f, Yasin_2009, Yildiz_2016e, teWildt_2010}, 9 cases (12\%) ingested a magnet \cite{Ali_2020f, Bhumi_2024f, Cauchi_2002, Liu_2005, Naji_2012f, Ohno_2005, Tanrikulu_2015e, Tay_2004, Wildhaber_2005}, 2 cases (3\%) ingested a button battery \cite{Berry_2021e, Bhumi_2024f}. \paragraph*{Outcomes} 48 cases (67\%) experienced a complication \cite{Ali_2017, Ali_2020f, Apikotoa_2022f, Atayan_2016, Beecroft_1998, Benoist_2019e, Berry_2021e, Bhasin_2014, Bhumi_2024f, CamachoDorado_2018, Cauchi_2002, Cox_2007, Csaky_1998e, DelgadoSalazar_2020c, DivsalarP._2023a, Emamhadi_2018, Farhadi_2024h, Fry_2010, Gardner_2017h, Goldman_1998f, Jin_2023, Kariholu_2008, Kerestes_2019, Kobiela_2015, Kumar_2001, Kumar_2019f, Liu_2005, Losanoff_1996, Mesfin_2022a, Misra_2013, Naji_2012f, Ohno_2005, Sakellaridis_2008f, Sobnach_2011f, Sultan_2024f, Tanrikulu_2015e, Tay_2004, Thapa_2019f, Trgo_2012f, Tupesis_2004f, Wildhaber_2005, Wnęk_2015f, Yasin_2009, Yildiz_2016e}, 44 cases (61\%) underwent surgery \cite{Al-Faham_2020k, AlShaaibi_2021b, Alao_2006i, Ali_2017, Ali_2020f, Atayan_2016, Beecroft_1998, Bhasin_2014, CamachoDorado_2018, Cauchi_2002, Chang_2017f, Cox_2007, Csaky_1998e, DelgadoSalazar_2020c, DivsalarP._2023a, Farhadi_2024h, Fry_2010, Gardner_2017h, Jin_2023, Kariholu_2008, Kerestes_2019, Kobiela_2015, Kumar_2019f, Liu_2005, Losanoff_1996, Losanoff_1997e, Mesfin_2022a, Misra_2013, Naji_2012f, Sobnach_2011f, Tanrikulu_2015e, Tay_2004, Thapa_2019f, Tupesis_2004f, Wildhaber_2005, Wnęk_2015f, Yasin_2009, Yildiz_2016e, fjbuilsRepeatedBehaviorDeliberate2024}, 31 cases (43\%) underwent endoscopy \cite{Akay_2015f, Ali_2022g, Apikotoa_2022f, Atayan_2016, Benoist_2019e, Berry_2021e, Bhasin_2014, Bhumi_2024f, CamachoDorado_2018, Chang_2017f, DelgadoSalazar_2020c, Gardner_2017h, Guinan_2019f, Hardy_2023g, Jehangir_2019h, Kariholu_2008, Li_2013, Liu_2005, Ohno_2005, Peixoto_2017f, Qureshi_2016, Riva_2018j, Sakellaridis_2008f, Sultan_2024f, Tammana_2012j, Tanrikulu_2015e, Trgo_2012f, Wadhwa_2015e, Wnęk_2015f, teWildt_2010}, 7 cases (10\%) were managed conservatively \cite{Ataya_2013, Bhattacharjee_2008, DivsalarP._2023a, Emamhadi_2018, Goldman_1998f, Kar_2015, Kumar_2001}, 2 cases (3\%) died \cite{Emamhadi_2018, Kumar_2001}. All 90 were male gender. 90 cases (100\%) were detained at the time of ingestion \cite{Elghali_2016, Karp_1991b, Lee_2007}, 88 cases (98\%) were intentional ingestions \cite{Elghali_2016, Karp_1991b, Lee_2007}, 30 cases (33\%) had a psychiatric history documented \cite{Elghali_2016, Karp_1991b, Lee_2007}, 2 cases (2\%) had a history of prior ingestion \cite{Elghali_2016}. No cases were reported for were psychiatric inpatients, were displaced people, were under the influence of alcohol at the time of ingestion, and had a severe disability history.
\paragraph*{Motivation}  70 cases (78\%) reported protest motivation \cite{Elghali_2016, Karp_1991b, Lee_2007}, 12 cases (13\%) reported psychiatric motivation \cite{Karp_1991b}, 6 cases (7\%) reported self-harm motivation \cite{Elghali_2016, Karp_1991b}. No cases were reported for psychosocial motivation and other motivation.
\paragraph*{Object Characteristics}  68 cases (76\%) involved sharp object ingestion \cite{Elghali_2016, Karp_1991b, Lee_2007}, 32 cases (36\%) involved long (\textgreater 5cm) object ingestion \cite{Lee_2007}, 25 cases (28\%) involved ingestion of multiple objects \cite{Elghali_2016, Lee_2007}. No cases were reported for button battery ingestion, magnet ingestion, and involved large diameter (\textgreater 2.5cm) object ingestion.
\paragraph*{Outcomes}  47 cases (52\%) underwent endoscopic intervention \cite{Elghali_2016, Lee_2007}, 29 cases (32\%) were managed conservatively \cite{Elghali_2016, Karp_1991b}, 15 cases (17\%) underwent surgical intervention \cite{Elghali_2016, Karp_1991b, Lee_2007}, 6 cases (7\%) reported complications \cite{Lee_2007}, 1 case (1\%) died \cite{Elghali_2016}.
\paragraph*{Geographical Location}Cases were recorded in 33 countries: 13 cases from USA \cite{Alao_2006i, Ataya_2013, Bhumi_2024f, Fry_2010, Guinan_2019f, Hardy_2023g, Jehangir_2019h, Kerestes_2019, Kumar_2001, Liu_2005, Tammana_2012j, Tay_2004, Tupesis_2004f}; 7 cases from India \cite{Bhasin_2014, Bhattacharjee_2008, Kar_2015, Kariholu_2008, Kumar_2019f, Misra_2013, Wadhwa_2015e} and UK \cite{Beecroft_1998, Berry_2021e, Cauchi_2002, Cox_2007, Gardner_2017h, Qureshi_2016}; 6 cases from Bulgaria \cite{Losanoff_1996, Losanoff_1997e}; 5 cases from Iran \cite{DivsalarP._2023a, Emamhadi_2018, Farhadi_2024h}; 4 cases from Turkey \cite{Akay_2015f, Atayan_2016, Tanrikulu_2015e, Yildiz_2016e}; 2 cases from China \cite{Jin_2023, Li_2013}, Poland \cite{Kobiela_2015, Wnęk_2015f}, and Spain \cite{CamachoDorado_2018, fjbuilsRepeatedBehaviorDeliberate2024}; 1 case from Australia \cite{Apikotoa_2022f}, Bahrain \cite{Ali_2020f}, Croatia \cite{Trgo_2012f}, Ecuador \cite{DelgadoSalazar_2020c}, Egypt \cite{Ali_2022g}, Ethiopia \cite{Mesfin_2022a}, Germany \cite{teWildt_2010}, Greece \cite{Sakellaridis_2008f}, Hungary \cite{Csaky_1998e}, Iraq \cite{Al-Faham_2020k}, Israel \cite{Goldman_1998f}, Italy \cite{Riva_2018j}, Japan \cite{Ohno_2005}, Nepal \cite{Thapa_2019f}, Netherlands \cite{Benoist_2019e}, Oman \cite{AlShaaibi_2021b}, Pakistan \cite{Yasin_2009}, Portugal \cite{Peixoto_2017f}, Qatar \cite{Ali_2017}, Saudi Arabia \cite{Sultan_2024f}, South Africa \cite{Sobnach_2011f}, Sweden \cite{Naji_2012f}, Switzerland \cite{Wildhaber_2005}, and Taiwan \cite{Chang_2017f}. \paragraph*{Gender} 43 cases (60\%) were male \cite{Akay_2015f, Al-Faham_2020k, Alao_2006i, Ali_2017, Ali_2022g, Apikotoa_2022f, Atayan_2016, Benoist_2019e, Berry_2021e, Bhumi_2024f, CamachoDorado_2018, Csaky_1998e, Emamhadi_2018, Farhadi_2024h, Fry_2010, Gardner_2017h, Guinan_2019f, Jehangir_2019h, Jin_2023, Kobiela_2015, Kumar_2001, Kumar_2019f, Liu_2005, Losanoff_1996, Losanoff_1997e, Mesfin_2022a, Misra_2013, Qureshi_2016, Riva_2018j, Sobnach_2011f, Tammana_2012j, Tanrikulu_2015e, Tay_2004, Thapa_2019f, Trgo_2012f, Wadhwa_2015e, Yasin_2009, teWildt_2010}, 28 cases (39\%) were female \cite{AlShaaibi_2021b, Ali_2020f, Ataya_2013, Beecroft_1998, Bhasin_2014, Bhattacharjee_2008, Cauchi_2002, Chang_2017f, Cox_2007, DelgadoSalazar_2020c, DivsalarP._2023a, Goldman_1998f, Hardy_2023g, Kar_2015, Kariholu_2008, Kerestes_2019, Li_2013, Naji_2012f, Ohno_2005, Peixoto_2017f, Sakellaridis_2008f, Sultan_2024f, Tupesis_2004f, Wildhaber_2005, Wnęk_2015f, Yildiz_2016e}, 1 case (1\%) had no gender recorded \cite{fjbuilsRepeatedBehaviorDeliberate2024}. \paragraph*{Age Group} 25 cases (35\%) were between 26 and 40 years of age \cite{Alao_2006i, Ali_2022g, Apikotoa_2022f, Ataya_2013, Benoist_2019e, Bhasin_2014, Chang_2017f, Cox_2007, DelgadoSalazar_2020c, Farhadi_2024h, Fry_2010, Gardner_2017h, Guinan_2019f, Jin_2023, Kumar_2019f, Losanoff_1996, Misra_2013, Qureshi_2016, Riva_2018j, Sakellaridis_2008f, Tammana_2012j, Trgo_2012f, Wnęk_2015f, Yildiz_2016e, fjbuilsRepeatedBehaviorDeliberate2024}, 18 cases (25\%) were between 18 and 25 years of age \cite{Akay_2015f, Ali_2017, Atayan_2016, Bhattacharjee_2008, Csaky_1998e, Kar_2015, Kariholu_2008, Kobiela_2015, Losanoff_1996, Losanoff_1997e, Mesfin_2022a, Peixoto_2017f, Sobnach_2011f, Tupesis_2004f, Yasin_2009}, 13 cases (18\%) were under 18 years of age \cite{AlShaaibi_2021b, Ali_2020f, Cauchi_2002, DivsalarP._2023a, Goldman_1998f, Liu_2005, Naji_2012f, Ohno_2005, Tanrikulu_2015e, Tay_2004, Wildhaber_2005}, 11 cases (15\%) were between 41 and 60 years of age \cite{Al-Faham_2020k, Bhumi_2024f, CamachoDorado_2018, Emamhadi_2018, Hardy_2023g, Jehangir_2019h, Kumar_2001, Sultan_2024f, Thapa_2019f, Wadhwa_2015e, teWildt_2010}, 3 cases (4\%) were over 60 years of age \cite{Beecroft_1998, Kerestes_2019, Li_2013}, 2 cases (3\%) had no age documented \cite{Berry_2021e}. \paragraph*{Population} 36 cases (50\%) had a psychiatric history \cite{AlShaaibi_2021b, Alao_2006i, Ali_2020f, Apikotoa_2022f, Ataya_2013, Atayan_2016, Beecroft_1998, CamachoDorado_2018, Chang_2017f, DelgadoSalazar_2020c, DivsalarP._2023a, Farhadi_2024h, Fry_2010, Guinan_2019f, Hardy_2023g, Jehangir_2019h, Jin_2023, Kar_2015, Kerestes_2019, Kobiela_2015, Kumar_2001, Kumar_2019f, Liu_2005, Mesfin_2022a, Misra_2013, Ohno_2005, Peixoto_2017f, Sakellaridis_2008f, Sultan_2024f, Tammana_2012j, Tanrikulu_2015e, Yildiz_2016e, fjbuilsRepeatedBehaviorDeliberate2024, teWildt_2010}, 19 cases (26\%) had ingested previously \cite{Alao_2006i, Apikotoa_2022f, Berry_2021e, Bhattacharjee_2008, Csaky_1998e, DivsalarP._2023a, Emamhadi_2018, Guinan_2019f, Jehangir_2019h, Jin_2023, Liu_2005, Sakellaridis_2008f, Tanrikulu_2015e, Thapa_2019f, Yildiz_2016e, fjbuilsRepeatedBehaviorDeliberate2024, teWildt_2010}, 12 cases (17\%) were detained persons \cite{Alao_2006i, Ali_2022g, Apikotoa_2022f, Losanoff_1996, Losanoff_1997e, Qureshi_2016, Tammana_2012j, Trgo_2012f}, 7 cases (10\%) were severely disabled \cite{Atayan_2016, Kerestes_2019, Liu_2005, Ohno_2005, Peixoto_2017f, Yildiz_2016e, teWildt_2010}, 4 cases (6\%) were psychiatric inpatients \cite{DivsalarP._2023a, fjbuilsRepeatedBehaviorDeliberate2024, teWildt_2010}, 3 cases (4\%) were under the influence of alcohol \cite{Benoist_2019e, Csaky_1998e, Thapa_2019f}, 2 cases (3\%) were displaced people \cite{Akay_2015f, Gardner_2017h}. \paragraph*{Motivation} 34 cases (47\%) had a psychiatric motivation \cite{Al-Faham_2020k, Alao_2006i, Ali_2020f, Apikotoa_2022f, Ataya_2013, Atayan_2016, Bhasin_2014, Bhattacharjee_2008, DelgadoSalazar_2020c, DivsalarP._2023a, Emamhadi_2018, Farhadi_2024h, Guinan_2019f, Hardy_2023g, Jehangir_2019h, Jin_2023, Kar_2015, Kariholu_2008, Kerestes_2019, Kobiela_2015, Kumar_2001, Kumar_2019f, Li_2013, Liu_2005, Misra_2013, Ohno_2005, Sakellaridis_2008f, Sultan_2024f, Tammana_2012j, Tanrikulu_2015e, Yasin_2009, teWildt_2010}, 21 cases (29\%) were motivated by self-harm intention \cite{Al-Faham_2020k, AlShaaibi_2021b, Alao_2006i, Ali_2017, CamachoDorado_2018, Chang_2017f, Cox_2007, Csaky_1998e, Fry_2010, Li_2013, Losanoff_1996, Losanoff_1997e, Mesfin_2022a, Sakellaridis_2008f, Tammana_2012j, Tanrikulu_2015e, fjbuilsRepeatedBehaviorDeliberate2024}, 17 cases (24\%) had a psychosocial motivation \cite{Akay_2015f, Benoist_2019e, Bhattacharjee_2008, Cauchi_2002, Goldman_1998f, Hardy_2023g, Kobiela_2015, Li_2013, Naji_2012f, Qureshi_2016, Riva_2018j, Sobnach_2011f, Tay_2004, Thapa_2019f, Tupesis_2004f, Wildhaber_2005, Wnęk_2015f}, 9 cases (12\%) were motivated by protest \cite{Bhumi_2024f, Gardner_2017h, Losanoff_1996, Losanoff_1997e, Tupesis_2004f}, 9 cases (12\%) had another documented motivation \cite{Ali_2020f, Ali_2022g, Emamhadi_2018, Guinan_2019f, Peixoto_2017f, Sakellaridis_2008f, Trgo_2012f, Wadhwa_2015e, Yildiz_2016e}. \paragraph*{Object Characteristics} 51 cases (71\%) ingested a large diameter object (\textgreater{}2.5cm) \cite{Akay_2015f, Al-Faham_2020k, AlShaaibi_2021b, Alao_2006i, Ali_2017, Ali_2022g, Apikotoa_2022f, Atayan_2016, Berry_2021e, Bhasin_2014, CamachoDorado_2018, Cauchi_2002, Chang_2017f, Cox_2007, Csaky_1998e, DivsalarP._2023a, Emamhadi_2018, Gardner_2017h, Guinan_2019f, Jehangir_2019h, Jin_2023, Kariholu_2008, Kerestes_2019, Kobiela_2015, Kumar_2001, Kumar_2019f, Losanoff_1996, Losanoff_1997e, Mesfin_2022a, Misra_2013, Naji_2012f, Ohno_2005, Peixoto_2017f, Qureshi_2016, Riva_2018j, Sakellaridis_2008f, Sultan_2024f, Tanrikulu_2015e, Thapa_2019f, Trgo_2012f, Wnęk_2015f, Yildiz_2016e, fjbuilsRepeatedBehaviorDeliberate2024, teWildt_2010}, 44 cases (61\%) ingested multiple objects \cite{Ali_2020f, Apikotoa_2022f, Ataya_2013, Atayan_2016, Beecroft_1998, Bhattacharjee_2008, Bhumi_2024f, CamachoDorado_2018, Cauchi_2002, Emamhadi_2018, Farhadi_2024h, Fry_2010, Goldman_1998f, Guinan_2019f, Hardy_2023g, Jehangir_2019h, Jin_2023, Kar_2015, Kariholu_2008, Kobiela_2015, Kumar_2001, Kumar_2019f, Li_2013, Liu_2005, Losanoff_1996, Mesfin_2022a, Misra_2013, Naji_2012f, Ohno_2005, Sobnach_2011f, Sultan_2024f, Tammana_2012j, Tanrikulu_2015e, Tay_2004, Thapa_2019f, Wadhwa_2015e, Wildhaber_2005, Yasin_2009, fjbuilsRepeatedBehaviorDeliberate2024, teWildt_2010}, 34 cases (47\%) ingested a sharp object \cite{AlShaaibi_2021b, Alao_2006i, Apikotoa_2022f, Ataya_2013, Benoist_2019e, Bhasin_2014, Bhattacharjee_2008, CamachoDorado_2018, Csaky_1998e, DelgadoSalazar_2020c, DivsalarP._2023a, Emamhadi_2018, Farhadi_2024h, Fry_2010, Guinan_2019f, Hardy_2023g, Jehangir_2019h, Jin_2023, Kariholu_2008, Kobiela_2015, Kumar_2019f, Losanoff_1996, Losanoff_1997e, Mesfin_2022a, Misra_2013, Sobnach_2011f, Yasin_2009, teWildt_2010}, 32 cases (44\%) ingested a long object (\textgreater{}5cm) \cite{Al-Faham_2020k, AlShaaibi_2021b, Ali_2017, Ali_2022g, Atayan_2016, Bhasin_2014, CamachoDorado_2018, Chang_2017f, Cox_2007, Csaky_1998e, DivsalarP._2023a, Emamhadi_2018, Fry_2010, Gardner_2017h, Jin_2023, Kariholu_2008, Kerestes_2019, Kobiela_2015, Kumar_2019f, Mesfin_2022a, Misra_2013, Ohno_2005, Qureshi_2016, Sakellaridis_2008f, Sultan_2024f, Thapa_2019f, Trgo_2012f, Yasin_2009, Yildiz_2016e, teWildt_2010}, 9 cases (12\%) ingested a magnet \cite{Ali_2020f, Bhumi_2024f, Cauchi_2002, Liu_2005, Naji_2012f, Ohno_2005, Tanrikulu_2015e, Tay_2004, Wildhaber_2005}, 2 cases (3\%) ingested a button battery \cite{Berry_2021e, Bhumi_2024f}. \paragraph*{Outcomes} 48 cases (67\%) experienced a complication \cite{Ali_2017, Ali_2020f, Apikotoa_2022f, Atayan_2016, Beecroft_1998, Benoist_2019e, Berry_2021e, Bhasin_2014, Bhumi_2024f, CamachoDorado_2018, Cauchi_2002, Cox_2007, Csaky_1998e, DelgadoSalazar_2020c, DivsalarP._2023a, Emamhadi_2018, Farhadi_2024h, Fry_2010, Gardner_2017h, Goldman_1998f, Jin_2023, Kariholu_2008, Kerestes_2019, Kobiela_2015, Kumar_2001, Kumar_2019f, Liu_2005, Losanoff_1996, Mesfin_2022a, Misra_2013, Naji_2012f, Ohno_2005, Sakellaridis_2008f, Sobnach_2011f, Sultan_2024f, Tanrikulu_2015e, Tay_2004, Thapa_2019f, Trgo_2012f, Tupesis_2004f, Wildhaber_2005, Wnęk_2015f, Yasin_2009, Yildiz_2016e}, 44 cases (61\%) underwent surgery \cite{Al-Faham_2020k, AlShaaibi_2021b, Alao_2006i, Ali_2017, Ali_2020f, Atayan_2016, Beecroft_1998, Bhasin_2014, CamachoDorado_2018, Cauchi_2002, Chang_2017f, Cox_2007, Csaky_1998e, DelgadoSalazar_2020c, DivsalarP._2023a, Farhadi_2024h, Fry_2010, Gardner_2017h, Jin_2023, Kariholu_2008, Kerestes_2019, Kobiela_2015, Kumar_2019f, Liu_2005, Losanoff_1996, Losanoff_1997e, Mesfin_2022a, Misra_2013, Naji_2012f, Sobnach_2011f, Tanrikulu_2015e, Tay_2004, Thapa_2019f, Tupesis_2004f, Wildhaber_2005, Wnęk_2015f, Yasin_2009, Yildiz_2016e, fjbuilsRepeatedBehaviorDeliberate2024}, 31 cases (43\%) underwent endoscopy \cite{Akay_2015f, Ali_2022g, Apikotoa_2022f, Atayan_2016, Benoist_2019e, Berry_2021e, Bhasin_2014, Bhumi_2024f, CamachoDorado_2018, Chang_2017f, DelgadoSalazar_2020c, Gardner_2017h, Guinan_2019f, Hardy_2023g, Jehangir_2019h, Kariholu_2008, Li_2013, Liu_2005, Ohno_2005, Peixoto_2017f, Qureshi_2016, Riva_2018j, Sakellaridis_2008f, Sultan_2024f, Tammana_2012j, Tanrikulu_2015e, Trgo_2012f, Wadhwa_2015e, Wnęk_2015f, teWildt_2010}, 7 cases (10\%) were managed conservatively \cite{Ataya_2013, Bhattacharjee_2008, DivsalarP._2023a, Emamhadi_2018, Goldman_1998f, Kar_2015, Kumar_2001}, 2 cases (3\%) died \cite{Emamhadi_2018, Kumar_2001}. All 90 were male gender. 90 cases (100\%) were detained at the time of ingestion \cite{Elghali_2016, Karp_1991b, Lee_2007}, 88 cases (98\%) were intentional ingestions \cite{Elghali_2016, Karp_1991b, Lee_2007}, 30 cases (33\%) had a psychiatric history documented \cite{Elghali_2016, Karp_1991b, Lee_2007}, 2 cases (2\%) had a history of prior ingestion \cite{Elghali_2016}. No cases were reported for were psychiatric inpatients, were displaced people, were under the influence of alcohol at the time of ingestion, and had a severe disability history.
\paragraph*{Motivation}  70 cases (78\%) reported protest motivation \cite{Elghali_2016, Karp_1991b, Lee_2007}, 12 cases (13\%) reported psychiatric motivation \cite{Karp_1991b}, 6 cases (7\%) reported self-harm motivation \cite{Elghali_2016, Karp_1991b}. No cases were reported for psychosocial motivation and other motivation.
\paragraph*{Object Characteristics}  68 cases (76\%) involved sharp object ingestion \cite{Elghali_2016, Karp_1991b, Lee_2007}, 32 cases (36\%) involved long (\textgreater 5cm) object ingestion \cite{Lee_2007}, 25 cases (28\%) involved ingestion of multiple objects \cite{Elghali_2016, Lee_2007}. No cases were reported for button battery ingestion, magnet ingestion, and involved large diameter (\textgreater 2.5cm) object ingestion.
\paragraph*{Outcomes}  47 cases (52\%) underwent endoscopic intervention \cite{Elghali_2016, Lee_2007}, 29 cases (32\%) were managed conservatively \cite{Elghali_2016, Karp_1991b}, 15 cases (17\%) underwent surgical intervention \cite{Elghali_2016, Karp_1991b, Lee_2007}, 6 cases (7\%) reported complications \cite{Lee_2007}, 1 case (1\%) died \cite{Elghali_2016}.
\paragraph*{Geographical Location}Cases were recorded in 33 countries: 13 cases from USA \cite{Alao_2006i, Ataya_2013, Bhumi_2024f, Fry_2010, Guinan_2019f, Hardy_2023g, Jehangir_2019h, Kerestes_2019, Kumar_2001, Liu_2005, Tammana_2012j, Tay_2004, Tupesis_2004f}; 7 cases from India \cite{Bhasin_2014, Bhattacharjee_2008, Kar_2015, Kariholu_2008, Kumar_2019f, Misra_2013, Wadhwa_2015e} and UK \cite{Beecroft_1998, Berry_2021e, Cauchi_2002, Cox_2007, Gardner_2017h, Qureshi_2016}; 6 cases from Bulgaria \cite{Losanoff_1996, Losanoff_1997e}; 5 cases from Iran \cite{DivsalarP._2023a, Emamhadi_2018, Farhadi_2024h}; 4 cases from Turkey \cite{Akay_2015f, Atayan_2016, Tanrikulu_2015e, Yildiz_2016e}; 2 cases from China \cite{Jin_2023, Li_2013}, Poland \cite{Kobiela_2015, Wnęk_2015f}, and Spain \cite{CamachoDorado_2018, fjbuilsRepeatedBehaviorDeliberate2024}; 1 case from Australia \cite{Apikotoa_2022f}, Bahrain \cite{Ali_2020f}, Croatia \cite{Trgo_2012f}, Ecuador \cite{DelgadoSalazar_2020c}, Egypt \cite{Ali_2022g}, Ethiopia \cite{Mesfin_2022a}, Germany \cite{teWildt_2010}, Greece \cite{Sakellaridis_2008f}, Hungary \cite{Csaky_1998e}, Iraq \cite{Al-Faham_2020k}, Israel \cite{Goldman_1998f}, Italy \cite{Riva_2018j}, Japan \cite{Ohno_2005}, Nepal \cite{Thapa_2019f}, Netherlands \cite{Benoist_2019e}, Oman \cite{AlShaaibi_2021b}, Pakistan \cite{Yasin_2009}, Portugal \cite{Peixoto_2017f}, Qatar \cite{Ali_2017}, Saudi Arabia \cite{Sultan_2024f}, South Africa \cite{Sobnach_2011f}, Sweden \cite{Naji_2012f}, Switzerland \cite{Wildhaber_2005}, and Taiwan \cite{Chang_2017f}. \paragraph*{Gender} 43 cases (60\%) were male \cite{Akay_2015f, Al-Faham_2020k, Alao_2006i, Ali_2017, Ali_2022g, Apikotoa_2022f, Atayan_2016, Benoist_2019e, Berry_2021e, Bhumi_2024f, CamachoDorado_2018, Csaky_1998e, Emamhadi_2018, Farhadi_2024h, Fry_2010, Gardner_2017h, Guinan_2019f, Jehangir_2019h, Jin_2023, Kobiela_2015, Kumar_2001, Kumar_2019f, Liu_2005, Losanoff_1996, Losanoff_1997e, Mesfin_2022a, Misra_2013, Qureshi_2016, Riva_2018j, Sobnach_2011f, Tammana_2012j, Tanrikulu_2015e, Tay_2004, Thapa_2019f, Trgo_2012f, Wadhwa_2015e, Yasin_2009, teWildt_2010}, 28 cases (39\%) were female \cite{AlShaaibi_2021b, Ali_2020f, Ataya_2013, Beecroft_1998, Bhasin_2014, Bhattacharjee_2008, Cauchi_2002, Chang_2017f, Cox_2007, DelgadoSalazar_2020c, DivsalarP._2023a, Goldman_1998f, Hardy_2023g, Kar_2015, Kariholu_2008, Kerestes_2019, Li_2013, Naji_2012f, Ohno_2005, Peixoto_2017f, Sakellaridis_2008f, Sultan_2024f, Tupesis_2004f, Wildhaber_2005, Wnęk_2015f, Yildiz_2016e}, 1 case (1\%) had no gender recorded \cite{fjbuilsRepeatedBehaviorDeliberate2024}. \paragraph*{Age Group} 25 cases (35\%) were between 26 and 40 years of age \cite{Alao_2006i, Ali_2022g, Apikotoa_2022f, Ataya_2013, Benoist_2019e, Bhasin_2014, Chang_2017f, Cox_2007, DelgadoSalazar_2020c, Farhadi_2024h, Fry_2010, Gardner_2017h, Guinan_2019f, Jin_2023, Kumar_2019f, Losanoff_1996, Misra_2013, Qureshi_2016, Riva_2018j, Sakellaridis_2008f, Tammana_2012j, Trgo_2012f, Wnęk_2015f, Yildiz_2016e, fjbuilsRepeatedBehaviorDeliberate2024}, 18 cases (25\%) were between 18 and 25 years of age \cite{Akay_2015f, Ali_2017, Atayan_2016, Bhattacharjee_2008, Csaky_1998e, Kar_2015, Kariholu_2008, Kobiela_2015, Losanoff_1996, Losanoff_1997e, Mesfin_2022a, Peixoto_2017f, Sobnach_2011f, Tupesis_2004f, Yasin_2009}, 13 cases (18\%) were under 18 years of age \cite{AlShaaibi_2021b, Ali_2020f, Cauchi_2002, DivsalarP._2023a, Goldman_1998f, Liu_2005, Naji_2012f, Ohno_2005, Tanrikulu_2015e, Tay_2004, Wildhaber_2005}, 11 cases (15\%) were between 41 and 60 years of age \cite{Al-Faham_2020k, Bhumi_2024f, CamachoDorado_2018, Emamhadi_2018, Hardy_2023g, Jehangir_2019h, Kumar_2001, Sultan_2024f, Thapa_2019f, Wadhwa_2015e, teWildt_2010}, 3 cases (4\%) were over 60 years of age \cite{Beecroft_1998, Kerestes_2019, Li_2013}, 2 cases (3\%) had no age documented \cite{Berry_2021e}. \paragraph*{Population} 36 cases (50\%) had a psychiatric history \cite{AlShaaibi_2021b, Alao_2006i, Ali_2020f, Apikotoa_2022f, Ataya_2013, Atayan_2016, Beecroft_1998, CamachoDorado_2018, Chang_2017f, DelgadoSalazar_2020c, DivsalarP._2023a, Farhadi_2024h, Fry_2010, Guinan_2019f, Hardy_2023g, Jehangir_2019h, Jin_2023, Kar_2015, Kerestes_2019, Kobiela_2015, Kumar_2001, Kumar_2019f, Liu_2005, Mesfin_2022a, Misra_2013, Ohno_2005, Peixoto_2017f, Sakellaridis_2008f, Sultan_2024f, Tammana_2012j, Tanrikulu_2015e, Yildiz_2016e, fjbuilsRepeatedBehaviorDeliberate2024, teWildt_2010}, 19 cases (26\%) had ingested previously \cite{Alao_2006i, Apikotoa_2022f, Berry_2021e, Bhattacharjee_2008, Csaky_1998e, DivsalarP._2023a, Emamhadi_2018, Guinan_2019f, Jehangir_2019h, Jin_2023, Liu_2005, Sakellaridis_2008f, Tanrikulu_2015e, Thapa_2019f, Yildiz_2016e, fjbuilsRepeatedBehaviorDeliberate2024, teWildt_2010}, 12 cases (17\%) were detained persons \cite{Alao_2006i, Ali_2022g, Apikotoa_2022f, Losanoff_1996, Losanoff_1997e, Qureshi_2016, Tammana_2012j, Trgo_2012f}, 7 cases (10\%) were severely disabled \cite{Atayan_2016, Kerestes_2019, Liu_2005, Ohno_2005, Peixoto_2017f, Yildiz_2016e, teWildt_2010}, 4 cases (6\%) were psychiatric inpatients \cite{DivsalarP._2023a, fjbuilsRepeatedBehaviorDeliberate2024, teWildt_2010}, 3 cases (4\%) were under the influence of alcohol \cite{Benoist_2019e, Csaky_1998e, Thapa_2019f}, 2 cases (3\%) were displaced people \cite{Akay_2015f, Gardner_2017h}. \paragraph*{Motivation} 34 cases (47\%) had a psychiatric motivation \cite{Al-Faham_2020k, Alao_2006i, Ali_2020f, Apikotoa_2022f, Ataya_2013, Atayan_2016, Bhasin_2014, Bhattacharjee_2008, DelgadoSalazar_2020c, DivsalarP._2023a, Emamhadi_2018, Farhadi_2024h, Guinan_2019f, Hardy_2023g, Jehangir_2019h, Jin_2023, Kar_2015, Kariholu_2008, Kerestes_2019, Kobiela_2015, Kumar_2001, Kumar_2019f, Li_2013, Liu_2005, Misra_2013, Ohno_2005, Sakellaridis_2008f, Sultan_2024f, Tammana_2012j, Tanrikulu_2015e, Yasin_2009, teWildt_2010}, 21 cases (29\%) were motivated by self-harm intention \cite{Al-Faham_2020k, AlShaaibi_2021b, Alao_2006i, Ali_2017, CamachoDorado_2018, Chang_2017f, Cox_2007, Csaky_1998e, Fry_2010, Li_2013, Losanoff_1996, Losanoff_1997e, Mesfin_2022a, Sakellaridis_2008f, Tammana_2012j, Tanrikulu_2015e, fjbuilsRepeatedBehaviorDeliberate2024}, 17 cases (24\%) had a psychosocial motivation \cite{Akay_2015f, Benoist_2019e, Bhattacharjee_2008, Cauchi_2002, Goldman_1998f, Hardy_2023g, Kobiela_2015, Li_2013, Naji_2012f, Qureshi_2016, Riva_2018j, Sobnach_2011f, Tay_2004, Thapa_2019f, Tupesis_2004f, Wildhaber_2005, Wnęk_2015f}, 9 cases (12\%) were motivated by protest \cite{Bhumi_2024f, Gardner_2017h, Losanoff_1996, Losanoff_1997e, Tupesis_2004f}, 9 cases (12\%) had another documented motivation \cite{Ali_2020f, Ali_2022g, Emamhadi_2018, Guinan_2019f, Peixoto_2017f, Sakellaridis_2008f, Trgo_2012f, Wadhwa_2015e, Yildiz_2016e}. \paragraph*{Object Characteristics} 51 cases (71\%) ingested a large diameter object (\textgreater{}2.5cm) \cite{Akay_2015f, Al-Faham_2020k, AlShaaibi_2021b, Alao_2006i, Ali_2017, Ali_2022g, Apikotoa_2022f, Atayan_2016, Berry_2021e, Bhasin_2014, CamachoDorado_2018, Cauchi_2002, Chang_2017f, Cox_2007, Csaky_1998e, DivsalarP._2023a, Emamhadi_2018, Gardner_2017h, Guinan_2019f, Jehangir_2019h, Jin_2023, Kariholu_2008, Kerestes_2019, Kobiela_2015, Kumar_2001, Kumar_2019f, Losanoff_1996, Losanoff_1997e, Mesfin_2022a, Misra_2013, Naji_2012f, Ohno_2005, Peixoto_2017f, Qureshi_2016, Riva_2018j, Sakellaridis_2008f, Sultan_2024f, Tanrikulu_2015e, Thapa_2019f, Trgo_2012f, Wnęk_2015f, Yildiz_2016e, fjbuilsRepeatedBehaviorDeliberate2024, teWildt_2010}, 44 cases (61\%) ingested multiple objects \cite{Ali_2020f, Apikotoa_2022f, Ataya_2013, Atayan_2016, Beecroft_1998, Bhattacharjee_2008, Bhumi_2024f, CamachoDorado_2018, Cauchi_2002, Emamhadi_2018, Farhadi_2024h, Fry_2010, Goldman_1998f, Guinan_2019f, Hardy_2023g, Jehangir_2019h, Jin_2023, Kar_2015, Kariholu_2008, Kobiela_2015, Kumar_2001, Kumar_2019f, Li_2013, Liu_2005, Losanoff_1996, Mesfin_2022a, Misra_2013, Naji_2012f, Ohno_2005, Sobnach_2011f, Sultan_2024f, Tammana_2012j, Tanrikulu_2015e, Tay_2004, Thapa_2019f, Wadhwa_2015e, Wildhaber_2005, Yasin_2009, fjbuilsRepeatedBehaviorDeliberate2024, teWildt_2010}, 34 cases (47\%) ingested a sharp object \cite{AlShaaibi_2021b, Alao_2006i, Apikotoa_2022f, Ataya_2013, Benoist_2019e, Bhasin_2014, Bhattacharjee_2008, CamachoDorado_2018, Csaky_1998e, DelgadoSalazar_2020c, DivsalarP._2023a, Emamhadi_2018, Farhadi_2024h, Fry_2010, Guinan_2019f, Hardy_2023g, Jehangir_2019h, Jin_2023, Kariholu_2008, Kobiela_2015, Kumar_2019f, Losanoff_1996, Losanoff_1997e, Mesfin_2022a, Misra_2013, Sobnach_2011f, Yasin_2009, teWildt_2010}, 32 cases (44\%) ingested a long object (\textgreater{}5cm) \cite{Al-Faham_2020k, AlShaaibi_2021b, Ali_2017, Ali_2022g, Atayan_2016, Bhasin_2014, CamachoDorado_2018, Chang_2017f, Cox_2007, Csaky_1998e, DivsalarP._2023a, Emamhadi_2018, Fry_2010, Gardner_2017h, Jin_2023, Kariholu_2008, Kerestes_2019, Kobiela_2015, Kumar_2019f, Mesfin_2022a, Misra_2013, Ohno_2005, Qureshi_2016, Sakellaridis_2008f, Sultan_2024f, Thapa_2019f, Trgo_2012f, Yasin_2009, Yildiz_2016e, teWildt_2010}, 9 cases (12\%) ingested a magnet \cite{Ali_2020f, Bhumi_2024f, Cauchi_2002, Liu_2005, Naji_2012f, Ohno_2005, Tanrikulu_2015e, Tay_2004, Wildhaber_2005}, 2 cases (3\%) ingested a button battery \cite{Berry_2021e, Bhumi_2024f}. \paragraph*{Outcomes} 48 cases (67\%) experienced a complication \cite{Ali_2017, Ali_2020f, Apikotoa_2022f, Atayan_2016, Beecroft_1998, Benoist_2019e, Berry_2021e, Bhasin_2014, Bhumi_2024f, CamachoDorado_2018, Cauchi_2002, Cox_2007, Csaky_1998e, DelgadoSalazar_2020c, DivsalarP._2023a, Emamhadi_2018, Farhadi_2024h, Fry_2010, Gardner_2017h, Goldman_1998f, Jin_2023, Kariholu_2008, Kerestes_2019, Kobiela_2015, Kumar_2001, Kumar_2019f, Liu_2005, Losanoff_1996, Mesfin_2022a, Misra_2013, Naji_2012f, Ohno_2005, Sakellaridis_2008f, Sobnach_2011f, Sultan_2024f, Tanrikulu_2015e, Tay_2004, Thapa_2019f, Trgo_2012f, Tupesis_2004f, Wildhaber_2005, Wnęk_2015f, Yasin_2009, Yildiz_2016e}, 44 cases (61\%) underwent surgery \cite{Al-Faham_2020k, AlShaaibi_2021b, Alao_2006i, Ali_2017, Ali_2020f, Atayan_2016, Beecroft_1998, Bhasin_2014, CamachoDorado_2018, Cauchi_2002, Chang_2017f, Cox_2007, Csaky_1998e, DelgadoSalazar_2020c, DivsalarP._2023a, Farhadi_2024h, Fry_2010, Gardner_2017h, Jin_2023, Kariholu_2008, Kerestes_2019, Kobiela_2015, Kumar_2019f, Liu_2005, Losanoff_1996, Losanoff_1997e, Mesfin_2022a, Misra_2013, Naji_2012f, Sobnach_2011f, Tanrikulu_2015e, Tay_2004, Thapa_2019f, Tupesis_2004f, Wildhaber_2005, Wnęk_2015f, Yasin_2009, Yildiz_2016e, fjbuilsRepeatedBehaviorDeliberate2024}, 31 cases (43\%) underwent endoscopy \cite{Akay_2015f, Ali_2022g, Apikotoa_2022f, Atayan_2016, Benoist_2019e, Berry_2021e, Bhasin_2014, Bhumi_2024f, CamachoDorado_2018, Chang_2017f, DelgadoSalazar_2020c, Gardner_2017h, Guinan_2019f, Hardy_2023g, Jehangir_2019h, Kariholu_2008, Li_2013, Liu_2005, Ohno_2005, Peixoto_2017f, Qureshi_2016, Riva_2018j, Sakellaridis_2008f, Sultan_2024f, Tammana_2012j, Tanrikulu_2015e, Trgo_2012f, Wadhwa_2015e, Wnęk_2015f, teWildt_2010}, 7 cases (10\%) were managed conservatively \cite{Ataya_2013, Bhattacharjee_2008, DivsalarP._2023a, Emamhadi_2018, Goldman_1998f, Kar_2015, Kumar_2001}, 2 cases (3\%) died \cite{Emamhadi_2018, Kumar_2001}. All 90 were male gender. 90 cases (100\%) were detained at the time of ingestion \cite{Elghali_2016, Karp_1991b, Lee_2007}, 88 cases (98\%) were intentional ingestions \cite{Elghali_2016, Karp_1991b, Lee_2007}, 30 cases (33\%) had a psychiatric history documented \cite{Elghali_2016, Karp_1991b, Lee_2007}, 2 cases (2\%) had a history of prior ingestion \cite{Elghali_2016}. No cases were reported for were psychiatric inpatients, were displaced people, were under the influence of alcohol at the time of ingestion, and had a severe disability history.
\paragraph*{Motivation}  70 cases (78\%) reported protest motivation \cite{Elghali_2016, Karp_1991b, Lee_2007}, 12 cases (13\%) reported psychiatric motivation \cite{Karp_1991b}, 6 cases (7\%) reported self-harm motivation \cite{Elghali_2016, Karp_1991b}. No cases were reported for psychosocial motivation and other motivation.
\paragraph*{Object Characteristics}  68 cases (76\%) involved sharp object ingestion \cite{Elghali_2016, Karp_1991b, Lee_2007}, 32 cases (36\%) involved long (\textgreater 5cm) object ingestion \cite{Lee_2007}, 25 cases (28\%) involved ingestion of multiple objects \cite{Elghali_2016, Lee_2007}. No cases were reported for button battery ingestion, magnet ingestion, and involved large diameter (\textgreater 2.5cm) object ingestion.
\paragraph*{Outcomes}  47 cases (52\%) underwent endoscopic intervention \cite{Elghali_2016, Lee_2007}, 29 cases (32\%) were managed conservatively \cite{Elghali_2016, Karp_1991b}, 15 cases (17\%) underwent surgical intervention \cite{Elghali_2016, Karp_1991b, Lee_2007}, 6 cases (7\%) reported complications \cite{Lee_2007}, 1 case (1\%) died \cite{Elghali_2016}.
\paragraph*{Geographical Location}Cases were recorded in 33 countries: 13 cases from USA \cite{Alao_2006i, Ataya_2013, Bhumi_2024f, Fry_2010, Guinan_2019f, Hardy_2023g, Jehangir_2019h, Kerestes_2019, Kumar_2001, Liu_2005, Tammana_2012j, Tay_2004, Tupesis_2004f}; 7 cases from India \cite{Bhasin_2014, Bhattacharjee_2008, Kar_2015, Kariholu_2008, Kumar_2019f, Misra_2013, Wadhwa_2015e} and UK \cite{Beecroft_1998, Berry_2021e, Cauchi_2002, Cox_2007, Gardner_2017h, Qureshi_2016}; 6 cases from Bulgaria \cite{Losanoff_1996, Losanoff_1997e}; 5 cases from Iran \cite{DivsalarP._2023a, Emamhadi_2018, Farhadi_2024h}; 4 cases from Turkey \cite{Akay_2015f, Atayan_2016, Tanrikulu_2015e, Yildiz_2016e}; 2 cases from China \cite{Jin_2023, Li_2013}, Poland \cite{Kobiela_2015, Wnęk_2015f}, and Spain \cite{CamachoDorado_2018, fjbuilsRepeatedBehaviorDeliberate2024}; 1 case from Australia \cite{Apikotoa_2022f}, Bahrain \cite{Ali_2020f}, Croatia \cite{Trgo_2012f}, Ecuador \cite{DelgadoSalazar_2020c}, Egypt \cite{Ali_2022g}, Ethiopia \cite{Mesfin_2022a}, Germany \cite{teWildt_2010}, Greece \cite{Sakellaridis_2008f}, Hungary \cite{Csaky_1998e}, Iraq \cite{Al-Faham_2020k}, Israel \cite{Goldman_1998f}, Italy \cite{Riva_2018j}, Japan \cite{Ohno_2005}, Nepal \cite{Thapa_2019f}, Netherlands \cite{Benoist_2019e}, Oman \cite{AlShaaibi_2021b}, Pakistan \cite{Yasin_2009}, Portugal \cite{Peixoto_2017f}, Qatar \cite{Ali_2017}, Saudi Arabia \cite{Sultan_2024f}, South Africa \cite{Sobnach_2011f}, Sweden \cite{Naji_2012f}, Switzerland \cite{Wildhaber_2005}, and Taiwan \cite{Chang_2017f}. \paragraph*{Gender} 43 cases (60\%) were male \cite{Akay_2015f, Al-Faham_2020k, Alao_2006i, Ali_2017, Ali_2022g, Apikotoa_2022f, Atayan_2016, Benoist_2019e, Berry_2021e, Bhumi_2024f, CamachoDorado_2018, Csaky_1998e, Emamhadi_2018, Farhadi_2024h, Fry_2010, Gardner_2017h, Guinan_2019f, Jehangir_2019h, Jin_2023, Kobiela_2015, Kumar_2001, Kumar_2019f, Liu_2005, Losanoff_1996, Losanoff_1997e, Mesfin_2022a, Misra_2013, Qureshi_2016, Riva_2018j, Sobnach_2011f, Tammana_2012j, Tanrikulu_2015e, Tay_2004, Thapa_2019f, Trgo_2012f, Wadhwa_2015e, Yasin_2009, teWildt_2010}, 28 cases (39\%) were female \cite{AlShaaibi_2021b, Ali_2020f, Ataya_2013, Beecroft_1998, Bhasin_2014, Bhattacharjee_2008, Cauchi_2002, Chang_2017f, Cox_2007, DelgadoSalazar_2020c, DivsalarP._2023a, Goldman_1998f, Hardy_2023g, Kar_2015, Kariholu_2008, Kerestes_2019, Li_2013, Naji_2012f, Ohno_2005, Peixoto_2017f, Sakellaridis_2008f, Sultan_2024f, Tupesis_2004f, Wildhaber_2005, Wnęk_2015f, Yildiz_2016e}, 1 case (1\%) had no gender recorded \cite{fjbuilsRepeatedBehaviorDeliberate2024}. \paragraph*{Age Group} 25 cases (35\%) were between 26 and 40 years of age \cite{Alao_2006i, Ali_2022g, Apikotoa_2022f, Ataya_2013, Benoist_2019e, Bhasin_2014, Chang_2017f, Cox_2007, DelgadoSalazar_2020c, Farhadi_2024h, Fry_2010, Gardner_2017h, Guinan_2019f, Jin_2023, Kumar_2019f, Losanoff_1996, Misra_2013, Qureshi_2016, Riva_2018j, Sakellaridis_2008f, Tammana_2012j, Trgo_2012f, Wnęk_2015f, Yildiz_2016e, fjbuilsRepeatedBehaviorDeliberate2024}, 18 cases (25\%) were between 18 and 25 years of age \cite{Akay_2015f, Ali_2017, Atayan_2016, Bhattacharjee_2008, Csaky_1998e, Kar_2015, Kariholu_2008, Kobiela_2015, Losanoff_1996, Losanoff_1997e, Mesfin_2022a, Peixoto_2017f, Sobnach_2011f, Tupesis_2004f, Yasin_2009}, 13 cases (18\%) were under 18 years of age \cite{AlShaaibi_2021b, Ali_2020f, Cauchi_2002, DivsalarP._2023a, Goldman_1998f, Liu_2005, Naji_2012f, Ohno_2005, Tanrikulu_2015e, Tay_2004, Wildhaber_2005}, 11 cases (15\%) were between 41 and 60 years of age \cite{Al-Faham_2020k, Bhumi_2024f, CamachoDorado_2018, Emamhadi_2018, Hardy_2023g, Jehangir_2019h, Kumar_2001, Sultan_2024f, Thapa_2019f, Wadhwa_2015e, teWildt_2010}, 3 cases (4\%) were over 60 years of age \cite{Beecroft_1998, Kerestes_2019, Li_2013}, 2 cases (3\%) had no age documented \cite{Berry_2021e}. \paragraph*{Population} 36 cases (50\%) had a psychiatric history \cite{AlShaaibi_2021b, Alao_2006i, Ali_2020f, Apikotoa_2022f, Ataya_2013, Atayan_2016, Beecroft_1998, CamachoDorado_2018, Chang_2017f, DelgadoSalazar_2020c, DivsalarP._2023a, Farhadi_2024h, Fry_2010, Guinan_2019f, Hardy_2023g, Jehangir_2019h, Jin_2023, Kar_2015, Kerestes_2019, Kobiela_2015, Kumar_2001, Kumar_2019f, Liu_2005, Mesfin_2022a, Misra_2013, Ohno_2005, Peixoto_2017f, Sakellaridis_2008f, Sultan_2024f, Tammana_2012j, Tanrikulu_2015e, Yildiz_2016e, fjbuilsRepeatedBehaviorDeliberate2024, teWildt_2010}, 19 cases (26\%) had ingested previously \cite{Alao_2006i, Apikotoa_2022f, Berry_2021e, Bhattacharjee_2008, Csaky_1998e, DivsalarP._2023a, Emamhadi_2018, Guinan_2019f, Jehangir_2019h, Jin_2023, Liu_2005, Sakellaridis_2008f, Tanrikulu_2015e, Thapa_2019f, Yildiz_2016e, fjbuilsRepeatedBehaviorDeliberate2024, teWildt_2010}, 12 cases (17\%) were detained persons \cite{Alao_2006i, Ali_2022g, Apikotoa_2022f, Losanoff_1996, Losanoff_1997e, Qureshi_2016, Tammana_2012j, Trgo_2012f}, 7 cases (10\%) were severely disabled \cite{Atayan_2016, Kerestes_2019, Liu_2005, Ohno_2005, Peixoto_2017f, Yildiz_2016e, teWildt_2010}, 4 cases (6\%) were psychiatric inpatients \cite{DivsalarP._2023a, fjbuilsRepeatedBehaviorDeliberate2024, teWildt_2010}, 3 cases (4\%) were under the influence of alcohol \cite{Benoist_2019e, Csaky_1998e, Thapa_2019f}, 2 cases (3\%) were displaced people \cite{Akay_2015f, Gardner_2017h}. \paragraph*{Motivation} 34 cases (47\%) had a psychiatric motivation \cite{Al-Faham_2020k, Alao_2006i, Ali_2020f, Apikotoa_2022f, Ataya_2013, Atayan_2016, Bhasin_2014, Bhattacharjee_2008, DelgadoSalazar_2020c, DivsalarP._2023a, Emamhadi_2018, Farhadi_2024h, Guinan_2019f, Hardy_2023g, Jehangir_2019h, Jin_2023, Kar_2015, Kariholu_2008, Kerestes_2019, Kobiela_2015, Kumar_2001, Kumar_2019f, Li_2013, Liu_2005, Misra_2013, Ohno_2005, Sakellaridis_2008f, Sultan_2024f, Tammana_2012j, Tanrikulu_2015e, Yasin_2009, teWildt_2010}, 21 cases (29\%) were motivated by self-harm intention \cite{Al-Faham_2020k, AlShaaibi_2021b, Alao_2006i, Ali_2017, CamachoDorado_2018, Chang_2017f, Cox_2007, Csaky_1998e, Fry_2010, Li_2013, Losanoff_1996, Losanoff_1997e, Mesfin_2022a, Sakellaridis_2008f, Tammana_2012j, Tanrikulu_2015e, fjbuilsRepeatedBehaviorDeliberate2024}, 17 cases (24\%) had a psychosocial motivation \cite{Akay_2015f, Benoist_2019e, Bhattacharjee_2008, Cauchi_2002, Goldman_1998f, Hardy_2023g, Kobiela_2015, Li_2013, Naji_2012f, Qureshi_2016, Riva_2018j, Sobnach_2011f, Tay_2004, Thapa_2019f, Tupesis_2004f, Wildhaber_2005, Wnęk_2015f}, 9 cases (12\%) were motivated by protest \cite{Bhumi_2024f, Gardner_2017h, Losanoff_1996, Losanoff_1997e, Tupesis_2004f}, 9 cases (12\%) had another documented motivation \cite{Ali_2020f, Ali_2022g, Emamhadi_2018, Guinan_2019f, Peixoto_2017f, Sakellaridis_2008f, Trgo_2012f, Wadhwa_2015e, Yildiz_2016e}. \paragraph*{Object Characteristics} 51 cases (71\%) ingested a large diameter object (\textgreater{}2.5cm) \cite{Akay_2015f, Al-Faham_2020k, AlShaaibi_2021b, Alao_2006i, Ali_2017, Ali_2022g, Apikotoa_2022f, Atayan_2016, Berry_2021e, Bhasin_2014, CamachoDorado_2018, Cauchi_2002, Chang_2017f, Cox_2007, Csaky_1998e, DivsalarP._2023a, Emamhadi_2018, Gardner_2017h, Guinan_2019f, Jehangir_2019h, Jin_2023, Kariholu_2008, Kerestes_2019, Kobiela_2015, Kumar_2001, Kumar_2019f, Losanoff_1996, Losanoff_1997e, Mesfin_2022a, Misra_2013, Naji_2012f, Ohno_2005, Peixoto_2017f, Qureshi_2016, Riva_2018j, Sakellaridis_2008f, Sultan_2024f, Tanrikulu_2015e, Thapa_2019f, Trgo_2012f, Wnęk_2015f, Yildiz_2016e, fjbuilsRepeatedBehaviorDeliberate2024, teWildt_2010}, 44 cases (61\%) ingested multiple objects \cite{Ali_2020f, Apikotoa_2022f, Ataya_2013, Atayan_2016, Beecroft_1998, Bhattacharjee_2008, Bhumi_2024f, CamachoDorado_2018, Cauchi_2002, Emamhadi_2018, Farhadi_2024h, Fry_2010, Goldman_1998f, Guinan_2019f, Hardy_2023g, Jehangir_2019h, Jin_2023, Kar_2015, Kariholu_2008, Kobiela_2015, Kumar_2001, Kumar_2019f, Li_2013, Liu_2005, Losanoff_1996, Mesfin_2022a, Misra_2013, Naji_2012f, Ohno_2005, Sobnach_2011f, Sultan_2024f, Tammana_2012j, Tanrikulu_2015e, Tay_2004, Thapa_2019f, Wadhwa_2015e, Wildhaber_2005, Yasin_2009, fjbuilsRepeatedBehaviorDeliberate2024, teWildt_2010}, 34 cases (47\%) ingested a sharp object \cite{AlShaaibi_2021b, Alao_2006i, Apikotoa_2022f, Ataya_2013, Benoist_2019e, Bhasin_2014, Bhattacharjee_2008, CamachoDorado_2018, Csaky_1998e, DelgadoSalazar_2020c, DivsalarP._2023a, Emamhadi_2018, Farhadi_2024h, Fry_2010, Guinan_2019f, Hardy_2023g, Jehangir_2019h, Jin_2023, Kariholu_2008, Kobiela_2015, Kumar_2019f, Losanoff_1996, Losanoff_1997e, Mesfin_2022a, Misra_2013, Sobnach_2011f, Yasin_2009, teWildt_2010}, 32 cases (44\%) ingested a long object (\textgreater{}5cm) \cite{Al-Faham_2020k, AlShaaibi_2021b, Ali_2017, Ali_2022g, Atayan_2016, Bhasin_2014, CamachoDorado_2018, Chang_2017f, Cox_2007, Csaky_1998e, DivsalarP._2023a, Emamhadi_2018, Fry_2010, Gardner_2017h, Jin_2023, Kariholu_2008, Kerestes_2019, Kobiela_2015, Kumar_2019f, Mesfin_2022a, Misra_2013, Ohno_2005, Qureshi_2016, Sakellaridis_2008f, Sultan_2024f, Thapa_2019f, Trgo_2012f, Yasin_2009, Yildiz_2016e, teWildt_2010}, 9 cases (12\%) ingested a magnet \cite{Ali_2020f, Bhumi_2024f, Cauchi_2002, Liu_2005, Naji_2012f, Ohno_2005, Tanrikulu_2015e, Tay_2004, Wildhaber_2005}, 2 cases (3\%) ingested a button battery \cite{Berry_2021e, Bhumi_2024f}. \paragraph*{Outcomes} 48 cases (67\%) experienced a complication \cite{Ali_2017, Ali_2020f, Apikotoa_2022f, Atayan_2016, Beecroft_1998, Benoist_2019e, Berry_2021e, Bhasin_2014, Bhumi_2024f, CamachoDorado_2018, Cauchi_2002, Cox_2007, Csaky_1998e, DelgadoSalazar_2020c, DivsalarP._2023a, Emamhadi_2018, Farhadi_2024h, Fry_2010, Gardner_2017h, Goldman_1998f, Jin_2023, Kariholu_2008, Kerestes_2019, Kobiela_2015, Kumar_2001, Kumar_2019f, Liu_2005, Losanoff_1996, Mesfin_2022a, Misra_2013, Naji_2012f, Ohno_2005, Sakellaridis_2008f, Sobnach_2011f, Sultan_2024f, Tanrikulu_2015e, Tay_2004, Thapa_2019f, Trgo_2012f, Tupesis_2004f, Wildhaber_2005, Wnęk_2015f, Yasin_2009, Yildiz_2016e}, 44 cases (61\%) underwent surgery \cite{Al-Faham_2020k, AlShaaibi_2021b, Alao_2006i, Ali_2017, Ali_2020f, Atayan_2016, Beecroft_1998, Bhasin_2014, CamachoDorado_2018, Cauchi_2002, Chang_2017f, Cox_2007, Csaky_1998e, DelgadoSalazar_2020c, DivsalarP._2023a, Farhadi_2024h, Fry_2010, Gardner_2017h, Jin_2023, Kariholu_2008, Kerestes_2019, Kobiela_2015, Kumar_2019f, Liu_2005, Losanoff_1996, Losanoff_1997e, Mesfin_2022a, Misra_2013, Naji_2012f, Sobnach_2011f, Tanrikulu_2015e, Tay_2004, Thapa_2019f, Tupesis_2004f, Wildhaber_2005, Wnęk_2015f, Yasin_2009, Yildiz_2016e, fjbuilsRepeatedBehaviorDeliberate2024}, 31 cases (43\%) underwent endoscopy \cite{Akay_2015f, Ali_2022g, Apikotoa_2022f, Atayan_2016, Benoist_2019e, Berry_2021e, Bhasin_2014, Bhumi_2024f, CamachoDorado_2018, Chang_2017f, DelgadoSalazar_2020c, Gardner_2017h, Guinan_2019f, Hardy_2023g, Jehangir_2019h, Kariholu_2008, Li_2013, Liu_2005, Ohno_2005, Peixoto_2017f, Qureshi_2016, Riva_2018j, Sakellaridis_2008f, Sultan_2024f, Tammana_2012j, Tanrikulu_2015e, Trgo_2012f, Wadhwa_2015e, Wnęk_2015f, teWildt_2010}, 7 cases (10\%) were managed conservatively \cite{Ataya_2013, Bhattacharjee_2008, DivsalarP._2023a, Emamhadi_2018, Goldman_1998f, Kar_2015, Kumar_2001}, 2 cases (3\%) died \cite{Emamhadi_2018, Kumar_2001}. All 90 were male gender. 90 cases (100\%) were detained at the time of ingestion \cite{Elghali_2016, Karp_1991b, Lee_2007}, 88 cases (98\%) were intentional ingestions \cite{Elghali_2016, Karp_1991b, Lee_2007}, 30 cases (33\%) had a psychiatric history documented \cite{Elghali_2016, Karp_1991b, Lee_2007}, 2 cases (2\%) had a history of prior ingestion \cite{Elghali_2016}. No cases were reported for were psychiatric inpatients, were displaced people, were under the influence of alcohol at the time of ingestion, and had a severe disability history.
\paragraph*{Motivation}  70 cases (78\%) reported protest motivation \cite{Elghali_2016, Karp_1991b, Lee_2007}, 12 cases (13\%) reported psychiatric motivation \cite{Karp_1991b}, 6 cases (7\%) reported self-harm motivation \cite{Elghali_2016, Karp_1991b}. No cases were reported for psychosocial motivation and other motivation.
\paragraph*{Object Characteristics}  68 cases (76\%) involved sharp object ingestion \cite{Elghali_2016, Karp_1991b, Lee_2007}, 32 cases (36\%) involved long (\textgreater 5cm) object ingestion \cite{Lee_2007}, 25 cases (28\%) involved ingestion of multiple objects \cite{Elghali_2016, Lee_2007}. No cases were reported for button battery ingestion, magnet ingestion, and involved large diameter (\textgreater 2.5cm) object ingestion.
\paragraph*{Outcomes}  47 cases (52\%) underwent endoscopic intervention \cite{Elghali_2016, Lee_2007}, 29 cases (32\%) were managed conservatively \cite{Elghali_2016, Karp_1991b}, 15 cases (17\%) underwent surgical intervention \cite{Elghali_2016, Karp_1991b, Lee_2007}, 6 cases (7\%) reported complications \cite{Lee_2007}, 1 case (1\%) died \cite{Elghali_2016}.
\paragraph*{Geographical Location}Cases were recorded in 33 countries: 13 cases from USA \cite{Alao_2006i, Ataya_2013, Bhumi_2024f, Fry_2010, Guinan_2019f, Hardy_2023g, Jehangir_2019h, Kerestes_2019, Kumar_2001, Liu_2005, Tammana_2012j, Tay_2004, Tupesis_2004f}; 7 cases from India \cite{Bhasin_2014, Bhattacharjee_2008, Kar_2015, Kariholu_2008, Kumar_2019f, Misra_2013, Wadhwa_2015e} and UK \cite{Beecroft_1998, Berry_2021e, Cauchi_2002, Cox_2007, Gardner_2017h, Qureshi_2016}; 6 cases from Bulgaria \cite{Losanoff_1996, Losanoff_1997e}; 5 cases from Iran \cite{DivsalarP._2023a, Emamhadi_2018, Farhadi_2024h}; 4 cases from Turkey \cite{Akay_2015f, Atayan_2016, Tanrikulu_2015e, Yildiz_2016e}; 2 cases from China \cite{Jin_2023, Li_2013}, Poland \cite{Kobiela_2015, Wnęk_2015f}, and Spain \cite{CamachoDorado_2018, fjbuilsRepeatedBehaviorDeliberate2024}; 1 case from Australia \cite{Apikotoa_2022f}, Bahrain \cite{Ali_2020f}, Croatia \cite{Trgo_2012f}, Ecuador \cite{DelgadoSalazar_2020c}, Egypt \cite{Ali_2022g}, Ethiopia \cite{Mesfin_2022a}, Germany \cite{teWildt_2010}, Greece \cite{Sakellaridis_2008f}, Hungary \cite{Csaky_1998e}, Iraq \cite{Al-Faham_2020k}, Israel \cite{Goldman_1998f}, Italy \cite{Riva_2018j}, Japan \cite{Ohno_2005}, Nepal \cite{Thapa_2019f}, Netherlands \cite{Benoist_2019e}, Oman \cite{AlShaaibi_2021b}, Pakistan \cite{Yasin_2009}, Portugal \cite{Peixoto_2017f}, Qatar \cite{Ali_2017}, Saudi Arabia \cite{Sultan_2024f}, South Africa \cite{Sobnach_2011f}, Sweden \cite{Naji_2012f}, Switzerland \cite{Wildhaber_2005}, and Taiwan \cite{Chang_2017f}. \paragraph*{Gender} 43 cases (60\%) were male \cite{Akay_2015f, Al-Faham_2020k, Alao_2006i, Ali_2017, Ali_2022g, Apikotoa_2022f, Atayan_2016, Benoist_2019e, Berry_2021e, Bhumi_2024f, CamachoDorado_2018, Csaky_1998e, Emamhadi_2018, Farhadi_2024h, Fry_2010, Gardner_2017h, Guinan_2019f, Jehangir_2019h, Jin_2023, Kobiela_2015, Kumar_2001, Kumar_2019f, Liu_2005, Losanoff_1996, Losanoff_1997e, Mesfin_2022a, Misra_2013, Qureshi_2016, Riva_2018j, Sobnach_2011f, Tammana_2012j, Tanrikulu_2015e, Tay_2004, Thapa_2019f, Trgo_2012f, Wadhwa_2015e, Yasin_2009, teWildt_2010}, 28 cases (39\%) were female \cite{AlShaaibi_2021b, Ali_2020f, Ataya_2013, Beecroft_1998, Bhasin_2014, Bhattacharjee_2008, Cauchi_2002, Chang_2017f, Cox_2007, DelgadoSalazar_2020c, DivsalarP._2023a, Goldman_1998f, Hardy_2023g, Kar_2015, Kariholu_2008, Kerestes_2019, Li_2013, Naji_2012f, Ohno_2005, Peixoto_2017f, Sakellaridis_2008f, Sultan_2024f, Tupesis_2004f, Wildhaber_2005, Wnęk_2015f, Yildiz_2016e}, 1 case (1\%) had no gender recorded \cite{fjbuilsRepeatedBehaviorDeliberate2024}. \paragraph*{Age Group} 25 cases (35\%) were between 26 and 40 years of age \cite{Alao_2006i, Ali_2022g, Apikotoa_2022f, Ataya_2013, Benoist_2019e, Bhasin_2014, Chang_2017f, Cox_2007, DelgadoSalazar_2020c, Farhadi_2024h, Fry_2010, Gardner_2017h, Guinan_2019f, Jin_2023, Kumar_2019f, Losanoff_1996, Misra_2013, Qureshi_2016, Riva_2018j, Sakellaridis_2008f, Tammana_2012j, Trgo_2012f, Wnęk_2015f, Yildiz_2016e, fjbuilsRepeatedBehaviorDeliberate2024}, 18 cases (25\%) were between 18 and 25 years of age \cite{Akay_2015f, Ali_2017, Atayan_2016, Bhattacharjee_2008, Csaky_1998e, Kar_2015, Kariholu_2008, Kobiela_2015, Losanoff_1996, Losanoff_1997e, Mesfin_2022a, Peixoto_2017f, Sobnach_2011f, Tupesis_2004f, Yasin_2009}, 13 cases (18\%) were under 18 years of age \cite{AlShaaibi_2021b, Ali_2020f, Cauchi_2002, DivsalarP._2023a, Goldman_1998f, Liu_2005, Naji_2012f, Ohno_2005, Tanrikulu_2015e, Tay_2004, Wildhaber_2005}, 11 cases (15\%) were between 41 and 60 years of age \cite{Al-Faham_2020k, Bhumi_2024f, CamachoDorado_2018, Emamhadi_2018, Hardy_2023g, Jehangir_2019h, Kumar_2001, Sultan_2024f, Thapa_2019f, Wadhwa_2015e, teWildt_2010}, 3 cases (4\%) were over 60 years of age \cite{Beecroft_1998, Kerestes_2019, Li_2013}, 2 cases (3\%) had no age documented \cite{Berry_2021e}. \paragraph*{Population} 36 cases (50\%) had a psychiatric history \cite{AlShaaibi_2021b, Alao_2006i, Ali_2020f, Apikotoa_2022f, Ataya_2013, Atayan_2016, Beecroft_1998, CamachoDorado_2018, Chang_2017f, DelgadoSalazar_2020c, DivsalarP._2023a, Farhadi_2024h, Fry_2010, Guinan_2019f, Hardy_2023g, Jehangir_2019h, Jin_2023, Kar_2015, Kerestes_2019, Kobiela_2015, Kumar_2001, Kumar_2019f, Liu_2005, Mesfin_2022a, Misra_2013, Ohno_2005, Peixoto_2017f, Sakellaridis_2008f, Sultan_2024f, Tammana_2012j, Tanrikulu_2015e, Yildiz_2016e, fjbuilsRepeatedBehaviorDeliberate2024, teWildt_2010}, 19 cases (26\%) had ingested previously \cite{Alao_2006i, Apikotoa_2022f, Berry_2021e, Bhattacharjee_2008, Csaky_1998e, DivsalarP._2023a, Emamhadi_2018, Guinan_2019f, Jehangir_2019h, Jin_2023, Liu_2005, Sakellaridis_2008f, Tanrikulu_2015e, Thapa_2019f, Yildiz_2016e, fjbuilsRepeatedBehaviorDeliberate2024, teWildt_2010}, 12 cases (17\%) were detained persons \cite{Alao_2006i, Ali_2022g, Apikotoa_2022f, Losanoff_1996, Losanoff_1997e, Qureshi_2016, Tammana_2012j, Trgo_2012f}, 7 cases (10\%) were severely disabled \cite{Atayan_2016, Kerestes_2019, Liu_2005, Ohno_2005, Peixoto_2017f, Yildiz_2016e, teWildt_2010}, 4 cases (6\%) were psychiatric inpatients \cite{DivsalarP._2023a, fjbuilsRepeatedBehaviorDeliberate2024, teWildt_2010}, 3 cases (4\%) were under the influence of alcohol \cite{Benoist_2019e, Csaky_1998e, Thapa_2019f}, 2 cases (3\%) were displaced people \cite{Akay_2015f, Gardner_2017h}. \paragraph*{Motivation} 34 cases (47\%) had a psychiatric motivation \cite{Al-Faham_2020k, Alao_2006i, Ali_2020f, Apikotoa_2022f, Ataya_2013, Atayan_2016, Bhasin_2014, Bhattacharjee_2008, DelgadoSalazar_2020c, DivsalarP._2023a, Emamhadi_2018, Farhadi_2024h, Guinan_2019f, Hardy_2023g, Jehangir_2019h, Jin_2023, Kar_2015, Kariholu_2008, Kerestes_2019, Kobiela_2015, Kumar_2001, Kumar_2019f, Li_2013, Liu_2005, Misra_2013, Ohno_2005, Sakellaridis_2008f, Sultan_2024f, Tammana_2012j, Tanrikulu_2015e, Yasin_2009, teWildt_2010}, 21 cases (29\%) were motivated by self-harm intention \cite{Al-Faham_2020k, AlShaaibi_2021b, Alao_2006i, Ali_2017, CamachoDorado_2018, Chang_2017f, Cox_2007, Csaky_1998e, Fry_2010, Li_2013, Losanoff_1996, Losanoff_1997e, Mesfin_2022a, Sakellaridis_2008f, Tammana_2012j, Tanrikulu_2015e, fjbuilsRepeatedBehaviorDeliberate2024}, 17 cases (24\%) had a psychosocial motivation \cite{Akay_2015f, Benoist_2019e, Bhattacharjee_2008, Cauchi_2002, Goldman_1998f, Hardy_2023g, Kobiela_2015, Li_2013, Naji_2012f, Qureshi_2016, Riva_2018j, Sobnach_2011f, Tay_2004, Thapa_2019f, Tupesis_2004f, Wildhaber_2005, Wnęk_2015f}, 9 cases (12\%) were motivated by protest \cite{Bhumi_2024f, Gardner_2017h, Losanoff_1996, Losanoff_1997e, Tupesis_2004f}, 9 cases (12\%) had another documented motivation \cite{Ali_2020f, Ali_2022g, Emamhadi_2018, Guinan_2019f, Peixoto_2017f, Sakellaridis_2008f, Trgo_2012f, Wadhwa_2015e, Yildiz_2016e}. \paragraph*{Object Characteristics} 51 cases (71\%) ingested a large diameter object (\textgreater{}2.5cm) \cite{Akay_2015f, Al-Faham_2020k, AlShaaibi_2021b, Alao_2006i, Ali_2017, Ali_2022g, Apikotoa_2022f, Atayan_2016, Berry_2021e, Bhasin_2014, CamachoDorado_2018, Cauchi_2002, Chang_2017f, Cox_2007, Csaky_1998e, DivsalarP._2023a, Emamhadi_2018, Gardner_2017h, Guinan_2019f, Jehangir_2019h, Jin_2023, Kariholu_2008, Kerestes_2019, Kobiela_2015, Kumar_2001, Kumar_2019f, Losanoff_1996, Losanoff_1997e, Mesfin_2022a, Misra_2013, Naji_2012f, Ohno_2005, Peixoto_2017f, Qureshi_2016, Riva_2018j, Sakellaridis_2008f, Sultan_2024f, Tanrikulu_2015e, Thapa_2019f, Trgo_2012f, Wnęk_2015f, Yildiz_2016e, fjbuilsRepeatedBehaviorDeliberate2024, teWildt_2010}, 44 cases (61\%) ingested multiple objects \cite{Ali_2020f, Apikotoa_2022f, Ataya_2013, Atayan_2016, Beecroft_1998, Bhattacharjee_2008, Bhumi_2024f, CamachoDorado_2018, Cauchi_2002, Emamhadi_2018, Farhadi_2024h, Fry_2010, Goldman_1998f, Guinan_2019f, Hardy_2023g, Jehangir_2019h, Jin_2023, Kar_2015, Kariholu_2008, Kobiela_2015, Kumar_2001, Kumar_2019f, Li_2013, Liu_2005, Losanoff_1996, Mesfin_2022a, Misra_2013, Naji_2012f, Ohno_2005, Sobnach_2011f, Sultan_2024f, Tammana_2012j, Tanrikulu_2015e, Tay_2004, Thapa_2019f, Wadhwa_2015e, Wildhaber_2005, Yasin_2009, fjbuilsRepeatedBehaviorDeliberate2024, teWildt_2010}, 34 cases (47\%) ingested a sharp object \cite{AlShaaibi_2021b, Alao_2006i, Apikotoa_2022f, Ataya_2013, Benoist_2019e, Bhasin_2014, Bhattacharjee_2008, CamachoDorado_2018, Csaky_1998e, DelgadoSalazar_2020c, DivsalarP._2023a, Emamhadi_2018, Farhadi_2024h, Fry_2010, Guinan_2019f, Hardy_2023g, Jehangir_2019h, Jin_2023, Kariholu_2008, Kobiela_2015, Kumar_2019f, Losanoff_1996, Losanoff_1997e, Mesfin_2022a, Misra_2013, Sobnach_2011f, Yasin_2009, teWildt_2010}, 32 cases (44\%) ingested a long object (\textgreater{}5cm) \cite{Al-Faham_2020k, AlShaaibi_2021b, Ali_2017, Ali_2022g, Atayan_2016, Bhasin_2014, CamachoDorado_2018, Chang_2017f, Cox_2007, Csaky_1998e, DivsalarP._2023a, Emamhadi_2018, Fry_2010, Gardner_2017h, Jin_2023, Kariholu_2008, Kerestes_2019, Kobiela_2015, Kumar_2019f, Mesfin_2022a, Misra_2013, Ohno_2005, Qureshi_2016, Sakellaridis_2008f, Sultan_2024f, Thapa_2019f, Trgo_2012f, Yasin_2009, Yildiz_2016e, teWildt_2010}, 9 cases (12\%) ingested a magnet \cite{Ali_2020f, Bhumi_2024f, Cauchi_2002, Liu_2005, Naji_2012f, Ohno_2005, Tanrikulu_2015e, Tay_2004, Wildhaber_2005}, 2 cases (3\%) ingested a button battery \cite{Berry_2021e, Bhumi_2024f}. \paragraph*{Outcomes} 48 cases (67\%) experienced a complication \cite{Ali_2017, Ali_2020f, Apikotoa_2022f, Atayan_2016, Beecroft_1998, Benoist_2019e, Berry_2021e, Bhasin_2014, Bhumi_2024f, CamachoDorado_2018, Cauchi_2002, Cox_2007, Csaky_1998e, DelgadoSalazar_2020c, DivsalarP._2023a, Emamhadi_2018, Farhadi_2024h, Fry_2010, Gardner_2017h, Goldman_1998f, Jin_2023, Kariholu_2008, Kerestes_2019, Kobiela_2015, Kumar_2001, Kumar_2019f, Liu_2005, Losanoff_1996, Mesfin_2022a, Misra_2013, Naji_2012f, Ohno_2005, Sakellaridis_2008f, Sobnach_2011f, Sultan_2024f, Tanrikulu_2015e, Tay_2004, Thapa_2019f, Trgo_2012f, Tupesis_2004f, Wildhaber_2005, Wnęk_2015f, Yasin_2009, Yildiz_2016e}, 44 cases (61\%) underwent surgery \cite{Al-Faham_2020k, AlShaaibi_2021b, Alao_2006i, Ali_2017, Ali_2020f, Atayan_2016, Beecroft_1998, Bhasin_2014, CamachoDorado_2018, Cauchi_2002, Chang_2017f, Cox_2007, Csaky_1998e, DelgadoSalazar_2020c, DivsalarP._2023a, Farhadi_2024h, Fry_2010, Gardner_2017h, Jin_2023, Kariholu_2008, Kerestes_2019, Kobiela_2015, Kumar_2019f, Liu_2005, Losanoff_1996, Losanoff_1997e, Mesfin_2022a, Misra_2013, Naji_2012f, Sobnach_2011f, Tanrikulu_2015e, Tay_2004, Thapa_2019f, Tupesis_2004f, Wildhaber_2005, Wnęk_2015f, Yasin_2009, Yildiz_2016e, fjbuilsRepeatedBehaviorDeliberate2024}, 31 cases (43\%) underwent endoscopy \cite{Akay_2015f, Ali_2022g, Apikotoa_2022f, Atayan_2016, Benoist_2019e, Berry_2021e, Bhasin_2014, Bhumi_2024f, CamachoDorado_2018, Chang_2017f, DelgadoSalazar_2020c, Gardner_2017h, Guinan_2019f, Hardy_2023g, Jehangir_2019h, Kariholu_2008, Li_2013, Liu_2005, Ohno_2005, Peixoto_2017f, Qureshi_2016, Riva_2018j, Sakellaridis_2008f, Sultan_2024f, Tammana_2012j, Tanrikulu_2015e, Trgo_2012f, Wadhwa_2015e, Wnęk_2015f, teWildt_2010}, 7 cases (10\%) were managed conservatively \cite{Ataya_2013, Bhattacharjee_2008, DivsalarP._2023a, Emamhadi_2018, Goldman_1998f, Kar_2015, Kumar_2001}, 2 cases (3\%) died \cite{Emamhadi_2018, Kumar_2001}. All 90 were male gender. 90 cases (100\%) were detained at the time of ingestion \cite{Elghali_2016, Karp_1991b, Lee_2007}, 88 cases (98\%) were intentional ingestions \cite{Elghali_2016, Karp_1991b, Lee_2007}, 30 cases (33\%) had a psychiatric history documented \cite{Elghali_2016, Karp_1991b, Lee_2007}, 2 cases (2\%) had a history of prior ingestion \cite{Elghali_2016}. No cases were reported for were psychiatric inpatients, were displaced people, were under the influence of alcohol at the time of ingestion, and had a severe disability history.
\paragraph*{Motivation}  70 cases (78\%) reported protest motivation \cite{Elghali_2016, Karp_1991b, Lee_2007}, 12 cases (13\%) reported psychiatric motivation \cite{Karp_1991b}, 6 cases (7\%) reported self-harm motivation \cite{Elghali_2016, Karp_1991b}. No cases were reported for psychosocial motivation and other motivation.
\paragraph*{Object Characteristics}  68 cases (76\%) involved sharp object ingestion \cite{Elghali_2016, Karp_1991b, Lee_2007}, 32 cases (36\%) involved long (\textgreater 5cm) object ingestion \cite{Lee_2007}, 25 cases (28\%) involved ingestion of multiple objects \cite{Elghali_2016, Lee_2007}. No cases were reported for button battery ingestion, magnet ingestion, and involved large diameter (\textgreater 2.5cm) object ingestion.
\paragraph*{Outcomes}  47 cases (52\%) underwent endoscopic intervention \cite{Elghali_2016, Lee_2007}, 29 cases (32\%) were managed conservatively \cite{Elghali_2016, Karp_1991b}, 15 cases (17\%) underwent surgical intervention \cite{Elghali_2016, Karp_1991b, Lee_2007}, 6 cases (7\%) reported complications \cite{Lee_2007}, 1 case (1\%) died \cite{Elghali_2016}.
\paragraph*{Geographical Location}Cases were recorded in 33 countries: 13 cases from USA \cite{Alao_2006i, Ataya_2013, Bhumi_2024f, Fry_2010, Guinan_2019f, Hardy_2023g, Jehangir_2019h, Kerestes_2019, Kumar_2001, Liu_2005, Tammana_2012j, Tay_2004, Tupesis_2004f}; 7 cases from India \cite{Bhasin_2014, Bhattacharjee_2008, Kar_2015, Kariholu_2008, Kumar_2019f, Misra_2013, Wadhwa_2015e} and UK \cite{Beecroft_1998, Berry_2021e, Cauchi_2002, Cox_2007, Gardner_2017h, Qureshi_2016}; 6 cases from Bulgaria \cite{Losanoff_1996, Losanoff_1997e}; 5 cases from Iran \cite{DivsalarP._2023a, Emamhadi_2018, Farhadi_2024h}; 4 cases from Turkey \cite{Akay_2015f, Atayan_2016, Tanrikulu_2015e, Yildiz_2016e}; 2 cases from China \cite{Jin_2023, Li_2013}, Poland \cite{Kobiela_2015, Wnęk_2015f}, and Spain \cite{CamachoDorado_2018, fjbuilsRepeatedBehaviorDeliberate2024}; 1 case from Australia \cite{Apikotoa_2022f}, Bahrain \cite{Ali_2020f}, Croatia \cite{Trgo_2012f}, Ecuador \cite{DelgadoSalazar_2020c}, Egypt \cite{Ali_2022g}, Ethiopia \cite{Mesfin_2022a}, Germany \cite{teWildt_2010}, Greece \cite{Sakellaridis_2008f}, Hungary \cite{Csaky_1998e}, Iraq \cite{Al-Faham_2020k}, Israel \cite{Goldman_1998f}, Italy \cite{Riva_2018j}, Japan \cite{Ohno_2005}, Nepal \cite{Thapa_2019f}, Netherlands \cite{Benoist_2019e}, Oman \cite{AlShaaibi_2021b}, Pakistan \cite{Yasin_2009}, Portugal \cite{Peixoto_2017f}, Qatar \cite{Ali_2017}, Saudi Arabia \cite{Sultan_2024f}, South Africa \cite{Sobnach_2011f}, Sweden \cite{Naji_2012f}, Switzerland \cite{Wildhaber_2005}, and Taiwan \cite{Chang_2017f}. \paragraph*{Gender} 43 cases (60\%) were male \cite{Akay_2015f, Al-Faham_2020k, Alao_2006i, Ali_2017, Ali_2022g, Apikotoa_2022f, Atayan_2016, Benoist_2019e, Berry_2021e, Bhumi_2024f, CamachoDorado_2018, Csaky_1998e, Emamhadi_2018, Farhadi_2024h, Fry_2010, Gardner_2017h, Guinan_2019f, Jehangir_2019h, Jin_2023, Kobiela_2015, Kumar_2001, Kumar_2019f, Liu_2005, Losanoff_1996, Losanoff_1997e, Mesfin_2022a, Misra_2013, Qureshi_2016, Riva_2018j, Sobnach_2011f, Tammana_2012j, Tanrikulu_2015e, Tay_2004, Thapa_2019f, Trgo_2012f, Wadhwa_2015e, Yasin_2009, teWildt_2010}, 28 cases (39\%) were female \cite{AlShaaibi_2021b, Ali_2020f, Ataya_2013, Beecroft_1998, Bhasin_2014, Bhattacharjee_2008, Cauchi_2002, Chang_2017f, Cox_2007, DelgadoSalazar_2020c, DivsalarP._2023a, Goldman_1998f, Hardy_2023g, Kar_2015, Kariholu_2008, Kerestes_2019, Li_2013, Naji_2012f, Ohno_2005, Peixoto_2017f, Sakellaridis_2008f, Sultan_2024f, Tupesis_2004f, Wildhaber_2005, Wnęk_2015f, Yildiz_2016e}, 1 case (1\%) had no gender recorded \cite{fjbuilsRepeatedBehaviorDeliberate2024}. \paragraph*{Age Group} 25 cases (35\%) were between 26 and 40 years of age \cite{Alao_2006i, Ali_2022g, Apikotoa_2022f, Ataya_2013, Benoist_2019e, Bhasin_2014, Chang_2017f, Cox_2007, DelgadoSalazar_2020c, Farhadi_2024h, Fry_2010, Gardner_2017h, Guinan_2019f, Jin_2023, Kumar_2019f, Losanoff_1996, Misra_2013, Qureshi_2016, Riva_2018j, Sakellaridis_2008f, Tammana_2012j, Trgo_2012f, Wnęk_2015f, Yildiz_2016e, fjbuilsRepeatedBehaviorDeliberate2024}, 18 cases (25\%) were between 18 and 25 years of age \cite{Akay_2015f, Ali_2017, Atayan_2016, Bhattacharjee_2008, Csaky_1998e, Kar_2015, Kariholu_2008, Kobiela_2015, Losanoff_1996, Losanoff_1997e, Mesfin_2022a, Peixoto_2017f, Sobnach_2011f, Tupesis_2004f, Yasin_2009}, 13 cases (18\%) were under 18 years of age \cite{AlShaaibi_2021b, Ali_2020f, Cauchi_2002, DivsalarP._2023a, Goldman_1998f, Liu_2005, Naji_2012f, Ohno_2005, Tanrikulu_2015e, Tay_2004, Wildhaber_2005}, 11 cases (15\%) were between 41 and 60 years of age \cite{Al-Faham_2020k, Bhumi_2024f, CamachoDorado_2018, Emamhadi_2018, Hardy_2023g, Jehangir_2019h, Kumar_2001, Sultan_2024f, Thapa_2019f, Wadhwa_2015e, teWildt_2010}, 3 cases (4\%) were over 60 years of age \cite{Beecroft_1998, Kerestes_2019, Li_2013}, 2 cases (3\%) had no age documented \cite{Berry_2021e}. \paragraph*{Population} 36 cases (50\%) had a psychiatric history \cite{AlShaaibi_2021b, Alao_2006i, Ali_2020f, Apikotoa_2022f, Ataya_2013, Atayan_2016, Beecroft_1998, CamachoDorado_2018, Chang_2017f, DelgadoSalazar_2020c, DivsalarP._2023a, Farhadi_2024h, Fry_2010, Guinan_2019f, Hardy_2023g, Jehangir_2019h, Jin_2023, Kar_2015, Kerestes_2019, Kobiela_2015, Kumar_2001, Kumar_2019f, Liu_2005, Mesfin_2022a, Misra_2013, Ohno_2005, Peixoto_2017f, Sakellaridis_2008f, Sultan_2024f, Tammana_2012j, Tanrikulu_2015e, Yildiz_2016e, fjbuilsRepeatedBehaviorDeliberate2024, teWildt_2010}, 19 cases (26\%) had ingested previously \cite{Alao_2006i, Apikotoa_2022f, Berry_2021e, Bhattacharjee_2008, Csaky_1998e, DivsalarP._2023a, Emamhadi_2018, Guinan_2019f, Jehangir_2019h, Jin_2023, Liu_2005, Sakellaridis_2008f, Tanrikulu_2015e, Thapa_2019f, Yildiz_2016e, fjbuilsRepeatedBehaviorDeliberate2024, teWildt_2010}, 12 cases (17\%) were detained persons \cite{Alao_2006i, Ali_2022g, Apikotoa_2022f, Losanoff_1996, Losanoff_1997e, Qureshi_2016, Tammana_2012j, Trgo_2012f}, 7 cases (10\%) were severely disabled \cite{Atayan_2016, Kerestes_2019, Liu_2005, Ohno_2005, Peixoto_2017f, Yildiz_2016e, teWildt_2010}, 4 cases (6\%) were psychiatric inpatients \cite{DivsalarP._2023a, fjbuilsRepeatedBehaviorDeliberate2024, teWildt_2010}, 3 cases (4\%) were under the influence of alcohol \cite{Benoist_2019e, Csaky_1998e, Thapa_2019f}, 2 cases (3\%) were displaced people \cite{Akay_2015f, Gardner_2017h}. \paragraph*{Motivation} 34 cases (47\%) had a psychiatric motivation \cite{Al-Faham_2020k, Alao_2006i, Ali_2020f, Apikotoa_2022f, Ataya_2013, Atayan_2016, Bhasin_2014, Bhattacharjee_2008, DelgadoSalazar_2020c, DivsalarP._2023a, Emamhadi_2018, Farhadi_2024h, Guinan_2019f, Hardy_2023g, Jehangir_2019h, Jin_2023, Kar_2015, Kariholu_2008, Kerestes_2019, Kobiela_2015, Kumar_2001, Kumar_2019f, Li_2013, Liu_2005, Misra_2013, Ohno_2005, Sakellaridis_2008f, Sultan_2024f, Tammana_2012j, Tanrikulu_2015e, Yasin_2009, teWildt_2010}, 21 cases (29\%) were motivated by self-harm intention \cite{Al-Faham_2020k, AlShaaibi_2021b, Alao_2006i, Ali_2017, CamachoDorado_2018, Chang_2017f, Cox_2007, Csaky_1998e, Fry_2010, Li_2013, Losanoff_1996, Losanoff_1997e, Mesfin_2022a, Sakellaridis_2008f, Tammana_2012j, Tanrikulu_2015e, fjbuilsRepeatedBehaviorDeliberate2024}, 17 cases (24\%) had a psychosocial motivation \cite{Akay_2015f, Benoist_2019e, Bhattacharjee_2008, Cauchi_2002, Goldman_1998f, Hardy_2023g, Kobiela_2015, Li_2013, Naji_2012f, Qureshi_2016, Riva_2018j, Sobnach_2011f, Tay_2004, Thapa_2019f, Tupesis_2004f, Wildhaber_2005, Wnęk_2015f}, 9 cases (12\%) were motivated by protest \cite{Bhumi_2024f, Gardner_2017h, Losanoff_1996, Losanoff_1997e, Tupesis_2004f}, 9 cases (12\%) had another documented motivation \cite{Ali_2020f, Ali_2022g, Emamhadi_2018, Guinan_2019f, Peixoto_2017f, Sakellaridis_2008f, Trgo_2012f, Wadhwa_2015e, Yildiz_2016e}. \paragraph*{Object Characteristics} 51 cases (71\%) ingested a large diameter object (\textgreater{}2.5cm) \cite{Akay_2015f, Al-Faham_2020k, AlShaaibi_2021b, Alao_2006i, Ali_2017, Ali_2022g, Apikotoa_2022f, Atayan_2016, Berry_2021e, Bhasin_2014, CamachoDorado_2018, Cauchi_2002, Chang_2017f, Cox_2007, Csaky_1998e, DivsalarP._2023a, Emamhadi_2018, Gardner_2017h, Guinan_2019f, Jehangir_2019h, Jin_2023, Kariholu_2008, Kerestes_2019, Kobiela_2015, Kumar_2001, Kumar_2019f, Losanoff_1996, Losanoff_1997e, Mesfin_2022a, Misra_2013, Naji_2012f, Ohno_2005, Peixoto_2017f, Qureshi_2016, Riva_2018j, Sakellaridis_2008f, Sultan_2024f, Tanrikulu_2015e, Thapa_2019f, Trgo_2012f, Wnęk_2015f, Yildiz_2016e, fjbuilsRepeatedBehaviorDeliberate2024, teWildt_2010}, 44 cases (61\%) ingested multiple objects \cite{Ali_2020f, Apikotoa_2022f, Ataya_2013, Atayan_2016, Beecroft_1998, Bhattacharjee_2008, Bhumi_2024f, CamachoDorado_2018, Cauchi_2002, Emamhadi_2018, Farhadi_2024h, Fry_2010, Goldman_1998f, Guinan_2019f, Hardy_2023g, Jehangir_2019h, Jin_2023, Kar_2015, Kariholu_2008, Kobiela_2015, Kumar_2001, Kumar_2019f, Li_2013, Liu_2005, Losanoff_1996, Mesfin_2022a, Misra_2013, Naji_2012f, Ohno_2005, Sobnach_2011f, Sultan_2024f, Tammana_2012j, Tanrikulu_2015e, Tay_2004, Thapa_2019f, Wadhwa_2015e, Wildhaber_2005, Yasin_2009, fjbuilsRepeatedBehaviorDeliberate2024, teWildt_2010}, 34 cases (47\%) ingested a sharp object \cite{AlShaaibi_2021b, Alao_2006i, Apikotoa_2022f, Ataya_2013, Benoist_2019e, Bhasin_2014, Bhattacharjee_2008, CamachoDorado_2018, Csaky_1998e, DelgadoSalazar_2020c, DivsalarP._2023a, Emamhadi_2018, Farhadi_2024h, Fry_2010, Guinan_2019f, Hardy_2023g, Jehangir_2019h, Jin_2023, Kariholu_2008, Kobiela_2015, Kumar_2019f, Losanoff_1996, Losanoff_1997e, Mesfin_2022a, Misra_2013, Sobnach_2011f, Yasin_2009, teWildt_2010}, 32 cases (44\%) ingested a long object (\textgreater{}5cm) \cite{Al-Faham_2020k, AlShaaibi_2021b, Ali_2017, Ali_2022g, Atayan_2016, Bhasin_2014, CamachoDorado_2018, Chang_2017f, Cox_2007, Csaky_1998e, DivsalarP._2023a, Emamhadi_2018, Fry_2010, Gardner_2017h, Jin_2023, Kariholu_2008, Kerestes_2019, Kobiela_2015, Kumar_2019f, Mesfin_2022a, Misra_2013, Ohno_2005, Qureshi_2016, Sakellaridis_2008f, Sultan_2024f, Thapa_2019f, Trgo_2012f, Yasin_2009, Yildiz_2016e, teWildt_2010}, 9 cases (12\%) ingested a magnet \cite{Ali_2020f, Bhumi_2024f, Cauchi_2002, Liu_2005, Naji_2012f, Ohno_2005, Tanrikulu_2015e, Tay_2004, Wildhaber_2005}, 2 cases (3\%) ingested a button battery \cite{Berry_2021e, Bhumi_2024f}. \paragraph*{Outcomes} 48 cases (67\%) experienced a complication \cite{Ali_2017, Ali_2020f, Apikotoa_2022f, Atayan_2016, Beecroft_1998, Benoist_2019e, Berry_2021e, Bhasin_2014, Bhumi_2024f, CamachoDorado_2018, Cauchi_2002, Cox_2007, Csaky_1998e, DelgadoSalazar_2020c, DivsalarP._2023a, Emamhadi_2018, Farhadi_2024h, Fry_2010, Gardner_2017h, Goldman_1998f, Jin_2023, Kariholu_2008, Kerestes_2019, Kobiela_2015, Kumar_2001, Kumar_2019f, Liu_2005, Losanoff_1996, Mesfin_2022a, Misra_2013, Naji_2012f, Ohno_2005, Sakellaridis_2008f, Sobnach_2011f, Sultan_2024f, Tanrikulu_2015e, Tay_2004, Thapa_2019f, Trgo_2012f, Tupesis_2004f, Wildhaber_2005, Wnęk_2015f, Yasin_2009, Yildiz_2016e}, 44 cases (61\%) underwent surgery \cite{Al-Faham_2020k, AlShaaibi_2021b, Alao_2006i, Ali_2017, Ali_2020f, Atayan_2016, Beecroft_1998, Bhasin_2014, CamachoDorado_2018, Cauchi_2002, Chang_2017f, Cox_2007, Csaky_1998e, DelgadoSalazar_2020c, DivsalarP._2023a, Farhadi_2024h, Fry_2010, Gardner_2017h, Jin_2023, Kariholu_2008, Kerestes_2019, Kobiela_2015, Kumar_2019f, Liu_2005, Losanoff_1996, Losanoff_1997e, Mesfin_2022a, Misra_2013, Naji_2012f, Sobnach_2011f, Tanrikulu_2015e, Tay_2004, Thapa_2019f, Tupesis_2004f, Wildhaber_2005, Wnęk_2015f, Yasin_2009, Yildiz_2016e, fjbuilsRepeatedBehaviorDeliberate2024}, 31 cases (43\%) underwent endoscopy \cite{Akay_2015f, Ali_2022g, Apikotoa_2022f, Atayan_2016, Benoist_2019e, Berry_2021e, Bhasin_2014, Bhumi_2024f, CamachoDorado_2018, Chang_2017f, DelgadoSalazar_2020c, Gardner_2017h, Guinan_2019f, Hardy_2023g, Jehangir_2019h, Kariholu_2008, Li_2013, Liu_2005, Ohno_2005, Peixoto_2017f, Qureshi_2016, Riva_2018j, Sakellaridis_2008f, Sultan_2024f, Tammana_2012j, Tanrikulu_2015e, Trgo_2012f, Wadhwa_2015e, Wnęk_2015f, teWildt_2010}, 7 cases (10\%) were managed conservatively \cite{Ataya_2013, Bhattacharjee_2008, DivsalarP._2023a, Emamhadi_2018, Goldman_1998f, Kar_2015, Kumar_2001}, 2 cases (3\%) died \cite{Emamhadi_2018, Kumar_2001}. All 90 were male gender. 90 cases (100\%) were detained at the time of ingestion \cite{Elghali_2016, Karp_1991b, Lee_2007}, 88 cases (98\%) were intentional ingestions \cite{Elghali_2016, Karp_1991b, Lee_2007}, 30 cases (33\%) had a psychiatric history documented \cite{Elghali_2016, Karp_1991b, Lee_2007}, 2 cases (2\%) had a history of prior ingestion \cite{Elghali_2016}. No cases were reported for were psychiatric inpatients, were displaced people, were under the influence of alcohol at the time of ingestion, and had a severe disability history.
\paragraph*{Motivation}  70 cases (78\%) reported protest motivation \cite{Elghali_2016, Karp_1991b, Lee_2007}, 12 cases (13\%) reported psychiatric motivation \cite{Karp_1991b}, 6 cases (7\%) reported self-harm motivation \cite{Elghali_2016, Karp_1991b}. No cases were reported for psychosocial motivation and other motivation.
\paragraph*{Object Characteristics}  68 cases (76\%) involved sharp object ingestion \cite{Elghali_2016, Karp_1991b, Lee_2007}, 32 cases (36\%) involved long (\textgreater 5cm) object ingestion \cite{Lee_2007}, 25 cases (28\%) involved ingestion of multiple objects \cite{Elghali_2016, Lee_2007}. No cases were reported for button battery ingestion, magnet ingestion, and involved large diameter (\textgreater 2.5cm) object ingestion.
\paragraph*{Outcomes}  47 cases (52\%) underwent endoscopic intervention \cite{Elghali_2016, Lee_2007}, 29 cases (32\%) were managed conservatively \cite{Elghali_2016, Karp_1991b}, 15 cases (17\%) underwent surgical intervention \cite{Elghali_2016, Karp_1991b, Lee_2007}, 6 cases (7\%) reported complications \cite{Lee_2007}, 1 case (1\%) died \cite{Elghali_2016}.
\paragraph*{Geographical Location}Cases were recorded in 33 countries: 13 cases from USA \cite{Alao_2006i, Ataya_2013, Bhumi_2024f, Fry_2010, Guinan_2019f, Hardy_2023g, Jehangir_2019h, Kerestes_2019, Kumar_2001, Liu_2005, Tammana_2012j, Tay_2004, Tupesis_2004f}; 7 cases from India \cite{Bhasin_2014, Bhattacharjee_2008, Kar_2015, Kariholu_2008, Kumar_2019f, Misra_2013, Wadhwa_2015e} and UK \cite{Beecroft_1998, Berry_2021e, Cauchi_2002, Cox_2007, Gardner_2017h, Qureshi_2016}; 6 cases from Bulgaria \cite{Losanoff_1996, Losanoff_1997e}; 5 cases from Iran \cite{DivsalarP._2023a, Emamhadi_2018, Farhadi_2024h}; 4 cases from Turkey \cite{Akay_2015f, Atayan_2016, Tanrikulu_2015e, Yildiz_2016e}; 2 cases from China \cite{Jin_2023, Li_2013}, Poland \cite{Kobiela_2015, Wnęk_2015f}, and Spain \cite{CamachoDorado_2018, fjbuilsRepeatedBehaviorDeliberate2024}; 1 case from Australia \cite{Apikotoa_2022f}, Bahrain \cite{Ali_2020f}, Croatia \cite{Trgo_2012f}, Ecuador \cite{DelgadoSalazar_2020c}, Egypt \cite{Ali_2022g}, Ethiopia \cite{Mesfin_2022a}, Germany \cite{teWildt_2010}, Greece \cite{Sakellaridis_2008f}, Hungary \cite{Csaky_1998e}, Iraq \cite{Al-Faham_2020k}, Israel \cite{Goldman_1998f}, Italy \cite{Riva_2018j}, Japan \cite{Ohno_2005}, Nepal \cite{Thapa_2019f}, Netherlands \cite{Benoist_2019e}, Oman \cite{AlShaaibi_2021b}, Pakistan \cite{Yasin_2009}, Portugal \cite{Peixoto_2017f}, Qatar \cite{Ali_2017}, Saudi Arabia \cite{Sultan_2024f}, South Africa \cite{Sobnach_2011f}, Sweden \cite{Naji_2012f}, Switzerland \cite{Wildhaber_2005}, and Taiwan \cite{Chang_2017f}. \paragraph*{Gender} 43 cases (60\%) were male \cite{Akay_2015f, Al-Faham_2020k, Alao_2006i, Ali_2017, Ali_2022g, Apikotoa_2022f, Atayan_2016, Benoist_2019e, Berry_2021e, Bhumi_2024f, CamachoDorado_2018, Csaky_1998e, Emamhadi_2018, Farhadi_2024h, Fry_2010, Gardner_2017h, Guinan_2019f, Jehangir_2019h, Jin_2023, Kobiela_2015, Kumar_2001, Kumar_2019f, Liu_2005, Losanoff_1996, Losanoff_1997e, Mesfin_2022a, Misra_2013, Qureshi_2016, Riva_2018j, Sobnach_2011f, Tammana_2012j, Tanrikulu_2015e, Tay_2004, Thapa_2019f, Trgo_2012f, Wadhwa_2015e, Yasin_2009, teWildt_2010}, 28 cases (39\%) were female \cite{AlShaaibi_2021b, Ali_2020f, Ataya_2013, Beecroft_1998, Bhasin_2014, Bhattacharjee_2008, Cauchi_2002, Chang_2017f, Cox_2007, DelgadoSalazar_2020c, DivsalarP._2023a, Goldman_1998f, Hardy_2023g, Kar_2015, Kariholu_2008, Kerestes_2019, Li_2013, Naji_2012f, Ohno_2005, Peixoto_2017f, Sakellaridis_2008f, Sultan_2024f, Tupesis_2004f, Wildhaber_2005, Wnęk_2015f, Yildiz_2016e}, 1 case (1\%) had no gender recorded \cite{fjbuilsRepeatedBehaviorDeliberate2024}. \paragraph*{Age Group} 25 cases (35\%) were between 26 and 40 years of age \cite{Alao_2006i, Ali_2022g, Apikotoa_2022f, Ataya_2013, Benoist_2019e, Bhasin_2014, Chang_2017f, Cox_2007, DelgadoSalazar_2020c, Farhadi_2024h, Fry_2010, Gardner_2017h, Guinan_2019f, Jin_2023, Kumar_2019f, Losanoff_1996, Misra_2013, Qureshi_2016, Riva_2018j, Sakellaridis_2008f, Tammana_2012j, Trgo_2012f, Wnęk_2015f, Yildiz_2016e, fjbuilsRepeatedBehaviorDeliberate2024}, 18 cases (25\%) were between 18 and 25 years of age \cite{Akay_2015f, Ali_2017, Atayan_2016, Bhattacharjee_2008, Csaky_1998e, Kar_2015, Kariholu_2008, Kobiela_2015, Losanoff_1996, Losanoff_1997e, Mesfin_2022a, Peixoto_2017f, Sobnach_2011f, Tupesis_2004f, Yasin_2009}, 13 cases (18\%) were under 18 years of age \cite{AlShaaibi_2021b, Ali_2020f, Cauchi_2002, DivsalarP._2023a, Goldman_1998f, Liu_2005, Naji_2012f, Ohno_2005, Tanrikulu_2015e, Tay_2004, Wildhaber_2005}, 11 cases (15\%) were between 41 and 60 years of age \cite{Al-Faham_2020k, Bhumi_2024f, CamachoDorado_2018, Emamhadi_2018, Hardy_2023g, Jehangir_2019h, Kumar_2001, Sultan_2024f, Thapa_2019f, Wadhwa_2015e, teWildt_2010}, 3 cases (4\%) were over 60 years of age \cite{Beecroft_1998, Kerestes_2019, Li_2013}, 2 cases (3\%) had no age documented \cite{Berry_2021e}. \paragraph*{Population} 36 cases (50\%) had a psychiatric history \cite{AlShaaibi_2021b, Alao_2006i, Ali_2020f, Apikotoa_2022f, Ataya_2013, Atayan_2016, Beecroft_1998, CamachoDorado_2018, Chang_2017f, DelgadoSalazar_2020c, DivsalarP._2023a, Farhadi_2024h, Fry_2010, Guinan_2019f, Hardy_2023g, Jehangir_2019h, Jin_2023, Kar_2015, Kerestes_2019, Kobiela_2015, Kumar_2001, Kumar_2019f, Liu_2005, Mesfin_2022a, Misra_2013, Ohno_2005, Peixoto_2017f, Sakellaridis_2008f, Sultan_2024f, Tammana_2012j, Tanrikulu_2015e, Yildiz_2016e, fjbuilsRepeatedBehaviorDeliberate2024, teWildt_2010}, 19 cases (26\%) had ingested previously \cite{Alao_2006i, Apikotoa_2022f, Berry_2021e, Bhattacharjee_2008, Csaky_1998e, DivsalarP._2023a, Emamhadi_2018, Guinan_2019f, Jehangir_2019h, Jin_2023, Liu_2005, Sakellaridis_2008f, Tanrikulu_2015e, Thapa_2019f, Yildiz_2016e, fjbuilsRepeatedBehaviorDeliberate2024, teWildt_2010}, 12 cases (17\%) were detained persons \cite{Alao_2006i, Ali_2022g, Apikotoa_2022f, Losanoff_1996, Losanoff_1997e, Qureshi_2016, Tammana_2012j, Trgo_2012f}, 7 cases (10\%) were severely disabled \cite{Atayan_2016, Kerestes_2019, Liu_2005, Ohno_2005, Peixoto_2017f, Yildiz_2016e, teWildt_2010}, 4 cases (6\%) were psychiatric inpatients \cite{DivsalarP._2023a, fjbuilsRepeatedBehaviorDeliberate2024, teWildt_2010}, 3 cases (4\%) were under the influence of alcohol \cite{Benoist_2019e, Csaky_1998e, Thapa_2019f}, 2 cases (3\%) were displaced people \cite{Akay_2015f, Gardner_2017h}. \paragraph*{Motivation} 34 cases (47\%) had a psychiatric motivation \cite{Al-Faham_2020k, Alao_2006i, Ali_2020f, Apikotoa_2022f, Ataya_2013, Atayan_2016, Bhasin_2014, Bhattacharjee_2008, DelgadoSalazar_2020c, DivsalarP._2023a, Emamhadi_2018, Farhadi_2024h, Guinan_2019f, Hardy_2023g, Jehangir_2019h, Jin_2023, Kar_2015, Kariholu_2008, Kerestes_2019, Kobiela_2015, Kumar_2001, Kumar_2019f, Li_2013, Liu_2005, Misra_2013, Ohno_2005, Sakellaridis_2008f, Sultan_2024f, Tammana_2012j, Tanrikulu_2015e, Yasin_2009, teWildt_2010}, 21 cases (29\%) were motivated by self-harm intention \cite{Al-Faham_2020k, AlShaaibi_2021b, Alao_2006i, Ali_2017, CamachoDorado_2018, Chang_2017f, Cox_2007, Csaky_1998e, Fry_2010, Li_2013, Losanoff_1996, Losanoff_1997e, Mesfin_2022a, Sakellaridis_2008f, Tammana_2012j, Tanrikulu_2015e, fjbuilsRepeatedBehaviorDeliberate2024}, 17 cases (24\%) had a psychosocial motivation \cite{Akay_2015f, Benoist_2019e, Bhattacharjee_2008, Cauchi_2002, Goldman_1998f, Hardy_2023g, Kobiela_2015, Li_2013, Naji_2012f, Qureshi_2016, Riva_2018j, Sobnach_2011f, Tay_2004, Thapa_2019f, Tupesis_2004f, Wildhaber_2005, Wnęk_2015f}, 9 cases (12\%) were motivated by protest \cite{Bhumi_2024f, Gardner_2017h, Losanoff_1996, Losanoff_1997e, Tupesis_2004f}, 9 cases (12\%) had another documented motivation \cite{Ali_2020f, Ali_2022g, Emamhadi_2018, Guinan_2019f, Peixoto_2017f, Sakellaridis_2008f, Trgo_2012f, Wadhwa_2015e, Yildiz_2016e}. \paragraph*{Object Characteristics} 51 cases (71\%) ingested a large diameter object (\textgreater{}2.5cm) \cite{Akay_2015f, Al-Faham_2020k, AlShaaibi_2021b, Alao_2006i, Ali_2017, Ali_2022g, Apikotoa_2022f, Atayan_2016, Berry_2021e, Bhasin_2014, CamachoDorado_2018, Cauchi_2002, Chang_2017f, Cox_2007, Csaky_1998e, DivsalarP._2023a, Emamhadi_2018, Gardner_2017h, Guinan_2019f, Jehangir_2019h, Jin_2023, Kariholu_2008, Kerestes_2019, Kobiela_2015, Kumar_2001, Kumar_2019f, Losanoff_1996, Losanoff_1997e, Mesfin_2022a, Misra_2013, Naji_2012f, Ohno_2005, Peixoto_2017f, Qureshi_2016, Riva_2018j, Sakellaridis_2008f, Sultan_2024f, Tanrikulu_2015e, Thapa_2019f, Trgo_2012f, Wnęk_2015f, Yildiz_2016e, fjbuilsRepeatedBehaviorDeliberate2024, teWildt_2010}, 44 cases (61\%) ingested multiple objects \cite{Ali_2020f, Apikotoa_2022f, Ataya_2013, Atayan_2016, Beecroft_1998, Bhattacharjee_2008, Bhumi_2024f, CamachoDorado_2018, Cauchi_2002, Emamhadi_2018, Farhadi_2024h, Fry_2010, Goldman_1998f, Guinan_2019f, Hardy_2023g, Jehangir_2019h, Jin_2023, Kar_2015, Kariholu_2008, Kobiela_2015, Kumar_2001, Kumar_2019f, Li_2013, Liu_2005, Losanoff_1996, Mesfin_2022a, Misra_2013, Naji_2012f, Ohno_2005, Sobnach_2011f, Sultan_2024f, Tammana_2012j, Tanrikulu_2015e, Tay_2004, Thapa_2019f, Wadhwa_2015e, Wildhaber_2005, Yasin_2009, fjbuilsRepeatedBehaviorDeliberate2024, teWildt_2010}, 34 cases (47\%) ingested a sharp object \cite{AlShaaibi_2021b, Alao_2006i, Apikotoa_2022f, Ataya_2013, Benoist_2019e, Bhasin_2014, Bhattacharjee_2008, CamachoDorado_2018, Csaky_1998e, DelgadoSalazar_2020c, DivsalarP._2023a, Emamhadi_2018, Farhadi_2024h, Fry_2010, Guinan_2019f, Hardy_2023g, Jehangir_2019h, Jin_2023, Kariholu_2008, Kobiela_2015, Kumar_2019f, Losanoff_1996, Losanoff_1997e, Mesfin_2022a, Misra_2013, Sobnach_2011f, Yasin_2009, teWildt_2010}, 32 cases (44\%) ingested a long object (\textgreater{}5cm) \cite{Al-Faham_2020k, AlShaaibi_2021b, Ali_2017, Ali_2022g, Atayan_2016, Bhasin_2014, CamachoDorado_2018, Chang_2017f, Cox_2007, Csaky_1998e, DivsalarP._2023a, Emamhadi_2018, Fry_2010, Gardner_2017h, Jin_2023, Kariholu_2008, Kerestes_2019, Kobiela_2015, Kumar_2019f, Mesfin_2022a, Misra_2013, Ohno_2005, Qureshi_2016, Sakellaridis_2008f, Sultan_2024f, Thapa_2019f, Trgo_2012f, Yasin_2009, Yildiz_2016e, teWildt_2010}, 9 cases (12\%) ingested a magnet \cite{Ali_2020f, Bhumi_2024f, Cauchi_2002, Liu_2005, Naji_2012f, Ohno_2005, Tanrikulu_2015e, Tay_2004, Wildhaber_2005}, 2 cases (3\%) ingested a button battery \cite{Berry_2021e, Bhumi_2024f}. \paragraph*{Outcomes} 48 cases (67\%) experienced a complication \cite{Ali_2017, Ali_2020f, Apikotoa_2022f, Atayan_2016, Beecroft_1998, Benoist_2019e, Berry_2021e, Bhasin_2014, Bhumi_2024f, CamachoDorado_2018, Cauchi_2002, Cox_2007, Csaky_1998e, DelgadoSalazar_2020c, DivsalarP._2023a, Emamhadi_2018, Farhadi_2024h, Fry_2010, Gardner_2017h, Goldman_1998f, Jin_2023, Kariholu_2008, Kerestes_2019, Kobiela_2015, Kumar_2001, Kumar_2019f, Liu_2005, Losanoff_1996, Mesfin_2022a, Misra_2013, Naji_2012f, Ohno_2005, Sakellaridis_2008f, Sobnach_2011f, Sultan_2024f, Tanrikulu_2015e, Tay_2004, Thapa_2019f, Trgo_2012f, Tupesis_2004f, Wildhaber_2005, Wnęk_2015f, Yasin_2009, Yildiz_2016e}, 44 cases (61\%) underwent surgery \cite{Al-Faham_2020k, AlShaaibi_2021b, Alao_2006i, Ali_2017, Ali_2020f, Atayan_2016, Beecroft_1998, Bhasin_2014, CamachoDorado_2018, Cauchi_2002, Chang_2017f, Cox_2007, Csaky_1998e, DelgadoSalazar_2020c, DivsalarP._2023a, Farhadi_2024h, Fry_2010, Gardner_2017h, Jin_2023, Kariholu_2008, Kerestes_2019, Kobiela_2015, Kumar_2019f, Liu_2005, Losanoff_1996, Losanoff_1997e, Mesfin_2022a, Misra_2013, Naji_2012f, Sobnach_2011f, Tanrikulu_2015e, Tay_2004, Thapa_2019f, Tupesis_2004f, Wildhaber_2005, Wnęk_2015f, Yasin_2009, Yildiz_2016e, fjbuilsRepeatedBehaviorDeliberate2024}, 31 cases (43\%) underwent endoscopy \cite{Akay_2015f, Ali_2022g, Apikotoa_2022f, Atayan_2016, Benoist_2019e, Berry_2021e, Bhasin_2014, Bhumi_2024f, CamachoDorado_2018, Chang_2017f, DelgadoSalazar_2020c, Gardner_2017h, Guinan_2019f, Hardy_2023g, Jehangir_2019h, Kariholu_2008, Li_2013, Liu_2005, Ohno_2005, Peixoto_2017f, Qureshi_2016, Riva_2018j, Sakellaridis_2008f, Sultan_2024f, Tammana_2012j, Tanrikulu_2015e, Trgo_2012f, Wadhwa_2015e, Wnęk_2015f, teWildt_2010}, 7 cases (10\%) were managed conservatively \cite{Ataya_2013, Bhattacharjee_2008, DivsalarP._2023a, Emamhadi_2018, Goldman_1998f, Kar_2015, Kumar_2001}, 2 cases (3\%) died \cite{Emamhadi_2018, Kumar_2001}. All 90 were male gender. 90 cases (100\%) were detained at the time of ingestion \cite{Elghali_2016, Karp_1991b, Lee_2007}, 88 cases (98\%) were intentional ingestions \cite{Elghali_2016, Karp_1991b, Lee_2007}, 30 cases (33\%) had a psychiatric history documented \cite{Elghali_2016, Karp_1991b, Lee_2007}, 2 cases (2\%) had a history of prior ingestion \cite{Elghali_2016}. No cases were reported for were psychiatric inpatients, were displaced people, were under the influence of alcohol at the time of ingestion, and had a severe disability history.
\paragraph*{Motivation}  70 cases (78\%) reported protest motivation \cite{Elghali_2016, Karp_1991b, Lee_2007}, 12 cases (13\%) reported psychiatric motivation \cite{Karp_1991b}, 6 cases (7\%) reported self-harm motivation \cite{Elghali_2016, Karp_1991b}. No cases were reported for psychosocial motivation and other motivation.
\paragraph*{Object Characteristics}  68 cases (76\%) involved sharp object ingestion \cite{Elghali_2016, Karp_1991b, Lee_2007}, 32 cases (36\%) involved long (\textgreater 5cm) object ingestion \cite{Lee_2007}, 25 cases (28\%) involved ingestion of multiple objects \cite{Elghali_2016, Lee_2007}. No cases were reported for button battery ingestion, magnet ingestion, and involved large diameter (\textgreater 2.5cm) object ingestion.
\paragraph*{Outcomes}  47 cases (52\%) underwent endoscopic intervention \cite{Elghali_2016, Lee_2007}, 29 cases (32\%) were managed conservatively \cite{Elghali_2016, Karp_1991b}, 15 cases (17\%) underwent surgical intervention \cite{Elghali_2016, Karp_1991b, Lee_2007}, 6 cases (7\%) reported complications \cite{Lee_2007}, 1 case (1\%) died \cite{Elghali_2016}.
\paragraph*{Geographical Location}Cases were recorded in 33 countries: 13 cases from USA \cite{Alao_2006i, Ataya_2013, Bhumi_2024f, Fry_2010, Guinan_2019f, Hardy_2023g, Jehangir_2019h, Kerestes_2019, Kumar_2001, Liu_2005, Tammana_2012j, Tay_2004, Tupesis_2004f}; 7 cases from India \cite{Bhasin_2014, Bhattacharjee_2008, Kar_2015, Kariholu_2008, Kumar_2019f, Misra_2013, Wadhwa_2015e} and UK \cite{Beecroft_1998, Berry_2021e, Cauchi_2002, Cox_2007, Gardner_2017h, Qureshi_2016}; 6 cases from Bulgaria \cite{Losanoff_1996, Losanoff_1997e}; 5 cases from Iran \cite{DivsalarP._2023a, Emamhadi_2018, Farhadi_2024h}; 4 cases from Turkey \cite{Akay_2015f, Atayan_2016, Tanrikulu_2015e, Yildiz_2016e}; 2 cases from China \cite{Jin_2023, Li_2013}, Poland \cite{Kobiela_2015, Wnęk_2015f}, and Spain \cite{CamachoDorado_2018, fjbuilsRepeatedBehaviorDeliberate2024}; 1 case from Australia \cite{Apikotoa_2022f}, Bahrain \cite{Ali_2020f}, Croatia \cite{Trgo_2012f}, Ecuador \cite{DelgadoSalazar_2020c}, Egypt \cite{Ali_2022g}, Ethiopia \cite{Mesfin_2022a}, Germany \cite{teWildt_2010}, Greece \cite{Sakellaridis_2008f}, Hungary \cite{Csaky_1998e}, Iraq \cite{Al-Faham_2020k}, Israel \cite{Goldman_1998f}, Italy \cite{Riva_2018j}, Japan \cite{Ohno_2005}, Nepal \cite{Thapa_2019f}, Netherlands \cite{Benoist_2019e}, Oman \cite{AlShaaibi_2021b}, Pakistan \cite{Yasin_2009}, Portugal \cite{Peixoto_2017f}, Qatar \cite{Ali_2017}, Saudi Arabia \cite{Sultan_2024f}, South Africa \cite{Sobnach_2011f}, Sweden \cite{Naji_2012f}, Switzerland \cite{Wildhaber_2005}, and Taiwan \cite{Chang_2017f}. \paragraph*{Gender} 43 cases (60\%) were male \cite{Akay_2015f, Al-Faham_2020k, Alao_2006i, Ali_2017, Ali_2022g, Apikotoa_2022f, Atayan_2016, Benoist_2019e, Berry_2021e, Bhumi_2024f, CamachoDorado_2018, Csaky_1998e, Emamhadi_2018, Farhadi_2024h, Fry_2010, Gardner_2017h, Guinan_2019f, Jehangir_2019h, Jin_2023, Kobiela_2015, Kumar_2001, Kumar_2019f, Liu_2005, Losanoff_1996, Losanoff_1997e, Mesfin_2022a, Misra_2013, Qureshi_2016, Riva_2018j, Sobnach_2011f, Tammana_2012j, Tanrikulu_2015e, Tay_2004, Thapa_2019f, Trgo_2012f, Wadhwa_2015e, Yasin_2009, teWildt_2010}, 28 cases (39\%) were female \cite{AlShaaibi_2021b, Ali_2020f, Ataya_2013, Beecroft_1998, Bhasin_2014, Bhattacharjee_2008, Cauchi_2002, Chang_2017f, Cox_2007, DelgadoSalazar_2020c, DivsalarP._2023a, Goldman_1998f, Hardy_2023g, Kar_2015, Kariholu_2008, Kerestes_2019, Li_2013, Naji_2012f, Ohno_2005, Peixoto_2017f, Sakellaridis_2008f, Sultan_2024f, Tupesis_2004f, Wildhaber_2005, Wnęk_2015f, Yildiz_2016e}, 1 case (1\%) had no gender recorded \cite{fjbuilsRepeatedBehaviorDeliberate2024}. \paragraph*{Age Group} 25 cases (35\%) were between 26 and 40 years of age \cite{Alao_2006i, Ali_2022g, Apikotoa_2022f, Ataya_2013, Benoist_2019e, Bhasin_2014, Chang_2017f, Cox_2007, DelgadoSalazar_2020c, Farhadi_2024h, Fry_2010, Gardner_2017h, Guinan_2019f, Jin_2023, Kumar_2019f, Losanoff_1996, Misra_2013, Qureshi_2016, Riva_2018j, Sakellaridis_2008f, Tammana_2012j, Trgo_2012f, Wnęk_2015f, Yildiz_2016e, fjbuilsRepeatedBehaviorDeliberate2024}, 18 cases (25\%) were between 18 and 25 years of age \cite{Akay_2015f, Ali_2017, Atayan_2016, Bhattacharjee_2008, Csaky_1998e, Kar_2015, Kariholu_2008, Kobiela_2015, Losanoff_1996, Losanoff_1997e, Mesfin_2022a, Peixoto_2017f, Sobnach_2011f, Tupesis_2004f, Yasin_2009}, 13 cases (18\%) were under 18 years of age \cite{AlShaaibi_2021b, Ali_2020f, Cauchi_2002, DivsalarP._2023a, Goldman_1998f, Liu_2005, Naji_2012f, Ohno_2005, Tanrikulu_2015e, Tay_2004, Wildhaber_2005}, 11 cases (15\%) were between 41 and 60 years of age \cite{Al-Faham_2020k, Bhumi_2024f, CamachoDorado_2018, Emamhadi_2018, Hardy_2023g, Jehangir_2019h, Kumar_2001, Sultan_2024f, Thapa_2019f, Wadhwa_2015e, teWildt_2010}, 3 cases (4\%) were over 60 years of age \cite{Beecroft_1998, Kerestes_2019, Li_2013}, 2 cases (3\%) had no age documented \cite{Berry_2021e}. \paragraph*{Population} 36 cases (50\%) had a psychiatric history \cite{AlShaaibi_2021b, Alao_2006i, Ali_2020f, Apikotoa_2022f, Ataya_2013, Atayan_2016, Beecroft_1998, CamachoDorado_2018, Chang_2017f, DelgadoSalazar_2020c, DivsalarP._2023a, Farhadi_2024h, Fry_2010, Guinan_2019f, Hardy_2023g, Jehangir_2019h, Jin_2023, Kar_2015, Kerestes_2019, Kobiela_2015, Kumar_2001, Kumar_2019f, Liu_2005, Mesfin_2022a, Misra_2013, Ohno_2005, Peixoto_2017f, Sakellaridis_2008f, Sultan_2024f, Tammana_2012j, Tanrikulu_2015e, Yildiz_2016e, fjbuilsRepeatedBehaviorDeliberate2024, teWildt_2010}, 19 cases (26\%) had ingested previously \cite{Alao_2006i, Apikotoa_2022f, Berry_2021e, Bhattacharjee_2008, Csaky_1998e, DivsalarP._2023a, Emamhadi_2018, Guinan_2019f, Jehangir_2019h, Jin_2023, Liu_2005, Sakellaridis_2008f, Tanrikulu_2015e, Thapa_2019f, Yildiz_2016e, fjbuilsRepeatedBehaviorDeliberate2024, teWildt_2010}, 12 cases (17\%) were detained persons \cite{Alao_2006i, Ali_2022g, Apikotoa_2022f, Losanoff_1996, Losanoff_1997e, Qureshi_2016, Tammana_2012j, Trgo_2012f}, 7 cases (10\%) were severely disabled \cite{Atayan_2016, Kerestes_2019, Liu_2005, Ohno_2005, Peixoto_2017f, Yildiz_2016e, teWildt_2010}, 4 cases (6\%) were psychiatric inpatients \cite{DivsalarP._2023a, fjbuilsRepeatedBehaviorDeliberate2024, teWildt_2010}, 3 cases (4\%) were under the influence of alcohol \cite{Benoist_2019e, Csaky_1998e, Thapa_2019f}, 2 cases (3\%) were displaced people \cite{Akay_2015f, Gardner_2017h}. \paragraph*{Motivation} 34 cases (47\%) had a psychiatric motivation \cite{Al-Faham_2020k, Alao_2006i, Ali_2020f, Apikotoa_2022f, Ataya_2013, Atayan_2016, Bhasin_2014, Bhattacharjee_2008, DelgadoSalazar_2020c, DivsalarP._2023a, Emamhadi_2018, Farhadi_2024h, Guinan_2019f, Hardy_2023g, Jehangir_2019h, Jin_2023, Kar_2015, Kariholu_2008, Kerestes_2019, Kobiela_2015, Kumar_2001, Kumar_2019f, Li_2013, Liu_2005, Misra_2013, Ohno_2005, Sakellaridis_2008f, Sultan_2024f, Tammana_2012j, Tanrikulu_2015e, Yasin_2009, teWildt_2010}, 21 cases (29\%) were motivated by self-harm intention \cite{Al-Faham_2020k, AlShaaibi_2021b, Alao_2006i, Ali_2017, CamachoDorado_2018, Chang_2017f, Cox_2007, Csaky_1998e, Fry_2010, Li_2013, Losanoff_1996, Losanoff_1997e, Mesfin_2022a, Sakellaridis_2008f, Tammana_2012j, Tanrikulu_2015e, fjbuilsRepeatedBehaviorDeliberate2024}, 17 cases (24\%) had a psychosocial motivation \cite{Akay_2015f, Benoist_2019e, Bhattacharjee_2008, Cauchi_2002, Goldman_1998f, Hardy_2023g, Kobiela_2015, Li_2013, Naji_2012f, Qureshi_2016, Riva_2018j, Sobnach_2011f, Tay_2004, Thapa_2019f, Tupesis_2004f, Wildhaber_2005, Wnęk_2015f}, 9 cases (12\%) were motivated by protest \cite{Bhumi_2024f, Gardner_2017h, Losanoff_1996, Losanoff_1997e, Tupesis_2004f}, 9 cases (12\%) had another documented motivation \cite{Ali_2020f, Ali_2022g, Emamhadi_2018, Guinan_2019f, Peixoto_2017f, Sakellaridis_2008f, Trgo_2012f, Wadhwa_2015e, Yildiz_2016e}. \paragraph*{Object Characteristics} 51 cases (71\%) ingested a large diameter object (\textgreater{}2.5cm) \cite{Akay_2015f, Al-Faham_2020k, AlShaaibi_2021b, Alao_2006i, Ali_2017, Ali_2022g, Apikotoa_2022f, Atayan_2016, Berry_2021e, Bhasin_2014, CamachoDorado_2018, Cauchi_2002, Chang_2017f, Cox_2007, Csaky_1998e, DivsalarP._2023a, Emamhadi_2018, Gardner_2017h, Guinan_2019f, Jehangir_2019h, Jin_2023, Kariholu_2008, Kerestes_2019, Kobiela_2015, Kumar_2001, Kumar_2019f, Losanoff_1996, Losanoff_1997e, Mesfin_2022a, Misra_2013, Naji_2012f, Ohno_2005, Peixoto_2017f, Qureshi_2016, Riva_2018j, Sakellaridis_2008f, Sultan_2024f, Tanrikulu_2015e, Thapa_2019f, Trgo_2012f, Wnęk_2015f, Yildiz_2016e, fjbuilsRepeatedBehaviorDeliberate2024, teWildt_2010}, 44 cases (61\%) ingested multiple objects \cite{Ali_2020f, Apikotoa_2022f, Ataya_2013, Atayan_2016, Beecroft_1998, Bhattacharjee_2008, Bhumi_2024f, CamachoDorado_2018, Cauchi_2002, Emamhadi_2018, Farhadi_2024h, Fry_2010, Goldman_1998f, Guinan_2019f, Hardy_2023g, Jehangir_2019h, Jin_2023, Kar_2015, Kariholu_2008, Kobiela_2015, Kumar_2001, Kumar_2019f, Li_2013, Liu_2005, Losanoff_1996, Mesfin_2022a, Misra_2013, Naji_2012f, Ohno_2005, Sobnach_2011f, Sultan_2024f, Tammana_2012j, Tanrikulu_2015e, Tay_2004, Thapa_2019f, Wadhwa_2015e, Wildhaber_2005, Yasin_2009, fjbuilsRepeatedBehaviorDeliberate2024, teWildt_2010}, 34 cases (47\%) ingested a sharp object \cite{AlShaaibi_2021b, Alao_2006i, Apikotoa_2022f, Ataya_2013, Benoist_2019e, Bhasin_2014, Bhattacharjee_2008, CamachoDorado_2018, Csaky_1998e, DelgadoSalazar_2020c, DivsalarP._2023a, Emamhadi_2018, Farhadi_2024h, Fry_2010, Guinan_2019f, Hardy_2023g, Jehangir_2019h, Jin_2023, Kariholu_2008, Kobiela_2015, Kumar_2019f, Losanoff_1996, Losanoff_1997e, Mesfin_2022a, Misra_2013, Sobnach_2011f, Yasin_2009, teWildt_2010}, 32 cases (44\%) ingested a long object (\textgreater{}5cm) \cite{Al-Faham_2020k, AlShaaibi_2021b, Ali_2017, Ali_2022g, Atayan_2016, Bhasin_2014, CamachoDorado_2018, Chang_2017f, Cox_2007, Csaky_1998e, DivsalarP._2023a, Emamhadi_2018, Fry_2010, Gardner_2017h, Jin_2023, Kariholu_2008, Kerestes_2019, Kobiela_2015, Kumar_2019f, Mesfin_2022a, Misra_2013, Ohno_2005, Qureshi_2016, Sakellaridis_2008f, Sultan_2024f, Thapa_2019f, Trgo_2012f, Yasin_2009, Yildiz_2016e, teWildt_2010}, 9 cases (12\%) ingested a magnet \cite{Ali_2020f, Bhumi_2024f, Cauchi_2002, Liu_2005, Naji_2012f, Ohno_2005, Tanrikulu_2015e, Tay_2004, Wildhaber_2005}, 2 cases (3\%) ingested a button battery \cite{Berry_2021e, Bhumi_2024f}. \paragraph*{Outcomes} 48 cases (67\%) experienced a complication \cite{Ali_2017, Ali_2020f, Apikotoa_2022f, Atayan_2016, Beecroft_1998, Benoist_2019e, Berry_2021e, Bhasin_2014, Bhumi_2024f, CamachoDorado_2018, Cauchi_2002, Cox_2007, Csaky_1998e, DelgadoSalazar_2020c, DivsalarP._2023a, Emamhadi_2018, Farhadi_2024h, Fry_2010, Gardner_2017h, Goldman_1998f, Jin_2023, Kariholu_2008, Kerestes_2019, Kobiela_2015, Kumar_2001, Kumar_2019f, Liu_2005, Losanoff_1996, Mesfin_2022a, Misra_2013, Naji_2012f, Ohno_2005, Sakellaridis_2008f, Sobnach_2011f, Sultan_2024f, Tanrikulu_2015e, Tay_2004, Thapa_2019f, Trgo_2012f, Tupesis_2004f, Wildhaber_2005, Wnęk_2015f, Yasin_2009, Yildiz_2016e}, 44 cases (61\%) underwent surgery \cite{Al-Faham_2020k, AlShaaibi_2021b, Alao_2006i, Ali_2017, Ali_2020f, Atayan_2016, Beecroft_1998, Bhasin_2014, CamachoDorado_2018, Cauchi_2002, Chang_2017f, Cox_2007, Csaky_1998e, DelgadoSalazar_2020c, DivsalarP._2023a, Farhadi_2024h, Fry_2010, Gardner_2017h, Jin_2023, Kariholu_2008, Kerestes_2019, Kobiela_2015, Kumar_2019f, Liu_2005, Losanoff_1996, Losanoff_1997e, Mesfin_2022a, Misra_2013, Naji_2012f, Sobnach_2011f, Tanrikulu_2015e, Tay_2004, Thapa_2019f, Tupesis_2004f, Wildhaber_2005, Wnęk_2015f, Yasin_2009, Yildiz_2016e, fjbuilsRepeatedBehaviorDeliberate2024}, 31 cases (43\%) underwent endoscopy \cite{Akay_2015f, Ali_2022g, Apikotoa_2022f, Atayan_2016, Benoist_2019e, Berry_2021e, Bhasin_2014, Bhumi_2024f, CamachoDorado_2018, Chang_2017f, DelgadoSalazar_2020c, Gardner_2017h, Guinan_2019f, Hardy_2023g, Jehangir_2019h, Kariholu_2008, Li_2013, Liu_2005, Ohno_2005, Peixoto_2017f, Qureshi_2016, Riva_2018j, Sakellaridis_2008f, Sultan_2024f, Tammana_2012j, Tanrikulu_2015e, Trgo_2012f, Wadhwa_2015e, Wnęk_2015f, teWildt_2010}, 7 cases (10\%) were managed conservatively \cite{Ataya_2013, Bhattacharjee_2008, DivsalarP._2023a, Emamhadi_2018, Goldman_1998f, Kar_2015, Kumar_2001}, 2 cases (3\%) died \cite{Emamhadi_2018, Kumar_2001}. All 90 were male gender. 90 cases (100\%) were detained at the time of ingestion \cite{Elghali_2016, Karp_1991b, Lee_2007}, 88 cases (98\%) were intentional ingestions \cite{Elghali_2016, Karp_1991b, Lee_2007}, 30 cases (33\%) had a psychiatric history documented \cite{Elghali_2016, Karp_1991b, Lee_2007}, 2 cases (2\%) had a history of prior ingestion \cite{Elghali_2016}. No cases were reported for were psychiatric inpatients, were displaced people, were under the influence of alcohol at the time of ingestion, and had a severe disability history.
\paragraph*{Motivation}  70 cases (78\%) reported protest motivation \cite{Elghali_2016, Karp_1991b, Lee_2007}, 12 cases (13\%) reported psychiatric motivation \cite{Karp_1991b}, 6 cases (7\%) reported self-harm motivation \cite{Elghali_2016, Karp_1991b}. No cases were reported for psychosocial motivation and other motivation.
\paragraph*{Object Characteristics}  68 cases (76\%) involved sharp object ingestion \cite{Elghali_2016, Karp_1991b, Lee_2007}, 32 cases (36\%) involved long (\textgreater 5cm) object ingestion \cite{Lee_2007}, 25 cases (28\%) involved ingestion of multiple objects \cite{Elghali_2016, Lee_2007}. No cases were reported for button battery ingestion, magnet ingestion, and involved large diameter (\textgreater 2.5cm) object ingestion.
\paragraph*{Outcomes}  47 cases (52\%) underwent endoscopic intervention \cite{Elghali_2016, Lee_2007}, 29 cases (32\%) were managed conservatively \cite{Elghali_2016, Karp_1991b}, 15 cases (17\%) underwent surgical intervention \cite{Elghali_2016, Karp_1991b, Lee_2007}, 6 cases (7\%) reported complications \cite{Lee_2007}, 1 case (1\%) died \cite{Elghali_2016}.
\paragraph*{Geographical Location}Cases were recorded in 33 countries: 13 cases from USA \cite{Alao_2006i, Ataya_2013, Bhumi_2024f, Fry_2010, Guinan_2019f, Hardy_2023g, Jehangir_2019h, Kerestes_2019, Kumar_2001, Liu_2005, Tammana_2012j, Tay_2004, Tupesis_2004f}; 7 cases from India \cite{Bhasin_2014, Bhattacharjee_2008, Kar_2015, Kariholu_2008, Kumar_2019f, Misra_2013, Wadhwa_2015e} and UK \cite{Beecroft_1998, Berry_2021e, Cauchi_2002, Cox_2007, Gardner_2017h, Qureshi_2016}; 6 cases from Bulgaria \cite{Losanoff_1996, Losanoff_1997e}; 5 cases from Iran \cite{DivsalarP._2023a, Emamhadi_2018, Farhadi_2024h}; 4 cases from Turkey \cite{Akay_2015f, Atayan_2016, Tanrikulu_2015e, Yildiz_2016e}; 2 cases from China \cite{Jin_2023, Li_2013}, Poland \cite{Kobiela_2015, Wnęk_2015f}, and Spain \cite{CamachoDorado_2018, fjbuilsRepeatedBehaviorDeliberate2024}; 1 case from Australia \cite{Apikotoa_2022f}, Bahrain \cite{Ali_2020f}, Croatia \cite{Trgo_2012f}, Ecuador \cite{DelgadoSalazar_2020c}, Egypt \cite{Ali_2022g}, Ethiopia \cite{Mesfin_2022a}, Germany \cite{teWildt_2010}, Greece \cite{Sakellaridis_2008f}, Hungary \cite{Csaky_1998e}, Iraq \cite{Al-Faham_2020k}, Israel \cite{Goldman_1998f}, Italy \cite{Riva_2018j}, Japan \cite{Ohno_2005}, Nepal \cite{Thapa_2019f}, Netherlands \cite{Benoist_2019e}, Oman \cite{AlShaaibi_2021b}, Pakistan \cite{Yasin_2009}, Portugal \cite{Peixoto_2017f}, Qatar \cite{Ali_2017}, Saudi Arabia \cite{Sultan_2024f}, South Africa \cite{Sobnach_2011f}, Sweden \cite{Naji_2012f}, Switzerland \cite{Wildhaber_2005}, and Taiwan \cite{Chang_2017f}. \paragraph*{Gender} 43 cases (60\%) were male \cite{Akay_2015f, Al-Faham_2020k, Alao_2006i, Ali_2017, Ali_2022g, Apikotoa_2022f, Atayan_2016, Benoist_2019e, Berry_2021e, Bhumi_2024f, CamachoDorado_2018, Csaky_1998e, Emamhadi_2018, Farhadi_2024h, Fry_2010, Gardner_2017h, Guinan_2019f, Jehangir_2019h, Jin_2023, Kobiela_2015, Kumar_2001, Kumar_2019f, Liu_2005, Losanoff_1996, Losanoff_1997e, Mesfin_2022a, Misra_2013, Qureshi_2016, Riva_2018j, Sobnach_2011f, Tammana_2012j, Tanrikulu_2015e, Tay_2004, Thapa_2019f, Trgo_2012f, Wadhwa_2015e, Yasin_2009, teWildt_2010}, 28 cases (39\%) were female \cite{AlShaaibi_2021b, Ali_2020f, Ataya_2013, Beecroft_1998, Bhasin_2014, Bhattacharjee_2008, Cauchi_2002, Chang_2017f, Cox_2007, DelgadoSalazar_2020c, DivsalarP._2023a, Goldman_1998f, Hardy_2023g, Kar_2015, Kariholu_2008, Kerestes_2019, Li_2013, Naji_2012f, Ohno_2005, Peixoto_2017f, Sakellaridis_2008f, Sultan_2024f, Tupesis_2004f, Wildhaber_2005, Wnęk_2015f, Yildiz_2016e}, 1 case (1\%) had no gender recorded \cite{fjbuilsRepeatedBehaviorDeliberate2024}. \paragraph*{Age Group} 25 cases (35\%) were between 26 and 40 years of age \cite{Alao_2006i, Ali_2022g, Apikotoa_2022f, Ataya_2013, Benoist_2019e, Bhasin_2014, Chang_2017f, Cox_2007, DelgadoSalazar_2020c, Farhadi_2024h, Fry_2010, Gardner_2017h, Guinan_2019f, Jin_2023, Kumar_2019f, Losanoff_1996, Misra_2013, Qureshi_2016, Riva_2018j, Sakellaridis_2008f, Tammana_2012j, Trgo_2012f, Wnęk_2015f, Yildiz_2016e, fjbuilsRepeatedBehaviorDeliberate2024}, 18 cases (25\%) were between 18 and 25 years of age \cite{Akay_2015f, Ali_2017, Atayan_2016, Bhattacharjee_2008, Csaky_1998e, Kar_2015, Kariholu_2008, Kobiela_2015, Losanoff_1996, Losanoff_1997e, Mesfin_2022a, Peixoto_2017f, Sobnach_2011f, Tupesis_2004f, Yasin_2009}, 13 cases (18\%) were under 18 years of age \cite{AlShaaibi_2021b, Ali_2020f, Cauchi_2002, DivsalarP._2023a, Goldman_1998f, Liu_2005, Naji_2012f, Ohno_2005, Tanrikulu_2015e, Tay_2004, Wildhaber_2005}, 11 cases (15\%) were between 41 and 60 years of age \cite{Al-Faham_2020k, Bhumi_2024f, CamachoDorado_2018, Emamhadi_2018, Hardy_2023g, Jehangir_2019h, Kumar_2001, Sultan_2024f, Thapa_2019f, Wadhwa_2015e, teWildt_2010}, 3 cases (4\%) were over 60 years of age \cite{Beecroft_1998, Kerestes_2019, Li_2013}, 2 cases (3\%) had no age documented \cite{Berry_2021e}. \paragraph*{Population} 36 cases (50\%) had a psychiatric history \cite{AlShaaibi_2021b, Alao_2006i, Ali_2020f, Apikotoa_2022f, Ataya_2013, Atayan_2016, Beecroft_1998, CamachoDorado_2018, Chang_2017f, DelgadoSalazar_2020c, DivsalarP._2023a, Farhadi_2024h, Fry_2010, Guinan_2019f, Hardy_2023g, Jehangir_2019h, Jin_2023, Kar_2015, Kerestes_2019, Kobiela_2015, Kumar_2001, Kumar_2019f, Liu_2005, Mesfin_2022a, Misra_2013, Ohno_2005, Peixoto_2017f, Sakellaridis_2008f, Sultan_2024f, Tammana_2012j, Tanrikulu_2015e, Yildiz_2016e, fjbuilsRepeatedBehaviorDeliberate2024, teWildt_2010}, 19 cases (26\%) had ingested previously \cite{Alao_2006i, Apikotoa_2022f, Berry_2021e, Bhattacharjee_2008, Csaky_1998e, DivsalarP._2023a, Emamhadi_2018, Guinan_2019f, Jehangir_2019h, Jin_2023, Liu_2005, Sakellaridis_2008f, Tanrikulu_2015e, Thapa_2019f, Yildiz_2016e, fjbuilsRepeatedBehaviorDeliberate2024, teWildt_2010}, 12 cases (17\%) were detained persons \cite{Alao_2006i, Ali_2022g, Apikotoa_2022f, Losanoff_1996, Losanoff_1997e, Qureshi_2016, Tammana_2012j, Trgo_2012f}, 7 cases (10\%) were severely disabled \cite{Atayan_2016, Kerestes_2019, Liu_2005, Ohno_2005, Peixoto_2017f, Yildiz_2016e, teWildt_2010}, 4 cases (6\%) were psychiatric inpatients \cite{DivsalarP._2023a, fjbuilsRepeatedBehaviorDeliberate2024, teWildt_2010}, 3 cases (4\%) were under the influence of alcohol \cite{Benoist_2019e, Csaky_1998e, Thapa_2019f}, 2 cases (3\%) were displaced people \cite{Akay_2015f, Gardner_2017h}. \paragraph*{Motivation} 34 cases (47\%) had a psychiatric motivation \cite{Al-Faham_2020k, Alao_2006i, Ali_2020f, Apikotoa_2022f, Ataya_2013, Atayan_2016, Bhasin_2014, Bhattacharjee_2008, DelgadoSalazar_2020c, DivsalarP._2023a, Emamhadi_2018, Farhadi_2024h, Guinan_2019f, Hardy_2023g, Jehangir_2019h, Jin_2023, Kar_2015, Kariholu_2008, Kerestes_2019, Kobiela_2015, Kumar_2001, Kumar_2019f, Li_2013, Liu_2005, Misra_2013, Ohno_2005, Sakellaridis_2008f, Sultan_2024f, Tammana_2012j, Tanrikulu_2015e, Yasin_2009, teWildt_2010}, 21 cases (29\%) were motivated by self-harm intention \cite{Al-Faham_2020k, AlShaaibi_2021b, Alao_2006i, Ali_2017, CamachoDorado_2018, Chang_2017f, Cox_2007, Csaky_1998e, Fry_2010, Li_2013, Losanoff_1996, Losanoff_1997e, Mesfin_2022a, Sakellaridis_2008f, Tammana_2012j, Tanrikulu_2015e, fjbuilsRepeatedBehaviorDeliberate2024}, 17 cases (24\%) had a psychosocial motivation \cite{Akay_2015f, Benoist_2019e, Bhattacharjee_2008, Cauchi_2002, Goldman_1998f, Hardy_2023g, Kobiela_2015, Li_2013, Naji_2012f, Qureshi_2016, Riva_2018j, Sobnach_2011f, Tay_2004, Thapa_2019f, Tupesis_2004f, Wildhaber_2005, Wnęk_2015f}, 9 cases (12\%) were motivated by protest \cite{Bhumi_2024f, Gardner_2017h, Losanoff_1996, Losanoff_1997e, Tupesis_2004f}, 9 cases (12\%) had another documented motivation \cite{Ali_2020f, Ali_2022g, Emamhadi_2018, Guinan_2019f, Peixoto_2017f, Sakellaridis_2008f, Trgo_2012f, Wadhwa_2015e, Yildiz_2016e}. \paragraph*{Object Characteristics} 51 cases (71\%) ingested a large diameter object (\textgreater{}2.5cm) \cite{Akay_2015f, Al-Faham_2020k, AlShaaibi_2021b, Alao_2006i, Ali_2017, Ali_2022g, Apikotoa_2022f, Atayan_2016, Berry_2021e, Bhasin_2014, CamachoDorado_2018, Cauchi_2002, Chang_2017f, Cox_2007, Csaky_1998e, DivsalarP._2023a, Emamhadi_2018, Gardner_2017h, Guinan_2019f, Jehangir_2019h, Jin_2023, Kariholu_2008, Kerestes_2019, Kobiela_2015, Kumar_2001, Kumar_2019f, Losanoff_1996, Losanoff_1997e, Mesfin_2022a, Misra_2013, Naji_2012f, Ohno_2005, Peixoto_2017f, Qureshi_2016, Riva_2018j, Sakellaridis_2008f, Sultan_2024f, Tanrikulu_2015e, Thapa_2019f, Trgo_2012f, Wnęk_2015f, Yildiz_2016e, fjbuilsRepeatedBehaviorDeliberate2024, teWildt_2010}, 44 cases (61\%) ingested multiple objects \cite{Ali_2020f, Apikotoa_2022f, Ataya_2013, Atayan_2016, Beecroft_1998, Bhattacharjee_2008, Bhumi_2024f, CamachoDorado_2018, Cauchi_2002, Emamhadi_2018, Farhadi_2024h, Fry_2010, Goldman_1998f, Guinan_2019f, Hardy_2023g, Jehangir_2019h, Jin_2023, Kar_2015, Kariholu_2008, Kobiela_2015, Kumar_2001, Kumar_2019f, Li_2013, Liu_2005, Losanoff_1996, Mesfin_2022a, Misra_2013, Naji_2012f, Ohno_2005, Sobnach_2011f, Sultan_2024f, Tammana_2012j, Tanrikulu_2015e, Tay_2004, Thapa_2019f, Wadhwa_2015e, Wildhaber_2005, Yasin_2009, fjbuilsRepeatedBehaviorDeliberate2024, teWildt_2010}, 34 cases (47\%) ingested a sharp object \cite{AlShaaibi_2021b, Alao_2006i, Apikotoa_2022f, Ataya_2013, Benoist_2019e, Bhasin_2014, Bhattacharjee_2008, CamachoDorado_2018, Csaky_1998e, DelgadoSalazar_2020c, DivsalarP._2023a, Emamhadi_2018, Farhadi_2024h, Fry_2010, Guinan_2019f, Hardy_2023g, Jehangir_2019h, Jin_2023, Kariholu_2008, Kobiela_2015, Kumar_2019f, Losanoff_1996, Losanoff_1997e, Mesfin_2022a, Misra_2013, Sobnach_2011f, Yasin_2009, teWildt_2010}, 32 cases (44\%) ingested a long object (\textgreater{}5cm) \cite{Al-Faham_2020k, AlShaaibi_2021b, Ali_2017, Ali_2022g, Atayan_2016, Bhasin_2014, CamachoDorado_2018, Chang_2017f, Cox_2007, Csaky_1998e, DivsalarP._2023a, Emamhadi_2018, Fry_2010, Gardner_2017h, Jin_2023, Kariholu_2008, Kerestes_2019, Kobiela_2015, Kumar_2019f, Mesfin_2022a, Misra_2013, Ohno_2005, Qureshi_2016, Sakellaridis_2008f, Sultan_2024f, Thapa_2019f, Trgo_2012f, Yasin_2009, Yildiz_2016e, teWildt_2010}, 9 cases (12\%) ingested a magnet \cite{Ali_2020f, Bhumi_2024f, Cauchi_2002, Liu_2005, Naji_2012f, Ohno_2005, Tanrikulu_2015e, Tay_2004, Wildhaber_2005}, 2 cases (3\%) ingested a button battery \cite{Berry_2021e, Bhumi_2024f}. \paragraph*{Outcomes} 48 cases (67\%) experienced a complication \cite{Ali_2017, Ali_2020f, Apikotoa_2022f, Atayan_2016, Beecroft_1998, Benoist_2019e, Berry_2021e, Bhasin_2014, Bhumi_2024f, CamachoDorado_2018, Cauchi_2002, Cox_2007, Csaky_1998e, DelgadoSalazar_2020c, DivsalarP._2023a, Emamhadi_2018, Farhadi_2024h, Fry_2010, Gardner_2017h, Goldman_1998f, Jin_2023, Kariholu_2008, Kerestes_2019, Kobiela_2015, Kumar_2001, Kumar_2019f, Liu_2005, Losanoff_1996, Mesfin_2022a, Misra_2013, Naji_2012f, Ohno_2005, Sakellaridis_2008f, Sobnach_2011f, Sultan_2024f, Tanrikulu_2015e, Tay_2004, Thapa_2019f, Trgo_2012f, Tupesis_2004f, Wildhaber_2005, Wnęk_2015f, Yasin_2009, Yildiz_2016e}, 44 cases (61\%) underwent surgery \cite{Al-Faham_2020k, AlShaaibi_2021b, Alao_2006i, Ali_2017, Ali_2020f, Atayan_2016, Beecroft_1998, Bhasin_2014, CamachoDorado_2018, Cauchi_2002, Chang_2017f, Cox_2007, Csaky_1998e, DelgadoSalazar_2020c, DivsalarP._2023a, Farhadi_2024h, Fry_2010, Gardner_2017h, Jin_2023, Kariholu_2008, Kerestes_2019, Kobiela_2015, Kumar_2019f, Liu_2005, Losanoff_1996, Losanoff_1997e, Mesfin_2022a, Misra_2013, Naji_2012f, Sobnach_2011f, Tanrikulu_2015e, Tay_2004, Thapa_2019f, Tupesis_2004f, Wildhaber_2005, Wnęk_2015f, Yasin_2009, Yildiz_2016e, fjbuilsRepeatedBehaviorDeliberate2024}, 31 cases (43\%) underwent endoscopy \cite{Akay_2015f, Ali_2022g, Apikotoa_2022f, Atayan_2016, Benoist_2019e, Berry_2021e, Bhasin_2014, Bhumi_2024f, CamachoDorado_2018, Chang_2017f, DelgadoSalazar_2020c, Gardner_2017h, Guinan_2019f, Hardy_2023g, Jehangir_2019h, Kariholu_2008, Li_2013, Liu_2005, Ohno_2005, Peixoto_2017f, Qureshi_2016, Riva_2018j, Sakellaridis_2008f, Sultan_2024f, Tammana_2012j, Tanrikulu_2015e, Trgo_2012f, Wadhwa_2015e, Wnęk_2015f, teWildt_2010}, 7 cases (10\%) were managed conservatively \cite{Ataya_2013, Bhattacharjee_2008, DivsalarP._2023a, Emamhadi_2018, Goldman_1998f, Kar_2015, Kumar_2001}, 2 cases (3\%) died \cite{Emamhadi_2018, Kumar_2001}. All 90 were male gender. 90 cases (100\%) were detained at the time of ingestion \cite{Elghali_2016, Karp_1991b, Lee_2007}, 88 cases (98\%) were intentional ingestions \cite{Elghali_2016, Karp_1991b, Lee_2007}, 30 cases (33\%) had a psychiatric history documented \cite{Elghali_2016, Karp_1991b, Lee_2007}, 2 cases (2\%) had a history of prior ingestion \cite{Elghali_2016}. No cases were reported for were psychiatric inpatients, were displaced people, were under the influence of alcohol at the time of ingestion, and had a severe disability history.
\paragraph*{Motivation}  70 cases (78\%) reported protest motivation \cite{Elghali_2016, Karp_1991b, Lee_2007}, 12 cases (13\%) reported psychiatric motivation \cite{Karp_1991b}, 6 cases (7\%) reported self-harm motivation \cite{Elghali_2016, Karp_1991b}. No cases were reported for psychosocial motivation and other motivation.
\paragraph*{Object Characteristics}  68 cases (76\%) involved sharp object ingestion \cite{Elghali_2016, Karp_1991b, Lee_2007}, 32 cases (36\%) involved long (\textgreater 5cm) object ingestion \cite{Lee_2007}, 25 cases (28\%) involved ingestion of multiple objects \cite{Elghali_2016, Lee_2007}. No cases were reported for button battery ingestion, magnet ingestion, and involved large diameter (\textgreater 2.5cm) object ingestion.
\paragraph*{Outcomes}  47 cases (52\%) underwent endoscopic intervention \cite{Elghali_2016, Lee_2007}, 29 cases (32\%) were managed conservatively \cite{Elghali_2016, Karp_1991b}, 15 cases (17\%) underwent surgical intervention \cite{Elghali_2016, Karp_1991b, Lee_2007}, 6 cases (7\%) reported complications \cite{Lee_2007}, 1 case (1\%) died \cite{Elghali_2016}.
\paragraph*{Geographical Location}Cases were recorded in 33 countries: 13 cases from USA \cite{Alao_2006i, Ataya_2013, Bhumi_2024f, Fry_2010, Guinan_2019f, Hardy_2023g, Jehangir_2019h, Kerestes_2019, Kumar_2001, Liu_2005, Tammana_2012j, Tay_2004, Tupesis_2004f}; 7 cases from India \cite{Bhasin_2014, Bhattacharjee_2008, Kar_2015, Kariholu_2008, Kumar_2019f, Misra_2013, Wadhwa_2015e} and UK \cite{Beecroft_1998, Berry_2021e, Cauchi_2002, Cox_2007, Gardner_2017h, Qureshi_2016}; 6 cases from Bulgaria \cite{Losanoff_1996, Losanoff_1997e}; 5 cases from Iran \cite{DivsalarP._2023a, Emamhadi_2018, Farhadi_2024h}; 4 cases from Turkey \cite{Akay_2015f, Atayan_2016, Tanrikulu_2015e, Yildiz_2016e}; 2 cases from China \cite{Jin_2023, Li_2013}, Poland \cite{Kobiela_2015, Wnęk_2015f}, and Spain \cite{CamachoDorado_2018, fjbuilsRepeatedBehaviorDeliberate2024}; 1 case from Australia \cite{Apikotoa_2022f}, Bahrain \cite{Ali_2020f}, Croatia \cite{Trgo_2012f}, Ecuador \cite{DelgadoSalazar_2020c}, Egypt \cite{Ali_2022g}, Ethiopia \cite{Mesfin_2022a}, Germany \cite{teWildt_2010}, Greece \cite{Sakellaridis_2008f}, Hungary \cite{Csaky_1998e}, Iraq \cite{Al-Faham_2020k}, Israel \cite{Goldman_1998f}, Italy \cite{Riva_2018j}, Japan \cite{Ohno_2005}, Nepal \cite{Thapa_2019f}, Netherlands \cite{Benoist_2019e}, Oman \cite{AlShaaibi_2021b}, Pakistan \cite{Yasin_2009}, Portugal \cite{Peixoto_2017f}, Qatar \cite{Ali_2017}, Saudi Arabia \cite{Sultan_2024f}, South Africa \cite{Sobnach_2011f}, Sweden \cite{Naji_2012f}, Switzerland \cite{Wildhaber_2005}, and Taiwan \cite{Chang_2017f}. \paragraph*{Gender} 43 cases (60\%) were male \cite{Akay_2015f, Al-Faham_2020k, Alao_2006i, Ali_2017, Ali_2022g, Apikotoa_2022f, Atayan_2016, Benoist_2019e, Berry_2021e, Bhumi_2024f, CamachoDorado_2018, Csaky_1998e, Emamhadi_2018, Farhadi_2024h, Fry_2010, Gardner_2017h, Guinan_2019f, Jehangir_2019h, Jin_2023, Kobiela_2015, Kumar_2001, Kumar_2019f, Liu_2005, Losanoff_1996, Losanoff_1997e, Mesfin_2022a, Misra_2013, Qureshi_2016, Riva_2018j, Sobnach_2011f, Tammana_2012j, Tanrikulu_2015e, Tay_2004, Thapa_2019f, Trgo_2012f, Wadhwa_2015e, Yasin_2009, teWildt_2010}, 28 cases (39\%) were female \cite{AlShaaibi_2021b, Ali_2020f, Ataya_2013, Beecroft_1998, Bhasin_2014, Bhattacharjee_2008, Cauchi_2002, Chang_2017f, Cox_2007, DelgadoSalazar_2020c, DivsalarP._2023a, Goldman_1998f, Hardy_2023g, Kar_2015, Kariholu_2008, Kerestes_2019, Li_2013, Naji_2012f, Ohno_2005, Peixoto_2017f, Sakellaridis_2008f, Sultan_2024f, Tupesis_2004f, Wildhaber_2005, Wnęk_2015f, Yildiz_2016e}, 1 case (1\%) had no gender recorded \cite{fjbuilsRepeatedBehaviorDeliberate2024}. \paragraph*{Age Group} 25 cases (35\%) were between 26 and 40 years of age \cite{Alao_2006i, Ali_2022g, Apikotoa_2022f, Ataya_2013, Benoist_2019e, Bhasin_2014, Chang_2017f, Cox_2007, DelgadoSalazar_2020c, Farhadi_2024h, Fry_2010, Gardner_2017h, Guinan_2019f, Jin_2023, Kumar_2019f, Losanoff_1996, Misra_2013, Qureshi_2016, Riva_2018j, Sakellaridis_2008f, Tammana_2012j, Trgo_2012f, Wnęk_2015f, Yildiz_2016e, fjbuilsRepeatedBehaviorDeliberate2024}, 18 cases (25\%) were between 18 and 25 years of age \cite{Akay_2015f, Ali_2017, Atayan_2016, Bhattacharjee_2008, Csaky_1998e, Kar_2015, Kariholu_2008, Kobiela_2015, Losanoff_1996, Losanoff_1997e, Mesfin_2022a, Peixoto_2017f, Sobnach_2011f, Tupesis_2004f, Yasin_2009}, 13 cases (18\%) were under 18 years of age \cite{AlShaaibi_2021b, Ali_2020f, Cauchi_2002, DivsalarP._2023a, Goldman_1998f, Liu_2005, Naji_2012f, Ohno_2005, Tanrikulu_2015e, Tay_2004, Wildhaber_2005}, 11 cases (15\%) were between 41 and 60 years of age \cite{Al-Faham_2020k, Bhumi_2024f, CamachoDorado_2018, Emamhadi_2018, Hardy_2023g, Jehangir_2019h, Kumar_2001, Sultan_2024f, Thapa_2019f, Wadhwa_2015e, teWildt_2010}, 3 cases (4\%) were over 60 years of age \cite{Beecroft_1998, Kerestes_2019, Li_2013}, 2 cases (3\%) had no age documented \cite{Berry_2021e}. \paragraph*{Population} 36 cases (50\%) had a psychiatric history \cite{AlShaaibi_2021b, Alao_2006i, Ali_2020f, Apikotoa_2022f, Ataya_2013, Atayan_2016, Beecroft_1998, CamachoDorado_2018, Chang_2017f, DelgadoSalazar_2020c, DivsalarP._2023a, Farhadi_2024h, Fry_2010, Guinan_2019f, Hardy_2023g, Jehangir_2019h, Jin_2023, Kar_2015, Kerestes_2019, Kobiela_2015, Kumar_2001, Kumar_2019f, Liu_2005, Mesfin_2022a, Misra_2013, Ohno_2005, Peixoto_2017f, Sakellaridis_2008f, Sultan_2024f, Tammana_2012j, Tanrikulu_2015e, Yildiz_2016e, fjbuilsRepeatedBehaviorDeliberate2024, teWildt_2010}, 19 cases (26\%) had ingested previously \cite{Alao_2006i, Apikotoa_2022f, Berry_2021e, Bhattacharjee_2008, Csaky_1998e, DivsalarP._2023a, Emamhadi_2018, Guinan_2019f, Jehangir_2019h, Jin_2023, Liu_2005, Sakellaridis_2008f, Tanrikulu_2015e, Thapa_2019f, Yildiz_2016e, fjbuilsRepeatedBehaviorDeliberate2024, teWildt_2010}, 12 cases (17\%) were detained persons \cite{Alao_2006i, Ali_2022g, Apikotoa_2022f, Losanoff_1996, Losanoff_1997e, Qureshi_2016, Tammana_2012j, Trgo_2012f}, 7 cases (10\%) were severely disabled \cite{Atayan_2016, Kerestes_2019, Liu_2005, Ohno_2005, Peixoto_2017f, Yildiz_2016e, teWildt_2010}, 4 cases (6\%) were psychiatric inpatients \cite{DivsalarP._2023a, fjbuilsRepeatedBehaviorDeliberate2024, teWildt_2010}, 3 cases (4\%) were under the influence of alcohol \cite{Benoist_2019e, Csaky_1998e, Thapa_2019f}, 2 cases (3\%) were displaced people \cite{Akay_2015f, Gardner_2017h}. \paragraph*{Motivation} 34 cases (47\%) had a psychiatric motivation \cite{Al-Faham_2020k, Alao_2006i, Ali_2020f, Apikotoa_2022f, Ataya_2013, Atayan_2016, Bhasin_2014, Bhattacharjee_2008, DelgadoSalazar_2020c, DivsalarP._2023a, Emamhadi_2018, Farhadi_2024h, Guinan_2019f, Hardy_2023g, Jehangir_2019h, Jin_2023, Kar_2015, Kariholu_2008, Kerestes_2019, Kobiela_2015, Kumar_2001, Kumar_2019f, Li_2013, Liu_2005, Misra_2013, Ohno_2005, Sakellaridis_2008f, Sultan_2024f, Tammana_2012j, Tanrikulu_2015e, Yasin_2009, teWildt_2010}, 21 cases (29\%) were motivated by self-harm intention \cite{Al-Faham_2020k, AlShaaibi_2021b, Alao_2006i, Ali_2017, CamachoDorado_2018, Chang_2017f, Cox_2007, Csaky_1998e, Fry_2010, Li_2013, Losanoff_1996, Losanoff_1997e, Mesfin_2022a, Sakellaridis_2008f, Tammana_2012j, Tanrikulu_2015e, fjbuilsRepeatedBehaviorDeliberate2024}, 17 cases (24\%) had a psychosocial motivation \cite{Akay_2015f, Benoist_2019e, Bhattacharjee_2008, Cauchi_2002, Goldman_1998f, Hardy_2023g, Kobiela_2015, Li_2013, Naji_2012f, Qureshi_2016, Riva_2018j, Sobnach_2011f, Tay_2004, Thapa_2019f, Tupesis_2004f, Wildhaber_2005, Wnęk_2015f}, 9 cases (12\%) were motivated by protest \cite{Bhumi_2024f, Gardner_2017h, Losanoff_1996, Losanoff_1997e, Tupesis_2004f}, 9 cases (12\%) had another documented motivation \cite{Ali_2020f, Ali_2022g, Emamhadi_2018, Guinan_2019f, Peixoto_2017f, Sakellaridis_2008f, Trgo_2012f, Wadhwa_2015e, Yildiz_2016e}. \paragraph*{Object Characteristics} 51 cases (71\%) ingested a large diameter object (\textgreater{}2.5cm) \cite{Akay_2015f, Al-Faham_2020k, AlShaaibi_2021b, Alao_2006i, Ali_2017, Ali_2022g, Apikotoa_2022f, Atayan_2016, Berry_2021e, Bhasin_2014, CamachoDorado_2018, Cauchi_2002, Chang_2017f, Cox_2007, Csaky_1998e, DivsalarP._2023a, Emamhadi_2018, Gardner_2017h, Guinan_2019f, Jehangir_2019h, Jin_2023, Kariholu_2008, Kerestes_2019, Kobiela_2015, Kumar_2001, Kumar_2019f, Losanoff_1996, Losanoff_1997e, Mesfin_2022a, Misra_2013, Naji_2012f, Ohno_2005, Peixoto_2017f, Qureshi_2016, Riva_2018j, Sakellaridis_2008f, Sultan_2024f, Tanrikulu_2015e, Thapa_2019f, Trgo_2012f, Wnęk_2015f, Yildiz_2016e, fjbuilsRepeatedBehaviorDeliberate2024, teWildt_2010}, 44 cases (61\%) ingested multiple objects \cite{Ali_2020f, Apikotoa_2022f, Ataya_2013, Atayan_2016, Beecroft_1998, Bhattacharjee_2008, Bhumi_2024f, CamachoDorado_2018, Cauchi_2002, Emamhadi_2018, Farhadi_2024h, Fry_2010, Goldman_1998f, Guinan_2019f, Hardy_2023g, Jehangir_2019h, Jin_2023, Kar_2015, Kariholu_2008, Kobiela_2015, Kumar_2001, Kumar_2019f, Li_2013, Liu_2005, Losanoff_1996, Mesfin_2022a, Misra_2013, Naji_2012f, Ohno_2005, Sobnach_2011f, Sultan_2024f, Tammana_2012j, Tanrikulu_2015e, Tay_2004, Thapa_2019f, Wadhwa_2015e, Wildhaber_2005, Yasin_2009, fjbuilsRepeatedBehaviorDeliberate2024, teWildt_2010}, 34 cases (47\%) ingested a sharp object \cite{AlShaaibi_2021b, Alao_2006i, Apikotoa_2022f, Ataya_2013, Benoist_2019e, Bhasin_2014, Bhattacharjee_2008, CamachoDorado_2018, Csaky_1998e, DelgadoSalazar_2020c, DivsalarP._2023a, Emamhadi_2018, Farhadi_2024h, Fry_2010, Guinan_2019f, Hardy_2023g, Jehangir_2019h, Jin_2023, Kariholu_2008, Kobiela_2015, Kumar_2019f, Losanoff_1996, Losanoff_1997e, Mesfin_2022a, Misra_2013, Sobnach_2011f, Yasin_2009, teWildt_2010}, 32 cases (44\%) ingested a long object (\textgreater{}5cm) \cite{Al-Faham_2020k, AlShaaibi_2021b, Ali_2017, Ali_2022g, Atayan_2016, Bhasin_2014, CamachoDorado_2018, Chang_2017f, Cox_2007, Csaky_1998e, DivsalarP._2023a, Emamhadi_2018, Fry_2010, Gardner_2017h, Jin_2023, Kariholu_2008, Kerestes_2019, Kobiela_2015, Kumar_2019f, Mesfin_2022a, Misra_2013, Ohno_2005, Qureshi_2016, Sakellaridis_2008f, Sultan_2024f, Thapa_2019f, Trgo_2012f, Yasin_2009, Yildiz_2016e, teWildt_2010}, 9 cases (12\%) ingested a magnet \cite{Ali_2020f, Bhumi_2024f, Cauchi_2002, Liu_2005, Naji_2012f, Ohno_2005, Tanrikulu_2015e, Tay_2004, Wildhaber_2005}, 2 cases (3\%) ingested a button battery \cite{Berry_2021e, Bhumi_2024f}. \paragraph*{Outcomes} 48 cases (67\%) experienced a complication \cite{Ali_2017, Ali_2020f, Apikotoa_2022f, Atayan_2016, Beecroft_1998, Benoist_2019e, Berry_2021e, Bhasin_2014, Bhumi_2024f, CamachoDorado_2018, Cauchi_2002, Cox_2007, Csaky_1998e, DelgadoSalazar_2020c, DivsalarP._2023a, Emamhadi_2018, Farhadi_2024h, Fry_2010, Gardner_2017h, Goldman_1998f, Jin_2023, Kariholu_2008, Kerestes_2019, Kobiela_2015, Kumar_2001, Kumar_2019f, Liu_2005, Losanoff_1996, Mesfin_2022a, Misra_2013, Naji_2012f, Ohno_2005, Sakellaridis_2008f, Sobnach_2011f, Sultan_2024f, Tanrikulu_2015e, Tay_2004, Thapa_2019f, Trgo_2012f, Tupesis_2004f, Wildhaber_2005, Wnęk_2015f, Yasin_2009, Yildiz_2016e}, 44 cases (61\%) underwent surgery \cite{Al-Faham_2020k, AlShaaibi_2021b, Alao_2006i, Ali_2017, Ali_2020f, Atayan_2016, Beecroft_1998, Bhasin_2014, CamachoDorado_2018, Cauchi_2002, Chang_2017f, Cox_2007, Csaky_1998e, DelgadoSalazar_2020c, DivsalarP._2023a, Farhadi_2024h, Fry_2010, Gardner_2017h, Jin_2023, Kariholu_2008, Kerestes_2019, Kobiela_2015, Kumar_2019f, Liu_2005, Losanoff_1996, Losanoff_1997e, Mesfin_2022a, Misra_2013, Naji_2012f, Sobnach_2011f, Tanrikulu_2015e, Tay_2004, Thapa_2019f, Tupesis_2004f, Wildhaber_2005, Wnęk_2015f, Yasin_2009, Yildiz_2016e, fjbuilsRepeatedBehaviorDeliberate2024}, 31 cases (43\%) underwent endoscopy \cite{Akay_2015f, Ali_2022g, Apikotoa_2022f, Atayan_2016, Benoist_2019e, Berry_2021e, Bhasin_2014, Bhumi_2024f, CamachoDorado_2018, Chang_2017f, DelgadoSalazar_2020c, Gardner_2017h, Guinan_2019f, Hardy_2023g, Jehangir_2019h, Kariholu_2008, Li_2013, Liu_2005, Ohno_2005, Peixoto_2017f, Qureshi_2016, Riva_2018j, Sakellaridis_2008f, Sultan_2024f, Tammana_2012j, Tanrikulu_2015e, Trgo_2012f, Wadhwa_2015e, Wnęk_2015f, teWildt_2010}, 7 cases (10\%) were managed conservatively \cite{Ataya_2013, Bhattacharjee_2008, DivsalarP._2023a, Emamhadi_2018, Goldman_1998f, Kar_2015, Kumar_2001}, 2 cases (3\%) died \cite{Emamhadi_2018, Kumar_2001}. All 90 were male gender. 90 cases (100\%) were detained at the time of ingestion \cite{Elghali_2016, Karp_1991b, Lee_2007}, 88 cases (98\%) were intentional ingestions \cite{Elghali_2016, Karp_1991b, Lee_2007}, 30 cases (33\%) had a psychiatric history documented \cite{Elghali_2016, Karp_1991b, Lee_2007}, 2 cases (2\%) had a history of prior ingestion \cite{Elghali_2016}. No cases were reported for were psychiatric inpatients, were displaced people, were under the influence of alcohol at the time of ingestion, and had a severe disability history.
\paragraph*{Motivation}  70 cases (78\%) reported protest motivation \cite{Elghali_2016, Karp_1991b, Lee_2007}, 12 cases (13\%) reported psychiatric motivation \cite{Karp_1991b}, 6 cases (7\%) reported self-harm motivation \cite{Elghali_2016, Karp_1991b}. No cases were reported for psychosocial motivation and other motivation.
\paragraph*{Object Characteristics}  68 cases (76\%) involved sharp object ingestion \cite{Elghali_2016, Karp_1991b, Lee_2007}, 32 cases (36\%) involved long (\textgreater 5cm) object ingestion \cite{Lee_2007}, 25 cases (28\%) involved ingestion of multiple objects \cite{Elghali_2016, Lee_2007}. No cases were reported for button battery ingestion, magnet ingestion, and involved large diameter (\textgreater 2.5cm) object ingestion.
\paragraph*{Outcomes}  47 cases (52\%) underwent endoscopic intervention \cite{Elghali_2016, Lee_2007}, 29 cases (32\%) were managed conservatively \cite{Elghali_2016, Karp_1991b}, 15 cases (17\%) underwent surgical intervention \cite{Elghali_2016, Karp_1991b, Lee_2007}, 6 cases (7\%) reported complications \cite{Lee_2007}, 1 case (1\%) died \cite{Elghali_2016}.
\paragraph*{Geographical Location}Cases were recorded in 33 countries: 13 cases from USA \cite{Alao_2006i, Ataya_2013, Bhumi_2024f, Fry_2010, Guinan_2019f, Hardy_2023g, Jehangir_2019h, Kerestes_2019, Kumar_2001, Liu_2005, Tammana_2012j, Tay_2004, Tupesis_2004f}; 7 cases from India \cite{Bhasin_2014, Bhattacharjee_2008, Kar_2015, Kariholu_2008, Kumar_2019f, Misra_2013, Wadhwa_2015e} and UK \cite{Beecroft_1998, Berry_2021e, Cauchi_2002, Cox_2007, Gardner_2017h, Qureshi_2016}; 6 cases from Bulgaria \cite{Losanoff_1996, Losanoff_1997e}; 5 cases from Iran \cite{DivsalarP._2023a, Emamhadi_2018, Farhadi_2024h}; 4 cases from Turkey \cite{Akay_2015f, Atayan_2016, Tanrikulu_2015e, Yildiz_2016e}; 2 cases from China \cite{Jin_2023, Li_2013}, Poland \cite{Kobiela_2015, Wnęk_2015f}, and Spain \cite{CamachoDorado_2018, fjbuilsRepeatedBehaviorDeliberate2024}; 1 case from Australia \cite{Apikotoa_2022f}, Bahrain \cite{Ali_2020f}, Croatia \cite{Trgo_2012f}, Ecuador \cite{DelgadoSalazar_2020c}, Egypt \cite{Ali_2022g}, Ethiopia \cite{Mesfin_2022a}, Germany \cite{teWildt_2010}, Greece \cite{Sakellaridis_2008f}, Hungary \cite{Csaky_1998e}, Iraq \cite{Al-Faham_2020k}, Israel \cite{Goldman_1998f}, Italy \cite{Riva_2018j}, Japan \cite{Ohno_2005}, Nepal \cite{Thapa_2019f}, Netherlands \cite{Benoist_2019e}, Oman \cite{AlShaaibi_2021b}, Pakistan \cite{Yasin_2009}, Portugal \cite{Peixoto_2017f}, Qatar \cite{Ali_2017}, Saudi Arabia \cite{Sultan_2024f}, South Africa \cite{Sobnach_2011f}, Sweden \cite{Naji_2012f}, Switzerland \cite{Wildhaber_2005}, and Taiwan \cite{Chang_2017f}. \paragraph*{Gender} 43 cases (60\%) were male \cite{Akay_2015f, Al-Faham_2020k, Alao_2006i, Ali_2017, Ali_2022g, Apikotoa_2022f, Atayan_2016, Benoist_2019e, Berry_2021e, Bhumi_2024f, CamachoDorado_2018, Csaky_1998e, Emamhadi_2018, Farhadi_2024h, Fry_2010, Gardner_2017h, Guinan_2019f, Jehangir_2019h, Jin_2023, Kobiela_2015, Kumar_2001, Kumar_2019f, Liu_2005, Losanoff_1996, Losanoff_1997e, Mesfin_2022a, Misra_2013, Qureshi_2016, Riva_2018j, Sobnach_2011f, Tammana_2012j, Tanrikulu_2015e, Tay_2004, Thapa_2019f, Trgo_2012f, Wadhwa_2015e, Yasin_2009, teWildt_2010}, 28 cases (39\%) were female \cite{AlShaaibi_2021b, Ali_2020f, Ataya_2013, Beecroft_1998, Bhasin_2014, Bhattacharjee_2008, Cauchi_2002, Chang_2017f, Cox_2007, DelgadoSalazar_2020c, DivsalarP._2023a, Goldman_1998f, Hardy_2023g, Kar_2015, Kariholu_2008, Kerestes_2019, Li_2013, Naji_2012f, Ohno_2005, Peixoto_2017f, Sakellaridis_2008f, Sultan_2024f, Tupesis_2004f, Wildhaber_2005, Wnęk_2015f, Yildiz_2016e}, 1 case (1\%) had no gender recorded \cite{fjbuilsRepeatedBehaviorDeliberate2024}. \paragraph*{Age Group} 25 cases (35\%) were between 26 and 40 years of age \cite{Alao_2006i, Ali_2022g, Apikotoa_2022f, Ataya_2013, Benoist_2019e, Bhasin_2014, Chang_2017f, Cox_2007, DelgadoSalazar_2020c, Farhadi_2024h, Fry_2010, Gardner_2017h, Guinan_2019f, Jin_2023, Kumar_2019f, Losanoff_1996, Misra_2013, Qureshi_2016, Riva_2018j, Sakellaridis_2008f, Tammana_2012j, Trgo_2012f, Wnęk_2015f, Yildiz_2016e, fjbuilsRepeatedBehaviorDeliberate2024}, 18 cases (25\%) were between 18 and 25 years of age \cite{Akay_2015f, Ali_2017, Atayan_2016, Bhattacharjee_2008, Csaky_1998e, Kar_2015, Kariholu_2008, Kobiela_2015, Losanoff_1996, Losanoff_1997e, Mesfin_2022a, Peixoto_2017f, Sobnach_2011f, Tupesis_2004f, Yasin_2009}, 13 cases (18\%) were under 18 years of age \cite{AlShaaibi_2021b, Ali_2020f, Cauchi_2002, DivsalarP._2023a, Goldman_1998f, Liu_2005, Naji_2012f, Ohno_2005, Tanrikulu_2015e, Tay_2004, Wildhaber_2005}, 11 cases (15\%) were between 41 and 60 years of age \cite{Al-Faham_2020k, Bhumi_2024f, CamachoDorado_2018, Emamhadi_2018, Hardy_2023g, Jehangir_2019h, Kumar_2001, Sultan_2024f, Thapa_2019f, Wadhwa_2015e, teWildt_2010}, 3 cases (4\%) were over 60 years of age \cite{Beecroft_1998, Kerestes_2019, Li_2013}, 2 cases (3\%) had no age documented \cite{Berry_2021e}. \paragraph*{Population} 36 cases (50\%) had a psychiatric history \cite{AlShaaibi_2021b, Alao_2006i, Ali_2020f, Apikotoa_2022f, Ataya_2013, Atayan_2016, Beecroft_1998, CamachoDorado_2018, Chang_2017f, DelgadoSalazar_2020c, DivsalarP._2023a, Farhadi_2024h, Fry_2010, Guinan_2019f, Hardy_2023g, Jehangir_2019h, Jin_2023, Kar_2015, Kerestes_2019, Kobiela_2015, Kumar_2001, Kumar_2019f, Liu_2005, Mesfin_2022a, Misra_2013, Ohno_2005, Peixoto_2017f, Sakellaridis_2008f, Sultan_2024f, Tammana_2012j, Tanrikulu_2015e, Yildiz_2016e, fjbuilsRepeatedBehaviorDeliberate2024, teWildt_2010}, 19 cases (26\%) had ingested previously \cite{Alao_2006i, Apikotoa_2022f, Berry_2021e, Bhattacharjee_2008, Csaky_1998e, DivsalarP._2023a, Emamhadi_2018, Guinan_2019f, Jehangir_2019h, Jin_2023, Liu_2005, Sakellaridis_2008f, Tanrikulu_2015e, Thapa_2019f, Yildiz_2016e, fjbuilsRepeatedBehaviorDeliberate2024, teWildt_2010}, 12 cases (17\%) were detained persons \cite{Alao_2006i, Ali_2022g, Apikotoa_2022f, Losanoff_1996, Losanoff_1997e, Qureshi_2016, Tammana_2012j, Trgo_2012f}, 7 cases (10\%) were severely disabled \cite{Atayan_2016, Kerestes_2019, Liu_2005, Ohno_2005, Peixoto_2017f, Yildiz_2016e, teWildt_2010}, 4 cases (6\%) were psychiatric inpatients \cite{DivsalarP._2023a, fjbuilsRepeatedBehaviorDeliberate2024, teWildt_2010}, 3 cases (4\%) were under the influence of alcohol \cite{Benoist_2019e, Csaky_1998e, Thapa_2019f}, 2 cases (3\%) were displaced people \cite{Akay_2015f, Gardner_2017h}. \paragraph*{Motivation} 34 cases (47\%) had a psychiatric motivation \cite{Al-Faham_2020k, Alao_2006i, Ali_2020f, Apikotoa_2022f, Ataya_2013, Atayan_2016, Bhasin_2014, Bhattacharjee_2008, DelgadoSalazar_2020c, DivsalarP._2023a, Emamhadi_2018, Farhadi_2024h, Guinan_2019f, Hardy_2023g, Jehangir_2019h, Jin_2023, Kar_2015, Kariholu_2008, Kerestes_2019, Kobiela_2015, Kumar_2001, Kumar_2019f, Li_2013, Liu_2005, Misra_2013, Ohno_2005, Sakellaridis_2008f, Sultan_2024f, Tammana_2012j, Tanrikulu_2015e, Yasin_2009, teWildt_2010}, 21 cases (29\%) were motivated by self-harm intention \cite{Al-Faham_2020k, AlShaaibi_2021b, Alao_2006i, Ali_2017, CamachoDorado_2018, Chang_2017f, Cox_2007, Csaky_1998e, Fry_2010, Li_2013, Losanoff_1996, Losanoff_1997e, Mesfin_2022a, Sakellaridis_2008f, Tammana_2012j, Tanrikulu_2015e, fjbuilsRepeatedBehaviorDeliberate2024}, 17 cases (24\%) had a psychosocial motivation \cite{Akay_2015f, Benoist_2019e, Bhattacharjee_2008, Cauchi_2002, Goldman_1998f, Hardy_2023g, Kobiela_2015, Li_2013, Naji_2012f, Qureshi_2016, Riva_2018j, Sobnach_2011f, Tay_2004, Thapa_2019f, Tupesis_2004f, Wildhaber_2005, Wnęk_2015f}, 9 cases (12\%) were motivated by protest \cite{Bhumi_2024f, Gardner_2017h, Losanoff_1996, Losanoff_1997e, Tupesis_2004f}, 9 cases (12\%) had another documented motivation \cite{Ali_2020f, Ali_2022g, Emamhadi_2018, Guinan_2019f, Peixoto_2017f, Sakellaridis_2008f, Trgo_2012f, Wadhwa_2015e, Yildiz_2016e}. \paragraph*{Object Characteristics} 51 cases (71\%) ingested a large diameter object (\textgreater{}2.5cm) \cite{Akay_2015f, Al-Faham_2020k, AlShaaibi_2021b, Alao_2006i, Ali_2017, Ali_2022g, Apikotoa_2022f, Atayan_2016, Berry_2021e, Bhasin_2014, CamachoDorado_2018, Cauchi_2002, Chang_2017f, Cox_2007, Csaky_1998e, DivsalarP._2023a, Emamhadi_2018, Gardner_2017h, Guinan_2019f, Jehangir_2019h, Jin_2023, Kariholu_2008, Kerestes_2019, Kobiela_2015, Kumar_2001, Kumar_2019f, Losanoff_1996, Losanoff_1997e, Mesfin_2022a, Misra_2013, Naji_2012f, Ohno_2005, Peixoto_2017f, Qureshi_2016, Riva_2018j, Sakellaridis_2008f, Sultan_2024f, Tanrikulu_2015e, Thapa_2019f, Trgo_2012f, Wnęk_2015f, Yildiz_2016e, fjbuilsRepeatedBehaviorDeliberate2024, teWildt_2010}, 44 cases (61\%) ingested multiple objects \cite{Ali_2020f, Apikotoa_2022f, Ataya_2013, Atayan_2016, Beecroft_1998, Bhattacharjee_2008, Bhumi_2024f, CamachoDorado_2018, Cauchi_2002, Emamhadi_2018, Farhadi_2024h, Fry_2010, Goldman_1998f, Guinan_2019f, Hardy_2023g, Jehangir_2019h, Jin_2023, Kar_2015, Kariholu_2008, Kobiela_2015, Kumar_2001, Kumar_2019f, Li_2013, Liu_2005, Losanoff_1996, Mesfin_2022a, Misra_2013, Naji_2012f, Ohno_2005, Sobnach_2011f, Sultan_2024f, Tammana_2012j, Tanrikulu_2015e, Tay_2004, Thapa_2019f, Wadhwa_2015e, Wildhaber_2005, Yasin_2009, fjbuilsRepeatedBehaviorDeliberate2024, teWildt_2010}, 34 cases (47\%) ingested a sharp object \cite{AlShaaibi_2021b, Alao_2006i, Apikotoa_2022f, Ataya_2013, Benoist_2019e, Bhasin_2014, Bhattacharjee_2008, CamachoDorado_2018, Csaky_1998e, DelgadoSalazar_2020c, DivsalarP._2023a, Emamhadi_2018, Farhadi_2024h, Fry_2010, Guinan_2019f, Hardy_2023g, Jehangir_2019h, Jin_2023, Kariholu_2008, Kobiela_2015, Kumar_2019f, Losanoff_1996, Losanoff_1997e, Mesfin_2022a, Misra_2013, Sobnach_2011f, Yasin_2009, teWildt_2010}, 32 cases (44\%) ingested a long object (\textgreater{}5cm) \cite{Al-Faham_2020k, AlShaaibi_2021b, Ali_2017, Ali_2022g, Atayan_2016, Bhasin_2014, CamachoDorado_2018, Chang_2017f, Cox_2007, Csaky_1998e, DivsalarP._2023a, Emamhadi_2018, Fry_2010, Gardner_2017h, Jin_2023, Kariholu_2008, Kerestes_2019, Kobiela_2015, Kumar_2019f, Mesfin_2022a, Misra_2013, Ohno_2005, Qureshi_2016, Sakellaridis_2008f, Sultan_2024f, Thapa_2019f, Trgo_2012f, Yasin_2009, Yildiz_2016e, teWildt_2010}, 9 cases (12\%) ingested a magnet \cite{Ali_2020f, Bhumi_2024f, Cauchi_2002, Liu_2005, Naji_2012f, Ohno_2005, Tanrikulu_2015e, Tay_2004, Wildhaber_2005}, 2 cases (3\%) ingested a button battery \cite{Berry_2021e, Bhumi_2024f}. \paragraph*{Outcomes} 48 cases (67\%) experienced a complication \cite{Ali_2017, Ali_2020f, Apikotoa_2022f, Atayan_2016, Beecroft_1998, Benoist_2019e, Berry_2021e, Bhasin_2014, Bhumi_2024f, CamachoDorado_2018, Cauchi_2002, Cox_2007, Csaky_1998e, DelgadoSalazar_2020c, DivsalarP._2023a, Emamhadi_2018, Farhadi_2024h, Fry_2010, Gardner_2017h, Goldman_1998f, Jin_2023, Kariholu_2008, Kerestes_2019, Kobiela_2015, Kumar_2001, Kumar_2019f, Liu_2005, Losanoff_1996, Mesfin_2022a, Misra_2013, Naji_2012f, Ohno_2005, Sakellaridis_2008f, Sobnach_2011f, Sultan_2024f, Tanrikulu_2015e, Tay_2004, Thapa_2019f, Trgo_2012f, Tupesis_2004f, Wildhaber_2005, Wnęk_2015f, Yasin_2009, Yildiz_2016e}, 44 cases (61\%) underwent surgery \cite{Al-Faham_2020k, AlShaaibi_2021b, Alao_2006i, Ali_2017, Ali_2020f, Atayan_2016, Beecroft_1998, Bhasin_2014, CamachoDorado_2018, Cauchi_2002, Chang_2017f, Cox_2007, Csaky_1998e, DelgadoSalazar_2020c, DivsalarP._2023a, Farhadi_2024h, Fry_2010, Gardner_2017h, Jin_2023, Kariholu_2008, Kerestes_2019, Kobiela_2015, Kumar_2019f, Liu_2005, Losanoff_1996, Losanoff_1997e, Mesfin_2022a, Misra_2013, Naji_2012f, Sobnach_2011f, Tanrikulu_2015e, Tay_2004, Thapa_2019f, Tupesis_2004f, Wildhaber_2005, Wnęk_2015f, Yasin_2009, Yildiz_2016e, fjbuilsRepeatedBehaviorDeliberate2024}, 31 cases (43\%) underwent endoscopy \cite{Akay_2015f, Ali_2022g, Apikotoa_2022f, Atayan_2016, Benoist_2019e, Berry_2021e, Bhasin_2014, Bhumi_2024f, CamachoDorado_2018, Chang_2017f, DelgadoSalazar_2020c, Gardner_2017h, Guinan_2019f, Hardy_2023g, Jehangir_2019h, Kariholu_2008, Li_2013, Liu_2005, Ohno_2005, Peixoto_2017f, Qureshi_2016, Riva_2018j, Sakellaridis_2008f, Sultan_2024f, Tammana_2012j, Tanrikulu_2015e, Trgo_2012f, Wadhwa_2015e, Wnęk_2015f, teWildt_2010}, 7 cases (10\%) were managed conservatively \cite{Ataya_2013, Bhattacharjee_2008, DivsalarP._2023a, Emamhadi_2018, Goldman_1998f, Kar_2015, Kumar_2001}, 2 cases (3\%) died \cite{Emamhadi_2018, Kumar_2001}. All 90 were male gender. 90 cases (100\%) were detained at the time of ingestion \cite{Elghali_2016, Karp_1991b, Lee_2007}, 88 cases (98\%) were intentional ingestions \cite{Elghali_2016, Karp_1991b, Lee_2007}, 30 cases (33\%) had a psychiatric history documented \cite{Elghali_2016, Karp_1991b, Lee_2007}, 2 cases (2\%) had a history of prior ingestion \cite{Elghali_2016}. No cases were reported for were psychiatric inpatients, were displaced people, were under the influence of alcohol at the time of ingestion, and had a severe disability history.
\paragraph*{Motivation}  70 cases (78\%) reported protest motivation \cite{Elghali_2016, Karp_1991b, Lee_2007}, 12 cases (13\%) reported psychiatric motivation \cite{Karp_1991b}, 6 cases (7\%) reported self-harm motivation \cite{Elghali_2016, Karp_1991b}. No cases were reported for psychosocial motivation and other motivation.
\paragraph*{Object Characteristics}  68 cases (76\%) involved sharp object ingestion \cite{Elghali_2016, Karp_1991b, Lee_2007}, 32 cases (36\%) involved long (\textgreater 5cm) object ingestion \cite{Lee_2007}, 25 cases (28\%) involved ingestion of multiple objects \cite{Elghali_2016, Lee_2007}. No cases were reported for button battery ingestion, magnet ingestion, and involved large diameter (\textgreater 2.5cm) object ingestion.
\paragraph*{Outcomes}  47 cases (52\%) underwent endoscopic intervention \cite{Elghali_2016, Lee_2007}, 29 cases (32\%) were managed conservatively \cite{Elghali_2016, Karp_1991b}, 15 cases (17\%) underwent surgical intervention \cite{Elghali_2016, Karp_1991b, Lee_2007}, 6 cases (7\%) reported complications \cite{Lee_2007}, 1 case (1\%) died \cite{Elghali_2016}.
\paragraph*{Geographical Location}Cases were recorded in 33 countries: 13 cases from USA \cite{Alao_2006i, Ataya_2013, Bhumi_2024f, Fry_2010, Guinan_2019f, Hardy_2023g, Jehangir_2019h, Kerestes_2019, Kumar_2001, Liu_2005, Tammana_2012j, Tay_2004, Tupesis_2004f}; 7 cases from India \cite{Bhasin_2014, Bhattacharjee_2008, Kar_2015, Kariholu_2008, Kumar_2019f, Misra_2013, Wadhwa_2015e} and UK \cite{Beecroft_1998, Berry_2021e, Cauchi_2002, Cox_2007, Gardner_2017h, Qureshi_2016}; 6 cases from Bulgaria \cite{Losanoff_1996, Losanoff_1997e}; 5 cases from Iran \cite{DivsalarP._2023a, Emamhadi_2018, Farhadi_2024h}; 4 cases from Turkey \cite{Akay_2015f, Atayan_2016, Tanrikulu_2015e, Yildiz_2016e}; 2 cases from China \cite{Jin_2023, Li_2013}, Poland \cite{Kobiela_2015, Wnęk_2015f}, and Spain \cite{CamachoDorado_2018, fjbuilsRepeatedBehaviorDeliberate2024}; 1 case from Australia \cite{Apikotoa_2022f}, Bahrain \cite{Ali_2020f}, Croatia \cite{Trgo_2012f}, Ecuador \cite{DelgadoSalazar_2020c}, Egypt \cite{Ali_2022g}, Ethiopia \cite{Mesfin_2022a}, Germany \cite{teWildt_2010}, Greece \cite{Sakellaridis_2008f}, Hungary \cite{Csaky_1998e}, Iraq \cite{Al-Faham_2020k}, Israel \cite{Goldman_1998f}, Italy \cite{Riva_2018j}, Japan \cite{Ohno_2005}, Nepal \cite{Thapa_2019f}, Netherlands \cite{Benoist_2019e}, Oman \cite{AlShaaibi_2021b}, Pakistan \cite{Yasin_2009}, Portugal \cite{Peixoto_2017f}, Qatar \cite{Ali_2017}, Saudi Arabia \cite{Sultan_2024f}, South Africa \cite{Sobnach_2011f}, Sweden \cite{Naji_2012f}, Switzerland \cite{Wildhaber_2005}, and Taiwan \cite{Chang_2017f}. \paragraph*{Gender} 43 cases (60\%) were male \cite{Akay_2015f, Al-Faham_2020k, Alao_2006i, Ali_2017, Ali_2022g, Apikotoa_2022f, Atayan_2016, Benoist_2019e, Berry_2021e, Bhumi_2024f, CamachoDorado_2018, Csaky_1998e, Emamhadi_2018, Farhadi_2024h, Fry_2010, Gardner_2017h, Guinan_2019f, Jehangir_2019h, Jin_2023, Kobiela_2015, Kumar_2001, Kumar_2019f, Liu_2005, Losanoff_1996, Losanoff_1997e, Mesfin_2022a, Misra_2013, Qureshi_2016, Riva_2018j, Sobnach_2011f, Tammana_2012j, Tanrikulu_2015e, Tay_2004, Thapa_2019f, Trgo_2012f, Wadhwa_2015e, Yasin_2009, teWildt_2010}, 28 cases (39\%) were female \cite{AlShaaibi_2021b, Ali_2020f, Ataya_2013, Beecroft_1998, Bhasin_2014, Bhattacharjee_2008, Cauchi_2002, Chang_2017f, Cox_2007, DelgadoSalazar_2020c, DivsalarP._2023a, Goldman_1998f, Hardy_2023g, Kar_2015, Kariholu_2008, Kerestes_2019, Li_2013, Naji_2012f, Ohno_2005, Peixoto_2017f, Sakellaridis_2008f, Sultan_2024f, Tupesis_2004f, Wildhaber_2005, Wnęk_2015f, Yildiz_2016e}, 1 case (1\%) had no gender recorded \cite{fjbuilsRepeatedBehaviorDeliberate2024}. \paragraph*{Age Group} 25 cases (35\%) were between 26 and 40 years of age \cite{Alao_2006i, Ali_2022g, Apikotoa_2022f, Ataya_2013, Benoist_2019e, Bhasin_2014, Chang_2017f, Cox_2007, DelgadoSalazar_2020c, Farhadi_2024h, Fry_2010, Gardner_2017h, Guinan_2019f, Jin_2023, Kumar_2019f, Losanoff_1996, Misra_2013, Qureshi_2016, Riva_2018j, Sakellaridis_2008f, Tammana_2012j, Trgo_2012f, Wnęk_2015f, Yildiz_2016e, fjbuilsRepeatedBehaviorDeliberate2024}, 18 cases (25\%) were between 18 and 25 years of age \cite{Akay_2015f, Ali_2017, Atayan_2016, Bhattacharjee_2008, Csaky_1998e, Kar_2015, Kariholu_2008, Kobiela_2015, Losanoff_1996, Losanoff_1997e, Mesfin_2022a, Peixoto_2017f, Sobnach_2011f, Tupesis_2004f, Yasin_2009}, 13 cases (18\%) were under 18 years of age \cite{AlShaaibi_2021b, Ali_2020f, Cauchi_2002, DivsalarP._2023a, Goldman_1998f, Liu_2005, Naji_2012f, Ohno_2005, Tanrikulu_2015e, Tay_2004, Wildhaber_2005}, 11 cases (15\%) were between 41 and 60 years of age \cite{Al-Faham_2020k, Bhumi_2024f, CamachoDorado_2018, Emamhadi_2018, Hardy_2023g, Jehangir_2019h, Kumar_2001, Sultan_2024f, Thapa_2019f, Wadhwa_2015e, teWildt_2010}, 3 cases (4\%) were over 60 years of age \cite{Beecroft_1998, Kerestes_2019, Li_2013}, 2 cases (3\%) had no age documented \cite{Berry_2021e}. \paragraph*{Population} 36 cases (50\%) had a psychiatric history \cite{AlShaaibi_2021b, Alao_2006i, Ali_2020f, Apikotoa_2022f, Ataya_2013, Atayan_2016, Beecroft_1998, CamachoDorado_2018, Chang_2017f, DelgadoSalazar_2020c, DivsalarP._2023a, Farhadi_2024h, Fry_2010, Guinan_2019f, Hardy_2023g, Jehangir_2019h, Jin_2023, Kar_2015, Kerestes_2019, Kobiela_2015, Kumar_2001, Kumar_2019f, Liu_2005, Mesfin_2022a, Misra_2013, Ohno_2005, Peixoto_2017f, Sakellaridis_2008f, Sultan_2024f, Tammana_2012j, Tanrikulu_2015e, Yildiz_2016e, fjbuilsRepeatedBehaviorDeliberate2024, teWildt_2010}, 19 cases (26\%) had ingested previously \cite{Alao_2006i, Apikotoa_2022f, Berry_2021e, Bhattacharjee_2008, Csaky_1998e, DivsalarP._2023a, Emamhadi_2018, Guinan_2019f, Jehangir_2019h, Jin_2023, Liu_2005, Sakellaridis_2008f, Tanrikulu_2015e, Thapa_2019f, Yildiz_2016e, fjbuilsRepeatedBehaviorDeliberate2024, teWildt_2010}, 12 cases (17\%) were detained persons \cite{Alao_2006i, Ali_2022g, Apikotoa_2022f, Losanoff_1996, Losanoff_1997e, Qureshi_2016, Tammana_2012j, Trgo_2012f}, 7 cases (10\%) were severely disabled \cite{Atayan_2016, Kerestes_2019, Liu_2005, Ohno_2005, Peixoto_2017f, Yildiz_2016e, teWildt_2010}, 4 cases (6\%) were psychiatric inpatients \cite{DivsalarP._2023a, fjbuilsRepeatedBehaviorDeliberate2024, teWildt_2010}, 3 cases (4\%) were under the influence of alcohol \cite{Benoist_2019e, Csaky_1998e, Thapa_2019f}, 2 cases (3\%) were displaced people \cite{Akay_2015f, Gardner_2017h}. \paragraph*{Motivation} 34 cases (47\%) had a psychiatric motivation \cite{Al-Faham_2020k, Alao_2006i, Ali_2020f, Apikotoa_2022f, Ataya_2013, Atayan_2016, Bhasin_2014, Bhattacharjee_2008, DelgadoSalazar_2020c, DivsalarP._2023a, Emamhadi_2018, Farhadi_2024h, Guinan_2019f, Hardy_2023g, Jehangir_2019h, Jin_2023, Kar_2015, Kariholu_2008, Kerestes_2019, Kobiela_2015, Kumar_2001, Kumar_2019f, Li_2013, Liu_2005, Misra_2013, Ohno_2005, Sakellaridis_2008f, Sultan_2024f, Tammana_2012j, Tanrikulu_2015e, Yasin_2009, teWildt_2010}, 21 cases (29\%) were motivated by self-harm intention \cite{Al-Faham_2020k, AlShaaibi_2021b, Alao_2006i, Ali_2017, CamachoDorado_2018, Chang_2017f, Cox_2007, Csaky_1998e, Fry_2010, Li_2013, Losanoff_1996, Losanoff_1997e, Mesfin_2022a, Sakellaridis_2008f, Tammana_2012j, Tanrikulu_2015e, fjbuilsRepeatedBehaviorDeliberate2024}, 17 cases (24\%) had a psychosocial motivation \cite{Akay_2015f, Benoist_2019e, Bhattacharjee_2008, Cauchi_2002, Goldman_1998f, Hardy_2023g, Kobiela_2015, Li_2013, Naji_2012f, Qureshi_2016, Riva_2018j, Sobnach_2011f, Tay_2004, Thapa_2019f, Tupesis_2004f, Wildhaber_2005, Wnęk_2015f}, 9 cases (12\%) were motivated by protest \cite{Bhumi_2024f, Gardner_2017h, Losanoff_1996, Losanoff_1997e, Tupesis_2004f}, 9 cases (12\%) had another documented motivation \cite{Ali_2020f, Ali_2022g, Emamhadi_2018, Guinan_2019f, Peixoto_2017f, Sakellaridis_2008f, Trgo_2012f, Wadhwa_2015e, Yildiz_2016e}. \paragraph*{Object Characteristics} 51 cases (71\%) ingested a large diameter object (\textgreater{}2.5cm) \cite{Akay_2015f, Al-Faham_2020k, AlShaaibi_2021b, Alao_2006i, Ali_2017, Ali_2022g, Apikotoa_2022f, Atayan_2016, Berry_2021e, Bhasin_2014, CamachoDorado_2018, Cauchi_2002, Chang_2017f, Cox_2007, Csaky_1998e, DivsalarP._2023a, Emamhadi_2018, Gardner_2017h, Guinan_2019f, Jehangir_2019h, Jin_2023, Kariholu_2008, Kerestes_2019, Kobiela_2015, Kumar_2001, Kumar_2019f, Losanoff_1996, Losanoff_1997e, Mesfin_2022a, Misra_2013, Naji_2012f, Ohno_2005, Peixoto_2017f, Qureshi_2016, Riva_2018j, Sakellaridis_2008f, Sultan_2024f, Tanrikulu_2015e, Thapa_2019f, Trgo_2012f, Wnęk_2015f, Yildiz_2016e, fjbuilsRepeatedBehaviorDeliberate2024, teWildt_2010}, 44 cases (61\%) ingested multiple objects \cite{Ali_2020f, Apikotoa_2022f, Ataya_2013, Atayan_2016, Beecroft_1998, Bhattacharjee_2008, Bhumi_2024f, CamachoDorado_2018, Cauchi_2002, Emamhadi_2018, Farhadi_2024h, Fry_2010, Goldman_1998f, Guinan_2019f, Hardy_2023g, Jehangir_2019h, Jin_2023, Kar_2015, Kariholu_2008, Kobiela_2015, Kumar_2001, Kumar_2019f, Li_2013, Liu_2005, Losanoff_1996, Mesfin_2022a, Misra_2013, Naji_2012f, Ohno_2005, Sobnach_2011f, Sultan_2024f, Tammana_2012j, Tanrikulu_2015e, Tay_2004, Thapa_2019f, Wadhwa_2015e, Wildhaber_2005, Yasin_2009, fjbuilsRepeatedBehaviorDeliberate2024, teWildt_2010}, 34 cases (47\%) ingested a sharp object \cite{AlShaaibi_2021b, Alao_2006i, Apikotoa_2022f, Ataya_2013, Benoist_2019e, Bhasin_2014, Bhattacharjee_2008, CamachoDorado_2018, Csaky_1998e, DelgadoSalazar_2020c, DivsalarP._2023a, Emamhadi_2018, Farhadi_2024h, Fry_2010, Guinan_2019f, Hardy_2023g, Jehangir_2019h, Jin_2023, Kariholu_2008, Kobiela_2015, Kumar_2019f, Losanoff_1996, Losanoff_1997e, Mesfin_2022a, Misra_2013, Sobnach_2011f, Yasin_2009, teWildt_2010}, 32 cases (44\%) ingested a long object (\textgreater{}5cm) \cite{Al-Faham_2020k, AlShaaibi_2021b, Ali_2017, Ali_2022g, Atayan_2016, Bhasin_2014, CamachoDorado_2018, Chang_2017f, Cox_2007, Csaky_1998e, DivsalarP._2023a, Emamhadi_2018, Fry_2010, Gardner_2017h, Jin_2023, Kariholu_2008, Kerestes_2019, Kobiela_2015, Kumar_2019f, Mesfin_2022a, Misra_2013, Ohno_2005, Qureshi_2016, Sakellaridis_2008f, Sultan_2024f, Thapa_2019f, Trgo_2012f, Yasin_2009, Yildiz_2016e, teWildt_2010}, 9 cases (12\%) ingested a magnet \cite{Ali_2020f, Bhumi_2024f, Cauchi_2002, Liu_2005, Naji_2012f, Ohno_2005, Tanrikulu_2015e, Tay_2004, Wildhaber_2005}, 2 cases (3\%) ingested a button battery \cite{Berry_2021e, Bhumi_2024f}. \paragraph*{Outcomes} 48 cases (67\%) experienced a complication \cite{Ali_2017, Ali_2020f, Apikotoa_2022f, Atayan_2016, Beecroft_1998, Benoist_2019e, Berry_2021e, Bhasin_2014, Bhumi_2024f, CamachoDorado_2018, Cauchi_2002, Cox_2007, Csaky_1998e, DelgadoSalazar_2020c, DivsalarP._2023a, Emamhadi_2018, Farhadi_2024h, Fry_2010, Gardner_2017h, Goldman_1998f, Jin_2023, Kariholu_2008, Kerestes_2019, Kobiela_2015, Kumar_2001, Kumar_2019f, Liu_2005, Losanoff_1996, Mesfin_2022a, Misra_2013, Naji_2012f, Ohno_2005, Sakellaridis_2008f, Sobnach_2011f, Sultan_2024f, Tanrikulu_2015e, Tay_2004, Thapa_2019f, Trgo_2012f, Tupesis_2004f, Wildhaber_2005, Wnęk_2015f, Yasin_2009, Yildiz_2016e}, 44 cases (61\%) underwent surgery \cite{Al-Faham_2020k, AlShaaibi_2021b, Alao_2006i, Ali_2017, Ali_2020f, Atayan_2016, Beecroft_1998, Bhasin_2014, CamachoDorado_2018, Cauchi_2002, Chang_2017f, Cox_2007, Csaky_1998e, DelgadoSalazar_2020c, DivsalarP._2023a, Farhadi_2024h, Fry_2010, Gardner_2017h, Jin_2023, Kariholu_2008, Kerestes_2019, Kobiela_2015, Kumar_2019f, Liu_2005, Losanoff_1996, Losanoff_1997e, Mesfin_2022a, Misra_2013, Naji_2012f, Sobnach_2011f, Tanrikulu_2015e, Tay_2004, Thapa_2019f, Tupesis_2004f, Wildhaber_2005, Wnęk_2015f, Yasin_2009, Yildiz_2016e, fjbuilsRepeatedBehaviorDeliberate2024}, 31 cases (43\%) underwent endoscopy \cite{Akay_2015f, Ali_2022g, Apikotoa_2022f, Atayan_2016, Benoist_2019e, Berry_2021e, Bhasin_2014, Bhumi_2024f, CamachoDorado_2018, Chang_2017f, DelgadoSalazar_2020c, Gardner_2017h, Guinan_2019f, Hardy_2023g, Jehangir_2019h, Kariholu_2008, Li_2013, Liu_2005, Ohno_2005, Peixoto_2017f, Qureshi_2016, Riva_2018j, Sakellaridis_2008f, Sultan_2024f, Tammana_2012j, Tanrikulu_2015e, Trgo_2012f, Wadhwa_2015e, Wnęk_2015f, teWildt_2010}, 7 cases (10\%) were managed conservatively \cite{Ataya_2013, Bhattacharjee_2008, DivsalarP._2023a, Emamhadi_2018, Goldman_1998f, Kar_2015, Kumar_2001}, 2 cases (3\%) died \cite{Emamhadi_2018, Kumar_2001}. All 90 were male gender. 90 cases (100\%) were detained at the time of ingestion \cite{Elghali_2016, Karp_1991b, Lee_2007}, 88 cases (98\%) were intentional ingestions \cite{Elghali_2016, Karp_1991b, Lee_2007}, 30 cases (33\%) had a psychiatric history documented \cite{Elghali_2016, Karp_1991b, Lee_2007}, 2 cases (2\%) had a history of prior ingestion \cite{Elghali_2016}. No cases were reported for were psychiatric inpatients, were displaced people, were under the influence of alcohol at the time of ingestion, and had a severe disability history.
\paragraph*{Motivation}  70 cases (78\%) reported protest motivation \cite{Elghali_2016, Karp_1991b, Lee_2007}, 12 cases (13\%) reported psychiatric motivation \cite{Karp_1991b}, 6 cases (7\%) reported self-harm motivation \cite{Elghali_2016, Karp_1991b}. No cases were reported for psychosocial motivation and other motivation.
\paragraph*{Object Characteristics}  68 cases (76\%) involved sharp object ingestion \cite{Elghali_2016, Karp_1991b, Lee_2007}, 32 cases (36\%) involved long (\textgreater 5cm) object ingestion \cite{Lee_2007}, 25 cases (28\%) involved ingestion of multiple objects \cite{Elghali_2016, Lee_2007}. No cases were reported for button battery ingestion, magnet ingestion, and involved large diameter (\textgreater 2.5cm) object ingestion.
\paragraph*{Outcomes}  47 cases (52\%) underwent endoscopic intervention \cite{Elghali_2016, Lee_2007}, 29 cases (32\%) were managed conservatively \cite{Elghali_2016, Karp_1991b}, 15 cases (17\%) underwent surgical intervention \cite{Elghali_2016, Karp_1991b, Lee_2007}, 6 cases (7\%) reported complications \cite{Lee_2007}, 1 case (1\%) died \cite{Elghali_2016}.
\paragraph*{Geographical Location}Cases were recorded in 33 countries: 13 cases from USA \cite{Alao_2006i, Ataya_2013, Bhumi_2024f, Fry_2010, Guinan_2019f, Hardy_2023g, Jehangir_2019h, Kerestes_2019, Kumar_2001, Liu_2005, Tammana_2012j, Tay_2004, Tupesis_2004f}; 7 cases from India \cite{Bhasin_2014, Bhattacharjee_2008, Kar_2015, Kariholu_2008, Kumar_2019f, Misra_2013, Wadhwa_2015e} and UK \cite{Beecroft_1998, Berry_2021e, Cauchi_2002, Cox_2007, Gardner_2017h, Qureshi_2016}; 6 cases from Bulgaria \cite{Losanoff_1996, Losanoff_1997e}; 5 cases from Iran \cite{DivsalarP._2023a, Emamhadi_2018, Farhadi_2024h}; 4 cases from Turkey \cite{Akay_2015f, Atayan_2016, Tanrikulu_2015e, Yildiz_2016e}; 2 cases from China \cite{Jin_2023, Li_2013}, Poland \cite{Kobiela_2015, Wnęk_2015f}, and Spain \cite{CamachoDorado_2018, fjbuilsRepeatedBehaviorDeliberate2024}; 1 case from Australia \cite{Apikotoa_2022f}, Bahrain \cite{Ali_2020f}, Croatia \cite{Trgo_2012f}, Ecuador \cite{DelgadoSalazar_2020c}, Egypt \cite{Ali_2022g}, Ethiopia \cite{Mesfin_2022a}, Germany \cite{teWildt_2010}, Greece \cite{Sakellaridis_2008f}, Hungary \cite{Csaky_1998e}, Iraq \cite{Al-Faham_2020k}, Israel \cite{Goldman_1998f}, Italy \cite{Riva_2018j}, Japan \cite{Ohno_2005}, Nepal \cite{Thapa_2019f}, Netherlands \cite{Benoist_2019e}, Oman \cite{AlShaaibi_2021b}, Pakistan \cite{Yasin_2009}, Portugal \cite{Peixoto_2017f}, Qatar \cite{Ali_2017}, Saudi Arabia \cite{Sultan_2024f}, South Africa \cite{Sobnach_2011f}, Sweden \cite{Naji_2012f}, Switzerland \cite{Wildhaber_2005}, and Taiwan \cite{Chang_2017f}. \paragraph*{Gender} 43 cases (60\%) were male \cite{Akay_2015f, Al-Faham_2020k, Alao_2006i, Ali_2017, Ali_2022g, Apikotoa_2022f, Atayan_2016, Benoist_2019e, Berry_2021e, Bhumi_2024f, CamachoDorado_2018, Csaky_1998e, Emamhadi_2018, Farhadi_2024h, Fry_2010, Gardner_2017h, Guinan_2019f, Jehangir_2019h, Jin_2023, Kobiela_2015, Kumar_2001, Kumar_2019f, Liu_2005, Losanoff_1996, Losanoff_1997e, Mesfin_2022a, Misra_2013, Qureshi_2016, Riva_2018j, Sobnach_2011f, Tammana_2012j, Tanrikulu_2015e, Tay_2004, Thapa_2019f, Trgo_2012f, Wadhwa_2015e, Yasin_2009, teWildt_2010}, 28 cases (39\%) were female \cite{AlShaaibi_2021b, Ali_2020f, Ataya_2013, Beecroft_1998, Bhasin_2014, Bhattacharjee_2008, Cauchi_2002, Chang_2017f, Cox_2007, DelgadoSalazar_2020c, DivsalarP._2023a, Goldman_1998f, Hardy_2023g, Kar_2015, Kariholu_2008, Kerestes_2019, Li_2013, Naji_2012f, Ohno_2005, Peixoto_2017f, Sakellaridis_2008f, Sultan_2024f, Tupesis_2004f, Wildhaber_2005, Wnęk_2015f, Yildiz_2016e}, 1 case (1\%) had no gender recorded \cite{fjbuilsRepeatedBehaviorDeliberate2024}. \paragraph*{Age Group} 25 cases (35\%) were between 26 and 40 years of age \cite{Alao_2006i, Ali_2022g, Apikotoa_2022f, Ataya_2013, Benoist_2019e, Bhasin_2014, Chang_2017f, Cox_2007, DelgadoSalazar_2020c, Farhadi_2024h, Fry_2010, Gardner_2017h, Guinan_2019f, Jin_2023, Kumar_2019f, Losanoff_1996, Misra_2013, Qureshi_2016, Riva_2018j, Sakellaridis_2008f, Tammana_2012j, Trgo_2012f, Wnęk_2015f, Yildiz_2016e, fjbuilsRepeatedBehaviorDeliberate2024}, 18 cases (25\%) were between 18 and 25 years of age \cite{Akay_2015f, Ali_2017, Atayan_2016, Bhattacharjee_2008, Csaky_1998e, Kar_2015, Kariholu_2008, Kobiela_2015, Losanoff_1996, Losanoff_1997e, Mesfin_2022a, Peixoto_2017f, Sobnach_2011f, Tupesis_2004f, Yasin_2009}, 13 cases (18\%) were under 18 years of age \cite{AlShaaibi_2021b, Ali_2020f, Cauchi_2002, DivsalarP._2023a, Goldman_1998f, Liu_2005, Naji_2012f, Ohno_2005, Tanrikulu_2015e, Tay_2004, Wildhaber_2005}, 11 cases (15\%) were between 41 and 60 years of age \cite{Al-Faham_2020k, Bhumi_2024f, CamachoDorado_2018, Emamhadi_2018, Hardy_2023g, Jehangir_2019h, Kumar_2001, Sultan_2024f, Thapa_2019f, Wadhwa_2015e, teWildt_2010}, 3 cases (4\%) were over 60 years of age \cite{Beecroft_1998, Kerestes_2019, Li_2013}, 2 cases (3\%) had no age documented \cite{Berry_2021e}. \paragraph*{Population} 36 cases (50\%) had a psychiatric history \cite{AlShaaibi_2021b, Alao_2006i, Ali_2020f, Apikotoa_2022f, Ataya_2013, Atayan_2016, Beecroft_1998, CamachoDorado_2018, Chang_2017f, DelgadoSalazar_2020c, DivsalarP._2023a, Farhadi_2024h, Fry_2010, Guinan_2019f, Hardy_2023g, Jehangir_2019h, Jin_2023, Kar_2015, Kerestes_2019, Kobiela_2015, Kumar_2001, Kumar_2019f, Liu_2005, Mesfin_2022a, Misra_2013, Ohno_2005, Peixoto_2017f, Sakellaridis_2008f, Sultan_2024f, Tammana_2012j, Tanrikulu_2015e, Yildiz_2016e, fjbuilsRepeatedBehaviorDeliberate2024, teWildt_2010}, 19 cases (26\%) had ingested previously \cite{Alao_2006i, Apikotoa_2022f, Berry_2021e, Bhattacharjee_2008, Csaky_1998e, DivsalarP._2023a, Emamhadi_2018, Guinan_2019f, Jehangir_2019h, Jin_2023, Liu_2005, Sakellaridis_2008f, Tanrikulu_2015e, Thapa_2019f, Yildiz_2016e, fjbuilsRepeatedBehaviorDeliberate2024, teWildt_2010}, 12 cases (17\%) were detained persons \cite{Alao_2006i, Ali_2022g, Apikotoa_2022f, Losanoff_1996, Losanoff_1997e, Qureshi_2016, Tammana_2012j, Trgo_2012f}, 7 cases (10\%) were severely disabled \cite{Atayan_2016, Kerestes_2019, Liu_2005, Ohno_2005, Peixoto_2017f, Yildiz_2016e, teWildt_2010}, 4 cases (6\%) were psychiatric inpatients \cite{DivsalarP._2023a, fjbuilsRepeatedBehaviorDeliberate2024, teWildt_2010}, 3 cases (4\%) were under the influence of alcohol \cite{Benoist_2019e, Csaky_1998e, Thapa_2019f}, 2 cases (3\%) were displaced people \cite{Akay_2015f, Gardner_2017h}. \paragraph*{Motivation} 34 cases (47\%) had a psychiatric motivation \cite{Al-Faham_2020k, Alao_2006i, Ali_2020f, Apikotoa_2022f, Ataya_2013, Atayan_2016, Bhasin_2014, Bhattacharjee_2008, DelgadoSalazar_2020c, DivsalarP._2023a, Emamhadi_2018, Farhadi_2024h, Guinan_2019f, Hardy_2023g, Jehangir_2019h, Jin_2023, Kar_2015, Kariholu_2008, Kerestes_2019, Kobiela_2015, Kumar_2001, Kumar_2019f, Li_2013, Liu_2005, Misra_2013, Ohno_2005, Sakellaridis_2008f, Sultan_2024f, Tammana_2012j, Tanrikulu_2015e, Yasin_2009, teWildt_2010}, 21 cases (29\%) were motivated by self-harm intention \cite{Al-Faham_2020k, AlShaaibi_2021b, Alao_2006i, Ali_2017, CamachoDorado_2018, Chang_2017f, Cox_2007, Csaky_1998e, Fry_2010, Li_2013, Losanoff_1996, Losanoff_1997e, Mesfin_2022a, Sakellaridis_2008f, Tammana_2012j, Tanrikulu_2015e, fjbuilsRepeatedBehaviorDeliberate2024}, 17 cases (24\%) had a psychosocial motivation \cite{Akay_2015f, Benoist_2019e, Bhattacharjee_2008, Cauchi_2002, Goldman_1998f, Hardy_2023g, Kobiela_2015, Li_2013, Naji_2012f, Qureshi_2016, Riva_2018j, Sobnach_2011f, Tay_2004, Thapa_2019f, Tupesis_2004f, Wildhaber_2005, Wnęk_2015f}, 9 cases (12\%) were motivated by protest \cite{Bhumi_2024f, Gardner_2017h, Losanoff_1996, Losanoff_1997e, Tupesis_2004f}, 9 cases (12\%) had another documented motivation \cite{Ali_2020f, Ali_2022g, Emamhadi_2018, Guinan_2019f, Peixoto_2017f, Sakellaridis_2008f, Trgo_2012f, Wadhwa_2015e, Yildiz_2016e}. \paragraph*{Object Characteristics} 51 cases (71\%) ingested a large diameter object (\textgreater{}2.5cm) \cite{Akay_2015f, Al-Faham_2020k, AlShaaibi_2021b, Alao_2006i, Ali_2017, Ali_2022g, Apikotoa_2022f, Atayan_2016, Berry_2021e, Bhasin_2014, CamachoDorado_2018, Cauchi_2002, Chang_2017f, Cox_2007, Csaky_1998e, DivsalarP._2023a, Emamhadi_2018, Gardner_2017h, Guinan_2019f, Jehangir_2019h, Jin_2023, Kariholu_2008, Kerestes_2019, Kobiela_2015, Kumar_2001, Kumar_2019f, Losanoff_1996, Losanoff_1997e, Mesfin_2022a, Misra_2013, Naji_2012f, Ohno_2005, Peixoto_2017f, Qureshi_2016, Riva_2018j, Sakellaridis_2008f, Sultan_2024f, Tanrikulu_2015e, Thapa_2019f, Trgo_2012f, Wnęk_2015f, Yildiz_2016e, fjbuilsRepeatedBehaviorDeliberate2024, teWildt_2010}, 44 cases (61\%) ingested multiple objects \cite{Ali_2020f, Apikotoa_2022f, Ataya_2013, Atayan_2016, Beecroft_1998, Bhattacharjee_2008, Bhumi_2024f, CamachoDorado_2018, Cauchi_2002, Emamhadi_2018, Farhadi_2024h, Fry_2010, Goldman_1998f, Guinan_2019f, Hardy_2023g, Jehangir_2019h, Jin_2023, Kar_2015, Kariholu_2008, Kobiela_2015, Kumar_2001, Kumar_2019f, Li_2013, Liu_2005, Losanoff_1996, Mesfin_2022a, Misra_2013, Naji_2012f, Ohno_2005, Sobnach_2011f, Sultan_2024f, Tammana_2012j, Tanrikulu_2015e, Tay_2004, Thapa_2019f, Wadhwa_2015e, Wildhaber_2005, Yasin_2009, fjbuilsRepeatedBehaviorDeliberate2024, teWildt_2010}, 34 cases (47\%) ingested a sharp object \cite{AlShaaibi_2021b, Alao_2006i, Apikotoa_2022f, Ataya_2013, Benoist_2019e, Bhasin_2014, Bhattacharjee_2008, CamachoDorado_2018, Csaky_1998e, DelgadoSalazar_2020c, DivsalarP._2023a, Emamhadi_2018, Farhadi_2024h, Fry_2010, Guinan_2019f, Hardy_2023g, Jehangir_2019h, Jin_2023, Kariholu_2008, Kobiela_2015, Kumar_2019f, Losanoff_1996, Losanoff_1997e, Mesfin_2022a, Misra_2013, Sobnach_2011f, Yasin_2009, teWildt_2010}, 32 cases (44\%) ingested a long object (\textgreater{}5cm) \cite{Al-Faham_2020k, AlShaaibi_2021b, Ali_2017, Ali_2022g, Atayan_2016, Bhasin_2014, CamachoDorado_2018, Chang_2017f, Cox_2007, Csaky_1998e, DivsalarP._2023a, Emamhadi_2018, Fry_2010, Gardner_2017h, Jin_2023, Kariholu_2008, Kerestes_2019, Kobiela_2015, Kumar_2019f, Mesfin_2022a, Misra_2013, Ohno_2005, Qureshi_2016, Sakellaridis_2008f, Sultan_2024f, Thapa_2019f, Trgo_2012f, Yasin_2009, Yildiz_2016e, teWildt_2010}, 9 cases (12\%) ingested a magnet \cite{Ali_2020f, Bhumi_2024f, Cauchi_2002, Liu_2005, Naji_2012f, Ohno_2005, Tanrikulu_2015e, Tay_2004, Wildhaber_2005}, 2 cases (3\%) ingested a button battery \cite{Berry_2021e, Bhumi_2024f}. \paragraph*{Outcomes} 48 cases (67\%) experienced a complication \cite{Ali_2017, Ali_2020f, Apikotoa_2022f, Atayan_2016, Beecroft_1998, Benoist_2019e, Berry_2021e, Bhasin_2014, Bhumi_2024f, CamachoDorado_2018, Cauchi_2002, Cox_2007, Csaky_1998e, DelgadoSalazar_2020c, DivsalarP._2023a, Emamhadi_2018, Farhadi_2024h, Fry_2010, Gardner_2017h, Goldman_1998f, Jin_2023, Kariholu_2008, Kerestes_2019, Kobiela_2015, Kumar_2001, Kumar_2019f, Liu_2005, Losanoff_1996, Mesfin_2022a, Misra_2013, Naji_2012f, Ohno_2005, Sakellaridis_2008f, Sobnach_2011f, Sultan_2024f, Tanrikulu_2015e, Tay_2004, Thapa_2019f, Trgo_2012f, Tupesis_2004f, Wildhaber_2005, Wnęk_2015f, Yasin_2009, Yildiz_2016e}, 44 cases (61\%) underwent surgery \cite{Al-Faham_2020k, AlShaaibi_2021b, Alao_2006i, Ali_2017, Ali_2020f, Atayan_2016, Beecroft_1998, Bhasin_2014, CamachoDorado_2018, Cauchi_2002, Chang_2017f, Cox_2007, Csaky_1998e, DelgadoSalazar_2020c, DivsalarP._2023a, Farhadi_2024h, Fry_2010, Gardner_2017h, Jin_2023, Kariholu_2008, Kerestes_2019, Kobiela_2015, Kumar_2019f, Liu_2005, Losanoff_1996, Losanoff_1997e, Mesfin_2022a, Misra_2013, Naji_2012f, Sobnach_2011f, Tanrikulu_2015e, Tay_2004, Thapa_2019f, Tupesis_2004f, Wildhaber_2005, Wnęk_2015f, Yasin_2009, Yildiz_2016e, fjbuilsRepeatedBehaviorDeliberate2024}, 31 cases (43\%) underwent endoscopy \cite{Akay_2015f, Ali_2022g, Apikotoa_2022f, Atayan_2016, Benoist_2019e, Berry_2021e, Bhasin_2014, Bhumi_2024f, CamachoDorado_2018, Chang_2017f, DelgadoSalazar_2020c, Gardner_2017h, Guinan_2019f, Hardy_2023g, Jehangir_2019h, Kariholu_2008, Li_2013, Liu_2005, Ohno_2005, Peixoto_2017f, Qureshi_2016, Riva_2018j, Sakellaridis_2008f, Sultan_2024f, Tammana_2012j, Tanrikulu_2015e, Trgo_2012f, Wadhwa_2015e, Wnęk_2015f, teWildt_2010}, 7 cases (10\%) were managed conservatively \cite{Ataya_2013, Bhattacharjee_2008, DivsalarP._2023a, Emamhadi_2018, Goldman_1998f, Kar_2015, Kumar_2001}, 2 cases (3\%) died \cite{Emamhadi_2018, Kumar_2001}. All 90 were male gender. 90 cases (100\%) were detained at the time of ingestion \cite{Elghali_2016, Karp_1991b, Lee_2007}, 88 cases (98\%) were intentional ingestions \cite{Elghali_2016, Karp_1991b, Lee_2007}, 30 cases (33\%) had a psychiatric history documented \cite{Elghali_2016, Karp_1991b, Lee_2007}, 2 cases (2\%) had a history of prior ingestion \cite{Elghali_2016}. No cases were reported for were psychiatric inpatients, were displaced people, were under the influence of alcohol at the time of ingestion, and had a severe disability history.
\paragraph*{Motivation}  70 cases (78\%) reported protest motivation \cite{Elghali_2016, Karp_1991b, Lee_2007}, 12 cases (13\%) reported psychiatric motivation \cite{Karp_1991b}, 6 cases (7\%) reported self-harm motivation \cite{Elghali_2016, Karp_1991b}. No cases were reported for psychosocial motivation and other motivation.
\paragraph*{Object Characteristics}  68 cases (76\%) involved sharp object ingestion \cite{Elghali_2016, Karp_1991b, Lee_2007}, 32 cases (36\%) involved long (\textgreater 5cm) object ingestion \cite{Lee_2007}, 25 cases (28\%) involved ingestion of multiple objects \cite{Elghali_2016, Lee_2007}. No cases were reported for button battery ingestion, magnet ingestion, and involved large diameter (\textgreater 2.5cm) object ingestion.
\paragraph*{Outcomes}  47 cases (52\%) underwent endoscopic intervention \cite{Elghali_2016, Lee_2007}, 29 cases (32\%) were managed conservatively \cite{Elghali_2016, Karp_1991b}, 15 cases (17\%) underwent surgical intervention \cite{Elghali_2016, Karp_1991b, Lee_2007}, 6 cases (7\%) reported complications \cite{Lee_2007}, 1 case (1\%) died \cite{Elghali_2016}.
\paragraph*{Geographical Location}Cases were recorded in 33 countries: 13 cases from USA \cite{Alao_2006i, Ataya_2013, Bhumi_2024f, Fry_2010, Guinan_2019f, Hardy_2023g, Jehangir_2019h, Kerestes_2019, Kumar_2001, Liu_2005, Tammana_2012j, Tay_2004, Tupesis_2004f}; 7 cases from India \cite{Bhasin_2014, Bhattacharjee_2008, Kar_2015, Kariholu_2008, Kumar_2019f, Misra_2013, Wadhwa_2015e} and UK \cite{Beecroft_1998, Berry_2021e, Cauchi_2002, Cox_2007, Gardner_2017h, Qureshi_2016}; 6 cases from Bulgaria \cite{Losanoff_1996, Losanoff_1997e}; 5 cases from Iran \cite{DivsalarP._2023a, Emamhadi_2018, Farhadi_2024h}; 4 cases from Turkey \cite{Akay_2015f, Atayan_2016, Tanrikulu_2015e, Yildiz_2016e}; 2 cases from China \cite{Jin_2023, Li_2013}, Poland \cite{Kobiela_2015, Wnęk_2015f}, and Spain \cite{CamachoDorado_2018, fjbuilsRepeatedBehaviorDeliberate2024}; 1 case from Australia \cite{Apikotoa_2022f}, Bahrain \cite{Ali_2020f}, Croatia \cite{Trgo_2012f}, Ecuador \cite{DelgadoSalazar_2020c}, Egypt \cite{Ali_2022g}, Ethiopia \cite{Mesfin_2022a}, Germany \cite{teWildt_2010}, Greece \cite{Sakellaridis_2008f}, Hungary \cite{Csaky_1998e}, Iraq \cite{Al-Faham_2020k}, Israel \cite{Goldman_1998f}, Italy \cite{Riva_2018j}, Japan \cite{Ohno_2005}, Nepal \cite{Thapa_2019f}, Netherlands \cite{Benoist_2019e}, Oman \cite{AlShaaibi_2021b}, Pakistan \cite{Yasin_2009}, Portugal \cite{Peixoto_2017f}, Qatar \cite{Ali_2017}, Saudi Arabia \cite{Sultan_2024f}, South Africa \cite{Sobnach_2011f}, Sweden \cite{Naji_2012f}, Switzerland \cite{Wildhaber_2005}, and Taiwan \cite{Chang_2017f}. \paragraph*{Gender} 43 cases (60\%) were male \cite{Akay_2015f, Al-Faham_2020k, Alao_2006i, Ali_2017, Ali_2022g, Apikotoa_2022f, Atayan_2016, Benoist_2019e, Berry_2021e, Bhumi_2024f, CamachoDorado_2018, Csaky_1998e, Emamhadi_2018, Farhadi_2024h, Fry_2010, Gardner_2017h, Guinan_2019f, Jehangir_2019h, Jin_2023, Kobiela_2015, Kumar_2001, Kumar_2019f, Liu_2005, Losanoff_1996, Losanoff_1997e, Mesfin_2022a, Misra_2013, Qureshi_2016, Riva_2018j, Sobnach_2011f, Tammana_2012j, Tanrikulu_2015e, Tay_2004, Thapa_2019f, Trgo_2012f, Wadhwa_2015e, Yasin_2009, teWildt_2010}, 28 cases (39\%) were female \cite{AlShaaibi_2021b, Ali_2020f, Ataya_2013, Beecroft_1998, Bhasin_2014, Bhattacharjee_2008, Cauchi_2002, Chang_2017f, Cox_2007, DelgadoSalazar_2020c, DivsalarP._2023a, Goldman_1998f, Hardy_2023g, Kar_2015, Kariholu_2008, Kerestes_2019, Li_2013, Naji_2012f, Ohno_2005, Peixoto_2017f, Sakellaridis_2008f, Sultan_2024f, Tupesis_2004f, Wildhaber_2005, Wnęk_2015f, Yildiz_2016e}, 1 case (1\%) had no gender recorded \cite{fjbuilsRepeatedBehaviorDeliberate2024}. \paragraph*{Age Group} 25 cases (35\%) were between 26 and 40 years of age \cite{Alao_2006i, Ali_2022g, Apikotoa_2022f, Ataya_2013, Benoist_2019e, Bhasin_2014, Chang_2017f, Cox_2007, DelgadoSalazar_2020c, Farhadi_2024h, Fry_2010, Gardner_2017h, Guinan_2019f, Jin_2023, Kumar_2019f, Losanoff_1996, Misra_2013, Qureshi_2016, Riva_2018j, Sakellaridis_2008f, Tammana_2012j, Trgo_2012f, Wnęk_2015f, Yildiz_2016e, fjbuilsRepeatedBehaviorDeliberate2024}, 18 cases (25\%) were between 18 and 25 years of age \cite{Akay_2015f, Ali_2017, Atayan_2016, Bhattacharjee_2008, Csaky_1998e, Kar_2015, Kariholu_2008, Kobiela_2015, Losanoff_1996, Losanoff_1997e, Mesfin_2022a, Peixoto_2017f, Sobnach_2011f, Tupesis_2004f, Yasin_2009}, 13 cases (18\%) were under 18 years of age \cite{AlShaaibi_2021b, Ali_2020f, Cauchi_2002, DivsalarP._2023a, Goldman_1998f, Liu_2005, Naji_2012f, Ohno_2005, Tanrikulu_2015e, Tay_2004, Wildhaber_2005}, 11 cases (15\%) were between 41 and 60 years of age \cite{Al-Faham_2020k, Bhumi_2024f, CamachoDorado_2018, Emamhadi_2018, Hardy_2023g, Jehangir_2019h, Kumar_2001, Sultan_2024f, Thapa_2019f, Wadhwa_2015e, teWildt_2010}, 3 cases (4\%) were over 60 years of age \cite{Beecroft_1998, Kerestes_2019, Li_2013}, 2 cases (3\%) had no age documented \cite{Berry_2021e}. \paragraph*{Population} 36 cases (50\%) had a psychiatric history \cite{AlShaaibi_2021b, Alao_2006i, Ali_2020f, Apikotoa_2022f, Ataya_2013, Atayan_2016, Beecroft_1998, CamachoDorado_2018, Chang_2017f, DelgadoSalazar_2020c, DivsalarP._2023a, Farhadi_2024h, Fry_2010, Guinan_2019f, Hardy_2023g, Jehangir_2019h, Jin_2023, Kar_2015, Kerestes_2019, Kobiela_2015, Kumar_2001, Kumar_2019f, Liu_2005, Mesfin_2022a, Misra_2013, Ohno_2005, Peixoto_2017f, Sakellaridis_2008f, Sultan_2024f, Tammana_2012j, Tanrikulu_2015e, Yildiz_2016e, fjbuilsRepeatedBehaviorDeliberate2024, teWildt_2010}, 19 cases (26\%) had ingested previously \cite{Alao_2006i, Apikotoa_2022f, Berry_2021e, Bhattacharjee_2008, Csaky_1998e, DivsalarP._2023a, Emamhadi_2018, Guinan_2019f, Jehangir_2019h, Jin_2023, Liu_2005, Sakellaridis_2008f, Tanrikulu_2015e, Thapa_2019f, Yildiz_2016e, fjbuilsRepeatedBehaviorDeliberate2024, teWildt_2010}, 12 cases (17\%) were detained persons \cite{Alao_2006i, Ali_2022g, Apikotoa_2022f, Losanoff_1996, Losanoff_1997e, Qureshi_2016, Tammana_2012j, Trgo_2012f}, 7 cases (10\%) were severely disabled \cite{Atayan_2016, Kerestes_2019, Liu_2005, Ohno_2005, Peixoto_2017f, Yildiz_2016e, teWildt_2010}, 4 cases (6\%) were psychiatric inpatients \cite{DivsalarP._2023a, fjbuilsRepeatedBehaviorDeliberate2024, teWildt_2010}, 3 cases (4\%) were under the influence of alcohol \cite{Benoist_2019e, Csaky_1998e, Thapa_2019f}, 2 cases (3\%) were displaced people \cite{Akay_2015f, Gardner_2017h}. \paragraph*{Motivation} 34 cases (47\%) had a psychiatric motivation \cite{Al-Faham_2020k, Alao_2006i, Ali_2020f, Apikotoa_2022f, Ataya_2013, Atayan_2016, Bhasin_2014, Bhattacharjee_2008, DelgadoSalazar_2020c, DivsalarP._2023a, Emamhadi_2018, Farhadi_2024h, Guinan_2019f, Hardy_2023g, Jehangir_2019h, Jin_2023, Kar_2015, Kariholu_2008, Kerestes_2019, Kobiela_2015, Kumar_2001, Kumar_2019f, Li_2013, Liu_2005, Misra_2013, Ohno_2005, Sakellaridis_2008f, Sultan_2024f, Tammana_2012j, Tanrikulu_2015e, Yasin_2009, teWildt_2010}, 21 cases (29\%) were motivated by self-harm intention \cite{Al-Faham_2020k, AlShaaibi_2021b, Alao_2006i, Ali_2017, CamachoDorado_2018, Chang_2017f, Cox_2007, Csaky_1998e, Fry_2010, Li_2013, Losanoff_1996, Losanoff_1997e, Mesfin_2022a, Sakellaridis_2008f, Tammana_2012j, Tanrikulu_2015e, fjbuilsRepeatedBehaviorDeliberate2024}, 17 cases (24\%) had a psychosocial motivation \cite{Akay_2015f, Benoist_2019e, Bhattacharjee_2008, Cauchi_2002, Goldman_1998f, Hardy_2023g, Kobiela_2015, Li_2013, Naji_2012f, Qureshi_2016, Riva_2018j, Sobnach_2011f, Tay_2004, Thapa_2019f, Tupesis_2004f, Wildhaber_2005, Wnęk_2015f}, 9 cases (12\%) were motivated by protest \cite{Bhumi_2024f, Gardner_2017h, Losanoff_1996, Losanoff_1997e, Tupesis_2004f}, 9 cases (12\%) had another documented motivation \cite{Ali_2020f, Ali_2022g, Emamhadi_2018, Guinan_2019f, Peixoto_2017f, Sakellaridis_2008f, Trgo_2012f, Wadhwa_2015e, Yildiz_2016e}. \paragraph*{Object Characteristics} 51 cases (71\%) ingested a large diameter object (\textgreater{}2.5cm) \cite{Akay_2015f, Al-Faham_2020k, AlShaaibi_2021b, Alao_2006i, Ali_2017, Ali_2022g, Apikotoa_2022f, Atayan_2016, Berry_2021e, Bhasin_2014, CamachoDorado_2018, Cauchi_2002, Chang_2017f, Cox_2007, Csaky_1998e, DivsalarP._2023a, Emamhadi_2018, Gardner_2017h, Guinan_2019f, Jehangir_2019h, Jin_2023, Kariholu_2008, Kerestes_2019, Kobiela_2015, Kumar_2001, Kumar_2019f, Losanoff_1996, Losanoff_1997e, Mesfin_2022a, Misra_2013, Naji_2012f, Ohno_2005, Peixoto_2017f, Qureshi_2016, Riva_2018j, Sakellaridis_2008f, Sultan_2024f, Tanrikulu_2015e, Thapa_2019f, Trgo_2012f, Wnęk_2015f, Yildiz_2016e, fjbuilsRepeatedBehaviorDeliberate2024, teWildt_2010}, 44 cases (61\%) ingested multiple objects \cite{Ali_2020f, Apikotoa_2022f, Ataya_2013, Atayan_2016, Beecroft_1998, Bhattacharjee_2008, Bhumi_2024f, CamachoDorado_2018, Cauchi_2002, Emamhadi_2018, Farhadi_2024h, Fry_2010, Goldman_1998f, Guinan_2019f, Hardy_2023g, Jehangir_2019h, Jin_2023, Kar_2015, Kariholu_2008, Kobiela_2015, Kumar_2001, Kumar_2019f, Li_2013, Liu_2005, Losanoff_1996, Mesfin_2022a, Misra_2013, Naji_2012f, Ohno_2005, Sobnach_2011f, Sultan_2024f, Tammana_2012j, Tanrikulu_2015e, Tay_2004, Thapa_2019f, Wadhwa_2015e, Wildhaber_2005, Yasin_2009, fjbuilsRepeatedBehaviorDeliberate2024, teWildt_2010}, 34 cases (47\%) ingested a sharp object \cite{AlShaaibi_2021b, Alao_2006i, Apikotoa_2022f, Ataya_2013, Benoist_2019e, Bhasin_2014, Bhattacharjee_2008, CamachoDorado_2018, Csaky_1998e, DelgadoSalazar_2020c, DivsalarP._2023a, Emamhadi_2018, Farhadi_2024h, Fry_2010, Guinan_2019f, Hardy_2023g, Jehangir_2019h, Jin_2023, Kariholu_2008, Kobiela_2015, Kumar_2019f, Losanoff_1996, Losanoff_1997e, Mesfin_2022a, Misra_2013, Sobnach_2011f, Yasin_2009, teWildt_2010}, 32 cases (44\%) ingested a long object (\textgreater{}5cm) \cite{Al-Faham_2020k, AlShaaibi_2021b, Ali_2017, Ali_2022g, Atayan_2016, Bhasin_2014, CamachoDorado_2018, Chang_2017f, Cox_2007, Csaky_1998e, DivsalarP._2023a, Emamhadi_2018, Fry_2010, Gardner_2017h, Jin_2023, Kariholu_2008, Kerestes_2019, Kobiela_2015, Kumar_2019f, Mesfin_2022a, Misra_2013, Ohno_2005, Qureshi_2016, Sakellaridis_2008f, Sultan_2024f, Thapa_2019f, Trgo_2012f, Yasin_2009, Yildiz_2016e, teWildt_2010}, 9 cases (12\%) ingested a magnet \cite{Ali_2020f, Bhumi_2024f, Cauchi_2002, Liu_2005, Naji_2012f, Ohno_2005, Tanrikulu_2015e, Tay_2004, Wildhaber_2005}, 2 cases (3\%) ingested a button battery \cite{Berry_2021e, Bhumi_2024f}. \paragraph*{Outcomes} 48 cases (67\%) experienced a complication \cite{Ali_2017, Ali_2020f, Apikotoa_2022f, Atayan_2016, Beecroft_1998, Benoist_2019e, Berry_2021e, Bhasin_2014, Bhumi_2024f, CamachoDorado_2018, Cauchi_2002, Cox_2007, Csaky_1998e, DelgadoSalazar_2020c, DivsalarP._2023a, Emamhadi_2018, Farhadi_2024h, Fry_2010, Gardner_2017h, Goldman_1998f, Jin_2023, Kariholu_2008, Kerestes_2019, Kobiela_2015, Kumar_2001, Kumar_2019f, Liu_2005, Losanoff_1996, Mesfin_2022a, Misra_2013, Naji_2012f, Ohno_2005, Sakellaridis_2008f, Sobnach_2011f, Sultan_2024f, Tanrikulu_2015e, Tay_2004, Thapa_2019f, Trgo_2012f, Tupesis_2004f, Wildhaber_2005, Wnęk_2015f, Yasin_2009, Yildiz_2016e}, 44 cases (61\%) underwent surgery \cite{Al-Faham_2020k, AlShaaibi_2021b, Alao_2006i, Ali_2017, Ali_2020f, Atayan_2016, Beecroft_1998, Bhasin_2014, CamachoDorado_2018, Cauchi_2002, Chang_2017f, Cox_2007, Csaky_1998e, DelgadoSalazar_2020c, DivsalarP._2023a, Farhadi_2024h, Fry_2010, Gardner_2017h, Jin_2023, Kariholu_2008, Kerestes_2019, Kobiela_2015, Kumar_2019f, Liu_2005, Losanoff_1996, Losanoff_1997e, Mesfin_2022a, Misra_2013, Naji_2012f, Sobnach_2011f, Tanrikulu_2015e, Tay_2004, Thapa_2019f, Tupesis_2004f, Wildhaber_2005, Wnęk_2015f, Yasin_2009, Yildiz_2016e, fjbuilsRepeatedBehaviorDeliberate2024}, 31 cases (43\%) underwent endoscopy \cite{Akay_2015f, Ali_2022g, Apikotoa_2022f, Atayan_2016, Benoist_2019e, Berry_2021e, Bhasin_2014, Bhumi_2024f, CamachoDorado_2018, Chang_2017f, DelgadoSalazar_2020c, Gardner_2017h, Guinan_2019f, Hardy_2023g, Jehangir_2019h, Kariholu_2008, Li_2013, Liu_2005, Ohno_2005, Peixoto_2017f, Qureshi_2016, Riva_2018j, Sakellaridis_2008f, Sultan_2024f, Tammana_2012j, Tanrikulu_2015e, Trgo_2012f, Wadhwa_2015e, Wnęk_2015f, teWildt_2010}, 7 cases (10\%) were managed conservatively \cite{Ataya_2013, Bhattacharjee_2008, DivsalarP._2023a, Emamhadi_2018, Goldman_1998f, Kar_2015, Kumar_2001}, 2 cases (3\%) died \cite{Emamhadi_2018, Kumar_2001}. All 90 were male gender. 90 cases (100\%) were detained at the time of ingestion \cite{Elghali_2016, Karp_1991b, Lee_2007}, 88 cases (98\%) were intentional ingestions \cite{Elghali_2016, Karp_1991b, Lee_2007}, 30 cases (33\%) had a psychiatric history documented \cite{Elghali_2016, Karp_1991b, Lee_2007}, 2 cases (2\%) had a history of prior ingestion \cite{Elghali_2016}. No cases were reported for were psychiatric inpatients, were displaced people, were under the influence of alcohol at the time of ingestion, and had a severe disability history.
\paragraph*{Motivation}  70 cases (78\%) reported protest motivation \cite{Elghali_2016, Karp_1991b, Lee_2007}, 12 cases (13\%) reported psychiatric motivation \cite{Karp_1991b}, 6 cases (7\%) reported self-harm motivation \cite{Elghali_2016, Karp_1991b}. No cases were reported for psychosocial motivation and other motivation.
\paragraph*{Object Characteristics}  68 cases (76\%) involved sharp object ingestion \cite{Elghali_2016, Karp_1991b, Lee_2007}, 32 cases (36\%) involved long (\textgreater 5cm) object ingestion \cite{Lee_2007}, 25 cases (28\%) involved ingestion of multiple objects \cite{Elghali_2016, Lee_2007}. No cases were reported for button battery ingestion, magnet ingestion, and involved large diameter (\textgreater 2.5cm) object ingestion.
\paragraph*{Outcomes}  47 cases (52\%) underwent endoscopic intervention \cite{Elghali_2016, Lee_2007}, 29 cases (32\%) were managed conservatively \cite{Elghali_2016, Karp_1991b}, 15 cases (17\%) underwent surgical intervention \cite{Elghali_2016, Karp_1991b, Lee_2007}, 6 cases (7\%) reported complications \cite{Lee_2007}, 1 case (1\%) died \cite{Elghali_2016}.
\paragraph*{Geographical Location}Cases were recorded in 33 countries: 13 cases from USA \cite{Alao_2006i, Ataya_2013, Bhumi_2024f, Fry_2010, Guinan_2019f, Hardy_2023g, Jehangir_2019h, Kerestes_2019, Kumar_2001, Liu_2005, Tammana_2012j, Tay_2004, Tupesis_2004f}; 7 cases from India \cite{Bhasin_2014, Bhattacharjee_2008, Kar_2015, Kariholu_2008, Kumar_2019f, Misra_2013, Wadhwa_2015e} and UK \cite{Beecroft_1998, Berry_2021e, Cauchi_2002, Cox_2007, Gardner_2017h, Qureshi_2016}; 6 cases from Bulgaria \cite{Losanoff_1996, Losanoff_1997e}; 5 cases from Iran \cite{DivsalarP._2023a, Emamhadi_2018, Farhadi_2024h}; 4 cases from Turkey \cite{Akay_2015f, Atayan_2016, Tanrikulu_2015e, Yildiz_2016e}; 2 cases from China \cite{Jin_2023, Li_2013}, Poland \cite{Kobiela_2015, Wnęk_2015f}, and Spain \cite{CamachoDorado_2018, fjbuilsRepeatedBehaviorDeliberate2024}; 1 case from Australia \cite{Apikotoa_2022f}, Bahrain \cite{Ali_2020f}, Croatia \cite{Trgo_2012f}, Ecuador \cite{DelgadoSalazar_2020c}, Egypt \cite{Ali_2022g}, Ethiopia \cite{Mesfin_2022a}, Germany \cite{teWildt_2010}, Greece \cite{Sakellaridis_2008f}, Hungary \cite{Csaky_1998e}, Iraq \cite{Al-Faham_2020k}, Israel \cite{Goldman_1998f}, Italy \cite{Riva_2018j}, Japan \cite{Ohno_2005}, Nepal \cite{Thapa_2019f}, Netherlands \cite{Benoist_2019e}, Oman \cite{AlShaaibi_2021b}, Pakistan \cite{Yasin_2009}, Portugal \cite{Peixoto_2017f}, Qatar \cite{Ali_2017}, Saudi Arabia \cite{Sultan_2024f}, South Africa \cite{Sobnach_2011f}, Sweden \cite{Naji_2012f}, Switzerland \cite{Wildhaber_2005}, and Taiwan \cite{Chang_2017f}. \paragraph*{Gender} 43 cases (60\%) were male \cite{Akay_2015f, Al-Faham_2020k, Alao_2006i, Ali_2017, Ali_2022g, Apikotoa_2022f, Atayan_2016, Benoist_2019e, Berry_2021e, Bhumi_2024f, CamachoDorado_2018, Csaky_1998e, Emamhadi_2018, Farhadi_2024h, Fry_2010, Gardner_2017h, Guinan_2019f, Jehangir_2019h, Jin_2023, Kobiela_2015, Kumar_2001, Kumar_2019f, Liu_2005, Losanoff_1996, Losanoff_1997e, Mesfin_2022a, Misra_2013, Qureshi_2016, Riva_2018j, Sobnach_2011f, Tammana_2012j, Tanrikulu_2015e, Tay_2004, Thapa_2019f, Trgo_2012f, Wadhwa_2015e, Yasin_2009, teWildt_2010}, 28 cases (39\%) were female \cite{AlShaaibi_2021b, Ali_2020f, Ataya_2013, Beecroft_1998, Bhasin_2014, Bhattacharjee_2008, Cauchi_2002, Chang_2017f, Cox_2007, DelgadoSalazar_2020c, DivsalarP._2023a, Goldman_1998f, Hardy_2023g, Kar_2015, Kariholu_2008, Kerestes_2019, Li_2013, Naji_2012f, Ohno_2005, Peixoto_2017f, Sakellaridis_2008f, Sultan_2024f, Tupesis_2004f, Wildhaber_2005, Wnęk_2015f, Yildiz_2016e}, 1 case (1\%) had no gender recorded \cite{fjbuilsRepeatedBehaviorDeliberate2024}. \paragraph*{Age Group} 25 cases (35\%) were between 26 and 40 years of age \cite{Alao_2006i, Ali_2022g, Apikotoa_2022f, Ataya_2013, Benoist_2019e, Bhasin_2014, Chang_2017f, Cox_2007, DelgadoSalazar_2020c, Farhadi_2024h, Fry_2010, Gardner_2017h, Guinan_2019f, Jin_2023, Kumar_2019f, Losanoff_1996, Misra_2013, Qureshi_2016, Riva_2018j, Sakellaridis_2008f, Tammana_2012j, Trgo_2012f, Wnęk_2015f, Yildiz_2016e, fjbuilsRepeatedBehaviorDeliberate2024}, 18 cases (25\%) were between 18 and 25 years of age \cite{Akay_2015f, Ali_2017, Atayan_2016, Bhattacharjee_2008, Csaky_1998e, Kar_2015, Kariholu_2008, Kobiela_2015, Losanoff_1996, Losanoff_1997e, Mesfin_2022a, Peixoto_2017f, Sobnach_2011f, Tupesis_2004f, Yasin_2009}, 13 cases (18\%) were under 18 years of age \cite{AlShaaibi_2021b, Ali_2020f, Cauchi_2002, DivsalarP._2023a, Goldman_1998f, Liu_2005, Naji_2012f, Ohno_2005, Tanrikulu_2015e, Tay_2004, Wildhaber_2005}, 11 cases (15\%) were between 41 and 60 years of age \cite{Al-Faham_2020k, Bhumi_2024f, CamachoDorado_2018, Emamhadi_2018, Hardy_2023g, Jehangir_2019h, Kumar_2001, Sultan_2024f, Thapa_2019f, Wadhwa_2015e, teWildt_2010}, 3 cases (4\%) were over 60 years of age \cite{Beecroft_1998, Kerestes_2019, Li_2013}, 2 cases (3\%) had no age documented \cite{Berry_2021e}. \paragraph*{Population} 36 cases (50\%) had a psychiatric history \cite{AlShaaibi_2021b, Alao_2006i, Ali_2020f, Apikotoa_2022f, Ataya_2013, Atayan_2016, Beecroft_1998, CamachoDorado_2018, Chang_2017f, DelgadoSalazar_2020c, DivsalarP._2023a, Farhadi_2024h, Fry_2010, Guinan_2019f, Hardy_2023g, Jehangir_2019h, Jin_2023, Kar_2015, Kerestes_2019, Kobiela_2015, Kumar_2001, Kumar_2019f, Liu_2005, Mesfin_2022a, Misra_2013, Ohno_2005, Peixoto_2017f, Sakellaridis_2008f, Sultan_2024f, Tammana_2012j, Tanrikulu_2015e, Yildiz_2016e, fjbuilsRepeatedBehaviorDeliberate2024, teWildt_2010}, 19 cases (26\%) had ingested previously \cite{Alao_2006i, Apikotoa_2022f, Berry_2021e, Bhattacharjee_2008, Csaky_1998e, DivsalarP._2023a, Emamhadi_2018, Guinan_2019f, Jehangir_2019h, Jin_2023, Liu_2005, Sakellaridis_2008f, Tanrikulu_2015e, Thapa_2019f, Yildiz_2016e, fjbuilsRepeatedBehaviorDeliberate2024, teWildt_2010}, 12 cases (17\%) were detained persons \cite{Alao_2006i, Ali_2022g, Apikotoa_2022f, Losanoff_1996, Losanoff_1997e, Qureshi_2016, Tammana_2012j, Trgo_2012f}, 7 cases (10\%) were severely disabled \cite{Atayan_2016, Kerestes_2019, Liu_2005, Ohno_2005, Peixoto_2017f, Yildiz_2016e, teWildt_2010}, 4 cases (6\%) were psychiatric inpatients \cite{DivsalarP._2023a, fjbuilsRepeatedBehaviorDeliberate2024, teWildt_2010}, 3 cases (4\%) were under the influence of alcohol \cite{Benoist_2019e, Csaky_1998e, Thapa_2019f}, 2 cases (3\%) were displaced people \cite{Akay_2015f, Gardner_2017h}. \paragraph*{Motivation} 34 cases (47\%) had a psychiatric motivation \cite{Al-Faham_2020k, Alao_2006i, Ali_2020f, Apikotoa_2022f, Ataya_2013, Atayan_2016, Bhasin_2014, Bhattacharjee_2008, DelgadoSalazar_2020c, DivsalarP._2023a, Emamhadi_2018, Farhadi_2024h, Guinan_2019f, Hardy_2023g, Jehangir_2019h, Jin_2023, Kar_2015, Kariholu_2008, Kerestes_2019, Kobiela_2015, Kumar_2001, Kumar_2019f, Li_2013, Liu_2005, Misra_2013, Ohno_2005, Sakellaridis_2008f, Sultan_2024f, Tammana_2012j, Tanrikulu_2015e, Yasin_2009, teWildt_2010}, 21 cases (29\%) were motivated by self-harm intention \cite{Al-Faham_2020k, AlShaaibi_2021b, Alao_2006i, Ali_2017, CamachoDorado_2018, Chang_2017f, Cox_2007, Csaky_1998e, Fry_2010, Li_2013, Losanoff_1996, Losanoff_1997e, Mesfin_2022a, Sakellaridis_2008f, Tammana_2012j, Tanrikulu_2015e, fjbuilsRepeatedBehaviorDeliberate2024}, 17 cases (24\%) had a psychosocial motivation \cite{Akay_2015f, Benoist_2019e, Bhattacharjee_2008, Cauchi_2002, Goldman_1998f, Hardy_2023g, Kobiela_2015, Li_2013, Naji_2012f, Qureshi_2016, Riva_2018j, Sobnach_2011f, Tay_2004, Thapa_2019f, Tupesis_2004f, Wildhaber_2005, Wnęk_2015f}, 9 cases (12\%) were motivated by protest \cite{Bhumi_2024f, Gardner_2017h, Losanoff_1996, Losanoff_1997e, Tupesis_2004f}, 9 cases (12\%) had another documented motivation \cite{Ali_2020f, Ali_2022g, Emamhadi_2018, Guinan_2019f, Peixoto_2017f, Sakellaridis_2008f, Trgo_2012f, Wadhwa_2015e, Yildiz_2016e}. \paragraph*{Object Characteristics} 51 cases (71\%) ingested a large diameter object (\textgreater{}2.5cm) \cite{Akay_2015f, Al-Faham_2020k, AlShaaibi_2021b, Alao_2006i, Ali_2017, Ali_2022g, Apikotoa_2022f, Atayan_2016, Berry_2021e, Bhasin_2014, CamachoDorado_2018, Cauchi_2002, Chang_2017f, Cox_2007, Csaky_1998e, DivsalarP._2023a, Emamhadi_2018, Gardner_2017h, Guinan_2019f, Jehangir_2019h, Jin_2023, Kariholu_2008, Kerestes_2019, Kobiela_2015, Kumar_2001, Kumar_2019f, Losanoff_1996, Losanoff_1997e, Mesfin_2022a, Misra_2013, Naji_2012f, Ohno_2005, Peixoto_2017f, Qureshi_2016, Riva_2018j, Sakellaridis_2008f, Sultan_2024f, Tanrikulu_2015e, Thapa_2019f, Trgo_2012f, Wnęk_2015f, Yildiz_2016e, fjbuilsRepeatedBehaviorDeliberate2024, teWildt_2010}, 44 cases (61\%) ingested multiple objects \cite{Ali_2020f, Apikotoa_2022f, Ataya_2013, Atayan_2016, Beecroft_1998, Bhattacharjee_2008, Bhumi_2024f, CamachoDorado_2018, Cauchi_2002, Emamhadi_2018, Farhadi_2024h, Fry_2010, Goldman_1998f, Guinan_2019f, Hardy_2023g, Jehangir_2019h, Jin_2023, Kar_2015, Kariholu_2008, Kobiela_2015, Kumar_2001, Kumar_2019f, Li_2013, Liu_2005, Losanoff_1996, Mesfin_2022a, Misra_2013, Naji_2012f, Ohno_2005, Sobnach_2011f, Sultan_2024f, Tammana_2012j, Tanrikulu_2015e, Tay_2004, Thapa_2019f, Wadhwa_2015e, Wildhaber_2005, Yasin_2009, fjbuilsRepeatedBehaviorDeliberate2024, teWildt_2010}, 34 cases (47\%) ingested a sharp object \cite{AlShaaibi_2021b, Alao_2006i, Apikotoa_2022f, Ataya_2013, Benoist_2019e, Bhasin_2014, Bhattacharjee_2008, CamachoDorado_2018, Csaky_1998e, DelgadoSalazar_2020c, DivsalarP._2023a, Emamhadi_2018, Farhadi_2024h, Fry_2010, Guinan_2019f, Hardy_2023g, Jehangir_2019h, Jin_2023, Kariholu_2008, Kobiela_2015, Kumar_2019f, Losanoff_1996, Losanoff_1997e, Mesfin_2022a, Misra_2013, Sobnach_2011f, Yasin_2009, teWildt_2010}, 32 cases (44\%) ingested a long object (\textgreater{}5cm) \cite{Al-Faham_2020k, AlShaaibi_2021b, Ali_2017, Ali_2022g, Atayan_2016, Bhasin_2014, CamachoDorado_2018, Chang_2017f, Cox_2007, Csaky_1998e, DivsalarP._2023a, Emamhadi_2018, Fry_2010, Gardner_2017h, Jin_2023, Kariholu_2008, Kerestes_2019, Kobiela_2015, Kumar_2019f, Mesfin_2022a, Misra_2013, Ohno_2005, Qureshi_2016, Sakellaridis_2008f, Sultan_2024f, Thapa_2019f, Trgo_2012f, Yasin_2009, Yildiz_2016e, teWildt_2010}, 9 cases (12\%) ingested a magnet \cite{Ali_2020f, Bhumi_2024f, Cauchi_2002, Liu_2005, Naji_2012f, Ohno_2005, Tanrikulu_2015e, Tay_2004, Wildhaber_2005}, 2 cases (3\%) ingested a button battery \cite{Berry_2021e, Bhumi_2024f}. \paragraph*{Outcomes} 48 cases (67\%) experienced a complication \cite{Ali_2017, Ali_2020f, Apikotoa_2022f, Atayan_2016, Beecroft_1998, Benoist_2019e, Berry_2021e, Bhasin_2014, Bhumi_2024f, CamachoDorado_2018, Cauchi_2002, Cox_2007, Csaky_1998e, DelgadoSalazar_2020c, DivsalarP._2023a, Emamhadi_2018, Farhadi_2024h, Fry_2010, Gardner_2017h, Goldman_1998f, Jin_2023, Kariholu_2008, Kerestes_2019, Kobiela_2015, Kumar_2001, Kumar_2019f, Liu_2005, Losanoff_1996, Mesfin_2022a, Misra_2013, Naji_2012f, Ohno_2005, Sakellaridis_2008f, Sobnach_2011f, Sultan_2024f, Tanrikulu_2015e, Tay_2004, Thapa_2019f, Trgo_2012f, Tupesis_2004f, Wildhaber_2005, Wnęk_2015f, Yasin_2009, Yildiz_2016e}, 44 cases (61\%) underwent surgery \cite{Al-Faham_2020k, AlShaaibi_2021b, Alao_2006i, Ali_2017, Ali_2020f, Atayan_2016, Beecroft_1998, Bhasin_2014, CamachoDorado_2018, Cauchi_2002, Chang_2017f, Cox_2007, Csaky_1998e, DelgadoSalazar_2020c, DivsalarP._2023a, Farhadi_2024h, Fry_2010, Gardner_2017h, Jin_2023, Kariholu_2008, Kerestes_2019, Kobiela_2015, Kumar_2019f, Liu_2005, Losanoff_1996, Losanoff_1997e, Mesfin_2022a, Misra_2013, Naji_2012f, Sobnach_2011f, Tanrikulu_2015e, Tay_2004, Thapa_2019f, Tupesis_2004f, Wildhaber_2005, Wnęk_2015f, Yasin_2009, Yildiz_2016e, fjbuilsRepeatedBehaviorDeliberate2024}, 31 cases (43\%) underwent endoscopy \cite{Akay_2015f, Ali_2022g, Apikotoa_2022f, Atayan_2016, Benoist_2019e, Berry_2021e, Bhasin_2014, Bhumi_2024f, CamachoDorado_2018, Chang_2017f, DelgadoSalazar_2020c, Gardner_2017h, Guinan_2019f, Hardy_2023g, Jehangir_2019h, Kariholu_2008, Li_2013, Liu_2005, Ohno_2005, Peixoto_2017f, Qureshi_2016, Riva_2018j, Sakellaridis_2008f, Sultan_2024f, Tammana_2012j, Tanrikulu_2015e, Trgo_2012f, Wadhwa_2015e, Wnęk_2015f, teWildt_2010}, 7 cases (10\%) were managed conservatively \cite{Ataya_2013, Bhattacharjee_2008, DivsalarP._2023a, Emamhadi_2018, Goldman_1998f, Kar_2015, Kumar_2001}, 2 cases (3\%) died \cite{Emamhadi_2018, Kumar_2001}. All 90 were male gender. 90 cases (100\%) were detained at the time of ingestion \cite{Elghali_2016, Karp_1991b, Lee_2007}, 88 cases (98\%) were intentional ingestions \cite{Elghali_2016, Karp_1991b, Lee_2007}, 30 cases (33\%) had a psychiatric history documented \cite{Elghali_2016, Karp_1991b, Lee_2007}, 2 cases (2\%) had a history of prior ingestion \cite{Elghali_2016}. No cases were reported for were psychiatric inpatients, were displaced people, were under the influence of alcohol at the time of ingestion, and had a severe disability history.
\paragraph*{Motivation}  70 cases (78\%) reported protest motivation \cite{Elghali_2016, Karp_1991b, Lee_2007}, 12 cases (13\%) reported psychiatric motivation \cite{Karp_1991b}, 6 cases (7\%) reported self-harm motivation \cite{Elghali_2016, Karp_1991b}. No cases were reported for psychosocial motivation and other motivation.
\paragraph*{Object Characteristics}  68 cases (76\%) involved sharp object ingestion \cite{Elghali_2016, Karp_1991b, Lee_2007}, 32 cases (36\%) involved long (\textgreater 5cm) object ingestion \cite{Lee_2007}, 25 cases (28\%) involved ingestion of multiple objects \cite{Elghali_2016, Lee_2007}. No cases were reported for button battery ingestion, magnet ingestion, and involved large diameter (\textgreater 2.5cm) object ingestion.
\paragraph*{Outcomes}  47 cases (52\%) underwent endoscopic intervention \cite{Elghali_2016, Lee_2007}, 29 cases (32\%) were managed conservatively \cite{Elghali_2016, Karp_1991b}, 15 cases (17\%) underwent surgical intervention \cite{Elghali_2016, Karp_1991b, Lee_2007}, 6 cases (7\%) reported complications \cite{Lee_2007}, 1 case (1\%) died \cite{Elghali_2016}.
\paragraph*{Geographical Location}Cases were recorded in 33 countries: 13 cases from USA \cite{Alao_2006i, Ataya_2013, Bhumi_2024f, Fry_2010, Guinan_2019f, Hardy_2023g, Jehangir_2019h, Kerestes_2019, Kumar_2001, Liu_2005, Tammana_2012j, Tay_2004, Tupesis_2004f}; 7 cases from India \cite{Bhasin_2014, Bhattacharjee_2008, Kar_2015, Kariholu_2008, Kumar_2019f, Misra_2013, Wadhwa_2015e} and UK \cite{Beecroft_1998, Berry_2021e, Cauchi_2002, Cox_2007, Gardner_2017h, Qureshi_2016}; 6 cases from Bulgaria \cite{Losanoff_1996, Losanoff_1997e}; 5 cases from Iran \cite{DivsalarP._2023a, Emamhadi_2018, Farhadi_2024h}; 4 cases from Turkey \cite{Akay_2015f, Atayan_2016, Tanrikulu_2015e, Yildiz_2016e}; 2 cases from China \cite{Jin_2023, Li_2013}, Poland \cite{Kobiela_2015, Wnęk_2015f}, and Spain \cite{CamachoDorado_2018, fjbuilsRepeatedBehaviorDeliberate2024}; 1 case from Australia \cite{Apikotoa_2022f}, Bahrain \cite{Ali_2020f}, Croatia \cite{Trgo_2012f}, Ecuador \cite{DelgadoSalazar_2020c}, Egypt \cite{Ali_2022g}, Ethiopia \cite{Mesfin_2022a}, Germany \cite{teWildt_2010}, Greece \cite{Sakellaridis_2008f}, Hungary \cite{Csaky_1998e}, Iraq \cite{Al-Faham_2020k}, Israel \cite{Goldman_1998f}, Italy \cite{Riva_2018j}, Japan \cite{Ohno_2005}, Nepal \cite{Thapa_2019f}, Netherlands \cite{Benoist_2019e}, Oman \cite{AlShaaibi_2021b}, Pakistan \cite{Yasin_2009}, Portugal \cite{Peixoto_2017f}, Qatar \cite{Ali_2017}, Saudi Arabia \cite{Sultan_2024f}, South Africa \cite{Sobnach_2011f}, Sweden \cite{Naji_2012f}, Switzerland \cite{Wildhaber_2005}, and Taiwan \cite{Chang_2017f}. \paragraph*{Gender} 43 cases (60\%) were male \cite{Akay_2015f, Al-Faham_2020k, Alao_2006i, Ali_2017, Ali_2022g, Apikotoa_2022f, Atayan_2016, Benoist_2019e, Berry_2021e, Bhumi_2024f, CamachoDorado_2018, Csaky_1998e, Emamhadi_2018, Farhadi_2024h, Fry_2010, Gardner_2017h, Guinan_2019f, Jehangir_2019h, Jin_2023, Kobiela_2015, Kumar_2001, Kumar_2019f, Liu_2005, Losanoff_1996, Losanoff_1997e, Mesfin_2022a, Misra_2013, Qureshi_2016, Riva_2018j, Sobnach_2011f, Tammana_2012j, Tanrikulu_2015e, Tay_2004, Thapa_2019f, Trgo_2012f, Wadhwa_2015e, Yasin_2009, teWildt_2010}, 28 cases (39\%) were female \cite{AlShaaibi_2021b, Ali_2020f, Ataya_2013, Beecroft_1998, Bhasin_2014, Bhattacharjee_2008, Cauchi_2002, Chang_2017f, Cox_2007, DelgadoSalazar_2020c, DivsalarP._2023a, Goldman_1998f, Hardy_2023g, Kar_2015, Kariholu_2008, Kerestes_2019, Li_2013, Naji_2012f, Ohno_2005, Peixoto_2017f, Sakellaridis_2008f, Sultan_2024f, Tupesis_2004f, Wildhaber_2005, Wnęk_2015f, Yildiz_2016e}, 1 case (1\%) had no gender recorded \cite{fjbuilsRepeatedBehaviorDeliberate2024}. \paragraph*{Age Group} 25 cases (35\%) were between 26 and 40 years of age \cite{Alao_2006i, Ali_2022g, Apikotoa_2022f, Ataya_2013, Benoist_2019e, Bhasin_2014, Chang_2017f, Cox_2007, DelgadoSalazar_2020c, Farhadi_2024h, Fry_2010, Gardner_2017h, Guinan_2019f, Jin_2023, Kumar_2019f, Losanoff_1996, Misra_2013, Qureshi_2016, Riva_2018j, Sakellaridis_2008f, Tammana_2012j, Trgo_2012f, Wnęk_2015f, Yildiz_2016e, fjbuilsRepeatedBehaviorDeliberate2024}, 18 cases (25\%) were between 18 and 25 years of age \cite{Akay_2015f, Ali_2017, Atayan_2016, Bhattacharjee_2008, Csaky_1998e, Kar_2015, Kariholu_2008, Kobiela_2015, Losanoff_1996, Losanoff_1997e, Mesfin_2022a, Peixoto_2017f, Sobnach_2011f, Tupesis_2004f, Yasin_2009}, 13 cases (18\%) were under 18 years of age \cite{AlShaaibi_2021b, Ali_2020f, Cauchi_2002, DivsalarP._2023a, Goldman_1998f, Liu_2005, Naji_2012f, Ohno_2005, Tanrikulu_2015e, Tay_2004, Wildhaber_2005}, 11 cases (15\%) were between 41 and 60 years of age \cite{Al-Faham_2020k, Bhumi_2024f, CamachoDorado_2018, Emamhadi_2018, Hardy_2023g, Jehangir_2019h, Kumar_2001, Sultan_2024f, Thapa_2019f, Wadhwa_2015e, teWildt_2010}, 3 cases (4\%) were over 60 years of age \cite{Beecroft_1998, Kerestes_2019, Li_2013}, 2 cases (3\%) had no age documented \cite{Berry_2021e}. \paragraph*{Population} 36 cases (50\%) had a psychiatric history \cite{AlShaaibi_2021b, Alao_2006i, Ali_2020f, Apikotoa_2022f, Ataya_2013, Atayan_2016, Beecroft_1998, CamachoDorado_2018, Chang_2017f, DelgadoSalazar_2020c, DivsalarP._2023a, Farhadi_2024h, Fry_2010, Guinan_2019f, Hardy_2023g, Jehangir_2019h, Jin_2023, Kar_2015, Kerestes_2019, Kobiela_2015, Kumar_2001, Kumar_2019f, Liu_2005, Mesfin_2022a, Misra_2013, Ohno_2005, Peixoto_2017f, Sakellaridis_2008f, Sultan_2024f, Tammana_2012j, Tanrikulu_2015e, Yildiz_2016e, fjbuilsRepeatedBehaviorDeliberate2024, teWildt_2010}, 19 cases (26\%) had ingested previously \cite{Alao_2006i, Apikotoa_2022f, Berry_2021e, Bhattacharjee_2008, Csaky_1998e, DivsalarP._2023a, Emamhadi_2018, Guinan_2019f, Jehangir_2019h, Jin_2023, Liu_2005, Sakellaridis_2008f, Tanrikulu_2015e, Thapa_2019f, Yildiz_2016e, fjbuilsRepeatedBehaviorDeliberate2024, teWildt_2010}, 12 cases (17\%) were detained persons \cite{Alao_2006i, Ali_2022g, Apikotoa_2022f, Losanoff_1996, Losanoff_1997e, Qureshi_2016, Tammana_2012j, Trgo_2012f}, 7 cases (10\%) were severely disabled \cite{Atayan_2016, Kerestes_2019, Liu_2005, Ohno_2005, Peixoto_2017f, Yildiz_2016e, teWildt_2010}, 4 cases (6\%) were psychiatric inpatients \cite{DivsalarP._2023a, fjbuilsRepeatedBehaviorDeliberate2024, teWildt_2010}, 3 cases (4\%) were under the influence of alcohol \cite{Benoist_2019e, Csaky_1998e, Thapa_2019f}, 2 cases (3\%) were displaced people \cite{Akay_2015f, Gardner_2017h}. \paragraph*{Motivation} 34 cases (47\%) had a psychiatric motivation \cite{Al-Faham_2020k, Alao_2006i, Ali_2020f, Apikotoa_2022f, Ataya_2013, Atayan_2016, Bhasin_2014, Bhattacharjee_2008, DelgadoSalazar_2020c, DivsalarP._2023a, Emamhadi_2018, Farhadi_2024h, Guinan_2019f, Hardy_2023g, Jehangir_2019h, Jin_2023, Kar_2015, Kariholu_2008, Kerestes_2019, Kobiela_2015, Kumar_2001, Kumar_2019f, Li_2013, Liu_2005, Misra_2013, Ohno_2005, Sakellaridis_2008f, Sultan_2024f, Tammana_2012j, Tanrikulu_2015e, Yasin_2009, teWildt_2010}, 21 cases (29\%) were motivated by self-harm intention \cite{Al-Faham_2020k, AlShaaibi_2021b, Alao_2006i, Ali_2017, CamachoDorado_2018, Chang_2017f, Cox_2007, Csaky_1998e, Fry_2010, Li_2013, Losanoff_1996, Losanoff_1997e, Mesfin_2022a, Sakellaridis_2008f, Tammana_2012j, Tanrikulu_2015e, fjbuilsRepeatedBehaviorDeliberate2024}, 17 cases (24\%) had a psychosocial motivation \cite{Akay_2015f, Benoist_2019e, Bhattacharjee_2008, Cauchi_2002, Goldman_1998f, Hardy_2023g, Kobiela_2015, Li_2013, Naji_2012f, Qureshi_2016, Riva_2018j, Sobnach_2011f, Tay_2004, Thapa_2019f, Tupesis_2004f, Wildhaber_2005, Wnęk_2015f}, 9 cases (12\%) were motivated by protest \cite{Bhumi_2024f, Gardner_2017h, Losanoff_1996, Losanoff_1997e, Tupesis_2004f}, 9 cases (12\%) had another documented motivation \cite{Ali_2020f, Ali_2022g, Emamhadi_2018, Guinan_2019f, Peixoto_2017f, Sakellaridis_2008f, Trgo_2012f, Wadhwa_2015e, Yildiz_2016e}. \paragraph*{Object Characteristics} 51 cases (71\%) ingested a large diameter object (\textgreater{}2.5cm) \cite{Akay_2015f, Al-Faham_2020k, AlShaaibi_2021b, Alao_2006i, Ali_2017, Ali_2022g, Apikotoa_2022f, Atayan_2016, Berry_2021e, Bhasin_2014, CamachoDorado_2018, Cauchi_2002, Chang_2017f, Cox_2007, Csaky_1998e, DivsalarP._2023a, Emamhadi_2018, Gardner_2017h, Guinan_2019f, Jehangir_2019h, Jin_2023, Kariholu_2008, Kerestes_2019, Kobiela_2015, Kumar_2001, Kumar_2019f, Losanoff_1996, Losanoff_1997e, Mesfin_2022a, Misra_2013, Naji_2012f, Ohno_2005, Peixoto_2017f, Qureshi_2016, Riva_2018j, Sakellaridis_2008f, Sultan_2024f, Tanrikulu_2015e, Thapa_2019f, Trgo_2012f, Wnęk_2015f, Yildiz_2016e, fjbuilsRepeatedBehaviorDeliberate2024, teWildt_2010}, 44 cases (61\%) ingested multiple objects \cite{Ali_2020f, Apikotoa_2022f, Ataya_2013, Atayan_2016, Beecroft_1998, Bhattacharjee_2008, Bhumi_2024f, CamachoDorado_2018, Cauchi_2002, Emamhadi_2018, Farhadi_2024h, Fry_2010, Goldman_1998f, Guinan_2019f, Hardy_2023g, Jehangir_2019h, Jin_2023, Kar_2015, Kariholu_2008, Kobiela_2015, Kumar_2001, Kumar_2019f, Li_2013, Liu_2005, Losanoff_1996, Mesfin_2022a, Misra_2013, Naji_2012f, Ohno_2005, Sobnach_2011f, Sultan_2024f, Tammana_2012j, Tanrikulu_2015e, Tay_2004, Thapa_2019f, Wadhwa_2015e, Wildhaber_2005, Yasin_2009, fjbuilsRepeatedBehaviorDeliberate2024, teWildt_2010}, 34 cases (47\%) ingested a sharp object \cite{AlShaaibi_2021b, Alao_2006i, Apikotoa_2022f, Ataya_2013, Benoist_2019e, Bhasin_2014, Bhattacharjee_2008, CamachoDorado_2018, Csaky_1998e, DelgadoSalazar_2020c, DivsalarP._2023a, Emamhadi_2018, Farhadi_2024h, Fry_2010, Guinan_2019f, Hardy_2023g, Jehangir_2019h, Jin_2023, Kariholu_2008, Kobiela_2015, Kumar_2019f, Losanoff_1996, Losanoff_1997e, Mesfin_2022a, Misra_2013, Sobnach_2011f, Yasin_2009, teWildt_2010}, 32 cases (44\%) ingested a long object (\textgreater{}5cm) \cite{Al-Faham_2020k, AlShaaibi_2021b, Ali_2017, Ali_2022g, Atayan_2016, Bhasin_2014, CamachoDorado_2018, Chang_2017f, Cox_2007, Csaky_1998e, DivsalarP._2023a, Emamhadi_2018, Fry_2010, Gardner_2017h, Jin_2023, Kariholu_2008, Kerestes_2019, Kobiela_2015, Kumar_2019f, Mesfin_2022a, Misra_2013, Ohno_2005, Qureshi_2016, Sakellaridis_2008f, Sultan_2024f, Thapa_2019f, Trgo_2012f, Yasin_2009, Yildiz_2016e, teWildt_2010}, 9 cases (12\%) ingested a magnet \cite{Ali_2020f, Bhumi_2024f, Cauchi_2002, Liu_2005, Naji_2012f, Ohno_2005, Tanrikulu_2015e, Tay_2004, Wildhaber_2005}, 2 cases (3\%) ingested a button battery \cite{Berry_2021e, Bhumi_2024f}. \paragraph*{Outcomes} 48 cases (67\%) experienced a complication \cite{Ali_2017, Ali_2020f, Apikotoa_2022f, Atayan_2016, Beecroft_1998, Benoist_2019e, Berry_2021e, Bhasin_2014, Bhumi_2024f, CamachoDorado_2018, Cauchi_2002, Cox_2007, Csaky_1998e, DelgadoSalazar_2020c, DivsalarP._2023a, Emamhadi_2018, Farhadi_2024h, Fry_2010, Gardner_2017h, Goldman_1998f, Jin_2023, Kariholu_2008, Kerestes_2019, Kobiela_2015, Kumar_2001, Kumar_2019f, Liu_2005, Losanoff_1996, Mesfin_2022a, Misra_2013, Naji_2012f, Ohno_2005, Sakellaridis_2008f, Sobnach_2011f, Sultan_2024f, Tanrikulu_2015e, Tay_2004, Thapa_2019f, Trgo_2012f, Tupesis_2004f, Wildhaber_2005, Wnęk_2015f, Yasin_2009, Yildiz_2016e}, 44 cases (61\%) underwent surgery \cite{Al-Faham_2020k, AlShaaibi_2021b, Alao_2006i, Ali_2017, Ali_2020f, Atayan_2016, Beecroft_1998, Bhasin_2014, CamachoDorado_2018, Cauchi_2002, Chang_2017f, Cox_2007, Csaky_1998e, DelgadoSalazar_2020c, DivsalarP._2023a, Farhadi_2024h, Fry_2010, Gardner_2017h, Jin_2023, Kariholu_2008, Kerestes_2019, Kobiela_2015, Kumar_2019f, Liu_2005, Losanoff_1996, Losanoff_1997e, Mesfin_2022a, Misra_2013, Naji_2012f, Sobnach_2011f, Tanrikulu_2015e, Tay_2004, Thapa_2019f, Tupesis_2004f, Wildhaber_2005, Wnęk_2015f, Yasin_2009, Yildiz_2016e, fjbuilsRepeatedBehaviorDeliberate2024}, 31 cases (43\%) underwent endoscopy \cite{Akay_2015f, Ali_2022g, Apikotoa_2022f, Atayan_2016, Benoist_2019e, Berry_2021e, Bhasin_2014, Bhumi_2024f, CamachoDorado_2018, Chang_2017f, DelgadoSalazar_2020c, Gardner_2017h, Guinan_2019f, Hardy_2023g, Jehangir_2019h, Kariholu_2008, Li_2013, Liu_2005, Ohno_2005, Peixoto_2017f, Qureshi_2016, Riva_2018j, Sakellaridis_2008f, Sultan_2024f, Tammana_2012j, Tanrikulu_2015e, Trgo_2012f, Wadhwa_2015e, Wnęk_2015f, teWildt_2010}, 7 cases (10\%) were managed conservatively \cite{Ataya_2013, Bhattacharjee_2008, DivsalarP._2023a, Emamhadi_2018, Goldman_1998f, Kar_2015, Kumar_2001}, 2 cases (3\%) died \cite{Emamhadi_2018, Kumar_2001}. All 90 were male gender. 90 cases (100\%) were detained at the time of ingestion \cite{Elghali_2016, Karp_1991b, Lee_2007}, 88 cases (98\%) were intentional ingestions \cite{Elghali_2016, Karp_1991b, Lee_2007}, 30 cases (33\%) had a psychiatric history documented \cite{Elghali_2016, Karp_1991b, Lee_2007}, 2 cases (2\%) had a history of prior ingestion \cite{Elghali_2016}. No cases were reported for were psychiatric inpatients, were displaced people, were under the influence of alcohol at the time of ingestion, and had a severe disability history.
\paragraph*{Motivation}  70 cases (78\%) reported protest motivation \cite{Elghali_2016, Karp_1991b, Lee_2007}, 12 cases (13\%) reported psychiatric motivation \cite{Karp_1991b}, 6 cases (7\%) reported self-harm motivation \cite{Elghali_2016, Karp_1991b}. No cases were reported for psychosocial motivation and other motivation.
\paragraph*{Object Characteristics}  68 cases (76\%) involved sharp object ingestion \cite{Elghali_2016, Karp_1991b, Lee_2007}, 32 cases (36\%) involved long (\textgreater 5cm) object ingestion \cite{Lee_2007}, 25 cases (28\%) involved ingestion of multiple objects \cite{Elghali_2016, Lee_2007}. No cases were reported for button battery ingestion, magnet ingestion, and involved large diameter (\textgreater 2.5cm) object ingestion.
\paragraph*{Outcomes}  47 cases (52\%) underwent endoscopic intervention \cite{Elghali_2016, Lee_2007}, 29 cases (32\%) were managed conservatively \cite{Elghali_2016, Karp_1991b}, 15 cases (17\%) underwent surgical intervention \cite{Elghali_2016, Karp_1991b, Lee_2007}, 6 cases (7\%) reported complications \cite{Lee_2007}, 1 case (1\%) died \cite{Elghali_2016}.
\paragraph*{Geographical Location}Cases were recorded in 33 countries: 13 cases from USA \cite{Alao_2006i, Ataya_2013, Bhumi_2024f, Fry_2010, Guinan_2019f, Hardy_2023g, Jehangir_2019h, Kerestes_2019, Kumar_2001, Liu_2005, Tammana_2012j, Tay_2004, Tupesis_2004f}; 7 cases from India \cite{Bhasin_2014, Bhattacharjee_2008, Kar_2015, Kariholu_2008, Kumar_2019f, Misra_2013, Wadhwa_2015e} and UK \cite{Beecroft_1998, Berry_2021e, Cauchi_2002, Cox_2007, Gardner_2017h, Qureshi_2016}; 6 cases from Bulgaria \cite{Losanoff_1996, Losanoff_1997e}; 5 cases from Iran \cite{DivsalarP._2023a, Emamhadi_2018, Farhadi_2024h}; 4 cases from Turkey \cite{Akay_2015f, Atayan_2016, Tanrikulu_2015e, Yildiz_2016e}; 2 cases from China \cite{Jin_2023, Li_2013}, Poland \cite{Kobiela_2015, Wnęk_2015f}, and Spain \cite{CamachoDorado_2018, fjbuilsRepeatedBehaviorDeliberate2024}; 1 case from Australia \cite{Apikotoa_2022f}, Bahrain \cite{Ali_2020f}, Croatia \cite{Trgo_2012f}, Ecuador \cite{DelgadoSalazar_2020c}, Egypt \cite{Ali_2022g}, Ethiopia \cite{Mesfin_2022a}, Germany \cite{teWildt_2010}, Greece \cite{Sakellaridis_2008f}, Hungary \cite{Csaky_1998e}, Iraq \cite{Al-Faham_2020k}, Israel \cite{Goldman_1998f}, Italy \cite{Riva_2018j}, Japan \cite{Ohno_2005}, Nepal \cite{Thapa_2019f}, Netherlands \cite{Benoist_2019e}, Oman \cite{AlShaaibi_2021b}, Pakistan \cite{Yasin_2009}, Portugal \cite{Peixoto_2017f}, Qatar \cite{Ali_2017}, Saudi Arabia \cite{Sultan_2024f}, South Africa \cite{Sobnach_2011f}, Sweden \cite{Naji_2012f}, Switzerland \cite{Wildhaber_2005}, and Taiwan \cite{Chang_2017f}. \paragraph*{Gender} 43 cases (60\%) were male \cite{Akay_2015f, Al-Faham_2020k, Alao_2006i, Ali_2017, Ali_2022g, Apikotoa_2022f, Atayan_2016, Benoist_2019e, Berry_2021e, Bhumi_2024f, CamachoDorado_2018, Csaky_1998e, Emamhadi_2018, Farhadi_2024h, Fry_2010, Gardner_2017h, Guinan_2019f, Jehangir_2019h, Jin_2023, Kobiela_2015, Kumar_2001, Kumar_2019f, Liu_2005, Losanoff_1996, Losanoff_1997e, Mesfin_2022a, Misra_2013, Qureshi_2016, Riva_2018j, Sobnach_2011f, Tammana_2012j, Tanrikulu_2015e, Tay_2004, Thapa_2019f, Trgo_2012f, Wadhwa_2015e, Yasin_2009, teWildt_2010}, 28 cases (39\%) were female \cite{AlShaaibi_2021b, Ali_2020f, Ataya_2013, Beecroft_1998, Bhasin_2014, Bhattacharjee_2008, Cauchi_2002, Chang_2017f, Cox_2007, DelgadoSalazar_2020c, DivsalarP._2023a, Goldman_1998f, Hardy_2023g, Kar_2015, Kariholu_2008, Kerestes_2019, Li_2013, Naji_2012f, Ohno_2005, Peixoto_2017f, Sakellaridis_2008f, Sultan_2024f, Tupesis_2004f, Wildhaber_2005, Wnęk_2015f, Yildiz_2016e}, 1 case (1\%) had no gender recorded \cite{fjbuilsRepeatedBehaviorDeliberate2024}. \paragraph*{Age Group} 25 cases (35\%) were between 26 and 40 years of age \cite{Alao_2006i, Ali_2022g, Apikotoa_2022f, Ataya_2013, Benoist_2019e, Bhasin_2014, Chang_2017f, Cox_2007, DelgadoSalazar_2020c, Farhadi_2024h, Fry_2010, Gardner_2017h, Guinan_2019f, Jin_2023, Kumar_2019f, Losanoff_1996, Misra_2013, Qureshi_2016, Riva_2018j, Sakellaridis_2008f, Tammana_2012j, Trgo_2012f, Wnęk_2015f, Yildiz_2016e, fjbuilsRepeatedBehaviorDeliberate2024}, 18 cases (25\%) were between 18 and 25 years of age \cite{Akay_2015f, Ali_2017, Atayan_2016, Bhattacharjee_2008, Csaky_1998e, Kar_2015, Kariholu_2008, Kobiela_2015, Losanoff_1996, Losanoff_1997e, Mesfin_2022a, Peixoto_2017f, Sobnach_2011f, Tupesis_2004f, Yasin_2009}, 13 cases (18\%) were under 18 years of age \cite{AlShaaibi_2021b, Ali_2020f, Cauchi_2002, DivsalarP._2023a, Goldman_1998f, Liu_2005, Naji_2012f, Ohno_2005, Tanrikulu_2015e, Tay_2004, Wildhaber_2005}, 11 cases (15\%) were between 41 and 60 years of age \cite{Al-Faham_2020k, Bhumi_2024f, CamachoDorado_2018, Emamhadi_2018, Hardy_2023g, Jehangir_2019h, Kumar_2001, Sultan_2024f, Thapa_2019f, Wadhwa_2015e, teWildt_2010}, 3 cases (4\%) were over 60 years of age \cite{Beecroft_1998, Kerestes_2019, Li_2013}, 2 cases (3\%) had no age documented \cite{Berry_2021e}. \paragraph*{Population} 36 cases (50\%) had a psychiatric history \cite{AlShaaibi_2021b, Alao_2006i, Ali_2020f, Apikotoa_2022f, Ataya_2013, Atayan_2016, Beecroft_1998, CamachoDorado_2018, Chang_2017f, DelgadoSalazar_2020c, DivsalarP._2023a, Farhadi_2024h, Fry_2010, Guinan_2019f, Hardy_2023g, Jehangir_2019h, Jin_2023, Kar_2015, Kerestes_2019, Kobiela_2015, Kumar_2001, Kumar_2019f, Liu_2005, Mesfin_2022a, Misra_2013, Ohno_2005, Peixoto_2017f, Sakellaridis_2008f, Sultan_2024f, Tammana_2012j, Tanrikulu_2015e, Yildiz_2016e, fjbuilsRepeatedBehaviorDeliberate2024, teWildt_2010}, 19 cases (26\%) had ingested previously \cite{Alao_2006i, Apikotoa_2022f, Berry_2021e, Bhattacharjee_2008, Csaky_1998e, DivsalarP._2023a, Emamhadi_2018, Guinan_2019f, Jehangir_2019h, Jin_2023, Liu_2005, Sakellaridis_2008f, Tanrikulu_2015e, Thapa_2019f, Yildiz_2016e, fjbuilsRepeatedBehaviorDeliberate2024, teWildt_2010}, 12 cases (17\%) were detained persons \cite{Alao_2006i, Ali_2022g, Apikotoa_2022f, Losanoff_1996, Losanoff_1997e, Qureshi_2016, Tammana_2012j, Trgo_2012f}, 7 cases (10\%) were severely disabled \cite{Atayan_2016, Kerestes_2019, Liu_2005, Ohno_2005, Peixoto_2017f, Yildiz_2016e, teWildt_2010}, 4 cases (6\%) were psychiatric inpatients \cite{DivsalarP._2023a, fjbuilsRepeatedBehaviorDeliberate2024, teWildt_2010}, 3 cases (4\%) were under the influence of alcohol \cite{Benoist_2019e, Csaky_1998e, Thapa_2019f}, 2 cases (3\%) were displaced people \cite{Akay_2015f, Gardner_2017h}. \paragraph*{Motivation} 34 cases (47\%) had a psychiatric motivation \cite{Al-Faham_2020k, Alao_2006i, Ali_2020f, Apikotoa_2022f, Ataya_2013, Atayan_2016, Bhasin_2014, Bhattacharjee_2008, DelgadoSalazar_2020c, DivsalarP._2023a, Emamhadi_2018, Farhadi_2024h, Guinan_2019f, Hardy_2023g, Jehangir_2019h, Jin_2023, Kar_2015, Kariholu_2008, Kerestes_2019, Kobiela_2015, Kumar_2001, Kumar_2019f, Li_2013, Liu_2005, Misra_2013, Ohno_2005, Sakellaridis_2008f, Sultan_2024f, Tammana_2012j, Tanrikulu_2015e, Yasin_2009, teWildt_2010}, 21 cases (29\%) were motivated by self-harm intention \cite{Al-Faham_2020k, AlShaaibi_2021b, Alao_2006i, Ali_2017, CamachoDorado_2018, Chang_2017f, Cox_2007, Csaky_1998e, Fry_2010, Li_2013, Losanoff_1996, Losanoff_1997e, Mesfin_2022a, Sakellaridis_2008f, Tammana_2012j, Tanrikulu_2015e, fjbuilsRepeatedBehaviorDeliberate2024}, 17 cases (24\%) had a psychosocial motivation \cite{Akay_2015f, Benoist_2019e, Bhattacharjee_2008, Cauchi_2002, Goldman_1998f, Hardy_2023g, Kobiela_2015, Li_2013, Naji_2012f, Qureshi_2016, Riva_2018j, Sobnach_2011f, Tay_2004, Thapa_2019f, Tupesis_2004f, Wildhaber_2005, Wnęk_2015f}, 9 cases (12\%) were motivated by protest \cite{Bhumi_2024f, Gardner_2017h, Losanoff_1996, Losanoff_1997e, Tupesis_2004f}, 9 cases (12\%) had another documented motivation \cite{Ali_2020f, Ali_2022g, Emamhadi_2018, Guinan_2019f, Peixoto_2017f, Sakellaridis_2008f, Trgo_2012f, Wadhwa_2015e, Yildiz_2016e}. \paragraph*{Object Characteristics} 51 cases (71\%) ingested a large diameter object (\textgreater{}2.5cm) \cite{Akay_2015f, Al-Faham_2020k, AlShaaibi_2021b, Alao_2006i, Ali_2017, Ali_2022g, Apikotoa_2022f, Atayan_2016, Berry_2021e, Bhasin_2014, CamachoDorado_2018, Cauchi_2002, Chang_2017f, Cox_2007, Csaky_1998e, DivsalarP._2023a, Emamhadi_2018, Gardner_2017h, Guinan_2019f, Jehangir_2019h, Jin_2023, Kariholu_2008, Kerestes_2019, Kobiela_2015, Kumar_2001, Kumar_2019f, Losanoff_1996, Losanoff_1997e, Mesfin_2022a, Misra_2013, Naji_2012f, Ohno_2005, Peixoto_2017f, Qureshi_2016, Riva_2018j, Sakellaridis_2008f, Sultan_2024f, Tanrikulu_2015e, Thapa_2019f, Trgo_2012f, Wnęk_2015f, Yildiz_2016e, fjbuilsRepeatedBehaviorDeliberate2024, teWildt_2010}, 44 cases (61\%) ingested multiple objects \cite{Ali_2020f, Apikotoa_2022f, Ataya_2013, Atayan_2016, Beecroft_1998, Bhattacharjee_2008, Bhumi_2024f, CamachoDorado_2018, Cauchi_2002, Emamhadi_2018, Farhadi_2024h, Fry_2010, Goldman_1998f, Guinan_2019f, Hardy_2023g, Jehangir_2019h, Jin_2023, Kar_2015, Kariholu_2008, Kobiela_2015, Kumar_2001, Kumar_2019f, Li_2013, Liu_2005, Losanoff_1996, Mesfin_2022a, Misra_2013, Naji_2012f, Ohno_2005, Sobnach_2011f, Sultan_2024f, Tammana_2012j, Tanrikulu_2015e, Tay_2004, Thapa_2019f, Wadhwa_2015e, Wildhaber_2005, Yasin_2009, fjbuilsRepeatedBehaviorDeliberate2024, teWildt_2010}, 34 cases (47\%) ingested a sharp object \cite{AlShaaibi_2021b, Alao_2006i, Apikotoa_2022f, Ataya_2013, Benoist_2019e, Bhasin_2014, Bhattacharjee_2008, CamachoDorado_2018, Csaky_1998e, DelgadoSalazar_2020c, DivsalarP._2023a, Emamhadi_2018, Farhadi_2024h, Fry_2010, Guinan_2019f, Hardy_2023g, Jehangir_2019h, Jin_2023, Kariholu_2008, Kobiela_2015, Kumar_2019f, Losanoff_1996, Losanoff_1997e, Mesfin_2022a, Misra_2013, Sobnach_2011f, Yasin_2009, teWildt_2010}, 32 cases (44\%) ingested a long object (\textgreater{}5cm) \cite{Al-Faham_2020k, AlShaaibi_2021b, Ali_2017, Ali_2022g, Atayan_2016, Bhasin_2014, CamachoDorado_2018, Chang_2017f, Cox_2007, Csaky_1998e, DivsalarP._2023a, Emamhadi_2018, Fry_2010, Gardner_2017h, Jin_2023, Kariholu_2008, Kerestes_2019, Kobiela_2015, Kumar_2019f, Mesfin_2022a, Misra_2013, Ohno_2005, Qureshi_2016, Sakellaridis_2008f, Sultan_2024f, Thapa_2019f, Trgo_2012f, Yasin_2009, Yildiz_2016e, teWildt_2010}, 9 cases (12\%) ingested a magnet \cite{Ali_2020f, Bhumi_2024f, Cauchi_2002, Liu_2005, Naji_2012f, Ohno_2005, Tanrikulu_2015e, Tay_2004, Wildhaber_2005}, 2 cases (3\%) ingested a button battery \cite{Berry_2021e, Bhumi_2024f}. \paragraph*{Outcomes} 48 cases (67\%) experienced a complication \cite{Ali_2017, Ali_2020f, Apikotoa_2022f, Atayan_2016, Beecroft_1998, Benoist_2019e, Berry_2021e, Bhasin_2014, Bhumi_2024f, CamachoDorado_2018, Cauchi_2002, Cox_2007, Csaky_1998e, DelgadoSalazar_2020c, DivsalarP._2023a, Emamhadi_2018, Farhadi_2024h, Fry_2010, Gardner_2017h, Goldman_1998f, Jin_2023, Kariholu_2008, Kerestes_2019, Kobiela_2015, Kumar_2001, Kumar_2019f, Liu_2005, Losanoff_1996, Mesfin_2022a, Misra_2013, Naji_2012f, Ohno_2005, Sakellaridis_2008f, Sobnach_2011f, Sultan_2024f, Tanrikulu_2015e, Tay_2004, Thapa_2019f, Trgo_2012f, Tupesis_2004f, Wildhaber_2005, Wnęk_2015f, Yasin_2009, Yildiz_2016e}, 44 cases (61\%) underwent surgery \cite{Al-Faham_2020k, AlShaaibi_2021b, Alao_2006i, Ali_2017, Ali_2020f, Atayan_2016, Beecroft_1998, Bhasin_2014, CamachoDorado_2018, Cauchi_2002, Chang_2017f, Cox_2007, Csaky_1998e, DelgadoSalazar_2020c, DivsalarP._2023a, Farhadi_2024h, Fry_2010, Gardner_2017h, Jin_2023, Kariholu_2008, Kerestes_2019, Kobiela_2015, Kumar_2019f, Liu_2005, Losanoff_1996, Losanoff_1997e, Mesfin_2022a, Misra_2013, Naji_2012f, Sobnach_2011f, Tanrikulu_2015e, Tay_2004, Thapa_2019f, Tupesis_2004f, Wildhaber_2005, Wnęk_2015f, Yasin_2009, Yildiz_2016e, fjbuilsRepeatedBehaviorDeliberate2024}, 31 cases (43\%) underwent endoscopy \cite{Akay_2015f, Ali_2022g, Apikotoa_2022f, Atayan_2016, Benoist_2019e, Berry_2021e, Bhasin_2014, Bhumi_2024f, CamachoDorado_2018, Chang_2017f, DelgadoSalazar_2020c, Gardner_2017h, Guinan_2019f, Hardy_2023g, Jehangir_2019h, Kariholu_2008, Li_2013, Liu_2005, Ohno_2005, Peixoto_2017f, Qureshi_2016, Riva_2018j, Sakellaridis_2008f, Sultan_2024f, Tammana_2012j, Tanrikulu_2015e, Trgo_2012f, Wadhwa_2015e, Wnęk_2015f, teWildt_2010}, 7 cases (10\%) were managed conservatively \cite{Ataya_2013, Bhattacharjee_2008, DivsalarP._2023a, Emamhadi_2018, Goldman_1998f, Kar_2015, Kumar_2001}, 2 cases (3\%) died \cite{Emamhadi_2018, Kumar_2001}. All 90 were male gender. 90 cases (100\%) were detained at the time of ingestion \cite{Elghali_2016, Karp_1991b, Lee_2007}, 88 cases (98\%) were intentional ingestions \cite{Elghali_2016, Karp_1991b, Lee_2007}, 30 cases (33\%) had a psychiatric history documented \cite{Elghali_2016, Karp_1991b, Lee_2007}, 2 cases (2\%) had a history of prior ingestion \cite{Elghali_2016}. No cases were reported for were psychiatric inpatients, were displaced people, were under the influence of alcohol at the time of ingestion, and had a severe disability history.
\paragraph*{Motivation}  70 cases (78\%) reported protest motivation \cite{Elghali_2016, Karp_1991b, Lee_2007}, 12 cases (13\%) reported psychiatric motivation \cite{Karp_1991b}, 6 cases (7\%) reported self-harm motivation \cite{Elghali_2016, Karp_1991b}. No cases were reported for psychosocial motivation and other motivation.
\paragraph*{Object Characteristics}  68 cases (76\%) involved sharp object ingestion \cite{Elghali_2016, Karp_1991b, Lee_2007}, 32 cases (36\%) involved long (\textgreater 5cm) object ingestion \cite{Lee_2007}, 25 cases (28\%) involved ingestion of multiple objects \cite{Elghali_2016, Lee_2007}. No cases were reported for button battery ingestion, magnet ingestion, and involved large diameter (\textgreater 2.5cm) object ingestion.
\paragraph*{Outcomes}  47 cases (52\%) underwent endoscopic intervention \cite{Elghali_2016, Lee_2007}, 29 cases (32\%) were managed conservatively \cite{Elghali_2016, Karp_1991b}, 15 cases (17\%) underwent surgical intervention \cite{Elghali_2016, Karp_1991b, Lee_2007}, 6 cases (7\%) reported complications \cite{Lee_2007}, 1 case (1\%) died \cite{Elghali_2016}.
\paragraph*{Geographical Location}Cases were recorded in 33 countries: 13 cases from USA \cite{Alao_2006i, Ataya_2013, Bhumi_2024f, Fry_2010, Guinan_2019f, Hardy_2023g, Jehangir_2019h, Kerestes_2019, Kumar_2001, Liu_2005, Tammana_2012j, Tay_2004, Tupesis_2004f}; 7 cases from India \cite{Bhasin_2014, Bhattacharjee_2008, Kar_2015, Kariholu_2008, Kumar_2019f, Misra_2013, Wadhwa_2015e} and UK \cite{Beecroft_1998, Berry_2021e, Cauchi_2002, Cox_2007, Gardner_2017h, Qureshi_2016}; 6 cases from Bulgaria \cite{Losanoff_1996, Losanoff_1997e}; 5 cases from Iran \cite{DivsalarP._2023a, Emamhadi_2018, Farhadi_2024h}; 4 cases from Turkey \cite{Akay_2015f, Atayan_2016, Tanrikulu_2015e, Yildiz_2016e}; 2 cases from China \cite{Jin_2023, Li_2013}, Poland \cite{Kobiela_2015, Wnęk_2015f}, and Spain \cite{CamachoDorado_2018, fjbuilsRepeatedBehaviorDeliberate2024}; 1 case from Australia \cite{Apikotoa_2022f}, Bahrain \cite{Ali_2020f}, Croatia \cite{Trgo_2012f}, Ecuador \cite{DelgadoSalazar_2020c}, Egypt \cite{Ali_2022g}, Ethiopia \cite{Mesfin_2022a}, Germany \cite{teWildt_2010}, Greece \cite{Sakellaridis_2008f}, Hungary \cite{Csaky_1998e}, Iraq \cite{Al-Faham_2020k}, Israel \cite{Goldman_1998f}, Italy \cite{Riva_2018j}, Japan \cite{Ohno_2005}, Nepal \cite{Thapa_2019f}, Netherlands \cite{Benoist_2019e}, Oman \cite{AlShaaibi_2021b}, Pakistan \cite{Yasin_2009}, Portugal \cite{Peixoto_2017f}, Qatar \cite{Ali_2017}, Saudi Arabia \cite{Sultan_2024f}, South Africa \cite{Sobnach_2011f}, Sweden \cite{Naji_2012f}, Switzerland \cite{Wildhaber_2005}, and Taiwan \cite{Chang_2017f}. \paragraph*{Gender} 43 cases (60\%) were male \cite{Akay_2015f, Al-Faham_2020k, Alao_2006i, Ali_2017, Ali_2022g, Apikotoa_2022f, Atayan_2016, Benoist_2019e, Berry_2021e, Bhumi_2024f, CamachoDorado_2018, Csaky_1998e, Emamhadi_2018, Farhadi_2024h, Fry_2010, Gardner_2017h, Guinan_2019f, Jehangir_2019h, Jin_2023, Kobiela_2015, Kumar_2001, Kumar_2019f, Liu_2005, Losanoff_1996, Losanoff_1997e, Mesfin_2022a, Misra_2013, Qureshi_2016, Riva_2018j, Sobnach_2011f, Tammana_2012j, Tanrikulu_2015e, Tay_2004, Thapa_2019f, Trgo_2012f, Wadhwa_2015e, Yasin_2009, teWildt_2010}, 28 cases (39\%) were female \cite{AlShaaibi_2021b, Ali_2020f, Ataya_2013, Beecroft_1998, Bhasin_2014, Bhattacharjee_2008, Cauchi_2002, Chang_2017f, Cox_2007, DelgadoSalazar_2020c, DivsalarP._2023a, Goldman_1998f, Hardy_2023g, Kar_2015, Kariholu_2008, Kerestes_2019, Li_2013, Naji_2012f, Ohno_2005, Peixoto_2017f, Sakellaridis_2008f, Sultan_2024f, Tupesis_2004f, Wildhaber_2005, Wnęk_2015f, Yildiz_2016e}, 1 case (1\%) had no gender recorded \cite{fjbuilsRepeatedBehaviorDeliberate2024}. \paragraph*{Age Group} 25 cases (35\%) were between 26 and 40 years of age \cite{Alao_2006i, Ali_2022g, Apikotoa_2022f, Ataya_2013, Benoist_2019e, Bhasin_2014, Chang_2017f, Cox_2007, DelgadoSalazar_2020c, Farhadi_2024h, Fry_2010, Gardner_2017h, Guinan_2019f, Jin_2023, Kumar_2019f, Losanoff_1996, Misra_2013, Qureshi_2016, Riva_2018j, Sakellaridis_2008f, Tammana_2012j, Trgo_2012f, Wnęk_2015f, Yildiz_2016e, fjbuilsRepeatedBehaviorDeliberate2024}, 18 cases (25\%) were between 18 and 25 years of age \cite{Akay_2015f, Ali_2017, Atayan_2016, Bhattacharjee_2008, Csaky_1998e, Kar_2015, Kariholu_2008, Kobiela_2015, Losanoff_1996, Losanoff_1997e, Mesfin_2022a, Peixoto_2017f, Sobnach_2011f, Tupesis_2004f, Yasin_2009}, 13 cases (18\%) were under 18 years of age \cite{AlShaaibi_2021b, Ali_2020f, Cauchi_2002, DivsalarP._2023a, Goldman_1998f, Liu_2005, Naji_2012f, Ohno_2005, Tanrikulu_2015e, Tay_2004, Wildhaber_2005}, 11 cases (15\%) were between 41 and 60 years of age \cite{Al-Faham_2020k, Bhumi_2024f, CamachoDorado_2018, Emamhadi_2018, Hardy_2023g, Jehangir_2019h, Kumar_2001, Sultan_2024f, Thapa_2019f, Wadhwa_2015e, teWildt_2010}, 3 cases (4\%) were over 60 years of age \cite{Beecroft_1998, Kerestes_2019, Li_2013}, 2 cases (3\%) had no age documented \cite{Berry_2021e}. \paragraph*{Population} 36 cases (50\%) had a psychiatric history \cite{AlShaaibi_2021b, Alao_2006i, Ali_2020f, Apikotoa_2022f, Ataya_2013, Atayan_2016, Beecroft_1998, CamachoDorado_2018, Chang_2017f, DelgadoSalazar_2020c, DivsalarP._2023a, Farhadi_2024h, Fry_2010, Guinan_2019f, Hardy_2023g, Jehangir_2019h, Jin_2023, Kar_2015, Kerestes_2019, Kobiela_2015, Kumar_2001, Kumar_2019f, Liu_2005, Mesfin_2022a, Misra_2013, Ohno_2005, Peixoto_2017f, Sakellaridis_2008f, Sultan_2024f, Tammana_2012j, Tanrikulu_2015e, Yildiz_2016e, fjbuilsRepeatedBehaviorDeliberate2024, teWildt_2010}, 19 cases (26\%) had ingested previously \cite{Alao_2006i, Apikotoa_2022f, Berry_2021e, Bhattacharjee_2008, Csaky_1998e, DivsalarP._2023a, Emamhadi_2018, Guinan_2019f, Jehangir_2019h, Jin_2023, Liu_2005, Sakellaridis_2008f, Tanrikulu_2015e, Thapa_2019f, Yildiz_2016e, fjbuilsRepeatedBehaviorDeliberate2024, teWildt_2010}, 12 cases (17\%) were detained persons \cite{Alao_2006i, Ali_2022g, Apikotoa_2022f, Losanoff_1996, Losanoff_1997e, Qureshi_2016, Tammana_2012j, Trgo_2012f}, 7 cases (10\%) were severely disabled \cite{Atayan_2016, Kerestes_2019, Liu_2005, Ohno_2005, Peixoto_2017f, Yildiz_2016e, teWildt_2010}, 4 cases (6\%) were psychiatric inpatients \cite{DivsalarP._2023a, fjbuilsRepeatedBehaviorDeliberate2024, teWildt_2010}, 3 cases (4\%) were under the influence of alcohol \cite{Benoist_2019e, Csaky_1998e, Thapa_2019f}, 2 cases (3\%) were displaced people \cite{Akay_2015f, Gardner_2017h}. \paragraph*{Motivation} 34 cases (47\%) had a psychiatric motivation \cite{Al-Faham_2020k, Alao_2006i, Ali_2020f, Apikotoa_2022f, Ataya_2013, Atayan_2016, Bhasin_2014, Bhattacharjee_2008, DelgadoSalazar_2020c, DivsalarP._2023a, Emamhadi_2018, Farhadi_2024h, Guinan_2019f, Hardy_2023g, Jehangir_2019h, Jin_2023, Kar_2015, Kariholu_2008, Kerestes_2019, Kobiela_2015, Kumar_2001, Kumar_2019f, Li_2013, Liu_2005, Misra_2013, Ohno_2005, Sakellaridis_2008f, Sultan_2024f, Tammana_2012j, Tanrikulu_2015e, Yasin_2009, teWildt_2010}, 21 cases (29\%) were motivated by self-harm intention \cite{Al-Faham_2020k, AlShaaibi_2021b, Alao_2006i, Ali_2017, CamachoDorado_2018, Chang_2017f, Cox_2007, Csaky_1998e, Fry_2010, Li_2013, Losanoff_1996, Losanoff_1997e, Mesfin_2022a, Sakellaridis_2008f, Tammana_2012j, Tanrikulu_2015e, fjbuilsRepeatedBehaviorDeliberate2024}, 17 cases (24\%) had a psychosocial motivation \cite{Akay_2015f, Benoist_2019e, Bhattacharjee_2008, Cauchi_2002, Goldman_1998f, Hardy_2023g, Kobiela_2015, Li_2013, Naji_2012f, Qureshi_2016, Riva_2018j, Sobnach_2011f, Tay_2004, Thapa_2019f, Tupesis_2004f, Wildhaber_2005, Wnęk_2015f}, 9 cases (12\%) were motivated by protest \cite{Bhumi_2024f, Gardner_2017h, Losanoff_1996, Losanoff_1997e, Tupesis_2004f}, 9 cases (12\%) had another documented motivation \cite{Ali_2020f, Ali_2022g, Emamhadi_2018, Guinan_2019f, Peixoto_2017f, Sakellaridis_2008f, Trgo_2012f, Wadhwa_2015e, Yildiz_2016e}. \paragraph*{Object Characteristics} 51 cases (71\%) ingested a large diameter object (\textgreater{}2.5cm) \cite{Akay_2015f, Al-Faham_2020k, AlShaaibi_2021b, Alao_2006i, Ali_2017, Ali_2022g, Apikotoa_2022f, Atayan_2016, Berry_2021e, Bhasin_2014, CamachoDorado_2018, Cauchi_2002, Chang_2017f, Cox_2007, Csaky_1998e, DivsalarP._2023a, Emamhadi_2018, Gardner_2017h, Guinan_2019f, Jehangir_2019h, Jin_2023, Kariholu_2008, Kerestes_2019, Kobiela_2015, Kumar_2001, Kumar_2019f, Losanoff_1996, Losanoff_1997e, Mesfin_2022a, Misra_2013, Naji_2012f, Ohno_2005, Peixoto_2017f, Qureshi_2016, Riva_2018j, Sakellaridis_2008f, Sultan_2024f, Tanrikulu_2015e, Thapa_2019f, Trgo_2012f, Wnęk_2015f, Yildiz_2016e, fjbuilsRepeatedBehaviorDeliberate2024, teWildt_2010}, 44 cases (61\%) ingested multiple objects \cite{Ali_2020f, Apikotoa_2022f, Ataya_2013, Atayan_2016, Beecroft_1998, Bhattacharjee_2008, Bhumi_2024f, CamachoDorado_2018, Cauchi_2002, Emamhadi_2018, Farhadi_2024h, Fry_2010, Goldman_1998f, Guinan_2019f, Hardy_2023g, Jehangir_2019h, Jin_2023, Kar_2015, Kariholu_2008, Kobiela_2015, Kumar_2001, Kumar_2019f, Li_2013, Liu_2005, Losanoff_1996, Mesfin_2022a, Misra_2013, Naji_2012f, Ohno_2005, Sobnach_2011f, Sultan_2024f, Tammana_2012j, Tanrikulu_2015e, Tay_2004, Thapa_2019f, Wadhwa_2015e, Wildhaber_2005, Yasin_2009, fjbuilsRepeatedBehaviorDeliberate2024, teWildt_2010}, 34 cases (47\%) ingested a sharp object \cite{AlShaaibi_2021b, Alao_2006i, Apikotoa_2022f, Ataya_2013, Benoist_2019e, Bhasin_2014, Bhattacharjee_2008, CamachoDorado_2018, Csaky_1998e, DelgadoSalazar_2020c, DivsalarP._2023a, Emamhadi_2018, Farhadi_2024h, Fry_2010, Guinan_2019f, Hardy_2023g, Jehangir_2019h, Jin_2023, Kariholu_2008, Kobiela_2015, Kumar_2019f, Losanoff_1996, Losanoff_1997e, Mesfin_2022a, Misra_2013, Sobnach_2011f, Yasin_2009, teWildt_2010}, 32 cases (44\%) ingested a long object (\textgreater{}5cm) \cite{Al-Faham_2020k, AlShaaibi_2021b, Ali_2017, Ali_2022g, Atayan_2016, Bhasin_2014, CamachoDorado_2018, Chang_2017f, Cox_2007, Csaky_1998e, DivsalarP._2023a, Emamhadi_2018, Fry_2010, Gardner_2017h, Jin_2023, Kariholu_2008, Kerestes_2019, Kobiela_2015, Kumar_2019f, Mesfin_2022a, Misra_2013, Ohno_2005, Qureshi_2016, Sakellaridis_2008f, Sultan_2024f, Thapa_2019f, Trgo_2012f, Yasin_2009, Yildiz_2016e, teWildt_2010}, 9 cases (12\%) ingested a magnet \cite{Ali_2020f, Bhumi_2024f, Cauchi_2002, Liu_2005, Naji_2012f, Ohno_2005, Tanrikulu_2015e, Tay_2004, Wildhaber_2005}, 2 cases (3\%) ingested a button battery \cite{Berry_2021e, Bhumi_2024f}. \paragraph*{Outcomes} 48 cases (67\%) experienced a complication \cite{Ali_2017, Ali_2020f, Apikotoa_2022f, Atayan_2016, Beecroft_1998, Benoist_2019e, Berry_2021e, Bhasin_2014, Bhumi_2024f, CamachoDorado_2018, Cauchi_2002, Cox_2007, Csaky_1998e, DelgadoSalazar_2020c, DivsalarP._2023a, Emamhadi_2018, Farhadi_2024h, Fry_2010, Gardner_2017h, Goldman_1998f, Jin_2023, Kariholu_2008, Kerestes_2019, Kobiela_2015, Kumar_2001, Kumar_2019f, Liu_2005, Losanoff_1996, Mesfin_2022a, Misra_2013, Naji_2012f, Ohno_2005, Sakellaridis_2008f, Sobnach_2011f, Sultan_2024f, Tanrikulu_2015e, Tay_2004, Thapa_2019f, Trgo_2012f, Tupesis_2004f, Wildhaber_2005, Wnęk_2015f, Yasin_2009, Yildiz_2016e}, 44 cases (61\%) underwent surgery \cite{Al-Faham_2020k, AlShaaibi_2021b, Alao_2006i, Ali_2017, Ali_2020f, Atayan_2016, Beecroft_1998, Bhasin_2014, CamachoDorado_2018, Cauchi_2002, Chang_2017f, Cox_2007, Csaky_1998e, DelgadoSalazar_2020c, DivsalarP._2023a, Farhadi_2024h, Fry_2010, Gardner_2017h, Jin_2023, Kariholu_2008, Kerestes_2019, Kobiela_2015, Kumar_2019f, Liu_2005, Losanoff_1996, Losanoff_1997e, Mesfin_2022a, Misra_2013, Naji_2012f, Sobnach_2011f, Tanrikulu_2015e, Tay_2004, Thapa_2019f, Tupesis_2004f, Wildhaber_2005, Wnęk_2015f, Yasin_2009, Yildiz_2016e, fjbuilsRepeatedBehaviorDeliberate2024}, 31 cases (43\%) underwent endoscopy \cite{Akay_2015f, Ali_2022g, Apikotoa_2022f, Atayan_2016, Benoist_2019e, Berry_2021e, Bhasin_2014, Bhumi_2024f, CamachoDorado_2018, Chang_2017f, DelgadoSalazar_2020c, Gardner_2017h, Guinan_2019f, Hardy_2023g, Jehangir_2019h, Kariholu_2008, Li_2013, Liu_2005, Ohno_2005, Peixoto_2017f, Qureshi_2016, Riva_2018j, Sakellaridis_2008f, Sultan_2024f, Tammana_2012j, Tanrikulu_2015e, Trgo_2012f, Wadhwa_2015e, Wnęk_2015f, teWildt_2010}, 7 cases (10\%) were managed conservatively \cite{Ataya_2013, Bhattacharjee_2008, DivsalarP._2023a, Emamhadi_2018, Goldman_1998f, Kar_2015, Kumar_2001}, 2 cases (3\%) died \cite{Emamhadi_2018, Kumar_2001}. All 90 were male gender. 90 cases (100\%) were detained at the time of ingestion \cite{Elghali_2016, Karp_1991b, Lee_2007}, 88 cases (98\%) were intentional ingestions \cite{Elghali_2016, Karp_1991b, Lee_2007}, 30 cases (33\%) had a psychiatric history documented \cite{Elghali_2016, Karp_1991b, Lee_2007}, 2 cases (2\%) had a history of prior ingestion \cite{Elghali_2016}. No cases were reported for were psychiatric inpatients, were displaced people, were under the influence of alcohol at the time of ingestion, and had a severe disability history.
\paragraph*{Motivation}  70 cases (78\%) reported protest motivation \cite{Elghali_2016, Karp_1991b, Lee_2007}, 12 cases (13\%) reported psychiatric motivation \cite{Karp_1991b}, 6 cases (7\%) reported self-harm motivation \cite{Elghali_2016, Karp_1991b}. No cases were reported for psychosocial motivation and other motivation.
\paragraph*{Object Characteristics}  68 cases (76\%) involved sharp object ingestion \cite{Elghali_2016, Karp_1991b, Lee_2007}, 32 cases (36\%) involved long (\textgreater 5cm) object ingestion \cite{Lee_2007}, 25 cases (28\%) involved ingestion of multiple objects \cite{Elghali_2016, Lee_2007}. No cases were reported for button battery ingestion, magnet ingestion, and involved large diameter (\textgreater 2.5cm) object ingestion.
\paragraph*{Outcomes}  47 cases (52\%) underwent endoscopic intervention \cite{Elghali_2016, Lee_2007}, 29 cases (32\%) were managed conservatively \cite{Elghali_2016, Karp_1991b}, 15 cases (17\%) underwent surgical intervention \cite{Elghali_2016, Karp_1991b, Lee_2007}, 6 cases (7\%) reported complications \cite{Lee_2007}, 1 case (1\%) died \cite{Elghali_2016}.
\paragraph*{Geographical Location}Cases were recorded in 33 countries: 13 cases from USA \cite{Alao_2006i, Ataya_2013, Bhumi_2024f, Fry_2010, Guinan_2019f, Hardy_2023g, Jehangir_2019h, Kerestes_2019, Kumar_2001, Liu_2005, Tammana_2012j, Tay_2004, Tupesis_2004f}; 7 cases from India \cite{Bhasin_2014, Bhattacharjee_2008, Kar_2015, Kariholu_2008, Kumar_2019f, Misra_2013, Wadhwa_2015e} and UK \cite{Beecroft_1998, Berry_2021e, Cauchi_2002, Cox_2007, Gardner_2017h, Qureshi_2016}; 6 cases from Bulgaria \cite{Losanoff_1996, Losanoff_1997e}; 5 cases from Iran \cite{DivsalarP._2023a, Emamhadi_2018, Farhadi_2024h}; 4 cases from Turkey \cite{Akay_2015f, Atayan_2016, Tanrikulu_2015e, Yildiz_2016e}; 2 cases from China \cite{Jin_2023, Li_2013}, Poland \cite{Kobiela_2015, Wnęk_2015f}, and Spain \cite{CamachoDorado_2018, fjbuilsRepeatedBehaviorDeliberate2024}; 1 case from Australia \cite{Apikotoa_2022f}, Bahrain \cite{Ali_2020f}, Croatia \cite{Trgo_2012f}, Ecuador \cite{DelgadoSalazar_2020c}, Egypt \cite{Ali_2022g}, Ethiopia \cite{Mesfin_2022a}, Germany \cite{teWildt_2010}, Greece \cite{Sakellaridis_2008f}, Hungary \cite{Csaky_1998e}, Iraq \cite{Al-Faham_2020k}, Israel \cite{Goldman_1998f}, Italy \cite{Riva_2018j}, Japan \cite{Ohno_2005}, Nepal \cite{Thapa_2019f}, Netherlands \cite{Benoist_2019e}, Oman \cite{AlShaaibi_2021b}, Pakistan \cite{Yasin_2009}, Portugal \cite{Peixoto_2017f}, Qatar \cite{Ali_2017}, Saudi Arabia \cite{Sultan_2024f}, South Africa \cite{Sobnach_2011f}, Sweden \cite{Naji_2012f}, Switzerland \cite{Wildhaber_2005}, and Taiwan \cite{Chang_2017f}. \paragraph*{Gender} 43 cases (60\%) were male \cite{Akay_2015f, Al-Faham_2020k, Alao_2006i, Ali_2017, Ali_2022g, Apikotoa_2022f, Atayan_2016, Benoist_2019e, Berry_2021e, Bhumi_2024f, CamachoDorado_2018, Csaky_1998e, Emamhadi_2018, Farhadi_2024h, Fry_2010, Gardner_2017h, Guinan_2019f, Jehangir_2019h, Jin_2023, Kobiela_2015, Kumar_2001, Kumar_2019f, Liu_2005, Losanoff_1996, Losanoff_1997e, Mesfin_2022a, Misra_2013, Qureshi_2016, Riva_2018j, Sobnach_2011f, Tammana_2012j, Tanrikulu_2015e, Tay_2004, Thapa_2019f, Trgo_2012f, Wadhwa_2015e, Yasin_2009, teWildt_2010}, 28 cases (39\%) were female \cite{AlShaaibi_2021b, Ali_2020f, Ataya_2013, Beecroft_1998, Bhasin_2014, Bhattacharjee_2008, Cauchi_2002, Chang_2017f, Cox_2007, DelgadoSalazar_2020c, DivsalarP._2023a, Goldman_1998f, Hardy_2023g, Kar_2015, Kariholu_2008, Kerestes_2019, Li_2013, Naji_2012f, Ohno_2005, Peixoto_2017f, Sakellaridis_2008f, Sultan_2024f, Tupesis_2004f, Wildhaber_2005, Wnęk_2015f, Yildiz_2016e}, 1 case (1\%) had no gender recorded \cite{fjbuilsRepeatedBehaviorDeliberate2024}. \paragraph*{Age Group} 25 cases (35\%) were between 26 and 40 years of age \cite{Alao_2006i, Ali_2022g, Apikotoa_2022f, Ataya_2013, Benoist_2019e, Bhasin_2014, Chang_2017f, Cox_2007, DelgadoSalazar_2020c, Farhadi_2024h, Fry_2010, Gardner_2017h, Guinan_2019f, Jin_2023, Kumar_2019f, Losanoff_1996, Misra_2013, Qureshi_2016, Riva_2018j, Sakellaridis_2008f, Tammana_2012j, Trgo_2012f, Wnęk_2015f, Yildiz_2016e, fjbuilsRepeatedBehaviorDeliberate2024}, 18 cases (25\%) were between 18 and 25 years of age \cite{Akay_2015f, Ali_2017, Atayan_2016, Bhattacharjee_2008, Csaky_1998e, Kar_2015, Kariholu_2008, Kobiela_2015, Losanoff_1996, Losanoff_1997e, Mesfin_2022a, Peixoto_2017f, Sobnach_2011f, Tupesis_2004f, Yasin_2009}, 13 cases (18\%) were under 18 years of age \cite{AlShaaibi_2021b, Ali_2020f, Cauchi_2002, DivsalarP._2023a, Goldman_1998f, Liu_2005, Naji_2012f, Ohno_2005, Tanrikulu_2015e, Tay_2004, Wildhaber_2005}, 11 cases (15\%) were between 41 and 60 years of age \cite{Al-Faham_2020k, Bhumi_2024f, CamachoDorado_2018, Emamhadi_2018, Hardy_2023g, Jehangir_2019h, Kumar_2001, Sultan_2024f, Thapa_2019f, Wadhwa_2015e, teWildt_2010}, 3 cases (4\%) were over 60 years of age \cite{Beecroft_1998, Kerestes_2019, Li_2013}, 2 cases (3\%) had no age documented \cite{Berry_2021e}. \paragraph*{Population} 36 cases (50\%) had a psychiatric history \cite{AlShaaibi_2021b, Alao_2006i, Ali_2020f, Apikotoa_2022f, Ataya_2013, Atayan_2016, Beecroft_1998, CamachoDorado_2018, Chang_2017f, DelgadoSalazar_2020c, DivsalarP._2023a, Farhadi_2024h, Fry_2010, Guinan_2019f, Hardy_2023g, Jehangir_2019h, Jin_2023, Kar_2015, Kerestes_2019, Kobiela_2015, Kumar_2001, Kumar_2019f, Liu_2005, Mesfin_2022a, Misra_2013, Ohno_2005, Peixoto_2017f, Sakellaridis_2008f, Sultan_2024f, Tammana_2012j, Tanrikulu_2015e, Yildiz_2016e, fjbuilsRepeatedBehaviorDeliberate2024, teWildt_2010}, 19 cases (26\%) had ingested previously \cite{Alao_2006i, Apikotoa_2022f, Berry_2021e, Bhattacharjee_2008, Csaky_1998e, DivsalarP._2023a, Emamhadi_2018, Guinan_2019f, Jehangir_2019h, Jin_2023, Liu_2005, Sakellaridis_2008f, Tanrikulu_2015e, Thapa_2019f, Yildiz_2016e, fjbuilsRepeatedBehaviorDeliberate2024, teWildt_2010}, 12 cases (17\%) were detained persons \cite{Alao_2006i, Ali_2022g, Apikotoa_2022f, Losanoff_1996, Losanoff_1997e, Qureshi_2016, Tammana_2012j, Trgo_2012f}, 7 cases (10\%) were severely disabled \cite{Atayan_2016, Kerestes_2019, Liu_2005, Ohno_2005, Peixoto_2017f, Yildiz_2016e, teWildt_2010}, 4 cases (6\%) were psychiatric inpatients \cite{DivsalarP._2023a, fjbuilsRepeatedBehaviorDeliberate2024, teWildt_2010}, 3 cases (4\%) were under the influence of alcohol \cite{Benoist_2019e, Csaky_1998e, Thapa_2019f}, 2 cases (3\%) were displaced people \cite{Akay_2015f, Gardner_2017h}. \paragraph*{Motivation} 34 cases (47\%) had a psychiatric motivation \cite{Al-Faham_2020k, Alao_2006i, Ali_2020f, Apikotoa_2022f, Ataya_2013, Atayan_2016, Bhasin_2014, Bhattacharjee_2008, DelgadoSalazar_2020c, DivsalarP._2023a, Emamhadi_2018, Farhadi_2024h, Guinan_2019f, Hardy_2023g, Jehangir_2019h, Jin_2023, Kar_2015, Kariholu_2008, Kerestes_2019, Kobiela_2015, Kumar_2001, Kumar_2019f, Li_2013, Liu_2005, Misra_2013, Ohno_2005, Sakellaridis_2008f, Sultan_2024f, Tammana_2012j, Tanrikulu_2015e, Yasin_2009, teWildt_2010}, 21 cases (29\%) were motivated by self-harm intention \cite{Al-Faham_2020k, AlShaaibi_2021b, Alao_2006i, Ali_2017, CamachoDorado_2018, Chang_2017f, Cox_2007, Csaky_1998e, Fry_2010, Li_2013, Losanoff_1996, Losanoff_1997e, Mesfin_2022a, Sakellaridis_2008f, Tammana_2012j, Tanrikulu_2015e, fjbuilsRepeatedBehaviorDeliberate2024}, 17 cases (24\%) had a psychosocial motivation \cite{Akay_2015f, Benoist_2019e, Bhattacharjee_2008, Cauchi_2002, Goldman_1998f, Hardy_2023g, Kobiela_2015, Li_2013, Naji_2012f, Qureshi_2016, Riva_2018j, Sobnach_2011f, Tay_2004, Thapa_2019f, Tupesis_2004f, Wildhaber_2005, Wnęk_2015f}, 9 cases (12\%) were motivated by protest \cite{Bhumi_2024f, Gardner_2017h, Losanoff_1996, Losanoff_1997e, Tupesis_2004f}, 9 cases (12\%) had another documented motivation \cite{Ali_2020f, Ali_2022g, Emamhadi_2018, Guinan_2019f, Peixoto_2017f, Sakellaridis_2008f, Trgo_2012f, Wadhwa_2015e, Yildiz_2016e}. \paragraph*{Object Characteristics} 51 cases (71\%) ingested a large diameter object (\textgreater{}2.5cm) \cite{Akay_2015f, Al-Faham_2020k, AlShaaibi_2021b, Alao_2006i, Ali_2017, Ali_2022g, Apikotoa_2022f, Atayan_2016, Berry_2021e, Bhasin_2014, CamachoDorado_2018, Cauchi_2002, Chang_2017f, Cox_2007, Csaky_1998e, DivsalarP._2023a, Emamhadi_2018, Gardner_2017h, Guinan_2019f, Jehangir_2019h, Jin_2023, Kariholu_2008, Kerestes_2019, Kobiela_2015, Kumar_2001, Kumar_2019f, Losanoff_1996, Losanoff_1997e, Mesfin_2022a, Misra_2013, Naji_2012f, Ohno_2005, Peixoto_2017f, Qureshi_2016, Riva_2018j, Sakellaridis_2008f, Sultan_2024f, Tanrikulu_2015e, Thapa_2019f, Trgo_2012f, Wnęk_2015f, Yildiz_2016e, fjbuilsRepeatedBehaviorDeliberate2024, teWildt_2010}, 44 cases (61\%) ingested multiple objects \cite{Ali_2020f, Apikotoa_2022f, Ataya_2013, Atayan_2016, Beecroft_1998, Bhattacharjee_2008, Bhumi_2024f, CamachoDorado_2018, Cauchi_2002, Emamhadi_2018, Farhadi_2024h, Fry_2010, Goldman_1998f, Guinan_2019f, Hardy_2023g, Jehangir_2019h, Jin_2023, Kar_2015, Kariholu_2008, Kobiela_2015, Kumar_2001, Kumar_2019f, Li_2013, Liu_2005, Losanoff_1996, Mesfin_2022a, Misra_2013, Naji_2012f, Ohno_2005, Sobnach_2011f, Sultan_2024f, Tammana_2012j, Tanrikulu_2015e, Tay_2004, Thapa_2019f, Wadhwa_2015e, Wildhaber_2005, Yasin_2009, fjbuilsRepeatedBehaviorDeliberate2024, teWildt_2010}, 34 cases (47\%) ingested a sharp object \cite{AlShaaibi_2021b, Alao_2006i, Apikotoa_2022f, Ataya_2013, Benoist_2019e, Bhasin_2014, Bhattacharjee_2008, CamachoDorado_2018, Csaky_1998e, DelgadoSalazar_2020c, DivsalarP._2023a, Emamhadi_2018, Farhadi_2024h, Fry_2010, Guinan_2019f, Hardy_2023g, Jehangir_2019h, Jin_2023, Kariholu_2008, Kobiela_2015, Kumar_2019f, Losanoff_1996, Losanoff_1997e, Mesfin_2022a, Misra_2013, Sobnach_2011f, Yasin_2009, teWildt_2010}, 32 cases (44\%) ingested a long object (\textgreater{}5cm) \cite{Al-Faham_2020k, AlShaaibi_2021b, Ali_2017, Ali_2022g, Atayan_2016, Bhasin_2014, CamachoDorado_2018, Chang_2017f, Cox_2007, Csaky_1998e, DivsalarP._2023a, Emamhadi_2018, Fry_2010, Gardner_2017h, Jin_2023, Kariholu_2008, Kerestes_2019, Kobiela_2015, Kumar_2019f, Mesfin_2022a, Misra_2013, Ohno_2005, Qureshi_2016, Sakellaridis_2008f, Sultan_2024f, Thapa_2019f, Trgo_2012f, Yasin_2009, Yildiz_2016e, teWildt_2010}, 9 cases (12\%) ingested a magnet \cite{Ali_2020f, Bhumi_2024f, Cauchi_2002, Liu_2005, Naji_2012f, Ohno_2005, Tanrikulu_2015e, Tay_2004, Wildhaber_2005}, 2 cases (3\%) ingested a button battery \cite{Berry_2021e, Bhumi_2024f}. \paragraph*{Outcomes} 48 cases (67\%) experienced a complication \cite{Ali_2017, Ali_2020f, Apikotoa_2022f, Atayan_2016, Beecroft_1998, Benoist_2019e, Berry_2021e, Bhasin_2014, Bhumi_2024f, CamachoDorado_2018, Cauchi_2002, Cox_2007, Csaky_1998e, DelgadoSalazar_2020c, DivsalarP._2023a, Emamhadi_2018, Farhadi_2024h, Fry_2010, Gardner_2017h, Goldman_1998f, Jin_2023, Kariholu_2008, Kerestes_2019, Kobiela_2015, Kumar_2001, Kumar_2019f, Liu_2005, Losanoff_1996, Mesfin_2022a, Misra_2013, Naji_2012f, Ohno_2005, Sakellaridis_2008f, Sobnach_2011f, Sultan_2024f, Tanrikulu_2015e, Tay_2004, Thapa_2019f, Trgo_2012f, Tupesis_2004f, Wildhaber_2005, Wnęk_2015f, Yasin_2009, Yildiz_2016e}, 44 cases (61\%) underwent surgery \cite{Al-Faham_2020k, AlShaaibi_2021b, Alao_2006i, Ali_2017, Ali_2020f, Atayan_2016, Beecroft_1998, Bhasin_2014, CamachoDorado_2018, Cauchi_2002, Chang_2017f, Cox_2007, Csaky_1998e, DelgadoSalazar_2020c, DivsalarP._2023a, Farhadi_2024h, Fry_2010, Gardner_2017h, Jin_2023, Kariholu_2008, Kerestes_2019, Kobiela_2015, Kumar_2019f, Liu_2005, Losanoff_1996, Losanoff_1997e, Mesfin_2022a, Misra_2013, Naji_2012f, Sobnach_2011f, Tanrikulu_2015e, Tay_2004, Thapa_2019f, Tupesis_2004f, Wildhaber_2005, Wnęk_2015f, Yasin_2009, Yildiz_2016e, fjbuilsRepeatedBehaviorDeliberate2024}, 31 cases (43\%) underwent endoscopy \cite{Akay_2015f, Ali_2022g, Apikotoa_2022f, Atayan_2016, Benoist_2019e, Berry_2021e, Bhasin_2014, Bhumi_2024f, CamachoDorado_2018, Chang_2017f, DelgadoSalazar_2020c, Gardner_2017h, Guinan_2019f, Hardy_2023g, Jehangir_2019h, Kariholu_2008, Li_2013, Liu_2005, Ohno_2005, Peixoto_2017f, Qureshi_2016, Riva_2018j, Sakellaridis_2008f, Sultan_2024f, Tammana_2012j, Tanrikulu_2015e, Trgo_2012f, Wadhwa_2015e, Wnęk_2015f, teWildt_2010}, 7 cases (10\%) were managed conservatively \cite{Ataya_2013, Bhattacharjee_2008, DivsalarP._2023a, Emamhadi_2018, Goldman_1998f, Kar_2015, Kumar_2001}, 2 cases (3\%) died \cite{Emamhadi_2018, Kumar_2001}. All 90 were male gender. 90 cases (100\%) were detained at the time of ingestion \cite{Elghali_2016, Karp_1991b, Lee_2007}, 88 cases (98\%) were intentional ingestions \cite{Elghali_2016, Karp_1991b, Lee_2007}, 30 cases (33\%) had a psychiatric history documented \cite{Elghali_2016, Karp_1991b, Lee_2007}, 2 cases (2\%) had a history of prior ingestion \cite{Elghali_2016}. No cases were reported for were psychiatric inpatients, were displaced people, were under the influence of alcohol at the time of ingestion, and had a severe disability history.
\paragraph*{Motivation}  70 cases (78\%) reported protest motivation \cite{Elghali_2016, Karp_1991b, Lee_2007}, 12 cases (13\%) reported psychiatric motivation \cite{Karp_1991b}, 6 cases (7\%) reported self-harm motivation \cite{Elghali_2016, Karp_1991b}. No cases were reported for psychosocial motivation and other motivation.
\paragraph*{Object Characteristics}  68 cases (76\%) involved sharp object ingestion \cite{Elghali_2016, Karp_1991b, Lee_2007}, 32 cases (36\%) involved long (\textgreater 5cm) object ingestion \cite{Lee_2007}, 25 cases (28\%) involved ingestion of multiple objects \cite{Elghali_2016, Lee_2007}. No cases were reported for button battery ingestion, magnet ingestion, and involved large diameter (\textgreater 2.5cm) object ingestion.
\paragraph*{Outcomes}  47 cases (52\%) underwent endoscopic intervention \cite{Elghali_2016, Lee_2007}, 29 cases (32\%) were managed conservatively \cite{Elghali_2016, Karp_1991b}, 15 cases (17\%) underwent surgical intervention \cite{Elghali_2016, Karp_1991b, Lee_2007}, 6 cases (7\%) reported complications \cite{Lee_2007}, 1 case (1\%) died \cite{Elghali_2016}.
\paragraph*{Geographical Location}Cases were recorded in 33 countries: 13 cases from USA \cite{Alao_2006i, Ataya_2013, Bhumi_2024f, Fry_2010, Guinan_2019f, Hardy_2023g, Jehangir_2019h, Kerestes_2019, Kumar_2001, Liu_2005, Tammana_2012j, Tay_2004, Tupesis_2004f}; 7 cases from India \cite{Bhasin_2014, Bhattacharjee_2008, Kar_2015, Kariholu_2008, Kumar_2019f, Misra_2013, Wadhwa_2015e} and UK \cite{Beecroft_1998, Berry_2021e, Cauchi_2002, Cox_2007, Gardner_2017h, Qureshi_2016}; 6 cases from Bulgaria \cite{Losanoff_1996, Losanoff_1997e}; 5 cases from Iran \cite{DivsalarP._2023a, Emamhadi_2018, Farhadi_2024h}; 4 cases from Turkey \cite{Akay_2015f, Atayan_2016, Tanrikulu_2015e, Yildiz_2016e}; 2 cases from China \cite{Jin_2023, Li_2013}, Poland \cite{Kobiela_2015, Wnęk_2015f}, and Spain \cite{CamachoDorado_2018, fjbuilsRepeatedBehaviorDeliberate2024}; 1 case from Australia \cite{Apikotoa_2022f}, Bahrain \cite{Ali_2020f}, Croatia \cite{Trgo_2012f}, Ecuador \cite{DelgadoSalazar_2020c}, Egypt \cite{Ali_2022g}, Ethiopia \cite{Mesfin_2022a}, Germany \cite{teWildt_2010}, Greece \cite{Sakellaridis_2008f}, Hungary \cite{Csaky_1998e}, Iraq \cite{Al-Faham_2020k}, Israel \cite{Goldman_1998f}, Italy \cite{Riva_2018j}, Japan \cite{Ohno_2005}, Nepal \cite{Thapa_2019f}, Netherlands \cite{Benoist_2019e}, Oman \cite{AlShaaibi_2021b}, Pakistan \cite{Yasin_2009}, Portugal \cite{Peixoto_2017f}, Qatar \cite{Ali_2017}, Saudi Arabia \cite{Sultan_2024f}, South Africa \cite{Sobnach_2011f}, Sweden \cite{Naji_2012f}, Switzerland \cite{Wildhaber_2005}, and Taiwan \cite{Chang_2017f}. \paragraph*{Gender} 43 cases (60\%) were male \cite{Akay_2015f, Al-Faham_2020k, Alao_2006i, Ali_2017, Ali_2022g, Apikotoa_2022f, Atayan_2016, Benoist_2019e, Berry_2021e, Bhumi_2024f, CamachoDorado_2018, Csaky_1998e, Emamhadi_2018, Farhadi_2024h, Fry_2010, Gardner_2017h, Guinan_2019f, Jehangir_2019h, Jin_2023, Kobiela_2015, Kumar_2001, Kumar_2019f, Liu_2005, Losanoff_1996, Losanoff_1997e, Mesfin_2022a, Misra_2013, Qureshi_2016, Riva_2018j, Sobnach_2011f, Tammana_2012j, Tanrikulu_2015e, Tay_2004, Thapa_2019f, Trgo_2012f, Wadhwa_2015e, Yasin_2009, teWildt_2010}, 28 cases (39\%) were female \cite{AlShaaibi_2021b, Ali_2020f, Ataya_2013, Beecroft_1998, Bhasin_2014, Bhattacharjee_2008, Cauchi_2002, Chang_2017f, Cox_2007, DelgadoSalazar_2020c, DivsalarP._2023a, Goldman_1998f, Hardy_2023g, Kar_2015, Kariholu_2008, Kerestes_2019, Li_2013, Naji_2012f, Ohno_2005, Peixoto_2017f, Sakellaridis_2008f, Sultan_2024f, Tupesis_2004f, Wildhaber_2005, Wnęk_2015f, Yildiz_2016e}, 1 case (1\%) had no gender recorded \cite{fjbuilsRepeatedBehaviorDeliberate2024}. \paragraph*{Age Group} 25 cases (35\%) were between 26 and 40 years of age \cite{Alao_2006i, Ali_2022g, Apikotoa_2022f, Ataya_2013, Benoist_2019e, Bhasin_2014, Chang_2017f, Cox_2007, DelgadoSalazar_2020c, Farhadi_2024h, Fry_2010, Gardner_2017h, Guinan_2019f, Jin_2023, Kumar_2019f, Losanoff_1996, Misra_2013, Qureshi_2016, Riva_2018j, Sakellaridis_2008f, Tammana_2012j, Trgo_2012f, Wnęk_2015f, Yildiz_2016e, fjbuilsRepeatedBehaviorDeliberate2024}, 18 cases (25\%) were between 18 and 25 years of age \cite{Akay_2015f, Ali_2017, Atayan_2016, Bhattacharjee_2008, Csaky_1998e, Kar_2015, Kariholu_2008, Kobiela_2015, Losanoff_1996, Losanoff_1997e, Mesfin_2022a, Peixoto_2017f, Sobnach_2011f, Tupesis_2004f, Yasin_2009}, 13 cases (18\%) were under 18 years of age \cite{AlShaaibi_2021b, Ali_2020f, Cauchi_2002, DivsalarP._2023a, Goldman_1998f, Liu_2005, Naji_2012f, Ohno_2005, Tanrikulu_2015e, Tay_2004, Wildhaber_2005}, 11 cases (15\%) were between 41 and 60 years of age \cite{Al-Faham_2020k, Bhumi_2024f, CamachoDorado_2018, Emamhadi_2018, Hardy_2023g, Jehangir_2019h, Kumar_2001, Sultan_2024f, Thapa_2019f, Wadhwa_2015e, teWildt_2010}, 3 cases (4\%) were over 60 years of age \cite{Beecroft_1998, Kerestes_2019, Li_2013}, 2 cases (3\%) had no age documented \cite{Berry_2021e}. \paragraph*{Population} 36 cases (50\%) had a psychiatric history \cite{AlShaaibi_2021b, Alao_2006i, Ali_2020f, Apikotoa_2022f, Ataya_2013, Atayan_2016, Beecroft_1998, CamachoDorado_2018, Chang_2017f, DelgadoSalazar_2020c, DivsalarP._2023a, Farhadi_2024h, Fry_2010, Guinan_2019f, Hardy_2023g, Jehangir_2019h, Jin_2023, Kar_2015, Kerestes_2019, Kobiela_2015, Kumar_2001, Kumar_2019f, Liu_2005, Mesfin_2022a, Misra_2013, Ohno_2005, Peixoto_2017f, Sakellaridis_2008f, Sultan_2024f, Tammana_2012j, Tanrikulu_2015e, Yildiz_2016e, fjbuilsRepeatedBehaviorDeliberate2024, teWildt_2010}, 19 cases (26\%) had ingested previously \cite{Alao_2006i, Apikotoa_2022f, Berry_2021e, Bhattacharjee_2008, Csaky_1998e, DivsalarP._2023a, Emamhadi_2018, Guinan_2019f, Jehangir_2019h, Jin_2023, Liu_2005, Sakellaridis_2008f, Tanrikulu_2015e, Thapa_2019f, Yildiz_2016e, fjbuilsRepeatedBehaviorDeliberate2024, teWildt_2010}, 12 cases (17\%) were detained persons \cite{Alao_2006i, Ali_2022g, Apikotoa_2022f, Losanoff_1996, Losanoff_1997e, Qureshi_2016, Tammana_2012j, Trgo_2012f}, 7 cases (10\%) were severely disabled \cite{Atayan_2016, Kerestes_2019, Liu_2005, Ohno_2005, Peixoto_2017f, Yildiz_2016e, teWildt_2010}, 4 cases (6\%) were psychiatric inpatients \cite{DivsalarP._2023a, fjbuilsRepeatedBehaviorDeliberate2024, teWildt_2010}, 3 cases (4\%) were under the influence of alcohol \cite{Benoist_2019e, Csaky_1998e, Thapa_2019f}, 2 cases (3\%) were displaced people \cite{Akay_2015f, Gardner_2017h}. \paragraph*{Motivation} 34 cases (47\%) had a psychiatric motivation \cite{Al-Faham_2020k, Alao_2006i, Ali_2020f, Apikotoa_2022f, Ataya_2013, Atayan_2016, Bhasin_2014, Bhattacharjee_2008, DelgadoSalazar_2020c, DivsalarP._2023a, Emamhadi_2018, Farhadi_2024h, Guinan_2019f, Hardy_2023g, Jehangir_2019h, Jin_2023, Kar_2015, Kariholu_2008, Kerestes_2019, Kobiela_2015, Kumar_2001, Kumar_2019f, Li_2013, Liu_2005, Misra_2013, Ohno_2005, Sakellaridis_2008f, Sultan_2024f, Tammana_2012j, Tanrikulu_2015e, Yasin_2009, teWildt_2010}, 21 cases (29\%) were motivated by self-harm intention \cite{Al-Faham_2020k, AlShaaibi_2021b, Alao_2006i, Ali_2017, CamachoDorado_2018, Chang_2017f, Cox_2007, Csaky_1998e, Fry_2010, Li_2013, Losanoff_1996, Losanoff_1997e, Mesfin_2022a, Sakellaridis_2008f, Tammana_2012j, Tanrikulu_2015e, fjbuilsRepeatedBehaviorDeliberate2024}, 17 cases (24\%) had a psychosocial motivation \cite{Akay_2015f, Benoist_2019e, Bhattacharjee_2008, Cauchi_2002, Goldman_1998f, Hardy_2023g, Kobiela_2015, Li_2013, Naji_2012f, Qureshi_2016, Riva_2018j, Sobnach_2011f, Tay_2004, Thapa_2019f, Tupesis_2004f, Wildhaber_2005, Wnęk_2015f}, 9 cases (12\%) were motivated by protest \cite{Bhumi_2024f, Gardner_2017h, Losanoff_1996, Losanoff_1997e, Tupesis_2004f}, 9 cases (12\%) had another documented motivation \cite{Ali_2020f, Ali_2022g, Emamhadi_2018, Guinan_2019f, Peixoto_2017f, Sakellaridis_2008f, Trgo_2012f, Wadhwa_2015e, Yildiz_2016e}. \paragraph*{Object Characteristics} 51 cases (71\%) ingested a large diameter object (\textgreater{}2.5cm) \cite{Akay_2015f, Al-Faham_2020k, AlShaaibi_2021b, Alao_2006i, Ali_2017, Ali_2022g, Apikotoa_2022f, Atayan_2016, Berry_2021e, Bhasin_2014, CamachoDorado_2018, Cauchi_2002, Chang_2017f, Cox_2007, Csaky_1998e, DivsalarP._2023a, Emamhadi_2018, Gardner_2017h, Guinan_2019f, Jehangir_2019h, Jin_2023, Kariholu_2008, Kerestes_2019, Kobiela_2015, Kumar_2001, Kumar_2019f, Losanoff_1996, Losanoff_1997e, Mesfin_2022a, Misra_2013, Naji_2012f, Ohno_2005, Peixoto_2017f, Qureshi_2016, Riva_2018j, Sakellaridis_2008f, Sultan_2024f, Tanrikulu_2015e, Thapa_2019f, Trgo_2012f, Wnęk_2015f, Yildiz_2016e, fjbuilsRepeatedBehaviorDeliberate2024, teWildt_2010}, 44 cases (61\%) ingested multiple objects \cite{Ali_2020f, Apikotoa_2022f, Ataya_2013, Atayan_2016, Beecroft_1998, Bhattacharjee_2008, Bhumi_2024f, CamachoDorado_2018, Cauchi_2002, Emamhadi_2018, Farhadi_2024h, Fry_2010, Goldman_1998f, Guinan_2019f, Hardy_2023g, Jehangir_2019h, Jin_2023, Kar_2015, Kariholu_2008, Kobiela_2015, Kumar_2001, Kumar_2019f, Li_2013, Liu_2005, Losanoff_1996, Mesfin_2022a, Misra_2013, Naji_2012f, Ohno_2005, Sobnach_2011f, Sultan_2024f, Tammana_2012j, Tanrikulu_2015e, Tay_2004, Thapa_2019f, Wadhwa_2015e, Wildhaber_2005, Yasin_2009, fjbuilsRepeatedBehaviorDeliberate2024, teWildt_2010}, 34 cases (47\%) ingested a sharp object \cite{AlShaaibi_2021b, Alao_2006i, Apikotoa_2022f, Ataya_2013, Benoist_2019e, Bhasin_2014, Bhattacharjee_2008, CamachoDorado_2018, Csaky_1998e, DelgadoSalazar_2020c, DivsalarP._2023a, Emamhadi_2018, Farhadi_2024h, Fry_2010, Guinan_2019f, Hardy_2023g, Jehangir_2019h, Jin_2023, Kariholu_2008, Kobiela_2015, Kumar_2019f, Losanoff_1996, Losanoff_1997e, Mesfin_2022a, Misra_2013, Sobnach_2011f, Yasin_2009, teWildt_2010}, 32 cases (44\%) ingested a long object (\textgreater{}5cm) \cite{Al-Faham_2020k, AlShaaibi_2021b, Ali_2017, Ali_2022g, Atayan_2016, Bhasin_2014, CamachoDorado_2018, Chang_2017f, Cox_2007, Csaky_1998e, DivsalarP._2023a, Emamhadi_2018, Fry_2010, Gardner_2017h, Jin_2023, Kariholu_2008, Kerestes_2019, Kobiela_2015, Kumar_2019f, Mesfin_2022a, Misra_2013, Ohno_2005, Qureshi_2016, Sakellaridis_2008f, Sultan_2024f, Thapa_2019f, Trgo_2012f, Yasin_2009, Yildiz_2016e, teWildt_2010}, 9 cases (12\%) ingested a magnet \cite{Ali_2020f, Bhumi_2024f, Cauchi_2002, Liu_2005, Naji_2012f, Ohno_2005, Tanrikulu_2015e, Tay_2004, Wildhaber_2005}, 2 cases (3\%) ingested a button battery \cite{Berry_2021e, Bhumi_2024f}. \paragraph*{Outcomes} 48 cases (67\%) experienced a complication \cite{Ali_2017, Ali_2020f, Apikotoa_2022f, Atayan_2016, Beecroft_1998, Benoist_2019e, Berry_2021e, Bhasin_2014, Bhumi_2024f, CamachoDorado_2018, Cauchi_2002, Cox_2007, Csaky_1998e, DelgadoSalazar_2020c, DivsalarP._2023a, Emamhadi_2018, Farhadi_2024h, Fry_2010, Gardner_2017h, Goldman_1998f, Jin_2023, Kariholu_2008, Kerestes_2019, Kobiela_2015, Kumar_2001, Kumar_2019f, Liu_2005, Losanoff_1996, Mesfin_2022a, Misra_2013, Naji_2012f, Ohno_2005, Sakellaridis_2008f, Sobnach_2011f, Sultan_2024f, Tanrikulu_2015e, Tay_2004, Thapa_2019f, Trgo_2012f, Tupesis_2004f, Wildhaber_2005, Wnęk_2015f, Yasin_2009, Yildiz_2016e}, 44 cases (61\%) underwent surgery \cite{Al-Faham_2020k, AlShaaibi_2021b, Alao_2006i, Ali_2017, Ali_2020f, Atayan_2016, Beecroft_1998, Bhasin_2014, CamachoDorado_2018, Cauchi_2002, Chang_2017f, Cox_2007, Csaky_1998e, DelgadoSalazar_2020c, DivsalarP._2023a, Farhadi_2024h, Fry_2010, Gardner_2017h, Jin_2023, Kariholu_2008, Kerestes_2019, Kobiela_2015, Kumar_2019f, Liu_2005, Losanoff_1996, Losanoff_1997e, Mesfin_2022a, Misra_2013, Naji_2012f, Sobnach_2011f, Tanrikulu_2015e, Tay_2004, Thapa_2019f, Tupesis_2004f, Wildhaber_2005, Wnęk_2015f, Yasin_2009, Yildiz_2016e, fjbuilsRepeatedBehaviorDeliberate2024}, 31 cases (43\%) underwent endoscopy \cite{Akay_2015f, Ali_2022g, Apikotoa_2022f, Atayan_2016, Benoist_2019e, Berry_2021e, Bhasin_2014, Bhumi_2024f, CamachoDorado_2018, Chang_2017f, DelgadoSalazar_2020c, Gardner_2017h, Guinan_2019f, Hardy_2023g, Jehangir_2019h, Kariholu_2008, Li_2013, Liu_2005, Ohno_2005, Peixoto_2017f, Qureshi_2016, Riva_2018j, Sakellaridis_2008f, Sultan_2024f, Tammana_2012j, Tanrikulu_2015e, Trgo_2012f, Wadhwa_2015e, Wnęk_2015f, teWildt_2010}, 7 cases (10\%) were managed conservatively \cite{Ataya_2013, Bhattacharjee_2008, DivsalarP._2023a, Emamhadi_2018, Goldman_1998f, Kar_2015, Kumar_2001}, 2 cases (3\%) died \cite{Emamhadi_2018, Kumar_2001}. All 90 were male gender. 90 cases (100\%) were detained at the time of ingestion \cite{Elghali_2016, Karp_1991b, Lee_2007}, 88 cases (98\%) were intentional ingestions \cite{Elghali_2016, Karp_1991b, Lee_2007}, 30 cases (33\%) had a psychiatric history documented \cite{Elghali_2016, Karp_1991b, Lee_2007}, 2 cases (2\%) had a history of prior ingestion \cite{Elghali_2016}. No cases were reported for were psychiatric inpatients, were displaced people, were under the influence of alcohol at the time of ingestion, and had a severe disability history.
\paragraph*{Motivation}  70 cases (78\%) reported protest motivation \cite{Elghali_2016, Karp_1991b, Lee_2007}, 12 cases (13\%) reported psychiatric motivation \cite{Karp_1991b}, 6 cases (7\%) reported self-harm motivation \cite{Elghali_2016, Karp_1991b}. No cases were reported for psychosocial motivation and other motivation.
\paragraph*{Object Characteristics}  68 cases (76\%) involved sharp object ingestion \cite{Elghali_2016, Karp_1991b, Lee_2007}, 32 cases (36\%) involved long (\textgreater 5cm) object ingestion \cite{Lee_2007}, 25 cases (28\%) involved ingestion of multiple objects \cite{Elghali_2016, Lee_2007}. No cases were reported for button battery ingestion, magnet ingestion, and involved large diameter (\textgreater 2.5cm) object ingestion.
\paragraph*{Outcomes}  47 cases (52\%) underwent endoscopic intervention \cite{Elghali_2016, Lee_2007}, 29 cases (32\%) were managed conservatively \cite{Elghali_2016, Karp_1991b}, 15 cases (17\%) underwent surgical intervention \cite{Elghali_2016, Karp_1991b, Lee_2007}, 6 cases (7\%) reported complications \cite{Lee_2007}, 1 case (1\%) died \cite{Elghali_2016}.
\paragraph*{Geographical Location}Cases were recorded in 33 countries: 13 cases from USA \cite{Alao_2006i, Ataya_2013, Bhumi_2024f, Fry_2010, Guinan_2019f, Hardy_2023g, Jehangir_2019h, Kerestes_2019, Kumar_2001, Liu_2005, Tammana_2012j, Tay_2004, Tupesis_2004f}; 7 cases from India \cite{Bhasin_2014, Bhattacharjee_2008, Kar_2015, Kariholu_2008, Kumar_2019f, Misra_2013, Wadhwa_2015e} and UK \cite{Beecroft_1998, Berry_2021e, Cauchi_2002, Cox_2007, Gardner_2017h, Qureshi_2016}; 6 cases from Bulgaria \cite{Losanoff_1996, Losanoff_1997e}; 5 cases from Iran \cite{DivsalarP._2023a, Emamhadi_2018, Farhadi_2024h}; 4 cases from Turkey \cite{Akay_2015f, Atayan_2016, Tanrikulu_2015e, Yildiz_2016e}; 2 cases from China \cite{Jin_2023, Li_2013}, Poland \cite{Kobiela_2015, Wnęk_2015f}, and Spain \cite{CamachoDorado_2018, fjbuilsRepeatedBehaviorDeliberate2024}; 1 case from Australia \cite{Apikotoa_2022f}, Bahrain \cite{Ali_2020f}, Croatia \cite{Trgo_2012f}, Ecuador \cite{DelgadoSalazar_2020c}, Egypt \cite{Ali_2022g}, Ethiopia \cite{Mesfin_2022a}, Germany \cite{teWildt_2010}, Greece \cite{Sakellaridis_2008f}, Hungary \cite{Csaky_1998e}, Iraq \cite{Al-Faham_2020k}, Israel \cite{Goldman_1998f}, Italy \cite{Riva_2018j}, Japan \cite{Ohno_2005}, Nepal \cite{Thapa_2019f}, Netherlands \cite{Benoist_2019e}, Oman \cite{AlShaaibi_2021b}, Pakistan \cite{Yasin_2009}, Portugal \cite{Peixoto_2017f}, Qatar \cite{Ali_2017}, Saudi Arabia \cite{Sultan_2024f}, South Africa \cite{Sobnach_2011f}, Sweden \cite{Naji_2012f}, Switzerland \cite{Wildhaber_2005}, and Taiwan \cite{Chang_2017f}. \paragraph*{Gender} 43 cases (60\%) were male \cite{Akay_2015f, Al-Faham_2020k, Alao_2006i, Ali_2017, Ali_2022g, Apikotoa_2022f, Atayan_2016, Benoist_2019e, Berry_2021e, Bhumi_2024f, CamachoDorado_2018, Csaky_1998e, Emamhadi_2018, Farhadi_2024h, Fry_2010, Gardner_2017h, Guinan_2019f, Jehangir_2019h, Jin_2023, Kobiela_2015, Kumar_2001, Kumar_2019f, Liu_2005, Losanoff_1996, Losanoff_1997e, Mesfin_2022a, Misra_2013, Qureshi_2016, Riva_2018j, Sobnach_2011f, Tammana_2012j, Tanrikulu_2015e, Tay_2004, Thapa_2019f, Trgo_2012f, Wadhwa_2015e, Yasin_2009, teWildt_2010}, 28 cases (39\%) were female \cite{AlShaaibi_2021b, Ali_2020f, Ataya_2013, Beecroft_1998, Bhasin_2014, Bhattacharjee_2008, Cauchi_2002, Chang_2017f, Cox_2007, DelgadoSalazar_2020c, DivsalarP._2023a, Goldman_1998f, Hardy_2023g, Kar_2015, Kariholu_2008, Kerestes_2019, Li_2013, Naji_2012f, Ohno_2005, Peixoto_2017f, Sakellaridis_2008f, Sultan_2024f, Tupesis_2004f, Wildhaber_2005, Wnęk_2015f, Yildiz_2016e}, 1 case (1\%) had no gender recorded \cite{fjbuilsRepeatedBehaviorDeliberate2024}. \paragraph*{Age Group} 25 cases (35\%) were between 26 and 40 years of age \cite{Alao_2006i, Ali_2022g, Apikotoa_2022f, Ataya_2013, Benoist_2019e, Bhasin_2014, Chang_2017f, Cox_2007, DelgadoSalazar_2020c, Farhadi_2024h, Fry_2010, Gardner_2017h, Guinan_2019f, Jin_2023, Kumar_2019f, Losanoff_1996, Misra_2013, Qureshi_2016, Riva_2018j, Sakellaridis_2008f, Tammana_2012j, Trgo_2012f, Wnęk_2015f, Yildiz_2016e, fjbuilsRepeatedBehaviorDeliberate2024}, 18 cases (25\%) were between 18 and 25 years of age \cite{Akay_2015f, Ali_2017, Atayan_2016, Bhattacharjee_2008, Csaky_1998e, Kar_2015, Kariholu_2008, Kobiela_2015, Losanoff_1996, Losanoff_1997e, Mesfin_2022a, Peixoto_2017f, Sobnach_2011f, Tupesis_2004f, Yasin_2009}, 13 cases (18\%) were under 18 years of age \cite{AlShaaibi_2021b, Ali_2020f, Cauchi_2002, DivsalarP._2023a, Goldman_1998f, Liu_2005, Naji_2012f, Ohno_2005, Tanrikulu_2015e, Tay_2004, Wildhaber_2005}, 11 cases (15\%) were between 41 and 60 years of age \cite{Al-Faham_2020k, Bhumi_2024f, CamachoDorado_2018, Emamhadi_2018, Hardy_2023g, Jehangir_2019h, Kumar_2001, Sultan_2024f, Thapa_2019f, Wadhwa_2015e, teWildt_2010}, 3 cases (4\%) were over 60 years of age \cite{Beecroft_1998, Kerestes_2019, Li_2013}, 2 cases (3\%) had no age documented \cite{Berry_2021e}. \paragraph*{Population} 36 cases (50\%) had a psychiatric history \cite{AlShaaibi_2021b, Alao_2006i, Ali_2020f, Apikotoa_2022f, Ataya_2013, Atayan_2016, Beecroft_1998, CamachoDorado_2018, Chang_2017f, DelgadoSalazar_2020c, DivsalarP._2023a, Farhadi_2024h, Fry_2010, Guinan_2019f, Hardy_2023g, Jehangir_2019h, Jin_2023, Kar_2015, Kerestes_2019, Kobiela_2015, Kumar_2001, Kumar_2019f, Liu_2005, Mesfin_2022a, Misra_2013, Ohno_2005, Peixoto_2017f, Sakellaridis_2008f, Sultan_2024f, Tammana_2012j, Tanrikulu_2015e, Yildiz_2016e, fjbuilsRepeatedBehaviorDeliberate2024, teWildt_2010}, 19 cases (26\%) had ingested previously \cite{Alao_2006i, Apikotoa_2022f, Berry_2021e, Bhattacharjee_2008, Csaky_1998e, DivsalarP._2023a, Emamhadi_2018, Guinan_2019f, Jehangir_2019h, Jin_2023, Liu_2005, Sakellaridis_2008f, Tanrikulu_2015e, Thapa_2019f, Yildiz_2016e, fjbuilsRepeatedBehaviorDeliberate2024, teWildt_2010}, 12 cases (17\%) were detained persons \cite{Alao_2006i, Ali_2022g, Apikotoa_2022f, Losanoff_1996, Losanoff_1997e, Qureshi_2016, Tammana_2012j, Trgo_2012f}, 7 cases (10\%) were severely disabled \cite{Atayan_2016, Kerestes_2019, Liu_2005, Ohno_2005, Peixoto_2017f, Yildiz_2016e, teWildt_2010}, 4 cases (6\%) were psychiatric inpatients \cite{DivsalarP._2023a, fjbuilsRepeatedBehaviorDeliberate2024, teWildt_2010}, 3 cases (4\%) were under the influence of alcohol \cite{Benoist_2019e, Csaky_1998e, Thapa_2019f}, 2 cases (3\%) were displaced people \cite{Akay_2015f, Gardner_2017h}. \paragraph*{Motivation} 34 cases (47\%) had a psychiatric motivation \cite{Al-Faham_2020k, Alao_2006i, Ali_2020f, Apikotoa_2022f, Ataya_2013, Atayan_2016, Bhasin_2014, Bhattacharjee_2008, DelgadoSalazar_2020c, DivsalarP._2023a, Emamhadi_2018, Farhadi_2024h, Guinan_2019f, Hardy_2023g, Jehangir_2019h, Jin_2023, Kar_2015, Kariholu_2008, Kerestes_2019, Kobiela_2015, Kumar_2001, Kumar_2019f, Li_2013, Liu_2005, Misra_2013, Ohno_2005, Sakellaridis_2008f, Sultan_2024f, Tammana_2012j, Tanrikulu_2015e, Yasin_2009, teWildt_2010}, 21 cases (29\%) were motivated by self-harm intention \cite{Al-Faham_2020k, AlShaaibi_2021b, Alao_2006i, Ali_2017, CamachoDorado_2018, Chang_2017f, Cox_2007, Csaky_1998e, Fry_2010, Li_2013, Losanoff_1996, Losanoff_1997e, Mesfin_2022a, Sakellaridis_2008f, Tammana_2012j, Tanrikulu_2015e, fjbuilsRepeatedBehaviorDeliberate2024}, 17 cases (24\%) had a psychosocial motivation \cite{Akay_2015f, Benoist_2019e, Bhattacharjee_2008, Cauchi_2002, Goldman_1998f, Hardy_2023g, Kobiela_2015, Li_2013, Naji_2012f, Qureshi_2016, Riva_2018j, Sobnach_2011f, Tay_2004, Thapa_2019f, Tupesis_2004f, Wildhaber_2005, Wnęk_2015f}, 9 cases (12\%) were motivated by protest \cite{Bhumi_2024f, Gardner_2017h, Losanoff_1996, Losanoff_1997e, Tupesis_2004f}, 9 cases (12\%) had another documented motivation \cite{Ali_2020f, Ali_2022g, Emamhadi_2018, Guinan_2019f, Peixoto_2017f, Sakellaridis_2008f, Trgo_2012f, Wadhwa_2015e, Yildiz_2016e}. \paragraph*{Object Characteristics} 51 cases (71\%) ingested a large diameter object (\textgreater{}2.5cm) \cite{Akay_2015f, Al-Faham_2020k, AlShaaibi_2021b, Alao_2006i, Ali_2017, Ali_2022g, Apikotoa_2022f, Atayan_2016, Berry_2021e, Bhasin_2014, CamachoDorado_2018, Cauchi_2002, Chang_2017f, Cox_2007, Csaky_1998e, DivsalarP._2023a, Emamhadi_2018, Gardner_2017h, Guinan_2019f, Jehangir_2019h, Jin_2023, Kariholu_2008, Kerestes_2019, Kobiela_2015, Kumar_2001, Kumar_2019f, Losanoff_1996, Losanoff_1997e, Mesfin_2022a, Misra_2013, Naji_2012f, Ohno_2005, Peixoto_2017f, Qureshi_2016, Riva_2018j, Sakellaridis_2008f, Sultan_2024f, Tanrikulu_2015e, Thapa_2019f, Trgo_2012f, Wnęk_2015f, Yildiz_2016e, fjbuilsRepeatedBehaviorDeliberate2024, teWildt_2010}, 44 cases (61\%) ingested multiple objects \cite{Ali_2020f, Apikotoa_2022f, Ataya_2013, Atayan_2016, Beecroft_1998, Bhattacharjee_2008, Bhumi_2024f, CamachoDorado_2018, Cauchi_2002, Emamhadi_2018, Farhadi_2024h, Fry_2010, Goldman_1998f, Guinan_2019f, Hardy_2023g, Jehangir_2019h, Jin_2023, Kar_2015, Kariholu_2008, Kobiela_2015, Kumar_2001, Kumar_2019f, Li_2013, Liu_2005, Losanoff_1996, Mesfin_2022a, Misra_2013, Naji_2012f, Ohno_2005, Sobnach_2011f, Sultan_2024f, Tammana_2012j, Tanrikulu_2015e, Tay_2004, Thapa_2019f, Wadhwa_2015e, Wildhaber_2005, Yasin_2009, fjbuilsRepeatedBehaviorDeliberate2024, teWildt_2010}, 34 cases (47\%) ingested a sharp object \cite{AlShaaibi_2021b, Alao_2006i, Apikotoa_2022f, Ataya_2013, Benoist_2019e, Bhasin_2014, Bhattacharjee_2008, CamachoDorado_2018, Csaky_1998e, DelgadoSalazar_2020c, DivsalarP._2023a, Emamhadi_2018, Farhadi_2024h, Fry_2010, Guinan_2019f, Hardy_2023g, Jehangir_2019h, Jin_2023, Kariholu_2008, Kobiela_2015, Kumar_2019f, Losanoff_1996, Losanoff_1997e, Mesfin_2022a, Misra_2013, Sobnach_2011f, Yasin_2009, teWildt_2010}, 32 cases (44\%) ingested a long object (\textgreater{}5cm) \cite{Al-Faham_2020k, AlShaaibi_2021b, Ali_2017, Ali_2022g, Atayan_2016, Bhasin_2014, CamachoDorado_2018, Chang_2017f, Cox_2007, Csaky_1998e, DivsalarP._2023a, Emamhadi_2018, Fry_2010, Gardner_2017h, Jin_2023, Kariholu_2008, Kerestes_2019, Kobiela_2015, Kumar_2019f, Mesfin_2022a, Misra_2013, Ohno_2005, Qureshi_2016, Sakellaridis_2008f, Sultan_2024f, Thapa_2019f, Trgo_2012f, Yasin_2009, Yildiz_2016e, teWildt_2010}, 9 cases (12\%) ingested a magnet \cite{Ali_2020f, Bhumi_2024f, Cauchi_2002, Liu_2005, Naji_2012f, Ohno_2005, Tanrikulu_2015e, Tay_2004, Wildhaber_2005}, 2 cases (3\%) ingested a button battery \cite{Berry_2021e, Bhumi_2024f}. \paragraph*{Outcomes} 48 cases (67\%) experienced a complication \cite{Ali_2017, Ali_2020f, Apikotoa_2022f, Atayan_2016, Beecroft_1998, Benoist_2019e, Berry_2021e, Bhasin_2014, Bhumi_2024f, CamachoDorado_2018, Cauchi_2002, Cox_2007, Csaky_1998e, DelgadoSalazar_2020c, DivsalarP._2023a, Emamhadi_2018, Farhadi_2024h, Fry_2010, Gardner_2017h, Goldman_1998f, Jin_2023, Kariholu_2008, Kerestes_2019, Kobiela_2015, Kumar_2001, Kumar_2019f, Liu_2005, Losanoff_1996, Mesfin_2022a, Misra_2013, Naji_2012f, Ohno_2005, Sakellaridis_2008f, Sobnach_2011f, Sultan_2024f, Tanrikulu_2015e, Tay_2004, Thapa_2019f, Trgo_2012f, Tupesis_2004f, Wildhaber_2005, Wnęk_2015f, Yasin_2009, Yildiz_2016e}, 44 cases (61\%) underwent surgery \cite{Al-Faham_2020k, AlShaaibi_2021b, Alao_2006i, Ali_2017, Ali_2020f, Atayan_2016, Beecroft_1998, Bhasin_2014, CamachoDorado_2018, Cauchi_2002, Chang_2017f, Cox_2007, Csaky_1998e, DelgadoSalazar_2020c, DivsalarP._2023a, Farhadi_2024h, Fry_2010, Gardner_2017h, Jin_2023, Kariholu_2008, Kerestes_2019, Kobiela_2015, Kumar_2019f, Liu_2005, Losanoff_1996, Losanoff_1997e, Mesfin_2022a, Misra_2013, Naji_2012f, Sobnach_2011f, Tanrikulu_2015e, Tay_2004, Thapa_2019f, Tupesis_2004f, Wildhaber_2005, Wnęk_2015f, Yasin_2009, Yildiz_2016e, fjbuilsRepeatedBehaviorDeliberate2024}, 31 cases (43\%) underwent endoscopy \cite{Akay_2015f, Ali_2022g, Apikotoa_2022f, Atayan_2016, Benoist_2019e, Berry_2021e, Bhasin_2014, Bhumi_2024f, CamachoDorado_2018, Chang_2017f, DelgadoSalazar_2020c, Gardner_2017h, Guinan_2019f, Hardy_2023g, Jehangir_2019h, Kariholu_2008, Li_2013, Liu_2005, Ohno_2005, Peixoto_2017f, Qureshi_2016, Riva_2018j, Sakellaridis_2008f, Sultan_2024f, Tammana_2012j, Tanrikulu_2015e, Trgo_2012f, Wadhwa_2015e, Wnęk_2015f, teWildt_2010}, 7 cases (10\%) were managed conservatively \cite{Ataya_2013, Bhattacharjee_2008, DivsalarP._2023a, Emamhadi_2018, Goldman_1998f, Kar_2015, Kumar_2001}, 2 cases (3\%) died \cite{Emamhadi_2018, Kumar_2001}. All 90 were male gender. 90 cases (100\%) were detained at the time of ingestion \cite{Elghali_2016, Karp_1991b, Lee_2007}, 88 cases (98\%) were intentional ingestions \cite{Elghali_2016, Karp_1991b, Lee_2007}, 30 cases (33\%) had a psychiatric history documented \cite{Elghali_2016, Karp_1991b, Lee_2007}, 2 cases (2\%) had a history of prior ingestion \cite{Elghali_2016}. No cases were reported for were psychiatric inpatients, were displaced people, were under the influence of alcohol at the time of ingestion, and had a severe disability history.
\paragraph*{Motivation}  70 cases (78\%) reported protest motivation \cite{Elghali_2016, Karp_1991b, Lee_2007}, 12 cases (13\%) reported psychiatric motivation \cite{Karp_1991b}, 6 cases (7\%) reported self-harm motivation \cite{Elghali_2016, Karp_1991b}. No cases were reported for psychosocial motivation and other motivation.
\paragraph*{Object Characteristics}  68 cases (76\%) involved sharp object ingestion \cite{Elghali_2016, Karp_1991b, Lee_2007}, 32 cases (36\%) involved long (\textgreater 5cm) object ingestion \cite{Lee_2007}, 25 cases (28\%) involved ingestion of multiple objects \cite{Elghali_2016, Lee_2007}. No cases were reported for button battery ingestion, magnet ingestion, and involved large diameter (\textgreater 2.5cm) object ingestion.
\paragraph*{Outcomes}  47 cases (52\%) underwent endoscopic intervention \cite{Elghali_2016, Lee_2007}, 29 cases (32\%) were managed conservatively \cite{Elghali_2016, Karp_1991b}, 15 cases (17\%) underwent surgical intervention \cite{Elghali_2016, Karp_1991b, Lee_2007}, 6 cases (7\%) reported complications \cite{Lee_2007}, 1 case (1\%) died \cite{Elghali_2016}.
\paragraph*{Geographical Location}Cases were recorded in 33 countries: 13 cases from USA \cite{Alao_2006i, Ataya_2013, Bhumi_2024f, Fry_2010, Guinan_2019f, Hardy_2023g, Jehangir_2019h, Kerestes_2019, Kumar_2001, Liu_2005, Tammana_2012j, Tay_2004, Tupesis_2004f}; 7 cases from India \cite{Bhasin_2014, Bhattacharjee_2008, Kar_2015, Kariholu_2008, Kumar_2019f, Misra_2013, Wadhwa_2015e} and UK \cite{Beecroft_1998, Berry_2021e, Cauchi_2002, Cox_2007, Gardner_2017h, Qureshi_2016}; 6 cases from Bulgaria \cite{Losanoff_1996, Losanoff_1997e}; 5 cases from Iran \cite{DivsalarP._2023a, Emamhadi_2018, Farhadi_2024h}; 4 cases from Turkey \cite{Akay_2015f, Atayan_2016, Tanrikulu_2015e, Yildiz_2016e}; 2 cases from China \cite{Jin_2023, Li_2013}, Poland \cite{Kobiela_2015, Wnęk_2015f}, and Spain \cite{CamachoDorado_2018, fjbuilsRepeatedBehaviorDeliberate2024}; 1 case from Australia \cite{Apikotoa_2022f}, Bahrain \cite{Ali_2020f}, Croatia \cite{Trgo_2012f}, Ecuador \cite{DelgadoSalazar_2020c}, Egypt \cite{Ali_2022g}, Ethiopia \cite{Mesfin_2022a}, Germany \cite{teWildt_2010}, Greece \cite{Sakellaridis_2008f}, Hungary \cite{Csaky_1998e}, Iraq \cite{Al-Faham_2020k}, Israel \cite{Goldman_1998f}, Italy \cite{Riva_2018j}, Japan \cite{Ohno_2005}, Nepal \cite{Thapa_2019f}, Netherlands \cite{Benoist_2019e}, Oman \cite{AlShaaibi_2021b}, Pakistan \cite{Yasin_2009}, Portugal \cite{Peixoto_2017f}, Qatar \cite{Ali_2017}, Saudi Arabia \cite{Sultan_2024f}, South Africa \cite{Sobnach_2011f}, Sweden \cite{Naji_2012f}, Switzerland \cite{Wildhaber_2005}, and Taiwan \cite{Chang_2017f}. \paragraph*{Gender} 43 cases (60\%) were male \cite{Akay_2015f, Al-Faham_2020k, Alao_2006i, Ali_2017, Ali_2022g, Apikotoa_2022f, Atayan_2016, Benoist_2019e, Berry_2021e, Bhumi_2024f, CamachoDorado_2018, Csaky_1998e, Emamhadi_2018, Farhadi_2024h, Fry_2010, Gardner_2017h, Guinan_2019f, Jehangir_2019h, Jin_2023, Kobiela_2015, Kumar_2001, Kumar_2019f, Liu_2005, Losanoff_1996, Losanoff_1997e, Mesfin_2022a, Misra_2013, Qureshi_2016, Riva_2018j, Sobnach_2011f, Tammana_2012j, Tanrikulu_2015e, Tay_2004, Thapa_2019f, Trgo_2012f, Wadhwa_2015e, Yasin_2009, teWildt_2010}, 28 cases (39\%) were female \cite{AlShaaibi_2021b, Ali_2020f, Ataya_2013, Beecroft_1998, Bhasin_2014, Bhattacharjee_2008, Cauchi_2002, Chang_2017f, Cox_2007, DelgadoSalazar_2020c, DivsalarP._2023a, Goldman_1998f, Hardy_2023g, Kar_2015, Kariholu_2008, Kerestes_2019, Li_2013, Naji_2012f, Ohno_2005, Peixoto_2017f, Sakellaridis_2008f, Sultan_2024f, Tupesis_2004f, Wildhaber_2005, Wnęk_2015f, Yildiz_2016e}, 1 case (1\%) had no gender recorded \cite{fjbuilsRepeatedBehaviorDeliberate2024}. \paragraph*{Age Group} 25 cases (35\%) were between 26 and 40 years of age \cite{Alao_2006i, Ali_2022g, Apikotoa_2022f, Ataya_2013, Benoist_2019e, Bhasin_2014, Chang_2017f, Cox_2007, DelgadoSalazar_2020c, Farhadi_2024h, Fry_2010, Gardner_2017h, Guinan_2019f, Jin_2023, Kumar_2019f, Losanoff_1996, Misra_2013, Qureshi_2016, Riva_2018j, Sakellaridis_2008f, Tammana_2012j, Trgo_2012f, Wnęk_2015f, Yildiz_2016e, fjbuilsRepeatedBehaviorDeliberate2024}, 18 cases (25\%) were between 18 and 25 years of age \cite{Akay_2015f, Ali_2017, Atayan_2016, Bhattacharjee_2008, Csaky_1998e, Kar_2015, Kariholu_2008, Kobiela_2015, Losanoff_1996, Losanoff_1997e, Mesfin_2022a, Peixoto_2017f, Sobnach_2011f, Tupesis_2004f, Yasin_2009}, 13 cases (18\%) were under 18 years of age \cite{AlShaaibi_2021b, Ali_2020f, Cauchi_2002, DivsalarP._2023a, Goldman_1998f, Liu_2005, Naji_2012f, Ohno_2005, Tanrikulu_2015e, Tay_2004, Wildhaber_2005}, 11 cases (15\%) were between 41 and 60 years of age \cite{Al-Faham_2020k, Bhumi_2024f, CamachoDorado_2018, Emamhadi_2018, Hardy_2023g, Jehangir_2019h, Kumar_2001, Sultan_2024f, Thapa_2019f, Wadhwa_2015e, teWildt_2010}, 3 cases (4\%) were over 60 years of age \cite{Beecroft_1998, Kerestes_2019, Li_2013}, 2 cases (3\%) had no age documented \cite{Berry_2021e}. \paragraph*{Population} 36 cases (50\%) had a psychiatric history \cite{AlShaaibi_2021b, Alao_2006i, Ali_2020f, Apikotoa_2022f, Ataya_2013, Atayan_2016, Beecroft_1998, CamachoDorado_2018, Chang_2017f, DelgadoSalazar_2020c, DivsalarP._2023a, Farhadi_2024h, Fry_2010, Guinan_2019f, Hardy_2023g, Jehangir_2019h, Jin_2023, Kar_2015, Kerestes_2019, Kobiela_2015, Kumar_2001, Kumar_2019f, Liu_2005, Mesfin_2022a, Misra_2013, Ohno_2005, Peixoto_2017f, Sakellaridis_2008f, Sultan_2024f, Tammana_2012j, Tanrikulu_2015e, Yildiz_2016e, fjbuilsRepeatedBehaviorDeliberate2024, teWildt_2010}, 19 cases (26\%) had ingested previously \cite{Alao_2006i, Apikotoa_2022f, Berry_2021e, Bhattacharjee_2008, Csaky_1998e, DivsalarP._2023a, Emamhadi_2018, Guinan_2019f, Jehangir_2019h, Jin_2023, Liu_2005, Sakellaridis_2008f, Tanrikulu_2015e, Thapa_2019f, Yildiz_2016e, fjbuilsRepeatedBehaviorDeliberate2024, teWildt_2010}, 12 cases (17\%) were detained persons \cite{Alao_2006i, Ali_2022g, Apikotoa_2022f, Losanoff_1996, Losanoff_1997e, Qureshi_2016, Tammana_2012j, Trgo_2012f}, 7 cases (10\%) were severely disabled \cite{Atayan_2016, Kerestes_2019, Liu_2005, Ohno_2005, Peixoto_2017f, Yildiz_2016e, teWildt_2010}, 4 cases (6\%) were psychiatric inpatients \cite{DivsalarP._2023a, fjbuilsRepeatedBehaviorDeliberate2024, teWildt_2010}, 3 cases (4\%) were under the influence of alcohol \cite{Benoist_2019e, Csaky_1998e, Thapa_2019f}, 2 cases (3\%) were displaced people \cite{Akay_2015f, Gardner_2017h}. \paragraph*{Motivation} 34 cases (47\%) had a psychiatric motivation \cite{Al-Faham_2020k, Alao_2006i, Ali_2020f, Apikotoa_2022f, Ataya_2013, Atayan_2016, Bhasin_2014, Bhattacharjee_2008, DelgadoSalazar_2020c, DivsalarP._2023a, Emamhadi_2018, Farhadi_2024h, Guinan_2019f, Hardy_2023g, Jehangir_2019h, Jin_2023, Kar_2015, Kariholu_2008, Kerestes_2019, Kobiela_2015, Kumar_2001, Kumar_2019f, Li_2013, Liu_2005, Misra_2013, Ohno_2005, Sakellaridis_2008f, Sultan_2024f, Tammana_2012j, Tanrikulu_2015e, Yasin_2009, teWildt_2010}, 21 cases (29\%) were motivated by self-harm intention \cite{Al-Faham_2020k, AlShaaibi_2021b, Alao_2006i, Ali_2017, CamachoDorado_2018, Chang_2017f, Cox_2007, Csaky_1998e, Fry_2010, Li_2013, Losanoff_1996, Losanoff_1997e, Mesfin_2022a, Sakellaridis_2008f, Tammana_2012j, Tanrikulu_2015e, fjbuilsRepeatedBehaviorDeliberate2024}, 17 cases (24\%) had a psychosocial motivation \cite{Akay_2015f, Benoist_2019e, Bhattacharjee_2008, Cauchi_2002, Goldman_1998f, Hardy_2023g, Kobiela_2015, Li_2013, Naji_2012f, Qureshi_2016, Riva_2018j, Sobnach_2011f, Tay_2004, Thapa_2019f, Tupesis_2004f, Wildhaber_2005, Wnęk_2015f}, 9 cases (12\%) were motivated by protest \cite{Bhumi_2024f, Gardner_2017h, Losanoff_1996, Losanoff_1997e, Tupesis_2004f}, 9 cases (12\%) had another documented motivation \cite{Ali_2020f, Ali_2022g, Emamhadi_2018, Guinan_2019f, Peixoto_2017f, Sakellaridis_2008f, Trgo_2012f, Wadhwa_2015e, Yildiz_2016e}. \paragraph*{Object Characteristics} 51 cases (71\%) ingested a large diameter object (\textgreater{}2.5cm) \cite{Akay_2015f, Al-Faham_2020k, AlShaaibi_2021b, Alao_2006i, Ali_2017, Ali_2022g, Apikotoa_2022f, Atayan_2016, Berry_2021e, Bhasin_2014, CamachoDorado_2018, Cauchi_2002, Chang_2017f, Cox_2007, Csaky_1998e, DivsalarP._2023a, Emamhadi_2018, Gardner_2017h, Guinan_2019f, Jehangir_2019h, Jin_2023, Kariholu_2008, Kerestes_2019, Kobiela_2015, Kumar_2001, Kumar_2019f, Losanoff_1996, Losanoff_1997e, Mesfin_2022a, Misra_2013, Naji_2012f, Ohno_2005, Peixoto_2017f, Qureshi_2016, Riva_2018j, Sakellaridis_2008f, Sultan_2024f, Tanrikulu_2015e, Thapa_2019f, Trgo_2012f, Wnęk_2015f, Yildiz_2016e, fjbuilsRepeatedBehaviorDeliberate2024, teWildt_2010}, 44 cases (61\%) ingested multiple objects \cite{Ali_2020f, Apikotoa_2022f, Ataya_2013, Atayan_2016, Beecroft_1998, Bhattacharjee_2008, Bhumi_2024f, CamachoDorado_2018, Cauchi_2002, Emamhadi_2018, Farhadi_2024h, Fry_2010, Goldman_1998f, Guinan_2019f, Hardy_2023g, Jehangir_2019h, Jin_2023, Kar_2015, Kariholu_2008, Kobiela_2015, Kumar_2001, Kumar_2019f, Li_2013, Liu_2005, Losanoff_1996, Mesfin_2022a, Misra_2013, Naji_2012f, Ohno_2005, Sobnach_2011f, Sultan_2024f, Tammana_2012j, Tanrikulu_2015e, Tay_2004, Thapa_2019f, Wadhwa_2015e, Wildhaber_2005, Yasin_2009, fjbuilsRepeatedBehaviorDeliberate2024, teWildt_2010}, 34 cases (47\%) ingested a sharp object \cite{AlShaaibi_2021b, Alao_2006i, Apikotoa_2022f, Ataya_2013, Benoist_2019e, Bhasin_2014, Bhattacharjee_2008, CamachoDorado_2018, Csaky_1998e, DelgadoSalazar_2020c, DivsalarP._2023a, Emamhadi_2018, Farhadi_2024h, Fry_2010, Guinan_2019f, Hardy_2023g, Jehangir_2019h, Jin_2023, Kariholu_2008, Kobiela_2015, Kumar_2019f, Losanoff_1996, Losanoff_1997e, Mesfin_2022a, Misra_2013, Sobnach_2011f, Yasin_2009, teWildt_2010}, 32 cases (44\%) ingested a long object (\textgreater{}5cm) \cite{Al-Faham_2020k, AlShaaibi_2021b, Ali_2017, Ali_2022g, Atayan_2016, Bhasin_2014, CamachoDorado_2018, Chang_2017f, Cox_2007, Csaky_1998e, DivsalarP._2023a, Emamhadi_2018, Fry_2010, Gardner_2017h, Jin_2023, Kariholu_2008, Kerestes_2019, Kobiela_2015, Kumar_2019f, Mesfin_2022a, Misra_2013, Ohno_2005, Qureshi_2016, Sakellaridis_2008f, Sultan_2024f, Thapa_2019f, Trgo_2012f, Yasin_2009, Yildiz_2016e, teWildt_2010}, 9 cases (12\%) ingested a magnet \cite{Ali_2020f, Bhumi_2024f, Cauchi_2002, Liu_2005, Naji_2012f, Ohno_2005, Tanrikulu_2015e, Tay_2004, Wildhaber_2005}, 2 cases (3\%) ingested a button battery \cite{Berry_2021e, Bhumi_2024f}. \paragraph*{Outcomes} 48 cases (67\%) experienced a complication \cite{Ali_2017, Ali_2020f, Apikotoa_2022f, Atayan_2016, Beecroft_1998, Benoist_2019e, Berry_2021e, Bhasin_2014, Bhumi_2024f, CamachoDorado_2018, Cauchi_2002, Cox_2007, Csaky_1998e, DelgadoSalazar_2020c, DivsalarP._2023a, Emamhadi_2018, Farhadi_2024h, Fry_2010, Gardner_2017h, Goldman_1998f, Jin_2023, Kariholu_2008, Kerestes_2019, Kobiela_2015, Kumar_2001, Kumar_2019f, Liu_2005, Losanoff_1996, Mesfin_2022a, Misra_2013, Naji_2012f, Ohno_2005, Sakellaridis_2008f, Sobnach_2011f, Sultan_2024f, Tanrikulu_2015e, Tay_2004, Thapa_2019f, Trgo_2012f, Tupesis_2004f, Wildhaber_2005, Wnęk_2015f, Yasin_2009, Yildiz_2016e}, 44 cases (61\%) underwent surgery \cite{Al-Faham_2020k, AlShaaibi_2021b, Alao_2006i, Ali_2017, Ali_2020f, Atayan_2016, Beecroft_1998, Bhasin_2014, CamachoDorado_2018, Cauchi_2002, Chang_2017f, Cox_2007, Csaky_1998e, DelgadoSalazar_2020c, DivsalarP._2023a, Farhadi_2024h, Fry_2010, Gardner_2017h, Jin_2023, Kariholu_2008, Kerestes_2019, Kobiela_2015, Kumar_2019f, Liu_2005, Losanoff_1996, Losanoff_1997e, Mesfin_2022a, Misra_2013, Naji_2012f, Sobnach_2011f, Tanrikulu_2015e, Tay_2004, Thapa_2019f, Tupesis_2004f, Wildhaber_2005, Wnęk_2015f, Yasin_2009, Yildiz_2016e, fjbuilsRepeatedBehaviorDeliberate2024}, 31 cases (43\%) underwent endoscopy \cite{Akay_2015f, Ali_2022g, Apikotoa_2022f, Atayan_2016, Benoist_2019e, Berry_2021e, Bhasin_2014, Bhumi_2024f, CamachoDorado_2018, Chang_2017f, DelgadoSalazar_2020c, Gardner_2017h, Guinan_2019f, Hardy_2023g, Jehangir_2019h, Kariholu_2008, Li_2013, Liu_2005, Ohno_2005, Peixoto_2017f, Qureshi_2016, Riva_2018j, Sakellaridis_2008f, Sultan_2024f, Tammana_2012j, Tanrikulu_2015e, Trgo_2012f, Wadhwa_2015e, Wnęk_2015f, teWildt_2010}, 7 cases (10\%) were managed conservatively \cite{Ataya_2013, Bhattacharjee_2008, DivsalarP._2023a, Emamhadi_2018, Goldman_1998f, Kar_2015, Kumar_2001}, 2 cases (3\%) died \cite{Emamhadi_2018, Kumar_2001}. All 90 were male gender. 90 cases (100\%) were detained at the time of ingestion \cite{Elghali_2016, Karp_1991b, Lee_2007}, 88 cases (98\%) were intentional ingestions \cite{Elghali_2016, Karp_1991b, Lee_2007}, 30 cases (33\%) had a psychiatric history documented \cite{Elghali_2016, Karp_1991b, Lee_2007}, 2 cases (2\%) had a history of prior ingestion \cite{Elghali_2016}. No cases were reported for were psychiatric inpatients, were displaced people, were under the influence of alcohol at the time of ingestion, and had a severe disability history.
\paragraph*{Motivation}  70 cases (78\%) reported protest motivation \cite{Elghali_2016, Karp_1991b, Lee_2007}, 12 cases (13\%) reported psychiatric motivation \cite{Karp_1991b}, 6 cases (7\%) reported self-harm motivation \cite{Elghali_2016, Karp_1991b}. No cases were reported for psychosocial motivation and other motivation.
\paragraph*{Object Characteristics}  68 cases (76\%) involved sharp object ingestion \cite{Elghali_2016, Karp_1991b, Lee_2007}, 32 cases (36\%) involved long (\textgreater 5cm) object ingestion \cite{Lee_2007}, 25 cases (28\%) involved ingestion of multiple objects \cite{Elghali_2016, Lee_2007}. No cases were reported for button battery ingestion, magnet ingestion, and involved large diameter (\textgreater 2.5cm) object ingestion.
\paragraph*{Outcomes}  47 cases (52\%) underwent endoscopic intervention \cite{Elghali_2016, Lee_2007}, 29 cases (32\%) were managed conservatively \cite{Elghali_2016, Karp_1991b}, 15 cases (17\%) underwent surgical intervention \cite{Elghali_2016, Karp_1991b, Lee_2007}, 6 cases (7\%) reported complications \cite{Lee_2007}, 1 case (1\%) died \cite{Elghali_2016}.
\paragraph*{Geographical Location}Cases were recorded in 33 countries: 13 cases from USA \cite{Alao_2006i, Ataya_2013, Bhumi_2024f, Fry_2010, Guinan_2019f, Hardy_2023g, Jehangir_2019h, Kerestes_2019, Kumar_2001, Liu_2005, Tammana_2012j, Tay_2004, Tupesis_2004f}; 7 cases from India \cite{Bhasin_2014, Bhattacharjee_2008, Kar_2015, Kariholu_2008, Kumar_2019f, Misra_2013, Wadhwa_2015e} and UK \cite{Beecroft_1998, Berry_2021e, Cauchi_2002, Cox_2007, Gardner_2017h, Qureshi_2016}; 6 cases from Bulgaria \cite{Losanoff_1996, Losanoff_1997e}; 5 cases from Iran \cite{DivsalarP._2023a, Emamhadi_2018, Farhadi_2024h}; 4 cases from Turkey \cite{Akay_2015f, Atayan_2016, Tanrikulu_2015e, Yildiz_2016e}; 2 cases from China \cite{Jin_2023, Li_2013}, Poland \cite{Kobiela_2015, Wnęk_2015f}, and Spain \cite{CamachoDorado_2018, fjbuilsRepeatedBehaviorDeliberate2024}; 1 case from Australia \cite{Apikotoa_2022f}, Bahrain \cite{Ali_2020f}, Croatia \cite{Trgo_2012f}, Ecuador \cite{DelgadoSalazar_2020c}, Egypt \cite{Ali_2022g}, Ethiopia \cite{Mesfin_2022a}, Germany \cite{teWildt_2010}, Greece \cite{Sakellaridis_2008f}, Hungary \cite{Csaky_1998e}, Iraq \cite{Al-Faham_2020k}, Israel \cite{Goldman_1998f}, Italy \cite{Riva_2018j}, Japan \cite{Ohno_2005}, Nepal \cite{Thapa_2019f}, Netherlands \cite{Benoist_2019e}, Oman \cite{AlShaaibi_2021b}, Pakistan \cite{Yasin_2009}, Portugal \cite{Peixoto_2017f}, Qatar \cite{Ali_2017}, Saudi Arabia \cite{Sultan_2024f}, South Africa \cite{Sobnach_2011f}, Sweden \cite{Naji_2012f}, Switzerland \cite{Wildhaber_2005}, and Taiwan \cite{Chang_2017f}. \paragraph*{Gender} 43 cases (60\%) were male \cite{Akay_2015f, Al-Faham_2020k, Alao_2006i, Ali_2017, Ali_2022g, Apikotoa_2022f, Atayan_2016, Benoist_2019e, Berry_2021e, Bhumi_2024f, CamachoDorado_2018, Csaky_1998e, Emamhadi_2018, Farhadi_2024h, Fry_2010, Gardner_2017h, Guinan_2019f, Jehangir_2019h, Jin_2023, Kobiela_2015, Kumar_2001, Kumar_2019f, Liu_2005, Losanoff_1996, Losanoff_1997e, Mesfin_2022a, Misra_2013, Qureshi_2016, Riva_2018j, Sobnach_2011f, Tammana_2012j, Tanrikulu_2015e, Tay_2004, Thapa_2019f, Trgo_2012f, Wadhwa_2015e, Yasin_2009, teWildt_2010}, 28 cases (39\%) were female \cite{AlShaaibi_2021b, Ali_2020f, Ataya_2013, Beecroft_1998, Bhasin_2014, Bhattacharjee_2008, Cauchi_2002, Chang_2017f, Cox_2007, DelgadoSalazar_2020c, DivsalarP._2023a, Goldman_1998f, Hardy_2023g, Kar_2015, Kariholu_2008, Kerestes_2019, Li_2013, Naji_2012f, Ohno_2005, Peixoto_2017f, Sakellaridis_2008f, Sultan_2024f, Tupesis_2004f, Wildhaber_2005, Wnęk_2015f, Yildiz_2016e}, 1 case (1\%) had no gender recorded \cite{fjbuilsRepeatedBehaviorDeliberate2024}. \paragraph*{Age Group} 25 cases (35\%) were between 26 and 40 years of age \cite{Alao_2006i, Ali_2022g, Apikotoa_2022f, Ataya_2013, Benoist_2019e, Bhasin_2014, Chang_2017f, Cox_2007, DelgadoSalazar_2020c, Farhadi_2024h, Fry_2010, Gardner_2017h, Guinan_2019f, Jin_2023, Kumar_2019f, Losanoff_1996, Misra_2013, Qureshi_2016, Riva_2018j, Sakellaridis_2008f, Tammana_2012j, Trgo_2012f, Wnęk_2015f, Yildiz_2016e, fjbuilsRepeatedBehaviorDeliberate2024}, 18 cases (25\%) were between 18 and 25 years of age \cite{Akay_2015f, Ali_2017, Atayan_2016, Bhattacharjee_2008, Csaky_1998e, Kar_2015, Kariholu_2008, Kobiela_2015, Losanoff_1996, Losanoff_1997e, Mesfin_2022a, Peixoto_2017f, Sobnach_2011f, Tupesis_2004f, Yasin_2009}, 13 cases (18\%) were under 18 years of age \cite{AlShaaibi_2021b, Ali_2020f, Cauchi_2002, DivsalarP._2023a, Goldman_1998f, Liu_2005, Naji_2012f, Ohno_2005, Tanrikulu_2015e, Tay_2004, Wildhaber_2005}, 11 cases (15\%) were between 41 and 60 years of age \cite{Al-Faham_2020k, Bhumi_2024f, CamachoDorado_2018, Emamhadi_2018, Hardy_2023g, Jehangir_2019h, Kumar_2001, Sultan_2024f, Thapa_2019f, Wadhwa_2015e, teWildt_2010}, 3 cases (4\%) were over 60 years of age \cite{Beecroft_1998, Kerestes_2019, Li_2013}, 2 cases (3\%) had no age documented \cite{Berry_2021e}. \paragraph*{Population} 36 cases (50\%) had a psychiatric history \cite{AlShaaibi_2021b, Alao_2006i, Ali_2020f, Apikotoa_2022f, Ataya_2013, Atayan_2016, Beecroft_1998, CamachoDorado_2018, Chang_2017f, DelgadoSalazar_2020c, DivsalarP._2023a, Farhadi_2024h, Fry_2010, Guinan_2019f, Hardy_2023g, Jehangir_2019h, Jin_2023, Kar_2015, Kerestes_2019, Kobiela_2015, Kumar_2001, Kumar_2019f, Liu_2005, Mesfin_2022a, Misra_2013, Ohno_2005, Peixoto_2017f, Sakellaridis_2008f, Sultan_2024f, Tammana_2012j, Tanrikulu_2015e, Yildiz_2016e, fjbuilsRepeatedBehaviorDeliberate2024, teWildt_2010}, 19 cases (26\%) had ingested previously \cite{Alao_2006i, Apikotoa_2022f, Berry_2021e, Bhattacharjee_2008, Csaky_1998e, DivsalarP._2023a, Emamhadi_2018, Guinan_2019f, Jehangir_2019h, Jin_2023, Liu_2005, Sakellaridis_2008f, Tanrikulu_2015e, Thapa_2019f, Yildiz_2016e, fjbuilsRepeatedBehaviorDeliberate2024, teWildt_2010}, 12 cases (17\%) were detained persons \cite{Alao_2006i, Ali_2022g, Apikotoa_2022f, Losanoff_1996, Losanoff_1997e, Qureshi_2016, Tammana_2012j, Trgo_2012f}, 7 cases (10\%) were severely disabled \cite{Atayan_2016, Kerestes_2019, Liu_2005, Ohno_2005, Peixoto_2017f, Yildiz_2016e, teWildt_2010}, 4 cases (6\%) were psychiatric inpatients \cite{DivsalarP._2023a, fjbuilsRepeatedBehaviorDeliberate2024, teWildt_2010}, 3 cases (4\%) were under the influence of alcohol \cite{Benoist_2019e, Csaky_1998e, Thapa_2019f}, 2 cases (3\%) were displaced people \cite{Akay_2015f, Gardner_2017h}. \paragraph*{Motivation} 34 cases (47\%) had a psychiatric motivation \cite{Al-Faham_2020k, Alao_2006i, Ali_2020f, Apikotoa_2022f, Ataya_2013, Atayan_2016, Bhasin_2014, Bhattacharjee_2008, DelgadoSalazar_2020c, DivsalarP._2023a, Emamhadi_2018, Farhadi_2024h, Guinan_2019f, Hardy_2023g, Jehangir_2019h, Jin_2023, Kar_2015, Kariholu_2008, Kerestes_2019, Kobiela_2015, Kumar_2001, Kumar_2019f, Li_2013, Liu_2005, Misra_2013, Ohno_2005, Sakellaridis_2008f, Sultan_2024f, Tammana_2012j, Tanrikulu_2015e, Yasin_2009, teWildt_2010}, 21 cases (29\%) were motivated by self-harm intention \cite{Al-Faham_2020k, AlShaaibi_2021b, Alao_2006i, Ali_2017, CamachoDorado_2018, Chang_2017f, Cox_2007, Csaky_1998e, Fry_2010, Li_2013, Losanoff_1996, Losanoff_1997e, Mesfin_2022a, Sakellaridis_2008f, Tammana_2012j, Tanrikulu_2015e, fjbuilsRepeatedBehaviorDeliberate2024}, 17 cases (24\%) had a psychosocial motivation \cite{Akay_2015f, Benoist_2019e, Bhattacharjee_2008, Cauchi_2002, Goldman_1998f, Hardy_2023g, Kobiela_2015, Li_2013, Naji_2012f, Qureshi_2016, Riva_2018j, Sobnach_2011f, Tay_2004, Thapa_2019f, Tupesis_2004f, Wildhaber_2005, Wnęk_2015f}, 9 cases (12\%) were motivated by protest \cite{Bhumi_2024f, Gardner_2017h, Losanoff_1996, Losanoff_1997e, Tupesis_2004f}, 9 cases (12\%) had another documented motivation \cite{Ali_2020f, Ali_2022g, Emamhadi_2018, Guinan_2019f, Peixoto_2017f, Sakellaridis_2008f, Trgo_2012f, Wadhwa_2015e, Yildiz_2016e}. \paragraph*{Object Characteristics} 51 cases (71\%) ingested a large diameter object (\textgreater{}2.5cm) \cite{Akay_2015f, Al-Faham_2020k, AlShaaibi_2021b, Alao_2006i, Ali_2017, Ali_2022g, Apikotoa_2022f, Atayan_2016, Berry_2021e, Bhasin_2014, CamachoDorado_2018, Cauchi_2002, Chang_2017f, Cox_2007, Csaky_1998e, DivsalarP._2023a, Emamhadi_2018, Gardner_2017h, Guinan_2019f, Jehangir_2019h, Jin_2023, Kariholu_2008, Kerestes_2019, Kobiela_2015, Kumar_2001, Kumar_2019f, Losanoff_1996, Losanoff_1997e, Mesfin_2022a, Misra_2013, Naji_2012f, Ohno_2005, Peixoto_2017f, Qureshi_2016, Riva_2018j, Sakellaridis_2008f, Sultan_2024f, Tanrikulu_2015e, Thapa_2019f, Trgo_2012f, Wnęk_2015f, Yildiz_2016e, fjbuilsRepeatedBehaviorDeliberate2024, teWildt_2010}, 44 cases (61\%) ingested multiple objects \cite{Ali_2020f, Apikotoa_2022f, Ataya_2013, Atayan_2016, Beecroft_1998, Bhattacharjee_2008, Bhumi_2024f, CamachoDorado_2018, Cauchi_2002, Emamhadi_2018, Farhadi_2024h, Fry_2010, Goldman_1998f, Guinan_2019f, Hardy_2023g, Jehangir_2019h, Jin_2023, Kar_2015, Kariholu_2008, Kobiela_2015, Kumar_2001, Kumar_2019f, Li_2013, Liu_2005, Losanoff_1996, Mesfin_2022a, Misra_2013, Naji_2012f, Ohno_2005, Sobnach_2011f, Sultan_2024f, Tammana_2012j, Tanrikulu_2015e, Tay_2004, Thapa_2019f, Wadhwa_2015e, Wildhaber_2005, Yasin_2009, fjbuilsRepeatedBehaviorDeliberate2024, teWildt_2010}, 34 cases (47\%) ingested a sharp object \cite{AlShaaibi_2021b, Alao_2006i, Apikotoa_2022f, Ataya_2013, Benoist_2019e, Bhasin_2014, Bhattacharjee_2008, CamachoDorado_2018, Csaky_1998e, DelgadoSalazar_2020c, DivsalarP._2023a, Emamhadi_2018, Farhadi_2024h, Fry_2010, Guinan_2019f, Hardy_2023g, Jehangir_2019h, Jin_2023, Kariholu_2008, Kobiela_2015, Kumar_2019f, Losanoff_1996, Losanoff_1997e, Mesfin_2022a, Misra_2013, Sobnach_2011f, Yasin_2009, teWildt_2010}, 32 cases (44\%) ingested a long object (\textgreater{}5cm) \cite{Al-Faham_2020k, AlShaaibi_2021b, Ali_2017, Ali_2022g, Atayan_2016, Bhasin_2014, CamachoDorado_2018, Chang_2017f, Cox_2007, Csaky_1998e, DivsalarP._2023a, Emamhadi_2018, Fry_2010, Gardner_2017h, Jin_2023, Kariholu_2008, Kerestes_2019, Kobiela_2015, Kumar_2019f, Mesfin_2022a, Misra_2013, Ohno_2005, Qureshi_2016, Sakellaridis_2008f, Sultan_2024f, Thapa_2019f, Trgo_2012f, Yasin_2009, Yildiz_2016e, teWildt_2010}, 9 cases (12\%) ingested a magnet \cite{Ali_2020f, Bhumi_2024f, Cauchi_2002, Liu_2005, Naji_2012f, Ohno_2005, Tanrikulu_2015e, Tay_2004, Wildhaber_2005}, 2 cases (3\%) ingested a button battery \cite{Berry_2021e, Bhumi_2024f}. \paragraph*{Outcomes} 48 cases (67\%) experienced a complication \cite{Ali_2017, Ali_2020f, Apikotoa_2022f, Atayan_2016, Beecroft_1998, Benoist_2019e, Berry_2021e, Bhasin_2014, Bhumi_2024f, CamachoDorado_2018, Cauchi_2002, Cox_2007, Csaky_1998e, DelgadoSalazar_2020c, DivsalarP._2023a, Emamhadi_2018, Farhadi_2024h, Fry_2010, Gardner_2017h, Goldman_1998f, Jin_2023, Kariholu_2008, Kerestes_2019, Kobiela_2015, Kumar_2001, Kumar_2019f, Liu_2005, Losanoff_1996, Mesfin_2022a, Misra_2013, Naji_2012f, Ohno_2005, Sakellaridis_2008f, Sobnach_2011f, Sultan_2024f, Tanrikulu_2015e, Tay_2004, Thapa_2019f, Trgo_2012f, Tupesis_2004f, Wildhaber_2005, Wnęk_2015f, Yasin_2009, Yildiz_2016e}, 44 cases (61\%) underwent surgery \cite{Al-Faham_2020k, AlShaaibi_2021b, Alao_2006i, Ali_2017, Ali_2020f, Atayan_2016, Beecroft_1998, Bhasin_2014, CamachoDorado_2018, Cauchi_2002, Chang_2017f, Cox_2007, Csaky_1998e, DelgadoSalazar_2020c, DivsalarP._2023a, Farhadi_2024h, Fry_2010, Gardner_2017h, Jin_2023, Kariholu_2008, Kerestes_2019, Kobiela_2015, Kumar_2019f, Liu_2005, Losanoff_1996, Losanoff_1997e, Mesfin_2022a, Misra_2013, Naji_2012f, Sobnach_2011f, Tanrikulu_2015e, Tay_2004, Thapa_2019f, Tupesis_2004f, Wildhaber_2005, Wnęk_2015f, Yasin_2009, Yildiz_2016e, fjbuilsRepeatedBehaviorDeliberate2024}, 31 cases (43\%) underwent endoscopy \cite{Akay_2015f, Ali_2022g, Apikotoa_2022f, Atayan_2016, Benoist_2019e, Berry_2021e, Bhasin_2014, Bhumi_2024f, CamachoDorado_2018, Chang_2017f, DelgadoSalazar_2020c, Gardner_2017h, Guinan_2019f, Hardy_2023g, Jehangir_2019h, Kariholu_2008, Li_2013, Liu_2005, Ohno_2005, Peixoto_2017f, Qureshi_2016, Riva_2018j, Sakellaridis_2008f, Sultan_2024f, Tammana_2012j, Tanrikulu_2015e, Trgo_2012f, Wadhwa_2015e, Wnęk_2015f, teWildt_2010}, 7 cases (10\%) were managed conservatively \cite{Ataya_2013, Bhattacharjee_2008, DivsalarP._2023a, Emamhadi_2018, Goldman_1998f, Kar_2015, Kumar_2001}, 2 cases (3\%) died \cite{Emamhadi_2018, Kumar_2001}. All 90 were male gender. 90 cases (100\%) were detained at the time of ingestion \cite{Elghali_2016, Karp_1991b, Lee_2007}, 88 cases (98\%) were intentional ingestions \cite{Elghali_2016, Karp_1991b, Lee_2007}, 30 cases (33\%) had a psychiatric history documented \cite{Elghali_2016, Karp_1991b, Lee_2007}, 2 cases (2\%) had a history of prior ingestion \cite{Elghali_2016}. No cases were reported for were psychiatric inpatients, were displaced people, were under the influence of alcohol at the time of ingestion, and had a severe disability history.
\paragraph*{Motivation}  70 cases (78\%) reported protest motivation \cite{Elghali_2016, Karp_1991b, Lee_2007}, 12 cases (13\%) reported psychiatric motivation \cite{Karp_1991b}, 6 cases (7\%) reported self-harm motivation \cite{Elghali_2016, Karp_1991b}. No cases were reported for psychosocial motivation and other motivation.
\paragraph*{Object Characteristics}  68 cases (76\%) involved sharp object ingestion \cite{Elghali_2016, Karp_1991b, Lee_2007}, 32 cases (36\%) involved long (\textgreater 5cm) object ingestion \cite{Lee_2007}, 25 cases (28\%) involved ingestion of multiple objects \cite{Elghali_2016, Lee_2007}. No cases were reported for button battery ingestion, magnet ingestion, and involved large diameter (\textgreater 2.5cm) object ingestion.
\paragraph*{Outcomes}  47 cases (52\%) underwent endoscopic intervention \cite{Elghali_2016, Lee_2007}, 29 cases (32\%) were managed conservatively \cite{Elghali_2016, Karp_1991b}, 15 cases (17\%) underwent surgical intervention \cite{Elghali_2016, Karp_1991b, Lee_2007}, 6 cases (7\%) reported complications \cite{Lee_2007}, 1 case (1\%) died \cite{Elghali_2016}.
\paragraph*{Geographical Location}Cases were recorded in 33 countries: 13 cases from USA \cite{Alao_2006i, Ataya_2013, Bhumi_2024f, Fry_2010, Guinan_2019f, Hardy_2023g, Jehangir_2019h, Kerestes_2019, Kumar_2001, Liu_2005, Tammana_2012j, Tay_2004, Tupesis_2004f}; 7 cases from India \cite{Bhasin_2014, Bhattacharjee_2008, Kar_2015, Kariholu_2008, Kumar_2019f, Misra_2013, Wadhwa_2015e} and UK \cite{Beecroft_1998, Berry_2021e, Cauchi_2002, Cox_2007, Gardner_2017h, Qureshi_2016}; 6 cases from Bulgaria \cite{Losanoff_1996, Losanoff_1997e}; 5 cases from Iran \cite{DivsalarP._2023a, Emamhadi_2018, Farhadi_2024h}; 4 cases from Turkey \cite{Akay_2015f, Atayan_2016, Tanrikulu_2015e, Yildiz_2016e}; 2 cases from China \cite{Jin_2023, Li_2013}, Poland \cite{Kobiela_2015, Wnęk_2015f}, and Spain \cite{CamachoDorado_2018, fjbuilsRepeatedBehaviorDeliberate2024}; 1 case from Australia \cite{Apikotoa_2022f}, Bahrain \cite{Ali_2020f}, Croatia \cite{Trgo_2012f}, Ecuador \cite{DelgadoSalazar_2020c}, Egypt \cite{Ali_2022g}, Ethiopia \cite{Mesfin_2022a}, Germany \cite{teWildt_2010}, Greece \cite{Sakellaridis_2008f}, Hungary \cite{Csaky_1998e}, Iraq \cite{Al-Faham_2020k}, Israel \cite{Goldman_1998f}, Italy \cite{Riva_2018j}, Japan \cite{Ohno_2005}, Nepal \cite{Thapa_2019f}, Netherlands \cite{Benoist_2019e}, Oman \cite{AlShaaibi_2021b}, Pakistan \cite{Yasin_2009}, Portugal \cite{Peixoto_2017f}, Qatar \cite{Ali_2017}, Saudi Arabia \cite{Sultan_2024f}, South Africa \cite{Sobnach_2011f}, Sweden \cite{Naji_2012f}, Switzerland \cite{Wildhaber_2005}, and Taiwan \cite{Chang_2017f}. \paragraph*{Gender} 43 cases (60\%) were male \cite{Akay_2015f, Al-Faham_2020k, Alao_2006i, Ali_2017, Ali_2022g, Apikotoa_2022f, Atayan_2016, Benoist_2019e, Berry_2021e, Bhumi_2024f, CamachoDorado_2018, Csaky_1998e, Emamhadi_2018, Farhadi_2024h, Fry_2010, Gardner_2017h, Guinan_2019f, Jehangir_2019h, Jin_2023, Kobiela_2015, Kumar_2001, Kumar_2019f, Liu_2005, Losanoff_1996, Losanoff_1997e, Mesfin_2022a, Misra_2013, Qureshi_2016, Riva_2018j, Sobnach_2011f, Tammana_2012j, Tanrikulu_2015e, Tay_2004, Thapa_2019f, Trgo_2012f, Wadhwa_2015e, Yasin_2009, teWildt_2010}, 28 cases (39\%) were female \cite{AlShaaibi_2021b, Ali_2020f, Ataya_2013, Beecroft_1998, Bhasin_2014, Bhattacharjee_2008, Cauchi_2002, Chang_2017f, Cox_2007, DelgadoSalazar_2020c, DivsalarP._2023a, Goldman_1998f, Hardy_2023g, Kar_2015, Kariholu_2008, Kerestes_2019, Li_2013, Naji_2012f, Ohno_2005, Peixoto_2017f, Sakellaridis_2008f, Sultan_2024f, Tupesis_2004f, Wildhaber_2005, Wnęk_2015f, Yildiz_2016e}, 1 case (1\%) had no gender recorded \cite{fjbuilsRepeatedBehaviorDeliberate2024}. \paragraph*{Age Group} 25 cases (35\%) were between 26 and 40 years of age \cite{Alao_2006i, Ali_2022g, Apikotoa_2022f, Ataya_2013, Benoist_2019e, Bhasin_2014, Chang_2017f, Cox_2007, DelgadoSalazar_2020c, Farhadi_2024h, Fry_2010, Gardner_2017h, Guinan_2019f, Jin_2023, Kumar_2019f, Losanoff_1996, Misra_2013, Qureshi_2016, Riva_2018j, Sakellaridis_2008f, Tammana_2012j, Trgo_2012f, Wnęk_2015f, Yildiz_2016e, fjbuilsRepeatedBehaviorDeliberate2024}, 18 cases (25\%) were between 18 and 25 years of age \cite{Akay_2015f, Ali_2017, Atayan_2016, Bhattacharjee_2008, Csaky_1998e, Kar_2015, Kariholu_2008, Kobiela_2015, Losanoff_1996, Losanoff_1997e, Mesfin_2022a, Peixoto_2017f, Sobnach_2011f, Tupesis_2004f, Yasin_2009}, 13 cases (18\%) were under 18 years of age \cite{AlShaaibi_2021b, Ali_2020f, Cauchi_2002, DivsalarP._2023a, Goldman_1998f, Liu_2005, Naji_2012f, Ohno_2005, Tanrikulu_2015e, Tay_2004, Wildhaber_2005}, 11 cases (15\%) were between 41 and 60 years of age \cite{Al-Faham_2020k, Bhumi_2024f, CamachoDorado_2018, Emamhadi_2018, Hardy_2023g, Jehangir_2019h, Kumar_2001, Sultan_2024f, Thapa_2019f, Wadhwa_2015e, teWildt_2010}, 3 cases (4\%) were over 60 years of age \cite{Beecroft_1998, Kerestes_2019, Li_2013}, 2 cases (3\%) had no age documented \cite{Berry_2021e}. \paragraph*{Population} 36 cases (50\%) had a psychiatric history \cite{AlShaaibi_2021b, Alao_2006i, Ali_2020f, Apikotoa_2022f, Ataya_2013, Atayan_2016, Beecroft_1998, CamachoDorado_2018, Chang_2017f, DelgadoSalazar_2020c, DivsalarP._2023a, Farhadi_2024h, Fry_2010, Guinan_2019f, Hardy_2023g, Jehangir_2019h, Jin_2023, Kar_2015, Kerestes_2019, Kobiela_2015, Kumar_2001, Kumar_2019f, Liu_2005, Mesfin_2022a, Misra_2013, Ohno_2005, Peixoto_2017f, Sakellaridis_2008f, Sultan_2024f, Tammana_2012j, Tanrikulu_2015e, Yildiz_2016e, fjbuilsRepeatedBehaviorDeliberate2024, teWildt_2010}, 19 cases (26\%) had ingested previously \cite{Alao_2006i, Apikotoa_2022f, Berry_2021e, Bhattacharjee_2008, Csaky_1998e, DivsalarP._2023a, Emamhadi_2018, Guinan_2019f, Jehangir_2019h, Jin_2023, Liu_2005, Sakellaridis_2008f, Tanrikulu_2015e, Thapa_2019f, Yildiz_2016e, fjbuilsRepeatedBehaviorDeliberate2024, teWildt_2010}, 12 cases (17\%) were detained persons \cite{Alao_2006i, Ali_2022g, Apikotoa_2022f, Losanoff_1996, Losanoff_1997e, Qureshi_2016, Tammana_2012j, Trgo_2012f}, 7 cases (10\%) were severely disabled \cite{Atayan_2016, Kerestes_2019, Liu_2005, Ohno_2005, Peixoto_2017f, Yildiz_2016e, teWildt_2010}, 4 cases (6\%) were psychiatric inpatients \cite{DivsalarP._2023a, fjbuilsRepeatedBehaviorDeliberate2024, teWildt_2010}, 3 cases (4\%) were under the influence of alcohol \cite{Benoist_2019e, Csaky_1998e, Thapa_2019f}, 2 cases (3\%) were displaced people \cite{Akay_2015f, Gardner_2017h}. \paragraph*{Motivation} 34 cases (47\%) had a psychiatric motivation \cite{Al-Faham_2020k, Alao_2006i, Ali_2020f, Apikotoa_2022f, Ataya_2013, Atayan_2016, Bhasin_2014, Bhattacharjee_2008, DelgadoSalazar_2020c, DivsalarP._2023a, Emamhadi_2018, Farhadi_2024h, Guinan_2019f, Hardy_2023g, Jehangir_2019h, Jin_2023, Kar_2015, Kariholu_2008, Kerestes_2019, Kobiela_2015, Kumar_2001, Kumar_2019f, Li_2013, Liu_2005, Misra_2013, Ohno_2005, Sakellaridis_2008f, Sultan_2024f, Tammana_2012j, Tanrikulu_2015e, Yasin_2009, teWildt_2010}, 21 cases (29\%) were motivated by self-harm intention \cite{Al-Faham_2020k, AlShaaibi_2021b, Alao_2006i, Ali_2017, CamachoDorado_2018, Chang_2017f, Cox_2007, Csaky_1998e, Fry_2010, Li_2013, Losanoff_1996, Losanoff_1997e, Mesfin_2022a, Sakellaridis_2008f, Tammana_2012j, Tanrikulu_2015e, fjbuilsRepeatedBehaviorDeliberate2024}, 17 cases (24\%) had a psychosocial motivation \cite{Akay_2015f, Benoist_2019e, Bhattacharjee_2008, Cauchi_2002, Goldman_1998f, Hardy_2023g, Kobiela_2015, Li_2013, Naji_2012f, Qureshi_2016, Riva_2018j, Sobnach_2011f, Tay_2004, Thapa_2019f, Tupesis_2004f, Wildhaber_2005, Wnęk_2015f}, 9 cases (12\%) were motivated by protest \cite{Bhumi_2024f, Gardner_2017h, Losanoff_1996, Losanoff_1997e, Tupesis_2004f}, 9 cases (12\%) had another documented motivation \cite{Ali_2020f, Ali_2022g, Emamhadi_2018, Guinan_2019f, Peixoto_2017f, Sakellaridis_2008f, Trgo_2012f, Wadhwa_2015e, Yildiz_2016e}. \paragraph*{Object Characteristics} 51 cases (71\%) ingested a large diameter object (\textgreater{}2.5cm) \cite{Akay_2015f, Al-Faham_2020k, AlShaaibi_2021b, Alao_2006i, Ali_2017, Ali_2022g, Apikotoa_2022f, Atayan_2016, Berry_2021e, Bhasin_2014, CamachoDorado_2018, Cauchi_2002, Chang_2017f, Cox_2007, Csaky_1998e, DivsalarP._2023a, Emamhadi_2018, Gardner_2017h, Guinan_2019f, Jehangir_2019h, Jin_2023, Kariholu_2008, Kerestes_2019, Kobiela_2015, Kumar_2001, Kumar_2019f, Losanoff_1996, Losanoff_1997e, Mesfin_2022a, Misra_2013, Naji_2012f, Ohno_2005, Peixoto_2017f, Qureshi_2016, Riva_2018j, Sakellaridis_2008f, Sultan_2024f, Tanrikulu_2015e, Thapa_2019f, Trgo_2012f, Wnęk_2015f, Yildiz_2016e, fjbuilsRepeatedBehaviorDeliberate2024, teWildt_2010}, 44 cases (61\%) ingested multiple objects \cite{Ali_2020f, Apikotoa_2022f, Ataya_2013, Atayan_2016, Beecroft_1998, Bhattacharjee_2008, Bhumi_2024f, CamachoDorado_2018, Cauchi_2002, Emamhadi_2018, Farhadi_2024h, Fry_2010, Goldman_1998f, Guinan_2019f, Hardy_2023g, Jehangir_2019h, Jin_2023, Kar_2015, Kariholu_2008, Kobiela_2015, Kumar_2001, Kumar_2019f, Li_2013, Liu_2005, Losanoff_1996, Mesfin_2022a, Misra_2013, Naji_2012f, Ohno_2005, Sobnach_2011f, Sultan_2024f, Tammana_2012j, Tanrikulu_2015e, Tay_2004, Thapa_2019f, Wadhwa_2015e, Wildhaber_2005, Yasin_2009, fjbuilsRepeatedBehaviorDeliberate2024, teWildt_2010}, 34 cases (47\%) ingested a sharp object \cite{AlShaaibi_2021b, Alao_2006i, Apikotoa_2022f, Ataya_2013, Benoist_2019e, Bhasin_2014, Bhattacharjee_2008, CamachoDorado_2018, Csaky_1998e, DelgadoSalazar_2020c, DivsalarP._2023a, Emamhadi_2018, Farhadi_2024h, Fry_2010, Guinan_2019f, Hardy_2023g, Jehangir_2019h, Jin_2023, Kariholu_2008, Kobiela_2015, Kumar_2019f, Losanoff_1996, Losanoff_1997e, Mesfin_2022a, Misra_2013, Sobnach_2011f, Yasin_2009, teWildt_2010}, 32 cases (44\%) ingested a long object (\textgreater{}5cm) \cite{Al-Faham_2020k, AlShaaibi_2021b, Ali_2017, Ali_2022g, Atayan_2016, Bhasin_2014, CamachoDorado_2018, Chang_2017f, Cox_2007, Csaky_1998e, DivsalarP._2023a, Emamhadi_2018, Fry_2010, Gardner_2017h, Jin_2023, Kariholu_2008, Kerestes_2019, Kobiela_2015, Kumar_2019f, Mesfin_2022a, Misra_2013, Ohno_2005, Qureshi_2016, Sakellaridis_2008f, Sultan_2024f, Thapa_2019f, Trgo_2012f, Yasin_2009, Yildiz_2016e, teWildt_2010}, 9 cases (12\%) ingested a magnet \cite{Ali_2020f, Bhumi_2024f, Cauchi_2002, Liu_2005, Naji_2012f, Ohno_2005, Tanrikulu_2015e, Tay_2004, Wildhaber_2005}, 2 cases (3\%) ingested a button battery \cite{Berry_2021e, Bhumi_2024f}. \paragraph*{Outcomes} 48 cases (67\%) experienced a complication \cite{Ali_2017, Ali_2020f, Apikotoa_2022f, Atayan_2016, Beecroft_1998, Benoist_2019e, Berry_2021e, Bhasin_2014, Bhumi_2024f, CamachoDorado_2018, Cauchi_2002, Cox_2007, Csaky_1998e, DelgadoSalazar_2020c, DivsalarP._2023a, Emamhadi_2018, Farhadi_2024h, Fry_2010, Gardner_2017h, Goldman_1998f, Jin_2023, Kariholu_2008, Kerestes_2019, Kobiela_2015, Kumar_2001, Kumar_2019f, Liu_2005, Losanoff_1996, Mesfin_2022a, Misra_2013, Naji_2012f, Ohno_2005, Sakellaridis_2008f, Sobnach_2011f, Sultan_2024f, Tanrikulu_2015e, Tay_2004, Thapa_2019f, Trgo_2012f, Tupesis_2004f, Wildhaber_2005, Wnęk_2015f, Yasin_2009, Yildiz_2016e}, 44 cases (61\%) underwent surgery \cite{Al-Faham_2020k, AlShaaibi_2021b, Alao_2006i, Ali_2017, Ali_2020f, Atayan_2016, Beecroft_1998, Bhasin_2014, CamachoDorado_2018, Cauchi_2002, Chang_2017f, Cox_2007, Csaky_1998e, DelgadoSalazar_2020c, DivsalarP._2023a, Farhadi_2024h, Fry_2010, Gardner_2017h, Jin_2023, Kariholu_2008, Kerestes_2019, Kobiela_2015, Kumar_2019f, Liu_2005, Losanoff_1996, Losanoff_1997e, Mesfin_2022a, Misra_2013, Naji_2012f, Sobnach_2011f, Tanrikulu_2015e, Tay_2004, Thapa_2019f, Tupesis_2004f, Wildhaber_2005, Wnęk_2015f, Yasin_2009, Yildiz_2016e, fjbuilsRepeatedBehaviorDeliberate2024}, 31 cases (43\%) underwent endoscopy \cite{Akay_2015f, Ali_2022g, Apikotoa_2022f, Atayan_2016, Benoist_2019e, Berry_2021e, Bhasin_2014, Bhumi_2024f, CamachoDorado_2018, Chang_2017f, DelgadoSalazar_2020c, Gardner_2017h, Guinan_2019f, Hardy_2023g, Jehangir_2019h, Kariholu_2008, Li_2013, Liu_2005, Ohno_2005, Peixoto_2017f, Qureshi_2016, Riva_2018j, Sakellaridis_2008f, Sultan_2024f, Tammana_2012j, Tanrikulu_2015e, Trgo_2012f, Wadhwa_2015e, Wnęk_2015f, teWildt_2010}, 7 cases (10\%) were managed conservatively \cite{Ataya_2013, Bhattacharjee_2008, DivsalarP._2023a, Emamhadi_2018, Goldman_1998f, Kar_2015, Kumar_2001}, 2 cases (3\%) died \cite{Emamhadi_2018, Kumar_2001}. All 90 were male gender. 90 cases (100\%) were detained at the time of ingestion \cite{Elghali_2016, Karp_1991b, Lee_2007}, 88 cases (98\%) were intentional ingestions \cite{Elghali_2016, Karp_1991b, Lee_2007}, 30 cases (33\%) had a psychiatric history documented \cite{Elghali_2016, Karp_1991b, Lee_2007}, 2 cases (2\%) had a history of prior ingestion \cite{Elghali_2016}. No cases were reported for were psychiatric inpatients, were displaced people, were under the influence of alcohol at the time of ingestion, and had a severe disability history.
\paragraph*{Motivation}  70 cases (78\%) reported protest motivation \cite{Elghali_2016, Karp_1991b, Lee_2007}, 12 cases (13\%) reported psychiatric motivation \cite{Karp_1991b}, 6 cases (7\%) reported self-harm motivation \cite{Elghali_2016, Karp_1991b}. No cases were reported for psychosocial motivation and other motivation.
\paragraph*{Object Characteristics}  68 cases (76\%) involved sharp object ingestion \cite{Elghali_2016, Karp_1991b, Lee_2007}, 32 cases (36\%) involved long (\textgreater 5cm) object ingestion \cite{Lee_2007}, 25 cases (28\%) involved ingestion of multiple objects \cite{Elghali_2016, Lee_2007}. No cases were reported for button battery ingestion, magnet ingestion, and involved large diameter (\textgreater 2.5cm) object ingestion.
\paragraph*{Outcomes}  47 cases (52\%) underwent endoscopic intervention \cite{Elghali_2016, Lee_2007}, 29 cases (32\%) were managed conservatively \cite{Elghali_2016, Karp_1991b}, 15 cases (17\%) underwent surgical intervention \cite{Elghali_2016, Karp_1991b, Lee_2007}, 6 cases (7\%) reported complications \cite{Lee_2007}, 1 case (1\%) died \cite{Elghali_2016}.
\paragraph*{Geographical Location}Cases were recorded in 33 countries: 13 cases from USA \cite{Alao_2006i, Ataya_2013, Bhumi_2024f, Fry_2010, Guinan_2019f, Hardy_2023g, Jehangir_2019h, Kerestes_2019, Kumar_2001, Liu_2005, Tammana_2012j, Tay_2004, Tupesis_2004f}; 7 cases from India \cite{Bhasin_2014, Bhattacharjee_2008, Kar_2015, Kariholu_2008, Kumar_2019f, Misra_2013, Wadhwa_2015e} and UK \cite{Beecroft_1998, Berry_2021e, Cauchi_2002, Cox_2007, Gardner_2017h, Qureshi_2016}; 6 cases from Bulgaria \cite{Losanoff_1996, Losanoff_1997e}; 5 cases from Iran \cite{DivsalarP._2023a, Emamhadi_2018, Farhadi_2024h}; 4 cases from Turkey \cite{Akay_2015f, Atayan_2016, Tanrikulu_2015e, Yildiz_2016e}; 2 cases from China \cite{Jin_2023, Li_2013}, Poland \cite{Kobiela_2015, Wnęk_2015f}, and Spain \cite{CamachoDorado_2018, fjbuilsRepeatedBehaviorDeliberate2024}; 1 case from Australia \cite{Apikotoa_2022f}, Bahrain \cite{Ali_2020f}, Croatia \cite{Trgo_2012f}, Ecuador \cite{DelgadoSalazar_2020c}, Egypt \cite{Ali_2022g}, Ethiopia \cite{Mesfin_2022a}, Germany \cite{teWildt_2010}, Greece \cite{Sakellaridis_2008f}, Hungary \cite{Csaky_1998e}, Iraq \cite{Al-Faham_2020k}, Israel \cite{Goldman_1998f}, Italy \cite{Riva_2018j}, Japan \cite{Ohno_2005}, Nepal \cite{Thapa_2019f}, Netherlands \cite{Benoist_2019e}, Oman \cite{AlShaaibi_2021b}, Pakistan \cite{Yasin_2009}, Portugal \cite{Peixoto_2017f}, Qatar \cite{Ali_2017}, Saudi Arabia \cite{Sultan_2024f}, South Africa \cite{Sobnach_2011f}, Sweden \cite{Naji_2012f}, Switzerland \cite{Wildhaber_2005}, and Taiwan \cite{Chang_2017f}. \paragraph*{Gender} 43 cases (60\%) were male \cite{Akay_2015f, Al-Faham_2020k, Alao_2006i, Ali_2017, Ali_2022g, Apikotoa_2022f, Atayan_2016, Benoist_2019e, Berry_2021e, Bhumi_2024f, CamachoDorado_2018, Csaky_1998e, Emamhadi_2018, Farhadi_2024h, Fry_2010, Gardner_2017h, Guinan_2019f, Jehangir_2019h, Jin_2023, Kobiela_2015, Kumar_2001, Kumar_2019f, Liu_2005, Losanoff_1996, Losanoff_1997e, Mesfin_2022a, Misra_2013, Qureshi_2016, Riva_2018j, Sobnach_2011f, Tammana_2012j, Tanrikulu_2015e, Tay_2004, Thapa_2019f, Trgo_2012f, Wadhwa_2015e, Yasin_2009, teWildt_2010}, 28 cases (39\%) were female \cite{AlShaaibi_2021b, Ali_2020f, Ataya_2013, Beecroft_1998, Bhasin_2014, Bhattacharjee_2008, Cauchi_2002, Chang_2017f, Cox_2007, DelgadoSalazar_2020c, DivsalarP._2023a, Goldman_1998f, Hardy_2023g, Kar_2015, Kariholu_2008, Kerestes_2019, Li_2013, Naji_2012f, Ohno_2005, Peixoto_2017f, Sakellaridis_2008f, Sultan_2024f, Tupesis_2004f, Wildhaber_2005, Wnęk_2015f, Yildiz_2016e}, 1 case (1\%) had no gender recorded \cite{fjbuilsRepeatedBehaviorDeliberate2024}. \paragraph*{Age Group} 25 cases (35\%) were between 26 and 40 years of age \cite{Alao_2006i, Ali_2022g, Apikotoa_2022f, Ataya_2013, Benoist_2019e, Bhasin_2014, Chang_2017f, Cox_2007, DelgadoSalazar_2020c, Farhadi_2024h, Fry_2010, Gardner_2017h, Guinan_2019f, Jin_2023, Kumar_2019f, Losanoff_1996, Misra_2013, Qureshi_2016, Riva_2018j, Sakellaridis_2008f, Tammana_2012j, Trgo_2012f, Wnęk_2015f, Yildiz_2016e, fjbuilsRepeatedBehaviorDeliberate2024}, 18 cases (25\%) were between 18 and 25 years of age \cite{Akay_2015f, Ali_2017, Atayan_2016, Bhattacharjee_2008, Csaky_1998e, Kar_2015, Kariholu_2008, Kobiela_2015, Losanoff_1996, Losanoff_1997e, Mesfin_2022a, Peixoto_2017f, Sobnach_2011f, Tupesis_2004f, Yasin_2009}, 13 cases (18\%) were under 18 years of age \cite{AlShaaibi_2021b, Ali_2020f, Cauchi_2002, DivsalarP._2023a, Goldman_1998f, Liu_2005, Naji_2012f, Ohno_2005, Tanrikulu_2015e, Tay_2004, Wildhaber_2005}, 11 cases (15\%) were between 41 and 60 years of age \cite{Al-Faham_2020k, Bhumi_2024f, CamachoDorado_2018, Emamhadi_2018, Hardy_2023g, Jehangir_2019h, Kumar_2001, Sultan_2024f, Thapa_2019f, Wadhwa_2015e, teWildt_2010}, 3 cases (4\%) were over 60 years of age \cite{Beecroft_1998, Kerestes_2019, Li_2013}, 2 cases (3\%) had no age documented \cite{Berry_2021e}. \paragraph*{Population} 36 cases (50\%) had a psychiatric history \cite{AlShaaibi_2021b, Alao_2006i, Ali_2020f, Apikotoa_2022f, Ataya_2013, Atayan_2016, Beecroft_1998, CamachoDorado_2018, Chang_2017f, DelgadoSalazar_2020c, DivsalarP._2023a, Farhadi_2024h, Fry_2010, Guinan_2019f, Hardy_2023g, Jehangir_2019h, Jin_2023, Kar_2015, Kerestes_2019, Kobiela_2015, Kumar_2001, Kumar_2019f, Liu_2005, Mesfin_2022a, Misra_2013, Ohno_2005, Peixoto_2017f, Sakellaridis_2008f, Sultan_2024f, Tammana_2012j, Tanrikulu_2015e, Yildiz_2016e, fjbuilsRepeatedBehaviorDeliberate2024, teWildt_2010}, 19 cases (26\%) had ingested previously \cite{Alao_2006i, Apikotoa_2022f, Berry_2021e, Bhattacharjee_2008, Csaky_1998e, DivsalarP._2023a, Emamhadi_2018, Guinan_2019f, Jehangir_2019h, Jin_2023, Liu_2005, Sakellaridis_2008f, Tanrikulu_2015e, Thapa_2019f, Yildiz_2016e, fjbuilsRepeatedBehaviorDeliberate2024, teWildt_2010}, 12 cases (17\%) were detained persons \cite{Alao_2006i, Ali_2022g, Apikotoa_2022f, Losanoff_1996, Losanoff_1997e, Qureshi_2016, Tammana_2012j, Trgo_2012f}, 7 cases (10\%) were severely disabled \cite{Atayan_2016, Kerestes_2019, Liu_2005, Ohno_2005, Peixoto_2017f, Yildiz_2016e, teWildt_2010}, 4 cases (6\%) were psychiatric inpatients \cite{DivsalarP._2023a, fjbuilsRepeatedBehaviorDeliberate2024, teWildt_2010}, 3 cases (4\%) were under the influence of alcohol \cite{Benoist_2019e, Csaky_1998e, Thapa_2019f}, 2 cases (3\%) were displaced people \cite{Akay_2015f, Gardner_2017h}. \paragraph*{Motivation} 34 cases (47\%) had a psychiatric motivation \cite{Al-Faham_2020k, Alao_2006i, Ali_2020f, Apikotoa_2022f, Ataya_2013, Atayan_2016, Bhasin_2014, Bhattacharjee_2008, DelgadoSalazar_2020c, DivsalarP._2023a, Emamhadi_2018, Farhadi_2024h, Guinan_2019f, Hardy_2023g, Jehangir_2019h, Jin_2023, Kar_2015, Kariholu_2008, Kerestes_2019, Kobiela_2015, Kumar_2001, Kumar_2019f, Li_2013, Liu_2005, Misra_2013, Ohno_2005, Sakellaridis_2008f, Sultan_2024f, Tammana_2012j, Tanrikulu_2015e, Yasin_2009, teWildt_2010}, 21 cases (29\%) were motivated by self-harm intention \cite{Al-Faham_2020k, AlShaaibi_2021b, Alao_2006i, Ali_2017, CamachoDorado_2018, Chang_2017f, Cox_2007, Csaky_1998e, Fry_2010, Li_2013, Losanoff_1996, Losanoff_1997e, Mesfin_2022a, Sakellaridis_2008f, Tammana_2012j, Tanrikulu_2015e, fjbuilsRepeatedBehaviorDeliberate2024}, 17 cases (24\%) had a psychosocial motivation \cite{Akay_2015f, Benoist_2019e, Bhattacharjee_2008, Cauchi_2002, Goldman_1998f, Hardy_2023g, Kobiela_2015, Li_2013, Naji_2012f, Qureshi_2016, Riva_2018j, Sobnach_2011f, Tay_2004, Thapa_2019f, Tupesis_2004f, Wildhaber_2005, Wnęk_2015f}, 9 cases (12\%) were motivated by protest \cite{Bhumi_2024f, Gardner_2017h, Losanoff_1996, Losanoff_1997e, Tupesis_2004f}, 9 cases (12\%) had another documented motivation \cite{Ali_2020f, Ali_2022g, Emamhadi_2018, Guinan_2019f, Peixoto_2017f, Sakellaridis_2008f, Trgo_2012f, Wadhwa_2015e, Yildiz_2016e}. \paragraph*{Object Characteristics} 51 cases (71\%) ingested a large diameter object (\textgreater{}2.5cm) \cite{Akay_2015f, Al-Faham_2020k, AlShaaibi_2021b, Alao_2006i, Ali_2017, Ali_2022g, Apikotoa_2022f, Atayan_2016, Berry_2021e, Bhasin_2014, CamachoDorado_2018, Cauchi_2002, Chang_2017f, Cox_2007, Csaky_1998e, DivsalarP._2023a, Emamhadi_2018, Gardner_2017h, Guinan_2019f, Jehangir_2019h, Jin_2023, Kariholu_2008, Kerestes_2019, Kobiela_2015, Kumar_2001, Kumar_2019f, Losanoff_1996, Losanoff_1997e, Mesfin_2022a, Misra_2013, Naji_2012f, Ohno_2005, Peixoto_2017f, Qureshi_2016, Riva_2018j, Sakellaridis_2008f, Sultan_2024f, Tanrikulu_2015e, Thapa_2019f, Trgo_2012f, Wnęk_2015f, Yildiz_2016e, fjbuilsRepeatedBehaviorDeliberate2024, teWildt_2010}, 44 cases (61\%) ingested multiple objects \cite{Ali_2020f, Apikotoa_2022f, Ataya_2013, Atayan_2016, Beecroft_1998, Bhattacharjee_2008, Bhumi_2024f, CamachoDorado_2018, Cauchi_2002, Emamhadi_2018, Farhadi_2024h, Fry_2010, Goldman_1998f, Guinan_2019f, Hardy_2023g, Jehangir_2019h, Jin_2023, Kar_2015, Kariholu_2008, Kobiela_2015, Kumar_2001, Kumar_2019f, Li_2013, Liu_2005, Losanoff_1996, Mesfin_2022a, Misra_2013, Naji_2012f, Ohno_2005, Sobnach_2011f, Sultan_2024f, Tammana_2012j, Tanrikulu_2015e, Tay_2004, Thapa_2019f, Wadhwa_2015e, Wildhaber_2005, Yasin_2009, fjbuilsRepeatedBehaviorDeliberate2024, teWildt_2010}, 34 cases (47\%) ingested a sharp object \cite{AlShaaibi_2021b, Alao_2006i, Apikotoa_2022f, Ataya_2013, Benoist_2019e, Bhasin_2014, Bhattacharjee_2008, CamachoDorado_2018, Csaky_1998e, DelgadoSalazar_2020c, DivsalarP._2023a, Emamhadi_2018, Farhadi_2024h, Fry_2010, Guinan_2019f, Hardy_2023g, Jehangir_2019h, Jin_2023, Kariholu_2008, Kobiela_2015, Kumar_2019f, Losanoff_1996, Losanoff_1997e, Mesfin_2022a, Misra_2013, Sobnach_2011f, Yasin_2009, teWildt_2010}, 32 cases (44\%) ingested a long object (\textgreater{}5cm) \cite{Al-Faham_2020k, AlShaaibi_2021b, Ali_2017, Ali_2022g, Atayan_2016, Bhasin_2014, CamachoDorado_2018, Chang_2017f, Cox_2007, Csaky_1998e, DivsalarP._2023a, Emamhadi_2018, Fry_2010, Gardner_2017h, Jin_2023, Kariholu_2008, Kerestes_2019, Kobiela_2015, Kumar_2019f, Mesfin_2022a, Misra_2013, Ohno_2005, Qureshi_2016, Sakellaridis_2008f, Sultan_2024f, Thapa_2019f, Trgo_2012f, Yasin_2009, Yildiz_2016e, teWildt_2010}, 9 cases (12\%) ingested a magnet \cite{Ali_2020f, Bhumi_2024f, Cauchi_2002, Liu_2005, Naji_2012f, Ohno_2005, Tanrikulu_2015e, Tay_2004, Wildhaber_2005}, 2 cases (3\%) ingested a button battery \cite{Berry_2021e, Bhumi_2024f}. \paragraph*{Outcomes} 48 cases (67\%) experienced a complication \cite{Ali_2017, Ali_2020f, Apikotoa_2022f, Atayan_2016, Beecroft_1998, Benoist_2019e, Berry_2021e, Bhasin_2014, Bhumi_2024f, CamachoDorado_2018, Cauchi_2002, Cox_2007, Csaky_1998e, DelgadoSalazar_2020c, DivsalarP._2023a, Emamhadi_2018, Farhadi_2024h, Fry_2010, Gardner_2017h, Goldman_1998f, Jin_2023, Kariholu_2008, Kerestes_2019, Kobiela_2015, Kumar_2001, Kumar_2019f, Liu_2005, Losanoff_1996, Mesfin_2022a, Misra_2013, Naji_2012f, Ohno_2005, Sakellaridis_2008f, Sobnach_2011f, Sultan_2024f, Tanrikulu_2015e, Tay_2004, Thapa_2019f, Trgo_2012f, Tupesis_2004f, Wildhaber_2005, Wnęk_2015f, Yasin_2009, Yildiz_2016e}, 44 cases (61\%) underwent surgery \cite{Al-Faham_2020k, AlShaaibi_2021b, Alao_2006i, Ali_2017, Ali_2020f, Atayan_2016, Beecroft_1998, Bhasin_2014, CamachoDorado_2018, Cauchi_2002, Chang_2017f, Cox_2007, Csaky_1998e, DelgadoSalazar_2020c, DivsalarP._2023a, Farhadi_2024h, Fry_2010, Gardner_2017h, Jin_2023, Kariholu_2008, Kerestes_2019, Kobiela_2015, Kumar_2019f, Liu_2005, Losanoff_1996, Losanoff_1997e, Mesfin_2022a, Misra_2013, Naji_2012f, Sobnach_2011f, Tanrikulu_2015e, Tay_2004, Thapa_2019f, Tupesis_2004f, Wildhaber_2005, Wnęk_2015f, Yasin_2009, Yildiz_2016e, fjbuilsRepeatedBehaviorDeliberate2024}, 31 cases (43\%) underwent endoscopy \cite{Akay_2015f, Ali_2022g, Apikotoa_2022f, Atayan_2016, Benoist_2019e, Berry_2021e, Bhasin_2014, Bhumi_2024f, CamachoDorado_2018, Chang_2017f, DelgadoSalazar_2020c, Gardner_2017h, Guinan_2019f, Hardy_2023g, Jehangir_2019h, Kariholu_2008, Li_2013, Liu_2005, Ohno_2005, Peixoto_2017f, Qureshi_2016, Riva_2018j, Sakellaridis_2008f, Sultan_2024f, Tammana_2012j, Tanrikulu_2015e, Trgo_2012f, Wadhwa_2015e, Wnęk_2015f, teWildt_2010}, 7 cases (10\%) were managed conservatively \cite{Ataya_2013, Bhattacharjee_2008, DivsalarP._2023a, Emamhadi_2018, Goldman_1998f, Kar_2015, Kumar_2001}, 2 cases (3\%) died \cite{Emamhadi_2018, Kumar_2001}. All 90 were male gender. 90 cases (100\%) were detained at the time of ingestion \cite{Elghali_2016, Karp_1991b, Lee_2007}, 88 cases (98\%) were intentional ingestions \cite{Elghali_2016, Karp_1991b, Lee_2007}, 30 cases (33\%) had a psychiatric history documented \cite{Elghali_2016, Karp_1991b, Lee_2007}, 2 cases (2\%) had a history of prior ingestion \cite{Elghali_2016}. No cases were reported for were psychiatric inpatients, were displaced people, were under the influence of alcohol at the time of ingestion, and had a severe disability history.
\paragraph*{Motivation}  70 cases (78\%) reported protest motivation \cite{Elghali_2016, Karp_1991b, Lee_2007}, 12 cases (13\%) reported psychiatric motivation \cite{Karp_1991b}, 6 cases (7\%) reported self-harm motivation \cite{Elghali_2016, Karp_1991b}. No cases were reported for psychosocial motivation and other motivation.
\paragraph*{Object Characteristics}  68 cases (76\%) involved sharp object ingestion \cite{Elghali_2016, Karp_1991b, Lee_2007}, 32 cases (36\%) involved long (\textgreater 5cm) object ingestion \cite{Lee_2007}, 25 cases (28\%) involved ingestion of multiple objects \cite{Elghali_2016, Lee_2007}. No cases were reported for button battery ingestion, magnet ingestion, and involved large diameter (\textgreater 2.5cm) object ingestion.
\paragraph*{Outcomes}  47 cases (52\%) underwent endoscopic intervention \cite{Elghali_2016, Lee_2007}, 29 cases (32\%) were managed conservatively \cite{Elghali_2016, Karp_1991b}, 15 cases (17\%) underwent surgical intervention \cite{Elghali_2016, Karp_1991b, Lee_2007}, 6 cases (7\%) reported complications \cite{Lee_2007}, 1 case (1\%) died \cite{Elghali_2016}.
\paragraph*{Geographical Location}Cases were recorded in 33 countries: 13 cases from USA \cite{Alao_2006i, Ataya_2013, Bhumi_2024f, Fry_2010, Guinan_2019f, Hardy_2023g, Jehangir_2019h, Kerestes_2019, Kumar_2001, Liu_2005, Tammana_2012j, Tay_2004, Tupesis_2004f}; 7 cases from India \cite{Bhasin_2014, Bhattacharjee_2008, Kar_2015, Kariholu_2008, Kumar_2019f, Misra_2013, Wadhwa_2015e} and UK \cite{Beecroft_1998, Berry_2021e, Cauchi_2002, Cox_2007, Gardner_2017h, Qureshi_2016}; 6 cases from Bulgaria \cite{Losanoff_1996, Losanoff_1997e}; 5 cases from Iran \cite{DivsalarP._2023a, Emamhadi_2018, Farhadi_2024h}; 4 cases from Turkey \cite{Akay_2015f, Atayan_2016, Tanrikulu_2015e, Yildiz_2016e}; 2 cases from China \cite{Jin_2023, Li_2013}, Poland \cite{Kobiela_2015, Wnęk_2015f}, and Spain \cite{CamachoDorado_2018, fjbuilsRepeatedBehaviorDeliberate2024}; 1 case from Australia \cite{Apikotoa_2022f}, Bahrain \cite{Ali_2020f}, Croatia \cite{Trgo_2012f}, Ecuador \cite{DelgadoSalazar_2020c}, Egypt \cite{Ali_2022g}, Ethiopia \cite{Mesfin_2022a}, Germany \cite{teWildt_2010}, Greece \cite{Sakellaridis_2008f}, Hungary \cite{Csaky_1998e}, Iraq \cite{Al-Faham_2020k}, Israel \cite{Goldman_1998f}, Italy \cite{Riva_2018j}, Japan \cite{Ohno_2005}, Nepal \cite{Thapa_2019f}, Netherlands \cite{Benoist_2019e}, Oman \cite{AlShaaibi_2021b}, Pakistan \cite{Yasin_2009}, Portugal \cite{Peixoto_2017f}, Qatar \cite{Ali_2017}, Saudi Arabia \cite{Sultan_2024f}, South Africa \cite{Sobnach_2011f}, Sweden \cite{Naji_2012f}, Switzerland \cite{Wildhaber_2005}, and Taiwan \cite{Chang_2017f}. \paragraph*{Gender} 43 cases (60\%) were male \cite{Akay_2015f, Al-Faham_2020k, Alao_2006i, Ali_2017, Ali_2022g, Apikotoa_2022f, Atayan_2016, Benoist_2019e, Berry_2021e, Bhumi_2024f, CamachoDorado_2018, Csaky_1998e, Emamhadi_2018, Farhadi_2024h, Fry_2010, Gardner_2017h, Guinan_2019f, Jehangir_2019h, Jin_2023, Kobiela_2015, Kumar_2001, Kumar_2019f, Liu_2005, Losanoff_1996, Losanoff_1997e, Mesfin_2022a, Misra_2013, Qureshi_2016, Riva_2018j, Sobnach_2011f, Tammana_2012j, Tanrikulu_2015e, Tay_2004, Thapa_2019f, Trgo_2012f, Wadhwa_2015e, Yasin_2009, teWildt_2010}, 28 cases (39\%) were female \cite{AlShaaibi_2021b, Ali_2020f, Ataya_2013, Beecroft_1998, Bhasin_2014, Bhattacharjee_2008, Cauchi_2002, Chang_2017f, Cox_2007, DelgadoSalazar_2020c, DivsalarP._2023a, Goldman_1998f, Hardy_2023g, Kar_2015, Kariholu_2008, Kerestes_2019, Li_2013, Naji_2012f, Ohno_2005, Peixoto_2017f, Sakellaridis_2008f, Sultan_2024f, Tupesis_2004f, Wildhaber_2005, Wnęk_2015f, Yildiz_2016e}, 1 case (1\%) had no gender recorded \cite{fjbuilsRepeatedBehaviorDeliberate2024}. \paragraph*{Age Group} 25 cases (35\%) were between 26 and 40 years of age \cite{Alao_2006i, Ali_2022g, Apikotoa_2022f, Ataya_2013, Benoist_2019e, Bhasin_2014, Chang_2017f, Cox_2007, DelgadoSalazar_2020c, Farhadi_2024h, Fry_2010, Gardner_2017h, Guinan_2019f, Jin_2023, Kumar_2019f, Losanoff_1996, Misra_2013, Qureshi_2016, Riva_2018j, Sakellaridis_2008f, Tammana_2012j, Trgo_2012f, Wnęk_2015f, Yildiz_2016e, fjbuilsRepeatedBehaviorDeliberate2024}, 18 cases (25\%) were between 18 and 25 years of age \cite{Akay_2015f, Ali_2017, Atayan_2016, Bhattacharjee_2008, Csaky_1998e, Kar_2015, Kariholu_2008, Kobiela_2015, Losanoff_1996, Losanoff_1997e, Mesfin_2022a, Peixoto_2017f, Sobnach_2011f, Tupesis_2004f, Yasin_2009}, 13 cases (18\%) were under 18 years of age \cite{AlShaaibi_2021b, Ali_2020f, Cauchi_2002, DivsalarP._2023a, Goldman_1998f, Liu_2005, Naji_2012f, Ohno_2005, Tanrikulu_2015e, Tay_2004, Wildhaber_2005}, 11 cases (15\%) were between 41 and 60 years of age \cite{Al-Faham_2020k, Bhumi_2024f, CamachoDorado_2018, Emamhadi_2018, Hardy_2023g, Jehangir_2019h, Kumar_2001, Sultan_2024f, Thapa_2019f, Wadhwa_2015e, teWildt_2010}, 3 cases (4\%) were over 60 years of age \cite{Beecroft_1998, Kerestes_2019, Li_2013}, 2 cases (3\%) had no age documented \cite{Berry_2021e}. \paragraph*{Population} 36 cases (50\%) had a psychiatric history \cite{AlShaaibi_2021b, Alao_2006i, Ali_2020f, Apikotoa_2022f, Ataya_2013, Atayan_2016, Beecroft_1998, CamachoDorado_2018, Chang_2017f, DelgadoSalazar_2020c, DivsalarP._2023a, Farhadi_2024h, Fry_2010, Guinan_2019f, Hardy_2023g, Jehangir_2019h, Jin_2023, Kar_2015, Kerestes_2019, Kobiela_2015, Kumar_2001, Kumar_2019f, Liu_2005, Mesfin_2022a, Misra_2013, Ohno_2005, Peixoto_2017f, Sakellaridis_2008f, Sultan_2024f, Tammana_2012j, Tanrikulu_2015e, Yildiz_2016e, fjbuilsRepeatedBehaviorDeliberate2024, teWildt_2010}, 19 cases (26\%) had ingested previously \cite{Alao_2006i, Apikotoa_2022f, Berry_2021e, Bhattacharjee_2008, Csaky_1998e, DivsalarP._2023a, Emamhadi_2018, Guinan_2019f, Jehangir_2019h, Jin_2023, Liu_2005, Sakellaridis_2008f, Tanrikulu_2015e, Thapa_2019f, Yildiz_2016e, fjbuilsRepeatedBehaviorDeliberate2024, teWildt_2010}, 12 cases (17\%) were detained persons \cite{Alao_2006i, Ali_2022g, Apikotoa_2022f, Losanoff_1996, Losanoff_1997e, Qureshi_2016, Tammana_2012j, Trgo_2012f}, 7 cases (10\%) were severely disabled \cite{Atayan_2016, Kerestes_2019, Liu_2005, Ohno_2005, Peixoto_2017f, Yildiz_2016e, teWildt_2010}, 4 cases (6\%) were psychiatric inpatients \cite{DivsalarP._2023a, fjbuilsRepeatedBehaviorDeliberate2024, teWildt_2010}, 3 cases (4\%) were under the influence of alcohol \cite{Benoist_2019e, Csaky_1998e, Thapa_2019f}, 2 cases (3\%) were displaced people \cite{Akay_2015f, Gardner_2017h}. \paragraph*{Motivation} 34 cases (47\%) had a psychiatric motivation \cite{Al-Faham_2020k, Alao_2006i, Ali_2020f, Apikotoa_2022f, Ataya_2013, Atayan_2016, Bhasin_2014, Bhattacharjee_2008, DelgadoSalazar_2020c, DivsalarP._2023a, Emamhadi_2018, Farhadi_2024h, Guinan_2019f, Hardy_2023g, Jehangir_2019h, Jin_2023, Kar_2015, Kariholu_2008, Kerestes_2019, Kobiela_2015, Kumar_2001, Kumar_2019f, Li_2013, Liu_2005, Misra_2013, Ohno_2005, Sakellaridis_2008f, Sultan_2024f, Tammana_2012j, Tanrikulu_2015e, Yasin_2009, teWildt_2010}, 21 cases (29\%) were motivated by self-harm intention \cite{Al-Faham_2020k, AlShaaibi_2021b, Alao_2006i, Ali_2017, CamachoDorado_2018, Chang_2017f, Cox_2007, Csaky_1998e, Fry_2010, Li_2013, Losanoff_1996, Losanoff_1997e, Mesfin_2022a, Sakellaridis_2008f, Tammana_2012j, Tanrikulu_2015e, fjbuilsRepeatedBehaviorDeliberate2024}, 17 cases (24\%) had a psychosocial motivation \cite{Akay_2015f, Benoist_2019e, Bhattacharjee_2008, Cauchi_2002, Goldman_1998f, Hardy_2023g, Kobiela_2015, Li_2013, Naji_2012f, Qureshi_2016, Riva_2018j, Sobnach_2011f, Tay_2004, Thapa_2019f, Tupesis_2004f, Wildhaber_2005, Wnęk_2015f}, 9 cases (12\%) were motivated by protest \cite{Bhumi_2024f, Gardner_2017h, Losanoff_1996, Losanoff_1997e, Tupesis_2004f}, 9 cases (12\%) had another documented motivation \cite{Ali_2020f, Ali_2022g, Emamhadi_2018, Guinan_2019f, Peixoto_2017f, Sakellaridis_2008f, Trgo_2012f, Wadhwa_2015e, Yildiz_2016e}. \paragraph*{Object Characteristics} 51 cases (71\%) ingested a large diameter object (\textgreater{}2.5cm) \cite{Akay_2015f, Al-Faham_2020k, AlShaaibi_2021b, Alao_2006i, Ali_2017, Ali_2022g, Apikotoa_2022f, Atayan_2016, Berry_2021e, Bhasin_2014, CamachoDorado_2018, Cauchi_2002, Chang_2017f, Cox_2007, Csaky_1998e, DivsalarP._2023a, Emamhadi_2018, Gardner_2017h, Guinan_2019f, Jehangir_2019h, Jin_2023, Kariholu_2008, Kerestes_2019, Kobiela_2015, Kumar_2001, Kumar_2019f, Losanoff_1996, Losanoff_1997e, Mesfin_2022a, Misra_2013, Naji_2012f, Ohno_2005, Peixoto_2017f, Qureshi_2016, Riva_2018j, Sakellaridis_2008f, Sultan_2024f, Tanrikulu_2015e, Thapa_2019f, Trgo_2012f, Wnęk_2015f, Yildiz_2016e, fjbuilsRepeatedBehaviorDeliberate2024, teWildt_2010}, 44 cases (61\%) ingested multiple objects \cite{Ali_2020f, Apikotoa_2022f, Ataya_2013, Atayan_2016, Beecroft_1998, Bhattacharjee_2008, Bhumi_2024f, CamachoDorado_2018, Cauchi_2002, Emamhadi_2018, Farhadi_2024h, Fry_2010, Goldman_1998f, Guinan_2019f, Hardy_2023g, Jehangir_2019h, Jin_2023, Kar_2015, Kariholu_2008, Kobiela_2015, Kumar_2001, Kumar_2019f, Li_2013, Liu_2005, Losanoff_1996, Mesfin_2022a, Misra_2013, Naji_2012f, Ohno_2005, Sobnach_2011f, Sultan_2024f, Tammana_2012j, Tanrikulu_2015e, Tay_2004, Thapa_2019f, Wadhwa_2015e, Wildhaber_2005, Yasin_2009, fjbuilsRepeatedBehaviorDeliberate2024, teWildt_2010}, 34 cases (47\%) ingested a sharp object \cite{AlShaaibi_2021b, Alao_2006i, Apikotoa_2022f, Ataya_2013, Benoist_2019e, Bhasin_2014, Bhattacharjee_2008, CamachoDorado_2018, Csaky_1998e, DelgadoSalazar_2020c, DivsalarP._2023a, Emamhadi_2018, Farhadi_2024h, Fry_2010, Guinan_2019f, Hardy_2023g, Jehangir_2019h, Jin_2023, Kariholu_2008, Kobiela_2015, Kumar_2019f, Losanoff_1996, Losanoff_1997e, Mesfin_2022a, Misra_2013, Sobnach_2011f, Yasin_2009, teWildt_2010}, 32 cases (44\%) ingested a long object (\textgreater{}5cm) \cite{Al-Faham_2020k, AlShaaibi_2021b, Ali_2017, Ali_2022g, Atayan_2016, Bhasin_2014, CamachoDorado_2018, Chang_2017f, Cox_2007, Csaky_1998e, DivsalarP._2023a, Emamhadi_2018, Fry_2010, Gardner_2017h, Jin_2023, Kariholu_2008, Kerestes_2019, Kobiela_2015, Kumar_2019f, Mesfin_2022a, Misra_2013, Ohno_2005, Qureshi_2016, Sakellaridis_2008f, Sultan_2024f, Thapa_2019f, Trgo_2012f, Yasin_2009, Yildiz_2016e, teWildt_2010}, 9 cases (12\%) ingested a magnet \cite{Ali_2020f, Bhumi_2024f, Cauchi_2002, Liu_2005, Naji_2012f, Ohno_2005, Tanrikulu_2015e, Tay_2004, Wildhaber_2005}, 2 cases (3\%) ingested a button battery \cite{Berry_2021e, Bhumi_2024f}. \paragraph*{Outcomes} 48 cases (67\%) experienced a complication \cite{Ali_2017, Ali_2020f, Apikotoa_2022f, Atayan_2016, Beecroft_1998, Benoist_2019e, Berry_2021e, Bhasin_2014, Bhumi_2024f, CamachoDorado_2018, Cauchi_2002, Cox_2007, Csaky_1998e, DelgadoSalazar_2020c, DivsalarP._2023a, Emamhadi_2018, Farhadi_2024h, Fry_2010, Gardner_2017h, Goldman_1998f, Jin_2023, Kariholu_2008, Kerestes_2019, Kobiela_2015, Kumar_2001, Kumar_2019f, Liu_2005, Losanoff_1996, Mesfin_2022a, Misra_2013, Naji_2012f, Ohno_2005, Sakellaridis_2008f, Sobnach_2011f, Sultan_2024f, Tanrikulu_2015e, Tay_2004, Thapa_2019f, Trgo_2012f, Tupesis_2004f, Wildhaber_2005, Wnęk_2015f, Yasin_2009, Yildiz_2016e}, 44 cases (61\%) underwent surgery \cite{Al-Faham_2020k, AlShaaibi_2021b, Alao_2006i, Ali_2017, Ali_2020f, Atayan_2016, Beecroft_1998, Bhasin_2014, CamachoDorado_2018, Cauchi_2002, Chang_2017f, Cox_2007, Csaky_1998e, DelgadoSalazar_2020c, DivsalarP._2023a, Farhadi_2024h, Fry_2010, Gardner_2017h, Jin_2023, Kariholu_2008, Kerestes_2019, Kobiela_2015, Kumar_2019f, Liu_2005, Losanoff_1996, Losanoff_1997e, Mesfin_2022a, Misra_2013, Naji_2012f, Sobnach_2011f, Tanrikulu_2015e, Tay_2004, Thapa_2019f, Tupesis_2004f, Wildhaber_2005, Wnęk_2015f, Yasin_2009, Yildiz_2016e, fjbuilsRepeatedBehaviorDeliberate2024}, 31 cases (43\%) underwent endoscopy \cite{Akay_2015f, Ali_2022g, Apikotoa_2022f, Atayan_2016, Benoist_2019e, Berry_2021e, Bhasin_2014, Bhumi_2024f, CamachoDorado_2018, Chang_2017f, DelgadoSalazar_2020c, Gardner_2017h, Guinan_2019f, Hardy_2023g, Jehangir_2019h, Kariholu_2008, Li_2013, Liu_2005, Ohno_2005, Peixoto_2017f, Qureshi_2016, Riva_2018j, Sakellaridis_2008f, Sultan_2024f, Tammana_2012j, Tanrikulu_2015e, Trgo_2012f, Wadhwa_2015e, Wnęk_2015f, teWildt_2010}, 7 cases (10\%) were managed conservatively \cite{Ataya_2013, Bhattacharjee_2008, DivsalarP._2023a, Emamhadi_2018, Goldman_1998f, Kar_2015, Kumar_2001}, 2 cases (3\%) died \cite{Emamhadi_2018, Kumar_2001}. All 90 were male gender. 90 cases (100\%) were detained at the time of ingestion \cite{Elghali_2016, Karp_1991b, Lee_2007}, 88 cases (98\%) were intentional ingestions \cite{Elghali_2016, Karp_1991b, Lee_2007}, 30 cases (33\%) had a psychiatric history documented \cite{Elghali_2016, Karp_1991b, Lee_2007}, 2 cases (2\%) had a history of prior ingestion \cite{Elghali_2016}. No cases were reported for were psychiatric inpatients, were displaced people, were under the influence of alcohol at the time of ingestion, and had a severe disability history.
\paragraph*{Motivation}  70 cases (78\%) reported protest motivation \cite{Elghali_2016, Karp_1991b, Lee_2007}, 12 cases (13\%) reported psychiatric motivation \cite{Karp_1991b}, 6 cases (7\%) reported self-harm motivation \cite{Elghali_2016, Karp_1991b}. No cases were reported for psychosocial motivation and other motivation.
\paragraph*{Object Characteristics}  68 cases (76\%) involved sharp object ingestion \cite{Elghali_2016, Karp_1991b, Lee_2007}, 32 cases (36\%) involved long (\textgreater 5cm) object ingestion \cite{Lee_2007}, 25 cases (28\%) involved ingestion of multiple objects \cite{Elghali_2016, Lee_2007}. No cases were reported for button battery ingestion, magnet ingestion, and involved large diameter (\textgreater 2.5cm) object ingestion.
\paragraph*{Outcomes}  47 cases (52\%) underwent endoscopic intervention \cite{Elghali_2016, Lee_2007}, 29 cases (32\%) were managed conservatively \cite{Elghali_2016, Karp_1991b}, 15 cases (17\%) underwent surgical intervention \cite{Elghali_2016, Karp_1991b, Lee_2007}, 6 cases (7\%) reported complications \cite{Lee_2007}, 1 case (1\%) died \cite{Elghali_2016}.
\paragraph*{Geographical Location}Cases were recorded in 33 countries: 13 cases from USA \cite{Alao_2006i, Ataya_2013, Bhumi_2024f, Fry_2010, Guinan_2019f, Hardy_2023g, Jehangir_2019h, Kerestes_2019, Kumar_2001, Liu_2005, Tammana_2012j, Tay_2004, Tupesis_2004f}; 7 cases from India \cite{Bhasin_2014, Bhattacharjee_2008, Kar_2015, Kariholu_2008, Kumar_2019f, Misra_2013, Wadhwa_2015e} and UK \cite{Beecroft_1998, Berry_2021e, Cauchi_2002, Cox_2007, Gardner_2017h, Qureshi_2016}; 6 cases from Bulgaria \cite{Losanoff_1996, Losanoff_1997e}; 5 cases from Iran \cite{DivsalarP._2023a, Emamhadi_2018, Farhadi_2024h}; 4 cases from Turkey \cite{Akay_2015f, Atayan_2016, Tanrikulu_2015e, Yildiz_2016e}; 2 cases from China \cite{Jin_2023, Li_2013}, Poland \cite{Kobiela_2015, Wnęk_2015f}, and Spain \cite{CamachoDorado_2018, fjbuilsRepeatedBehaviorDeliberate2024}; 1 case from Australia \cite{Apikotoa_2022f}, Bahrain \cite{Ali_2020f}, Croatia \cite{Trgo_2012f}, Ecuador \cite{DelgadoSalazar_2020c}, Egypt \cite{Ali_2022g}, Ethiopia \cite{Mesfin_2022a}, Germany \cite{teWildt_2010}, Greece \cite{Sakellaridis_2008f}, Hungary \cite{Csaky_1998e}, Iraq \cite{Al-Faham_2020k}, Israel \cite{Goldman_1998f}, Italy \cite{Riva_2018j}, Japan \cite{Ohno_2005}, Nepal \cite{Thapa_2019f}, Netherlands \cite{Benoist_2019e}, Oman \cite{AlShaaibi_2021b}, Pakistan \cite{Yasin_2009}, Portugal \cite{Peixoto_2017f}, Qatar \cite{Ali_2017}, Saudi Arabia \cite{Sultan_2024f}, South Africa \cite{Sobnach_2011f}, Sweden \cite{Naji_2012f}, Switzerland \cite{Wildhaber_2005}, and Taiwan \cite{Chang_2017f}. \paragraph*{Gender} 43 cases (60\%) were male \cite{Akay_2015f, Al-Faham_2020k, Alao_2006i, Ali_2017, Ali_2022g, Apikotoa_2022f, Atayan_2016, Benoist_2019e, Berry_2021e, Bhumi_2024f, CamachoDorado_2018, Csaky_1998e, Emamhadi_2018, Farhadi_2024h, Fry_2010, Gardner_2017h, Guinan_2019f, Jehangir_2019h, Jin_2023, Kobiela_2015, Kumar_2001, Kumar_2019f, Liu_2005, Losanoff_1996, Losanoff_1997e, Mesfin_2022a, Misra_2013, Qureshi_2016, Riva_2018j, Sobnach_2011f, Tammana_2012j, Tanrikulu_2015e, Tay_2004, Thapa_2019f, Trgo_2012f, Wadhwa_2015e, Yasin_2009, teWildt_2010}, 28 cases (39\%) were female \cite{AlShaaibi_2021b, Ali_2020f, Ataya_2013, Beecroft_1998, Bhasin_2014, Bhattacharjee_2008, Cauchi_2002, Chang_2017f, Cox_2007, DelgadoSalazar_2020c, DivsalarP._2023a, Goldman_1998f, Hardy_2023g, Kar_2015, Kariholu_2008, Kerestes_2019, Li_2013, Naji_2012f, Ohno_2005, Peixoto_2017f, Sakellaridis_2008f, Sultan_2024f, Tupesis_2004f, Wildhaber_2005, Wnęk_2015f, Yildiz_2016e}, 1 case (1\%) had no gender recorded \cite{fjbuilsRepeatedBehaviorDeliberate2024}. \paragraph*{Age Group} 25 cases (35\%) were between 26 and 40 years of age \cite{Alao_2006i, Ali_2022g, Apikotoa_2022f, Ataya_2013, Benoist_2019e, Bhasin_2014, Chang_2017f, Cox_2007, DelgadoSalazar_2020c, Farhadi_2024h, Fry_2010, Gardner_2017h, Guinan_2019f, Jin_2023, Kumar_2019f, Losanoff_1996, Misra_2013, Qureshi_2016, Riva_2018j, Sakellaridis_2008f, Tammana_2012j, Trgo_2012f, Wnęk_2015f, Yildiz_2016e, fjbuilsRepeatedBehaviorDeliberate2024}, 18 cases (25\%) were between 18 and 25 years of age \cite{Akay_2015f, Ali_2017, Atayan_2016, Bhattacharjee_2008, Csaky_1998e, Kar_2015, Kariholu_2008, Kobiela_2015, Losanoff_1996, Losanoff_1997e, Mesfin_2022a, Peixoto_2017f, Sobnach_2011f, Tupesis_2004f, Yasin_2009}, 13 cases (18\%) were under 18 years of age \cite{AlShaaibi_2021b, Ali_2020f, Cauchi_2002, DivsalarP._2023a, Goldman_1998f, Liu_2005, Naji_2012f, Ohno_2005, Tanrikulu_2015e, Tay_2004, Wildhaber_2005}, 11 cases (15\%) were between 41 and 60 years of age \cite{Al-Faham_2020k, Bhumi_2024f, CamachoDorado_2018, Emamhadi_2018, Hardy_2023g, Jehangir_2019h, Kumar_2001, Sultan_2024f, Thapa_2019f, Wadhwa_2015e, teWildt_2010}, 3 cases (4\%) were over 60 years of age \cite{Beecroft_1998, Kerestes_2019, Li_2013}, 2 cases (3\%) had no age documented \cite{Berry_2021e}. \paragraph*{Population} 36 cases (50\%) had a psychiatric history \cite{AlShaaibi_2021b, Alao_2006i, Ali_2020f, Apikotoa_2022f, Ataya_2013, Atayan_2016, Beecroft_1998, CamachoDorado_2018, Chang_2017f, DelgadoSalazar_2020c, DivsalarP._2023a, Farhadi_2024h, Fry_2010, Guinan_2019f, Hardy_2023g, Jehangir_2019h, Jin_2023, Kar_2015, Kerestes_2019, Kobiela_2015, Kumar_2001, Kumar_2019f, Liu_2005, Mesfin_2022a, Misra_2013, Ohno_2005, Peixoto_2017f, Sakellaridis_2008f, Sultan_2024f, Tammana_2012j, Tanrikulu_2015e, Yildiz_2016e, fjbuilsRepeatedBehaviorDeliberate2024, teWildt_2010}, 19 cases (26\%) had ingested previously \cite{Alao_2006i, Apikotoa_2022f, Berry_2021e, Bhattacharjee_2008, Csaky_1998e, DivsalarP._2023a, Emamhadi_2018, Guinan_2019f, Jehangir_2019h, Jin_2023, Liu_2005, Sakellaridis_2008f, Tanrikulu_2015e, Thapa_2019f, Yildiz_2016e, fjbuilsRepeatedBehaviorDeliberate2024, teWildt_2010}, 12 cases (17\%) were detained persons \cite{Alao_2006i, Ali_2022g, Apikotoa_2022f, Losanoff_1996, Losanoff_1997e, Qureshi_2016, Tammana_2012j, Trgo_2012f}, 7 cases (10\%) were severely disabled \cite{Atayan_2016, Kerestes_2019, Liu_2005, Ohno_2005, Peixoto_2017f, Yildiz_2016e, teWildt_2010}, 4 cases (6\%) were psychiatric inpatients \cite{DivsalarP._2023a, fjbuilsRepeatedBehaviorDeliberate2024, teWildt_2010}, 3 cases (4\%) were under the influence of alcohol \cite{Benoist_2019e, Csaky_1998e, Thapa_2019f}, 2 cases (3\%) were displaced people \cite{Akay_2015f, Gardner_2017h}. \paragraph*{Motivation} 34 cases (47\%) had a psychiatric motivation \cite{Al-Faham_2020k, Alao_2006i, Ali_2020f, Apikotoa_2022f, Ataya_2013, Atayan_2016, Bhasin_2014, Bhattacharjee_2008, DelgadoSalazar_2020c, DivsalarP._2023a, Emamhadi_2018, Farhadi_2024h, Guinan_2019f, Hardy_2023g, Jehangir_2019h, Jin_2023, Kar_2015, Kariholu_2008, Kerestes_2019, Kobiela_2015, Kumar_2001, Kumar_2019f, Li_2013, Liu_2005, Misra_2013, Ohno_2005, Sakellaridis_2008f, Sultan_2024f, Tammana_2012j, Tanrikulu_2015e, Yasin_2009, teWildt_2010}, 21 cases (29\%) were motivated by self-harm intention \cite{Al-Faham_2020k, AlShaaibi_2021b, Alao_2006i, Ali_2017, CamachoDorado_2018, Chang_2017f, Cox_2007, Csaky_1998e, Fry_2010, Li_2013, Losanoff_1996, Losanoff_1997e, Mesfin_2022a, Sakellaridis_2008f, Tammana_2012j, Tanrikulu_2015e, fjbuilsRepeatedBehaviorDeliberate2024}, 17 cases (24\%) had a psychosocial motivation \cite{Akay_2015f, Benoist_2019e, Bhattacharjee_2008, Cauchi_2002, Goldman_1998f, Hardy_2023g, Kobiela_2015, Li_2013, Naji_2012f, Qureshi_2016, Riva_2018j, Sobnach_2011f, Tay_2004, Thapa_2019f, Tupesis_2004f, Wildhaber_2005, Wnęk_2015f}, 9 cases (12\%) were motivated by protest \cite{Bhumi_2024f, Gardner_2017h, Losanoff_1996, Losanoff_1997e, Tupesis_2004f}, 9 cases (12\%) had another documented motivation \cite{Ali_2020f, Ali_2022g, Emamhadi_2018, Guinan_2019f, Peixoto_2017f, Sakellaridis_2008f, Trgo_2012f, Wadhwa_2015e, Yildiz_2016e}. \paragraph*{Object Characteristics} 51 cases (71\%) ingested a large diameter object (\textgreater{}2.5cm) \cite{Akay_2015f, Al-Faham_2020k, AlShaaibi_2021b, Alao_2006i, Ali_2017, Ali_2022g, Apikotoa_2022f, Atayan_2016, Berry_2021e, Bhasin_2014, CamachoDorado_2018, Cauchi_2002, Chang_2017f, Cox_2007, Csaky_1998e, DivsalarP._2023a, Emamhadi_2018, Gardner_2017h, Guinan_2019f, Jehangir_2019h, Jin_2023, Kariholu_2008, Kerestes_2019, Kobiela_2015, Kumar_2001, Kumar_2019f, Losanoff_1996, Losanoff_1997e, Mesfin_2022a, Misra_2013, Naji_2012f, Ohno_2005, Peixoto_2017f, Qureshi_2016, Riva_2018j, Sakellaridis_2008f, Sultan_2024f, Tanrikulu_2015e, Thapa_2019f, Trgo_2012f, Wnęk_2015f, Yildiz_2016e, fjbuilsRepeatedBehaviorDeliberate2024, teWildt_2010}, 44 cases (61\%) ingested multiple objects \cite{Ali_2020f, Apikotoa_2022f, Ataya_2013, Atayan_2016, Beecroft_1998, Bhattacharjee_2008, Bhumi_2024f, CamachoDorado_2018, Cauchi_2002, Emamhadi_2018, Farhadi_2024h, Fry_2010, Goldman_1998f, Guinan_2019f, Hardy_2023g, Jehangir_2019h, Jin_2023, Kar_2015, Kariholu_2008, Kobiela_2015, Kumar_2001, Kumar_2019f, Li_2013, Liu_2005, Losanoff_1996, Mesfin_2022a, Misra_2013, Naji_2012f, Ohno_2005, Sobnach_2011f, Sultan_2024f, Tammana_2012j, Tanrikulu_2015e, Tay_2004, Thapa_2019f, Wadhwa_2015e, Wildhaber_2005, Yasin_2009, fjbuilsRepeatedBehaviorDeliberate2024, teWildt_2010}, 34 cases (47\%) ingested a sharp object \cite{AlShaaibi_2021b, Alao_2006i, Apikotoa_2022f, Ataya_2013, Benoist_2019e, Bhasin_2014, Bhattacharjee_2008, CamachoDorado_2018, Csaky_1998e, DelgadoSalazar_2020c, DivsalarP._2023a, Emamhadi_2018, Farhadi_2024h, Fry_2010, Guinan_2019f, Hardy_2023g, Jehangir_2019h, Jin_2023, Kariholu_2008, Kobiela_2015, Kumar_2019f, Losanoff_1996, Losanoff_1997e, Mesfin_2022a, Misra_2013, Sobnach_2011f, Yasin_2009, teWildt_2010}, 32 cases (44\%) ingested a long object (\textgreater{}5cm) \cite{Al-Faham_2020k, AlShaaibi_2021b, Ali_2017, Ali_2022g, Atayan_2016, Bhasin_2014, CamachoDorado_2018, Chang_2017f, Cox_2007, Csaky_1998e, DivsalarP._2023a, Emamhadi_2018, Fry_2010, Gardner_2017h, Jin_2023, Kariholu_2008, Kerestes_2019, Kobiela_2015, Kumar_2019f, Mesfin_2022a, Misra_2013, Ohno_2005, Qureshi_2016, Sakellaridis_2008f, Sultan_2024f, Thapa_2019f, Trgo_2012f, Yasin_2009, Yildiz_2016e, teWildt_2010}, 9 cases (12\%) ingested a magnet \cite{Ali_2020f, Bhumi_2024f, Cauchi_2002, Liu_2005, Naji_2012f, Ohno_2005, Tanrikulu_2015e, Tay_2004, Wildhaber_2005}, 2 cases (3\%) ingested a button battery \cite{Berry_2021e, Bhumi_2024f}. \paragraph*{Outcomes} 48 cases (67\%) experienced a complication \cite{Ali_2017, Ali_2020f, Apikotoa_2022f, Atayan_2016, Beecroft_1998, Benoist_2019e, Berry_2021e, Bhasin_2014, Bhumi_2024f, CamachoDorado_2018, Cauchi_2002, Cox_2007, Csaky_1998e, DelgadoSalazar_2020c, DivsalarP._2023a, Emamhadi_2018, Farhadi_2024h, Fry_2010, Gardner_2017h, Goldman_1998f, Jin_2023, Kariholu_2008, Kerestes_2019, Kobiela_2015, Kumar_2001, Kumar_2019f, Liu_2005, Losanoff_1996, Mesfin_2022a, Misra_2013, Naji_2012f, Ohno_2005, Sakellaridis_2008f, Sobnach_2011f, Sultan_2024f, Tanrikulu_2015e, Tay_2004, Thapa_2019f, Trgo_2012f, Tupesis_2004f, Wildhaber_2005, Wnęk_2015f, Yasin_2009, Yildiz_2016e}, 44 cases (61\%) underwent surgery \cite{Al-Faham_2020k, AlShaaibi_2021b, Alao_2006i, Ali_2017, Ali_2020f, Atayan_2016, Beecroft_1998, Bhasin_2014, CamachoDorado_2018, Cauchi_2002, Chang_2017f, Cox_2007, Csaky_1998e, DelgadoSalazar_2020c, DivsalarP._2023a, Farhadi_2024h, Fry_2010, Gardner_2017h, Jin_2023, Kariholu_2008, Kerestes_2019, Kobiela_2015, Kumar_2019f, Liu_2005, Losanoff_1996, Losanoff_1997e, Mesfin_2022a, Misra_2013, Naji_2012f, Sobnach_2011f, Tanrikulu_2015e, Tay_2004, Thapa_2019f, Tupesis_2004f, Wildhaber_2005, Wnęk_2015f, Yasin_2009, Yildiz_2016e, fjbuilsRepeatedBehaviorDeliberate2024}, 31 cases (43\%) underwent endoscopy \cite{Akay_2015f, Ali_2022g, Apikotoa_2022f, Atayan_2016, Benoist_2019e, Berry_2021e, Bhasin_2014, Bhumi_2024f, CamachoDorado_2018, Chang_2017f, DelgadoSalazar_2020c, Gardner_2017h, Guinan_2019f, Hardy_2023g, Jehangir_2019h, Kariholu_2008, Li_2013, Liu_2005, Ohno_2005, Peixoto_2017f, Qureshi_2016, Riva_2018j, Sakellaridis_2008f, Sultan_2024f, Tammana_2012j, Tanrikulu_2015e, Trgo_2012f, Wadhwa_2015e, Wnęk_2015f, teWildt_2010}, 7 cases (10\%) were managed conservatively \cite{Ataya_2013, Bhattacharjee_2008, DivsalarP._2023a, Emamhadi_2018, Goldman_1998f, Kar_2015, Kumar_2001}, 2 cases (3\%) died \cite{Emamhadi_2018, Kumar_2001}. All 90 were male gender. 90 cases (100\%) were detained at the time of ingestion \cite{Elghali_2016, Karp_1991b, Lee_2007}, 88 cases (98\%) were intentional ingestions \cite{Elghali_2016, Karp_1991b, Lee_2007}, 30 cases (33\%) had a psychiatric history documented \cite{Elghali_2016, Karp_1991b, Lee_2007}, 2 cases (2\%) had a history of prior ingestion \cite{Elghali_2016}. No cases were reported for were psychiatric inpatients, were displaced people, were under the influence of alcohol at the time of ingestion, and had a severe disability history.
\paragraph*{Motivation}  70 cases (78\%) reported protest motivation \cite{Elghali_2016, Karp_1991b, Lee_2007}, 12 cases (13\%) reported psychiatric motivation \cite{Karp_1991b}, 6 cases (7\%) reported self-harm motivation \cite{Elghali_2016, Karp_1991b}. No cases were reported for psychosocial motivation and other motivation.
\paragraph*{Object Characteristics}  68 cases (76\%) involved sharp object ingestion \cite{Elghali_2016, Karp_1991b, Lee_2007}, 32 cases (36\%) involved long (\textgreater 5cm) object ingestion \cite{Lee_2007}, 25 cases (28\%) involved ingestion of multiple objects \cite{Elghali_2016, Lee_2007}. No cases were reported for button battery ingestion, magnet ingestion, and involved large diameter (\textgreater 2.5cm) object ingestion.
\paragraph*{Outcomes}  47 cases (52\%) underwent endoscopic intervention \cite{Elghali_2016, Lee_2007}, 29 cases (32\%) were managed conservatively \cite{Elghali_2016, Karp_1991b}, 15 cases (17\%) underwent surgical intervention \cite{Elghali_2016, Karp_1991b, Lee_2007}, 6 cases (7\%) reported complications \cite{Lee_2007}, 1 case (1\%) died \cite{Elghali_2016}.
\paragraph*{Geographical Location}Cases were recorded in 33 countries: 13 cases from USA \cite{Alao_2006i, Ataya_2013, Bhumi_2024f, Fry_2010, Guinan_2019f, Hardy_2023g, Jehangir_2019h, Kerestes_2019, Kumar_2001, Liu_2005, Tammana_2012j, Tay_2004, Tupesis_2004f}; 7 cases from India \cite{Bhasin_2014, Bhattacharjee_2008, Kar_2015, Kariholu_2008, Kumar_2019f, Misra_2013, Wadhwa_2015e} and UK \cite{Beecroft_1998, Berry_2021e, Cauchi_2002, Cox_2007, Gardner_2017h, Qureshi_2016}; 6 cases from Bulgaria \cite{Losanoff_1996, Losanoff_1997e}; 5 cases from Iran \cite{DivsalarP._2023a, Emamhadi_2018, Farhadi_2024h}; 4 cases from Turkey \cite{Akay_2015f, Atayan_2016, Tanrikulu_2015e, Yildiz_2016e}; 2 cases from China \cite{Jin_2023, Li_2013}, Poland \cite{Kobiela_2015, Wnęk_2015f}, and Spain \cite{CamachoDorado_2018, fjbuilsRepeatedBehaviorDeliberate2024}; 1 case from Australia \cite{Apikotoa_2022f}, Bahrain \cite{Ali_2020f}, Croatia \cite{Trgo_2012f}, Ecuador \cite{DelgadoSalazar_2020c}, Egypt \cite{Ali_2022g}, Ethiopia \cite{Mesfin_2022a}, Germany \cite{teWildt_2010}, Greece \cite{Sakellaridis_2008f}, Hungary \cite{Csaky_1998e}, Iraq \cite{Al-Faham_2020k}, Israel \cite{Goldman_1998f}, Italy \cite{Riva_2018j}, Japan \cite{Ohno_2005}, Nepal \cite{Thapa_2019f}, Netherlands \cite{Benoist_2019e}, Oman \cite{AlShaaibi_2021b}, Pakistan \cite{Yasin_2009}, Portugal \cite{Peixoto_2017f}, Qatar \cite{Ali_2017}, Saudi Arabia \cite{Sultan_2024f}, South Africa \cite{Sobnach_2011f}, Sweden \cite{Naji_2012f}, Switzerland \cite{Wildhaber_2005}, and Taiwan \cite{Chang_2017f}. \paragraph*{Gender} 43 cases (60\%) were male \cite{Akay_2015f, Al-Faham_2020k, Alao_2006i, Ali_2017, Ali_2022g, Apikotoa_2022f, Atayan_2016, Benoist_2019e, Berry_2021e, Bhumi_2024f, CamachoDorado_2018, Csaky_1998e, Emamhadi_2018, Farhadi_2024h, Fry_2010, Gardner_2017h, Guinan_2019f, Jehangir_2019h, Jin_2023, Kobiela_2015, Kumar_2001, Kumar_2019f, Liu_2005, Losanoff_1996, Losanoff_1997e, Mesfin_2022a, Misra_2013, Qureshi_2016, Riva_2018j, Sobnach_2011f, Tammana_2012j, Tanrikulu_2015e, Tay_2004, Thapa_2019f, Trgo_2012f, Wadhwa_2015e, Yasin_2009, teWildt_2010}, 28 cases (39\%) were female \cite{AlShaaibi_2021b, Ali_2020f, Ataya_2013, Beecroft_1998, Bhasin_2014, Bhattacharjee_2008, Cauchi_2002, Chang_2017f, Cox_2007, DelgadoSalazar_2020c, DivsalarP._2023a, Goldman_1998f, Hardy_2023g, Kar_2015, Kariholu_2008, Kerestes_2019, Li_2013, Naji_2012f, Ohno_2005, Peixoto_2017f, Sakellaridis_2008f, Sultan_2024f, Tupesis_2004f, Wildhaber_2005, Wnęk_2015f, Yildiz_2016e}, 1 case (1\%) had no gender recorded \cite{fjbuilsRepeatedBehaviorDeliberate2024}. \paragraph*{Age Group} 25 cases (35\%) were between 26 and 40 years of age \cite{Alao_2006i, Ali_2022g, Apikotoa_2022f, Ataya_2013, Benoist_2019e, Bhasin_2014, Chang_2017f, Cox_2007, DelgadoSalazar_2020c, Farhadi_2024h, Fry_2010, Gardner_2017h, Guinan_2019f, Jin_2023, Kumar_2019f, Losanoff_1996, Misra_2013, Qureshi_2016, Riva_2018j, Sakellaridis_2008f, Tammana_2012j, Trgo_2012f, Wnęk_2015f, Yildiz_2016e, fjbuilsRepeatedBehaviorDeliberate2024}, 18 cases (25\%) were between 18 and 25 years of age \cite{Akay_2015f, Ali_2017, Atayan_2016, Bhattacharjee_2008, Csaky_1998e, Kar_2015, Kariholu_2008, Kobiela_2015, Losanoff_1996, Losanoff_1997e, Mesfin_2022a, Peixoto_2017f, Sobnach_2011f, Tupesis_2004f, Yasin_2009}, 13 cases (18\%) were under 18 years of age \cite{AlShaaibi_2021b, Ali_2020f, Cauchi_2002, DivsalarP._2023a, Goldman_1998f, Liu_2005, Naji_2012f, Ohno_2005, Tanrikulu_2015e, Tay_2004, Wildhaber_2005}, 11 cases (15\%) were between 41 and 60 years of age \cite{Al-Faham_2020k, Bhumi_2024f, CamachoDorado_2018, Emamhadi_2018, Hardy_2023g, Jehangir_2019h, Kumar_2001, Sultan_2024f, Thapa_2019f, Wadhwa_2015e, teWildt_2010}, 3 cases (4\%) were over 60 years of age \cite{Beecroft_1998, Kerestes_2019, Li_2013}, 2 cases (3\%) had no age documented \cite{Berry_2021e}. \paragraph*{Population} 36 cases (50\%) had a psychiatric history \cite{AlShaaibi_2021b, Alao_2006i, Ali_2020f, Apikotoa_2022f, Ataya_2013, Atayan_2016, Beecroft_1998, CamachoDorado_2018, Chang_2017f, DelgadoSalazar_2020c, DivsalarP._2023a, Farhadi_2024h, Fry_2010, Guinan_2019f, Hardy_2023g, Jehangir_2019h, Jin_2023, Kar_2015, Kerestes_2019, Kobiela_2015, Kumar_2001, Kumar_2019f, Liu_2005, Mesfin_2022a, Misra_2013, Ohno_2005, Peixoto_2017f, Sakellaridis_2008f, Sultan_2024f, Tammana_2012j, Tanrikulu_2015e, Yildiz_2016e, fjbuilsRepeatedBehaviorDeliberate2024, teWildt_2010}, 19 cases (26\%) had ingested previously \cite{Alao_2006i, Apikotoa_2022f, Berry_2021e, Bhattacharjee_2008, Csaky_1998e, DivsalarP._2023a, Emamhadi_2018, Guinan_2019f, Jehangir_2019h, Jin_2023, Liu_2005, Sakellaridis_2008f, Tanrikulu_2015e, Thapa_2019f, Yildiz_2016e, fjbuilsRepeatedBehaviorDeliberate2024, teWildt_2010}, 12 cases (17\%) were detained persons \cite{Alao_2006i, Ali_2022g, Apikotoa_2022f, Losanoff_1996, Losanoff_1997e, Qureshi_2016, Tammana_2012j, Trgo_2012f}, 7 cases (10\%) were severely disabled \cite{Atayan_2016, Kerestes_2019, Liu_2005, Ohno_2005, Peixoto_2017f, Yildiz_2016e, teWildt_2010}, 4 cases (6\%) were psychiatric inpatients \cite{DivsalarP._2023a, fjbuilsRepeatedBehaviorDeliberate2024, teWildt_2010}, 3 cases (4\%) were under the influence of alcohol \cite{Benoist_2019e, Csaky_1998e, Thapa_2019f}, 2 cases (3\%) were displaced people \cite{Akay_2015f, Gardner_2017h}. \paragraph*{Motivation} 34 cases (47\%) had a psychiatric motivation \cite{Al-Faham_2020k, Alao_2006i, Ali_2020f, Apikotoa_2022f, Ataya_2013, Atayan_2016, Bhasin_2014, Bhattacharjee_2008, DelgadoSalazar_2020c, DivsalarP._2023a, Emamhadi_2018, Farhadi_2024h, Guinan_2019f, Hardy_2023g, Jehangir_2019h, Jin_2023, Kar_2015, Kariholu_2008, Kerestes_2019, Kobiela_2015, Kumar_2001, Kumar_2019f, Li_2013, Liu_2005, Misra_2013, Ohno_2005, Sakellaridis_2008f, Sultan_2024f, Tammana_2012j, Tanrikulu_2015e, Yasin_2009, teWildt_2010}, 21 cases (29\%) were motivated by self-harm intention \cite{Al-Faham_2020k, AlShaaibi_2021b, Alao_2006i, Ali_2017, CamachoDorado_2018, Chang_2017f, Cox_2007, Csaky_1998e, Fry_2010, Li_2013, Losanoff_1996, Losanoff_1997e, Mesfin_2022a, Sakellaridis_2008f, Tammana_2012j, Tanrikulu_2015e, fjbuilsRepeatedBehaviorDeliberate2024}, 17 cases (24\%) had a psychosocial motivation \cite{Akay_2015f, Benoist_2019e, Bhattacharjee_2008, Cauchi_2002, Goldman_1998f, Hardy_2023g, Kobiela_2015, Li_2013, Naji_2012f, Qureshi_2016, Riva_2018j, Sobnach_2011f, Tay_2004, Thapa_2019f, Tupesis_2004f, Wildhaber_2005, Wnęk_2015f}, 9 cases (12\%) were motivated by protest \cite{Bhumi_2024f, Gardner_2017h, Losanoff_1996, Losanoff_1997e, Tupesis_2004f}, 9 cases (12\%) had another documented motivation \cite{Ali_2020f, Ali_2022g, Emamhadi_2018, Guinan_2019f, Peixoto_2017f, Sakellaridis_2008f, Trgo_2012f, Wadhwa_2015e, Yildiz_2016e}. \paragraph*{Object Characteristics} 51 cases (71\%) ingested a large diameter object (\textgreater{}2.5cm) \cite{Akay_2015f, Al-Faham_2020k, AlShaaibi_2021b, Alao_2006i, Ali_2017, Ali_2022g, Apikotoa_2022f, Atayan_2016, Berry_2021e, Bhasin_2014, CamachoDorado_2018, Cauchi_2002, Chang_2017f, Cox_2007, Csaky_1998e, DivsalarP._2023a, Emamhadi_2018, Gardner_2017h, Guinan_2019f, Jehangir_2019h, Jin_2023, Kariholu_2008, Kerestes_2019, Kobiela_2015, Kumar_2001, Kumar_2019f, Losanoff_1996, Losanoff_1997e, Mesfin_2022a, Misra_2013, Naji_2012f, Ohno_2005, Peixoto_2017f, Qureshi_2016, Riva_2018j, Sakellaridis_2008f, Sultan_2024f, Tanrikulu_2015e, Thapa_2019f, Trgo_2012f, Wnęk_2015f, Yildiz_2016e, fjbuilsRepeatedBehaviorDeliberate2024, teWildt_2010}, 44 cases (61\%) ingested multiple objects \cite{Ali_2020f, Apikotoa_2022f, Ataya_2013, Atayan_2016, Beecroft_1998, Bhattacharjee_2008, Bhumi_2024f, CamachoDorado_2018, Cauchi_2002, Emamhadi_2018, Farhadi_2024h, Fry_2010, Goldman_1998f, Guinan_2019f, Hardy_2023g, Jehangir_2019h, Jin_2023, Kar_2015, Kariholu_2008, Kobiela_2015, Kumar_2001, Kumar_2019f, Li_2013, Liu_2005, Losanoff_1996, Mesfin_2022a, Misra_2013, Naji_2012f, Ohno_2005, Sobnach_2011f, Sultan_2024f, Tammana_2012j, Tanrikulu_2015e, Tay_2004, Thapa_2019f, Wadhwa_2015e, Wildhaber_2005, Yasin_2009, fjbuilsRepeatedBehaviorDeliberate2024, teWildt_2010}, 34 cases (47\%) ingested a sharp object \cite{AlShaaibi_2021b, Alao_2006i, Apikotoa_2022f, Ataya_2013, Benoist_2019e, Bhasin_2014, Bhattacharjee_2008, CamachoDorado_2018, Csaky_1998e, DelgadoSalazar_2020c, DivsalarP._2023a, Emamhadi_2018, Farhadi_2024h, Fry_2010, Guinan_2019f, Hardy_2023g, Jehangir_2019h, Jin_2023, Kariholu_2008, Kobiela_2015, Kumar_2019f, Losanoff_1996, Losanoff_1997e, Mesfin_2022a, Misra_2013, Sobnach_2011f, Yasin_2009, teWildt_2010}, 32 cases (44\%) ingested a long object (\textgreater{}5cm) \cite{Al-Faham_2020k, AlShaaibi_2021b, Ali_2017, Ali_2022g, Atayan_2016, Bhasin_2014, CamachoDorado_2018, Chang_2017f, Cox_2007, Csaky_1998e, DivsalarP._2023a, Emamhadi_2018, Fry_2010, Gardner_2017h, Jin_2023, Kariholu_2008, Kerestes_2019, Kobiela_2015, Kumar_2019f, Mesfin_2022a, Misra_2013, Ohno_2005, Qureshi_2016, Sakellaridis_2008f, Sultan_2024f, Thapa_2019f, Trgo_2012f, Yasin_2009, Yildiz_2016e, teWildt_2010}, 9 cases (12\%) ingested a magnet \cite{Ali_2020f, Bhumi_2024f, Cauchi_2002, Liu_2005, Naji_2012f, Ohno_2005, Tanrikulu_2015e, Tay_2004, Wildhaber_2005}, 2 cases (3\%) ingested a button battery \cite{Berry_2021e, Bhumi_2024f}. \paragraph*{Outcomes} 48 cases (67\%) experienced a complication \cite{Ali_2017, Ali_2020f, Apikotoa_2022f, Atayan_2016, Beecroft_1998, Benoist_2019e, Berry_2021e, Bhasin_2014, Bhumi_2024f, CamachoDorado_2018, Cauchi_2002, Cox_2007, Csaky_1998e, DelgadoSalazar_2020c, DivsalarP._2023a, Emamhadi_2018, Farhadi_2024h, Fry_2010, Gardner_2017h, Goldman_1998f, Jin_2023, Kariholu_2008, Kerestes_2019, Kobiela_2015, Kumar_2001, Kumar_2019f, Liu_2005, Losanoff_1996, Mesfin_2022a, Misra_2013, Naji_2012f, Ohno_2005, Sakellaridis_2008f, Sobnach_2011f, Sultan_2024f, Tanrikulu_2015e, Tay_2004, Thapa_2019f, Trgo_2012f, Tupesis_2004f, Wildhaber_2005, Wnęk_2015f, Yasin_2009, Yildiz_2016e}, 44 cases (61\%) underwent surgery \cite{Al-Faham_2020k, AlShaaibi_2021b, Alao_2006i, Ali_2017, Ali_2020f, Atayan_2016, Beecroft_1998, Bhasin_2014, CamachoDorado_2018, Cauchi_2002, Chang_2017f, Cox_2007, Csaky_1998e, DelgadoSalazar_2020c, DivsalarP._2023a, Farhadi_2024h, Fry_2010, Gardner_2017h, Jin_2023, Kariholu_2008, Kerestes_2019, Kobiela_2015, Kumar_2019f, Liu_2005, Losanoff_1996, Losanoff_1997e, Mesfin_2022a, Misra_2013, Naji_2012f, Sobnach_2011f, Tanrikulu_2015e, Tay_2004, Thapa_2019f, Tupesis_2004f, Wildhaber_2005, Wnęk_2015f, Yasin_2009, Yildiz_2016e, fjbuilsRepeatedBehaviorDeliberate2024}, 31 cases (43\%) underwent endoscopy \cite{Akay_2015f, Ali_2022g, Apikotoa_2022f, Atayan_2016, Benoist_2019e, Berry_2021e, Bhasin_2014, Bhumi_2024f, CamachoDorado_2018, Chang_2017f, DelgadoSalazar_2020c, Gardner_2017h, Guinan_2019f, Hardy_2023g, Jehangir_2019h, Kariholu_2008, Li_2013, Liu_2005, Ohno_2005, Peixoto_2017f, Qureshi_2016, Riva_2018j, Sakellaridis_2008f, Sultan_2024f, Tammana_2012j, Tanrikulu_2015e, Trgo_2012f, Wadhwa_2015e, Wnęk_2015f, teWildt_2010}, 7 cases (10\%) were managed conservatively \cite{Ataya_2013, Bhattacharjee_2008, DivsalarP._2023a, Emamhadi_2018, Goldman_1998f, Kar_2015, Kumar_2001}, 2 cases (3\%) died \cite{Emamhadi_2018, Kumar_2001}. All 90 were male gender. 90 cases (100\%) were detained at the time of ingestion \cite{Elghali_2016, Karp_1991b, Lee_2007}, 88 cases (98\%) were intentional ingestions \cite{Elghali_2016, Karp_1991b, Lee_2007}, 30 cases (33\%) had a psychiatric history documented \cite{Elghali_2016, Karp_1991b, Lee_2007}, 2 cases (2\%) had a history of prior ingestion \cite{Elghali_2016}. No cases were reported for were psychiatric inpatients, were displaced people, were under the influence of alcohol at the time of ingestion, and had a severe disability history.
\paragraph*{Motivation}  70 cases (78\%) reported protest motivation \cite{Elghali_2016, Karp_1991b, Lee_2007}, 12 cases (13\%) reported psychiatric motivation \cite{Karp_1991b}, 6 cases (7\%) reported self-harm motivation \cite{Elghali_2016, Karp_1991b}. No cases were reported for psychosocial motivation and other motivation.
\paragraph*{Object Characteristics}  68 cases (76\%) involved sharp object ingestion \cite{Elghali_2016, Karp_1991b, Lee_2007}, 32 cases (36\%) involved long (\textgreater 5cm) object ingestion \cite{Lee_2007}, 25 cases (28\%) involved ingestion of multiple objects \cite{Elghali_2016, Lee_2007}. No cases were reported for button battery ingestion, magnet ingestion, and involved large diameter (\textgreater 2.5cm) object ingestion.
\paragraph*{Outcomes}  47 cases (52\%) underwent endoscopic intervention \cite{Elghali_2016, Lee_2007}, 29 cases (32\%) were managed conservatively \cite{Elghali_2016, Karp_1991b}, 15 cases (17\%) underwent surgical intervention \cite{Elghali_2016, Karp_1991b, Lee_2007}, 6 cases (7\%) reported complications \cite{Lee_2007}, 1 case (1\%) died \cite{Elghali_2016}.
\paragraph*{Geographical Location}Cases were recorded in 33 countries: 13 cases from USA \cite{Alao_2006i, Ataya_2013, Bhumi_2024f, Fry_2010, Guinan_2019f, Hardy_2023g, Jehangir_2019h, Kerestes_2019, Kumar_2001, Liu_2005, Tammana_2012j, Tay_2004, Tupesis_2004f}; 7 cases from India \cite{Bhasin_2014, Bhattacharjee_2008, Kar_2015, Kariholu_2008, Kumar_2019f, Misra_2013, Wadhwa_2015e} and UK \cite{Beecroft_1998, Berry_2021e, Cauchi_2002, Cox_2007, Gardner_2017h, Qureshi_2016}; 6 cases from Bulgaria \cite{Losanoff_1996, Losanoff_1997e}; 5 cases from Iran \cite{DivsalarP._2023a, Emamhadi_2018, Farhadi_2024h}; 4 cases from Turkey \cite{Akay_2015f, Atayan_2016, Tanrikulu_2015e, Yildiz_2016e}; 2 cases from China \cite{Jin_2023, Li_2013}, Poland \cite{Kobiela_2015, Wnęk_2015f}, and Spain \cite{CamachoDorado_2018, fjbuilsRepeatedBehaviorDeliberate2024}; 1 case from Australia \cite{Apikotoa_2022f}, Bahrain \cite{Ali_2020f}, Croatia \cite{Trgo_2012f}, Ecuador \cite{DelgadoSalazar_2020c}, Egypt \cite{Ali_2022g}, Ethiopia \cite{Mesfin_2022a}, Germany \cite{teWildt_2010}, Greece \cite{Sakellaridis_2008f}, Hungary \cite{Csaky_1998e}, Iraq \cite{Al-Faham_2020k}, Israel \cite{Goldman_1998f}, Italy \cite{Riva_2018j}, Japan \cite{Ohno_2005}, Nepal \cite{Thapa_2019f}, Netherlands \cite{Benoist_2019e}, Oman \cite{AlShaaibi_2021b}, Pakistan \cite{Yasin_2009}, Portugal \cite{Peixoto_2017f}, Qatar \cite{Ali_2017}, Saudi Arabia \cite{Sultan_2024f}, South Africa \cite{Sobnach_2011f}, Sweden \cite{Naji_2012f}, Switzerland \cite{Wildhaber_2005}, and Taiwan \cite{Chang_2017f}. \paragraph*{Gender} 43 cases (60\%) were male \cite{Akay_2015f, Al-Faham_2020k, Alao_2006i, Ali_2017, Ali_2022g, Apikotoa_2022f, Atayan_2016, Benoist_2019e, Berry_2021e, Bhumi_2024f, CamachoDorado_2018, Csaky_1998e, Emamhadi_2018, Farhadi_2024h, Fry_2010, Gardner_2017h, Guinan_2019f, Jehangir_2019h, Jin_2023, Kobiela_2015, Kumar_2001, Kumar_2019f, Liu_2005, Losanoff_1996, Losanoff_1997e, Mesfin_2022a, Misra_2013, Qureshi_2016, Riva_2018j, Sobnach_2011f, Tammana_2012j, Tanrikulu_2015e, Tay_2004, Thapa_2019f, Trgo_2012f, Wadhwa_2015e, Yasin_2009, teWildt_2010}, 28 cases (39\%) were female \cite{AlShaaibi_2021b, Ali_2020f, Ataya_2013, Beecroft_1998, Bhasin_2014, Bhattacharjee_2008, Cauchi_2002, Chang_2017f, Cox_2007, DelgadoSalazar_2020c, DivsalarP._2023a, Goldman_1998f, Hardy_2023g, Kar_2015, Kariholu_2008, Kerestes_2019, Li_2013, Naji_2012f, Ohno_2005, Peixoto_2017f, Sakellaridis_2008f, Sultan_2024f, Tupesis_2004f, Wildhaber_2005, Wnęk_2015f, Yildiz_2016e}, 1 case (1\%) had no gender recorded \cite{fjbuilsRepeatedBehaviorDeliberate2024}. \paragraph*{Age Group} 25 cases (35\%) were between 26 and 40 years of age \cite{Alao_2006i, Ali_2022g, Apikotoa_2022f, Ataya_2013, Benoist_2019e, Bhasin_2014, Chang_2017f, Cox_2007, DelgadoSalazar_2020c, Farhadi_2024h, Fry_2010, Gardner_2017h, Guinan_2019f, Jin_2023, Kumar_2019f, Losanoff_1996, Misra_2013, Qureshi_2016, Riva_2018j, Sakellaridis_2008f, Tammana_2012j, Trgo_2012f, Wnęk_2015f, Yildiz_2016e, fjbuilsRepeatedBehaviorDeliberate2024}, 18 cases (25\%) were between 18 and 25 years of age \cite{Akay_2015f, Ali_2017, Atayan_2016, Bhattacharjee_2008, Csaky_1998e, Kar_2015, Kariholu_2008, Kobiela_2015, Losanoff_1996, Losanoff_1997e, Mesfin_2022a, Peixoto_2017f, Sobnach_2011f, Tupesis_2004f, Yasin_2009}, 13 cases (18\%) were under 18 years of age \cite{AlShaaibi_2021b, Ali_2020f, Cauchi_2002, DivsalarP._2023a, Goldman_1998f, Liu_2005, Naji_2012f, Ohno_2005, Tanrikulu_2015e, Tay_2004, Wildhaber_2005}, 11 cases (15\%) were between 41 and 60 years of age \cite{Al-Faham_2020k, Bhumi_2024f, CamachoDorado_2018, Emamhadi_2018, Hardy_2023g, Jehangir_2019h, Kumar_2001, Sultan_2024f, Thapa_2019f, Wadhwa_2015e, teWildt_2010}, 3 cases (4\%) were over 60 years of age \cite{Beecroft_1998, Kerestes_2019, Li_2013}, 2 cases (3\%) had no age documented \cite{Berry_2021e}. \paragraph*{Population} 36 cases (50\%) had a psychiatric history \cite{AlShaaibi_2021b, Alao_2006i, Ali_2020f, Apikotoa_2022f, Ataya_2013, Atayan_2016, Beecroft_1998, CamachoDorado_2018, Chang_2017f, DelgadoSalazar_2020c, DivsalarP._2023a, Farhadi_2024h, Fry_2010, Guinan_2019f, Hardy_2023g, Jehangir_2019h, Jin_2023, Kar_2015, Kerestes_2019, Kobiela_2015, Kumar_2001, Kumar_2019f, Liu_2005, Mesfin_2022a, Misra_2013, Ohno_2005, Peixoto_2017f, Sakellaridis_2008f, Sultan_2024f, Tammana_2012j, Tanrikulu_2015e, Yildiz_2016e, fjbuilsRepeatedBehaviorDeliberate2024, teWildt_2010}, 19 cases (26\%) had ingested previously \cite{Alao_2006i, Apikotoa_2022f, Berry_2021e, Bhattacharjee_2008, Csaky_1998e, DivsalarP._2023a, Emamhadi_2018, Guinan_2019f, Jehangir_2019h, Jin_2023, Liu_2005, Sakellaridis_2008f, Tanrikulu_2015e, Thapa_2019f, Yildiz_2016e, fjbuilsRepeatedBehaviorDeliberate2024, teWildt_2010}, 12 cases (17\%) were detained persons \cite{Alao_2006i, Ali_2022g, Apikotoa_2022f, Losanoff_1996, Losanoff_1997e, Qureshi_2016, Tammana_2012j, Trgo_2012f}, 7 cases (10\%) were severely disabled \cite{Atayan_2016, Kerestes_2019, Liu_2005, Ohno_2005, Peixoto_2017f, Yildiz_2016e, teWildt_2010}, 4 cases (6\%) were psychiatric inpatients \cite{DivsalarP._2023a, fjbuilsRepeatedBehaviorDeliberate2024, teWildt_2010}, 3 cases (4\%) were under the influence of alcohol \cite{Benoist_2019e, Csaky_1998e, Thapa_2019f}, 2 cases (3\%) were displaced people \cite{Akay_2015f, Gardner_2017h}. \paragraph*{Motivation} 34 cases (47\%) had a psychiatric motivation \cite{Al-Faham_2020k, Alao_2006i, Ali_2020f, Apikotoa_2022f, Ataya_2013, Atayan_2016, Bhasin_2014, Bhattacharjee_2008, DelgadoSalazar_2020c, DivsalarP._2023a, Emamhadi_2018, Farhadi_2024h, Guinan_2019f, Hardy_2023g, Jehangir_2019h, Jin_2023, Kar_2015, Kariholu_2008, Kerestes_2019, Kobiela_2015, Kumar_2001, Kumar_2019f, Li_2013, Liu_2005, Misra_2013, Ohno_2005, Sakellaridis_2008f, Sultan_2024f, Tammana_2012j, Tanrikulu_2015e, Yasin_2009, teWildt_2010}, 21 cases (29\%) were motivated by self-harm intention \cite{Al-Faham_2020k, AlShaaibi_2021b, Alao_2006i, Ali_2017, CamachoDorado_2018, Chang_2017f, Cox_2007, Csaky_1998e, Fry_2010, Li_2013, Losanoff_1996, Losanoff_1997e, Mesfin_2022a, Sakellaridis_2008f, Tammana_2012j, Tanrikulu_2015e, fjbuilsRepeatedBehaviorDeliberate2024}, 17 cases (24\%) had a psychosocial motivation \cite{Akay_2015f, Benoist_2019e, Bhattacharjee_2008, Cauchi_2002, Goldman_1998f, Hardy_2023g, Kobiela_2015, Li_2013, Naji_2012f, Qureshi_2016, Riva_2018j, Sobnach_2011f, Tay_2004, Thapa_2019f, Tupesis_2004f, Wildhaber_2005, Wnęk_2015f}, 9 cases (12\%) were motivated by protest \cite{Bhumi_2024f, Gardner_2017h, Losanoff_1996, Losanoff_1997e, Tupesis_2004f}, 9 cases (12\%) had another documented motivation \cite{Ali_2020f, Ali_2022g, Emamhadi_2018, Guinan_2019f, Peixoto_2017f, Sakellaridis_2008f, Trgo_2012f, Wadhwa_2015e, Yildiz_2016e}. \paragraph*{Object Characteristics} 51 cases (71\%) ingested a large diameter object (\textgreater{}2.5cm) \cite{Akay_2015f, Al-Faham_2020k, AlShaaibi_2021b, Alao_2006i, Ali_2017, Ali_2022g, Apikotoa_2022f, Atayan_2016, Berry_2021e, Bhasin_2014, CamachoDorado_2018, Cauchi_2002, Chang_2017f, Cox_2007, Csaky_1998e, DivsalarP._2023a, Emamhadi_2018, Gardner_2017h, Guinan_2019f, Jehangir_2019h, Jin_2023, Kariholu_2008, Kerestes_2019, Kobiela_2015, Kumar_2001, Kumar_2019f, Losanoff_1996, Losanoff_1997e, Mesfin_2022a, Misra_2013, Naji_2012f, Ohno_2005, Peixoto_2017f, Qureshi_2016, Riva_2018j, Sakellaridis_2008f, Sultan_2024f, Tanrikulu_2015e, Thapa_2019f, Trgo_2012f, Wnęk_2015f, Yildiz_2016e, fjbuilsRepeatedBehaviorDeliberate2024, teWildt_2010}, 44 cases (61\%) ingested multiple objects \cite{Ali_2020f, Apikotoa_2022f, Ataya_2013, Atayan_2016, Beecroft_1998, Bhattacharjee_2008, Bhumi_2024f, CamachoDorado_2018, Cauchi_2002, Emamhadi_2018, Farhadi_2024h, Fry_2010, Goldman_1998f, Guinan_2019f, Hardy_2023g, Jehangir_2019h, Jin_2023, Kar_2015, Kariholu_2008, Kobiela_2015, Kumar_2001, Kumar_2019f, Li_2013, Liu_2005, Losanoff_1996, Mesfin_2022a, Misra_2013, Naji_2012f, Ohno_2005, Sobnach_2011f, Sultan_2024f, Tammana_2012j, Tanrikulu_2015e, Tay_2004, Thapa_2019f, Wadhwa_2015e, Wildhaber_2005, Yasin_2009, fjbuilsRepeatedBehaviorDeliberate2024, teWildt_2010}, 34 cases (47\%) ingested a sharp object \cite{AlShaaibi_2021b, Alao_2006i, Apikotoa_2022f, Ataya_2013, Benoist_2019e, Bhasin_2014, Bhattacharjee_2008, CamachoDorado_2018, Csaky_1998e, DelgadoSalazar_2020c, DivsalarP._2023a, Emamhadi_2018, Farhadi_2024h, Fry_2010, Guinan_2019f, Hardy_2023g, Jehangir_2019h, Jin_2023, Kariholu_2008, Kobiela_2015, Kumar_2019f, Losanoff_1996, Losanoff_1997e, Mesfin_2022a, Misra_2013, Sobnach_2011f, Yasin_2009, teWildt_2010}, 32 cases (44\%) ingested a long object (\textgreater{}5cm) \cite{Al-Faham_2020k, AlShaaibi_2021b, Ali_2017, Ali_2022g, Atayan_2016, Bhasin_2014, CamachoDorado_2018, Chang_2017f, Cox_2007, Csaky_1998e, DivsalarP._2023a, Emamhadi_2018, Fry_2010, Gardner_2017h, Jin_2023, Kariholu_2008, Kerestes_2019, Kobiela_2015, Kumar_2019f, Mesfin_2022a, Misra_2013, Ohno_2005, Qureshi_2016, Sakellaridis_2008f, Sultan_2024f, Thapa_2019f, Trgo_2012f, Yasin_2009, Yildiz_2016e, teWildt_2010}, 9 cases (12\%) ingested a magnet \cite{Ali_2020f, Bhumi_2024f, Cauchi_2002, Liu_2005, Naji_2012f, Ohno_2005, Tanrikulu_2015e, Tay_2004, Wildhaber_2005}, 2 cases (3\%) ingested a button battery \cite{Berry_2021e, Bhumi_2024f}. \paragraph*{Outcomes} 48 cases (67\%) experienced a complication \cite{Ali_2017, Ali_2020f, Apikotoa_2022f, Atayan_2016, Beecroft_1998, Benoist_2019e, Berry_2021e, Bhasin_2014, Bhumi_2024f, CamachoDorado_2018, Cauchi_2002, Cox_2007, Csaky_1998e, DelgadoSalazar_2020c, DivsalarP._2023a, Emamhadi_2018, Farhadi_2024h, Fry_2010, Gardner_2017h, Goldman_1998f, Jin_2023, Kariholu_2008, Kerestes_2019, Kobiela_2015, Kumar_2001, Kumar_2019f, Liu_2005, Losanoff_1996, Mesfin_2022a, Misra_2013, Naji_2012f, Ohno_2005, Sakellaridis_2008f, Sobnach_2011f, Sultan_2024f, Tanrikulu_2015e, Tay_2004, Thapa_2019f, Trgo_2012f, Tupesis_2004f, Wildhaber_2005, Wnęk_2015f, Yasin_2009, Yildiz_2016e}, 44 cases (61\%) underwent surgery \cite{Al-Faham_2020k, AlShaaibi_2021b, Alao_2006i, Ali_2017, Ali_2020f, Atayan_2016, Beecroft_1998, Bhasin_2014, CamachoDorado_2018, Cauchi_2002, Chang_2017f, Cox_2007, Csaky_1998e, DelgadoSalazar_2020c, DivsalarP._2023a, Farhadi_2024h, Fry_2010, Gardner_2017h, Jin_2023, Kariholu_2008, Kerestes_2019, Kobiela_2015, Kumar_2019f, Liu_2005, Losanoff_1996, Losanoff_1997e, Mesfin_2022a, Misra_2013, Naji_2012f, Sobnach_2011f, Tanrikulu_2015e, Tay_2004, Thapa_2019f, Tupesis_2004f, Wildhaber_2005, Wnęk_2015f, Yasin_2009, Yildiz_2016e, fjbuilsRepeatedBehaviorDeliberate2024}, 31 cases (43\%) underwent endoscopy \cite{Akay_2015f, Ali_2022g, Apikotoa_2022f, Atayan_2016, Benoist_2019e, Berry_2021e, Bhasin_2014, Bhumi_2024f, CamachoDorado_2018, Chang_2017f, DelgadoSalazar_2020c, Gardner_2017h, Guinan_2019f, Hardy_2023g, Jehangir_2019h, Kariholu_2008, Li_2013, Liu_2005, Ohno_2005, Peixoto_2017f, Qureshi_2016, Riva_2018j, Sakellaridis_2008f, Sultan_2024f, Tammana_2012j, Tanrikulu_2015e, Trgo_2012f, Wadhwa_2015e, Wnęk_2015f, teWildt_2010}, 7 cases (10\%) were managed conservatively \cite{Ataya_2013, Bhattacharjee_2008, DivsalarP._2023a, Emamhadi_2018, Goldman_1998f, Kar_2015, Kumar_2001}, 2 cases (3\%) died \cite{Emamhadi_2018, Kumar_2001}. All 90 were male gender. 90 cases (100\%) were detained at the time of ingestion \cite{Elghali_2016, Karp_1991b, Lee_2007}, 88 cases (98\%) were intentional ingestions \cite{Elghali_2016, Karp_1991b, Lee_2007}, 30 cases (33\%) had a psychiatric history documented \cite{Elghali_2016, Karp_1991b, Lee_2007}, 2 cases (2\%) had a history of prior ingestion \cite{Elghali_2016}. No cases were reported for were psychiatric inpatients, were displaced people, were under the influence of alcohol at the time of ingestion, and had a severe disability history.
\paragraph*{Motivation}  70 cases (78\%) reported protest motivation \cite{Elghali_2016, Karp_1991b, Lee_2007}, 12 cases (13\%) reported psychiatric motivation \cite{Karp_1991b}, 6 cases (7\%) reported self-harm motivation \cite{Elghali_2016, Karp_1991b}. No cases were reported for psychosocial motivation and other motivation.
\paragraph*{Object Characteristics}  68 cases (76\%) involved sharp object ingestion \cite{Elghali_2016, Karp_1991b, Lee_2007}, 32 cases (36\%) involved long (\textgreater 5cm) object ingestion \cite{Lee_2007}, 25 cases (28\%) involved ingestion of multiple objects \cite{Elghali_2016, Lee_2007}. No cases were reported for button battery ingestion, magnet ingestion, and involved large diameter (\textgreater 2.5cm) object ingestion.
\paragraph*{Outcomes}  47 cases (52\%) underwent endoscopic intervention \cite{Elghali_2016, Lee_2007}, 29 cases (32\%) were managed conservatively \cite{Elghali_2016, Karp_1991b}, 15 cases (17\%) underwent surgical intervention \cite{Elghali_2016, Karp_1991b, Lee_2007}, 6 cases (7\%) reported complications \cite{Lee_2007}, 1 case (1\%) died \cite{Elghali_2016}.
\paragraph*{Geographical Location}Cases were recorded in 33 countries: 13 cases from USA \cite{Alao_2006i, Ataya_2013, Bhumi_2024f, Fry_2010, Guinan_2019f, Hardy_2023g, Jehangir_2019h, Kerestes_2019, Kumar_2001, Liu_2005, Tammana_2012j, Tay_2004, Tupesis_2004f}; 7 cases from India \cite{Bhasin_2014, Bhattacharjee_2008, Kar_2015, Kariholu_2008, Kumar_2019f, Misra_2013, Wadhwa_2015e} and UK \cite{Beecroft_1998, Berry_2021e, Cauchi_2002, Cox_2007, Gardner_2017h, Qureshi_2016}; 6 cases from Bulgaria \cite{Losanoff_1996, Losanoff_1997e}; 5 cases from Iran \cite{DivsalarP._2023a, Emamhadi_2018, Farhadi_2024h}; 4 cases from Turkey \cite{Akay_2015f, Atayan_2016, Tanrikulu_2015e, Yildiz_2016e}; 2 cases from China \cite{Jin_2023, Li_2013}, Poland \cite{Kobiela_2015, Wnęk_2015f}, and Spain \cite{CamachoDorado_2018, fjbuilsRepeatedBehaviorDeliberate2024}; 1 case from Australia \cite{Apikotoa_2022f}, Bahrain \cite{Ali_2020f}, Croatia \cite{Trgo_2012f}, Ecuador \cite{DelgadoSalazar_2020c}, Egypt \cite{Ali_2022g}, Ethiopia \cite{Mesfin_2022a}, Germany \cite{teWildt_2010}, Greece \cite{Sakellaridis_2008f}, Hungary \cite{Csaky_1998e}, Iraq \cite{Al-Faham_2020k}, Israel \cite{Goldman_1998f}, Italy \cite{Riva_2018j}, Japan \cite{Ohno_2005}, Nepal \cite{Thapa_2019f}, Netherlands \cite{Benoist_2019e}, Oman \cite{AlShaaibi_2021b}, Pakistan \cite{Yasin_2009}, Portugal \cite{Peixoto_2017f}, Qatar \cite{Ali_2017}, Saudi Arabia \cite{Sultan_2024f}, South Africa \cite{Sobnach_2011f}, Sweden \cite{Naji_2012f}, Switzerland \cite{Wildhaber_2005}, and Taiwan \cite{Chang_2017f}. \paragraph*{Gender} 43 cases (60\%) were male \cite{Akay_2015f, Al-Faham_2020k, Alao_2006i, Ali_2017, Ali_2022g, Apikotoa_2022f, Atayan_2016, Benoist_2019e, Berry_2021e, Bhumi_2024f, CamachoDorado_2018, Csaky_1998e, Emamhadi_2018, Farhadi_2024h, Fry_2010, Gardner_2017h, Guinan_2019f, Jehangir_2019h, Jin_2023, Kobiela_2015, Kumar_2001, Kumar_2019f, Liu_2005, Losanoff_1996, Losanoff_1997e, Mesfin_2022a, Misra_2013, Qureshi_2016, Riva_2018j, Sobnach_2011f, Tammana_2012j, Tanrikulu_2015e, Tay_2004, Thapa_2019f, Trgo_2012f, Wadhwa_2015e, Yasin_2009, teWildt_2010}, 28 cases (39\%) were female \cite{AlShaaibi_2021b, Ali_2020f, Ataya_2013, Beecroft_1998, Bhasin_2014, Bhattacharjee_2008, Cauchi_2002, Chang_2017f, Cox_2007, DelgadoSalazar_2020c, DivsalarP._2023a, Goldman_1998f, Hardy_2023g, Kar_2015, Kariholu_2008, Kerestes_2019, Li_2013, Naji_2012f, Ohno_2005, Peixoto_2017f, Sakellaridis_2008f, Sultan_2024f, Tupesis_2004f, Wildhaber_2005, Wnęk_2015f, Yildiz_2016e}, 1 case (1\%) had no gender recorded \cite{fjbuilsRepeatedBehaviorDeliberate2024}. \paragraph*{Age Group} 25 cases (35\%) were between 26 and 40 years of age \cite{Alao_2006i, Ali_2022g, Apikotoa_2022f, Ataya_2013, Benoist_2019e, Bhasin_2014, Chang_2017f, Cox_2007, DelgadoSalazar_2020c, Farhadi_2024h, Fry_2010, Gardner_2017h, Guinan_2019f, Jin_2023, Kumar_2019f, Losanoff_1996, Misra_2013, Qureshi_2016, Riva_2018j, Sakellaridis_2008f, Tammana_2012j, Trgo_2012f, Wnęk_2015f, Yildiz_2016e, fjbuilsRepeatedBehaviorDeliberate2024}, 18 cases (25\%) were between 18 and 25 years of age \cite{Akay_2015f, Ali_2017, Atayan_2016, Bhattacharjee_2008, Csaky_1998e, Kar_2015, Kariholu_2008, Kobiela_2015, Losanoff_1996, Losanoff_1997e, Mesfin_2022a, Peixoto_2017f, Sobnach_2011f, Tupesis_2004f, Yasin_2009}, 13 cases (18\%) were under 18 years of age \cite{AlShaaibi_2021b, Ali_2020f, Cauchi_2002, DivsalarP._2023a, Goldman_1998f, Liu_2005, Naji_2012f, Ohno_2005, Tanrikulu_2015e, Tay_2004, Wildhaber_2005}, 11 cases (15\%) were between 41 and 60 years of age \cite{Al-Faham_2020k, Bhumi_2024f, CamachoDorado_2018, Emamhadi_2018, Hardy_2023g, Jehangir_2019h, Kumar_2001, Sultan_2024f, Thapa_2019f, Wadhwa_2015e, teWildt_2010}, 3 cases (4\%) were over 60 years of age \cite{Beecroft_1998, Kerestes_2019, Li_2013}, 2 cases (3\%) had no age documented \cite{Berry_2021e}. \paragraph*{Population} 36 cases (50\%) had a psychiatric history \cite{AlShaaibi_2021b, Alao_2006i, Ali_2020f, Apikotoa_2022f, Ataya_2013, Atayan_2016, Beecroft_1998, CamachoDorado_2018, Chang_2017f, DelgadoSalazar_2020c, DivsalarP._2023a, Farhadi_2024h, Fry_2010, Guinan_2019f, Hardy_2023g, Jehangir_2019h, Jin_2023, Kar_2015, Kerestes_2019, Kobiela_2015, Kumar_2001, Kumar_2019f, Liu_2005, Mesfin_2022a, Misra_2013, Ohno_2005, Peixoto_2017f, Sakellaridis_2008f, Sultan_2024f, Tammana_2012j, Tanrikulu_2015e, Yildiz_2016e, fjbuilsRepeatedBehaviorDeliberate2024, teWildt_2010}, 19 cases (26\%) had ingested previously \cite{Alao_2006i, Apikotoa_2022f, Berry_2021e, Bhattacharjee_2008, Csaky_1998e, DivsalarP._2023a, Emamhadi_2018, Guinan_2019f, Jehangir_2019h, Jin_2023, Liu_2005, Sakellaridis_2008f, Tanrikulu_2015e, Thapa_2019f, Yildiz_2016e, fjbuilsRepeatedBehaviorDeliberate2024, teWildt_2010}, 12 cases (17\%) were detained persons \cite{Alao_2006i, Ali_2022g, Apikotoa_2022f, Losanoff_1996, Losanoff_1997e, Qureshi_2016, Tammana_2012j, Trgo_2012f}, 7 cases (10\%) were severely disabled \cite{Atayan_2016, Kerestes_2019, Liu_2005, Ohno_2005, Peixoto_2017f, Yildiz_2016e, teWildt_2010}, 4 cases (6\%) were psychiatric inpatients \cite{DivsalarP._2023a, fjbuilsRepeatedBehaviorDeliberate2024, teWildt_2010}, 3 cases (4\%) were under the influence of alcohol \cite{Benoist_2019e, Csaky_1998e, Thapa_2019f}, 2 cases (3\%) were displaced people \cite{Akay_2015f, Gardner_2017h}. \paragraph*{Motivation} 34 cases (47\%) had a psychiatric motivation \cite{Al-Faham_2020k, Alao_2006i, Ali_2020f, Apikotoa_2022f, Ataya_2013, Atayan_2016, Bhasin_2014, Bhattacharjee_2008, DelgadoSalazar_2020c, DivsalarP._2023a, Emamhadi_2018, Farhadi_2024h, Guinan_2019f, Hardy_2023g, Jehangir_2019h, Jin_2023, Kar_2015, Kariholu_2008, Kerestes_2019, Kobiela_2015, Kumar_2001, Kumar_2019f, Li_2013, Liu_2005, Misra_2013, Ohno_2005, Sakellaridis_2008f, Sultan_2024f, Tammana_2012j, Tanrikulu_2015e, Yasin_2009, teWildt_2010}, 21 cases (29\%) were motivated by self-harm intention \cite{Al-Faham_2020k, AlShaaibi_2021b, Alao_2006i, Ali_2017, CamachoDorado_2018, Chang_2017f, Cox_2007, Csaky_1998e, Fry_2010, Li_2013, Losanoff_1996, Losanoff_1997e, Mesfin_2022a, Sakellaridis_2008f, Tammana_2012j, Tanrikulu_2015e, fjbuilsRepeatedBehaviorDeliberate2024}, 17 cases (24\%) had a psychosocial motivation \cite{Akay_2015f, Benoist_2019e, Bhattacharjee_2008, Cauchi_2002, Goldman_1998f, Hardy_2023g, Kobiela_2015, Li_2013, Naji_2012f, Qureshi_2016, Riva_2018j, Sobnach_2011f, Tay_2004, Thapa_2019f, Tupesis_2004f, Wildhaber_2005, Wnęk_2015f}, 9 cases (12\%) were motivated by protest \cite{Bhumi_2024f, Gardner_2017h, Losanoff_1996, Losanoff_1997e, Tupesis_2004f}, 9 cases (12\%) had another documented motivation \cite{Ali_2020f, Ali_2022g, Emamhadi_2018, Guinan_2019f, Peixoto_2017f, Sakellaridis_2008f, Trgo_2012f, Wadhwa_2015e, Yildiz_2016e}. \paragraph*{Object Characteristics} 51 cases (71\%) ingested a large diameter object (\textgreater{}2.5cm) \cite{Akay_2015f, Al-Faham_2020k, AlShaaibi_2021b, Alao_2006i, Ali_2017, Ali_2022g, Apikotoa_2022f, Atayan_2016, Berry_2021e, Bhasin_2014, CamachoDorado_2018, Cauchi_2002, Chang_2017f, Cox_2007, Csaky_1998e, DivsalarP._2023a, Emamhadi_2018, Gardner_2017h, Guinan_2019f, Jehangir_2019h, Jin_2023, Kariholu_2008, Kerestes_2019, Kobiela_2015, Kumar_2001, Kumar_2019f, Losanoff_1996, Losanoff_1997e, Mesfin_2022a, Misra_2013, Naji_2012f, Ohno_2005, Peixoto_2017f, Qureshi_2016, Riva_2018j, Sakellaridis_2008f, Sultan_2024f, Tanrikulu_2015e, Thapa_2019f, Trgo_2012f, Wnęk_2015f, Yildiz_2016e, fjbuilsRepeatedBehaviorDeliberate2024, teWildt_2010}, 44 cases (61\%) ingested multiple objects \cite{Ali_2020f, Apikotoa_2022f, Ataya_2013, Atayan_2016, Beecroft_1998, Bhattacharjee_2008, Bhumi_2024f, CamachoDorado_2018, Cauchi_2002, Emamhadi_2018, Farhadi_2024h, Fry_2010, Goldman_1998f, Guinan_2019f, Hardy_2023g, Jehangir_2019h, Jin_2023, Kar_2015, Kariholu_2008, Kobiela_2015, Kumar_2001, Kumar_2019f, Li_2013, Liu_2005, Losanoff_1996, Mesfin_2022a, Misra_2013, Naji_2012f, Ohno_2005, Sobnach_2011f, Sultan_2024f, Tammana_2012j, Tanrikulu_2015e, Tay_2004, Thapa_2019f, Wadhwa_2015e, Wildhaber_2005, Yasin_2009, fjbuilsRepeatedBehaviorDeliberate2024, teWildt_2010}, 34 cases (47\%) ingested a sharp object \cite{AlShaaibi_2021b, Alao_2006i, Apikotoa_2022f, Ataya_2013, Benoist_2019e, Bhasin_2014, Bhattacharjee_2008, CamachoDorado_2018, Csaky_1998e, DelgadoSalazar_2020c, DivsalarP._2023a, Emamhadi_2018, Farhadi_2024h, Fry_2010, Guinan_2019f, Hardy_2023g, Jehangir_2019h, Jin_2023, Kariholu_2008, Kobiela_2015, Kumar_2019f, Losanoff_1996, Losanoff_1997e, Mesfin_2022a, Misra_2013, Sobnach_2011f, Yasin_2009, teWildt_2010}, 32 cases (44\%) ingested a long object (\textgreater{}5cm) \cite{Al-Faham_2020k, AlShaaibi_2021b, Ali_2017, Ali_2022g, Atayan_2016, Bhasin_2014, CamachoDorado_2018, Chang_2017f, Cox_2007, Csaky_1998e, DivsalarP._2023a, Emamhadi_2018, Fry_2010, Gardner_2017h, Jin_2023, Kariholu_2008, Kerestes_2019, Kobiela_2015, Kumar_2019f, Mesfin_2022a, Misra_2013, Ohno_2005, Qureshi_2016, Sakellaridis_2008f, Sultan_2024f, Thapa_2019f, Trgo_2012f, Yasin_2009, Yildiz_2016e, teWildt_2010}, 9 cases (12\%) ingested a magnet \cite{Ali_2020f, Bhumi_2024f, Cauchi_2002, Liu_2005, Naji_2012f, Ohno_2005, Tanrikulu_2015e, Tay_2004, Wildhaber_2005}, 2 cases (3\%) ingested a button battery \cite{Berry_2021e, Bhumi_2024f}. \paragraph*{Outcomes} 48 cases (67\%) experienced a complication \cite{Ali_2017, Ali_2020f, Apikotoa_2022f, Atayan_2016, Beecroft_1998, Benoist_2019e, Berry_2021e, Bhasin_2014, Bhumi_2024f, CamachoDorado_2018, Cauchi_2002, Cox_2007, Csaky_1998e, DelgadoSalazar_2020c, DivsalarP._2023a, Emamhadi_2018, Farhadi_2024h, Fry_2010, Gardner_2017h, Goldman_1998f, Jin_2023, Kariholu_2008, Kerestes_2019, Kobiela_2015, Kumar_2001, Kumar_2019f, Liu_2005, Losanoff_1996, Mesfin_2022a, Misra_2013, Naji_2012f, Ohno_2005, Sakellaridis_2008f, Sobnach_2011f, Sultan_2024f, Tanrikulu_2015e, Tay_2004, Thapa_2019f, Trgo_2012f, Tupesis_2004f, Wildhaber_2005, Wnęk_2015f, Yasin_2009, Yildiz_2016e}, 44 cases (61\%) underwent surgery \cite{Al-Faham_2020k, AlShaaibi_2021b, Alao_2006i, Ali_2017, Ali_2020f, Atayan_2016, Beecroft_1998, Bhasin_2014, CamachoDorado_2018, Cauchi_2002, Chang_2017f, Cox_2007, Csaky_1998e, DelgadoSalazar_2020c, DivsalarP._2023a, Farhadi_2024h, Fry_2010, Gardner_2017h, Jin_2023, Kariholu_2008, Kerestes_2019, Kobiela_2015, Kumar_2019f, Liu_2005, Losanoff_1996, Losanoff_1997e, Mesfin_2022a, Misra_2013, Naji_2012f, Sobnach_2011f, Tanrikulu_2015e, Tay_2004, Thapa_2019f, Tupesis_2004f, Wildhaber_2005, Wnęk_2015f, Yasin_2009, Yildiz_2016e, fjbuilsRepeatedBehaviorDeliberate2024}, 31 cases (43\%) underwent endoscopy \cite{Akay_2015f, Ali_2022g, Apikotoa_2022f, Atayan_2016, Benoist_2019e, Berry_2021e, Bhasin_2014, Bhumi_2024f, CamachoDorado_2018, Chang_2017f, DelgadoSalazar_2020c, Gardner_2017h, Guinan_2019f, Hardy_2023g, Jehangir_2019h, Kariholu_2008, Li_2013, Liu_2005, Ohno_2005, Peixoto_2017f, Qureshi_2016, Riva_2018j, Sakellaridis_2008f, Sultan_2024f, Tammana_2012j, Tanrikulu_2015e, Trgo_2012f, Wadhwa_2015e, Wnęk_2015f, teWildt_2010}, 7 cases (10\%) were managed conservatively \cite{Ataya_2013, Bhattacharjee_2008, DivsalarP._2023a, Emamhadi_2018, Goldman_1998f, Kar_2015, Kumar_2001}, 2 cases (3\%) died \cite{Emamhadi_2018, Kumar_2001}. All 90 were male gender. 90 cases (100\%) were detained at the time of ingestion \cite{Elghali_2016, Karp_1991b, Lee_2007}, 88 cases (98\%) were intentional ingestions \cite{Elghali_2016, Karp_1991b, Lee_2007}, 30 cases (33\%) had a psychiatric history documented \cite{Elghali_2016, Karp_1991b, Lee_2007}, 2 cases (2\%) had a history of prior ingestion \cite{Elghali_2016}. No cases were reported for were psychiatric inpatients, were displaced people, were under the influence of alcohol at the time of ingestion, and had a severe disability history.
\paragraph*{Motivation}  70 cases (78\%) reported protest motivation \cite{Elghali_2016, Karp_1991b, Lee_2007}, 12 cases (13\%) reported psychiatric motivation \cite{Karp_1991b}, 6 cases (7\%) reported self-harm motivation \cite{Elghali_2016, Karp_1991b}. No cases were reported for psychosocial motivation and other motivation.
\paragraph*{Object Characteristics}  68 cases (76\%) involved sharp object ingestion \cite{Elghali_2016, Karp_1991b, Lee_2007}, 32 cases (36\%) involved long (\textgreater 5cm) object ingestion \cite{Lee_2007}, 25 cases (28\%) involved ingestion of multiple objects \cite{Elghali_2016, Lee_2007}. No cases were reported for button battery ingestion, magnet ingestion, and involved large diameter (\textgreater 2.5cm) object ingestion.
\paragraph*{Outcomes}  47 cases (52\%) underwent endoscopic intervention \cite{Elghali_2016, Lee_2007}, 29 cases (32\%) were managed conservatively \cite{Elghali_2016, Karp_1991b}, 15 cases (17\%) underwent surgical intervention \cite{Elghali_2016, Karp_1991b, Lee_2007}, 6 cases (7\%) reported complications \cite{Lee_2007}, 1 case (1\%) died \cite{Elghali_2016}.
\paragraph*{Geographical Location}Cases were recorded in 33 countries: 13 cases from USA \cite{Alao_2006i, Ataya_2013, Bhumi_2024f, Fry_2010, Guinan_2019f, Hardy_2023g, Jehangir_2019h, Kerestes_2019, Kumar_2001, Liu_2005, Tammana_2012j, Tay_2004, Tupesis_2004f}; 7 cases from India \cite{Bhasin_2014, Bhattacharjee_2008, Kar_2015, Kariholu_2008, Kumar_2019f, Misra_2013, Wadhwa_2015e} and UK \cite{Beecroft_1998, Berry_2021e, Cauchi_2002, Cox_2007, Gardner_2017h, Qureshi_2016}; 6 cases from Bulgaria \cite{Losanoff_1996, Losanoff_1997e}; 5 cases from Iran \cite{DivsalarP._2023a, Emamhadi_2018, Farhadi_2024h}; 4 cases from Turkey \cite{Akay_2015f, Atayan_2016, Tanrikulu_2015e, Yildiz_2016e}; 2 cases from China \cite{Jin_2023, Li_2013}, Poland \cite{Kobiela_2015, Wnęk_2015f}, and Spain \cite{CamachoDorado_2018, fjbuilsRepeatedBehaviorDeliberate2024}; 1 case from Australia \cite{Apikotoa_2022f}, Bahrain \cite{Ali_2020f}, Croatia \cite{Trgo_2012f}, Ecuador \cite{DelgadoSalazar_2020c}, Egypt \cite{Ali_2022g}, Ethiopia \cite{Mesfin_2022a}, Germany \cite{teWildt_2010}, Greece \cite{Sakellaridis_2008f}, Hungary \cite{Csaky_1998e}, Iraq \cite{Al-Faham_2020k}, Israel \cite{Goldman_1998f}, Italy \cite{Riva_2018j}, Japan \cite{Ohno_2005}, Nepal \cite{Thapa_2019f}, Netherlands \cite{Benoist_2019e}, Oman \cite{AlShaaibi_2021b}, Pakistan \cite{Yasin_2009}, Portugal \cite{Peixoto_2017f}, Qatar \cite{Ali_2017}, Saudi Arabia \cite{Sultan_2024f}, South Africa \cite{Sobnach_2011f}, Sweden \cite{Naji_2012f}, Switzerland \cite{Wildhaber_2005}, and Taiwan \cite{Chang_2017f}. \paragraph*{Gender} 43 cases (60\%) were male \cite{Akay_2015f, Al-Faham_2020k, Alao_2006i, Ali_2017, Ali_2022g, Apikotoa_2022f, Atayan_2016, Benoist_2019e, Berry_2021e, Bhumi_2024f, CamachoDorado_2018, Csaky_1998e, Emamhadi_2018, Farhadi_2024h, Fry_2010, Gardner_2017h, Guinan_2019f, Jehangir_2019h, Jin_2023, Kobiela_2015, Kumar_2001, Kumar_2019f, Liu_2005, Losanoff_1996, Losanoff_1997e, Mesfin_2022a, Misra_2013, Qureshi_2016, Riva_2018j, Sobnach_2011f, Tammana_2012j, Tanrikulu_2015e, Tay_2004, Thapa_2019f, Trgo_2012f, Wadhwa_2015e, Yasin_2009, teWildt_2010}, 28 cases (39\%) were female \cite{AlShaaibi_2021b, Ali_2020f, Ataya_2013, Beecroft_1998, Bhasin_2014, Bhattacharjee_2008, Cauchi_2002, Chang_2017f, Cox_2007, DelgadoSalazar_2020c, DivsalarP._2023a, Goldman_1998f, Hardy_2023g, Kar_2015, Kariholu_2008, Kerestes_2019, Li_2013, Naji_2012f, Ohno_2005, Peixoto_2017f, Sakellaridis_2008f, Sultan_2024f, Tupesis_2004f, Wildhaber_2005, Wnęk_2015f, Yildiz_2016e}, 1 case (1\%) had no gender recorded \cite{fjbuilsRepeatedBehaviorDeliberate2024}. \paragraph*{Age Group} 25 cases (35\%) were between 26 and 40 years of age \cite{Alao_2006i, Ali_2022g, Apikotoa_2022f, Ataya_2013, Benoist_2019e, Bhasin_2014, Chang_2017f, Cox_2007, DelgadoSalazar_2020c, Farhadi_2024h, Fry_2010, Gardner_2017h, Guinan_2019f, Jin_2023, Kumar_2019f, Losanoff_1996, Misra_2013, Qureshi_2016, Riva_2018j, Sakellaridis_2008f, Tammana_2012j, Trgo_2012f, Wnęk_2015f, Yildiz_2016e, fjbuilsRepeatedBehaviorDeliberate2024}, 18 cases (25\%) were between 18 and 25 years of age \cite{Akay_2015f, Ali_2017, Atayan_2016, Bhattacharjee_2008, Csaky_1998e, Kar_2015, Kariholu_2008, Kobiela_2015, Losanoff_1996, Losanoff_1997e, Mesfin_2022a, Peixoto_2017f, Sobnach_2011f, Tupesis_2004f, Yasin_2009}, 13 cases (18\%) were under 18 years of age \cite{AlShaaibi_2021b, Ali_2020f, Cauchi_2002, DivsalarP._2023a, Goldman_1998f, Liu_2005, Naji_2012f, Ohno_2005, Tanrikulu_2015e, Tay_2004, Wildhaber_2005}, 11 cases (15\%) were between 41 and 60 years of age \cite{Al-Faham_2020k, Bhumi_2024f, CamachoDorado_2018, Emamhadi_2018, Hardy_2023g, Jehangir_2019h, Kumar_2001, Sultan_2024f, Thapa_2019f, Wadhwa_2015e, teWildt_2010}, 3 cases (4\%) were over 60 years of age \cite{Beecroft_1998, Kerestes_2019, Li_2013}, 2 cases (3\%) had no age documented \cite{Berry_2021e}. \paragraph*{Population} 36 cases (50\%) had a psychiatric history \cite{AlShaaibi_2021b, Alao_2006i, Ali_2020f, Apikotoa_2022f, Ataya_2013, Atayan_2016, Beecroft_1998, CamachoDorado_2018, Chang_2017f, DelgadoSalazar_2020c, DivsalarP._2023a, Farhadi_2024h, Fry_2010, Guinan_2019f, Hardy_2023g, Jehangir_2019h, Jin_2023, Kar_2015, Kerestes_2019, Kobiela_2015, Kumar_2001, Kumar_2019f, Liu_2005, Mesfin_2022a, Misra_2013, Ohno_2005, Peixoto_2017f, Sakellaridis_2008f, Sultan_2024f, Tammana_2012j, Tanrikulu_2015e, Yildiz_2016e, fjbuilsRepeatedBehaviorDeliberate2024, teWildt_2010}, 19 cases (26\%) had ingested previously \cite{Alao_2006i, Apikotoa_2022f, Berry_2021e, Bhattacharjee_2008, Csaky_1998e, DivsalarP._2023a, Emamhadi_2018, Guinan_2019f, Jehangir_2019h, Jin_2023, Liu_2005, Sakellaridis_2008f, Tanrikulu_2015e, Thapa_2019f, Yildiz_2016e, fjbuilsRepeatedBehaviorDeliberate2024, teWildt_2010}, 12 cases (17\%) were detained persons \cite{Alao_2006i, Ali_2022g, Apikotoa_2022f, Losanoff_1996, Losanoff_1997e, Qureshi_2016, Tammana_2012j, Trgo_2012f}, 7 cases (10\%) were severely disabled \cite{Atayan_2016, Kerestes_2019, Liu_2005, Ohno_2005, Peixoto_2017f, Yildiz_2016e, teWildt_2010}, 4 cases (6\%) were psychiatric inpatients \cite{DivsalarP._2023a, fjbuilsRepeatedBehaviorDeliberate2024, teWildt_2010}, 3 cases (4\%) were under the influence of alcohol \cite{Benoist_2019e, Csaky_1998e, Thapa_2019f}, 2 cases (3\%) were displaced people \cite{Akay_2015f, Gardner_2017h}. \paragraph*{Motivation} 34 cases (47\%) had a psychiatric motivation \cite{Al-Faham_2020k, Alao_2006i, Ali_2020f, Apikotoa_2022f, Ataya_2013, Atayan_2016, Bhasin_2014, Bhattacharjee_2008, DelgadoSalazar_2020c, DivsalarP._2023a, Emamhadi_2018, Farhadi_2024h, Guinan_2019f, Hardy_2023g, Jehangir_2019h, Jin_2023, Kar_2015, Kariholu_2008, Kerestes_2019, Kobiela_2015, Kumar_2001, Kumar_2019f, Li_2013, Liu_2005, Misra_2013, Ohno_2005, Sakellaridis_2008f, Sultan_2024f, Tammana_2012j, Tanrikulu_2015e, Yasin_2009, teWildt_2010}, 21 cases (29\%) were motivated by self-harm intention \cite{Al-Faham_2020k, AlShaaibi_2021b, Alao_2006i, Ali_2017, CamachoDorado_2018, Chang_2017f, Cox_2007, Csaky_1998e, Fry_2010, Li_2013, Losanoff_1996, Losanoff_1997e, Mesfin_2022a, Sakellaridis_2008f, Tammana_2012j, Tanrikulu_2015e, fjbuilsRepeatedBehaviorDeliberate2024}, 17 cases (24\%) had a psychosocial motivation \cite{Akay_2015f, Benoist_2019e, Bhattacharjee_2008, Cauchi_2002, Goldman_1998f, Hardy_2023g, Kobiela_2015, Li_2013, Naji_2012f, Qureshi_2016, Riva_2018j, Sobnach_2011f, Tay_2004, Thapa_2019f, Tupesis_2004f, Wildhaber_2005, Wnęk_2015f}, 9 cases (12\%) were motivated by protest \cite{Bhumi_2024f, Gardner_2017h, Losanoff_1996, Losanoff_1997e, Tupesis_2004f}, 9 cases (12\%) had another documented motivation \cite{Ali_2020f, Ali_2022g, Emamhadi_2018, Guinan_2019f, Peixoto_2017f, Sakellaridis_2008f, Trgo_2012f, Wadhwa_2015e, Yildiz_2016e}. \paragraph*{Object Characteristics} 51 cases (71\%) ingested a large diameter object (\textgreater{}2.5cm) \cite{Akay_2015f, Al-Faham_2020k, AlShaaibi_2021b, Alao_2006i, Ali_2017, Ali_2022g, Apikotoa_2022f, Atayan_2016, Berry_2021e, Bhasin_2014, CamachoDorado_2018, Cauchi_2002, Chang_2017f, Cox_2007, Csaky_1998e, DivsalarP._2023a, Emamhadi_2018, Gardner_2017h, Guinan_2019f, Jehangir_2019h, Jin_2023, Kariholu_2008, Kerestes_2019, Kobiela_2015, Kumar_2001, Kumar_2019f, Losanoff_1996, Losanoff_1997e, Mesfin_2022a, Misra_2013, Naji_2012f, Ohno_2005, Peixoto_2017f, Qureshi_2016, Riva_2018j, Sakellaridis_2008f, Sultan_2024f, Tanrikulu_2015e, Thapa_2019f, Trgo_2012f, Wnęk_2015f, Yildiz_2016e, fjbuilsRepeatedBehaviorDeliberate2024, teWildt_2010}, 44 cases (61\%) ingested multiple objects \cite{Ali_2020f, Apikotoa_2022f, Ataya_2013, Atayan_2016, Beecroft_1998, Bhattacharjee_2008, Bhumi_2024f, CamachoDorado_2018, Cauchi_2002, Emamhadi_2018, Farhadi_2024h, Fry_2010, Goldman_1998f, Guinan_2019f, Hardy_2023g, Jehangir_2019h, Jin_2023, Kar_2015, Kariholu_2008, Kobiela_2015, Kumar_2001, Kumar_2019f, Li_2013, Liu_2005, Losanoff_1996, Mesfin_2022a, Misra_2013, Naji_2012f, Ohno_2005, Sobnach_2011f, Sultan_2024f, Tammana_2012j, Tanrikulu_2015e, Tay_2004, Thapa_2019f, Wadhwa_2015e, Wildhaber_2005, Yasin_2009, fjbuilsRepeatedBehaviorDeliberate2024, teWildt_2010}, 34 cases (47\%) ingested a sharp object \cite{AlShaaibi_2021b, Alao_2006i, Apikotoa_2022f, Ataya_2013, Benoist_2019e, Bhasin_2014, Bhattacharjee_2008, CamachoDorado_2018, Csaky_1998e, DelgadoSalazar_2020c, DivsalarP._2023a, Emamhadi_2018, Farhadi_2024h, Fry_2010, Guinan_2019f, Hardy_2023g, Jehangir_2019h, Jin_2023, Kariholu_2008, Kobiela_2015, Kumar_2019f, Losanoff_1996, Losanoff_1997e, Mesfin_2022a, Misra_2013, Sobnach_2011f, Yasin_2009, teWildt_2010}, 32 cases (44\%) ingested a long object (\textgreater{}5cm) \cite{Al-Faham_2020k, AlShaaibi_2021b, Ali_2017, Ali_2022g, Atayan_2016, Bhasin_2014, CamachoDorado_2018, Chang_2017f, Cox_2007, Csaky_1998e, DivsalarP._2023a, Emamhadi_2018, Fry_2010, Gardner_2017h, Jin_2023, Kariholu_2008, Kerestes_2019, Kobiela_2015, Kumar_2019f, Mesfin_2022a, Misra_2013, Ohno_2005, Qureshi_2016, Sakellaridis_2008f, Sultan_2024f, Thapa_2019f, Trgo_2012f, Yasin_2009, Yildiz_2016e, teWildt_2010}, 9 cases (12\%) ingested a magnet \cite{Ali_2020f, Bhumi_2024f, Cauchi_2002, Liu_2005, Naji_2012f, Ohno_2005, Tanrikulu_2015e, Tay_2004, Wildhaber_2005}, 2 cases (3\%) ingested a button battery \cite{Berry_2021e, Bhumi_2024f}. \paragraph*{Outcomes} 48 cases (67\%) experienced a complication \cite{Ali_2017, Ali_2020f, Apikotoa_2022f, Atayan_2016, Beecroft_1998, Benoist_2019e, Berry_2021e, Bhasin_2014, Bhumi_2024f, CamachoDorado_2018, Cauchi_2002, Cox_2007, Csaky_1998e, DelgadoSalazar_2020c, DivsalarP._2023a, Emamhadi_2018, Farhadi_2024h, Fry_2010, Gardner_2017h, Goldman_1998f, Jin_2023, Kariholu_2008, Kerestes_2019, Kobiela_2015, Kumar_2001, Kumar_2019f, Liu_2005, Losanoff_1996, Mesfin_2022a, Misra_2013, Naji_2012f, Ohno_2005, Sakellaridis_2008f, Sobnach_2011f, Sultan_2024f, Tanrikulu_2015e, Tay_2004, Thapa_2019f, Trgo_2012f, Tupesis_2004f, Wildhaber_2005, Wnęk_2015f, Yasin_2009, Yildiz_2016e}, 44 cases (61\%) underwent surgery \cite{Al-Faham_2020k, AlShaaibi_2021b, Alao_2006i, Ali_2017, Ali_2020f, Atayan_2016, Beecroft_1998, Bhasin_2014, CamachoDorado_2018, Cauchi_2002, Chang_2017f, Cox_2007, Csaky_1998e, DelgadoSalazar_2020c, DivsalarP._2023a, Farhadi_2024h, Fry_2010, Gardner_2017h, Jin_2023, Kariholu_2008, Kerestes_2019, Kobiela_2015, Kumar_2019f, Liu_2005, Losanoff_1996, Losanoff_1997e, Mesfin_2022a, Misra_2013, Naji_2012f, Sobnach_2011f, Tanrikulu_2015e, Tay_2004, Thapa_2019f, Tupesis_2004f, Wildhaber_2005, Wnęk_2015f, Yasin_2009, Yildiz_2016e, fjbuilsRepeatedBehaviorDeliberate2024}, 31 cases (43\%) underwent endoscopy \cite{Akay_2015f, Ali_2022g, Apikotoa_2022f, Atayan_2016, Benoist_2019e, Berry_2021e, Bhasin_2014, Bhumi_2024f, CamachoDorado_2018, Chang_2017f, DelgadoSalazar_2020c, Gardner_2017h, Guinan_2019f, Hardy_2023g, Jehangir_2019h, Kariholu_2008, Li_2013, Liu_2005, Ohno_2005, Peixoto_2017f, Qureshi_2016, Riva_2018j, Sakellaridis_2008f, Sultan_2024f, Tammana_2012j, Tanrikulu_2015e, Trgo_2012f, Wadhwa_2015e, Wnęk_2015f, teWildt_2010}, 7 cases (10\%) were managed conservatively \cite{Ataya_2013, Bhattacharjee_2008, DivsalarP._2023a, Emamhadi_2018, Goldman_1998f, Kar_2015, Kumar_2001}, 2 cases (3\%) died \cite{Emamhadi_2018, Kumar_2001}. All 90 were male gender. 90 cases (100\%) were detained at the time of ingestion \cite{Elghali_2016, Karp_1991b, Lee_2007}, 88 cases (98\%) were intentional ingestions \cite{Elghali_2016, Karp_1991b, Lee_2007}, 30 cases (33\%) had a psychiatric history documented \cite{Elghali_2016, Karp_1991b, Lee_2007}, 2 cases (2\%) had a history of prior ingestion \cite{Elghali_2016}. No cases were reported for were psychiatric inpatients, were displaced people, were under the influence of alcohol at the time of ingestion, and had a severe disability history.
\paragraph*{Motivation}  70 cases (78\%) reported protest motivation \cite{Elghali_2016, Karp_1991b, Lee_2007}, 12 cases (13\%) reported psychiatric motivation \cite{Karp_1991b}, 6 cases (7\%) reported self-harm motivation \cite{Elghali_2016, Karp_1991b}. No cases were reported for psychosocial motivation and other motivation.
\paragraph*{Object Characteristics}  68 cases (76\%) involved sharp object ingestion \cite{Elghali_2016, Karp_1991b, Lee_2007}, 32 cases (36\%) involved long (\textgreater 5cm) object ingestion \cite{Lee_2007}, 25 cases (28\%) involved ingestion of multiple objects \cite{Elghali_2016, Lee_2007}. No cases were reported for button battery ingestion, magnet ingestion, and involved large diameter (\textgreater 2.5cm) object ingestion.
\paragraph*{Outcomes}  47 cases (52\%) underwent endoscopic intervention \cite{Elghali_2016, Lee_2007}, 29 cases (32\%) were managed conservatively \cite{Elghali_2016, Karp_1991b}, 15 cases (17\%) underwent surgical intervention \cite{Elghali_2016, Karp_1991b, Lee_2007}, 6 cases (7\%) reported complications \cite{Lee_2007}, 1 case (1\%) died \cite{Elghali_2016}.
\paragraph*{Geographical Location}Cases were recorded in 33 countries: 13 cases from USA \cite{Alao_2006i, Ataya_2013, Bhumi_2024f, Fry_2010, Guinan_2019f, Hardy_2023g, Jehangir_2019h, Kerestes_2019, Kumar_2001, Liu_2005, Tammana_2012j, Tay_2004, Tupesis_2004f}; 7 cases from India \cite{Bhasin_2014, Bhattacharjee_2008, Kar_2015, Kariholu_2008, Kumar_2019f, Misra_2013, Wadhwa_2015e} and UK \cite{Beecroft_1998, Berry_2021e, Cauchi_2002, Cox_2007, Gardner_2017h, Qureshi_2016}; 6 cases from Bulgaria \cite{Losanoff_1996, Losanoff_1997e}; 5 cases from Iran \cite{DivsalarP._2023a, Emamhadi_2018, Farhadi_2024h}; 4 cases from Turkey \cite{Akay_2015f, Atayan_2016, Tanrikulu_2015e, Yildiz_2016e}; 2 cases from China \cite{Jin_2023, Li_2013}, Poland \cite{Kobiela_2015, Wnęk_2015f}, and Spain \cite{CamachoDorado_2018, fjbuilsRepeatedBehaviorDeliberate2024}; 1 case from Australia \cite{Apikotoa_2022f}, Bahrain \cite{Ali_2020f}, Croatia \cite{Trgo_2012f}, Ecuador \cite{DelgadoSalazar_2020c}, Egypt \cite{Ali_2022g}, Ethiopia \cite{Mesfin_2022a}, Germany \cite{teWildt_2010}, Greece \cite{Sakellaridis_2008f}, Hungary \cite{Csaky_1998e}, Iraq \cite{Al-Faham_2020k}, Israel \cite{Goldman_1998f}, Italy \cite{Riva_2018j}, Japan \cite{Ohno_2005}, Nepal \cite{Thapa_2019f}, Netherlands \cite{Benoist_2019e}, Oman \cite{AlShaaibi_2021b}, Pakistan \cite{Yasin_2009}, Portugal \cite{Peixoto_2017f}, Qatar \cite{Ali_2017}, Saudi Arabia \cite{Sultan_2024f}, South Africa \cite{Sobnach_2011f}, Sweden \cite{Naji_2012f}, Switzerland \cite{Wildhaber_2005}, and Taiwan \cite{Chang_2017f}. \paragraph*{Gender} 43 cases (60\%) were male \cite{Akay_2015f, Al-Faham_2020k, Alao_2006i, Ali_2017, Ali_2022g, Apikotoa_2022f, Atayan_2016, Benoist_2019e, Berry_2021e, Bhumi_2024f, CamachoDorado_2018, Csaky_1998e, Emamhadi_2018, Farhadi_2024h, Fry_2010, Gardner_2017h, Guinan_2019f, Jehangir_2019h, Jin_2023, Kobiela_2015, Kumar_2001, Kumar_2019f, Liu_2005, Losanoff_1996, Losanoff_1997e, Mesfin_2022a, Misra_2013, Qureshi_2016, Riva_2018j, Sobnach_2011f, Tammana_2012j, Tanrikulu_2015e, Tay_2004, Thapa_2019f, Trgo_2012f, Wadhwa_2015e, Yasin_2009, teWildt_2010}, 28 cases (39\%) were female \cite{AlShaaibi_2021b, Ali_2020f, Ataya_2013, Beecroft_1998, Bhasin_2014, Bhattacharjee_2008, Cauchi_2002, Chang_2017f, Cox_2007, DelgadoSalazar_2020c, DivsalarP._2023a, Goldman_1998f, Hardy_2023g, Kar_2015, Kariholu_2008, Kerestes_2019, Li_2013, Naji_2012f, Ohno_2005, Peixoto_2017f, Sakellaridis_2008f, Sultan_2024f, Tupesis_2004f, Wildhaber_2005, Wnęk_2015f, Yildiz_2016e}, 1 case (1\%) had no gender recorded \cite{fjbuilsRepeatedBehaviorDeliberate2024}. \paragraph*{Age Group} 25 cases (35\%) were between 26 and 40 years of age \cite{Alao_2006i, Ali_2022g, Apikotoa_2022f, Ataya_2013, Benoist_2019e, Bhasin_2014, Chang_2017f, Cox_2007, DelgadoSalazar_2020c, Farhadi_2024h, Fry_2010, Gardner_2017h, Guinan_2019f, Jin_2023, Kumar_2019f, Losanoff_1996, Misra_2013, Qureshi_2016, Riva_2018j, Sakellaridis_2008f, Tammana_2012j, Trgo_2012f, Wnęk_2015f, Yildiz_2016e, fjbuilsRepeatedBehaviorDeliberate2024}, 18 cases (25\%) were between 18 and 25 years of age \cite{Akay_2015f, Ali_2017, Atayan_2016, Bhattacharjee_2008, Csaky_1998e, Kar_2015, Kariholu_2008, Kobiela_2015, Losanoff_1996, Losanoff_1997e, Mesfin_2022a, Peixoto_2017f, Sobnach_2011f, Tupesis_2004f, Yasin_2009}, 13 cases (18\%) were under 18 years of age \cite{AlShaaibi_2021b, Ali_2020f, Cauchi_2002, DivsalarP._2023a, Goldman_1998f, Liu_2005, Naji_2012f, Ohno_2005, Tanrikulu_2015e, Tay_2004, Wildhaber_2005}, 11 cases (15\%) were between 41 and 60 years of age \cite{Al-Faham_2020k, Bhumi_2024f, CamachoDorado_2018, Emamhadi_2018, Hardy_2023g, Jehangir_2019h, Kumar_2001, Sultan_2024f, Thapa_2019f, Wadhwa_2015e, teWildt_2010}, 3 cases (4\%) were over 60 years of age \cite{Beecroft_1998, Kerestes_2019, Li_2013}, 2 cases (3\%) had no age documented \cite{Berry_2021e}. \paragraph*{Population} 36 cases (50\%) had a psychiatric history \cite{AlShaaibi_2021b, Alao_2006i, Ali_2020f, Apikotoa_2022f, Ataya_2013, Atayan_2016, Beecroft_1998, CamachoDorado_2018, Chang_2017f, DelgadoSalazar_2020c, DivsalarP._2023a, Farhadi_2024h, Fry_2010, Guinan_2019f, Hardy_2023g, Jehangir_2019h, Jin_2023, Kar_2015, Kerestes_2019, Kobiela_2015, Kumar_2001, Kumar_2019f, Liu_2005, Mesfin_2022a, Misra_2013, Ohno_2005, Peixoto_2017f, Sakellaridis_2008f, Sultan_2024f, Tammana_2012j, Tanrikulu_2015e, Yildiz_2016e, fjbuilsRepeatedBehaviorDeliberate2024, teWildt_2010}, 19 cases (26\%) had ingested previously \cite{Alao_2006i, Apikotoa_2022f, Berry_2021e, Bhattacharjee_2008, Csaky_1998e, DivsalarP._2023a, Emamhadi_2018, Guinan_2019f, Jehangir_2019h, Jin_2023, Liu_2005, Sakellaridis_2008f, Tanrikulu_2015e, Thapa_2019f, Yildiz_2016e, fjbuilsRepeatedBehaviorDeliberate2024, teWildt_2010}, 12 cases (17\%) were detained persons \cite{Alao_2006i, Ali_2022g, Apikotoa_2022f, Losanoff_1996, Losanoff_1997e, Qureshi_2016, Tammana_2012j, Trgo_2012f}, 7 cases (10\%) were severely disabled \cite{Atayan_2016, Kerestes_2019, Liu_2005, Ohno_2005, Peixoto_2017f, Yildiz_2016e, teWildt_2010}, 4 cases (6\%) were psychiatric inpatients \cite{DivsalarP._2023a, fjbuilsRepeatedBehaviorDeliberate2024, teWildt_2010}, 3 cases (4\%) were under the influence of alcohol \cite{Benoist_2019e, Csaky_1998e, Thapa_2019f}, 2 cases (3\%) were displaced people \cite{Akay_2015f, Gardner_2017h}. \paragraph*{Motivation} 34 cases (47\%) had a psychiatric motivation \cite{Al-Faham_2020k, Alao_2006i, Ali_2020f, Apikotoa_2022f, Ataya_2013, Atayan_2016, Bhasin_2014, Bhattacharjee_2008, DelgadoSalazar_2020c, DivsalarP._2023a, Emamhadi_2018, Farhadi_2024h, Guinan_2019f, Hardy_2023g, Jehangir_2019h, Jin_2023, Kar_2015, Kariholu_2008, Kerestes_2019, Kobiela_2015, Kumar_2001, Kumar_2019f, Li_2013, Liu_2005, Misra_2013, Ohno_2005, Sakellaridis_2008f, Sultan_2024f, Tammana_2012j, Tanrikulu_2015e, Yasin_2009, teWildt_2010}, 21 cases (29\%) were motivated by self-harm intention \cite{Al-Faham_2020k, AlShaaibi_2021b, Alao_2006i, Ali_2017, CamachoDorado_2018, Chang_2017f, Cox_2007, Csaky_1998e, Fry_2010, Li_2013, Losanoff_1996, Losanoff_1997e, Mesfin_2022a, Sakellaridis_2008f, Tammana_2012j, Tanrikulu_2015e, fjbuilsRepeatedBehaviorDeliberate2024}, 17 cases (24\%) had a psychosocial motivation \cite{Akay_2015f, Benoist_2019e, Bhattacharjee_2008, Cauchi_2002, Goldman_1998f, Hardy_2023g, Kobiela_2015, Li_2013, Naji_2012f, Qureshi_2016, Riva_2018j, Sobnach_2011f, Tay_2004, Thapa_2019f, Tupesis_2004f, Wildhaber_2005, Wnęk_2015f}, 9 cases (12\%) were motivated by protest \cite{Bhumi_2024f, Gardner_2017h, Losanoff_1996, Losanoff_1997e, Tupesis_2004f}, 9 cases (12\%) had another documented motivation \cite{Ali_2020f, Ali_2022g, Emamhadi_2018, Guinan_2019f, Peixoto_2017f, Sakellaridis_2008f, Trgo_2012f, Wadhwa_2015e, Yildiz_2016e}. \paragraph*{Object Characteristics} 51 cases (71\%) ingested a large diameter object (\textgreater{}2.5cm) \cite{Akay_2015f, Al-Faham_2020k, AlShaaibi_2021b, Alao_2006i, Ali_2017, Ali_2022g, Apikotoa_2022f, Atayan_2016, Berry_2021e, Bhasin_2014, CamachoDorado_2018, Cauchi_2002, Chang_2017f, Cox_2007, Csaky_1998e, DivsalarP._2023a, Emamhadi_2018, Gardner_2017h, Guinan_2019f, Jehangir_2019h, Jin_2023, Kariholu_2008, Kerestes_2019, Kobiela_2015, Kumar_2001, Kumar_2019f, Losanoff_1996, Losanoff_1997e, Mesfin_2022a, Misra_2013, Naji_2012f, Ohno_2005, Peixoto_2017f, Qureshi_2016, Riva_2018j, Sakellaridis_2008f, Sultan_2024f, Tanrikulu_2015e, Thapa_2019f, Trgo_2012f, Wnęk_2015f, Yildiz_2016e, fjbuilsRepeatedBehaviorDeliberate2024, teWildt_2010}, 44 cases (61\%) ingested multiple objects \cite{Ali_2020f, Apikotoa_2022f, Ataya_2013, Atayan_2016, Beecroft_1998, Bhattacharjee_2008, Bhumi_2024f, CamachoDorado_2018, Cauchi_2002, Emamhadi_2018, Farhadi_2024h, Fry_2010, Goldman_1998f, Guinan_2019f, Hardy_2023g, Jehangir_2019h, Jin_2023, Kar_2015, Kariholu_2008, Kobiela_2015, Kumar_2001, Kumar_2019f, Li_2013, Liu_2005, Losanoff_1996, Mesfin_2022a, Misra_2013, Naji_2012f, Ohno_2005, Sobnach_2011f, Sultan_2024f, Tammana_2012j, Tanrikulu_2015e, Tay_2004, Thapa_2019f, Wadhwa_2015e, Wildhaber_2005, Yasin_2009, fjbuilsRepeatedBehaviorDeliberate2024, teWildt_2010}, 34 cases (47\%) ingested a sharp object \cite{AlShaaibi_2021b, Alao_2006i, Apikotoa_2022f, Ataya_2013, Benoist_2019e, Bhasin_2014, Bhattacharjee_2008, CamachoDorado_2018, Csaky_1998e, DelgadoSalazar_2020c, DivsalarP._2023a, Emamhadi_2018, Farhadi_2024h, Fry_2010, Guinan_2019f, Hardy_2023g, Jehangir_2019h, Jin_2023, Kariholu_2008, Kobiela_2015, Kumar_2019f, Losanoff_1996, Losanoff_1997e, Mesfin_2022a, Misra_2013, Sobnach_2011f, Yasin_2009, teWildt_2010}, 32 cases (44\%) ingested a long object (\textgreater{}5cm) \cite{Al-Faham_2020k, AlShaaibi_2021b, Ali_2017, Ali_2022g, Atayan_2016, Bhasin_2014, CamachoDorado_2018, Chang_2017f, Cox_2007, Csaky_1998e, DivsalarP._2023a, Emamhadi_2018, Fry_2010, Gardner_2017h, Jin_2023, Kariholu_2008, Kerestes_2019, Kobiela_2015, Kumar_2019f, Mesfin_2022a, Misra_2013, Ohno_2005, Qureshi_2016, Sakellaridis_2008f, Sultan_2024f, Thapa_2019f, Trgo_2012f, Yasin_2009, Yildiz_2016e, teWildt_2010}, 9 cases (12\%) ingested a magnet \cite{Ali_2020f, Bhumi_2024f, Cauchi_2002, Liu_2005, Naji_2012f, Ohno_2005, Tanrikulu_2015e, Tay_2004, Wildhaber_2005}, 2 cases (3\%) ingested a button battery \cite{Berry_2021e, Bhumi_2024f}. \paragraph*{Outcomes} 48 cases (67\%) experienced a complication \cite{Ali_2017, Ali_2020f, Apikotoa_2022f, Atayan_2016, Beecroft_1998, Benoist_2019e, Berry_2021e, Bhasin_2014, Bhumi_2024f, CamachoDorado_2018, Cauchi_2002, Cox_2007, Csaky_1998e, DelgadoSalazar_2020c, DivsalarP._2023a, Emamhadi_2018, Farhadi_2024h, Fry_2010, Gardner_2017h, Goldman_1998f, Jin_2023, Kariholu_2008, Kerestes_2019, Kobiela_2015, Kumar_2001, Kumar_2019f, Liu_2005, Losanoff_1996, Mesfin_2022a, Misra_2013, Naji_2012f, Ohno_2005, Sakellaridis_2008f, Sobnach_2011f, Sultan_2024f, Tanrikulu_2015e, Tay_2004, Thapa_2019f, Trgo_2012f, Tupesis_2004f, Wildhaber_2005, Wnęk_2015f, Yasin_2009, Yildiz_2016e}, 44 cases (61\%) underwent surgery \cite{Al-Faham_2020k, AlShaaibi_2021b, Alao_2006i, Ali_2017, Ali_2020f, Atayan_2016, Beecroft_1998, Bhasin_2014, CamachoDorado_2018, Cauchi_2002, Chang_2017f, Cox_2007, Csaky_1998e, DelgadoSalazar_2020c, DivsalarP._2023a, Farhadi_2024h, Fry_2010, Gardner_2017h, Jin_2023, Kariholu_2008, Kerestes_2019, Kobiela_2015, Kumar_2019f, Liu_2005, Losanoff_1996, Losanoff_1997e, Mesfin_2022a, Misra_2013, Naji_2012f, Sobnach_2011f, Tanrikulu_2015e, Tay_2004, Thapa_2019f, Tupesis_2004f, Wildhaber_2005, Wnęk_2015f, Yasin_2009, Yildiz_2016e, fjbuilsRepeatedBehaviorDeliberate2024}, 31 cases (43\%) underwent endoscopy \cite{Akay_2015f, Ali_2022g, Apikotoa_2022f, Atayan_2016, Benoist_2019e, Berry_2021e, Bhasin_2014, Bhumi_2024f, CamachoDorado_2018, Chang_2017f, DelgadoSalazar_2020c, Gardner_2017h, Guinan_2019f, Hardy_2023g, Jehangir_2019h, Kariholu_2008, Li_2013, Liu_2005, Ohno_2005, Peixoto_2017f, Qureshi_2016, Riva_2018j, Sakellaridis_2008f, Sultan_2024f, Tammana_2012j, Tanrikulu_2015e, Trgo_2012f, Wadhwa_2015e, Wnęk_2015f, teWildt_2010}, 7 cases (10\%) were managed conservatively \cite{Ataya_2013, Bhattacharjee_2008, DivsalarP._2023a, Emamhadi_2018, Goldman_1998f, Kar_2015, Kumar_2001}, 2 cases (3\%) died \cite{Emamhadi_2018, Kumar_2001}. All 90 were male gender. 90 cases (100\%) were detained at the time of ingestion \cite{Elghali_2016, Karp_1991b, Lee_2007}, 88 cases (98\%) were intentional ingestions \cite{Elghali_2016, Karp_1991b, Lee_2007}, 30 cases (33\%) had a psychiatric history documented \cite{Elghali_2016, Karp_1991b, Lee_2007}, 2 cases (2\%) had a history of prior ingestion \cite{Elghali_2016}. No cases were reported for were psychiatric inpatients, were displaced people, were under the influence of alcohol at the time of ingestion, and had a severe disability history.
\paragraph*{Motivation}  70 cases (78\%) reported protest motivation \cite{Elghali_2016, Karp_1991b, Lee_2007}, 12 cases (13\%) reported psychiatric motivation \cite{Karp_1991b}, 6 cases (7\%) reported self-harm motivation \cite{Elghali_2016, Karp_1991b}. No cases were reported for psychosocial motivation and other motivation.
\paragraph*{Object Characteristics}  68 cases (76\%) involved sharp object ingestion \cite{Elghali_2016, Karp_1991b, Lee_2007}, 32 cases (36\%) involved long (\textgreater 5cm) object ingestion \cite{Lee_2007}, 25 cases (28\%) involved ingestion of multiple objects \cite{Elghali_2016, Lee_2007}. No cases were reported for button battery ingestion, magnet ingestion, and involved large diameter (\textgreater 2.5cm) object ingestion.
\paragraph*{Outcomes}  47 cases (52\%) underwent endoscopic intervention \cite{Elghali_2016, Lee_2007}, 29 cases (32\%) were managed conservatively \cite{Elghali_2016, Karp_1991b}, 15 cases (17\%) underwent surgical intervention \cite{Elghali_2016, Karp_1991b, Lee_2007}, 6 cases (7\%) reported complications \cite{Lee_2007}, 1 case (1\%) died \cite{Elghali_2016}.
\paragraph*{Geographical Location}Cases were recorded in 33 countries: 13 cases from USA \cite{Alao_2006i, Ataya_2013, Bhumi_2024f, Fry_2010, Guinan_2019f, Hardy_2023g, Jehangir_2019h, Kerestes_2019, Kumar_2001, Liu_2005, Tammana_2012j, Tay_2004, Tupesis_2004f}; 7 cases from India \cite{Bhasin_2014, Bhattacharjee_2008, Kar_2015, Kariholu_2008, Kumar_2019f, Misra_2013, Wadhwa_2015e} and UK \cite{Beecroft_1998, Berry_2021e, Cauchi_2002, Cox_2007, Gardner_2017h, Qureshi_2016}; 6 cases from Bulgaria \cite{Losanoff_1996, Losanoff_1997e}; 5 cases from Iran \cite{DivsalarP._2023a, Emamhadi_2018, Farhadi_2024h}; 4 cases from Turkey \cite{Akay_2015f, Atayan_2016, Tanrikulu_2015e, Yildiz_2016e}; 2 cases from China \cite{Jin_2023, Li_2013}, Poland \cite{Kobiela_2015, Wnęk_2015f}, and Spain \cite{CamachoDorado_2018, fjbuilsRepeatedBehaviorDeliberate2024}; 1 case from Australia \cite{Apikotoa_2022f}, Bahrain \cite{Ali_2020f}, Croatia \cite{Trgo_2012f}, Ecuador \cite{DelgadoSalazar_2020c}, Egypt \cite{Ali_2022g}, Ethiopia \cite{Mesfin_2022a}, Germany \cite{teWildt_2010}, Greece \cite{Sakellaridis_2008f}, Hungary \cite{Csaky_1998e}, Iraq \cite{Al-Faham_2020k}, Israel \cite{Goldman_1998f}, Italy \cite{Riva_2018j}, Japan \cite{Ohno_2005}, Nepal \cite{Thapa_2019f}, Netherlands \cite{Benoist_2019e}, Oman \cite{AlShaaibi_2021b}, Pakistan \cite{Yasin_2009}, Portugal \cite{Peixoto_2017f}, Qatar \cite{Ali_2017}, Saudi Arabia \cite{Sultan_2024f}, South Africa \cite{Sobnach_2011f}, Sweden \cite{Naji_2012f}, Switzerland \cite{Wildhaber_2005}, and Taiwan \cite{Chang_2017f}. \paragraph*{Gender} 43 cases (60\%) were male \cite{Akay_2015f, Al-Faham_2020k, Alao_2006i, Ali_2017, Ali_2022g, Apikotoa_2022f, Atayan_2016, Benoist_2019e, Berry_2021e, Bhumi_2024f, CamachoDorado_2018, Csaky_1998e, Emamhadi_2018, Farhadi_2024h, Fry_2010, Gardner_2017h, Guinan_2019f, Jehangir_2019h, Jin_2023, Kobiela_2015, Kumar_2001, Kumar_2019f, Liu_2005, Losanoff_1996, Losanoff_1997e, Mesfin_2022a, Misra_2013, Qureshi_2016, Riva_2018j, Sobnach_2011f, Tammana_2012j, Tanrikulu_2015e, Tay_2004, Thapa_2019f, Trgo_2012f, Wadhwa_2015e, Yasin_2009, teWildt_2010}, 28 cases (39\%) were female \cite{AlShaaibi_2021b, Ali_2020f, Ataya_2013, Beecroft_1998, Bhasin_2014, Bhattacharjee_2008, Cauchi_2002, Chang_2017f, Cox_2007, DelgadoSalazar_2020c, DivsalarP._2023a, Goldman_1998f, Hardy_2023g, Kar_2015, Kariholu_2008, Kerestes_2019, Li_2013, Naji_2012f, Ohno_2005, Peixoto_2017f, Sakellaridis_2008f, Sultan_2024f, Tupesis_2004f, Wildhaber_2005, Wnęk_2015f, Yildiz_2016e}, 1 case (1\%) had no gender recorded \cite{fjbuilsRepeatedBehaviorDeliberate2024}. \paragraph*{Age Group} 25 cases (35\%) were between 26 and 40 years of age \cite{Alao_2006i, Ali_2022g, Apikotoa_2022f, Ataya_2013, Benoist_2019e, Bhasin_2014, Chang_2017f, Cox_2007, DelgadoSalazar_2020c, Farhadi_2024h, Fry_2010, Gardner_2017h, Guinan_2019f, Jin_2023, Kumar_2019f, Losanoff_1996, Misra_2013, Qureshi_2016, Riva_2018j, Sakellaridis_2008f, Tammana_2012j, Trgo_2012f, Wnęk_2015f, Yildiz_2016e, fjbuilsRepeatedBehaviorDeliberate2024}, 18 cases (25\%) were between 18 and 25 years of age \cite{Akay_2015f, Ali_2017, Atayan_2016, Bhattacharjee_2008, Csaky_1998e, Kar_2015, Kariholu_2008, Kobiela_2015, Losanoff_1996, Losanoff_1997e, Mesfin_2022a, Peixoto_2017f, Sobnach_2011f, Tupesis_2004f, Yasin_2009}, 13 cases (18\%) were under 18 years of age \cite{AlShaaibi_2021b, Ali_2020f, Cauchi_2002, DivsalarP._2023a, Goldman_1998f, Liu_2005, Naji_2012f, Ohno_2005, Tanrikulu_2015e, Tay_2004, Wildhaber_2005}, 11 cases (15\%) were between 41 and 60 years of age \cite{Al-Faham_2020k, Bhumi_2024f, CamachoDorado_2018, Emamhadi_2018, Hardy_2023g, Jehangir_2019h, Kumar_2001, Sultan_2024f, Thapa_2019f, Wadhwa_2015e, teWildt_2010}, 3 cases (4\%) were over 60 years of age \cite{Beecroft_1998, Kerestes_2019, Li_2013}, 2 cases (3\%) had no age documented \cite{Berry_2021e}. \paragraph*{Population} 36 cases (50\%) had a psychiatric history \cite{AlShaaibi_2021b, Alao_2006i, Ali_2020f, Apikotoa_2022f, Ataya_2013, Atayan_2016, Beecroft_1998, CamachoDorado_2018, Chang_2017f, DelgadoSalazar_2020c, DivsalarP._2023a, Farhadi_2024h, Fry_2010, Guinan_2019f, Hardy_2023g, Jehangir_2019h, Jin_2023, Kar_2015, Kerestes_2019, Kobiela_2015, Kumar_2001, Kumar_2019f, Liu_2005, Mesfin_2022a, Misra_2013, Ohno_2005, Peixoto_2017f, Sakellaridis_2008f, Sultan_2024f, Tammana_2012j, Tanrikulu_2015e, Yildiz_2016e, fjbuilsRepeatedBehaviorDeliberate2024, teWildt_2010}, 19 cases (26\%) had ingested previously \cite{Alao_2006i, Apikotoa_2022f, Berry_2021e, Bhattacharjee_2008, Csaky_1998e, DivsalarP._2023a, Emamhadi_2018, Guinan_2019f, Jehangir_2019h, Jin_2023, Liu_2005, Sakellaridis_2008f, Tanrikulu_2015e, Thapa_2019f, Yildiz_2016e, fjbuilsRepeatedBehaviorDeliberate2024, teWildt_2010}, 12 cases (17\%) were detained persons \cite{Alao_2006i, Ali_2022g, Apikotoa_2022f, Losanoff_1996, Losanoff_1997e, Qureshi_2016, Tammana_2012j, Trgo_2012f}, 7 cases (10\%) were severely disabled \cite{Atayan_2016, Kerestes_2019, Liu_2005, Ohno_2005, Peixoto_2017f, Yildiz_2016e, teWildt_2010}, 4 cases (6\%) were psychiatric inpatients \cite{DivsalarP._2023a, fjbuilsRepeatedBehaviorDeliberate2024, teWildt_2010}, 3 cases (4\%) were under the influence of alcohol \cite{Benoist_2019e, Csaky_1998e, Thapa_2019f}, 2 cases (3\%) were displaced people \cite{Akay_2015f, Gardner_2017h}. \paragraph*{Motivation} 34 cases (47\%) had a psychiatric motivation \cite{Al-Faham_2020k, Alao_2006i, Ali_2020f, Apikotoa_2022f, Ataya_2013, Atayan_2016, Bhasin_2014, Bhattacharjee_2008, DelgadoSalazar_2020c, DivsalarP._2023a, Emamhadi_2018, Farhadi_2024h, Guinan_2019f, Hardy_2023g, Jehangir_2019h, Jin_2023, Kar_2015, Kariholu_2008, Kerestes_2019, Kobiela_2015, Kumar_2001, Kumar_2019f, Li_2013, Liu_2005, Misra_2013, Ohno_2005, Sakellaridis_2008f, Sultan_2024f, Tammana_2012j, Tanrikulu_2015e, Yasin_2009, teWildt_2010}, 21 cases (29\%) were motivated by self-harm intention \cite{Al-Faham_2020k, AlShaaibi_2021b, Alao_2006i, Ali_2017, CamachoDorado_2018, Chang_2017f, Cox_2007, Csaky_1998e, Fry_2010, Li_2013, Losanoff_1996, Losanoff_1997e, Mesfin_2022a, Sakellaridis_2008f, Tammana_2012j, Tanrikulu_2015e, fjbuilsRepeatedBehaviorDeliberate2024}, 17 cases (24\%) had a psychosocial motivation \cite{Akay_2015f, Benoist_2019e, Bhattacharjee_2008, Cauchi_2002, Goldman_1998f, Hardy_2023g, Kobiela_2015, Li_2013, Naji_2012f, Qureshi_2016, Riva_2018j, Sobnach_2011f, Tay_2004, Thapa_2019f, Tupesis_2004f, Wildhaber_2005, Wnęk_2015f}, 9 cases (12\%) were motivated by protest \cite{Bhumi_2024f, Gardner_2017h, Losanoff_1996, Losanoff_1997e, Tupesis_2004f}, 9 cases (12\%) had another documented motivation \cite{Ali_2020f, Ali_2022g, Emamhadi_2018, Guinan_2019f, Peixoto_2017f, Sakellaridis_2008f, Trgo_2012f, Wadhwa_2015e, Yildiz_2016e}. \paragraph*{Object Characteristics} 51 cases (71\%) ingested a large diameter object (\textgreater{}2.5cm) \cite{Akay_2015f, Al-Faham_2020k, AlShaaibi_2021b, Alao_2006i, Ali_2017, Ali_2022g, Apikotoa_2022f, Atayan_2016, Berry_2021e, Bhasin_2014, CamachoDorado_2018, Cauchi_2002, Chang_2017f, Cox_2007, Csaky_1998e, DivsalarP._2023a, Emamhadi_2018, Gardner_2017h, Guinan_2019f, Jehangir_2019h, Jin_2023, Kariholu_2008, Kerestes_2019, Kobiela_2015, Kumar_2001, Kumar_2019f, Losanoff_1996, Losanoff_1997e, Mesfin_2022a, Misra_2013, Naji_2012f, Ohno_2005, Peixoto_2017f, Qureshi_2016, Riva_2018j, Sakellaridis_2008f, Sultan_2024f, Tanrikulu_2015e, Thapa_2019f, Trgo_2012f, Wnęk_2015f, Yildiz_2016e, fjbuilsRepeatedBehaviorDeliberate2024, teWildt_2010}, 44 cases (61\%) ingested multiple objects \cite{Ali_2020f, Apikotoa_2022f, Ataya_2013, Atayan_2016, Beecroft_1998, Bhattacharjee_2008, Bhumi_2024f, CamachoDorado_2018, Cauchi_2002, Emamhadi_2018, Farhadi_2024h, Fry_2010, Goldman_1998f, Guinan_2019f, Hardy_2023g, Jehangir_2019h, Jin_2023, Kar_2015, Kariholu_2008, Kobiela_2015, Kumar_2001, Kumar_2019f, Li_2013, Liu_2005, Losanoff_1996, Mesfin_2022a, Misra_2013, Naji_2012f, Ohno_2005, Sobnach_2011f, Sultan_2024f, Tammana_2012j, Tanrikulu_2015e, Tay_2004, Thapa_2019f, Wadhwa_2015e, Wildhaber_2005, Yasin_2009, fjbuilsRepeatedBehaviorDeliberate2024, teWildt_2010}, 34 cases (47\%) ingested a sharp object \cite{AlShaaibi_2021b, Alao_2006i, Apikotoa_2022f, Ataya_2013, Benoist_2019e, Bhasin_2014, Bhattacharjee_2008, CamachoDorado_2018, Csaky_1998e, DelgadoSalazar_2020c, DivsalarP._2023a, Emamhadi_2018, Farhadi_2024h, Fry_2010, Guinan_2019f, Hardy_2023g, Jehangir_2019h, Jin_2023, Kariholu_2008, Kobiela_2015, Kumar_2019f, Losanoff_1996, Losanoff_1997e, Mesfin_2022a, Misra_2013, Sobnach_2011f, Yasin_2009, teWildt_2010}, 32 cases (44\%) ingested a long object (\textgreater{}5cm) \cite{Al-Faham_2020k, AlShaaibi_2021b, Ali_2017, Ali_2022g, Atayan_2016, Bhasin_2014, CamachoDorado_2018, Chang_2017f, Cox_2007, Csaky_1998e, DivsalarP._2023a, Emamhadi_2018, Fry_2010, Gardner_2017h, Jin_2023, Kariholu_2008, Kerestes_2019, Kobiela_2015, Kumar_2019f, Mesfin_2022a, Misra_2013, Ohno_2005, Qureshi_2016, Sakellaridis_2008f, Sultan_2024f, Thapa_2019f, Trgo_2012f, Yasin_2009, Yildiz_2016e, teWildt_2010}, 9 cases (12\%) ingested a magnet \cite{Ali_2020f, Bhumi_2024f, Cauchi_2002, Liu_2005, Naji_2012f, Ohno_2005, Tanrikulu_2015e, Tay_2004, Wildhaber_2005}, 2 cases (3\%) ingested a button battery \cite{Berry_2021e, Bhumi_2024f}. \paragraph*{Outcomes} 48 cases (67\%) experienced a complication \cite{Ali_2017, Ali_2020f, Apikotoa_2022f, Atayan_2016, Beecroft_1998, Benoist_2019e, Berry_2021e, Bhasin_2014, Bhumi_2024f, CamachoDorado_2018, Cauchi_2002, Cox_2007, Csaky_1998e, DelgadoSalazar_2020c, DivsalarP._2023a, Emamhadi_2018, Farhadi_2024h, Fry_2010, Gardner_2017h, Goldman_1998f, Jin_2023, Kariholu_2008, Kerestes_2019, Kobiela_2015, Kumar_2001, Kumar_2019f, Liu_2005, Losanoff_1996, Mesfin_2022a, Misra_2013, Naji_2012f, Ohno_2005, Sakellaridis_2008f, Sobnach_2011f, Sultan_2024f, Tanrikulu_2015e, Tay_2004, Thapa_2019f, Trgo_2012f, Tupesis_2004f, Wildhaber_2005, Wnęk_2015f, Yasin_2009, Yildiz_2016e}, 44 cases (61\%) underwent surgery \cite{Al-Faham_2020k, AlShaaibi_2021b, Alao_2006i, Ali_2017, Ali_2020f, Atayan_2016, Beecroft_1998, Bhasin_2014, CamachoDorado_2018, Cauchi_2002, Chang_2017f, Cox_2007, Csaky_1998e, DelgadoSalazar_2020c, DivsalarP._2023a, Farhadi_2024h, Fry_2010, Gardner_2017h, Jin_2023, Kariholu_2008, Kerestes_2019, Kobiela_2015, Kumar_2019f, Liu_2005, Losanoff_1996, Losanoff_1997e, Mesfin_2022a, Misra_2013, Naji_2012f, Sobnach_2011f, Tanrikulu_2015e, Tay_2004, Thapa_2019f, Tupesis_2004f, Wildhaber_2005, Wnęk_2015f, Yasin_2009, Yildiz_2016e, fjbuilsRepeatedBehaviorDeliberate2024}, 31 cases (43\%) underwent endoscopy \cite{Akay_2015f, Ali_2022g, Apikotoa_2022f, Atayan_2016, Benoist_2019e, Berry_2021e, Bhasin_2014, Bhumi_2024f, CamachoDorado_2018, Chang_2017f, DelgadoSalazar_2020c, Gardner_2017h, Guinan_2019f, Hardy_2023g, Jehangir_2019h, Kariholu_2008, Li_2013, Liu_2005, Ohno_2005, Peixoto_2017f, Qureshi_2016, Riva_2018j, Sakellaridis_2008f, Sultan_2024f, Tammana_2012j, Tanrikulu_2015e, Trgo_2012f, Wadhwa_2015e, Wnęk_2015f, teWildt_2010}, 7 cases (10\%) were managed conservatively \cite{Ataya_2013, Bhattacharjee_2008, DivsalarP._2023a, Emamhadi_2018, Goldman_1998f, Kar_2015, Kumar_2001}, 2 cases (3\%) died \cite{Emamhadi_2018, Kumar_2001}. All 90 were male gender. 90 cases (100\%) were detained at the time of ingestion \cite{Elghali_2016, Karp_1991b, Lee_2007}, 88 cases (98\%) were intentional ingestions \cite{Elghali_2016, Karp_1991b, Lee_2007}, 30 cases (33\%) had a psychiatric history documented \cite{Elghali_2016, Karp_1991b, Lee_2007}, 2 cases (2\%) had a history of prior ingestion \cite{Elghali_2016}. No cases were reported for were psychiatric inpatients, were displaced people, were under the influence of alcohol at the time of ingestion, and had a severe disability history.
\paragraph*{Motivation}  70 cases (78\%) reported protest motivation \cite{Elghali_2016, Karp_1991b, Lee_2007}, 12 cases (13\%) reported psychiatric motivation \cite{Karp_1991b}, 6 cases (7\%) reported self-harm motivation \cite{Elghali_2016, Karp_1991b}. No cases were reported for psychosocial motivation and other motivation.
\paragraph*{Object Characteristics}  68 cases (76\%) involved sharp object ingestion \cite{Elghali_2016, Karp_1991b, Lee_2007}, 32 cases (36\%) involved long (\textgreater 5cm) object ingestion \cite{Lee_2007}, 25 cases (28\%) involved ingestion of multiple objects \cite{Elghali_2016, Lee_2007}. No cases were reported for button battery ingestion, magnet ingestion, and involved large diameter (\textgreater 2.5cm) object ingestion.
\paragraph*{Outcomes}  47 cases (52\%) underwent endoscopic intervention \cite{Elghali_2016, Lee_2007}, 29 cases (32\%) were managed conservatively \cite{Elghali_2016, Karp_1991b}, 15 cases (17\%) underwent surgical intervention \cite{Elghali_2016, Karp_1991b, Lee_2007}, 6 cases (7\%) reported complications \cite{Lee_2007}, 1 case (1\%) died \cite{Elghali_2016}.
\paragraph*{Geographical Location}Cases were recorded in 33 countries: 13 cases from USA \cite{Alao_2006i, Ataya_2013, Bhumi_2024f, Fry_2010, Guinan_2019f, Hardy_2023g, Jehangir_2019h, Kerestes_2019, Kumar_2001, Liu_2005, Tammana_2012j, Tay_2004, Tupesis_2004f}; 7 cases from India \cite{Bhasin_2014, Bhattacharjee_2008, Kar_2015, Kariholu_2008, Kumar_2019f, Misra_2013, Wadhwa_2015e} and UK \cite{Beecroft_1998, Berry_2021e, Cauchi_2002, Cox_2007, Gardner_2017h, Qureshi_2016}; 6 cases from Bulgaria \cite{Losanoff_1996, Losanoff_1997e}; 5 cases from Iran \cite{DivsalarP._2023a, Emamhadi_2018, Farhadi_2024h}; 4 cases from Turkey \cite{Akay_2015f, Atayan_2016, Tanrikulu_2015e, Yildiz_2016e}; 2 cases from China \cite{Jin_2023, Li_2013}, Poland \cite{Kobiela_2015, Wnęk_2015f}, and Spain \cite{CamachoDorado_2018, fjbuilsRepeatedBehaviorDeliberate2024}; 1 case from Australia \cite{Apikotoa_2022f}, Bahrain \cite{Ali_2020f}, Croatia \cite{Trgo_2012f}, Ecuador \cite{DelgadoSalazar_2020c}, Egypt \cite{Ali_2022g}, Ethiopia \cite{Mesfin_2022a}, Germany \cite{teWildt_2010}, Greece \cite{Sakellaridis_2008f}, Hungary \cite{Csaky_1998e}, Iraq \cite{Al-Faham_2020k}, Israel \cite{Goldman_1998f}, Italy \cite{Riva_2018j}, Japan \cite{Ohno_2005}, Nepal \cite{Thapa_2019f}, Netherlands \cite{Benoist_2019e}, Oman \cite{AlShaaibi_2021b}, Pakistan \cite{Yasin_2009}, Portugal \cite{Peixoto_2017f}, Qatar \cite{Ali_2017}, Saudi Arabia \cite{Sultan_2024f}, South Africa \cite{Sobnach_2011f}, Sweden \cite{Naji_2012f}, Switzerland \cite{Wildhaber_2005}, and Taiwan \cite{Chang_2017f}. \paragraph*{Gender} 43 cases (60\%) were male \cite{Akay_2015f, Al-Faham_2020k, Alao_2006i, Ali_2017, Ali_2022g, Apikotoa_2022f, Atayan_2016, Benoist_2019e, Berry_2021e, Bhumi_2024f, CamachoDorado_2018, Csaky_1998e, Emamhadi_2018, Farhadi_2024h, Fry_2010, Gardner_2017h, Guinan_2019f, Jehangir_2019h, Jin_2023, Kobiela_2015, Kumar_2001, Kumar_2019f, Liu_2005, Losanoff_1996, Losanoff_1997e, Mesfin_2022a, Misra_2013, Qureshi_2016, Riva_2018j, Sobnach_2011f, Tammana_2012j, Tanrikulu_2015e, Tay_2004, Thapa_2019f, Trgo_2012f, Wadhwa_2015e, Yasin_2009, teWildt_2010}, 28 cases (39\%) were female \cite{AlShaaibi_2021b, Ali_2020f, Ataya_2013, Beecroft_1998, Bhasin_2014, Bhattacharjee_2008, Cauchi_2002, Chang_2017f, Cox_2007, DelgadoSalazar_2020c, DivsalarP._2023a, Goldman_1998f, Hardy_2023g, Kar_2015, Kariholu_2008, Kerestes_2019, Li_2013, Naji_2012f, Ohno_2005, Peixoto_2017f, Sakellaridis_2008f, Sultan_2024f, Tupesis_2004f, Wildhaber_2005, Wnęk_2015f, Yildiz_2016e}, 1 case (1\%) had no gender recorded \cite{fjbuilsRepeatedBehaviorDeliberate2024}. \paragraph*{Age Group} 25 cases (35\%) were between 26 and 40 years of age \cite{Alao_2006i, Ali_2022g, Apikotoa_2022f, Ataya_2013, Benoist_2019e, Bhasin_2014, Chang_2017f, Cox_2007, DelgadoSalazar_2020c, Farhadi_2024h, Fry_2010, Gardner_2017h, Guinan_2019f, Jin_2023, Kumar_2019f, Losanoff_1996, Misra_2013, Qureshi_2016, Riva_2018j, Sakellaridis_2008f, Tammana_2012j, Trgo_2012f, Wnęk_2015f, Yildiz_2016e, fjbuilsRepeatedBehaviorDeliberate2024}, 18 cases (25\%) were between 18 and 25 years of age \cite{Akay_2015f, Ali_2017, Atayan_2016, Bhattacharjee_2008, Csaky_1998e, Kar_2015, Kariholu_2008, Kobiela_2015, Losanoff_1996, Losanoff_1997e, Mesfin_2022a, Peixoto_2017f, Sobnach_2011f, Tupesis_2004f, Yasin_2009}, 13 cases (18\%) were under 18 years of age \cite{AlShaaibi_2021b, Ali_2020f, Cauchi_2002, DivsalarP._2023a, Goldman_1998f, Liu_2005, Naji_2012f, Ohno_2005, Tanrikulu_2015e, Tay_2004, Wildhaber_2005}, 11 cases (15\%) were between 41 and 60 years of age \cite{Al-Faham_2020k, Bhumi_2024f, CamachoDorado_2018, Emamhadi_2018, Hardy_2023g, Jehangir_2019h, Kumar_2001, Sultan_2024f, Thapa_2019f, Wadhwa_2015e, teWildt_2010}, 3 cases (4\%) were over 60 years of age \cite{Beecroft_1998, Kerestes_2019, Li_2013}, 2 cases (3\%) had no age documented \cite{Berry_2021e}. \paragraph*{Population} 36 cases (50\%) had a psychiatric history \cite{AlShaaibi_2021b, Alao_2006i, Ali_2020f, Apikotoa_2022f, Ataya_2013, Atayan_2016, Beecroft_1998, CamachoDorado_2018, Chang_2017f, DelgadoSalazar_2020c, DivsalarP._2023a, Farhadi_2024h, Fry_2010, Guinan_2019f, Hardy_2023g, Jehangir_2019h, Jin_2023, Kar_2015, Kerestes_2019, Kobiela_2015, Kumar_2001, Kumar_2019f, Liu_2005, Mesfin_2022a, Misra_2013, Ohno_2005, Peixoto_2017f, Sakellaridis_2008f, Sultan_2024f, Tammana_2012j, Tanrikulu_2015e, Yildiz_2016e, fjbuilsRepeatedBehaviorDeliberate2024, teWildt_2010}, 19 cases (26\%) had ingested previously \cite{Alao_2006i, Apikotoa_2022f, Berry_2021e, Bhattacharjee_2008, Csaky_1998e, DivsalarP._2023a, Emamhadi_2018, Guinan_2019f, Jehangir_2019h, Jin_2023, Liu_2005, Sakellaridis_2008f, Tanrikulu_2015e, Thapa_2019f, Yildiz_2016e, fjbuilsRepeatedBehaviorDeliberate2024, teWildt_2010}, 12 cases (17\%) were detained persons \cite{Alao_2006i, Ali_2022g, Apikotoa_2022f, Losanoff_1996, Losanoff_1997e, Qureshi_2016, Tammana_2012j, Trgo_2012f}, 7 cases (10\%) were severely disabled \cite{Atayan_2016, Kerestes_2019, Liu_2005, Ohno_2005, Peixoto_2017f, Yildiz_2016e, teWildt_2010}, 4 cases (6\%) were psychiatric inpatients \cite{DivsalarP._2023a, fjbuilsRepeatedBehaviorDeliberate2024, teWildt_2010}, 3 cases (4\%) were under the influence of alcohol \cite{Benoist_2019e, Csaky_1998e, Thapa_2019f}, 2 cases (3\%) were displaced people \cite{Akay_2015f, Gardner_2017h}. \paragraph*{Motivation} 34 cases (47\%) had a psychiatric motivation \cite{Al-Faham_2020k, Alao_2006i, Ali_2020f, Apikotoa_2022f, Ataya_2013, Atayan_2016, Bhasin_2014, Bhattacharjee_2008, DelgadoSalazar_2020c, DivsalarP._2023a, Emamhadi_2018, Farhadi_2024h, Guinan_2019f, Hardy_2023g, Jehangir_2019h, Jin_2023, Kar_2015, Kariholu_2008, Kerestes_2019, Kobiela_2015, Kumar_2001, Kumar_2019f, Li_2013, Liu_2005, Misra_2013, Ohno_2005, Sakellaridis_2008f, Sultan_2024f, Tammana_2012j, Tanrikulu_2015e, Yasin_2009, teWildt_2010}, 21 cases (29\%) were motivated by self-harm intention \cite{Al-Faham_2020k, AlShaaibi_2021b, Alao_2006i, Ali_2017, CamachoDorado_2018, Chang_2017f, Cox_2007, Csaky_1998e, Fry_2010, Li_2013, Losanoff_1996, Losanoff_1997e, Mesfin_2022a, Sakellaridis_2008f, Tammana_2012j, Tanrikulu_2015e, fjbuilsRepeatedBehaviorDeliberate2024}, 17 cases (24\%) had a psychosocial motivation \cite{Akay_2015f, Benoist_2019e, Bhattacharjee_2008, Cauchi_2002, Goldman_1998f, Hardy_2023g, Kobiela_2015, Li_2013, Naji_2012f, Qureshi_2016, Riva_2018j, Sobnach_2011f, Tay_2004, Thapa_2019f, Tupesis_2004f, Wildhaber_2005, Wnęk_2015f}, 9 cases (12\%) were motivated by protest \cite{Bhumi_2024f, Gardner_2017h, Losanoff_1996, Losanoff_1997e, Tupesis_2004f}, 9 cases (12\%) had another documented motivation \cite{Ali_2020f, Ali_2022g, Emamhadi_2018, Guinan_2019f, Peixoto_2017f, Sakellaridis_2008f, Trgo_2012f, Wadhwa_2015e, Yildiz_2016e}. \paragraph*{Object Characteristics} 51 cases (71\%) ingested a large diameter object (\textgreater{}2.5cm) \cite{Akay_2015f, Al-Faham_2020k, AlShaaibi_2021b, Alao_2006i, Ali_2017, Ali_2022g, Apikotoa_2022f, Atayan_2016, Berry_2021e, Bhasin_2014, CamachoDorado_2018, Cauchi_2002, Chang_2017f, Cox_2007, Csaky_1998e, DivsalarP._2023a, Emamhadi_2018, Gardner_2017h, Guinan_2019f, Jehangir_2019h, Jin_2023, Kariholu_2008, Kerestes_2019, Kobiela_2015, Kumar_2001, Kumar_2019f, Losanoff_1996, Losanoff_1997e, Mesfin_2022a, Misra_2013, Naji_2012f, Ohno_2005, Peixoto_2017f, Qureshi_2016, Riva_2018j, Sakellaridis_2008f, Sultan_2024f, Tanrikulu_2015e, Thapa_2019f, Trgo_2012f, Wnęk_2015f, Yildiz_2016e, fjbuilsRepeatedBehaviorDeliberate2024, teWildt_2010}, 44 cases (61\%) ingested multiple objects \cite{Ali_2020f, Apikotoa_2022f, Ataya_2013, Atayan_2016, Beecroft_1998, Bhattacharjee_2008, Bhumi_2024f, CamachoDorado_2018, Cauchi_2002, Emamhadi_2018, Farhadi_2024h, Fry_2010, Goldman_1998f, Guinan_2019f, Hardy_2023g, Jehangir_2019h, Jin_2023, Kar_2015, Kariholu_2008, Kobiela_2015, Kumar_2001, Kumar_2019f, Li_2013, Liu_2005, Losanoff_1996, Mesfin_2022a, Misra_2013, Naji_2012f, Ohno_2005, Sobnach_2011f, Sultan_2024f, Tammana_2012j, Tanrikulu_2015e, Tay_2004, Thapa_2019f, Wadhwa_2015e, Wildhaber_2005, Yasin_2009, fjbuilsRepeatedBehaviorDeliberate2024, teWildt_2010}, 34 cases (47\%) ingested a sharp object \cite{AlShaaibi_2021b, Alao_2006i, Apikotoa_2022f, Ataya_2013, Benoist_2019e, Bhasin_2014, Bhattacharjee_2008, CamachoDorado_2018, Csaky_1998e, DelgadoSalazar_2020c, DivsalarP._2023a, Emamhadi_2018, Farhadi_2024h, Fry_2010, Guinan_2019f, Hardy_2023g, Jehangir_2019h, Jin_2023, Kariholu_2008, Kobiela_2015, Kumar_2019f, Losanoff_1996, Losanoff_1997e, Mesfin_2022a, Misra_2013, Sobnach_2011f, Yasin_2009, teWildt_2010}, 32 cases (44\%) ingested a long object (\textgreater{}5cm) \cite{Al-Faham_2020k, AlShaaibi_2021b, Ali_2017, Ali_2022g, Atayan_2016, Bhasin_2014, CamachoDorado_2018, Chang_2017f, Cox_2007, Csaky_1998e, DivsalarP._2023a, Emamhadi_2018, Fry_2010, Gardner_2017h, Jin_2023, Kariholu_2008, Kerestes_2019, Kobiela_2015, Kumar_2019f, Mesfin_2022a, Misra_2013, Ohno_2005, Qureshi_2016, Sakellaridis_2008f, Sultan_2024f, Thapa_2019f, Trgo_2012f, Yasin_2009, Yildiz_2016e, teWildt_2010}, 9 cases (12\%) ingested a magnet \cite{Ali_2020f, Bhumi_2024f, Cauchi_2002, Liu_2005, Naji_2012f, Ohno_2005, Tanrikulu_2015e, Tay_2004, Wildhaber_2005}, 2 cases (3\%) ingested a button battery \cite{Berry_2021e, Bhumi_2024f}. \paragraph*{Outcomes} 48 cases (67\%) experienced a complication \cite{Ali_2017, Ali_2020f, Apikotoa_2022f, Atayan_2016, Beecroft_1998, Benoist_2019e, Berry_2021e, Bhasin_2014, Bhumi_2024f, CamachoDorado_2018, Cauchi_2002, Cox_2007, Csaky_1998e, DelgadoSalazar_2020c, DivsalarP._2023a, Emamhadi_2018, Farhadi_2024h, Fry_2010, Gardner_2017h, Goldman_1998f, Jin_2023, Kariholu_2008, Kerestes_2019, Kobiela_2015, Kumar_2001, Kumar_2019f, Liu_2005, Losanoff_1996, Mesfin_2022a, Misra_2013, Naji_2012f, Ohno_2005, Sakellaridis_2008f, Sobnach_2011f, Sultan_2024f, Tanrikulu_2015e, Tay_2004, Thapa_2019f, Trgo_2012f, Tupesis_2004f, Wildhaber_2005, Wnęk_2015f, Yasin_2009, Yildiz_2016e}, 44 cases (61\%) underwent surgery \cite{Al-Faham_2020k, AlShaaibi_2021b, Alao_2006i, Ali_2017, Ali_2020f, Atayan_2016, Beecroft_1998, Bhasin_2014, CamachoDorado_2018, Cauchi_2002, Chang_2017f, Cox_2007, Csaky_1998e, DelgadoSalazar_2020c, DivsalarP._2023a, Farhadi_2024h, Fry_2010, Gardner_2017h, Jin_2023, Kariholu_2008, Kerestes_2019, Kobiela_2015, Kumar_2019f, Liu_2005, Losanoff_1996, Losanoff_1997e, Mesfin_2022a, Misra_2013, Naji_2012f, Sobnach_2011f, Tanrikulu_2015e, Tay_2004, Thapa_2019f, Tupesis_2004f, Wildhaber_2005, Wnęk_2015f, Yasin_2009, Yildiz_2016e, fjbuilsRepeatedBehaviorDeliberate2024}, 31 cases (43\%) underwent endoscopy \cite{Akay_2015f, Ali_2022g, Apikotoa_2022f, Atayan_2016, Benoist_2019e, Berry_2021e, Bhasin_2014, Bhumi_2024f, CamachoDorado_2018, Chang_2017f, DelgadoSalazar_2020c, Gardner_2017h, Guinan_2019f, Hardy_2023g, Jehangir_2019h, Kariholu_2008, Li_2013, Liu_2005, Ohno_2005, Peixoto_2017f, Qureshi_2016, Riva_2018j, Sakellaridis_2008f, Sultan_2024f, Tammana_2012j, Tanrikulu_2015e, Trgo_2012f, Wadhwa_2015e, Wnęk_2015f, teWildt_2010}, 7 cases (10\%) were managed conservatively \cite{Ataya_2013, Bhattacharjee_2008, DivsalarP._2023a, Emamhadi_2018, Goldman_1998f, Kar_2015, Kumar_2001}, 2 cases (3\%) died \cite{Emamhadi_2018, Kumar_2001}. All 90 were male gender. 90 cases (100\%) were detained at the time of ingestion \cite{Elghali_2016, Karp_1991b, Lee_2007}, 88 cases (98\%) were intentional ingestions \cite{Elghali_2016, Karp_1991b, Lee_2007}, 30 cases (33\%) had a psychiatric history documented \cite{Elghali_2016, Karp_1991b, Lee_2007}, 2 cases (2\%) had a history of prior ingestion \cite{Elghali_2016}. No cases were reported for were psychiatric inpatients, were displaced people, were under the influence of alcohol at the time of ingestion, and had a severe disability history.
\paragraph*{Motivation}  70 cases (78\%) reported protest motivation \cite{Elghali_2016, Karp_1991b, Lee_2007}, 12 cases (13\%) reported psychiatric motivation \cite{Karp_1991b}, 6 cases (7\%) reported self-harm motivation \cite{Elghali_2016, Karp_1991b}. No cases were reported for psychosocial motivation and other motivation.
\paragraph*{Object Characteristics}  68 cases (76\%) involved sharp object ingestion \cite{Elghali_2016, Karp_1991b, Lee_2007}, 32 cases (36\%) involved long (\textgreater 5cm) object ingestion \cite{Lee_2007}, 25 cases (28\%) involved ingestion of multiple objects \cite{Elghali_2016, Lee_2007}. No cases were reported for button battery ingestion, magnet ingestion, and involved large diameter (\textgreater 2.5cm) object ingestion.
\paragraph*{Outcomes}  47 cases (52\%) underwent endoscopic intervention \cite{Elghali_2016, Lee_2007}, 29 cases (32\%) were managed conservatively \cite{Elghali_2016, Karp_1991b}, 15 cases (17\%) underwent surgical intervention \cite{Elghali_2016, Karp_1991b, Lee_2007}, 6 cases (7\%) reported complications \cite{Lee_2007}, 1 case (1\%) died \cite{Elghali_2016}.
\paragraph*{Geographical Location}Cases were recorded in 33 countries: 13 cases from USA \cite{Alao_2006i, Ataya_2013, Bhumi_2024f, Fry_2010, Guinan_2019f, Hardy_2023g, Jehangir_2019h, Kerestes_2019, Kumar_2001, Liu_2005, Tammana_2012j, Tay_2004, Tupesis_2004f}; 7 cases from India \cite{Bhasin_2014, Bhattacharjee_2008, Kar_2015, Kariholu_2008, Kumar_2019f, Misra_2013, Wadhwa_2015e} and UK \cite{Beecroft_1998, Berry_2021e, Cauchi_2002, Cox_2007, Gardner_2017h, Qureshi_2016}; 6 cases from Bulgaria \cite{Losanoff_1996, Losanoff_1997e}; 5 cases from Iran \cite{DivsalarP._2023a, Emamhadi_2018, Farhadi_2024h}; 4 cases from Turkey \cite{Akay_2015f, Atayan_2016, Tanrikulu_2015e, Yildiz_2016e}; 2 cases from China \cite{Jin_2023, Li_2013}, Poland \cite{Kobiela_2015, Wnęk_2015f}, and Spain \cite{CamachoDorado_2018, fjbuilsRepeatedBehaviorDeliberate2024}; 1 case from Australia \cite{Apikotoa_2022f}, Bahrain \cite{Ali_2020f}, Croatia \cite{Trgo_2012f}, Ecuador \cite{DelgadoSalazar_2020c}, Egypt \cite{Ali_2022g}, Ethiopia \cite{Mesfin_2022a}, Germany \cite{teWildt_2010}, Greece \cite{Sakellaridis_2008f}, Hungary \cite{Csaky_1998e}, Iraq \cite{Al-Faham_2020k}, Israel \cite{Goldman_1998f}, Italy \cite{Riva_2018j}, Japan \cite{Ohno_2005}, Nepal \cite{Thapa_2019f}, Netherlands \cite{Benoist_2019e}, Oman \cite{AlShaaibi_2021b}, Pakistan \cite{Yasin_2009}, Portugal \cite{Peixoto_2017f}, Qatar \cite{Ali_2017}, Saudi Arabia \cite{Sultan_2024f}, South Africa \cite{Sobnach_2011f}, Sweden \cite{Naji_2012f}, Switzerland \cite{Wildhaber_2005}, and Taiwan \cite{Chang_2017f}. \paragraph*{Gender} 43 cases (60\%) were male \cite{Akay_2015f, Al-Faham_2020k, Alao_2006i, Ali_2017, Ali_2022g, Apikotoa_2022f, Atayan_2016, Benoist_2019e, Berry_2021e, Bhumi_2024f, CamachoDorado_2018, Csaky_1998e, Emamhadi_2018, Farhadi_2024h, Fry_2010, Gardner_2017h, Guinan_2019f, Jehangir_2019h, Jin_2023, Kobiela_2015, Kumar_2001, Kumar_2019f, Liu_2005, Losanoff_1996, Losanoff_1997e, Mesfin_2022a, Misra_2013, Qureshi_2016, Riva_2018j, Sobnach_2011f, Tammana_2012j, Tanrikulu_2015e, Tay_2004, Thapa_2019f, Trgo_2012f, Wadhwa_2015e, Yasin_2009, teWildt_2010}, 28 cases (39\%) were female \cite{AlShaaibi_2021b, Ali_2020f, Ataya_2013, Beecroft_1998, Bhasin_2014, Bhattacharjee_2008, Cauchi_2002, Chang_2017f, Cox_2007, DelgadoSalazar_2020c, DivsalarP._2023a, Goldman_1998f, Hardy_2023g, Kar_2015, Kariholu_2008, Kerestes_2019, Li_2013, Naji_2012f, Ohno_2005, Peixoto_2017f, Sakellaridis_2008f, Sultan_2024f, Tupesis_2004f, Wildhaber_2005, Wnęk_2015f, Yildiz_2016e}, 1 case (1\%) had no gender recorded \cite{fjbuilsRepeatedBehaviorDeliberate2024}. \paragraph*{Age Group} 25 cases (35\%) were between 26 and 40 years of age \cite{Alao_2006i, Ali_2022g, Apikotoa_2022f, Ataya_2013, Benoist_2019e, Bhasin_2014, Chang_2017f, Cox_2007, DelgadoSalazar_2020c, Farhadi_2024h, Fry_2010, Gardner_2017h, Guinan_2019f, Jin_2023, Kumar_2019f, Losanoff_1996, Misra_2013, Qureshi_2016, Riva_2018j, Sakellaridis_2008f, Tammana_2012j, Trgo_2012f, Wnęk_2015f, Yildiz_2016e, fjbuilsRepeatedBehaviorDeliberate2024}, 18 cases (25\%) were between 18 and 25 years of age \cite{Akay_2015f, Ali_2017, Atayan_2016, Bhattacharjee_2008, Csaky_1998e, Kar_2015, Kariholu_2008, Kobiela_2015, Losanoff_1996, Losanoff_1997e, Mesfin_2022a, Peixoto_2017f, Sobnach_2011f, Tupesis_2004f, Yasin_2009}, 13 cases (18\%) were under 18 years of age \cite{AlShaaibi_2021b, Ali_2020f, Cauchi_2002, DivsalarP._2023a, Goldman_1998f, Liu_2005, Naji_2012f, Ohno_2005, Tanrikulu_2015e, Tay_2004, Wildhaber_2005}, 11 cases (15\%) were between 41 and 60 years of age \cite{Al-Faham_2020k, Bhumi_2024f, CamachoDorado_2018, Emamhadi_2018, Hardy_2023g, Jehangir_2019h, Kumar_2001, Sultan_2024f, Thapa_2019f, Wadhwa_2015e, teWildt_2010}, 3 cases (4\%) were over 60 years of age \cite{Beecroft_1998, Kerestes_2019, Li_2013}, 2 cases (3\%) had no age documented \cite{Berry_2021e}. \paragraph*{Population} 36 cases (50\%) had a psychiatric history \cite{AlShaaibi_2021b, Alao_2006i, Ali_2020f, Apikotoa_2022f, Ataya_2013, Atayan_2016, Beecroft_1998, CamachoDorado_2018, Chang_2017f, DelgadoSalazar_2020c, DivsalarP._2023a, Farhadi_2024h, Fry_2010, Guinan_2019f, Hardy_2023g, Jehangir_2019h, Jin_2023, Kar_2015, Kerestes_2019, Kobiela_2015, Kumar_2001, Kumar_2019f, Liu_2005, Mesfin_2022a, Misra_2013, Ohno_2005, Peixoto_2017f, Sakellaridis_2008f, Sultan_2024f, Tammana_2012j, Tanrikulu_2015e, Yildiz_2016e, fjbuilsRepeatedBehaviorDeliberate2024, teWildt_2010}, 19 cases (26\%) had ingested previously \cite{Alao_2006i, Apikotoa_2022f, Berry_2021e, Bhattacharjee_2008, Csaky_1998e, DivsalarP._2023a, Emamhadi_2018, Guinan_2019f, Jehangir_2019h, Jin_2023, Liu_2005, Sakellaridis_2008f, Tanrikulu_2015e, Thapa_2019f, Yildiz_2016e, fjbuilsRepeatedBehaviorDeliberate2024, teWildt_2010}, 12 cases (17\%) were detained persons \cite{Alao_2006i, Ali_2022g, Apikotoa_2022f, Losanoff_1996, Losanoff_1997e, Qureshi_2016, Tammana_2012j, Trgo_2012f}, 7 cases (10\%) were severely disabled \cite{Atayan_2016, Kerestes_2019, Liu_2005, Ohno_2005, Peixoto_2017f, Yildiz_2016e, teWildt_2010}, 4 cases (6\%) were psychiatric inpatients \cite{DivsalarP._2023a, fjbuilsRepeatedBehaviorDeliberate2024, teWildt_2010}, 3 cases (4\%) were under the influence of alcohol \cite{Benoist_2019e, Csaky_1998e, Thapa_2019f}, 2 cases (3\%) were displaced people \cite{Akay_2015f, Gardner_2017h}. \paragraph*{Motivation} 34 cases (47\%) had a psychiatric motivation \cite{Al-Faham_2020k, Alao_2006i, Ali_2020f, Apikotoa_2022f, Ataya_2013, Atayan_2016, Bhasin_2014, Bhattacharjee_2008, DelgadoSalazar_2020c, DivsalarP._2023a, Emamhadi_2018, Farhadi_2024h, Guinan_2019f, Hardy_2023g, Jehangir_2019h, Jin_2023, Kar_2015, Kariholu_2008, Kerestes_2019, Kobiela_2015, Kumar_2001, Kumar_2019f, Li_2013, Liu_2005, Misra_2013, Ohno_2005, Sakellaridis_2008f, Sultan_2024f, Tammana_2012j, Tanrikulu_2015e, Yasin_2009, teWildt_2010}, 21 cases (29\%) were motivated by self-harm intention \cite{Al-Faham_2020k, AlShaaibi_2021b, Alao_2006i, Ali_2017, CamachoDorado_2018, Chang_2017f, Cox_2007, Csaky_1998e, Fry_2010, Li_2013, Losanoff_1996, Losanoff_1997e, Mesfin_2022a, Sakellaridis_2008f, Tammana_2012j, Tanrikulu_2015e, fjbuilsRepeatedBehaviorDeliberate2024}, 17 cases (24\%) had a psychosocial motivation \cite{Akay_2015f, Benoist_2019e, Bhattacharjee_2008, Cauchi_2002, Goldman_1998f, Hardy_2023g, Kobiela_2015, Li_2013, Naji_2012f, Qureshi_2016, Riva_2018j, Sobnach_2011f, Tay_2004, Thapa_2019f, Tupesis_2004f, Wildhaber_2005, Wnęk_2015f}, 9 cases (12\%) were motivated by protest \cite{Bhumi_2024f, Gardner_2017h, Losanoff_1996, Losanoff_1997e, Tupesis_2004f}, 9 cases (12\%) had another documented motivation \cite{Ali_2020f, Ali_2022g, Emamhadi_2018, Guinan_2019f, Peixoto_2017f, Sakellaridis_2008f, Trgo_2012f, Wadhwa_2015e, Yildiz_2016e}. \paragraph*{Object Characteristics} 51 cases (71\%) ingested a large diameter object (\textgreater{}2.5cm) \cite{Akay_2015f, Al-Faham_2020k, AlShaaibi_2021b, Alao_2006i, Ali_2017, Ali_2022g, Apikotoa_2022f, Atayan_2016, Berry_2021e, Bhasin_2014, CamachoDorado_2018, Cauchi_2002, Chang_2017f, Cox_2007, Csaky_1998e, DivsalarP._2023a, Emamhadi_2018, Gardner_2017h, Guinan_2019f, Jehangir_2019h, Jin_2023, Kariholu_2008, Kerestes_2019, Kobiela_2015, Kumar_2001, Kumar_2019f, Losanoff_1996, Losanoff_1997e, Mesfin_2022a, Misra_2013, Naji_2012f, Ohno_2005, Peixoto_2017f, Qureshi_2016, Riva_2018j, Sakellaridis_2008f, Sultan_2024f, Tanrikulu_2015e, Thapa_2019f, Trgo_2012f, Wnęk_2015f, Yildiz_2016e, fjbuilsRepeatedBehaviorDeliberate2024, teWildt_2010}, 44 cases (61\%) ingested multiple objects \cite{Ali_2020f, Apikotoa_2022f, Ataya_2013, Atayan_2016, Beecroft_1998, Bhattacharjee_2008, Bhumi_2024f, CamachoDorado_2018, Cauchi_2002, Emamhadi_2018, Farhadi_2024h, Fry_2010, Goldman_1998f, Guinan_2019f, Hardy_2023g, Jehangir_2019h, Jin_2023, Kar_2015, Kariholu_2008, Kobiela_2015, Kumar_2001, Kumar_2019f, Li_2013, Liu_2005, Losanoff_1996, Mesfin_2022a, Misra_2013, Naji_2012f, Ohno_2005, Sobnach_2011f, Sultan_2024f, Tammana_2012j, Tanrikulu_2015e, Tay_2004, Thapa_2019f, Wadhwa_2015e, Wildhaber_2005, Yasin_2009, fjbuilsRepeatedBehaviorDeliberate2024, teWildt_2010}, 34 cases (47\%) ingested a sharp object \cite{AlShaaibi_2021b, Alao_2006i, Apikotoa_2022f, Ataya_2013, Benoist_2019e, Bhasin_2014, Bhattacharjee_2008, CamachoDorado_2018, Csaky_1998e, DelgadoSalazar_2020c, DivsalarP._2023a, Emamhadi_2018, Farhadi_2024h, Fry_2010, Guinan_2019f, Hardy_2023g, Jehangir_2019h, Jin_2023, Kariholu_2008, Kobiela_2015, Kumar_2019f, Losanoff_1996, Losanoff_1997e, Mesfin_2022a, Misra_2013, Sobnach_2011f, Yasin_2009, teWildt_2010}, 32 cases (44\%) ingested a long object (\textgreater{}5cm) \cite{Al-Faham_2020k, AlShaaibi_2021b, Ali_2017, Ali_2022g, Atayan_2016, Bhasin_2014, CamachoDorado_2018, Chang_2017f, Cox_2007, Csaky_1998e, DivsalarP._2023a, Emamhadi_2018, Fry_2010, Gardner_2017h, Jin_2023, Kariholu_2008, Kerestes_2019, Kobiela_2015, Kumar_2019f, Mesfin_2022a, Misra_2013, Ohno_2005, Qureshi_2016, Sakellaridis_2008f, Sultan_2024f, Thapa_2019f, Trgo_2012f, Yasin_2009, Yildiz_2016e, teWildt_2010}, 9 cases (12\%) ingested a magnet \cite{Ali_2020f, Bhumi_2024f, Cauchi_2002, Liu_2005, Naji_2012f, Ohno_2005, Tanrikulu_2015e, Tay_2004, Wildhaber_2005}, 2 cases (3\%) ingested a button battery \cite{Berry_2021e, Bhumi_2024f}. \paragraph*{Outcomes} 48 cases (67\%) experienced a complication \cite{Ali_2017, Ali_2020f, Apikotoa_2022f, Atayan_2016, Beecroft_1998, Benoist_2019e, Berry_2021e, Bhasin_2014, Bhumi_2024f, CamachoDorado_2018, Cauchi_2002, Cox_2007, Csaky_1998e, DelgadoSalazar_2020c, DivsalarP._2023a, Emamhadi_2018, Farhadi_2024h, Fry_2010, Gardner_2017h, Goldman_1998f, Jin_2023, Kariholu_2008, Kerestes_2019, Kobiela_2015, Kumar_2001, Kumar_2019f, Liu_2005, Losanoff_1996, Mesfin_2022a, Misra_2013, Naji_2012f, Ohno_2005, Sakellaridis_2008f, Sobnach_2011f, Sultan_2024f, Tanrikulu_2015e, Tay_2004, Thapa_2019f, Trgo_2012f, Tupesis_2004f, Wildhaber_2005, Wnęk_2015f, Yasin_2009, Yildiz_2016e}, 44 cases (61\%) underwent surgery \cite{Al-Faham_2020k, AlShaaibi_2021b, Alao_2006i, Ali_2017, Ali_2020f, Atayan_2016, Beecroft_1998, Bhasin_2014, CamachoDorado_2018, Cauchi_2002, Chang_2017f, Cox_2007, Csaky_1998e, DelgadoSalazar_2020c, DivsalarP._2023a, Farhadi_2024h, Fry_2010, Gardner_2017h, Jin_2023, Kariholu_2008, Kerestes_2019, Kobiela_2015, Kumar_2019f, Liu_2005, Losanoff_1996, Losanoff_1997e, Mesfin_2022a, Misra_2013, Naji_2012f, Sobnach_2011f, Tanrikulu_2015e, Tay_2004, Thapa_2019f, Tupesis_2004f, Wildhaber_2005, Wnęk_2015f, Yasin_2009, Yildiz_2016e, fjbuilsRepeatedBehaviorDeliberate2024}, 31 cases (43\%) underwent endoscopy \cite{Akay_2015f, Ali_2022g, Apikotoa_2022f, Atayan_2016, Benoist_2019e, Berry_2021e, Bhasin_2014, Bhumi_2024f, CamachoDorado_2018, Chang_2017f, DelgadoSalazar_2020c, Gardner_2017h, Guinan_2019f, Hardy_2023g, Jehangir_2019h, Kariholu_2008, Li_2013, Liu_2005, Ohno_2005, Peixoto_2017f, Qureshi_2016, Riva_2018j, Sakellaridis_2008f, Sultan_2024f, Tammana_2012j, Tanrikulu_2015e, Trgo_2012f, Wadhwa_2015e, Wnęk_2015f, teWildt_2010}, 7 cases (10\%) were managed conservatively \cite{Ataya_2013, Bhattacharjee_2008, DivsalarP._2023a, Emamhadi_2018, Goldman_1998f, Kar_2015, Kumar_2001}, 2 cases (3\%) died \cite{Emamhadi_2018, Kumar_2001}. All 90 were male gender. 90 cases (100\%) were detained at the time of ingestion \cite{Elghali_2016, Karp_1991b, Lee_2007}, 88 cases (98\%) were intentional ingestions \cite{Elghali_2016, Karp_1991b, Lee_2007}, 30 cases (33\%) had a psychiatric history documented \cite{Elghali_2016, Karp_1991b, Lee_2007}, 2 cases (2\%) had a history of prior ingestion \cite{Elghali_2016}. No cases were reported for were psychiatric inpatients, were displaced people, were under the influence of alcohol at the time of ingestion, and had a severe disability history.
\paragraph*{Motivation}  70 cases (78\%) reported protest motivation \cite{Elghali_2016, Karp_1991b, Lee_2007}, 12 cases (13\%) reported psychiatric motivation \cite{Karp_1991b}, 6 cases (7\%) reported self-harm motivation \cite{Elghali_2016, Karp_1991b}. No cases were reported for psychosocial motivation and other motivation.
\paragraph*{Object Characteristics}  68 cases (76\%) involved sharp object ingestion \cite{Elghali_2016, Karp_1991b, Lee_2007}, 32 cases (36\%) involved long (\textgreater 5cm) object ingestion \cite{Lee_2007}, 25 cases (28\%) involved ingestion of multiple objects \cite{Elghali_2016, Lee_2007}. No cases were reported for button battery ingestion, magnet ingestion, and involved large diameter (\textgreater 2.5cm) object ingestion.
\paragraph*{Outcomes}  47 cases (52\%) underwent endoscopic intervention \cite{Elghali_2016, Lee_2007}, 29 cases (32\%) were managed conservatively \cite{Elghali_2016, Karp_1991b}, 15 cases (17\%) underwent surgical intervention \cite{Elghali_2016, Karp_1991b, Lee_2007}, 6 cases (7\%) reported complications \cite{Lee_2007}, 1 case (1\%) died \cite{Elghali_2016}.
\paragraph*{Geographical Location}Cases were recorded in 33 countries: 13 cases from USA \cite{Alao_2006i, Ataya_2013, Bhumi_2024f, Fry_2010, Guinan_2019f, Hardy_2023g, Jehangir_2019h, Kerestes_2019, Kumar_2001, Liu_2005, Tammana_2012j, Tay_2004, Tupesis_2004f}; 7 cases from India \cite{Bhasin_2014, Bhattacharjee_2008, Kar_2015, Kariholu_2008, Kumar_2019f, Misra_2013, Wadhwa_2015e} and UK \cite{Beecroft_1998, Berry_2021e, Cauchi_2002, Cox_2007, Gardner_2017h, Qureshi_2016}; 6 cases from Bulgaria \cite{Losanoff_1996, Losanoff_1997e}; 5 cases from Iran \cite{DivsalarP._2023a, Emamhadi_2018, Farhadi_2024h}; 4 cases from Turkey \cite{Akay_2015f, Atayan_2016, Tanrikulu_2015e, Yildiz_2016e}; 2 cases from China \cite{Jin_2023, Li_2013}, Poland \cite{Kobiela_2015, Wnęk_2015f}, and Spain \cite{CamachoDorado_2018, fjbuilsRepeatedBehaviorDeliberate2024}; 1 case from Australia \cite{Apikotoa_2022f}, Bahrain \cite{Ali_2020f}, Croatia \cite{Trgo_2012f}, Ecuador \cite{DelgadoSalazar_2020c}, Egypt \cite{Ali_2022g}, Ethiopia \cite{Mesfin_2022a}, Germany \cite{teWildt_2010}, Greece \cite{Sakellaridis_2008f}, Hungary \cite{Csaky_1998e}, Iraq \cite{Al-Faham_2020k}, Israel \cite{Goldman_1998f}, Italy \cite{Riva_2018j}, Japan \cite{Ohno_2005}, Nepal \cite{Thapa_2019f}, Netherlands \cite{Benoist_2019e}, Oman \cite{AlShaaibi_2021b}, Pakistan \cite{Yasin_2009}, Portugal \cite{Peixoto_2017f}, Qatar \cite{Ali_2017}, Saudi Arabia \cite{Sultan_2024f}, South Africa \cite{Sobnach_2011f}, Sweden \cite{Naji_2012f}, Switzerland \cite{Wildhaber_2005}, and Taiwan \cite{Chang_2017f}. \paragraph*{Gender} 43 cases (60\%) were male \cite{Akay_2015f, Al-Faham_2020k, Alao_2006i, Ali_2017, Ali_2022g, Apikotoa_2022f, Atayan_2016, Benoist_2019e, Berry_2021e, Bhumi_2024f, CamachoDorado_2018, Csaky_1998e, Emamhadi_2018, Farhadi_2024h, Fry_2010, Gardner_2017h, Guinan_2019f, Jehangir_2019h, Jin_2023, Kobiela_2015, Kumar_2001, Kumar_2019f, Liu_2005, Losanoff_1996, Losanoff_1997e, Mesfin_2022a, Misra_2013, Qureshi_2016, Riva_2018j, Sobnach_2011f, Tammana_2012j, Tanrikulu_2015e, Tay_2004, Thapa_2019f, Trgo_2012f, Wadhwa_2015e, Yasin_2009, teWildt_2010}, 28 cases (39\%) were female \cite{AlShaaibi_2021b, Ali_2020f, Ataya_2013, Beecroft_1998, Bhasin_2014, Bhattacharjee_2008, Cauchi_2002, Chang_2017f, Cox_2007, DelgadoSalazar_2020c, DivsalarP._2023a, Goldman_1998f, Hardy_2023g, Kar_2015, Kariholu_2008, Kerestes_2019, Li_2013, Naji_2012f, Ohno_2005, Peixoto_2017f, Sakellaridis_2008f, Sultan_2024f, Tupesis_2004f, Wildhaber_2005, Wnęk_2015f, Yildiz_2016e}, 1 case (1\%) had no gender recorded \cite{fjbuilsRepeatedBehaviorDeliberate2024}. \paragraph*{Age Group} 25 cases (35\%) were between 26 and 40 years of age \cite{Alao_2006i, Ali_2022g, Apikotoa_2022f, Ataya_2013, Benoist_2019e, Bhasin_2014, Chang_2017f, Cox_2007, DelgadoSalazar_2020c, Farhadi_2024h, Fry_2010, Gardner_2017h, Guinan_2019f, Jin_2023, Kumar_2019f, Losanoff_1996, Misra_2013, Qureshi_2016, Riva_2018j, Sakellaridis_2008f, Tammana_2012j, Trgo_2012f, Wnęk_2015f, Yildiz_2016e, fjbuilsRepeatedBehaviorDeliberate2024}, 18 cases (25\%) were between 18 and 25 years of age \cite{Akay_2015f, Ali_2017, Atayan_2016, Bhattacharjee_2008, Csaky_1998e, Kar_2015, Kariholu_2008, Kobiela_2015, Losanoff_1996, Losanoff_1997e, Mesfin_2022a, Peixoto_2017f, Sobnach_2011f, Tupesis_2004f, Yasin_2009}, 13 cases (18\%) were under 18 years of age \cite{AlShaaibi_2021b, Ali_2020f, Cauchi_2002, DivsalarP._2023a, Goldman_1998f, Liu_2005, Naji_2012f, Ohno_2005, Tanrikulu_2015e, Tay_2004, Wildhaber_2005}, 11 cases (15\%) were between 41 and 60 years of age \cite{Al-Faham_2020k, Bhumi_2024f, CamachoDorado_2018, Emamhadi_2018, Hardy_2023g, Jehangir_2019h, Kumar_2001, Sultan_2024f, Thapa_2019f, Wadhwa_2015e, teWildt_2010}, 3 cases (4\%) were over 60 years of age \cite{Beecroft_1998, Kerestes_2019, Li_2013}, 2 cases (3\%) had no age documented \cite{Berry_2021e}. \paragraph*{Population} 36 cases (50\%) had a psychiatric history \cite{AlShaaibi_2021b, Alao_2006i, Ali_2020f, Apikotoa_2022f, Ataya_2013, Atayan_2016, Beecroft_1998, CamachoDorado_2018, Chang_2017f, DelgadoSalazar_2020c, DivsalarP._2023a, Farhadi_2024h, Fry_2010, Guinan_2019f, Hardy_2023g, Jehangir_2019h, Jin_2023, Kar_2015, Kerestes_2019, Kobiela_2015, Kumar_2001, Kumar_2019f, Liu_2005, Mesfin_2022a, Misra_2013, Ohno_2005, Peixoto_2017f, Sakellaridis_2008f, Sultan_2024f, Tammana_2012j, Tanrikulu_2015e, Yildiz_2016e, fjbuilsRepeatedBehaviorDeliberate2024, teWildt_2010}, 19 cases (26\%) had ingested previously \cite{Alao_2006i, Apikotoa_2022f, Berry_2021e, Bhattacharjee_2008, Csaky_1998e, DivsalarP._2023a, Emamhadi_2018, Guinan_2019f, Jehangir_2019h, Jin_2023, Liu_2005, Sakellaridis_2008f, Tanrikulu_2015e, Thapa_2019f, Yildiz_2016e, fjbuilsRepeatedBehaviorDeliberate2024, teWildt_2010}, 12 cases (17\%) were detained persons \cite{Alao_2006i, Ali_2022g, Apikotoa_2022f, Losanoff_1996, Losanoff_1997e, Qureshi_2016, Tammana_2012j, Trgo_2012f}, 7 cases (10\%) were severely disabled \cite{Atayan_2016, Kerestes_2019, Liu_2005, Ohno_2005, Peixoto_2017f, Yildiz_2016e, teWildt_2010}, 4 cases (6\%) were psychiatric inpatients \cite{DivsalarP._2023a, fjbuilsRepeatedBehaviorDeliberate2024, teWildt_2010}, 3 cases (4\%) were under the influence of alcohol \cite{Benoist_2019e, Csaky_1998e, Thapa_2019f}, 2 cases (3\%) were displaced people \cite{Akay_2015f, Gardner_2017h}. \paragraph*{Motivation} 34 cases (47\%) had a psychiatric motivation \cite{Al-Faham_2020k, Alao_2006i, Ali_2020f, Apikotoa_2022f, Ataya_2013, Atayan_2016, Bhasin_2014, Bhattacharjee_2008, DelgadoSalazar_2020c, DivsalarP._2023a, Emamhadi_2018, Farhadi_2024h, Guinan_2019f, Hardy_2023g, Jehangir_2019h, Jin_2023, Kar_2015, Kariholu_2008, Kerestes_2019, Kobiela_2015, Kumar_2001, Kumar_2019f, Li_2013, Liu_2005, Misra_2013, Ohno_2005, Sakellaridis_2008f, Sultan_2024f, Tammana_2012j, Tanrikulu_2015e, Yasin_2009, teWildt_2010}, 21 cases (29\%) were motivated by self-harm intention \cite{Al-Faham_2020k, AlShaaibi_2021b, Alao_2006i, Ali_2017, CamachoDorado_2018, Chang_2017f, Cox_2007, Csaky_1998e, Fry_2010, Li_2013, Losanoff_1996, Losanoff_1997e, Mesfin_2022a, Sakellaridis_2008f, Tammana_2012j, Tanrikulu_2015e, fjbuilsRepeatedBehaviorDeliberate2024}, 17 cases (24\%) had a psychosocial motivation \cite{Akay_2015f, Benoist_2019e, Bhattacharjee_2008, Cauchi_2002, Goldman_1998f, Hardy_2023g, Kobiela_2015, Li_2013, Naji_2012f, Qureshi_2016, Riva_2018j, Sobnach_2011f, Tay_2004, Thapa_2019f, Tupesis_2004f, Wildhaber_2005, Wnęk_2015f}, 9 cases (12\%) were motivated by protest \cite{Bhumi_2024f, Gardner_2017h, Losanoff_1996, Losanoff_1997e, Tupesis_2004f}, 9 cases (12\%) had another documented motivation \cite{Ali_2020f, Ali_2022g, Emamhadi_2018, Guinan_2019f, Peixoto_2017f, Sakellaridis_2008f, Trgo_2012f, Wadhwa_2015e, Yildiz_2016e}. \paragraph*{Object Characteristics} 51 cases (71\%) ingested a large diameter object (\textgreater{}2.5cm) \cite{Akay_2015f, Al-Faham_2020k, AlShaaibi_2021b, Alao_2006i, Ali_2017, Ali_2022g, Apikotoa_2022f, Atayan_2016, Berry_2021e, Bhasin_2014, CamachoDorado_2018, Cauchi_2002, Chang_2017f, Cox_2007, Csaky_1998e, DivsalarP._2023a, Emamhadi_2018, Gardner_2017h, Guinan_2019f, Jehangir_2019h, Jin_2023, Kariholu_2008, Kerestes_2019, Kobiela_2015, Kumar_2001, Kumar_2019f, Losanoff_1996, Losanoff_1997e, Mesfin_2022a, Misra_2013, Naji_2012f, Ohno_2005, Peixoto_2017f, Qureshi_2016, Riva_2018j, Sakellaridis_2008f, Sultan_2024f, Tanrikulu_2015e, Thapa_2019f, Trgo_2012f, Wnęk_2015f, Yildiz_2016e, fjbuilsRepeatedBehaviorDeliberate2024, teWildt_2010}, 44 cases (61\%) ingested multiple objects \cite{Ali_2020f, Apikotoa_2022f, Ataya_2013, Atayan_2016, Beecroft_1998, Bhattacharjee_2008, Bhumi_2024f, CamachoDorado_2018, Cauchi_2002, Emamhadi_2018, Farhadi_2024h, Fry_2010, Goldman_1998f, Guinan_2019f, Hardy_2023g, Jehangir_2019h, Jin_2023, Kar_2015, Kariholu_2008, Kobiela_2015, Kumar_2001, Kumar_2019f, Li_2013, Liu_2005, Losanoff_1996, Mesfin_2022a, Misra_2013, Naji_2012f, Ohno_2005, Sobnach_2011f, Sultan_2024f, Tammana_2012j, Tanrikulu_2015e, Tay_2004, Thapa_2019f, Wadhwa_2015e, Wildhaber_2005, Yasin_2009, fjbuilsRepeatedBehaviorDeliberate2024, teWildt_2010}, 34 cases (47\%) ingested a sharp object \cite{AlShaaibi_2021b, Alao_2006i, Apikotoa_2022f, Ataya_2013, Benoist_2019e, Bhasin_2014, Bhattacharjee_2008, CamachoDorado_2018, Csaky_1998e, DelgadoSalazar_2020c, DivsalarP._2023a, Emamhadi_2018, Farhadi_2024h, Fry_2010, Guinan_2019f, Hardy_2023g, Jehangir_2019h, Jin_2023, Kariholu_2008, Kobiela_2015, Kumar_2019f, Losanoff_1996, Losanoff_1997e, Mesfin_2022a, Misra_2013, Sobnach_2011f, Yasin_2009, teWildt_2010}, 32 cases (44\%) ingested a long object (\textgreater{}5cm) \cite{Al-Faham_2020k, AlShaaibi_2021b, Ali_2017, Ali_2022g, Atayan_2016, Bhasin_2014, CamachoDorado_2018, Chang_2017f, Cox_2007, Csaky_1998e, DivsalarP._2023a, Emamhadi_2018, Fry_2010, Gardner_2017h, Jin_2023, Kariholu_2008, Kerestes_2019, Kobiela_2015, Kumar_2019f, Mesfin_2022a, Misra_2013, Ohno_2005, Qureshi_2016, Sakellaridis_2008f, Sultan_2024f, Thapa_2019f, Trgo_2012f, Yasin_2009, Yildiz_2016e, teWildt_2010}, 9 cases (12\%) ingested a magnet \cite{Ali_2020f, Bhumi_2024f, Cauchi_2002, Liu_2005, Naji_2012f, Ohno_2005, Tanrikulu_2015e, Tay_2004, Wildhaber_2005}, 2 cases (3\%) ingested a button battery \cite{Berry_2021e, Bhumi_2024f}. \paragraph*{Outcomes} 48 cases (67\%) experienced a complication \cite{Ali_2017, Ali_2020f, Apikotoa_2022f, Atayan_2016, Beecroft_1998, Benoist_2019e, Berry_2021e, Bhasin_2014, Bhumi_2024f, CamachoDorado_2018, Cauchi_2002, Cox_2007, Csaky_1998e, DelgadoSalazar_2020c, DivsalarP._2023a, Emamhadi_2018, Farhadi_2024h, Fry_2010, Gardner_2017h, Goldman_1998f, Jin_2023, Kariholu_2008, Kerestes_2019, Kobiela_2015, Kumar_2001, Kumar_2019f, Liu_2005, Losanoff_1996, Mesfin_2022a, Misra_2013, Naji_2012f, Ohno_2005, Sakellaridis_2008f, Sobnach_2011f, Sultan_2024f, Tanrikulu_2015e, Tay_2004, Thapa_2019f, Trgo_2012f, Tupesis_2004f, Wildhaber_2005, Wnęk_2015f, Yasin_2009, Yildiz_2016e}, 44 cases (61\%) underwent surgery \cite{Al-Faham_2020k, AlShaaibi_2021b, Alao_2006i, Ali_2017, Ali_2020f, Atayan_2016, Beecroft_1998, Bhasin_2014, CamachoDorado_2018, Cauchi_2002, Chang_2017f, Cox_2007, Csaky_1998e, DelgadoSalazar_2020c, DivsalarP._2023a, Farhadi_2024h, Fry_2010, Gardner_2017h, Jin_2023, Kariholu_2008, Kerestes_2019, Kobiela_2015, Kumar_2019f, Liu_2005, Losanoff_1996, Losanoff_1997e, Mesfin_2022a, Misra_2013, Naji_2012f, Sobnach_2011f, Tanrikulu_2015e, Tay_2004, Thapa_2019f, Tupesis_2004f, Wildhaber_2005, Wnęk_2015f, Yasin_2009, Yildiz_2016e, fjbuilsRepeatedBehaviorDeliberate2024}, 31 cases (43\%) underwent endoscopy \cite{Akay_2015f, Ali_2022g, Apikotoa_2022f, Atayan_2016, Benoist_2019e, Berry_2021e, Bhasin_2014, Bhumi_2024f, CamachoDorado_2018, Chang_2017f, DelgadoSalazar_2020c, Gardner_2017h, Guinan_2019f, Hardy_2023g, Jehangir_2019h, Kariholu_2008, Li_2013, Liu_2005, Ohno_2005, Peixoto_2017f, Qureshi_2016, Riva_2018j, Sakellaridis_2008f, Sultan_2024f, Tammana_2012j, Tanrikulu_2015e, Trgo_2012f, Wadhwa_2015e, Wnęk_2015f, teWildt_2010}, 7 cases (10\%) were managed conservatively \cite{Ataya_2013, Bhattacharjee_2008, DivsalarP._2023a, Emamhadi_2018, Goldman_1998f, Kar_2015, Kumar_2001}, 2 cases (3\%) died \cite{Emamhadi_2018, Kumar_2001}. All 90 were male gender. 90 cases (100\%) were detained at the time of ingestion \cite{Elghali_2016, Karp_1991b, Lee_2007}, 88 cases (98\%) were intentional ingestions \cite{Elghali_2016, Karp_1991b, Lee_2007}, 30 cases (33\%) had a psychiatric history documented \cite{Elghali_2016, Karp_1991b, Lee_2007}, 2 cases (2\%) had a history of prior ingestion \cite{Elghali_2016}. No cases were reported for were psychiatric inpatients, were displaced people, were under the influence of alcohol at the time of ingestion, and had a severe disability history.
\paragraph*{Motivation}  70 cases (78\%) reported protest motivation \cite{Elghali_2016, Karp_1991b, Lee_2007}, 12 cases (13\%) reported psychiatric motivation \cite{Karp_1991b}, 6 cases (7\%) reported self-harm motivation \cite{Elghali_2016, Karp_1991b}. No cases were reported for psychosocial motivation and other motivation.
\paragraph*{Object Characteristics}  68 cases (76\%) involved sharp object ingestion \cite{Elghali_2016, Karp_1991b, Lee_2007}, 32 cases (36\%) involved long (\textgreater 5cm) object ingestion \cite{Lee_2007}, 25 cases (28\%) involved ingestion of multiple objects \cite{Elghali_2016, Lee_2007}. No cases were reported for button battery ingestion, magnet ingestion, and involved large diameter (\textgreater 2.5cm) object ingestion.
\paragraph*{Outcomes}  47 cases (52\%) underwent endoscopic intervention \cite{Elghali_2016, Lee_2007}, 29 cases (32\%) were managed conservatively \cite{Elghali_2016, Karp_1991b}, 15 cases (17\%) underwent surgical intervention \cite{Elghali_2016, Karp_1991b, Lee_2007}, 6 cases (7\%) reported complications \cite{Lee_2007}, 1 case (1\%) died \cite{Elghali_2016}.
\paragraph*{Geographical Location}Cases were recorded in 33 countries: 13 cases from USA \cite{Alao_2006i, Ataya_2013, Bhumi_2024f, Fry_2010, Guinan_2019f, Hardy_2023g, Jehangir_2019h, Kerestes_2019, Kumar_2001, Liu_2005, Tammana_2012j, Tay_2004, Tupesis_2004f}; 7 cases from India \cite{Bhasin_2014, Bhattacharjee_2008, Kar_2015, Kariholu_2008, Kumar_2019f, Misra_2013, Wadhwa_2015e} and UK \cite{Beecroft_1998, Berry_2021e, Cauchi_2002, Cox_2007, Gardner_2017h, Qureshi_2016}; 6 cases from Bulgaria \cite{Losanoff_1996, Losanoff_1997e}; 5 cases from Iran \cite{DivsalarP._2023a, Emamhadi_2018, Farhadi_2024h}; 4 cases from Turkey \cite{Akay_2015f, Atayan_2016, Tanrikulu_2015e, Yildiz_2016e}; 2 cases from China \cite{Jin_2023, Li_2013}, Poland \cite{Kobiela_2015, Wnęk_2015f}, and Spain \cite{CamachoDorado_2018, fjbuilsRepeatedBehaviorDeliberate2024}; 1 case from Australia \cite{Apikotoa_2022f}, Bahrain \cite{Ali_2020f}, Croatia \cite{Trgo_2012f}, Ecuador \cite{DelgadoSalazar_2020c}, Egypt \cite{Ali_2022g}, Ethiopia \cite{Mesfin_2022a}, Germany \cite{teWildt_2010}, Greece \cite{Sakellaridis_2008f}, Hungary \cite{Csaky_1998e}, Iraq \cite{Al-Faham_2020k}, Israel \cite{Goldman_1998f}, Italy \cite{Riva_2018j}, Japan \cite{Ohno_2005}, Nepal \cite{Thapa_2019f}, Netherlands \cite{Benoist_2019e}, Oman \cite{AlShaaibi_2021b}, Pakistan \cite{Yasin_2009}, Portugal \cite{Peixoto_2017f}, Qatar \cite{Ali_2017}, Saudi Arabia \cite{Sultan_2024f}, South Africa \cite{Sobnach_2011f}, Sweden \cite{Naji_2012f}, Switzerland \cite{Wildhaber_2005}, and Taiwan \cite{Chang_2017f}. \paragraph*{Gender} 43 cases (60\%) were male \cite{Akay_2015f, Al-Faham_2020k, Alao_2006i, Ali_2017, Ali_2022g, Apikotoa_2022f, Atayan_2016, Benoist_2019e, Berry_2021e, Bhumi_2024f, CamachoDorado_2018, Csaky_1998e, Emamhadi_2018, Farhadi_2024h, Fry_2010, Gardner_2017h, Guinan_2019f, Jehangir_2019h, Jin_2023, Kobiela_2015, Kumar_2001, Kumar_2019f, Liu_2005, Losanoff_1996, Losanoff_1997e, Mesfin_2022a, Misra_2013, Qureshi_2016, Riva_2018j, Sobnach_2011f, Tammana_2012j, Tanrikulu_2015e, Tay_2004, Thapa_2019f, Trgo_2012f, Wadhwa_2015e, Yasin_2009, teWildt_2010}, 28 cases (39\%) were female \cite{AlShaaibi_2021b, Ali_2020f, Ataya_2013, Beecroft_1998, Bhasin_2014, Bhattacharjee_2008, Cauchi_2002, Chang_2017f, Cox_2007, DelgadoSalazar_2020c, DivsalarP._2023a, Goldman_1998f, Hardy_2023g, Kar_2015, Kariholu_2008, Kerestes_2019, Li_2013, Naji_2012f, Ohno_2005, Peixoto_2017f, Sakellaridis_2008f, Sultan_2024f, Tupesis_2004f, Wildhaber_2005, Wnęk_2015f, Yildiz_2016e}, 1 case (1\%) had no gender recorded \cite{fjbuilsRepeatedBehaviorDeliberate2024}. \paragraph*{Age Group} 25 cases (35\%) were between 26 and 40 years of age \cite{Alao_2006i, Ali_2022g, Apikotoa_2022f, Ataya_2013, Benoist_2019e, Bhasin_2014, Chang_2017f, Cox_2007, DelgadoSalazar_2020c, Farhadi_2024h, Fry_2010, Gardner_2017h, Guinan_2019f, Jin_2023, Kumar_2019f, Losanoff_1996, Misra_2013, Qureshi_2016, Riva_2018j, Sakellaridis_2008f, Tammana_2012j, Trgo_2012f, Wnęk_2015f, Yildiz_2016e, fjbuilsRepeatedBehaviorDeliberate2024}, 18 cases (25\%) were between 18 and 25 years of age \cite{Akay_2015f, Ali_2017, Atayan_2016, Bhattacharjee_2008, Csaky_1998e, Kar_2015, Kariholu_2008, Kobiela_2015, Losanoff_1996, Losanoff_1997e, Mesfin_2022a, Peixoto_2017f, Sobnach_2011f, Tupesis_2004f, Yasin_2009}, 13 cases (18\%) were under 18 years of age \cite{AlShaaibi_2021b, Ali_2020f, Cauchi_2002, DivsalarP._2023a, Goldman_1998f, Liu_2005, Naji_2012f, Ohno_2005, Tanrikulu_2015e, Tay_2004, Wildhaber_2005}, 11 cases (15\%) were between 41 and 60 years of age \cite{Al-Faham_2020k, Bhumi_2024f, CamachoDorado_2018, Emamhadi_2018, Hardy_2023g, Jehangir_2019h, Kumar_2001, Sultan_2024f, Thapa_2019f, Wadhwa_2015e, teWildt_2010}, 3 cases (4\%) were over 60 years of age \cite{Beecroft_1998, Kerestes_2019, Li_2013}, 2 cases (3\%) had no age documented \cite{Berry_2021e}. \paragraph*{Population} 36 cases (50\%) had a psychiatric history \cite{AlShaaibi_2021b, Alao_2006i, Ali_2020f, Apikotoa_2022f, Ataya_2013, Atayan_2016, Beecroft_1998, CamachoDorado_2018, Chang_2017f, DelgadoSalazar_2020c, DivsalarP._2023a, Farhadi_2024h, Fry_2010, Guinan_2019f, Hardy_2023g, Jehangir_2019h, Jin_2023, Kar_2015, Kerestes_2019, Kobiela_2015, Kumar_2001, Kumar_2019f, Liu_2005, Mesfin_2022a, Misra_2013, Ohno_2005, Peixoto_2017f, Sakellaridis_2008f, Sultan_2024f, Tammana_2012j, Tanrikulu_2015e, Yildiz_2016e, fjbuilsRepeatedBehaviorDeliberate2024, teWildt_2010}, 19 cases (26\%) had ingested previously \cite{Alao_2006i, Apikotoa_2022f, Berry_2021e, Bhattacharjee_2008, Csaky_1998e, DivsalarP._2023a, Emamhadi_2018, Guinan_2019f, Jehangir_2019h, Jin_2023, Liu_2005, Sakellaridis_2008f, Tanrikulu_2015e, Thapa_2019f, Yildiz_2016e, fjbuilsRepeatedBehaviorDeliberate2024, teWildt_2010}, 12 cases (17\%) were detained persons \cite{Alao_2006i, Ali_2022g, Apikotoa_2022f, Losanoff_1996, Losanoff_1997e, Qureshi_2016, Tammana_2012j, Trgo_2012f}, 7 cases (10\%) were severely disabled \cite{Atayan_2016, Kerestes_2019, Liu_2005, Ohno_2005, Peixoto_2017f, Yildiz_2016e, teWildt_2010}, 4 cases (6\%) were psychiatric inpatients \cite{DivsalarP._2023a, fjbuilsRepeatedBehaviorDeliberate2024, teWildt_2010}, 3 cases (4\%) were under the influence of alcohol \cite{Benoist_2019e, Csaky_1998e, Thapa_2019f}, 2 cases (3\%) were displaced people \cite{Akay_2015f, Gardner_2017h}. \paragraph*{Motivation} 34 cases (47\%) had a psychiatric motivation \cite{Al-Faham_2020k, Alao_2006i, Ali_2020f, Apikotoa_2022f, Ataya_2013, Atayan_2016, Bhasin_2014, Bhattacharjee_2008, DelgadoSalazar_2020c, DivsalarP._2023a, Emamhadi_2018, Farhadi_2024h, Guinan_2019f, Hardy_2023g, Jehangir_2019h, Jin_2023, Kar_2015, Kariholu_2008, Kerestes_2019, Kobiela_2015, Kumar_2001, Kumar_2019f, Li_2013, Liu_2005, Misra_2013, Ohno_2005, Sakellaridis_2008f, Sultan_2024f, Tammana_2012j, Tanrikulu_2015e, Yasin_2009, teWildt_2010}, 21 cases (29\%) were motivated by self-harm intention \cite{Al-Faham_2020k, AlShaaibi_2021b, Alao_2006i, Ali_2017, CamachoDorado_2018, Chang_2017f, Cox_2007, Csaky_1998e, Fry_2010, Li_2013, Losanoff_1996, Losanoff_1997e, Mesfin_2022a, Sakellaridis_2008f, Tammana_2012j, Tanrikulu_2015e, fjbuilsRepeatedBehaviorDeliberate2024}, 17 cases (24\%) had a psychosocial motivation \cite{Akay_2015f, Benoist_2019e, Bhattacharjee_2008, Cauchi_2002, Goldman_1998f, Hardy_2023g, Kobiela_2015, Li_2013, Naji_2012f, Qureshi_2016, Riva_2018j, Sobnach_2011f, Tay_2004, Thapa_2019f, Tupesis_2004f, Wildhaber_2005, Wnęk_2015f}, 9 cases (12\%) were motivated by protest \cite{Bhumi_2024f, Gardner_2017h, Losanoff_1996, Losanoff_1997e, Tupesis_2004f}, 9 cases (12\%) had another documented motivation \cite{Ali_2020f, Ali_2022g, Emamhadi_2018, Guinan_2019f, Peixoto_2017f, Sakellaridis_2008f, Trgo_2012f, Wadhwa_2015e, Yildiz_2016e}. \paragraph*{Object Characteristics} 51 cases (71\%) ingested a large diameter object (\textgreater{}2.5cm) \cite{Akay_2015f, Al-Faham_2020k, AlShaaibi_2021b, Alao_2006i, Ali_2017, Ali_2022g, Apikotoa_2022f, Atayan_2016, Berry_2021e, Bhasin_2014, CamachoDorado_2018, Cauchi_2002, Chang_2017f, Cox_2007, Csaky_1998e, DivsalarP._2023a, Emamhadi_2018, Gardner_2017h, Guinan_2019f, Jehangir_2019h, Jin_2023, Kariholu_2008, Kerestes_2019, Kobiela_2015, Kumar_2001, Kumar_2019f, Losanoff_1996, Losanoff_1997e, Mesfin_2022a, Misra_2013, Naji_2012f, Ohno_2005, Peixoto_2017f, Qureshi_2016, Riva_2018j, Sakellaridis_2008f, Sultan_2024f, Tanrikulu_2015e, Thapa_2019f, Trgo_2012f, Wnęk_2015f, Yildiz_2016e, fjbuilsRepeatedBehaviorDeliberate2024, teWildt_2010}, 44 cases (61\%) ingested multiple objects \cite{Ali_2020f, Apikotoa_2022f, Ataya_2013, Atayan_2016, Beecroft_1998, Bhattacharjee_2008, Bhumi_2024f, CamachoDorado_2018, Cauchi_2002, Emamhadi_2018, Farhadi_2024h, Fry_2010, Goldman_1998f, Guinan_2019f, Hardy_2023g, Jehangir_2019h, Jin_2023, Kar_2015, Kariholu_2008, Kobiela_2015, Kumar_2001, Kumar_2019f, Li_2013, Liu_2005, Losanoff_1996, Mesfin_2022a, Misra_2013, Naji_2012f, Ohno_2005, Sobnach_2011f, Sultan_2024f, Tammana_2012j, Tanrikulu_2015e, Tay_2004, Thapa_2019f, Wadhwa_2015e, Wildhaber_2005, Yasin_2009, fjbuilsRepeatedBehaviorDeliberate2024, teWildt_2010}, 34 cases (47\%) ingested a sharp object \cite{AlShaaibi_2021b, Alao_2006i, Apikotoa_2022f, Ataya_2013, Benoist_2019e, Bhasin_2014, Bhattacharjee_2008, CamachoDorado_2018, Csaky_1998e, DelgadoSalazar_2020c, DivsalarP._2023a, Emamhadi_2018, Farhadi_2024h, Fry_2010, Guinan_2019f, Hardy_2023g, Jehangir_2019h, Jin_2023, Kariholu_2008, Kobiela_2015, Kumar_2019f, Losanoff_1996, Losanoff_1997e, Mesfin_2022a, Misra_2013, Sobnach_2011f, Yasin_2009, teWildt_2010}, 32 cases (44\%) ingested a long object (\textgreater{}5cm) \cite{Al-Faham_2020k, AlShaaibi_2021b, Ali_2017, Ali_2022g, Atayan_2016, Bhasin_2014, CamachoDorado_2018, Chang_2017f, Cox_2007, Csaky_1998e, DivsalarP._2023a, Emamhadi_2018, Fry_2010, Gardner_2017h, Jin_2023, Kariholu_2008, Kerestes_2019, Kobiela_2015, Kumar_2019f, Mesfin_2022a, Misra_2013, Ohno_2005, Qureshi_2016, Sakellaridis_2008f, Sultan_2024f, Thapa_2019f, Trgo_2012f, Yasin_2009, Yildiz_2016e, teWildt_2010}, 9 cases (12\%) ingested a magnet \cite{Ali_2020f, Bhumi_2024f, Cauchi_2002, Liu_2005, Naji_2012f, Ohno_2005, Tanrikulu_2015e, Tay_2004, Wildhaber_2005}, 2 cases (3\%) ingested a button battery \cite{Berry_2021e, Bhumi_2024f}. \paragraph*{Outcomes} 48 cases (67\%) experienced a complication \cite{Ali_2017, Ali_2020f, Apikotoa_2022f, Atayan_2016, Beecroft_1998, Benoist_2019e, Berry_2021e, Bhasin_2014, Bhumi_2024f, CamachoDorado_2018, Cauchi_2002, Cox_2007, Csaky_1998e, DelgadoSalazar_2020c, DivsalarP._2023a, Emamhadi_2018, Farhadi_2024h, Fry_2010, Gardner_2017h, Goldman_1998f, Jin_2023, Kariholu_2008, Kerestes_2019, Kobiela_2015, Kumar_2001, Kumar_2019f, Liu_2005, Losanoff_1996, Mesfin_2022a, Misra_2013, Naji_2012f, Ohno_2005, Sakellaridis_2008f, Sobnach_2011f, Sultan_2024f, Tanrikulu_2015e, Tay_2004, Thapa_2019f, Trgo_2012f, Tupesis_2004f, Wildhaber_2005, Wnęk_2015f, Yasin_2009, Yildiz_2016e}, 44 cases (61\%) underwent surgery \cite{Al-Faham_2020k, AlShaaibi_2021b, Alao_2006i, Ali_2017, Ali_2020f, Atayan_2016, Beecroft_1998, Bhasin_2014, CamachoDorado_2018, Cauchi_2002, Chang_2017f, Cox_2007, Csaky_1998e, DelgadoSalazar_2020c, DivsalarP._2023a, Farhadi_2024h, Fry_2010, Gardner_2017h, Jin_2023, Kariholu_2008, Kerestes_2019, Kobiela_2015, Kumar_2019f, Liu_2005, Losanoff_1996, Losanoff_1997e, Mesfin_2022a, Misra_2013, Naji_2012f, Sobnach_2011f, Tanrikulu_2015e, Tay_2004, Thapa_2019f, Tupesis_2004f, Wildhaber_2005, Wnęk_2015f, Yasin_2009, Yildiz_2016e, fjbuilsRepeatedBehaviorDeliberate2024}, 31 cases (43\%) underwent endoscopy \cite{Akay_2015f, Ali_2022g, Apikotoa_2022f, Atayan_2016, Benoist_2019e, Berry_2021e, Bhasin_2014, Bhumi_2024f, CamachoDorado_2018, Chang_2017f, DelgadoSalazar_2020c, Gardner_2017h, Guinan_2019f, Hardy_2023g, Jehangir_2019h, Kariholu_2008, Li_2013, Liu_2005, Ohno_2005, Peixoto_2017f, Qureshi_2016, Riva_2018j, Sakellaridis_2008f, Sultan_2024f, Tammana_2012j, Tanrikulu_2015e, Trgo_2012f, Wadhwa_2015e, Wnęk_2015f, teWildt_2010}, 7 cases (10\%) were managed conservatively \cite{Ataya_2013, Bhattacharjee_2008, DivsalarP._2023a, Emamhadi_2018, Goldman_1998f, Kar_2015, Kumar_2001}, 2 cases (3\%) died \cite{Emamhadi_2018, Kumar_2001}. All 90 were male gender. 90 cases (100\%) were detained at the time of ingestion \cite{Elghali_2016, Karp_1991b, Lee_2007}, 88 cases (98\%) were intentional ingestions \cite{Elghali_2016, Karp_1991b, Lee_2007}, 30 cases (33\%) had a psychiatric history documented \cite{Elghali_2016, Karp_1991b, Lee_2007}, 2 cases (2\%) had a history of prior ingestion \cite{Elghali_2016}. No cases were reported for were psychiatric inpatients, were displaced people, were under the influence of alcohol at the time of ingestion, and had a severe disability history.
\paragraph*{Motivation}  70 cases (78\%) reported protest motivation \cite{Elghali_2016, Karp_1991b, Lee_2007}, 12 cases (13\%) reported psychiatric motivation \cite{Karp_1991b}, 6 cases (7\%) reported self-harm motivation \cite{Elghali_2016, Karp_1991b}. No cases were reported for psychosocial motivation and other motivation.
\paragraph*{Object Characteristics}  68 cases (76\%) involved sharp object ingestion \cite{Elghali_2016, Karp_1991b, Lee_2007}, 32 cases (36\%) involved long (\textgreater 5cm) object ingestion \cite{Lee_2007}, 25 cases (28\%) involved ingestion of multiple objects \cite{Elghali_2016, Lee_2007}. No cases were reported for button battery ingestion, magnet ingestion, and involved large diameter (\textgreater 2.5cm) object ingestion.
\paragraph*{Outcomes}  47 cases (52\%) underwent endoscopic intervention \cite{Elghali_2016, Lee_2007}, 29 cases (32\%) were managed conservatively \cite{Elghali_2016, Karp_1991b}, 15 cases (17\%) underwent surgical intervention \cite{Elghali_2016, Karp_1991b, Lee_2007}, 6 cases (7\%) reported complications \cite{Lee_2007}, 1 case (1\%) died \cite{Elghali_2016}.
\paragraph*{Geographical Location}Cases were recorded in 33 countries: 13 cases from USA \cite{Alao_2006i, Ataya_2013, Bhumi_2024f, Fry_2010, Guinan_2019f, Hardy_2023g, Jehangir_2019h, Kerestes_2019, Kumar_2001, Liu_2005, Tammana_2012j, Tay_2004, Tupesis_2004f}; 7 cases from India \cite{Bhasin_2014, Bhattacharjee_2008, Kar_2015, Kariholu_2008, Kumar_2019f, Misra_2013, Wadhwa_2015e} and UK \cite{Beecroft_1998, Berry_2021e, Cauchi_2002, Cox_2007, Gardner_2017h, Qureshi_2016}; 6 cases from Bulgaria \cite{Losanoff_1996, Losanoff_1997e}; 5 cases from Iran \cite{DivsalarP._2023a, Emamhadi_2018, Farhadi_2024h}; 4 cases from Turkey \cite{Akay_2015f, Atayan_2016, Tanrikulu_2015e, Yildiz_2016e}; 2 cases from China \cite{Jin_2023, Li_2013}, Poland \cite{Kobiela_2015, Wnęk_2015f}, and Spain \cite{CamachoDorado_2018, fjbuilsRepeatedBehaviorDeliberate2024}; 1 case from Australia \cite{Apikotoa_2022f}, Bahrain \cite{Ali_2020f}, Croatia \cite{Trgo_2012f}, Ecuador \cite{DelgadoSalazar_2020c}, Egypt \cite{Ali_2022g}, Ethiopia \cite{Mesfin_2022a}, Germany \cite{teWildt_2010}, Greece \cite{Sakellaridis_2008f}, Hungary \cite{Csaky_1998e}, Iraq \cite{Al-Faham_2020k}, Israel \cite{Goldman_1998f}, Italy \cite{Riva_2018j}, Japan \cite{Ohno_2005}, Nepal \cite{Thapa_2019f}, Netherlands \cite{Benoist_2019e}, Oman \cite{AlShaaibi_2021b}, Pakistan \cite{Yasin_2009}, Portugal \cite{Peixoto_2017f}, Qatar \cite{Ali_2017}, Saudi Arabia \cite{Sultan_2024f}, South Africa \cite{Sobnach_2011f}, Sweden \cite{Naji_2012f}, Switzerland \cite{Wildhaber_2005}, and Taiwan \cite{Chang_2017f}. \paragraph*{Gender} 43 cases (60\%) were male \cite{Akay_2015f, Al-Faham_2020k, Alao_2006i, Ali_2017, Ali_2022g, Apikotoa_2022f, Atayan_2016, Benoist_2019e, Berry_2021e, Bhumi_2024f, CamachoDorado_2018, Csaky_1998e, Emamhadi_2018, Farhadi_2024h, Fry_2010, Gardner_2017h, Guinan_2019f, Jehangir_2019h, Jin_2023, Kobiela_2015, Kumar_2001, Kumar_2019f, Liu_2005, Losanoff_1996, Losanoff_1997e, Mesfin_2022a, Misra_2013, Qureshi_2016, Riva_2018j, Sobnach_2011f, Tammana_2012j, Tanrikulu_2015e, Tay_2004, Thapa_2019f, Trgo_2012f, Wadhwa_2015e, Yasin_2009, teWildt_2010}, 28 cases (39\%) were female \cite{AlShaaibi_2021b, Ali_2020f, Ataya_2013, Beecroft_1998, Bhasin_2014, Bhattacharjee_2008, Cauchi_2002, Chang_2017f, Cox_2007, DelgadoSalazar_2020c, DivsalarP._2023a, Goldman_1998f, Hardy_2023g, Kar_2015, Kariholu_2008, Kerestes_2019, Li_2013, Naji_2012f, Ohno_2005, Peixoto_2017f, Sakellaridis_2008f, Sultan_2024f, Tupesis_2004f, Wildhaber_2005, Wnęk_2015f, Yildiz_2016e}, 1 case (1\%) had no gender recorded \cite{fjbuilsRepeatedBehaviorDeliberate2024}. \paragraph*{Age Group} 25 cases (35\%) were between 26 and 40 years of age \cite{Alao_2006i, Ali_2022g, Apikotoa_2022f, Ataya_2013, Benoist_2019e, Bhasin_2014, Chang_2017f, Cox_2007, DelgadoSalazar_2020c, Farhadi_2024h, Fry_2010, Gardner_2017h, Guinan_2019f, Jin_2023, Kumar_2019f, Losanoff_1996, Misra_2013, Qureshi_2016, Riva_2018j, Sakellaridis_2008f, Tammana_2012j, Trgo_2012f, Wnęk_2015f, Yildiz_2016e, fjbuilsRepeatedBehaviorDeliberate2024}, 18 cases (25\%) were between 18 and 25 years of age \cite{Akay_2015f, Ali_2017, Atayan_2016, Bhattacharjee_2008, Csaky_1998e, Kar_2015, Kariholu_2008, Kobiela_2015, Losanoff_1996, Losanoff_1997e, Mesfin_2022a, Peixoto_2017f, Sobnach_2011f, Tupesis_2004f, Yasin_2009}, 13 cases (18\%) were under 18 years of age \cite{AlShaaibi_2021b, Ali_2020f, Cauchi_2002, DivsalarP._2023a, Goldman_1998f, Liu_2005, Naji_2012f, Ohno_2005, Tanrikulu_2015e, Tay_2004, Wildhaber_2005}, 11 cases (15\%) were between 41 and 60 years of age \cite{Al-Faham_2020k, Bhumi_2024f, CamachoDorado_2018, Emamhadi_2018, Hardy_2023g, Jehangir_2019h, Kumar_2001, Sultan_2024f, Thapa_2019f, Wadhwa_2015e, teWildt_2010}, 3 cases (4\%) were over 60 years of age \cite{Beecroft_1998, Kerestes_2019, Li_2013}, 2 cases (3\%) had no age documented \cite{Berry_2021e}. \paragraph*{Population} 36 cases (50\%) had a psychiatric history \cite{AlShaaibi_2021b, Alao_2006i, Ali_2020f, Apikotoa_2022f, Ataya_2013, Atayan_2016, Beecroft_1998, CamachoDorado_2018, Chang_2017f, DelgadoSalazar_2020c, DivsalarP._2023a, Farhadi_2024h, Fry_2010, Guinan_2019f, Hardy_2023g, Jehangir_2019h, Jin_2023, Kar_2015, Kerestes_2019, Kobiela_2015, Kumar_2001, Kumar_2019f, Liu_2005, Mesfin_2022a, Misra_2013, Ohno_2005, Peixoto_2017f, Sakellaridis_2008f, Sultan_2024f, Tammana_2012j, Tanrikulu_2015e, Yildiz_2016e, fjbuilsRepeatedBehaviorDeliberate2024, teWildt_2010}, 19 cases (26\%) had ingested previously \cite{Alao_2006i, Apikotoa_2022f, Berry_2021e, Bhattacharjee_2008, Csaky_1998e, DivsalarP._2023a, Emamhadi_2018, Guinan_2019f, Jehangir_2019h, Jin_2023, Liu_2005, Sakellaridis_2008f, Tanrikulu_2015e, Thapa_2019f, Yildiz_2016e, fjbuilsRepeatedBehaviorDeliberate2024, teWildt_2010}, 12 cases (17\%) were detained persons \cite{Alao_2006i, Ali_2022g, Apikotoa_2022f, Losanoff_1996, Losanoff_1997e, Qureshi_2016, Tammana_2012j, Trgo_2012f}, 7 cases (10\%) were severely disabled \cite{Atayan_2016, Kerestes_2019, Liu_2005, Ohno_2005, Peixoto_2017f, Yildiz_2016e, teWildt_2010}, 4 cases (6\%) were psychiatric inpatients \cite{DivsalarP._2023a, fjbuilsRepeatedBehaviorDeliberate2024, teWildt_2010}, 3 cases (4\%) were under the influence of alcohol \cite{Benoist_2019e, Csaky_1998e, Thapa_2019f}, 2 cases (3\%) were displaced people \cite{Akay_2015f, Gardner_2017h}. \paragraph*{Motivation} 34 cases (47\%) had a psychiatric motivation \cite{Al-Faham_2020k, Alao_2006i, Ali_2020f, Apikotoa_2022f, Ataya_2013, Atayan_2016, Bhasin_2014, Bhattacharjee_2008, DelgadoSalazar_2020c, DivsalarP._2023a, Emamhadi_2018, Farhadi_2024h, Guinan_2019f, Hardy_2023g, Jehangir_2019h, Jin_2023, Kar_2015, Kariholu_2008, Kerestes_2019, Kobiela_2015, Kumar_2001, Kumar_2019f, Li_2013, Liu_2005, Misra_2013, Ohno_2005, Sakellaridis_2008f, Sultan_2024f, Tammana_2012j, Tanrikulu_2015e, Yasin_2009, teWildt_2010}, 21 cases (29\%) were motivated by self-harm intention \cite{Al-Faham_2020k, AlShaaibi_2021b, Alao_2006i, Ali_2017, CamachoDorado_2018, Chang_2017f, Cox_2007, Csaky_1998e, Fry_2010, Li_2013, Losanoff_1996, Losanoff_1997e, Mesfin_2022a, Sakellaridis_2008f, Tammana_2012j, Tanrikulu_2015e, fjbuilsRepeatedBehaviorDeliberate2024}, 17 cases (24\%) had a psychosocial motivation \cite{Akay_2015f, Benoist_2019e, Bhattacharjee_2008, Cauchi_2002, Goldman_1998f, Hardy_2023g, Kobiela_2015, Li_2013, Naji_2012f, Qureshi_2016, Riva_2018j, Sobnach_2011f, Tay_2004, Thapa_2019f, Tupesis_2004f, Wildhaber_2005, Wnęk_2015f}, 9 cases (12\%) were motivated by protest \cite{Bhumi_2024f, Gardner_2017h, Losanoff_1996, Losanoff_1997e, Tupesis_2004f}, 9 cases (12\%) had another documented motivation \cite{Ali_2020f, Ali_2022g, Emamhadi_2018, Guinan_2019f, Peixoto_2017f, Sakellaridis_2008f, Trgo_2012f, Wadhwa_2015e, Yildiz_2016e}. \paragraph*{Object Characteristics} 51 cases (71\%) ingested a large diameter object (\textgreater{}2.5cm) \cite{Akay_2015f, Al-Faham_2020k, AlShaaibi_2021b, Alao_2006i, Ali_2017, Ali_2022g, Apikotoa_2022f, Atayan_2016, Berry_2021e, Bhasin_2014, CamachoDorado_2018, Cauchi_2002, Chang_2017f, Cox_2007, Csaky_1998e, DivsalarP._2023a, Emamhadi_2018, Gardner_2017h, Guinan_2019f, Jehangir_2019h, Jin_2023, Kariholu_2008, Kerestes_2019, Kobiela_2015, Kumar_2001, Kumar_2019f, Losanoff_1996, Losanoff_1997e, Mesfin_2022a, Misra_2013, Naji_2012f, Ohno_2005, Peixoto_2017f, Qureshi_2016, Riva_2018j, Sakellaridis_2008f, Sultan_2024f, Tanrikulu_2015e, Thapa_2019f, Trgo_2012f, Wnęk_2015f, Yildiz_2016e, fjbuilsRepeatedBehaviorDeliberate2024, teWildt_2010}, 44 cases (61\%) ingested multiple objects \cite{Ali_2020f, Apikotoa_2022f, Ataya_2013, Atayan_2016, Beecroft_1998, Bhattacharjee_2008, Bhumi_2024f, CamachoDorado_2018, Cauchi_2002, Emamhadi_2018, Farhadi_2024h, Fry_2010, Goldman_1998f, Guinan_2019f, Hardy_2023g, Jehangir_2019h, Jin_2023, Kar_2015, Kariholu_2008, Kobiela_2015, Kumar_2001, Kumar_2019f, Li_2013, Liu_2005, Losanoff_1996, Mesfin_2022a, Misra_2013, Naji_2012f, Ohno_2005, Sobnach_2011f, Sultan_2024f, Tammana_2012j, Tanrikulu_2015e, Tay_2004, Thapa_2019f, Wadhwa_2015e, Wildhaber_2005, Yasin_2009, fjbuilsRepeatedBehaviorDeliberate2024, teWildt_2010}, 34 cases (47\%) ingested a sharp object \cite{AlShaaibi_2021b, Alao_2006i, Apikotoa_2022f, Ataya_2013, Benoist_2019e, Bhasin_2014, Bhattacharjee_2008, CamachoDorado_2018, Csaky_1998e, DelgadoSalazar_2020c, DivsalarP._2023a, Emamhadi_2018, Farhadi_2024h, Fry_2010, Guinan_2019f, Hardy_2023g, Jehangir_2019h, Jin_2023, Kariholu_2008, Kobiela_2015, Kumar_2019f, Losanoff_1996, Losanoff_1997e, Mesfin_2022a, Misra_2013, Sobnach_2011f, Yasin_2009, teWildt_2010}, 32 cases (44\%) ingested a long object (\textgreater{}5cm) \cite{Al-Faham_2020k, AlShaaibi_2021b, Ali_2017, Ali_2022g, Atayan_2016, Bhasin_2014, CamachoDorado_2018, Chang_2017f, Cox_2007, Csaky_1998e, DivsalarP._2023a, Emamhadi_2018, Fry_2010, Gardner_2017h, Jin_2023, Kariholu_2008, Kerestes_2019, Kobiela_2015, Kumar_2019f, Mesfin_2022a, Misra_2013, Ohno_2005, Qureshi_2016, Sakellaridis_2008f, Sultan_2024f, Thapa_2019f, Trgo_2012f, Yasin_2009, Yildiz_2016e, teWildt_2010}, 9 cases (12\%) ingested a magnet \cite{Ali_2020f, Bhumi_2024f, Cauchi_2002, Liu_2005, Naji_2012f, Ohno_2005, Tanrikulu_2015e, Tay_2004, Wildhaber_2005}, 2 cases (3\%) ingested a button battery \cite{Berry_2021e, Bhumi_2024f}. \paragraph*{Outcomes} 48 cases (67\%) experienced a complication \cite{Ali_2017, Ali_2020f, Apikotoa_2022f, Atayan_2016, Beecroft_1998, Benoist_2019e, Berry_2021e, Bhasin_2014, Bhumi_2024f, CamachoDorado_2018, Cauchi_2002, Cox_2007, Csaky_1998e, DelgadoSalazar_2020c, DivsalarP._2023a, Emamhadi_2018, Farhadi_2024h, Fry_2010, Gardner_2017h, Goldman_1998f, Jin_2023, Kariholu_2008, Kerestes_2019, Kobiela_2015, Kumar_2001, Kumar_2019f, Liu_2005, Losanoff_1996, Mesfin_2022a, Misra_2013, Naji_2012f, Ohno_2005, Sakellaridis_2008f, Sobnach_2011f, Sultan_2024f, Tanrikulu_2015e, Tay_2004, Thapa_2019f, Trgo_2012f, Tupesis_2004f, Wildhaber_2005, Wnęk_2015f, Yasin_2009, Yildiz_2016e}, 44 cases (61\%) underwent surgery \cite{Al-Faham_2020k, AlShaaibi_2021b, Alao_2006i, Ali_2017, Ali_2020f, Atayan_2016, Beecroft_1998, Bhasin_2014, CamachoDorado_2018, Cauchi_2002, Chang_2017f, Cox_2007, Csaky_1998e, DelgadoSalazar_2020c, DivsalarP._2023a, Farhadi_2024h, Fry_2010, Gardner_2017h, Jin_2023, Kariholu_2008, Kerestes_2019, Kobiela_2015, Kumar_2019f, Liu_2005, Losanoff_1996, Losanoff_1997e, Mesfin_2022a, Misra_2013, Naji_2012f, Sobnach_2011f, Tanrikulu_2015e, Tay_2004, Thapa_2019f, Tupesis_2004f, Wildhaber_2005, Wnęk_2015f, Yasin_2009, Yildiz_2016e, fjbuilsRepeatedBehaviorDeliberate2024}, 31 cases (43\%) underwent endoscopy \cite{Akay_2015f, Ali_2022g, Apikotoa_2022f, Atayan_2016, Benoist_2019e, Berry_2021e, Bhasin_2014, Bhumi_2024f, CamachoDorado_2018, Chang_2017f, DelgadoSalazar_2020c, Gardner_2017h, Guinan_2019f, Hardy_2023g, Jehangir_2019h, Kariholu_2008, Li_2013, Liu_2005, Ohno_2005, Peixoto_2017f, Qureshi_2016, Riva_2018j, Sakellaridis_2008f, Sultan_2024f, Tammana_2012j, Tanrikulu_2015e, Trgo_2012f, Wadhwa_2015e, Wnęk_2015f, teWildt_2010}, 7 cases (10\%) were managed conservatively \cite{Ataya_2013, Bhattacharjee_2008, DivsalarP._2023a, Emamhadi_2018, Goldman_1998f, Kar_2015, Kumar_2001}, 2 cases (3\%) died \cite{Emamhadi_2018, Kumar_2001}. All 90 were male gender. 90 cases (100\%) were detained at the time of ingestion \cite{Elghali_2016, Karp_1991b, Lee_2007}, 88 cases (98\%) were intentional ingestions \cite{Elghali_2016, Karp_1991b, Lee_2007}, 30 cases (33\%) had a psychiatric history documented \cite{Elghali_2016, Karp_1991b, Lee_2007}, 2 cases (2\%) had a history of prior ingestion \cite{Elghali_2016}. No cases were reported for were psychiatric inpatients, were displaced people, were under the influence of alcohol at the time of ingestion, and had a severe disability history.
\paragraph*{Motivation}  70 cases (78\%) reported protest motivation \cite{Elghali_2016, Karp_1991b, Lee_2007}, 12 cases (13\%) reported psychiatric motivation \cite{Karp_1991b}, 6 cases (7\%) reported self-harm motivation \cite{Elghali_2016, Karp_1991b}. No cases were reported for psychosocial motivation and other motivation.
\paragraph*{Object Characteristics}  68 cases (76\%) involved sharp object ingestion \cite{Elghali_2016, Karp_1991b, Lee_2007}, 32 cases (36\%) involved long (\textgreater 5cm) object ingestion \cite{Lee_2007}, 25 cases (28\%) involved ingestion of multiple objects \cite{Elghali_2016, Lee_2007}. No cases were reported for button battery ingestion, magnet ingestion, and involved large diameter (\textgreater 2.5cm) object ingestion.
\paragraph*{Outcomes}  47 cases (52\%) underwent endoscopic intervention \cite{Elghali_2016, Lee_2007}, 29 cases (32\%) were managed conservatively \cite{Elghali_2016, Karp_1991b}, 15 cases (17\%) underwent surgical intervention \cite{Elghali_2016, Karp_1991b, Lee_2007}, 6 cases (7\%) reported complications \cite{Lee_2007}, 1 case (1\%) died \cite{Elghali_2016}.
\paragraph*{Geographical Location}Cases were recorded in 33 countries: 13 cases from USA \cite{Alao_2006i, Ataya_2013, Bhumi_2024f, Fry_2010, Guinan_2019f, Hardy_2023g, Jehangir_2019h, Kerestes_2019, Kumar_2001, Liu_2005, Tammana_2012j, Tay_2004, Tupesis_2004f}; 7 cases from India \cite{Bhasin_2014, Bhattacharjee_2008, Kar_2015, Kariholu_2008, Kumar_2019f, Misra_2013, Wadhwa_2015e} and UK \cite{Beecroft_1998, Berry_2021e, Cauchi_2002, Cox_2007, Gardner_2017h, Qureshi_2016}; 6 cases from Bulgaria \cite{Losanoff_1996, Losanoff_1997e}; 5 cases from Iran \cite{DivsalarP._2023a, Emamhadi_2018, Farhadi_2024h}; 4 cases from Turkey \cite{Akay_2015f, Atayan_2016, Tanrikulu_2015e, Yildiz_2016e}; 2 cases from China \cite{Jin_2023, Li_2013}, Poland \cite{Kobiela_2015, Wnęk_2015f}, and Spain \cite{CamachoDorado_2018, fjbuilsRepeatedBehaviorDeliberate2024}; 1 case from Australia \cite{Apikotoa_2022f}, Bahrain \cite{Ali_2020f}, Croatia \cite{Trgo_2012f}, Ecuador \cite{DelgadoSalazar_2020c}, Egypt \cite{Ali_2022g}, Ethiopia \cite{Mesfin_2022a}, Germany \cite{teWildt_2010}, Greece \cite{Sakellaridis_2008f}, Hungary \cite{Csaky_1998e}, Iraq \cite{Al-Faham_2020k}, Israel \cite{Goldman_1998f}, Italy \cite{Riva_2018j}, Japan \cite{Ohno_2005}, Nepal \cite{Thapa_2019f}, Netherlands \cite{Benoist_2019e}, Oman \cite{AlShaaibi_2021b}, Pakistan \cite{Yasin_2009}, Portugal \cite{Peixoto_2017f}, Qatar \cite{Ali_2017}, Saudi Arabia \cite{Sultan_2024f}, South Africa \cite{Sobnach_2011f}, Sweden \cite{Naji_2012f}, Switzerland \cite{Wildhaber_2005}, and Taiwan \cite{Chang_2017f}. \paragraph*{Gender} 43 cases (60\%) were male \cite{Akay_2015f, Al-Faham_2020k, Alao_2006i, Ali_2017, Ali_2022g, Apikotoa_2022f, Atayan_2016, Benoist_2019e, Berry_2021e, Bhumi_2024f, CamachoDorado_2018, Csaky_1998e, Emamhadi_2018, Farhadi_2024h, Fry_2010, Gardner_2017h, Guinan_2019f, Jehangir_2019h, Jin_2023, Kobiela_2015, Kumar_2001, Kumar_2019f, Liu_2005, Losanoff_1996, Losanoff_1997e, Mesfin_2022a, Misra_2013, Qureshi_2016, Riva_2018j, Sobnach_2011f, Tammana_2012j, Tanrikulu_2015e, Tay_2004, Thapa_2019f, Trgo_2012f, Wadhwa_2015e, Yasin_2009, teWildt_2010}, 28 cases (39\%) were female \cite{AlShaaibi_2021b, Ali_2020f, Ataya_2013, Beecroft_1998, Bhasin_2014, Bhattacharjee_2008, Cauchi_2002, Chang_2017f, Cox_2007, DelgadoSalazar_2020c, DivsalarP._2023a, Goldman_1998f, Hardy_2023g, Kar_2015, Kariholu_2008, Kerestes_2019, Li_2013, Naji_2012f, Ohno_2005, Peixoto_2017f, Sakellaridis_2008f, Sultan_2024f, Tupesis_2004f, Wildhaber_2005, Wnęk_2015f, Yildiz_2016e}, 1 case (1\%) had no gender recorded \cite{fjbuilsRepeatedBehaviorDeliberate2024}. \paragraph*{Age Group} 25 cases (35\%) were between 26 and 40 years of age \cite{Alao_2006i, Ali_2022g, Apikotoa_2022f, Ataya_2013, Benoist_2019e, Bhasin_2014, Chang_2017f, Cox_2007, DelgadoSalazar_2020c, Farhadi_2024h, Fry_2010, Gardner_2017h, Guinan_2019f, Jin_2023, Kumar_2019f, Losanoff_1996, Misra_2013, Qureshi_2016, Riva_2018j, Sakellaridis_2008f, Tammana_2012j, Trgo_2012f, Wnęk_2015f, Yildiz_2016e, fjbuilsRepeatedBehaviorDeliberate2024}, 18 cases (25\%) were between 18 and 25 years of age \cite{Akay_2015f, Ali_2017, Atayan_2016, Bhattacharjee_2008, Csaky_1998e, Kar_2015, Kariholu_2008, Kobiela_2015, Losanoff_1996, Losanoff_1997e, Mesfin_2022a, Peixoto_2017f, Sobnach_2011f, Tupesis_2004f, Yasin_2009}, 13 cases (18\%) were under 18 years of age \cite{AlShaaibi_2021b, Ali_2020f, Cauchi_2002, DivsalarP._2023a, Goldman_1998f, Liu_2005, Naji_2012f, Ohno_2005, Tanrikulu_2015e, Tay_2004, Wildhaber_2005}, 11 cases (15\%) were between 41 and 60 years of age \cite{Al-Faham_2020k, Bhumi_2024f, CamachoDorado_2018, Emamhadi_2018, Hardy_2023g, Jehangir_2019h, Kumar_2001, Sultan_2024f, Thapa_2019f, Wadhwa_2015e, teWildt_2010}, 3 cases (4\%) were over 60 years of age \cite{Beecroft_1998, Kerestes_2019, Li_2013}, 2 cases (3\%) had no age documented \cite{Berry_2021e}. \paragraph*{Population} 36 cases (50\%) had a psychiatric history \cite{AlShaaibi_2021b, Alao_2006i, Ali_2020f, Apikotoa_2022f, Ataya_2013, Atayan_2016, Beecroft_1998, CamachoDorado_2018, Chang_2017f, DelgadoSalazar_2020c, DivsalarP._2023a, Farhadi_2024h, Fry_2010, Guinan_2019f, Hardy_2023g, Jehangir_2019h, Jin_2023, Kar_2015, Kerestes_2019, Kobiela_2015, Kumar_2001, Kumar_2019f, Liu_2005, Mesfin_2022a, Misra_2013, Ohno_2005, Peixoto_2017f, Sakellaridis_2008f, Sultan_2024f, Tammana_2012j, Tanrikulu_2015e, Yildiz_2016e, fjbuilsRepeatedBehaviorDeliberate2024, teWildt_2010}, 19 cases (26\%) had ingested previously \cite{Alao_2006i, Apikotoa_2022f, Berry_2021e, Bhattacharjee_2008, Csaky_1998e, DivsalarP._2023a, Emamhadi_2018, Guinan_2019f, Jehangir_2019h, Jin_2023, Liu_2005, Sakellaridis_2008f, Tanrikulu_2015e, Thapa_2019f, Yildiz_2016e, fjbuilsRepeatedBehaviorDeliberate2024, teWildt_2010}, 12 cases (17\%) were detained persons \cite{Alao_2006i, Ali_2022g, Apikotoa_2022f, Losanoff_1996, Losanoff_1997e, Qureshi_2016, Tammana_2012j, Trgo_2012f}, 7 cases (10\%) were severely disabled \cite{Atayan_2016, Kerestes_2019, Liu_2005, Ohno_2005, Peixoto_2017f, Yildiz_2016e, teWildt_2010}, 4 cases (6\%) were psychiatric inpatients \cite{DivsalarP._2023a, fjbuilsRepeatedBehaviorDeliberate2024, teWildt_2010}, 3 cases (4\%) were under the influence of alcohol \cite{Benoist_2019e, Csaky_1998e, Thapa_2019f}, 2 cases (3\%) were displaced people \cite{Akay_2015f, Gardner_2017h}. \paragraph*{Motivation} 34 cases (47\%) had a psychiatric motivation \cite{Al-Faham_2020k, Alao_2006i, Ali_2020f, Apikotoa_2022f, Ataya_2013, Atayan_2016, Bhasin_2014, Bhattacharjee_2008, DelgadoSalazar_2020c, DivsalarP._2023a, Emamhadi_2018, Farhadi_2024h, Guinan_2019f, Hardy_2023g, Jehangir_2019h, Jin_2023, Kar_2015, Kariholu_2008, Kerestes_2019, Kobiela_2015, Kumar_2001, Kumar_2019f, Li_2013, Liu_2005, Misra_2013, Ohno_2005, Sakellaridis_2008f, Sultan_2024f, Tammana_2012j, Tanrikulu_2015e, Yasin_2009, teWildt_2010}, 21 cases (29\%) were motivated by self-harm intention \cite{Al-Faham_2020k, AlShaaibi_2021b, Alao_2006i, Ali_2017, CamachoDorado_2018, Chang_2017f, Cox_2007, Csaky_1998e, Fry_2010, Li_2013, Losanoff_1996, Losanoff_1997e, Mesfin_2022a, Sakellaridis_2008f, Tammana_2012j, Tanrikulu_2015e, fjbuilsRepeatedBehaviorDeliberate2024}, 17 cases (24\%) had a psychosocial motivation \cite{Akay_2015f, Benoist_2019e, Bhattacharjee_2008, Cauchi_2002, Goldman_1998f, Hardy_2023g, Kobiela_2015, Li_2013, Naji_2012f, Qureshi_2016, Riva_2018j, Sobnach_2011f, Tay_2004, Thapa_2019f, Tupesis_2004f, Wildhaber_2005, Wnęk_2015f}, 9 cases (12\%) were motivated by protest \cite{Bhumi_2024f, Gardner_2017h, Losanoff_1996, Losanoff_1997e, Tupesis_2004f}, 9 cases (12\%) had another documented motivation \cite{Ali_2020f, Ali_2022g, Emamhadi_2018, Guinan_2019f, Peixoto_2017f, Sakellaridis_2008f, Trgo_2012f, Wadhwa_2015e, Yildiz_2016e}. \paragraph*{Object Characteristics} 51 cases (71\%) ingested a large diameter object (\textgreater{}2.5cm) \cite{Akay_2015f, Al-Faham_2020k, AlShaaibi_2021b, Alao_2006i, Ali_2017, Ali_2022g, Apikotoa_2022f, Atayan_2016, Berry_2021e, Bhasin_2014, CamachoDorado_2018, Cauchi_2002, Chang_2017f, Cox_2007, Csaky_1998e, DivsalarP._2023a, Emamhadi_2018, Gardner_2017h, Guinan_2019f, Jehangir_2019h, Jin_2023, Kariholu_2008, Kerestes_2019, Kobiela_2015, Kumar_2001, Kumar_2019f, Losanoff_1996, Losanoff_1997e, Mesfin_2022a, Misra_2013, Naji_2012f, Ohno_2005, Peixoto_2017f, Qureshi_2016, Riva_2018j, Sakellaridis_2008f, Sultan_2024f, Tanrikulu_2015e, Thapa_2019f, Trgo_2012f, Wnęk_2015f, Yildiz_2016e, fjbuilsRepeatedBehaviorDeliberate2024, teWildt_2010}, 44 cases (61\%) ingested multiple objects \cite{Ali_2020f, Apikotoa_2022f, Ataya_2013, Atayan_2016, Beecroft_1998, Bhattacharjee_2008, Bhumi_2024f, CamachoDorado_2018, Cauchi_2002, Emamhadi_2018, Farhadi_2024h, Fry_2010, Goldman_1998f, Guinan_2019f, Hardy_2023g, Jehangir_2019h, Jin_2023, Kar_2015, Kariholu_2008, Kobiela_2015, Kumar_2001, Kumar_2019f, Li_2013, Liu_2005, Losanoff_1996, Mesfin_2022a, Misra_2013, Naji_2012f, Ohno_2005, Sobnach_2011f, Sultan_2024f, Tammana_2012j, Tanrikulu_2015e, Tay_2004, Thapa_2019f, Wadhwa_2015e, Wildhaber_2005, Yasin_2009, fjbuilsRepeatedBehaviorDeliberate2024, teWildt_2010}, 34 cases (47\%) ingested a sharp object \cite{AlShaaibi_2021b, Alao_2006i, Apikotoa_2022f, Ataya_2013, Benoist_2019e, Bhasin_2014, Bhattacharjee_2008, CamachoDorado_2018, Csaky_1998e, DelgadoSalazar_2020c, DivsalarP._2023a, Emamhadi_2018, Farhadi_2024h, Fry_2010, Guinan_2019f, Hardy_2023g, Jehangir_2019h, Jin_2023, Kariholu_2008, Kobiela_2015, Kumar_2019f, Losanoff_1996, Losanoff_1997e, Mesfin_2022a, Misra_2013, Sobnach_2011f, Yasin_2009, teWildt_2010}, 32 cases (44\%) ingested a long object (\textgreater{}5cm) \cite{Al-Faham_2020k, AlShaaibi_2021b, Ali_2017, Ali_2022g, Atayan_2016, Bhasin_2014, CamachoDorado_2018, Chang_2017f, Cox_2007, Csaky_1998e, DivsalarP._2023a, Emamhadi_2018, Fry_2010, Gardner_2017h, Jin_2023, Kariholu_2008, Kerestes_2019, Kobiela_2015, Kumar_2019f, Mesfin_2022a, Misra_2013, Ohno_2005, Qureshi_2016, Sakellaridis_2008f, Sultan_2024f, Thapa_2019f, Trgo_2012f, Yasin_2009, Yildiz_2016e, teWildt_2010}, 9 cases (12\%) ingested a magnet \cite{Ali_2020f, Bhumi_2024f, Cauchi_2002, Liu_2005, Naji_2012f, Ohno_2005, Tanrikulu_2015e, Tay_2004, Wildhaber_2005}, 2 cases (3\%) ingested a button battery \cite{Berry_2021e, Bhumi_2024f}. \paragraph*{Outcomes} 48 cases (67\%) experienced a complication \cite{Ali_2017, Ali_2020f, Apikotoa_2022f, Atayan_2016, Beecroft_1998, Benoist_2019e, Berry_2021e, Bhasin_2014, Bhumi_2024f, CamachoDorado_2018, Cauchi_2002, Cox_2007, Csaky_1998e, DelgadoSalazar_2020c, DivsalarP._2023a, Emamhadi_2018, Farhadi_2024h, Fry_2010, Gardner_2017h, Goldman_1998f, Jin_2023, Kariholu_2008, Kerestes_2019, Kobiela_2015, Kumar_2001, Kumar_2019f, Liu_2005, Losanoff_1996, Mesfin_2022a, Misra_2013, Naji_2012f, Ohno_2005, Sakellaridis_2008f, Sobnach_2011f, Sultan_2024f, Tanrikulu_2015e, Tay_2004, Thapa_2019f, Trgo_2012f, Tupesis_2004f, Wildhaber_2005, Wnęk_2015f, Yasin_2009, Yildiz_2016e}, 44 cases (61\%) underwent surgery \cite{Al-Faham_2020k, AlShaaibi_2021b, Alao_2006i, Ali_2017, Ali_2020f, Atayan_2016, Beecroft_1998, Bhasin_2014, CamachoDorado_2018, Cauchi_2002, Chang_2017f, Cox_2007, Csaky_1998e, DelgadoSalazar_2020c, DivsalarP._2023a, Farhadi_2024h, Fry_2010, Gardner_2017h, Jin_2023, Kariholu_2008, Kerestes_2019, Kobiela_2015, Kumar_2019f, Liu_2005, Losanoff_1996, Losanoff_1997e, Mesfin_2022a, Misra_2013, Naji_2012f, Sobnach_2011f, Tanrikulu_2015e, Tay_2004, Thapa_2019f, Tupesis_2004f, Wildhaber_2005, Wnęk_2015f, Yasin_2009, Yildiz_2016e, fjbuilsRepeatedBehaviorDeliberate2024}, 31 cases (43\%) underwent endoscopy \cite{Akay_2015f, Ali_2022g, Apikotoa_2022f, Atayan_2016, Benoist_2019e, Berry_2021e, Bhasin_2014, Bhumi_2024f, CamachoDorado_2018, Chang_2017f, DelgadoSalazar_2020c, Gardner_2017h, Guinan_2019f, Hardy_2023g, Jehangir_2019h, Kariholu_2008, Li_2013, Liu_2005, Ohno_2005, Peixoto_2017f, Qureshi_2016, Riva_2018j, Sakellaridis_2008f, Sultan_2024f, Tammana_2012j, Tanrikulu_2015e, Trgo_2012f, Wadhwa_2015e, Wnęk_2015f, teWildt_2010}, 7 cases (10\%) were managed conservatively \cite{Ataya_2013, Bhattacharjee_2008, DivsalarP._2023a, Emamhadi_2018, Goldman_1998f, Kar_2015, Kumar_2001}, 2 cases (3\%) died \cite{Emamhadi_2018, Kumar_2001}. All 90 were male gender. 90 cases (100\%) were detained at the time of ingestion \cite{Elghali_2016, Karp_1991b, Lee_2007}, 88 cases (98\%) were intentional ingestions \cite{Elghali_2016, Karp_1991b, Lee_2007}, 30 cases (33\%) had a psychiatric history documented \cite{Elghali_2016, Karp_1991b, Lee_2007}, 2 cases (2\%) had a history of prior ingestion \cite{Elghali_2016}. No cases were reported for were psychiatric inpatients, were displaced people, were under the influence of alcohol at the time of ingestion, and had a severe disability history.
\paragraph*{Motivation}  70 cases (78\%) reported protest motivation \cite{Elghali_2016, Karp_1991b, Lee_2007}, 12 cases (13\%) reported psychiatric motivation \cite{Karp_1991b}, 6 cases (7\%) reported self-harm motivation \cite{Elghali_2016, Karp_1991b}. No cases were reported for psychosocial motivation and other motivation.
\paragraph*{Object Characteristics}  68 cases (76\%) involved sharp object ingestion \cite{Elghali_2016, Karp_1991b, Lee_2007}, 32 cases (36\%) involved long (\textgreater 5cm) object ingestion \cite{Lee_2007}, 25 cases (28\%) involved ingestion of multiple objects \cite{Elghali_2016, Lee_2007}. No cases were reported for button battery ingestion, magnet ingestion, and involved large diameter (\textgreater 2.5cm) object ingestion.
\paragraph*{Outcomes}  47 cases (52\%) underwent endoscopic intervention \cite{Elghali_2016, Lee_2007}, 29 cases (32\%) were managed conservatively \cite{Elghali_2016, Karp_1991b}, 15 cases (17\%) underwent surgical intervention \cite{Elghali_2016, Karp_1991b, Lee_2007}, 6 cases (7\%) reported complications \cite{Lee_2007}, 1 case (1\%) died \cite{Elghali_2016}.
\paragraph*{Geographical Location}Cases were recorded in 33 countries: 13 cases from USA \cite{Alao_2006i, Ataya_2013, Bhumi_2024f, Fry_2010, Guinan_2019f, Hardy_2023g, Jehangir_2019h, Kerestes_2019, Kumar_2001, Liu_2005, Tammana_2012j, Tay_2004, Tupesis_2004f}; 7 cases from India \cite{Bhasin_2014, Bhattacharjee_2008, Kar_2015, Kariholu_2008, Kumar_2019f, Misra_2013, Wadhwa_2015e} and UK \cite{Beecroft_1998, Berry_2021e, Cauchi_2002, Cox_2007, Gardner_2017h, Qureshi_2016}; 6 cases from Bulgaria \cite{Losanoff_1996, Losanoff_1997e}; 5 cases from Iran \cite{DivsalarP._2023a, Emamhadi_2018, Farhadi_2024h}; 4 cases from Turkey \cite{Akay_2015f, Atayan_2016, Tanrikulu_2015e, Yildiz_2016e}; 2 cases from China \cite{Jin_2023, Li_2013}, Poland \cite{Kobiela_2015, Wnęk_2015f}, and Spain \cite{CamachoDorado_2018, fjbuilsRepeatedBehaviorDeliberate2024}; 1 case from Australia \cite{Apikotoa_2022f}, Bahrain \cite{Ali_2020f}, Croatia \cite{Trgo_2012f}, Ecuador \cite{DelgadoSalazar_2020c}, Egypt \cite{Ali_2022g}, Ethiopia \cite{Mesfin_2022a}, Germany \cite{teWildt_2010}, Greece \cite{Sakellaridis_2008f}, Hungary \cite{Csaky_1998e}, Iraq \cite{Al-Faham_2020k}, Israel \cite{Goldman_1998f}, Italy \cite{Riva_2018j}, Japan \cite{Ohno_2005}, Nepal \cite{Thapa_2019f}, Netherlands \cite{Benoist_2019e}, Oman \cite{AlShaaibi_2021b}, Pakistan \cite{Yasin_2009}, Portugal \cite{Peixoto_2017f}, Qatar \cite{Ali_2017}, Saudi Arabia \cite{Sultan_2024f}, South Africa \cite{Sobnach_2011f}, Sweden \cite{Naji_2012f}, Switzerland \cite{Wildhaber_2005}, and Taiwan \cite{Chang_2017f}. \paragraph*{Gender} 43 cases (60\%) were male \cite{Akay_2015f, Al-Faham_2020k, Alao_2006i, Ali_2017, Ali_2022g, Apikotoa_2022f, Atayan_2016, Benoist_2019e, Berry_2021e, Bhumi_2024f, CamachoDorado_2018, Csaky_1998e, Emamhadi_2018, Farhadi_2024h, Fry_2010, Gardner_2017h, Guinan_2019f, Jehangir_2019h, Jin_2023, Kobiela_2015, Kumar_2001, Kumar_2019f, Liu_2005, Losanoff_1996, Losanoff_1997e, Mesfin_2022a, Misra_2013, Qureshi_2016, Riva_2018j, Sobnach_2011f, Tammana_2012j, Tanrikulu_2015e, Tay_2004, Thapa_2019f, Trgo_2012f, Wadhwa_2015e, Yasin_2009, teWildt_2010}, 28 cases (39\%) were female \cite{AlShaaibi_2021b, Ali_2020f, Ataya_2013, Beecroft_1998, Bhasin_2014, Bhattacharjee_2008, Cauchi_2002, Chang_2017f, Cox_2007, DelgadoSalazar_2020c, DivsalarP._2023a, Goldman_1998f, Hardy_2023g, Kar_2015, Kariholu_2008, Kerestes_2019, Li_2013, Naji_2012f, Ohno_2005, Peixoto_2017f, Sakellaridis_2008f, Sultan_2024f, Tupesis_2004f, Wildhaber_2005, Wnęk_2015f, Yildiz_2016e}, 1 case (1\%) had no gender recorded \cite{fjbuilsRepeatedBehaviorDeliberate2024}. \paragraph*{Age Group} 25 cases (35\%) were between 26 and 40 years of age \cite{Alao_2006i, Ali_2022g, Apikotoa_2022f, Ataya_2013, Benoist_2019e, Bhasin_2014, Chang_2017f, Cox_2007, DelgadoSalazar_2020c, Farhadi_2024h, Fry_2010, Gardner_2017h, Guinan_2019f, Jin_2023, Kumar_2019f, Losanoff_1996, Misra_2013, Qureshi_2016, Riva_2018j, Sakellaridis_2008f, Tammana_2012j, Trgo_2012f, Wnęk_2015f, Yildiz_2016e, fjbuilsRepeatedBehaviorDeliberate2024}, 18 cases (25\%) were between 18 and 25 years of age \cite{Akay_2015f, Ali_2017, Atayan_2016, Bhattacharjee_2008, Csaky_1998e, Kar_2015, Kariholu_2008, Kobiela_2015, Losanoff_1996, Losanoff_1997e, Mesfin_2022a, Peixoto_2017f, Sobnach_2011f, Tupesis_2004f, Yasin_2009}, 13 cases (18\%) were under 18 years of age \cite{AlShaaibi_2021b, Ali_2020f, Cauchi_2002, DivsalarP._2023a, Goldman_1998f, Liu_2005, Naji_2012f, Ohno_2005, Tanrikulu_2015e, Tay_2004, Wildhaber_2005}, 11 cases (15\%) were between 41 and 60 years of age \cite{Al-Faham_2020k, Bhumi_2024f, CamachoDorado_2018, Emamhadi_2018, Hardy_2023g, Jehangir_2019h, Kumar_2001, Sultan_2024f, Thapa_2019f, Wadhwa_2015e, teWildt_2010}, 3 cases (4\%) were over 60 years of age \cite{Beecroft_1998, Kerestes_2019, Li_2013}, 2 cases (3\%) had no age documented \cite{Berry_2021e}. \paragraph*{Population} 36 cases (50\%) had a psychiatric history \cite{AlShaaibi_2021b, Alao_2006i, Ali_2020f, Apikotoa_2022f, Ataya_2013, Atayan_2016, Beecroft_1998, CamachoDorado_2018, Chang_2017f, DelgadoSalazar_2020c, DivsalarP._2023a, Farhadi_2024h, Fry_2010, Guinan_2019f, Hardy_2023g, Jehangir_2019h, Jin_2023, Kar_2015, Kerestes_2019, Kobiela_2015, Kumar_2001, Kumar_2019f, Liu_2005, Mesfin_2022a, Misra_2013, Ohno_2005, Peixoto_2017f, Sakellaridis_2008f, Sultan_2024f, Tammana_2012j, Tanrikulu_2015e, Yildiz_2016e, fjbuilsRepeatedBehaviorDeliberate2024, teWildt_2010}, 19 cases (26\%) had ingested previously \cite{Alao_2006i, Apikotoa_2022f, Berry_2021e, Bhattacharjee_2008, Csaky_1998e, DivsalarP._2023a, Emamhadi_2018, Guinan_2019f, Jehangir_2019h, Jin_2023, Liu_2005, Sakellaridis_2008f, Tanrikulu_2015e, Thapa_2019f, Yildiz_2016e, fjbuilsRepeatedBehaviorDeliberate2024, teWildt_2010}, 12 cases (17\%) were detained persons \cite{Alao_2006i, Ali_2022g, Apikotoa_2022f, Losanoff_1996, Losanoff_1997e, Qureshi_2016, Tammana_2012j, Trgo_2012f}, 7 cases (10\%) were severely disabled \cite{Atayan_2016, Kerestes_2019, Liu_2005, Ohno_2005, Peixoto_2017f, Yildiz_2016e, teWildt_2010}, 4 cases (6\%) were psychiatric inpatients \cite{DivsalarP._2023a, fjbuilsRepeatedBehaviorDeliberate2024, teWildt_2010}, 3 cases (4\%) were under the influence of alcohol \cite{Benoist_2019e, Csaky_1998e, Thapa_2019f}, 2 cases (3\%) were displaced people \cite{Akay_2015f, Gardner_2017h}. \paragraph*{Motivation} 34 cases (47\%) had a psychiatric motivation \cite{Al-Faham_2020k, Alao_2006i, Ali_2020f, Apikotoa_2022f, Ataya_2013, Atayan_2016, Bhasin_2014, Bhattacharjee_2008, DelgadoSalazar_2020c, DivsalarP._2023a, Emamhadi_2018, Farhadi_2024h, Guinan_2019f, Hardy_2023g, Jehangir_2019h, Jin_2023, Kar_2015, Kariholu_2008, Kerestes_2019, Kobiela_2015, Kumar_2001, Kumar_2019f, Li_2013, Liu_2005, Misra_2013, Ohno_2005, Sakellaridis_2008f, Sultan_2024f, Tammana_2012j, Tanrikulu_2015e, Yasin_2009, teWildt_2010}, 21 cases (29\%) were motivated by self-harm intention \cite{Al-Faham_2020k, AlShaaibi_2021b, Alao_2006i, Ali_2017, CamachoDorado_2018, Chang_2017f, Cox_2007, Csaky_1998e, Fry_2010, Li_2013, Losanoff_1996, Losanoff_1997e, Mesfin_2022a, Sakellaridis_2008f, Tammana_2012j, Tanrikulu_2015e, fjbuilsRepeatedBehaviorDeliberate2024}, 17 cases (24\%) had a psychosocial motivation \cite{Akay_2015f, Benoist_2019e, Bhattacharjee_2008, Cauchi_2002, Goldman_1998f, Hardy_2023g, Kobiela_2015, Li_2013, Naji_2012f, Qureshi_2016, Riva_2018j, Sobnach_2011f, Tay_2004, Thapa_2019f, Tupesis_2004f, Wildhaber_2005, Wnęk_2015f}, 9 cases (12\%) were motivated by protest \cite{Bhumi_2024f, Gardner_2017h, Losanoff_1996, Losanoff_1997e, Tupesis_2004f}, 9 cases (12\%) had another documented motivation \cite{Ali_2020f, Ali_2022g, Emamhadi_2018, Guinan_2019f, Peixoto_2017f, Sakellaridis_2008f, Trgo_2012f, Wadhwa_2015e, Yildiz_2016e}. \paragraph*{Object Characteristics} 51 cases (71\%) ingested a large diameter object (\textgreater{}2.5cm) \cite{Akay_2015f, Al-Faham_2020k, AlShaaibi_2021b, Alao_2006i, Ali_2017, Ali_2022g, Apikotoa_2022f, Atayan_2016, Berry_2021e, Bhasin_2014, CamachoDorado_2018, Cauchi_2002, Chang_2017f, Cox_2007, Csaky_1998e, DivsalarP._2023a, Emamhadi_2018, Gardner_2017h, Guinan_2019f, Jehangir_2019h, Jin_2023, Kariholu_2008, Kerestes_2019, Kobiela_2015, Kumar_2001, Kumar_2019f, Losanoff_1996, Losanoff_1997e, Mesfin_2022a, Misra_2013, Naji_2012f, Ohno_2005, Peixoto_2017f, Qureshi_2016, Riva_2018j, Sakellaridis_2008f, Sultan_2024f, Tanrikulu_2015e, Thapa_2019f, Trgo_2012f, Wnęk_2015f, Yildiz_2016e, fjbuilsRepeatedBehaviorDeliberate2024, teWildt_2010}, 44 cases (61\%) ingested multiple objects \cite{Ali_2020f, Apikotoa_2022f, Ataya_2013, Atayan_2016, Beecroft_1998, Bhattacharjee_2008, Bhumi_2024f, CamachoDorado_2018, Cauchi_2002, Emamhadi_2018, Farhadi_2024h, Fry_2010, Goldman_1998f, Guinan_2019f, Hardy_2023g, Jehangir_2019h, Jin_2023, Kar_2015, Kariholu_2008, Kobiela_2015, Kumar_2001, Kumar_2019f, Li_2013, Liu_2005, Losanoff_1996, Mesfin_2022a, Misra_2013, Naji_2012f, Ohno_2005, Sobnach_2011f, Sultan_2024f, Tammana_2012j, Tanrikulu_2015e, Tay_2004, Thapa_2019f, Wadhwa_2015e, Wildhaber_2005, Yasin_2009, fjbuilsRepeatedBehaviorDeliberate2024, teWildt_2010}, 34 cases (47\%) ingested a sharp object \cite{AlShaaibi_2021b, Alao_2006i, Apikotoa_2022f, Ataya_2013, Benoist_2019e, Bhasin_2014, Bhattacharjee_2008, CamachoDorado_2018, Csaky_1998e, DelgadoSalazar_2020c, DivsalarP._2023a, Emamhadi_2018, Farhadi_2024h, Fry_2010, Guinan_2019f, Hardy_2023g, Jehangir_2019h, Jin_2023, Kariholu_2008, Kobiela_2015, Kumar_2019f, Losanoff_1996, Losanoff_1997e, Mesfin_2022a, Misra_2013, Sobnach_2011f, Yasin_2009, teWildt_2010}, 32 cases (44\%) ingested a long object (\textgreater{}5cm) \cite{Al-Faham_2020k, AlShaaibi_2021b, Ali_2017, Ali_2022g, Atayan_2016, Bhasin_2014, CamachoDorado_2018, Chang_2017f, Cox_2007, Csaky_1998e, DivsalarP._2023a, Emamhadi_2018, Fry_2010, Gardner_2017h, Jin_2023, Kariholu_2008, Kerestes_2019, Kobiela_2015, Kumar_2019f, Mesfin_2022a, Misra_2013, Ohno_2005, Qureshi_2016, Sakellaridis_2008f, Sultan_2024f, Thapa_2019f, Trgo_2012f, Yasin_2009, Yildiz_2016e, teWildt_2010}, 9 cases (12\%) ingested a magnet \cite{Ali_2020f, Bhumi_2024f, Cauchi_2002, Liu_2005, Naji_2012f, Ohno_2005, Tanrikulu_2015e, Tay_2004, Wildhaber_2005}, 2 cases (3\%) ingested a button battery \cite{Berry_2021e, Bhumi_2024f}. \paragraph*{Outcomes} 48 cases (67\%) experienced a complication \cite{Ali_2017, Ali_2020f, Apikotoa_2022f, Atayan_2016, Beecroft_1998, Benoist_2019e, Berry_2021e, Bhasin_2014, Bhumi_2024f, CamachoDorado_2018, Cauchi_2002, Cox_2007, Csaky_1998e, DelgadoSalazar_2020c, DivsalarP._2023a, Emamhadi_2018, Farhadi_2024h, Fry_2010, Gardner_2017h, Goldman_1998f, Jin_2023, Kariholu_2008, Kerestes_2019, Kobiela_2015, Kumar_2001, Kumar_2019f, Liu_2005, Losanoff_1996, Mesfin_2022a, Misra_2013, Naji_2012f, Ohno_2005, Sakellaridis_2008f, Sobnach_2011f, Sultan_2024f, Tanrikulu_2015e, Tay_2004, Thapa_2019f, Trgo_2012f, Tupesis_2004f, Wildhaber_2005, Wnęk_2015f, Yasin_2009, Yildiz_2016e}, 44 cases (61\%) underwent surgery \cite{Al-Faham_2020k, AlShaaibi_2021b, Alao_2006i, Ali_2017, Ali_2020f, Atayan_2016, Beecroft_1998, Bhasin_2014, CamachoDorado_2018, Cauchi_2002, Chang_2017f, Cox_2007, Csaky_1998e, DelgadoSalazar_2020c, DivsalarP._2023a, Farhadi_2024h, Fry_2010, Gardner_2017h, Jin_2023, Kariholu_2008, Kerestes_2019, Kobiela_2015, Kumar_2019f, Liu_2005, Losanoff_1996, Losanoff_1997e, Mesfin_2022a, Misra_2013, Naji_2012f, Sobnach_2011f, Tanrikulu_2015e, Tay_2004, Thapa_2019f, Tupesis_2004f, Wildhaber_2005, Wnęk_2015f, Yasin_2009, Yildiz_2016e, fjbuilsRepeatedBehaviorDeliberate2024}, 31 cases (43\%) underwent endoscopy \cite{Akay_2015f, Ali_2022g, Apikotoa_2022f, Atayan_2016, Benoist_2019e, Berry_2021e, Bhasin_2014, Bhumi_2024f, CamachoDorado_2018, Chang_2017f, DelgadoSalazar_2020c, Gardner_2017h, Guinan_2019f, Hardy_2023g, Jehangir_2019h, Kariholu_2008, Li_2013, Liu_2005, Ohno_2005, Peixoto_2017f, Qureshi_2016, Riva_2018j, Sakellaridis_2008f, Sultan_2024f, Tammana_2012j, Tanrikulu_2015e, Trgo_2012f, Wadhwa_2015e, Wnęk_2015f, teWildt_2010}, 7 cases (10\%) were managed conservatively \cite{Ataya_2013, Bhattacharjee_2008, DivsalarP._2023a, Emamhadi_2018, Goldman_1998f, Kar_2015, Kumar_2001}, 2 cases (3\%) died \cite{Emamhadi_2018, Kumar_2001}. All 90 were male gender. 90 cases (100\%) were detained at the time of ingestion \cite{Elghali_2016, Karp_1991b, Lee_2007}, 88 cases (98\%) were intentional ingestions \cite{Elghali_2016, Karp_1991b, Lee_2007}, 30 cases (33\%) had a psychiatric history documented \cite{Elghali_2016, Karp_1991b, Lee_2007}, 2 cases (2\%) had a history of prior ingestion \cite{Elghali_2016}. No cases were reported for were psychiatric inpatients, were displaced people, were under the influence of alcohol at the time of ingestion, and had a severe disability history.
\paragraph*{Motivation}  70 cases (78\%) reported protest motivation \cite{Elghali_2016, Karp_1991b, Lee_2007}, 12 cases (13\%) reported psychiatric motivation \cite{Karp_1991b}, 6 cases (7\%) reported self-harm motivation \cite{Elghali_2016, Karp_1991b}. No cases were reported for psychosocial motivation and other motivation.
\paragraph*{Object Characteristics}  68 cases (76\%) involved sharp object ingestion \cite{Elghali_2016, Karp_1991b, Lee_2007}, 32 cases (36\%) involved long (\textgreater 5cm) object ingestion \cite{Lee_2007}, 25 cases (28\%) involved ingestion of multiple objects \cite{Elghali_2016, Lee_2007}. No cases were reported for button battery ingestion, magnet ingestion, and involved large diameter (\textgreater 2.5cm) object ingestion.
\paragraph*{Outcomes}  47 cases (52\%) underwent endoscopic intervention \cite{Elghali_2016, Lee_2007}, 29 cases (32\%) were managed conservatively \cite{Elghali_2016, Karp_1991b}, 15 cases (17\%) underwent surgical intervention \cite{Elghali_2016, Karp_1991b, Lee_2007}, 6 cases (7\%) reported complications \cite{Lee_2007}, 1 case (1\%) died \cite{Elghali_2016}.
\paragraph*{Geographical Location}Cases were recorded in 33 countries: 13 cases from USA \cite{Alao_2006i, Ataya_2013, Bhumi_2024f, Fry_2010, Guinan_2019f, Hardy_2023g, Jehangir_2019h, Kerestes_2019, Kumar_2001, Liu_2005, Tammana_2012j, Tay_2004, Tupesis_2004f}; 7 cases from India \cite{Bhasin_2014, Bhattacharjee_2008, Kar_2015, Kariholu_2008, Kumar_2019f, Misra_2013, Wadhwa_2015e} and UK \cite{Beecroft_1998, Berry_2021e, Cauchi_2002, Cox_2007, Gardner_2017h, Qureshi_2016}; 6 cases from Bulgaria \cite{Losanoff_1996, Losanoff_1997e}; 5 cases from Iran \cite{DivsalarP._2023a, Emamhadi_2018, Farhadi_2024h}; 4 cases from Turkey \cite{Akay_2015f, Atayan_2016, Tanrikulu_2015e, Yildiz_2016e}; 2 cases from China \cite{Jin_2023, Li_2013}, Poland \cite{Kobiela_2015, Wnęk_2015f}, and Spain \cite{CamachoDorado_2018, fjbuilsRepeatedBehaviorDeliberate2024}; 1 case from Australia \cite{Apikotoa_2022f}, Bahrain \cite{Ali_2020f}, Croatia \cite{Trgo_2012f}, Ecuador \cite{DelgadoSalazar_2020c}, Egypt \cite{Ali_2022g}, Ethiopia \cite{Mesfin_2022a}, Germany \cite{teWildt_2010}, Greece \cite{Sakellaridis_2008f}, Hungary \cite{Csaky_1998e}, Iraq \cite{Al-Faham_2020k}, Israel \cite{Goldman_1998f}, Italy \cite{Riva_2018j}, Japan \cite{Ohno_2005}, Nepal \cite{Thapa_2019f}, Netherlands \cite{Benoist_2019e}, Oman \cite{AlShaaibi_2021b}, Pakistan \cite{Yasin_2009}, Portugal \cite{Peixoto_2017f}, Qatar \cite{Ali_2017}, Saudi Arabia \cite{Sultan_2024f}, South Africa \cite{Sobnach_2011f}, Sweden \cite{Naji_2012f}, Switzerland \cite{Wildhaber_2005}, and Taiwan \cite{Chang_2017f}. \paragraph*{Gender} 43 cases (60\%) were male \cite{Akay_2015f, Al-Faham_2020k, Alao_2006i, Ali_2017, Ali_2022g, Apikotoa_2022f, Atayan_2016, Benoist_2019e, Berry_2021e, Bhumi_2024f, CamachoDorado_2018, Csaky_1998e, Emamhadi_2018, Farhadi_2024h, Fry_2010, Gardner_2017h, Guinan_2019f, Jehangir_2019h, Jin_2023, Kobiela_2015, Kumar_2001, Kumar_2019f, Liu_2005, Losanoff_1996, Losanoff_1997e, Mesfin_2022a, Misra_2013, Qureshi_2016, Riva_2018j, Sobnach_2011f, Tammana_2012j, Tanrikulu_2015e, Tay_2004, Thapa_2019f, Trgo_2012f, Wadhwa_2015e, Yasin_2009, teWildt_2010}, 28 cases (39\%) were female \cite{AlShaaibi_2021b, Ali_2020f, Ataya_2013, Beecroft_1998, Bhasin_2014, Bhattacharjee_2008, Cauchi_2002, Chang_2017f, Cox_2007, DelgadoSalazar_2020c, DivsalarP._2023a, Goldman_1998f, Hardy_2023g, Kar_2015, Kariholu_2008, Kerestes_2019, Li_2013, Naji_2012f, Ohno_2005, Peixoto_2017f, Sakellaridis_2008f, Sultan_2024f, Tupesis_2004f, Wildhaber_2005, Wnęk_2015f, Yildiz_2016e}, 1 case (1\%) had no gender recorded \cite{fjbuilsRepeatedBehaviorDeliberate2024}. \paragraph*{Age Group} 25 cases (35\%) were between 26 and 40 years of age \cite{Alao_2006i, Ali_2022g, Apikotoa_2022f, Ataya_2013, Benoist_2019e, Bhasin_2014, Chang_2017f, Cox_2007, DelgadoSalazar_2020c, Farhadi_2024h, Fry_2010, Gardner_2017h, Guinan_2019f, Jin_2023, Kumar_2019f, Losanoff_1996, Misra_2013, Qureshi_2016, Riva_2018j, Sakellaridis_2008f, Tammana_2012j, Trgo_2012f, Wnęk_2015f, Yildiz_2016e, fjbuilsRepeatedBehaviorDeliberate2024}, 18 cases (25\%) were between 18 and 25 years of age \cite{Akay_2015f, Ali_2017, Atayan_2016, Bhattacharjee_2008, Csaky_1998e, Kar_2015, Kariholu_2008, Kobiela_2015, Losanoff_1996, Losanoff_1997e, Mesfin_2022a, Peixoto_2017f, Sobnach_2011f, Tupesis_2004f, Yasin_2009}, 13 cases (18\%) were under 18 years of age \cite{AlShaaibi_2021b, Ali_2020f, Cauchi_2002, DivsalarP._2023a, Goldman_1998f, Liu_2005, Naji_2012f, Ohno_2005, Tanrikulu_2015e, Tay_2004, Wildhaber_2005}, 11 cases (15\%) were between 41 and 60 years of age \cite{Al-Faham_2020k, Bhumi_2024f, CamachoDorado_2018, Emamhadi_2018, Hardy_2023g, Jehangir_2019h, Kumar_2001, Sultan_2024f, Thapa_2019f, Wadhwa_2015e, teWildt_2010}, 3 cases (4\%) were over 60 years of age \cite{Beecroft_1998, Kerestes_2019, Li_2013}, 2 cases (3\%) had no age documented \cite{Berry_2021e}. \paragraph*{Population} 36 cases (50\%) had a psychiatric history \cite{AlShaaibi_2021b, Alao_2006i, Ali_2020f, Apikotoa_2022f, Ataya_2013, Atayan_2016, Beecroft_1998, CamachoDorado_2018, Chang_2017f, DelgadoSalazar_2020c, DivsalarP._2023a, Farhadi_2024h, Fry_2010, Guinan_2019f, Hardy_2023g, Jehangir_2019h, Jin_2023, Kar_2015, Kerestes_2019, Kobiela_2015, Kumar_2001, Kumar_2019f, Liu_2005, Mesfin_2022a, Misra_2013, Ohno_2005, Peixoto_2017f, Sakellaridis_2008f, Sultan_2024f, Tammana_2012j, Tanrikulu_2015e, Yildiz_2016e, fjbuilsRepeatedBehaviorDeliberate2024, teWildt_2010}, 19 cases (26\%) had ingested previously \cite{Alao_2006i, Apikotoa_2022f, Berry_2021e, Bhattacharjee_2008, Csaky_1998e, DivsalarP._2023a, Emamhadi_2018, Guinan_2019f, Jehangir_2019h, Jin_2023, Liu_2005, Sakellaridis_2008f, Tanrikulu_2015e, Thapa_2019f, Yildiz_2016e, fjbuilsRepeatedBehaviorDeliberate2024, teWildt_2010}, 12 cases (17\%) were detained persons \cite{Alao_2006i, Ali_2022g, Apikotoa_2022f, Losanoff_1996, Losanoff_1997e, Qureshi_2016, Tammana_2012j, Trgo_2012f}, 7 cases (10\%) were severely disabled \cite{Atayan_2016, Kerestes_2019, Liu_2005, Ohno_2005, Peixoto_2017f, Yildiz_2016e, teWildt_2010}, 4 cases (6\%) were psychiatric inpatients \cite{DivsalarP._2023a, fjbuilsRepeatedBehaviorDeliberate2024, teWildt_2010}, 3 cases (4\%) were under the influence of alcohol \cite{Benoist_2019e, Csaky_1998e, Thapa_2019f}, 2 cases (3\%) were displaced people \cite{Akay_2015f, Gardner_2017h}. \paragraph*{Motivation} 34 cases (47\%) had a psychiatric motivation \cite{Al-Faham_2020k, Alao_2006i, Ali_2020f, Apikotoa_2022f, Ataya_2013, Atayan_2016, Bhasin_2014, Bhattacharjee_2008, DelgadoSalazar_2020c, DivsalarP._2023a, Emamhadi_2018, Farhadi_2024h, Guinan_2019f, Hardy_2023g, Jehangir_2019h, Jin_2023, Kar_2015, Kariholu_2008, Kerestes_2019, Kobiela_2015, Kumar_2001, Kumar_2019f, Li_2013, Liu_2005, Misra_2013, Ohno_2005, Sakellaridis_2008f, Sultan_2024f, Tammana_2012j, Tanrikulu_2015e, Yasin_2009, teWildt_2010}, 21 cases (29\%) were motivated by self-harm intention \cite{Al-Faham_2020k, AlShaaibi_2021b, Alao_2006i, Ali_2017, CamachoDorado_2018, Chang_2017f, Cox_2007, Csaky_1998e, Fry_2010, Li_2013, Losanoff_1996, Losanoff_1997e, Mesfin_2022a, Sakellaridis_2008f, Tammana_2012j, Tanrikulu_2015e, fjbuilsRepeatedBehaviorDeliberate2024}, 17 cases (24\%) had a psychosocial motivation \cite{Akay_2015f, Benoist_2019e, Bhattacharjee_2008, Cauchi_2002, Goldman_1998f, Hardy_2023g, Kobiela_2015, Li_2013, Naji_2012f, Qureshi_2016, Riva_2018j, Sobnach_2011f, Tay_2004, Thapa_2019f, Tupesis_2004f, Wildhaber_2005, Wnęk_2015f}, 9 cases (12\%) were motivated by protest \cite{Bhumi_2024f, Gardner_2017h, Losanoff_1996, Losanoff_1997e, Tupesis_2004f}, 9 cases (12\%) had another documented motivation \cite{Ali_2020f, Ali_2022g, Emamhadi_2018, Guinan_2019f, Peixoto_2017f, Sakellaridis_2008f, Trgo_2012f, Wadhwa_2015e, Yildiz_2016e}. \paragraph*{Object Characteristics} 51 cases (71\%) ingested a large diameter object (\textgreater{}2.5cm) \cite{Akay_2015f, Al-Faham_2020k, AlShaaibi_2021b, Alao_2006i, Ali_2017, Ali_2022g, Apikotoa_2022f, Atayan_2016, Berry_2021e, Bhasin_2014, CamachoDorado_2018, Cauchi_2002, Chang_2017f, Cox_2007, Csaky_1998e, DivsalarP._2023a, Emamhadi_2018, Gardner_2017h, Guinan_2019f, Jehangir_2019h, Jin_2023, Kariholu_2008, Kerestes_2019, Kobiela_2015, Kumar_2001, Kumar_2019f, Losanoff_1996, Losanoff_1997e, Mesfin_2022a, Misra_2013, Naji_2012f, Ohno_2005, Peixoto_2017f, Qureshi_2016, Riva_2018j, Sakellaridis_2008f, Sultan_2024f, Tanrikulu_2015e, Thapa_2019f, Trgo_2012f, Wnęk_2015f, Yildiz_2016e, fjbuilsRepeatedBehaviorDeliberate2024, teWildt_2010}, 44 cases (61\%) ingested multiple objects \cite{Ali_2020f, Apikotoa_2022f, Ataya_2013, Atayan_2016, Beecroft_1998, Bhattacharjee_2008, Bhumi_2024f, CamachoDorado_2018, Cauchi_2002, Emamhadi_2018, Farhadi_2024h, Fry_2010, Goldman_1998f, Guinan_2019f, Hardy_2023g, Jehangir_2019h, Jin_2023, Kar_2015, Kariholu_2008, Kobiela_2015, Kumar_2001, Kumar_2019f, Li_2013, Liu_2005, Losanoff_1996, Mesfin_2022a, Misra_2013, Naji_2012f, Ohno_2005, Sobnach_2011f, Sultan_2024f, Tammana_2012j, Tanrikulu_2015e, Tay_2004, Thapa_2019f, Wadhwa_2015e, Wildhaber_2005, Yasin_2009, fjbuilsRepeatedBehaviorDeliberate2024, teWildt_2010}, 34 cases (47\%) ingested a sharp object \cite{AlShaaibi_2021b, Alao_2006i, Apikotoa_2022f, Ataya_2013, Benoist_2019e, Bhasin_2014, Bhattacharjee_2008, CamachoDorado_2018, Csaky_1998e, DelgadoSalazar_2020c, DivsalarP._2023a, Emamhadi_2018, Farhadi_2024h, Fry_2010, Guinan_2019f, Hardy_2023g, Jehangir_2019h, Jin_2023, Kariholu_2008, Kobiela_2015, Kumar_2019f, Losanoff_1996, Losanoff_1997e, Mesfin_2022a, Misra_2013, Sobnach_2011f, Yasin_2009, teWildt_2010}, 32 cases (44\%) ingested a long object (\textgreater{}5cm) \cite{Al-Faham_2020k, AlShaaibi_2021b, Ali_2017, Ali_2022g, Atayan_2016, Bhasin_2014, CamachoDorado_2018, Chang_2017f, Cox_2007, Csaky_1998e, DivsalarP._2023a, Emamhadi_2018, Fry_2010, Gardner_2017h, Jin_2023, Kariholu_2008, Kerestes_2019, Kobiela_2015, Kumar_2019f, Mesfin_2022a, Misra_2013, Ohno_2005, Qureshi_2016, Sakellaridis_2008f, Sultan_2024f, Thapa_2019f, Trgo_2012f, Yasin_2009, Yildiz_2016e, teWildt_2010}, 9 cases (12\%) ingested a magnet \cite{Ali_2020f, Bhumi_2024f, Cauchi_2002, Liu_2005, Naji_2012f, Ohno_2005, Tanrikulu_2015e, Tay_2004, Wildhaber_2005}, 2 cases (3\%) ingested a button battery \cite{Berry_2021e, Bhumi_2024f}. \paragraph*{Outcomes} 48 cases (67\%) experienced a complication \cite{Ali_2017, Ali_2020f, Apikotoa_2022f, Atayan_2016, Beecroft_1998, Benoist_2019e, Berry_2021e, Bhasin_2014, Bhumi_2024f, CamachoDorado_2018, Cauchi_2002, Cox_2007, Csaky_1998e, DelgadoSalazar_2020c, DivsalarP._2023a, Emamhadi_2018, Farhadi_2024h, Fry_2010, Gardner_2017h, Goldman_1998f, Jin_2023, Kariholu_2008, Kerestes_2019, Kobiela_2015, Kumar_2001, Kumar_2019f, Liu_2005, Losanoff_1996, Mesfin_2022a, Misra_2013, Naji_2012f, Ohno_2005, Sakellaridis_2008f, Sobnach_2011f, Sultan_2024f, Tanrikulu_2015e, Tay_2004, Thapa_2019f, Trgo_2012f, Tupesis_2004f, Wildhaber_2005, Wnęk_2015f, Yasin_2009, Yildiz_2016e}, 44 cases (61\%) underwent surgery \cite{Al-Faham_2020k, AlShaaibi_2021b, Alao_2006i, Ali_2017, Ali_2020f, Atayan_2016, Beecroft_1998, Bhasin_2014, CamachoDorado_2018, Cauchi_2002, Chang_2017f, Cox_2007, Csaky_1998e, DelgadoSalazar_2020c, DivsalarP._2023a, Farhadi_2024h, Fry_2010, Gardner_2017h, Jin_2023, Kariholu_2008, Kerestes_2019, Kobiela_2015, Kumar_2019f, Liu_2005, Losanoff_1996, Losanoff_1997e, Mesfin_2022a, Misra_2013, Naji_2012f, Sobnach_2011f, Tanrikulu_2015e, Tay_2004, Thapa_2019f, Tupesis_2004f, Wildhaber_2005, Wnęk_2015f, Yasin_2009, Yildiz_2016e, fjbuilsRepeatedBehaviorDeliberate2024}, 31 cases (43\%) underwent endoscopy \cite{Akay_2015f, Ali_2022g, Apikotoa_2022f, Atayan_2016, Benoist_2019e, Berry_2021e, Bhasin_2014, Bhumi_2024f, CamachoDorado_2018, Chang_2017f, DelgadoSalazar_2020c, Gardner_2017h, Guinan_2019f, Hardy_2023g, Jehangir_2019h, Kariholu_2008, Li_2013, Liu_2005, Ohno_2005, Peixoto_2017f, Qureshi_2016, Riva_2018j, Sakellaridis_2008f, Sultan_2024f, Tammana_2012j, Tanrikulu_2015e, Trgo_2012f, Wadhwa_2015e, Wnęk_2015f, teWildt_2010}, 7 cases (10\%) were managed conservatively \cite{Ataya_2013, Bhattacharjee_2008, DivsalarP._2023a, Emamhadi_2018, Goldman_1998f, Kar_2015, Kumar_2001}, 2 cases (3\%) died \cite{Emamhadi_2018, Kumar_2001}. All 90 were male gender. 90 cases (100\%) were detained at the time of ingestion \cite{Elghali_2016, Karp_1991b, Lee_2007}, 88 cases (98\%) were intentional ingestions \cite{Elghali_2016, Karp_1991b, Lee_2007}, 30 cases (33\%) had a psychiatric history documented \cite{Elghali_2016, Karp_1991b, Lee_2007}, 2 cases (2\%) had a history of prior ingestion \cite{Elghali_2016}. No cases were reported for were psychiatric inpatients, were displaced people, were under the influence of alcohol at the time of ingestion, and had a severe disability history.
\paragraph*{Motivation}  70 cases (78\%) reported protest motivation \cite{Elghali_2016, Karp_1991b, Lee_2007}, 12 cases (13\%) reported psychiatric motivation \cite{Karp_1991b}, 6 cases (7\%) reported self-harm motivation \cite{Elghali_2016, Karp_1991b}. No cases were reported for psychosocial motivation and other motivation.
\paragraph*{Object Characteristics}  68 cases (76\%) involved sharp object ingestion \cite{Elghali_2016, Karp_1991b, Lee_2007}, 32 cases (36\%) involved long (\textgreater 5cm) object ingestion \cite{Lee_2007}, 25 cases (28\%) involved ingestion of multiple objects \cite{Elghali_2016, Lee_2007}. No cases were reported for button battery ingestion, magnet ingestion, and involved large diameter (\textgreater 2.5cm) object ingestion.
\paragraph*{Outcomes}  47 cases (52\%) underwent endoscopic intervention \cite{Elghali_2016, Lee_2007}, 29 cases (32\%) were managed conservatively \cite{Elghali_2016, Karp_1991b}, 15 cases (17\%) underwent surgical intervention \cite{Elghali_2016, Karp_1991b, Lee_2007}, 6 cases (7\%) reported complications \cite{Lee_2007}, 1 case (1\%) died \cite{Elghali_2016}.
\paragraph*{Geographical Location}Cases were recorded in 33 countries: 13 cases from USA \cite{Alao_2006i, Ataya_2013, Bhumi_2024f, Fry_2010, Guinan_2019f, Hardy_2023g, Jehangir_2019h, Kerestes_2019, Kumar_2001, Liu_2005, Tammana_2012j, Tay_2004, Tupesis_2004f}; 7 cases from India \cite{Bhasin_2014, Bhattacharjee_2008, Kar_2015, Kariholu_2008, Kumar_2019f, Misra_2013, Wadhwa_2015e} and UK \cite{Beecroft_1998, Berry_2021e, Cauchi_2002, Cox_2007, Gardner_2017h, Qureshi_2016}; 6 cases from Bulgaria \cite{Losanoff_1996, Losanoff_1997e}; 5 cases from Iran \cite{DivsalarP._2023a, Emamhadi_2018, Farhadi_2024h}; 4 cases from Turkey \cite{Akay_2015f, Atayan_2016, Tanrikulu_2015e, Yildiz_2016e}; 2 cases from China \cite{Jin_2023, Li_2013}, Poland \cite{Kobiela_2015, Wnęk_2015f}, and Spain \cite{CamachoDorado_2018, fjbuilsRepeatedBehaviorDeliberate2024}; 1 case from Australia \cite{Apikotoa_2022f}, Bahrain \cite{Ali_2020f}, Croatia \cite{Trgo_2012f}, Ecuador \cite{DelgadoSalazar_2020c}, Egypt \cite{Ali_2022g}, Ethiopia \cite{Mesfin_2022a}, Germany \cite{teWildt_2010}, Greece \cite{Sakellaridis_2008f}, Hungary \cite{Csaky_1998e}, Iraq \cite{Al-Faham_2020k}, Israel \cite{Goldman_1998f}, Italy \cite{Riva_2018j}, Japan \cite{Ohno_2005}, Nepal \cite{Thapa_2019f}, Netherlands \cite{Benoist_2019e}, Oman \cite{AlShaaibi_2021b}, Pakistan \cite{Yasin_2009}, Portugal \cite{Peixoto_2017f}, Qatar \cite{Ali_2017}, Saudi Arabia \cite{Sultan_2024f}, South Africa \cite{Sobnach_2011f}, Sweden \cite{Naji_2012f}, Switzerland \cite{Wildhaber_2005}, and Taiwan \cite{Chang_2017f}. \paragraph*{Gender} 43 cases (60\%) were male \cite{Akay_2015f, Al-Faham_2020k, Alao_2006i, Ali_2017, Ali_2022g, Apikotoa_2022f, Atayan_2016, Benoist_2019e, Berry_2021e, Bhumi_2024f, CamachoDorado_2018, Csaky_1998e, Emamhadi_2018, Farhadi_2024h, Fry_2010, Gardner_2017h, Guinan_2019f, Jehangir_2019h, Jin_2023, Kobiela_2015, Kumar_2001, Kumar_2019f, Liu_2005, Losanoff_1996, Losanoff_1997e, Mesfin_2022a, Misra_2013, Qureshi_2016, Riva_2018j, Sobnach_2011f, Tammana_2012j, Tanrikulu_2015e, Tay_2004, Thapa_2019f, Trgo_2012f, Wadhwa_2015e, Yasin_2009, teWildt_2010}, 28 cases (39\%) were female \cite{AlShaaibi_2021b, Ali_2020f, Ataya_2013, Beecroft_1998, Bhasin_2014, Bhattacharjee_2008, Cauchi_2002, Chang_2017f, Cox_2007, DelgadoSalazar_2020c, DivsalarP._2023a, Goldman_1998f, Hardy_2023g, Kar_2015, Kariholu_2008, Kerestes_2019, Li_2013, Naji_2012f, Ohno_2005, Peixoto_2017f, Sakellaridis_2008f, Sultan_2024f, Tupesis_2004f, Wildhaber_2005, Wnęk_2015f, Yildiz_2016e}, 1 case (1\%) had no gender recorded \cite{fjbuilsRepeatedBehaviorDeliberate2024}. \paragraph*{Age Group} 25 cases (35\%) were between 26 and 40 years of age \cite{Alao_2006i, Ali_2022g, Apikotoa_2022f, Ataya_2013, Benoist_2019e, Bhasin_2014, Chang_2017f, Cox_2007, DelgadoSalazar_2020c, Farhadi_2024h, Fry_2010, Gardner_2017h, Guinan_2019f, Jin_2023, Kumar_2019f, Losanoff_1996, Misra_2013, Qureshi_2016, Riva_2018j, Sakellaridis_2008f, Tammana_2012j, Trgo_2012f, Wnęk_2015f, Yildiz_2016e, fjbuilsRepeatedBehaviorDeliberate2024}, 18 cases (25\%) were between 18 and 25 years of age \cite{Akay_2015f, Ali_2017, Atayan_2016, Bhattacharjee_2008, Csaky_1998e, Kar_2015, Kariholu_2008, Kobiela_2015, Losanoff_1996, Losanoff_1997e, Mesfin_2022a, Peixoto_2017f, Sobnach_2011f, Tupesis_2004f, Yasin_2009}, 13 cases (18\%) were under 18 years of age \cite{AlShaaibi_2021b, Ali_2020f, Cauchi_2002, DivsalarP._2023a, Goldman_1998f, Liu_2005, Naji_2012f, Ohno_2005, Tanrikulu_2015e, Tay_2004, Wildhaber_2005}, 11 cases (15\%) were between 41 and 60 years of age \cite{Al-Faham_2020k, Bhumi_2024f, CamachoDorado_2018, Emamhadi_2018, Hardy_2023g, Jehangir_2019h, Kumar_2001, Sultan_2024f, Thapa_2019f, Wadhwa_2015e, teWildt_2010}, 3 cases (4\%) were over 60 years of age \cite{Beecroft_1998, Kerestes_2019, Li_2013}, 2 cases (3\%) had no age documented \cite{Berry_2021e}. \paragraph*{Population} 36 cases (50\%) had a psychiatric history \cite{AlShaaibi_2021b, Alao_2006i, Ali_2020f, Apikotoa_2022f, Ataya_2013, Atayan_2016, Beecroft_1998, CamachoDorado_2018, Chang_2017f, DelgadoSalazar_2020c, DivsalarP._2023a, Farhadi_2024h, Fry_2010, Guinan_2019f, Hardy_2023g, Jehangir_2019h, Jin_2023, Kar_2015, Kerestes_2019, Kobiela_2015, Kumar_2001, Kumar_2019f, Liu_2005, Mesfin_2022a, Misra_2013, Ohno_2005, Peixoto_2017f, Sakellaridis_2008f, Sultan_2024f, Tammana_2012j, Tanrikulu_2015e, Yildiz_2016e, fjbuilsRepeatedBehaviorDeliberate2024, teWildt_2010}, 19 cases (26\%) had ingested previously \cite{Alao_2006i, Apikotoa_2022f, Berry_2021e, Bhattacharjee_2008, Csaky_1998e, DivsalarP._2023a, Emamhadi_2018, Guinan_2019f, Jehangir_2019h, Jin_2023, Liu_2005, Sakellaridis_2008f, Tanrikulu_2015e, Thapa_2019f, Yildiz_2016e, fjbuilsRepeatedBehaviorDeliberate2024, teWildt_2010}, 12 cases (17\%) were detained persons \cite{Alao_2006i, Ali_2022g, Apikotoa_2022f, Losanoff_1996, Losanoff_1997e, Qureshi_2016, Tammana_2012j, Trgo_2012f}, 7 cases (10\%) were severely disabled \cite{Atayan_2016, Kerestes_2019, Liu_2005, Ohno_2005, Peixoto_2017f, Yildiz_2016e, teWildt_2010}, 4 cases (6\%) were psychiatric inpatients \cite{DivsalarP._2023a, fjbuilsRepeatedBehaviorDeliberate2024, teWildt_2010}, 3 cases (4\%) were under the influence of alcohol \cite{Benoist_2019e, Csaky_1998e, Thapa_2019f}, 2 cases (3\%) were displaced people \cite{Akay_2015f, Gardner_2017h}. \paragraph*{Motivation} 34 cases (47\%) had a psychiatric motivation \cite{Al-Faham_2020k, Alao_2006i, Ali_2020f, Apikotoa_2022f, Ataya_2013, Atayan_2016, Bhasin_2014, Bhattacharjee_2008, DelgadoSalazar_2020c, DivsalarP._2023a, Emamhadi_2018, Farhadi_2024h, Guinan_2019f, Hardy_2023g, Jehangir_2019h, Jin_2023, Kar_2015, Kariholu_2008, Kerestes_2019, Kobiela_2015, Kumar_2001, Kumar_2019f, Li_2013, Liu_2005, Misra_2013, Ohno_2005, Sakellaridis_2008f, Sultan_2024f, Tammana_2012j, Tanrikulu_2015e, Yasin_2009, teWildt_2010}, 21 cases (29\%) were motivated by self-harm intention \cite{Al-Faham_2020k, AlShaaibi_2021b, Alao_2006i, Ali_2017, CamachoDorado_2018, Chang_2017f, Cox_2007, Csaky_1998e, Fry_2010, Li_2013, Losanoff_1996, Losanoff_1997e, Mesfin_2022a, Sakellaridis_2008f, Tammana_2012j, Tanrikulu_2015e, fjbuilsRepeatedBehaviorDeliberate2024}, 17 cases (24\%) had a psychosocial motivation \cite{Akay_2015f, Benoist_2019e, Bhattacharjee_2008, Cauchi_2002, Goldman_1998f, Hardy_2023g, Kobiela_2015, Li_2013, Naji_2012f, Qureshi_2016, Riva_2018j, Sobnach_2011f, Tay_2004, Thapa_2019f, Tupesis_2004f, Wildhaber_2005, Wnęk_2015f}, 9 cases (12\%) were motivated by protest \cite{Bhumi_2024f, Gardner_2017h, Losanoff_1996, Losanoff_1997e, Tupesis_2004f}, 9 cases (12\%) had another documented motivation \cite{Ali_2020f, Ali_2022g, Emamhadi_2018, Guinan_2019f, Peixoto_2017f, Sakellaridis_2008f, Trgo_2012f, Wadhwa_2015e, Yildiz_2016e}. \paragraph*{Object Characteristics} 51 cases (71\%) ingested a large diameter object (\textgreater{}2.5cm) \cite{Akay_2015f, Al-Faham_2020k, AlShaaibi_2021b, Alao_2006i, Ali_2017, Ali_2022g, Apikotoa_2022f, Atayan_2016, Berry_2021e, Bhasin_2014, CamachoDorado_2018, Cauchi_2002, Chang_2017f, Cox_2007, Csaky_1998e, DivsalarP._2023a, Emamhadi_2018, Gardner_2017h, Guinan_2019f, Jehangir_2019h, Jin_2023, Kariholu_2008, Kerestes_2019, Kobiela_2015, Kumar_2001, Kumar_2019f, Losanoff_1996, Losanoff_1997e, Mesfin_2022a, Misra_2013, Naji_2012f, Ohno_2005, Peixoto_2017f, Qureshi_2016, Riva_2018j, Sakellaridis_2008f, Sultan_2024f, Tanrikulu_2015e, Thapa_2019f, Trgo_2012f, Wnęk_2015f, Yildiz_2016e, fjbuilsRepeatedBehaviorDeliberate2024, teWildt_2010}, 44 cases (61\%) ingested multiple objects \cite{Ali_2020f, Apikotoa_2022f, Ataya_2013, Atayan_2016, Beecroft_1998, Bhattacharjee_2008, Bhumi_2024f, CamachoDorado_2018, Cauchi_2002, Emamhadi_2018, Farhadi_2024h, Fry_2010, Goldman_1998f, Guinan_2019f, Hardy_2023g, Jehangir_2019h, Jin_2023, Kar_2015, Kariholu_2008, Kobiela_2015, Kumar_2001, Kumar_2019f, Li_2013, Liu_2005, Losanoff_1996, Mesfin_2022a, Misra_2013, Naji_2012f, Ohno_2005, Sobnach_2011f, Sultan_2024f, Tammana_2012j, Tanrikulu_2015e, Tay_2004, Thapa_2019f, Wadhwa_2015e, Wildhaber_2005, Yasin_2009, fjbuilsRepeatedBehaviorDeliberate2024, teWildt_2010}, 34 cases (47\%) ingested a sharp object \cite{AlShaaibi_2021b, Alao_2006i, Apikotoa_2022f, Ataya_2013, Benoist_2019e, Bhasin_2014, Bhattacharjee_2008, CamachoDorado_2018, Csaky_1998e, DelgadoSalazar_2020c, DivsalarP._2023a, Emamhadi_2018, Farhadi_2024h, Fry_2010, Guinan_2019f, Hardy_2023g, Jehangir_2019h, Jin_2023, Kariholu_2008, Kobiela_2015, Kumar_2019f, Losanoff_1996, Losanoff_1997e, Mesfin_2022a, Misra_2013, Sobnach_2011f, Yasin_2009, teWildt_2010}, 32 cases (44\%) ingested a long object (\textgreater{}5cm) \cite{Al-Faham_2020k, AlShaaibi_2021b, Ali_2017, Ali_2022g, Atayan_2016, Bhasin_2014, CamachoDorado_2018, Chang_2017f, Cox_2007, Csaky_1998e, DivsalarP._2023a, Emamhadi_2018, Fry_2010, Gardner_2017h, Jin_2023, Kariholu_2008, Kerestes_2019, Kobiela_2015, Kumar_2019f, Mesfin_2022a, Misra_2013, Ohno_2005, Qureshi_2016, Sakellaridis_2008f, Sultan_2024f, Thapa_2019f, Trgo_2012f, Yasin_2009, Yildiz_2016e, teWildt_2010}, 9 cases (12\%) ingested a magnet \cite{Ali_2020f, Bhumi_2024f, Cauchi_2002, Liu_2005, Naji_2012f, Ohno_2005, Tanrikulu_2015e, Tay_2004, Wildhaber_2005}, 2 cases (3\%) ingested a button battery \cite{Berry_2021e, Bhumi_2024f}. \paragraph*{Outcomes} 48 cases (67\%) experienced a complication \cite{Ali_2017, Ali_2020f, Apikotoa_2022f, Atayan_2016, Beecroft_1998, Benoist_2019e, Berry_2021e, Bhasin_2014, Bhumi_2024f, CamachoDorado_2018, Cauchi_2002, Cox_2007, Csaky_1998e, DelgadoSalazar_2020c, DivsalarP._2023a, Emamhadi_2018, Farhadi_2024h, Fry_2010, Gardner_2017h, Goldman_1998f, Jin_2023, Kariholu_2008, Kerestes_2019, Kobiela_2015, Kumar_2001, Kumar_2019f, Liu_2005, Losanoff_1996, Mesfin_2022a, Misra_2013, Naji_2012f, Ohno_2005, Sakellaridis_2008f, Sobnach_2011f, Sultan_2024f, Tanrikulu_2015e, Tay_2004, Thapa_2019f, Trgo_2012f, Tupesis_2004f, Wildhaber_2005, Wnęk_2015f, Yasin_2009, Yildiz_2016e}, 44 cases (61\%) underwent surgery \cite{Al-Faham_2020k, AlShaaibi_2021b, Alao_2006i, Ali_2017, Ali_2020f, Atayan_2016, Beecroft_1998, Bhasin_2014, CamachoDorado_2018, Cauchi_2002, Chang_2017f, Cox_2007, Csaky_1998e, DelgadoSalazar_2020c, DivsalarP._2023a, Farhadi_2024h, Fry_2010, Gardner_2017h, Jin_2023, Kariholu_2008, Kerestes_2019, Kobiela_2015, Kumar_2019f, Liu_2005, Losanoff_1996, Losanoff_1997e, Mesfin_2022a, Misra_2013, Naji_2012f, Sobnach_2011f, Tanrikulu_2015e, Tay_2004, Thapa_2019f, Tupesis_2004f, Wildhaber_2005, Wnęk_2015f, Yasin_2009, Yildiz_2016e, fjbuilsRepeatedBehaviorDeliberate2024}, 31 cases (43\%) underwent endoscopy \cite{Akay_2015f, Ali_2022g, Apikotoa_2022f, Atayan_2016, Benoist_2019e, Berry_2021e, Bhasin_2014, Bhumi_2024f, CamachoDorado_2018, Chang_2017f, DelgadoSalazar_2020c, Gardner_2017h, Guinan_2019f, Hardy_2023g, Jehangir_2019h, Kariholu_2008, Li_2013, Liu_2005, Ohno_2005, Peixoto_2017f, Qureshi_2016, Riva_2018j, Sakellaridis_2008f, Sultan_2024f, Tammana_2012j, Tanrikulu_2015e, Trgo_2012f, Wadhwa_2015e, Wnęk_2015f, teWildt_2010}, 7 cases (10\%) were managed conservatively \cite{Ataya_2013, Bhattacharjee_2008, DivsalarP._2023a, Emamhadi_2018, Goldman_1998f, Kar_2015, Kumar_2001}, 2 cases (3\%) died \cite{Emamhadi_2018, Kumar_2001}. All 90 were male gender. 90 cases (100\%) were detained at the time of ingestion \cite{Elghali_2016, Karp_1991b, Lee_2007}, 88 cases (98\%) were intentional ingestions \cite{Elghali_2016, Karp_1991b, Lee_2007}, 30 cases (33\%) had a psychiatric history documented \cite{Elghali_2016, Karp_1991b, Lee_2007}, 2 cases (2\%) had a history of prior ingestion \cite{Elghali_2016}. No cases were reported for were psychiatric inpatients, were displaced people, were under the influence of alcohol at the time of ingestion, and had a severe disability history.
\paragraph*{Motivation}  70 cases (78\%) reported protest motivation \cite{Elghali_2016, Karp_1991b, Lee_2007}, 12 cases (13\%) reported psychiatric motivation \cite{Karp_1991b}, 6 cases (7\%) reported self-harm motivation \cite{Elghali_2016, Karp_1991b}. No cases were reported for psychosocial motivation and other motivation.
\paragraph*{Object Characteristics}  68 cases (76\%) involved sharp object ingestion \cite{Elghali_2016, Karp_1991b, Lee_2007}, 32 cases (36\%) involved long (\textgreater 5cm) object ingestion \cite{Lee_2007}, 25 cases (28\%) involved ingestion of multiple objects \cite{Elghali_2016, Lee_2007}. No cases were reported for button battery ingestion, magnet ingestion, and involved large diameter (\textgreater 2.5cm) object ingestion.
\paragraph*{Outcomes}  47 cases (52\%) underwent endoscopic intervention \cite{Elghali_2016, Lee_2007}, 29 cases (32\%) were managed conservatively \cite{Elghali_2016, Karp_1991b}, 15 cases (17\%) underwent surgical intervention \cite{Elghali_2016, Karp_1991b, Lee_2007}, 6 cases (7\%) reported complications \cite{Lee_2007}, 1 case (1\%) died \cite{Elghali_2016}.
\paragraph*{Geographical Location}Cases were recorded in 33 countries: 13 cases from USA \cite{Alao_2006i, Ataya_2013, Bhumi_2024f, Fry_2010, Guinan_2019f, Hardy_2023g, Jehangir_2019h, Kerestes_2019, Kumar_2001, Liu_2005, Tammana_2012j, Tay_2004, Tupesis_2004f}; 7 cases from India \cite{Bhasin_2014, Bhattacharjee_2008, Kar_2015, Kariholu_2008, Kumar_2019f, Misra_2013, Wadhwa_2015e} and UK \cite{Beecroft_1998, Berry_2021e, Cauchi_2002, Cox_2007, Gardner_2017h, Qureshi_2016}; 6 cases from Bulgaria \cite{Losanoff_1996, Losanoff_1997e}; 5 cases from Iran \cite{DivsalarP._2023a, Emamhadi_2018, Farhadi_2024h}; 4 cases from Turkey \cite{Akay_2015f, Atayan_2016, Tanrikulu_2015e, Yildiz_2016e}; 2 cases from China \cite{Jin_2023, Li_2013}, Poland \cite{Kobiela_2015, Wnęk_2015f}, and Spain \cite{CamachoDorado_2018, fjbuilsRepeatedBehaviorDeliberate2024}; 1 case from Australia \cite{Apikotoa_2022f}, Bahrain \cite{Ali_2020f}, Croatia \cite{Trgo_2012f}, Ecuador \cite{DelgadoSalazar_2020c}, Egypt \cite{Ali_2022g}, Ethiopia \cite{Mesfin_2022a}, Germany \cite{teWildt_2010}, Greece \cite{Sakellaridis_2008f}, Hungary \cite{Csaky_1998e}, Iraq \cite{Al-Faham_2020k}, Israel \cite{Goldman_1998f}, Italy \cite{Riva_2018j}, Japan \cite{Ohno_2005}, Nepal \cite{Thapa_2019f}, Netherlands \cite{Benoist_2019e}, Oman \cite{AlShaaibi_2021b}, Pakistan \cite{Yasin_2009}, Portugal \cite{Peixoto_2017f}, Qatar \cite{Ali_2017}, Saudi Arabia \cite{Sultan_2024f}, South Africa \cite{Sobnach_2011f}, Sweden \cite{Naji_2012f}, Switzerland \cite{Wildhaber_2005}, and Taiwan \cite{Chang_2017f}. \paragraph*{Gender} 43 cases (60\%) were male \cite{Akay_2015f, Al-Faham_2020k, Alao_2006i, Ali_2017, Ali_2022g, Apikotoa_2022f, Atayan_2016, Benoist_2019e, Berry_2021e, Bhumi_2024f, CamachoDorado_2018, Csaky_1998e, Emamhadi_2018, Farhadi_2024h, Fry_2010, Gardner_2017h, Guinan_2019f, Jehangir_2019h, Jin_2023, Kobiela_2015, Kumar_2001, Kumar_2019f, Liu_2005, Losanoff_1996, Losanoff_1997e, Mesfin_2022a, Misra_2013, Qureshi_2016, Riva_2018j, Sobnach_2011f, Tammana_2012j, Tanrikulu_2015e, Tay_2004, Thapa_2019f, Trgo_2012f, Wadhwa_2015e, Yasin_2009, teWildt_2010}, 28 cases (39\%) were female \cite{AlShaaibi_2021b, Ali_2020f, Ataya_2013, Beecroft_1998, Bhasin_2014, Bhattacharjee_2008, Cauchi_2002, Chang_2017f, Cox_2007, DelgadoSalazar_2020c, DivsalarP._2023a, Goldman_1998f, Hardy_2023g, Kar_2015, Kariholu_2008, Kerestes_2019, Li_2013, Naji_2012f, Ohno_2005, Peixoto_2017f, Sakellaridis_2008f, Sultan_2024f, Tupesis_2004f, Wildhaber_2005, Wnęk_2015f, Yildiz_2016e}, 1 case (1\%) had no gender recorded \cite{fjbuilsRepeatedBehaviorDeliberate2024}. \paragraph*{Age Group} 25 cases (35\%) were between 26 and 40 years of age \cite{Alao_2006i, Ali_2022g, Apikotoa_2022f, Ataya_2013, Benoist_2019e, Bhasin_2014, Chang_2017f, Cox_2007, DelgadoSalazar_2020c, Farhadi_2024h, Fry_2010, Gardner_2017h, Guinan_2019f, Jin_2023, Kumar_2019f, Losanoff_1996, Misra_2013, Qureshi_2016, Riva_2018j, Sakellaridis_2008f, Tammana_2012j, Trgo_2012f, Wnęk_2015f, Yildiz_2016e, fjbuilsRepeatedBehaviorDeliberate2024}, 18 cases (25\%) were between 18 and 25 years of age \cite{Akay_2015f, Ali_2017, Atayan_2016, Bhattacharjee_2008, Csaky_1998e, Kar_2015, Kariholu_2008, Kobiela_2015, Losanoff_1996, Losanoff_1997e, Mesfin_2022a, Peixoto_2017f, Sobnach_2011f, Tupesis_2004f, Yasin_2009}, 13 cases (18\%) were under 18 years of age \cite{AlShaaibi_2021b, Ali_2020f, Cauchi_2002, DivsalarP._2023a, Goldman_1998f, Liu_2005, Naji_2012f, Ohno_2005, Tanrikulu_2015e, Tay_2004, Wildhaber_2005}, 11 cases (15\%) were between 41 and 60 years of age \cite{Al-Faham_2020k, Bhumi_2024f, CamachoDorado_2018, Emamhadi_2018, Hardy_2023g, Jehangir_2019h, Kumar_2001, Sultan_2024f, Thapa_2019f, Wadhwa_2015e, teWildt_2010}, 3 cases (4\%) were over 60 years of age \cite{Beecroft_1998, Kerestes_2019, Li_2013}, 2 cases (3\%) had no age documented \cite{Berry_2021e}. \paragraph*{Population} 36 cases (50\%) had a psychiatric history \cite{AlShaaibi_2021b, Alao_2006i, Ali_2020f, Apikotoa_2022f, Ataya_2013, Atayan_2016, Beecroft_1998, CamachoDorado_2018, Chang_2017f, DelgadoSalazar_2020c, DivsalarP._2023a, Farhadi_2024h, Fry_2010, Guinan_2019f, Hardy_2023g, Jehangir_2019h, Jin_2023, Kar_2015, Kerestes_2019, Kobiela_2015, Kumar_2001, Kumar_2019f, Liu_2005, Mesfin_2022a, Misra_2013, Ohno_2005, Peixoto_2017f, Sakellaridis_2008f, Sultan_2024f, Tammana_2012j, Tanrikulu_2015e, Yildiz_2016e, fjbuilsRepeatedBehaviorDeliberate2024, teWildt_2010}, 19 cases (26\%) had ingested previously \cite{Alao_2006i, Apikotoa_2022f, Berry_2021e, Bhattacharjee_2008, Csaky_1998e, DivsalarP._2023a, Emamhadi_2018, Guinan_2019f, Jehangir_2019h, Jin_2023, Liu_2005, Sakellaridis_2008f, Tanrikulu_2015e, Thapa_2019f, Yildiz_2016e, fjbuilsRepeatedBehaviorDeliberate2024, teWildt_2010}, 12 cases (17\%) were detained persons \cite{Alao_2006i, Ali_2022g, Apikotoa_2022f, Losanoff_1996, Losanoff_1997e, Qureshi_2016, Tammana_2012j, Trgo_2012f}, 7 cases (10\%) were severely disabled \cite{Atayan_2016, Kerestes_2019, Liu_2005, Ohno_2005, Peixoto_2017f, Yildiz_2016e, teWildt_2010}, 4 cases (6\%) were psychiatric inpatients \cite{DivsalarP._2023a, fjbuilsRepeatedBehaviorDeliberate2024, teWildt_2010}, 3 cases (4\%) were under the influence of alcohol \cite{Benoist_2019e, Csaky_1998e, Thapa_2019f}, 2 cases (3\%) were displaced people \cite{Akay_2015f, Gardner_2017h}. \paragraph*{Motivation} 34 cases (47\%) had a psychiatric motivation \cite{Al-Faham_2020k, Alao_2006i, Ali_2020f, Apikotoa_2022f, Ataya_2013, Atayan_2016, Bhasin_2014, Bhattacharjee_2008, DelgadoSalazar_2020c, DivsalarP._2023a, Emamhadi_2018, Farhadi_2024h, Guinan_2019f, Hardy_2023g, Jehangir_2019h, Jin_2023, Kar_2015, Kariholu_2008, Kerestes_2019, Kobiela_2015, Kumar_2001, Kumar_2019f, Li_2013, Liu_2005, Misra_2013, Ohno_2005, Sakellaridis_2008f, Sultan_2024f, Tammana_2012j, Tanrikulu_2015e, Yasin_2009, teWildt_2010}, 21 cases (29\%) were motivated by self-harm intention \cite{Al-Faham_2020k, AlShaaibi_2021b, Alao_2006i, Ali_2017, CamachoDorado_2018, Chang_2017f, Cox_2007, Csaky_1998e, Fry_2010, Li_2013, Losanoff_1996, Losanoff_1997e, Mesfin_2022a, Sakellaridis_2008f, Tammana_2012j, Tanrikulu_2015e, fjbuilsRepeatedBehaviorDeliberate2024}, 17 cases (24\%) had a psychosocial motivation \cite{Akay_2015f, Benoist_2019e, Bhattacharjee_2008, Cauchi_2002, Goldman_1998f, Hardy_2023g, Kobiela_2015, Li_2013, Naji_2012f, Qureshi_2016, Riva_2018j, Sobnach_2011f, Tay_2004, Thapa_2019f, Tupesis_2004f, Wildhaber_2005, Wnęk_2015f}, 9 cases (12\%) were motivated by protest \cite{Bhumi_2024f, Gardner_2017h, Losanoff_1996, Losanoff_1997e, Tupesis_2004f}, 9 cases (12\%) had another documented motivation \cite{Ali_2020f, Ali_2022g, Emamhadi_2018, Guinan_2019f, Peixoto_2017f, Sakellaridis_2008f, Trgo_2012f, Wadhwa_2015e, Yildiz_2016e}. \paragraph*{Object Characteristics} 51 cases (71\%) ingested a large diameter object (\textgreater{}2.5cm) \cite{Akay_2015f, Al-Faham_2020k, AlShaaibi_2021b, Alao_2006i, Ali_2017, Ali_2022g, Apikotoa_2022f, Atayan_2016, Berry_2021e, Bhasin_2014, CamachoDorado_2018, Cauchi_2002, Chang_2017f, Cox_2007, Csaky_1998e, DivsalarP._2023a, Emamhadi_2018, Gardner_2017h, Guinan_2019f, Jehangir_2019h, Jin_2023, Kariholu_2008, Kerestes_2019, Kobiela_2015, Kumar_2001, Kumar_2019f, Losanoff_1996, Losanoff_1997e, Mesfin_2022a, Misra_2013, Naji_2012f, Ohno_2005, Peixoto_2017f, Qureshi_2016, Riva_2018j, Sakellaridis_2008f, Sultan_2024f, Tanrikulu_2015e, Thapa_2019f, Trgo_2012f, Wnęk_2015f, Yildiz_2016e, fjbuilsRepeatedBehaviorDeliberate2024, teWildt_2010}, 44 cases (61\%) ingested multiple objects \cite{Ali_2020f, Apikotoa_2022f, Ataya_2013, Atayan_2016, Beecroft_1998, Bhattacharjee_2008, Bhumi_2024f, CamachoDorado_2018, Cauchi_2002, Emamhadi_2018, Farhadi_2024h, Fry_2010, Goldman_1998f, Guinan_2019f, Hardy_2023g, Jehangir_2019h, Jin_2023, Kar_2015, Kariholu_2008, Kobiela_2015, Kumar_2001, Kumar_2019f, Li_2013, Liu_2005, Losanoff_1996, Mesfin_2022a, Misra_2013, Naji_2012f, Ohno_2005, Sobnach_2011f, Sultan_2024f, Tammana_2012j, Tanrikulu_2015e, Tay_2004, Thapa_2019f, Wadhwa_2015e, Wildhaber_2005, Yasin_2009, fjbuilsRepeatedBehaviorDeliberate2024, teWildt_2010}, 34 cases (47\%) ingested a sharp object \cite{AlShaaibi_2021b, Alao_2006i, Apikotoa_2022f, Ataya_2013, Benoist_2019e, Bhasin_2014, Bhattacharjee_2008, CamachoDorado_2018, Csaky_1998e, DelgadoSalazar_2020c, DivsalarP._2023a, Emamhadi_2018, Farhadi_2024h, Fry_2010, Guinan_2019f, Hardy_2023g, Jehangir_2019h, Jin_2023, Kariholu_2008, Kobiela_2015, Kumar_2019f, Losanoff_1996, Losanoff_1997e, Mesfin_2022a, Misra_2013, Sobnach_2011f, Yasin_2009, teWildt_2010}, 32 cases (44\%) ingested a long object (\textgreater{}5cm) \cite{Al-Faham_2020k, AlShaaibi_2021b, Ali_2017, Ali_2022g, Atayan_2016, Bhasin_2014, CamachoDorado_2018, Chang_2017f, Cox_2007, Csaky_1998e, DivsalarP._2023a, Emamhadi_2018, Fry_2010, Gardner_2017h, Jin_2023, Kariholu_2008, Kerestes_2019, Kobiela_2015, Kumar_2019f, Mesfin_2022a, Misra_2013, Ohno_2005, Qureshi_2016, Sakellaridis_2008f, Sultan_2024f, Thapa_2019f, Trgo_2012f, Yasin_2009, Yildiz_2016e, teWildt_2010}, 9 cases (12\%) ingested a magnet \cite{Ali_2020f, Bhumi_2024f, Cauchi_2002, Liu_2005, Naji_2012f, Ohno_2005, Tanrikulu_2015e, Tay_2004, Wildhaber_2005}, 2 cases (3\%) ingested a button battery \cite{Berry_2021e, Bhumi_2024f}. \paragraph*{Outcomes} 48 cases (67\%) experienced a complication \cite{Ali_2017, Ali_2020f, Apikotoa_2022f, Atayan_2016, Beecroft_1998, Benoist_2019e, Berry_2021e, Bhasin_2014, Bhumi_2024f, CamachoDorado_2018, Cauchi_2002, Cox_2007, Csaky_1998e, DelgadoSalazar_2020c, DivsalarP._2023a, Emamhadi_2018, Farhadi_2024h, Fry_2010, Gardner_2017h, Goldman_1998f, Jin_2023, Kariholu_2008, Kerestes_2019, Kobiela_2015, Kumar_2001, Kumar_2019f, Liu_2005, Losanoff_1996, Mesfin_2022a, Misra_2013, Naji_2012f, Ohno_2005, Sakellaridis_2008f, Sobnach_2011f, Sultan_2024f, Tanrikulu_2015e, Tay_2004, Thapa_2019f, Trgo_2012f, Tupesis_2004f, Wildhaber_2005, Wnęk_2015f, Yasin_2009, Yildiz_2016e}, 44 cases (61\%) underwent surgery \cite{Al-Faham_2020k, AlShaaibi_2021b, Alao_2006i, Ali_2017, Ali_2020f, Atayan_2016, Beecroft_1998, Bhasin_2014, CamachoDorado_2018, Cauchi_2002, Chang_2017f, Cox_2007, Csaky_1998e, DelgadoSalazar_2020c, DivsalarP._2023a, Farhadi_2024h, Fry_2010, Gardner_2017h, Jin_2023, Kariholu_2008, Kerestes_2019, Kobiela_2015, Kumar_2019f, Liu_2005, Losanoff_1996, Losanoff_1997e, Mesfin_2022a, Misra_2013, Naji_2012f, Sobnach_2011f, Tanrikulu_2015e, Tay_2004, Thapa_2019f, Tupesis_2004f, Wildhaber_2005, Wnęk_2015f, Yasin_2009, Yildiz_2016e, fjbuilsRepeatedBehaviorDeliberate2024}, 31 cases (43\%) underwent endoscopy \cite{Akay_2015f, Ali_2022g, Apikotoa_2022f, Atayan_2016, Benoist_2019e, Berry_2021e, Bhasin_2014, Bhumi_2024f, CamachoDorado_2018, Chang_2017f, DelgadoSalazar_2020c, Gardner_2017h, Guinan_2019f, Hardy_2023g, Jehangir_2019h, Kariholu_2008, Li_2013, Liu_2005, Ohno_2005, Peixoto_2017f, Qureshi_2016, Riva_2018j, Sakellaridis_2008f, Sultan_2024f, Tammana_2012j, Tanrikulu_2015e, Trgo_2012f, Wadhwa_2015e, Wnęk_2015f, teWildt_2010}, 7 cases (10\%) were managed conservatively \cite{Ataya_2013, Bhattacharjee_2008, DivsalarP._2023a, Emamhadi_2018, Goldman_1998f, Kar_2015, Kumar_2001}, 2 cases (3\%) died \cite{Emamhadi_2018, Kumar_2001}. All 90 were male gender. 90 cases (100\%) were detained at the time of ingestion \cite{Elghali_2016, Karp_1991b, Lee_2007}, 88 cases (98\%) were intentional ingestions \cite{Elghali_2016, Karp_1991b, Lee_2007}, 30 cases (33\%) had a psychiatric history documented \cite{Elghali_2016, Karp_1991b, Lee_2007}, 2 cases (2\%) had a history of prior ingestion \cite{Elghali_2016}. No cases were reported for were psychiatric inpatients, were displaced people, were under the influence of alcohol at the time of ingestion, and had a severe disability history.
\paragraph*{Motivation}  70 cases (78\%) reported protest motivation \cite{Elghali_2016, Karp_1991b, Lee_2007}, 12 cases (13\%) reported psychiatric motivation \cite{Karp_1991b}, 6 cases (7\%) reported self-harm motivation \cite{Elghali_2016, Karp_1991b}. No cases were reported for psychosocial motivation and other motivation.
\paragraph*{Object Characteristics}  68 cases (76\%) involved sharp object ingestion \cite{Elghali_2016, Karp_1991b, Lee_2007}, 32 cases (36\%) involved long (\textgreater 5cm) object ingestion \cite{Lee_2007}, 25 cases (28\%) involved ingestion of multiple objects \cite{Elghali_2016, Lee_2007}. No cases were reported for button battery ingestion, magnet ingestion, and involved large diameter (\textgreater 2.5cm) object ingestion.
\paragraph*{Outcomes}  47 cases (52\%) underwent endoscopic intervention \cite{Elghali_2016, Lee_2007}, 29 cases (32\%) were managed conservatively \cite{Elghali_2016, Karp_1991b}, 15 cases (17\%) underwent surgical intervention \cite{Elghali_2016, Karp_1991b, Lee_2007}, 6 cases (7\%) reported complications \cite{Lee_2007}, 1 case (1\%) died \cite{Elghali_2016}.
\paragraph*{Geographical Location}Cases were recorded in 33 countries: 13 cases from USA \cite{Alao_2006i, Ataya_2013, Bhumi_2024f, Fry_2010, Guinan_2019f, Hardy_2023g, Jehangir_2019h, Kerestes_2019, Kumar_2001, Liu_2005, Tammana_2012j, Tay_2004, Tupesis_2004f}; 7 cases from India \cite{Bhasin_2014, Bhattacharjee_2008, Kar_2015, Kariholu_2008, Kumar_2019f, Misra_2013, Wadhwa_2015e} and UK \cite{Beecroft_1998, Berry_2021e, Cauchi_2002, Cox_2007, Gardner_2017h, Qureshi_2016}; 6 cases from Bulgaria \cite{Losanoff_1996, Losanoff_1997e}; 5 cases from Iran \cite{DivsalarP._2023a, Emamhadi_2018, Farhadi_2024h}; 4 cases from Turkey \cite{Akay_2015f, Atayan_2016, Tanrikulu_2015e, Yildiz_2016e}; 2 cases from China \cite{Jin_2023, Li_2013}, Poland \cite{Kobiela_2015, Wnęk_2015f}, and Spain \cite{CamachoDorado_2018, fjbuilsRepeatedBehaviorDeliberate2024}; 1 case from Australia \cite{Apikotoa_2022f}, Bahrain \cite{Ali_2020f}, Croatia \cite{Trgo_2012f}, Ecuador \cite{DelgadoSalazar_2020c}, Egypt \cite{Ali_2022g}, Ethiopia \cite{Mesfin_2022a}, Germany \cite{teWildt_2010}, Greece \cite{Sakellaridis_2008f}, Hungary \cite{Csaky_1998e}, Iraq \cite{Al-Faham_2020k}, Israel \cite{Goldman_1998f}, Italy \cite{Riva_2018j}, Japan \cite{Ohno_2005}, Nepal \cite{Thapa_2019f}, Netherlands \cite{Benoist_2019e}, Oman \cite{AlShaaibi_2021b}, Pakistan \cite{Yasin_2009}, Portugal \cite{Peixoto_2017f}, Qatar \cite{Ali_2017}, Saudi Arabia \cite{Sultan_2024f}, South Africa \cite{Sobnach_2011f}, Sweden \cite{Naji_2012f}, Switzerland \cite{Wildhaber_2005}, and Taiwan \cite{Chang_2017f}. \paragraph*{Gender} 43 cases (60\%) were male \cite{Akay_2015f, Al-Faham_2020k, Alao_2006i, Ali_2017, Ali_2022g, Apikotoa_2022f, Atayan_2016, Benoist_2019e, Berry_2021e, Bhumi_2024f, CamachoDorado_2018, Csaky_1998e, Emamhadi_2018, Farhadi_2024h, Fry_2010, Gardner_2017h, Guinan_2019f, Jehangir_2019h, Jin_2023, Kobiela_2015, Kumar_2001, Kumar_2019f, Liu_2005, Losanoff_1996, Losanoff_1997e, Mesfin_2022a, Misra_2013, Qureshi_2016, Riva_2018j, Sobnach_2011f, Tammana_2012j, Tanrikulu_2015e, Tay_2004, Thapa_2019f, Trgo_2012f, Wadhwa_2015e, Yasin_2009, teWildt_2010}, 28 cases (39\%) were female \cite{AlShaaibi_2021b, Ali_2020f, Ataya_2013, Beecroft_1998, Bhasin_2014, Bhattacharjee_2008, Cauchi_2002, Chang_2017f, Cox_2007, DelgadoSalazar_2020c, DivsalarP._2023a, Goldman_1998f, Hardy_2023g, Kar_2015, Kariholu_2008, Kerestes_2019, Li_2013, Naji_2012f, Ohno_2005, Peixoto_2017f, Sakellaridis_2008f, Sultan_2024f, Tupesis_2004f, Wildhaber_2005, Wnęk_2015f, Yildiz_2016e}, 1 case (1\%) had no gender recorded \cite{fjbuilsRepeatedBehaviorDeliberate2024}. \paragraph*{Age Group} 25 cases (35\%) were between 26 and 40 years of age \cite{Alao_2006i, Ali_2022g, Apikotoa_2022f, Ataya_2013, Benoist_2019e, Bhasin_2014, Chang_2017f, Cox_2007, DelgadoSalazar_2020c, Farhadi_2024h, Fry_2010, Gardner_2017h, Guinan_2019f, Jin_2023, Kumar_2019f, Losanoff_1996, Misra_2013, Qureshi_2016, Riva_2018j, Sakellaridis_2008f, Tammana_2012j, Trgo_2012f, Wnęk_2015f, Yildiz_2016e, fjbuilsRepeatedBehaviorDeliberate2024}, 18 cases (25\%) were between 18 and 25 years of age \cite{Akay_2015f, Ali_2017, Atayan_2016, Bhattacharjee_2008, Csaky_1998e, Kar_2015, Kariholu_2008, Kobiela_2015, Losanoff_1996, Losanoff_1997e, Mesfin_2022a, Peixoto_2017f, Sobnach_2011f, Tupesis_2004f, Yasin_2009}, 13 cases (18\%) were under 18 years of age \cite{AlShaaibi_2021b, Ali_2020f, Cauchi_2002, DivsalarP._2023a, Goldman_1998f, Liu_2005, Naji_2012f, Ohno_2005, Tanrikulu_2015e, Tay_2004, Wildhaber_2005}, 11 cases (15\%) were between 41 and 60 years of age \cite{Al-Faham_2020k, Bhumi_2024f, CamachoDorado_2018, Emamhadi_2018, Hardy_2023g, Jehangir_2019h, Kumar_2001, Sultan_2024f, Thapa_2019f, Wadhwa_2015e, teWildt_2010}, 3 cases (4\%) were over 60 years of age \cite{Beecroft_1998, Kerestes_2019, Li_2013}, 2 cases (3\%) had no age documented \cite{Berry_2021e}. \paragraph*{Population} 36 cases (50\%) had a psychiatric history \cite{AlShaaibi_2021b, Alao_2006i, Ali_2020f, Apikotoa_2022f, Ataya_2013, Atayan_2016, Beecroft_1998, CamachoDorado_2018, Chang_2017f, DelgadoSalazar_2020c, DivsalarP._2023a, Farhadi_2024h, Fry_2010, Guinan_2019f, Hardy_2023g, Jehangir_2019h, Jin_2023, Kar_2015, Kerestes_2019, Kobiela_2015, Kumar_2001, Kumar_2019f, Liu_2005, Mesfin_2022a, Misra_2013, Ohno_2005, Peixoto_2017f, Sakellaridis_2008f, Sultan_2024f, Tammana_2012j, Tanrikulu_2015e, Yildiz_2016e, fjbuilsRepeatedBehaviorDeliberate2024, teWildt_2010}, 19 cases (26\%) had ingested previously \cite{Alao_2006i, Apikotoa_2022f, Berry_2021e, Bhattacharjee_2008, Csaky_1998e, DivsalarP._2023a, Emamhadi_2018, Guinan_2019f, Jehangir_2019h, Jin_2023, Liu_2005, Sakellaridis_2008f, Tanrikulu_2015e, Thapa_2019f, Yildiz_2016e, fjbuilsRepeatedBehaviorDeliberate2024, teWildt_2010}, 12 cases (17\%) were detained persons \cite{Alao_2006i, Ali_2022g, Apikotoa_2022f, Losanoff_1996, Losanoff_1997e, Qureshi_2016, Tammana_2012j, Trgo_2012f}, 7 cases (10\%) were severely disabled \cite{Atayan_2016, Kerestes_2019, Liu_2005, Ohno_2005, Peixoto_2017f, Yildiz_2016e, teWildt_2010}, 4 cases (6\%) were psychiatric inpatients \cite{DivsalarP._2023a, fjbuilsRepeatedBehaviorDeliberate2024, teWildt_2010}, 3 cases (4\%) were under the influence of alcohol \cite{Benoist_2019e, Csaky_1998e, Thapa_2019f}, 2 cases (3\%) were displaced people \cite{Akay_2015f, Gardner_2017h}. \paragraph*{Motivation} 34 cases (47\%) had a psychiatric motivation \cite{Al-Faham_2020k, Alao_2006i, Ali_2020f, Apikotoa_2022f, Ataya_2013, Atayan_2016, Bhasin_2014, Bhattacharjee_2008, DelgadoSalazar_2020c, DivsalarP._2023a, Emamhadi_2018, Farhadi_2024h, Guinan_2019f, Hardy_2023g, Jehangir_2019h, Jin_2023, Kar_2015, Kariholu_2008, Kerestes_2019, Kobiela_2015, Kumar_2001, Kumar_2019f, Li_2013, Liu_2005, Misra_2013, Ohno_2005, Sakellaridis_2008f, Sultan_2024f, Tammana_2012j, Tanrikulu_2015e, Yasin_2009, teWildt_2010}, 21 cases (29\%) were motivated by self-harm intention \cite{Al-Faham_2020k, AlShaaibi_2021b, Alao_2006i, Ali_2017, CamachoDorado_2018, Chang_2017f, Cox_2007, Csaky_1998e, Fry_2010, Li_2013, Losanoff_1996, Losanoff_1997e, Mesfin_2022a, Sakellaridis_2008f, Tammana_2012j, Tanrikulu_2015e, fjbuilsRepeatedBehaviorDeliberate2024}, 17 cases (24\%) had a psychosocial motivation \cite{Akay_2015f, Benoist_2019e, Bhattacharjee_2008, Cauchi_2002, Goldman_1998f, Hardy_2023g, Kobiela_2015, Li_2013, Naji_2012f, Qureshi_2016, Riva_2018j, Sobnach_2011f, Tay_2004, Thapa_2019f, Tupesis_2004f, Wildhaber_2005, Wnęk_2015f}, 9 cases (12\%) were motivated by protest \cite{Bhumi_2024f, Gardner_2017h, Losanoff_1996, Losanoff_1997e, Tupesis_2004f}, 9 cases (12\%) had another documented motivation \cite{Ali_2020f, Ali_2022g, Emamhadi_2018, Guinan_2019f, Peixoto_2017f, Sakellaridis_2008f, Trgo_2012f, Wadhwa_2015e, Yildiz_2016e}. \paragraph*{Object Characteristics} 51 cases (71\%) ingested a large diameter object (\textgreater{}2.5cm) \cite{Akay_2015f, Al-Faham_2020k, AlShaaibi_2021b, Alao_2006i, Ali_2017, Ali_2022g, Apikotoa_2022f, Atayan_2016, Berry_2021e, Bhasin_2014, CamachoDorado_2018, Cauchi_2002, Chang_2017f, Cox_2007, Csaky_1998e, DivsalarP._2023a, Emamhadi_2018, Gardner_2017h, Guinan_2019f, Jehangir_2019h, Jin_2023, Kariholu_2008, Kerestes_2019, Kobiela_2015, Kumar_2001, Kumar_2019f, Losanoff_1996, Losanoff_1997e, Mesfin_2022a, Misra_2013, Naji_2012f, Ohno_2005, Peixoto_2017f, Qureshi_2016, Riva_2018j, Sakellaridis_2008f, Sultan_2024f, Tanrikulu_2015e, Thapa_2019f, Trgo_2012f, Wnęk_2015f, Yildiz_2016e, fjbuilsRepeatedBehaviorDeliberate2024, teWildt_2010}, 44 cases (61\%) ingested multiple objects \cite{Ali_2020f, Apikotoa_2022f, Ataya_2013, Atayan_2016, Beecroft_1998, Bhattacharjee_2008, Bhumi_2024f, CamachoDorado_2018, Cauchi_2002, Emamhadi_2018, Farhadi_2024h, Fry_2010, Goldman_1998f, Guinan_2019f, Hardy_2023g, Jehangir_2019h, Jin_2023, Kar_2015, Kariholu_2008, Kobiela_2015, Kumar_2001, Kumar_2019f, Li_2013, Liu_2005, Losanoff_1996, Mesfin_2022a, Misra_2013, Naji_2012f, Ohno_2005, Sobnach_2011f, Sultan_2024f, Tammana_2012j, Tanrikulu_2015e, Tay_2004, Thapa_2019f, Wadhwa_2015e, Wildhaber_2005, Yasin_2009, fjbuilsRepeatedBehaviorDeliberate2024, teWildt_2010}, 34 cases (47\%) ingested a sharp object \cite{AlShaaibi_2021b, Alao_2006i, Apikotoa_2022f, Ataya_2013, Benoist_2019e, Bhasin_2014, Bhattacharjee_2008, CamachoDorado_2018, Csaky_1998e, DelgadoSalazar_2020c, DivsalarP._2023a, Emamhadi_2018, Farhadi_2024h, Fry_2010, Guinan_2019f, Hardy_2023g, Jehangir_2019h, Jin_2023, Kariholu_2008, Kobiela_2015, Kumar_2019f, Losanoff_1996, Losanoff_1997e, Mesfin_2022a, Misra_2013, Sobnach_2011f, Yasin_2009, teWildt_2010}, 32 cases (44\%) ingested a long object (\textgreater{}5cm) \cite{Al-Faham_2020k, AlShaaibi_2021b, Ali_2017, Ali_2022g, Atayan_2016, Bhasin_2014, CamachoDorado_2018, Chang_2017f, Cox_2007, Csaky_1998e, DivsalarP._2023a, Emamhadi_2018, Fry_2010, Gardner_2017h, Jin_2023, Kariholu_2008, Kerestes_2019, Kobiela_2015, Kumar_2019f, Mesfin_2022a, Misra_2013, Ohno_2005, Qureshi_2016, Sakellaridis_2008f, Sultan_2024f, Thapa_2019f, Trgo_2012f, Yasin_2009, Yildiz_2016e, teWildt_2010}, 9 cases (12\%) ingested a magnet \cite{Ali_2020f, Bhumi_2024f, Cauchi_2002, Liu_2005, Naji_2012f, Ohno_2005, Tanrikulu_2015e, Tay_2004, Wildhaber_2005}, 2 cases (3\%) ingested a button battery \cite{Berry_2021e, Bhumi_2024f}. \paragraph*{Outcomes} 48 cases (67\%) experienced a complication \cite{Ali_2017, Ali_2020f, Apikotoa_2022f, Atayan_2016, Beecroft_1998, Benoist_2019e, Berry_2021e, Bhasin_2014, Bhumi_2024f, CamachoDorado_2018, Cauchi_2002, Cox_2007, Csaky_1998e, DelgadoSalazar_2020c, DivsalarP._2023a, Emamhadi_2018, Farhadi_2024h, Fry_2010, Gardner_2017h, Goldman_1998f, Jin_2023, Kariholu_2008, Kerestes_2019, Kobiela_2015, Kumar_2001, Kumar_2019f, Liu_2005, Losanoff_1996, Mesfin_2022a, Misra_2013, Naji_2012f, Ohno_2005, Sakellaridis_2008f, Sobnach_2011f, Sultan_2024f, Tanrikulu_2015e, Tay_2004, Thapa_2019f, Trgo_2012f, Tupesis_2004f, Wildhaber_2005, Wnęk_2015f, Yasin_2009, Yildiz_2016e}, 44 cases (61\%) underwent surgery \cite{Al-Faham_2020k, AlShaaibi_2021b, Alao_2006i, Ali_2017, Ali_2020f, Atayan_2016, Beecroft_1998, Bhasin_2014, CamachoDorado_2018, Cauchi_2002, Chang_2017f, Cox_2007, Csaky_1998e, DelgadoSalazar_2020c, DivsalarP._2023a, Farhadi_2024h, Fry_2010, Gardner_2017h, Jin_2023, Kariholu_2008, Kerestes_2019, Kobiela_2015, Kumar_2019f, Liu_2005, Losanoff_1996, Losanoff_1997e, Mesfin_2022a, Misra_2013, Naji_2012f, Sobnach_2011f, Tanrikulu_2015e, Tay_2004, Thapa_2019f, Tupesis_2004f, Wildhaber_2005, Wnęk_2015f, Yasin_2009, Yildiz_2016e, fjbuilsRepeatedBehaviorDeliberate2024}, 31 cases (43\%) underwent endoscopy \cite{Akay_2015f, Ali_2022g, Apikotoa_2022f, Atayan_2016, Benoist_2019e, Berry_2021e, Bhasin_2014, Bhumi_2024f, CamachoDorado_2018, Chang_2017f, DelgadoSalazar_2020c, Gardner_2017h, Guinan_2019f, Hardy_2023g, Jehangir_2019h, Kariholu_2008, Li_2013, Liu_2005, Ohno_2005, Peixoto_2017f, Qureshi_2016, Riva_2018j, Sakellaridis_2008f, Sultan_2024f, Tammana_2012j, Tanrikulu_2015e, Trgo_2012f, Wadhwa_2015e, Wnęk_2015f, teWildt_2010}, 7 cases (10\%) were managed conservatively \cite{Ataya_2013, Bhattacharjee_2008, DivsalarP._2023a, Emamhadi_2018, Goldman_1998f, Kar_2015, Kumar_2001}, 2 cases (3\%) died \cite{Emamhadi_2018, Kumar_2001}. All 90 were male gender. 90 cases (100\%) were detained at the time of ingestion \cite{Elghali_2016, Karp_1991b, Lee_2007}, 88 cases (98\%) were intentional ingestions \cite{Elghali_2016, Karp_1991b, Lee_2007}, 30 cases (33\%) had a psychiatric history documented \cite{Elghali_2016, Karp_1991b, Lee_2007}, 2 cases (2\%) had a history of prior ingestion \cite{Elghali_2016}. No cases were reported for were psychiatric inpatients, were displaced people, were under the influence of alcohol at the time of ingestion, and had a severe disability history.
\paragraph*{Motivation}  70 cases (78\%) reported protest motivation \cite{Elghali_2016, Karp_1991b, Lee_2007}, 12 cases (13\%) reported psychiatric motivation \cite{Karp_1991b}, 6 cases (7\%) reported self-harm motivation \cite{Elghali_2016, Karp_1991b}. No cases were reported for psychosocial motivation and other motivation.
\paragraph*{Object Characteristics}  68 cases (76\%) involved sharp object ingestion \cite{Elghali_2016, Karp_1991b, Lee_2007}, 32 cases (36\%) involved long (\textgreater 5cm) object ingestion \cite{Lee_2007}, 25 cases (28\%) involved ingestion of multiple objects \cite{Elghali_2016, Lee_2007}. No cases were reported for button battery ingestion, magnet ingestion, and involved large diameter (\textgreater 2.5cm) object ingestion.
\paragraph*{Outcomes}  47 cases (52\%) underwent endoscopic intervention \cite{Elghali_2016, Lee_2007}, 29 cases (32\%) were managed conservatively \cite{Elghali_2016, Karp_1991b}, 15 cases (17\%) underwent surgical intervention \cite{Elghali_2016, Karp_1991b, Lee_2007}, 6 cases (7\%) reported complications \cite{Lee_2007}, 1 case (1\%) died \cite{Elghali_2016}.
\paragraph*{Geographical Location}Cases were recorded in 33 countries: 13 cases from USA \cite{Alao_2006i, Ataya_2013, Bhumi_2024f, Fry_2010, Guinan_2019f, Hardy_2023g, Jehangir_2019h, Kerestes_2019, Kumar_2001, Liu_2005, Tammana_2012j, Tay_2004, Tupesis_2004f}; 7 cases from India \cite{Bhasin_2014, Bhattacharjee_2008, Kar_2015, Kariholu_2008, Kumar_2019f, Misra_2013, Wadhwa_2015e} and UK \cite{Beecroft_1998, Berry_2021e, Cauchi_2002, Cox_2007, Gardner_2017h, Qureshi_2016}; 6 cases from Bulgaria \cite{Losanoff_1996, Losanoff_1997e}; 5 cases from Iran \cite{DivsalarP._2023a, Emamhadi_2018, Farhadi_2024h}; 4 cases from Turkey \cite{Akay_2015f, Atayan_2016, Tanrikulu_2015e, Yildiz_2016e}; 2 cases from China \cite{Jin_2023, Li_2013}, Poland \cite{Kobiela_2015, Wnęk_2015f}, and Spain \cite{CamachoDorado_2018, fjbuilsRepeatedBehaviorDeliberate2024}; 1 case from Australia \cite{Apikotoa_2022f}, Bahrain \cite{Ali_2020f}, Croatia \cite{Trgo_2012f}, Ecuador \cite{DelgadoSalazar_2020c}, Egypt \cite{Ali_2022g}, Ethiopia \cite{Mesfin_2022a}, Germany \cite{teWildt_2010}, Greece \cite{Sakellaridis_2008f}, Hungary \cite{Csaky_1998e}, Iraq \cite{Al-Faham_2020k}, Israel \cite{Goldman_1998f}, Italy \cite{Riva_2018j}, Japan \cite{Ohno_2005}, Nepal \cite{Thapa_2019f}, Netherlands \cite{Benoist_2019e}, Oman \cite{AlShaaibi_2021b}, Pakistan \cite{Yasin_2009}, Portugal \cite{Peixoto_2017f}, Qatar \cite{Ali_2017}, Saudi Arabia \cite{Sultan_2024f}, South Africa \cite{Sobnach_2011f}, Sweden \cite{Naji_2012f}, Switzerland \cite{Wildhaber_2005}, and Taiwan \cite{Chang_2017f}. \paragraph*{Gender} 43 cases (60\%) were male \cite{Akay_2015f, Al-Faham_2020k, Alao_2006i, Ali_2017, Ali_2022g, Apikotoa_2022f, Atayan_2016, Benoist_2019e, Berry_2021e, Bhumi_2024f, CamachoDorado_2018, Csaky_1998e, Emamhadi_2018, Farhadi_2024h, Fry_2010, Gardner_2017h, Guinan_2019f, Jehangir_2019h, Jin_2023, Kobiela_2015, Kumar_2001, Kumar_2019f, Liu_2005, Losanoff_1996, Losanoff_1997e, Mesfin_2022a, Misra_2013, Qureshi_2016, Riva_2018j, Sobnach_2011f, Tammana_2012j, Tanrikulu_2015e, Tay_2004, Thapa_2019f, Trgo_2012f, Wadhwa_2015e, Yasin_2009, teWildt_2010}, 28 cases (39\%) were female \cite{AlShaaibi_2021b, Ali_2020f, Ataya_2013, Beecroft_1998, Bhasin_2014, Bhattacharjee_2008, Cauchi_2002, Chang_2017f, Cox_2007, DelgadoSalazar_2020c, DivsalarP._2023a, Goldman_1998f, Hardy_2023g, Kar_2015, Kariholu_2008, Kerestes_2019, Li_2013, Naji_2012f, Ohno_2005, Peixoto_2017f, Sakellaridis_2008f, Sultan_2024f, Tupesis_2004f, Wildhaber_2005, Wnęk_2015f, Yildiz_2016e}, 1 case (1\%) had no gender recorded \cite{fjbuilsRepeatedBehaviorDeliberate2024}. \paragraph*{Age Group} 25 cases (35\%) were between 26 and 40 years of age \cite{Alao_2006i, Ali_2022g, Apikotoa_2022f, Ataya_2013, Benoist_2019e, Bhasin_2014, Chang_2017f, Cox_2007, DelgadoSalazar_2020c, Farhadi_2024h, Fry_2010, Gardner_2017h, Guinan_2019f, Jin_2023, Kumar_2019f, Losanoff_1996, Misra_2013, Qureshi_2016, Riva_2018j, Sakellaridis_2008f, Tammana_2012j, Trgo_2012f, Wnęk_2015f, Yildiz_2016e, fjbuilsRepeatedBehaviorDeliberate2024}, 18 cases (25\%) were between 18 and 25 years of age \cite{Akay_2015f, Ali_2017, Atayan_2016, Bhattacharjee_2008, Csaky_1998e, Kar_2015, Kariholu_2008, Kobiela_2015, Losanoff_1996, Losanoff_1997e, Mesfin_2022a, Peixoto_2017f, Sobnach_2011f, Tupesis_2004f, Yasin_2009}, 13 cases (18\%) were under 18 years of age \cite{AlShaaibi_2021b, Ali_2020f, Cauchi_2002, DivsalarP._2023a, Goldman_1998f, Liu_2005, Naji_2012f, Ohno_2005, Tanrikulu_2015e, Tay_2004, Wildhaber_2005}, 11 cases (15\%) were between 41 and 60 years of age \cite{Al-Faham_2020k, Bhumi_2024f, CamachoDorado_2018, Emamhadi_2018, Hardy_2023g, Jehangir_2019h, Kumar_2001, Sultan_2024f, Thapa_2019f, Wadhwa_2015e, teWildt_2010}, 3 cases (4\%) were over 60 years of age \cite{Beecroft_1998, Kerestes_2019, Li_2013}, 2 cases (3\%) had no age documented \cite{Berry_2021e}. \paragraph*{Population} 36 cases (50\%) had a psychiatric history \cite{AlShaaibi_2021b, Alao_2006i, Ali_2020f, Apikotoa_2022f, Ataya_2013, Atayan_2016, Beecroft_1998, CamachoDorado_2018, Chang_2017f, DelgadoSalazar_2020c, DivsalarP._2023a, Farhadi_2024h, Fry_2010, Guinan_2019f, Hardy_2023g, Jehangir_2019h, Jin_2023, Kar_2015, Kerestes_2019, Kobiela_2015, Kumar_2001, Kumar_2019f, Liu_2005, Mesfin_2022a, Misra_2013, Ohno_2005, Peixoto_2017f, Sakellaridis_2008f, Sultan_2024f, Tammana_2012j, Tanrikulu_2015e, Yildiz_2016e, fjbuilsRepeatedBehaviorDeliberate2024, teWildt_2010}, 19 cases (26\%) had ingested previously \cite{Alao_2006i, Apikotoa_2022f, Berry_2021e, Bhattacharjee_2008, Csaky_1998e, DivsalarP._2023a, Emamhadi_2018, Guinan_2019f, Jehangir_2019h, Jin_2023, Liu_2005, Sakellaridis_2008f, Tanrikulu_2015e, Thapa_2019f, Yildiz_2016e, fjbuilsRepeatedBehaviorDeliberate2024, teWildt_2010}, 12 cases (17\%) were detained persons \cite{Alao_2006i, Ali_2022g, Apikotoa_2022f, Losanoff_1996, Losanoff_1997e, Qureshi_2016, Tammana_2012j, Trgo_2012f}, 7 cases (10\%) were severely disabled \cite{Atayan_2016, Kerestes_2019, Liu_2005, Ohno_2005, Peixoto_2017f, Yildiz_2016e, teWildt_2010}, 4 cases (6\%) were psychiatric inpatients \cite{DivsalarP._2023a, fjbuilsRepeatedBehaviorDeliberate2024, teWildt_2010}, 3 cases (4\%) were under the influence of alcohol \cite{Benoist_2019e, Csaky_1998e, Thapa_2019f}, 2 cases (3\%) were displaced people \cite{Akay_2015f, Gardner_2017h}. \paragraph*{Motivation} 34 cases (47\%) had a psychiatric motivation \cite{Al-Faham_2020k, Alao_2006i, Ali_2020f, Apikotoa_2022f, Ataya_2013, Atayan_2016, Bhasin_2014, Bhattacharjee_2008, DelgadoSalazar_2020c, DivsalarP._2023a, Emamhadi_2018, Farhadi_2024h, Guinan_2019f, Hardy_2023g, Jehangir_2019h, Jin_2023, Kar_2015, Kariholu_2008, Kerestes_2019, Kobiela_2015, Kumar_2001, Kumar_2019f, Li_2013, Liu_2005, Misra_2013, Ohno_2005, Sakellaridis_2008f, Sultan_2024f, Tammana_2012j, Tanrikulu_2015e, Yasin_2009, teWildt_2010}, 21 cases (29\%) were motivated by self-harm intention \cite{Al-Faham_2020k, AlShaaibi_2021b, Alao_2006i, Ali_2017, CamachoDorado_2018, Chang_2017f, Cox_2007, Csaky_1998e, Fry_2010, Li_2013, Losanoff_1996, Losanoff_1997e, Mesfin_2022a, Sakellaridis_2008f, Tammana_2012j, Tanrikulu_2015e, fjbuilsRepeatedBehaviorDeliberate2024}, 17 cases (24\%) had a psychosocial motivation \cite{Akay_2015f, Benoist_2019e, Bhattacharjee_2008, Cauchi_2002, Goldman_1998f, Hardy_2023g, Kobiela_2015, Li_2013, Naji_2012f, Qureshi_2016, Riva_2018j, Sobnach_2011f, Tay_2004, Thapa_2019f, Tupesis_2004f, Wildhaber_2005, Wnęk_2015f}, 9 cases (12\%) were motivated by protest \cite{Bhumi_2024f, Gardner_2017h, Losanoff_1996, Losanoff_1997e, Tupesis_2004f}, 9 cases (12\%) had another documented motivation \cite{Ali_2020f, Ali_2022g, Emamhadi_2018, Guinan_2019f, Peixoto_2017f, Sakellaridis_2008f, Trgo_2012f, Wadhwa_2015e, Yildiz_2016e}. \paragraph*{Object Characteristics} 51 cases (71\%) ingested a large diameter object (\textgreater{}2.5cm) \cite{Akay_2015f, Al-Faham_2020k, AlShaaibi_2021b, Alao_2006i, Ali_2017, Ali_2022g, Apikotoa_2022f, Atayan_2016, Berry_2021e, Bhasin_2014, CamachoDorado_2018, Cauchi_2002, Chang_2017f, Cox_2007, Csaky_1998e, DivsalarP._2023a, Emamhadi_2018, Gardner_2017h, Guinan_2019f, Jehangir_2019h, Jin_2023, Kariholu_2008, Kerestes_2019, Kobiela_2015, Kumar_2001, Kumar_2019f, Losanoff_1996, Losanoff_1997e, Mesfin_2022a, Misra_2013, Naji_2012f, Ohno_2005, Peixoto_2017f, Qureshi_2016, Riva_2018j, Sakellaridis_2008f, Sultan_2024f, Tanrikulu_2015e, Thapa_2019f, Trgo_2012f, Wnęk_2015f, Yildiz_2016e, fjbuilsRepeatedBehaviorDeliberate2024, teWildt_2010}, 44 cases (61\%) ingested multiple objects \cite{Ali_2020f, Apikotoa_2022f, Ataya_2013, Atayan_2016, Beecroft_1998, Bhattacharjee_2008, Bhumi_2024f, CamachoDorado_2018, Cauchi_2002, Emamhadi_2018, Farhadi_2024h, Fry_2010, Goldman_1998f, Guinan_2019f, Hardy_2023g, Jehangir_2019h, Jin_2023, Kar_2015, Kariholu_2008, Kobiela_2015, Kumar_2001, Kumar_2019f, Li_2013, Liu_2005, Losanoff_1996, Mesfin_2022a, Misra_2013, Naji_2012f, Ohno_2005, Sobnach_2011f, Sultan_2024f, Tammana_2012j, Tanrikulu_2015e, Tay_2004, Thapa_2019f, Wadhwa_2015e, Wildhaber_2005, Yasin_2009, fjbuilsRepeatedBehaviorDeliberate2024, teWildt_2010}, 34 cases (47\%) ingested a sharp object \cite{AlShaaibi_2021b, Alao_2006i, Apikotoa_2022f, Ataya_2013, Benoist_2019e, Bhasin_2014, Bhattacharjee_2008, CamachoDorado_2018, Csaky_1998e, DelgadoSalazar_2020c, DivsalarP._2023a, Emamhadi_2018, Farhadi_2024h, Fry_2010, Guinan_2019f, Hardy_2023g, Jehangir_2019h, Jin_2023, Kariholu_2008, Kobiela_2015, Kumar_2019f, Losanoff_1996, Losanoff_1997e, Mesfin_2022a, Misra_2013, Sobnach_2011f, Yasin_2009, teWildt_2010}, 32 cases (44\%) ingested a long object (\textgreater{}5cm) \cite{Al-Faham_2020k, AlShaaibi_2021b, Ali_2017, Ali_2022g, Atayan_2016, Bhasin_2014, CamachoDorado_2018, Chang_2017f, Cox_2007, Csaky_1998e, DivsalarP._2023a, Emamhadi_2018, Fry_2010, Gardner_2017h, Jin_2023, Kariholu_2008, Kerestes_2019, Kobiela_2015, Kumar_2019f, Mesfin_2022a, Misra_2013, Ohno_2005, Qureshi_2016, Sakellaridis_2008f, Sultan_2024f, Thapa_2019f, Trgo_2012f, Yasin_2009, Yildiz_2016e, teWildt_2010}, 9 cases (12\%) ingested a magnet \cite{Ali_2020f, Bhumi_2024f, Cauchi_2002, Liu_2005, Naji_2012f, Ohno_2005, Tanrikulu_2015e, Tay_2004, Wildhaber_2005}, 2 cases (3\%) ingested a button battery \cite{Berry_2021e, Bhumi_2024f}. \paragraph*{Outcomes} 48 cases (67\%) experienced a complication \cite{Ali_2017, Ali_2020f, Apikotoa_2022f, Atayan_2016, Beecroft_1998, Benoist_2019e, Berry_2021e, Bhasin_2014, Bhumi_2024f, CamachoDorado_2018, Cauchi_2002, Cox_2007, Csaky_1998e, DelgadoSalazar_2020c, DivsalarP._2023a, Emamhadi_2018, Farhadi_2024h, Fry_2010, Gardner_2017h, Goldman_1998f, Jin_2023, Kariholu_2008, Kerestes_2019, Kobiela_2015, Kumar_2001, Kumar_2019f, Liu_2005, Losanoff_1996, Mesfin_2022a, Misra_2013, Naji_2012f, Ohno_2005, Sakellaridis_2008f, Sobnach_2011f, Sultan_2024f, Tanrikulu_2015e, Tay_2004, Thapa_2019f, Trgo_2012f, Tupesis_2004f, Wildhaber_2005, Wnęk_2015f, Yasin_2009, Yildiz_2016e}, 44 cases (61\%) underwent surgery \cite{Al-Faham_2020k, AlShaaibi_2021b, Alao_2006i, Ali_2017, Ali_2020f, Atayan_2016, Beecroft_1998, Bhasin_2014, CamachoDorado_2018, Cauchi_2002, Chang_2017f, Cox_2007, Csaky_1998e, DelgadoSalazar_2020c, DivsalarP._2023a, Farhadi_2024h, Fry_2010, Gardner_2017h, Jin_2023, Kariholu_2008, Kerestes_2019, Kobiela_2015, Kumar_2019f, Liu_2005, Losanoff_1996, Losanoff_1997e, Mesfin_2022a, Misra_2013, Naji_2012f, Sobnach_2011f, Tanrikulu_2015e, Tay_2004, Thapa_2019f, Tupesis_2004f, Wildhaber_2005, Wnęk_2015f, Yasin_2009, Yildiz_2016e, fjbuilsRepeatedBehaviorDeliberate2024}, 31 cases (43\%) underwent endoscopy \cite{Akay_2015f, Ali_2022g, Apikotoa_2022f, Atayan_2016, Benoist_2019e, Berry_2021e, Bhasin_2014, Bhumi_2024f, CamachoDorado_2018, Chang_2017f, DelgadoSalazar_2020c, Gardner_2017h, Guinan_2019f, Hardy_2023g, Jehangir_2019h, Kariholu_2008, Li_2013, Liu_2005, Ohno_2005, Peixoto_2017f, Qureshi_2016, Riva_2018j, Sakellaridis_2008f, Sultan_2024f, Tammana_2012j, Tanrikulu_2015e, Trgo_2012f, Wadhwa_2015e, Wnęk_2015f, teWildt_2010}, 7 cases (10\%) were managed conservatively \cite{Ataya_2013, Bhattacharjee_2008, DivsalarP._2023a, Emamhadi_2018, Goldman_1998f, Kar_2015, Kumar_2001}, 2 cases (3\%) died \cite{Emamhadi_2018, Kumar_2001}. All 90 were male gender. 90 cases (100\%) were detained at the time of ingestion \cite{Elghali_2016, Karp_1991b, Lee_2007}, 88 cases (98\%) were intentional ingestions \cite{Elghali_2016, Karp_1991b, Lee_2007}, 30 cases (33\%) had a psychiatric history documented \cite{Elghali_2016, Karp_1991b, Lee_2007}, 2 cases (2\%) had a history of prior ingestion \cite{Elghali_2016}. No cases were reported for were psychiatric inpatients, were displaced people, were under the influence of alcohol at the time of ingestion, and had a severe disability history.
\paragraph*{Motivation}  70 cases (78\%) reported protest motivation \cite{Elghali_2016, Karp_1991b, Lee_2007}, 12 cases (13\%) reported psychiatric motivation \cite{Karp_1991b}, 6 cases (7\%) reported self-harm motivation \cite{Elghali_2016, Karp_1991b}. No cases were reported for psychosocial motivation and other motivation.
\paragraph*{Object Characteristics}  68 cases (76\%) involved sharp object ingestion \cite{Elghali_2016, Karp_1991b, Lee_2007}, 32 cases (36\%) involved long (\textgreater 5cm) object ingestion \cite{Lee_2007}, 25 cases (28\%) involved ingestion of multiple objects \cite{Elghali_2016, Lee_2007}. No cases were reported for button battery ingestion, magnet ingestion, and involved large diameter (\textgreater 2.5cm) object ingestion.
\paragraph*{Outcomes}  47 cases (52\%) underwent endoscopic intervention \cite{Elghali_2016, Lee_2007}, 29 cases (32\%) were managed conservatively \cite{Elghali_2016, Karp_1991b}, 15 cases (17\%) underwent surgical intervention \cite{Elghali_2016, Karp_1991b, Lee_2007}, 6 cases (7\%) reported complications \cite{Lee_2007}, 1 case (1\%) died \cite{Elghali_2016}.
\paragraph*{Geographical Location}Cases were recorded in 33 countries: 13 cases from USA \cite{Alao_2006i, Ataya_2013, Bhumi_2024f, Fry_2010, Guinan_2019f, Hardy_2023g, Jehangir_2019h, Kerestes_2019, Kumar_2001, Liu_2005, Tammana_2012j, Tay_2004, Tupesis_2004f}; 7 cases from India \cite{Bhasin_2014, Bhattacharjee_2008, Kar_2015, Kariholu_2008, Kumar_2019f, Misra_2013, Wadhwa_2015e} and UK \cite{Beecroft_1998, Berry_2021e, Cauchi_2002, Cox_2007, Gardner_2017h, Qureshi_2016}; 6 cases from Bulgaria \cite{Losanoff_1996, Losanoff_1997e}; 5 cases from Iran \cite{DivsalarP._2023a, Emamhadi_2018, Farhadi_2024h}; 4 cases from Turkey \cite{Akay_2015f, Atayan_2016, Tanrikulu_2015e, Yildiz_2016e}; 2 cases from China \cite{Jin_2023, Li_2013}, Poland \cite{Kobiela_2015, Wnęk_2015f}, and Spain \cite{CamachoDorado_2018, fjbuilsRepeatedBehaviorDeliberate2024}; 1 case from Australia \cite{Apikotoa_2022f}, Bahrain \cite{Ali_2020f}, Croatia \cite{Trgo_2012f}, Ecuador \cite{DelgadoSalazar_2020c}, Egypt \cite{Ali_2022g}, Ethiopia \cite{Mesfin_2022a}, Germany \cite{teWildt_2010}, Greece \cite{Sakellaridis_2008f}, Hungary \cite{Csaky_1998e}, Iraq \cite{Al-Faham_2020k}, Israel \cite{Goldman_1998f}, Italy \cite{Riva_2018j}, Japan \cite{Ohno_2005}, Nepal \cite{Thapa_2019f}, Netherlands \cite{Benoist_2019e}, Oman \cite{AlShaaibi_2021b}, Pakistan \cite{Yasin_2009}, Portugal \cite{Peixoto_2017f}, Qatar \cite{Ali_2017}, Saudi Arabia \cite{Sultan_2024f}, South Africa \cite{Sobnach_2011f}, Sweden \cite{Naji_2012f}, Switzerland \cite{Wildhaber_2005}, and Taiwan \cite{Chang_2017f}. \paragraph*{Gender} 43 cases (60\%) were male \cite{Akay_2015f, Al-Faham_2020k, Alao_2006i, Ali_2017, Ali_2022g, Apikotoa_2022f, Atayan_2016, Benoist_2019e, Berry_2021e, Bhumi_2024f, CamachoDorado_2018, Csaky_1998e, Emamhadi_2018, Farhadi_2024h, Fry_2010, Gardner_2017h, Guinan_2019f, Jehangir_2019h, Jin_2023, Kobiela_2015, Kumar_2001, Kumar_2019f, Liu_2005, Losanoff_1996, Losanoff_1997e, Mesfin_2022a, Misra_2013, Qureshi_2016, Riva_2018j, Sobnach_2011f, Tammana_2012j, Tanrikulu_2015e, Tay_2004, Thapa_2019f, Trgo_2012f, Wadhwa_2015e, Yasin_2009, teWildt_2010}, 28 cases (39\%) were female \cite{AlShaaibi_2021b, Ali_2020f, Ataya_2013, Beecroft_1998, Bhasin_2014, Bhattacharjee_2008, Cauchi_2002, Chang_2017f, Cox_2007, DelgadoSalazar_2020c, DivsalarP._2023a, Goldman_1998f, Hardy_2023g, Kar_2015, Kariholu_2008, Kerestes_2019, Li_2013, Naji_2012f, Ohno_2005, Peixoto_2017f, Sakellaridis_2008f, Sultan_2024f, Tupesis_2004f, Wildhaber_2005, Wnęk_2015f, Yildiz_2016e}, 1 case (1\%) had no gender recorded \cite{fjbuilsRepeatedBehaviorDeliberate2024}. \paragraph*{Age Group} 25 cases (35\%) were between 26 and 40 years of age \cite{Alao_2006i, Ali_2022g, Apikotoa_2022f, Ataya_2013, Benoist_2019e, Bhasin_2014, Chang_2017f, Cox_2007, DelgadoSalazar_2020c, Farhadi_2024h, Fry_2010, Gardner_2017h, Guinan_2019f, Jin_2023, Kumar_2019f, Losanoff_1996, Misra_2013, Qureshi_2016, Riva_2018j, Sakellaridis_2008f, Tammana_2012j, Trgo_2012f, Wnęk_2015f, Yildiz_2016e, fjbuilsRepeatedBehaviorDeliberate2024}, 18 cases (25\%) were between 18 and 25 years of age \cite{Akay_2015f, Ali_2017, Atayan_2016, Bhattacharjee_2008, Csaky_1998e, Kar_2015, Kariholu_2008, Kobiela_2015, Losanoff_1996, Losanoff_1997e, Mesfin_2022a, Peixoto_2017f, Sobnach_2011f, Tupesis_2004f, Yasin_2009}, 13 cases (18\%) were under 18 years of age \cite{AlShaaibi_2021b, Ali_2020f, Cauchi_2002, DivsalarP._2023a, Goldman_1998f, Liu_2005, Naji_2012f, Ohno_2005, Tanrikulu_2015e, Tay_2004, Wildhaber_2005}, 11 cases (15\%) were between 41 and 60 years of age \cite{Al-Faham_2020k, Bhumi_2024f, CamachoDorado_2018, Emamhadi_2018, Hardy_2023g, Jehangir_2019h, Kumar_2001, Sultan_2024f, Thapa_2019f, Wadhwa_2015e, teWildt_2010}, 3 cases (4\%) were over 60 years of age \cite{Beecroft_1998, Kerestes_2019, Li_2013}, 2 cases (3\%) had no age documented \cite{Berry_2021e}. \paragraph*{Population} 36 cases (50\%) had a psychiatric history \cite{AlShaaibi_2021b, Alao_2006i, Ali_2020f, Apikotoa_2022f, Ataya_2013, Atayan_2016, Beecroft_1998, CamachoDorado_2018, Chang_2017f, DelgadoSalazar_2020c, DivsalarP._2023a, Farhadi_2024h, Fry_2010, Guinan_2019f, Hardy_2023g, Jehangir_2019h, Jin_2023, Kar_2015, Kerestes_2019, Kobiela_2015, Kumar_2001, Kumar_2019f, Liu_2005, Mesfin_2022a, Misra_2013, Ohno_2005, Peixoto_2017f, Sakellaridis_2008f, Sultan_2024f, Tammana_2012j, Tanrikulu_2015e, Yildiz_2016e, fjbuilsRepeatedBehaviorDeliberate2024, teWildt_2010}, 19 cases (26\%) had ingested previously \cite{Alao_2006i, Apikotoa_2022f, Berry_2021e, Bhattacharjee_2008, Csaky_1998e, DivsalarP._2023a, Emamhadi_2018, Guinan_2019f, Jehangir_2019h, Jin_2023, Liu_2005, Sakellaridis_2008f, Tanrikulu_2015e, Thapa_2019f, Yildiz_2016e, fjbuilsRepeatedBehaviorDeliberate2024, teWildt_2010}, 12 cases (17\%) were detained persons \cite{Alao_2006i, Ali_2022g, Apikotoa_2022f, Losanoff_1996, Losanoff_1997e, Qureshi_2016, Tammana_2012j, Trgo_2012f}, 7 cases (10\%) were severely disabled \cite{Atayan_2016, Kerestes_2019, Liu_2005, Ohno_2005, Peixoto_2017f, Yildiz_2016e, teWildt_2010}, 4 cases (6\%) were psychiatric inpatients \cite{DivsalarP._2023a, fjbuilsRepeatedBehaviorDeliberate2024, teWildt_2010}, 3 cases (4\%) were under the influence of alcohol \cite{Benoist_2019e, Csaky_1998e, Thapa_2019f}, 2 cases (3\%) were displaced people \cite{Akay_2015f, Gardner_2017h}. \paragraph*{Motivation} 34 cases (47\%) had a psychiatric motivation \cite{Al-Faham_2020k, Alao_2006i, Ali_2020f, Apikotoa_2022f, Ataya_2013, Atayan_2016, Bhasin_2014, Bhattacharjee_2008, DelgadoSalazar_2020c, DivsalarP._2023a, Emamhadi_2018, Farhadi_2024h, Guinan_2019f, Hardy_2023g, Jehangir_2019h, Jin_2023, Kar_2015, Kariholu_2008, Kerestes_2019, Kobiela_2015, Kumar_2001, Kumar_2019f, Li_2013, Liu_2005, Misra_2013, Ohno_2005, Sakellaridis_2008f, Sultan_2024f, Tammana_2012j, Tanrikulu_2015e, Yasin_2009, teWildt_2010}, 21 cases (29\%) were motivated by self-harm intention \cite{Al-Faham_2020k, AlShaaibi_2021b, Alao_2006i, Ali_2017, CamachoDorado_2018, Chang_2017f, Cox_2007, Csaky_1998e, Fry_2010, Li_2013, Losanoff_1996, Losanoff_1997e, Mesfin_2022a, Sakellaridis_2008f, Tammana_2012j, Tanrikulu_2015e, fjbuilsRepeatedBehaviorDeliberate2024}, 17 cases (24\%) had a psychosocial motivation \cite{Akay_2015f, Benoist_2019e, Bhattacharjee_2008, Cauchi_2002, Goldman_1998f, Hardy_2023g, Kobiela_2015, Li_2013, Naji_2012f, Qureshi_2016, Riva_2018j, Sobnach_2011f, Tay_2004, Thapa_2019f, Tupesis_2004f, Wildhaber_2005, Wnęk_2015f}, 9 cases (12\%) were motivated by protest \cite{Bhumi_2024f, Gardner_2017h, Losanoff_1996, Losanoff_1997e, Tupesis_2004f}, 9 cases (12\%) had another documented motivation \cite{Ali_2020f, Ali_2022g, Emamhadi_2018, Guinan_2019f, Peixoto_2017f, Sakellaridis_2008f, Trgo_2012f, Wadhwa_2015e, Yildiz_2016e}. \paragraph*{Object Characteristics} 51 cases (71\%) ingested a large diameter object (\textgreater{}2.5cm) \cite{Akay_2015f, Al-Faham_2020k, AlShaaibi_2021b, Alao_2006i, Ali_2017, Ali_2022g, Apikotoa_2022f, Atayan_2016, Berry_2021e, Bhasin_2014, CamachoDorado_2018, Cauchi_2002, Chang_2017f, Cox_2007, Csaky_1998e, DivsalarP._2023a, Emamhadi_2018, Gardner_2017h, Guinan_2019f, Jehangir_2019h, Jin_2023, Kariholu_2008, Kerestes_2019, Kobiela_2015, Kumar_2001, Kumar_2019f, Losanoff_1996, Losanoff_1997e, Mesfin_2022a, Misra_2013, Naji_2012f, Ohno_2005, Peixoto_2017f, Qureshi_2016, Riva_2018j, Sakellaridis_2008f, Sultan_2024f, Tanrikulu_2015e, Thapa_2019f, Trgo_2012f, Wnęk_2015f, Yildiz_2016e, fjbuilsRepeatedBehaviorDeliberate2024, teWildt_2010}, 44 cases (61\%) ingested multiple objects \cite{Ali_2020f, Apikotoa_2022f, Ataya_2013, Atayan_2016, Beecroft_1998, Bhattacharjee_2008, Bhumi_2024f, CamachoDorado_2018, Cauchi_2002, Emamhadi_2018, Farhadi_2024h, Fry_2010, Goldman_1998f, Guinan_2019f, Hardy_2023g, Jehangir_2019h, Jin_2023, Kar_2015, Kariholu_2008, Kobiela_2015, Kumar_2001, Kumar_2019f, Li_2013, Liu_2005, Losanoff_1996, Mesfin_2022a, Misra_2013, Naji_2012f, Ohno_2005, Sobnach_2011f, Sultan_2024f, Tammana_2012j, Tanrikulu_2015e, Tay_2004, Thapa_2019f, Wadhwa_2015e, Wildhaber_2005, Yasin_2009, fjbuilsRepeatedBehaviorDeliberate2024, teWildt_2010}, 34 cases (47\%) ingested a sharp object \cite{AlShaaibi_2021b, Alao_2006i, Apikotoa_2022f, Ataya_2013, Benoist_2019e, Bhasin_2014, Bhattacharjee_2008, CamachoDorado_2018, Csaky_1998e, DelgadoSalazar_2020c, DivsalarP._2023a, Emamhadi_2018, Farhadi_2024h, Fry_2010, Guinan_2019f, Hardy_2023g, Jehangir_2019h, Jin_2023, Kariholu_2008, Kobiela_2015, Kumar_2019f, Losanoff_1996, Losanoff_1997e, Mesfin_2022a, Misra_2013, Sobnach_2011f, Yasin_2009, teWildt_2010}, 32 cases (44\%) ingested a long object (\textgreater{}5cm) \cite{Al-Faham_2020k, AlShaaibi_2021b, Ali_2017, Ali_2022g, Atayan_2016, Bhasin_2014, CamachoDorado_2018, Chang_2017f, Cox_2007, Csaky_1998e, DivsalarP._2023a, Emamhadi_2018, Fry_2010, Gardner_2017h, Jin_2023, Kariholu_2008, Kerestes_2019, Kobiela_2015, Kumar_2019f, Mesfin_2022a, Misra_2013, Ohno_2005, Qureshi_2016, Sakellaridis_2008f, Sultan_2024f, Thapa_2019f, Trgo_2012f, Yasin_2009, Yildiz_2016e, teWildt_2010}, 9 cases (12\%) ingested a magnet \cite{Ali_2020f, Bhumi_2024f, Cauchi_2002, Liu_2005, Naji_2012f, Ohno_2005, Tanrikulu_2015e, Tay_2004, Wildhaber_2005}, 2 cases (3\%) ingested a button battery \cite{Berry_2021e, Bhumi_2024f}. \paragraph*{Outcomes} 48 cases (67\%) experienced a complication \cite{Ali_2017, Ali_2020f, Apikotoa_2022f, Atayan_2016, Beecroft_1998, Benoist_2019e, Berry_2021e, Bhasin_2014, Bhumi_2024f, CamachoDorado_2018, Cauchi_2002, Cox_2007, Csaky_1998e, DelgadoSalazar_2020c, DivsalarP._2023a, Emamhadi_2018, Farhadi_2024h, Fry_2010, Gardner_2017h, Goldman_1998f, Jin_2023, Kariholu_2008, Kerestes_2019, Kobiela_2015, Kumar_2001, Kumar_2019f, Liu_2005, Losanoff_1996, Mesfin_2022a, Misra_2013, Naji_2012f, Ohno_2005, Sakellaridis_2008f, Sobnach_2011f, Sultan_2024f, Tanrikulu_2015e, Tay_2004, Thapa_2019f, Trgo_2012f, Tupesis_2004f, Wildhaber_2005, Wnęk_2015f, Yasin_2009, Yildiz_2016e}, 44 cases (61\%) underwent surgery \cite{Al-Faham_2020k, AlShaaibi_2021b, Alao_2006i, Ali_2017, Ali_2020f, Atayan_2016, Beecroft_1998, Bhasin_2014, CamachoDorado_2018, Cauchi_2002, Chang_2017f, Cox_2007, Csaky_1998e, DelgadoSalazar_2020c, DivsalarP._2023a, Farhadi_2024h, Fry_2010, Gardner_2017h, Jin_2023, Kariholu_2008, Kerestes_2019, Kobiela_2015, Kumar_2019f, Liu_2005, Losanoff_1996, Losanoff_1997e, Mesfin_2022a, Misra_2013, Naji_2012f, Sobnach_2011f, Tanrikulu_2015e, Tay_2004, Thapa_2019f, Tupesis_2004f, Wildhaber_2005, Wnęk_2015f, Yasin_2009, Yildiz_2016e, fjbuilsRepeatedBehaviorDeliberate2024}, 31 cases (43\%) underwent endoscopy \cite{Akay_2015f, Ali_2022g, Apikotoa_2022f, Atayan_2016, Benoist_2019e, Berry_2021e, Bhasin_2014, Bhumi_2024f, CamachoDorado_2018, Chang_2017f, DelgadoSalazar_2020c, Gardner_2017h, Guinan_2019f, Hardy_2023g, Jehangir_2019h, Kariholu_2008, Li_2013, Liu_2005, Ohno_2005, Peixoto_2017f, Qureshi_2016, Riva_2018j, Sakellaridis_2008f, Sultan_2024f, Tammana_2012j, Tanrikulu_2015e, Trgo_2012f, Wadhwa_2015e, Wnęk_2015f, teWildt_2010}, 7 cases (10\%) were managed conservatively \cite{Ataya_2013, Bhattacharjee_2008, DivsalarP._2023a, Emamhadi_2018, Goldman_1998f, Kar_2015, Kumar_2001}, 2 cases (3\%) died \cite{Emamhadi_2018, Kumar_2001}. All 90 were male gender. 90 cases (100\%) were detained at the time of ingestion \cite{Elghali_2016, Karp_1991b, Lee_2007}, 88 cases (98\%) were intentional ingestions \cite{Elghali_2016, Karp_1991b, Lee_2007}, 30 cases (33\%) had a psychiatric history documented \cite{Elghali_2016, Karp_1991b, Lee_2007}, 2 cases (2\%) had a history of prior ingestion \cite{Elghali_2016}. No cases were reported for were psychiatric inpatients, were displaced people, were under the influence of alcohol at the time of ingestion, and had a severe disability history.
\paragraph*{Motivation}  70 cases (78\%) reported protest motivation \cite{Elghali_2016, Karp_1991b, Lee_2007}, 12 cases (13\%) reported psychiatric motivation \cite{Karp_1991b}, 6 cases (7\%) reported self-harm motivation \cite{Elghali_2016, Karp_1991b}. No cases were reported for psychosocial motivation and other motivation.
\paragraph*{Object Characteristics}  68 cases (76\%) involved sharp object ingestion \cite{Elghali_2016, Karp_1991b, Lee_2007}, 32 cases (36\%) involved long (\textgreater 5cm) object ingestion \cite{Lee_2007}, 25 cases (28\%) involved ingestion of multiple objects \cite{Elghali_2016, Lee_2007}. No cases were reported for button battery ingestion, magnet ingestion, and involved large diameter (\textgreater 2.5cm) object ingestion.
\paragraph*{Outcomes}  47 cases (52\%) underwent endoscopic intervention \cite{Elghali_2016, Lee_2007}, 29 cases (32\%) were managed conservatively \cite{Elghali_2016, Karp_1991b}, 15 cases (17\%) underwent surgical intervention \cite{Elghali_2016, Karp_1991b, Lee_2007}, 6 cases (7\%) reported complications \cite{Lee_2007}, 1 case (1\%) died \cite{Elghali_2016}.
\paragraph*{Geographical Location}Cases were recorded in 33 countries: 13 cases from USA \cite{Alao_2006i, Ataya_2013, Bhumi_2024f, Fry_2010, Guinan_2019f, Hardy_2023g, Jehangir_2019h, Kerestes_2019, Kumar_2001, Liu_2005, Tammana_2012j, Tay_2004, Tupesis_2004f}; 7 cases from India \cite{Bhasin_2014, Bhattacharjee_2008, Kar_2015, Kariholu_2008, Kumar_2019f, Misra_2013, Wadhwa_2015e} and UK \cite{Beecroft_1998, Berry_2021e, Cauchi_2002, Cox_2007, Gardner_2017h, Qureshi_2016}; 6 cases from Bulgaria \cite{Losanoff_1996, Losanoff_1997e}; 5 cases from Iran \cite{DivsalarP._2023a, Emamhadi_2018, Farhadi_2024h}; 4 cases from Turkey \cite{Akay_2015f, Atayan_2016, Tanrikulu_2015e, Yildiz_2016e}; 2 cases from China \cite{Jin_2023, Li_2013}, Poland \cite{Kobiela_2015, Wnęk_2015f}, and Spain \cite{CamachoDorado_2018, fjbuilsRepeatedBehaviorDeliberate2024}; 1 case from Australia \cite{Apikotoa_2022f}, Bahrain \cite{Ali_2020f}, Croatia \cite{Trgo_2012f}, Ecuador \cite{DelgadoSalazar_2020c}, Egypt \cite{Ali_2022g}, Ethiopia \cite{Mesfin_2022a}, Germany \cite{teWildt_2010}, Greece \cite{Sakellaridis_2008f}, Hungary \cite{Csaky_1998e}, Iraq \cite{Al-Faham_2020k}, Israel \cite{Goldman_1998f}, Italy \cite{Riva_2018j}, Japan \cite{Ohno_2005}, Nepal \cite{Thapa_2019f}, Netherlands \cite{Benoist_2019e}, Oman \cite{AlShaaibi_2021b}, Pakistan \cite{Yasin_2009}, Portugal \cite{Peixoto_2017f}, Qatar \cite{Ali_2017}, Saudi Arabia \cite{Sultan_2024f}, South Africa \cite{Sobnach_2011f}, Sweden \cite{Naji_2012f}, Switzerland \cite{Wildhaber_2005}, and Taiwan \cite{Chang_2017f}. \paragraph*{Gender} 43 cases (60\%) were male \cite{Akay_2015f, Al-Faham_2020k, Alao_2006i, Ali_2017, Ali_2022g, Apikotoa_2022f, Atayan_2016, Benoist_2019e, Berry_2021e, Bhumi_2024f, CamachoDorado_2018, Csaky_1998e, Emamhadi_2018, Farhadi_2024h, Fry_2010, Gardner_2017h, Guinan_2019f, Jehangir_2019h, Jin_2023, Kobiela_2015, Kumar_2001, Kumar_2019f, Liu_2005, Losanoff_1996, Losanoff_1997e, Mesfin_2022a, Misra_2013, Qureshi_2016, Riva_2018j, Sobnach_2011f, Tammana_2012j, Tanrikulu_2015e, Tay_2004, Thapa_2019f, Trgo_2012f, Wadhwa_2015e, Yasin_2009, teWildt_2010}, 28 cases (39\%) were female \cite{AlShaaibi_2021b, Ali_2020f, Ataya_2013, Beecroft_1998, Bhasin_2014, Bhattacharjee_2008, Cauchi_2002, Chang_2017f, Cox_2007, DelgadoSalazar_2020c, DivsalarP._2023a, Goldman_1998f, Hardy_2023g, Kar_2015, Kariholu_2008, Kerestes_2019, Li_2013, Naji_2012f, Ohno_2005, Peixoto_2017f, Sakellaridis_2008f, Sultan_2024f, Tupesis_2004f, Wildhaber_2005, Wnęk_2015f, Yildiz_2016e}, 1 case (1\%) had no gender recorded \cite{fjbuilsRepeatedBehaviorDeliberate2024}. \paragraph*{Age Group} 25 cases (35\%) were between 26 and 40 years of age \cite{Alao_2006i, Ali_2022g, Apikotoa_2022f, Ataya_2013, Benoist_2019e, Bhasin_2014, Chang_2017f, Cox_2007, DelgadoSalazar_2020c, Farhadi_2024h, Fry_2010, Gardner_2017h, Guinan_2019f, Jin_2023, Kumar_2019f, Losanoff_1996, Misra_2013, Qureshi_2016, Riva_2018j, Sakellaridis_2008f, Tammana_2012j, Trgo_2012f, Wnęk_2015f, Yildiz_2016e, fjbuilsRepeatedBehaviorDeliberate2024}, 18 cases (25\%) were between 18 and 25 years of age \cite{Akay_2015f, Ali_2017, Atayan_2016, Bhattacharjee_2008, Csaky_1998e, Kar_2015, Kariholu_2008, Kobiela_2015, Losanoff_1996, Losanoff_1997e, Mesfin_2022a, Peixoto_2017f, Sobnach_2011f, Tupesis_2004f, Yasin_2009}, 13 cases (18\%) were under 18 years of age \cite{AlShaaibi_2021b, Ali_2020f, Cauchi_2002, DivsalarP._2023a, Goldman_1998f, Liu_2005, Naji_2012f, Ohno_2005, Tanrikulu_2015e, Tay_2004, Wildhaber_2005}, 11 cases (15\%) were between 41 and 60 years of age \cite{Al-Faham_2020k, Bhumi_2024f, CamachoDorado_2018, Emamhadi_2018, Hardy_2023g, Jehangir_2019h, Kumar_2001, Sultan_2024f, Thapa_2019f, Wadhwa_2015e, teWildt_2010}, 3 cases (4\%) were over 60 years of age \cite{Beecroft_1998, Kerestes_2019, Li_2013}, 2 cases (3\%) had no age documented \cite{Berry_2021e}. \paragraph*{Population} 36 cases (50\%) had a psychiatric history \cite{AlShaaibi_2021b, Alao_2006i, Ali_2020f, Apikotoa_2022f, Ataya_2013, Atayan_2016, Beecroft_1998, CamachoDorado_2018, Chang_2017f, DelgadoSalazar_2020c, DivsalarP._2023a, Farhadi_2024h, Fry_2010, Guinan_2019f, Hardy_2023g, Jehangir_2019h, Jin_2023, Kar_2015, Kerestes_2019, Kobiela_2015, Kumar_2001, Kumar_2019f, Liu_2005, Mesfin_2022a, Misra_2013, Ohno_2005, Peixoto_2017f, Sakellaridis_2008f, Sultan_2024f, Tammana_2012j, Tanrikulu_2015e, Yildiz_2016e, fjbuilsRepeatedBehaviorDeliberate2024, teWildt_2010}, 19 cases (26\%) had ingested previously \cite{Alao_2006i, Apikotoa_2022f, Berry_2021e, Bhattacharjee_2008, Csaky_1998e, DivsalarP._2023a, Emamhadi_2018, Guinan_2019f, Jehangir_2019h, Jin_2023, Liu_2005, Sakellaridis_2008f, Tanrikulu_2015e, Thapa_2019f, Yildiz_2016e, fjbuilsRepeatedBehaviorDeliberate2024, teWildt_2010}, 12 cases (17\%) were detained persons \cite{Alao_2006i, Ali_2022g, Apikotoa_2022f, Losanoff_1996, Losanoff_1997e, Qureshi_2016, Tammana_2012j, Trgo_2012f}, 7 cases (10\%) were severely disabled \cite{Atayan_2016, Kerestes_2019, Liu_2005, Ohno_2005, Peixoto_2017f, Yildiz_2016e, teWildt_2010}, 4 cases (6\%) were psychiatric inpatients \cite{DivsalarP._2023a, fjbuilsRepeatedBehaviorDeliberate2024, teWildt_2010}, 3 cases (4\%) were under the influence of alcohol \cite{Benoist_2019e, Csaky_1998e, Thapa_2019f}, 2 cases (3\%) were displaced people \cite{Akay_2015f, Gardner_2017h}. \paragraph*{Motivation} 34 cases (47\%) had a psychiatric motivation \cite{Al-Faham_2020k, Alao_2006i, Ali_2020f, Apikotoa_2022f, Ataya_2013, Atayan_2016, Bhasin_2014, Bhattacharjee_2008, DelgadoSalazar_2020c, DivsalarP._2023a, Emamhadi_2018, Farhadi_2024h, Guinan_2019f, Hardy_2023g, Jehangir_2019h, Jin_2023, Kar_2015, Kariholu_2008, Kerestes_2019, Kobiela_2015, Kumar_2001, Kumar_2019f, Li_2013, Liu_2005, Misra_2013, Ohno_2005, Sakellaridis_2008f, Sultan_2024f, Tammana_2012j, Tanrikulu_2015e, Yasin_2009, teWildt_2010}, 21 cases (29\%) were motivated by self-harm intention \cite{Al-Faham_2020k, AlShaaibi_2021b, Alao_2006i, Ali_2017, CamachoDorado_2018, Chang_2017f, Cox_2007, Csaky_1998e, Fry_2010, Li_2013, Losanoff_1996, Losanoff_1997e, Mesfin_2022a, Sakellaridis_2008f, Tammana_2012j, Tanrikulu_2015e, fjbuilsRepeatedBehaviorDeliberate2024}, 17 cases (24\%) had a psychosocial motivation \cite{Akay_2015f, Benoist_2019e, Bhattacharjee_2008, Cauchi_2002, Goldman_1998f, Hardy_2023g, Kobiela_2015, Li_2013, Naji_2012f, Qureshi_2016, Riva_2018j, Sobnach_2011f, Tay_2004, Thapa_2019f, Tupesis_2004f, Wildhaber_2005, Wnęk_2015f}, 9 cases (12\%) were motivated by protest \cite{Bhumi_2024f, Gardner_2017h, Losanoff_1996, Losanoff_1997e, Tupesis_2004f}, 9 cases (12\%) had another documented motivation \cite{Ali_2020f, Ali_2022g, Emamhadi_2018, Guinan_2019f, Peixoto_2017f, Sakellaridis_2008f, Trgo_2012f, Wadhwa_2015e, Yildiz_2016e}. \paragraph*{Object Characteristics} 51 cases (71\%) ingested a large diameter object (\textgreater{}2.5cm) \cite{Akay_2015f, Al-Faham_2020k, AlShaaibi_2021b, Alao_2006i, Ali_2017, Ali_2022g, Apikotoa_2022f, Atayan_2016, Berry_2021e, Bhasin_2014, CamachoDorado_2018, Cauchi_2002, Chang_2017f, Cox_2007, Csaky_1998e, DivsalarP._2023a, Emamhadi_2018, Gardner_2017h, Guinan_2019f, Jehangir_2019h, Jin_2023, Kariholu_2008, Kerestes_2019, Kobiela_2015, Kumar_2001, Kumar_2019f, Losanoff_1996, Losanoff_1997e, Mesfin_2022a, Misra_2013, Naji_2012f, Ohno_2005, Peixoto_2017f, Qureshi_2016, Riva_2018j, Sakellaridis_2008f, Sultan_2024f, Tanrikulu_2015e, Thapa_2019f, Trgo_2012f, Wnęk_2015f, Yildiz_2016e, fjbuilsRepeatedBehaviorDeliberate2024, teWildt_2010}, 44 cases (61\%) ingested multiple objects \cite{Ali_2020f, Apikotoa_2022f, Ataya_2013, Atayan_2016, Beecroft_1998, Bhattacharjee_2008, Bhumi_2024f, CamachoDorado_2018, Cauchi_2002, Emamhadi_2018, Farhadi_2024h, Fry_2010, Goldman_1998f, Guinan_2019f, Hardy_2023g, Jehangir_2019h, Jin_2023, Kar_2015, Kariholu_2008, Kobiela_2015, Kumar_2001, Kumar_2019f, Li_2013, Liu_2005, Losanoff_1996, Mesfin_2022a, Misra_2013, Naji_2012f, Ohno_2005, Sobnach_2011f, Sultan_2024f, Tammana_2012j, Tanrikulu_2015e, Tay_2004, Thapa_2019f, Wadhwa_2015e, Wildhaber_2005, Yasin_2009, fjbuilsRepeatedBehaviorDeliberate2024, teWildt_2010}, 34 cases (47\%) ingested a sharp object \cite{AlShaaibi_2021b, Alao_2006i, Apikotoa_2022f, Ataya_2013, Benoist_2019e, Bhasin_2014, Bhattacharjee_2008, CamachoDorado_2018, Csaky_1998e, DelgadoSalazar_2020c, DivsalarP._2023a, Emamhadi_2018, Farhadi_2024h, Fry_2010, Guinan_2019f, Hardy_2023g, Jehangir_2019h, Jin_2023, Kariholu_2008, Kobiela_2015, Kumar_2019f, Losanoff_1996, Losanoff_1997e, Mesfin_2022a, Misra_2013, Sobnach_2011f, Yasin_2009, teWildt_2010}, 32 cases (44\%) ingested a long object (\textgreater{}5cm) \cite{Al-Faham_2020k, AlShaaibi_2021b, Ali_2017, Ali_2022g, Atayan_2016, Bhasin_2014, CamachoDorado_2018, Chang_2017f, Cox_2007, Csaky_1998e, DivsalarP._2023a, Emamhadi_2018, Fry_2010, Gardner_2017h, Jin_2023, Kariholu_2008, Kerestes_2019, Kobiela_2015, Kumar_2019f, Mesfin_2022a, Misra_2013, Ohno_2005, Qureshi_2016, Sakellaridis_2008f, Sultan_2024f, Thapa_2019f, Trgo_2012f, Yasin_2009, Yildiz_2016e, teWildt_2010}, 9 cases (12\%) ingested a magnet \cite{Ali_2020f, Bhumi_2024f, Cauchi_2002, Liu_2005, Naji_2012f, Ohno_2005, Tanrikulu_2015e, Tay_2004, Wildhaber_2005}, 2 cases (3\%) ingested a button battery \cite{Berry_2021e, Bhumi_2024f}. \paragraph*{Outcomes} 48 cases (67\%) experienced a complication \cite{Ali_2017, Ali_2020f, Apikotoa_2022f, Atayan_2016, Beecroft_1998, Benoist_2019e, Berry_2021e, Bhasin_2014, Bhumi_2024f, CamachoDorado_2018, Cauchi_2002, Cox_2007, Csaky_1998e, DelgadoSalazar_2020c, DivsalarP._2023a, Emamhadi_2018, Farhadi_2024h, Fry_2010, Gardner_2017h, Goldman_1998f, Jin_2023, Kariholu_2008, Kerestes_2019, Kobiela_2015, Kumar_2001, Kumar_2019f, Liu_2005, Losanoff_1996, Mesfin_2022a, Misra_2013, Naji_2012f, Ohno_2005, Sakellaridis_2008f, Sobnach_2011f, Sultan_2024f, Tanrikulu_2015e, Tay_2004, Thapa_2019f, Trgo_2012f, Tupesis_2004f, Wildhaber_2005, Wnęk_2015f, Yasin_2009, Yildiz_2016e}, 44 cases (61\%) underwent surgery \cite{Al-Faham_2020k, AlShaaibi_2021b, Alao_2006i, Ali_2017, Ali_2020f, Atayan_2016, Beecroft_1998, Bhasin_2014, CamachoDorado_2018, Cauchi_2002, Chang_2017f, Cox_2007, Csaky_1998e, DelgadoSalazar_2020c, DivsalarP._2023a, Farhadi_2024h, Fry_2010, Gardner_2017h, Jin_2023, Kariholu_2008, Kerestes_2019, Kobiela_2015, Kumar_2019f, Liu_2005, Losanoff_1996, Losanoff_1997e, Mesfin_2022a, Misra_2013, Naji_2012f, Sobnach_2011f, Tanrikulu_2015e, Tay_2004, Thapa_2019f, Tupesis_2004f, Wildhaber_2005, Wnęk_2015f, Yasin_2009, Yildiz_2016e, fjbuilsRepeatedBehaviorDeliberate2024}, 31 cases (43\%) underwent endoscopy \cite{Akay_2015f, Ali_2022g, Apikotoa_2022f, Atayan_2016, Benoist_2019e, Berry_2021e, Bhasin_2014, Bhumi_2024f, CamachoDorado_2018, Chang_2017f, DelgadoSalazar_2020c, Gardner_2017h, Guinan_2019f, Hardy_2023g, Jehangir_2019h, Kariholu_2008, Li_2013, Liu_2005, Ohno_2005, Peixoto_2017f, Qureshi_2016, Riva_2018j, Sakellaridis_2008f, Sultan_2024f, Tammana_2012j, Tanrikulu_2015e, Trgo_2012f, Wadhwa_2015e, Wnęk_2015f, teWildt_2010}, 7 cases (10\%) were managed conservatively \cite{Ataya_2013, Bhattacharjee_2008, DivsalarP._2023a, Emamhadi_2018, Goldman_1998f, Kar_2015, Kumar_2001}, 2 cases (3\%) died \cite{Emamhadi_2018, Kumar_2001}. All 90 were male gender. 90 cases (100\%) were detained at the time of ingestion \cite{Elghali_2016, Karp_1991b, Lee_2007}, 88 cases (98\%) were intentional ingestions \cite{Elghali_2016, Karp_1991b, Lee_2007}, 30 cases (33\%) had a psychiatric history documented \cite{Elghali_2016, Karp_1991b, Lee_2007}, 2 cases (2\%) had a history of prior ingestion \cite{Elghali_2016}. No cases were reported for were psychiatric inpatients, were displaced people, were under the influence of alcohol at the time of ingestion, and had a severe disability history.
\paragraph*{Motivation}  70 cases (78\%) reported protest motivation \cite{Elghali_2016, Karp_1991b, Lee_2007}, 12 cases (13\%) reported psychiatric motivation \cite{Karp_1991b}, 6 cases (7\%) reported self-harm motivation \cite{Elghali_2016, Karp_1991b}. No cases were reported for psychosocial motivation and other motivation.
\paragraph*{Object Characteristics}  68 cases (76\%) involved sharp object ingestion \cite{Elghali_2016, Karp_1991b, Lee_2007}, 32 cases (36\%) involved long (\textgreater 5cm) object ingestion \cite{Lee_2007}, 25 cases (28\%) involved ingestion of multiple objects \cite{Elghali_2016, Lee_2007}. No cases were reported for button battery ingestion, magnet ingestion, and involved large diameter (\textgreater 2.5cm) object ingestion.
\paragraph*{Outcomes}  47 cases (52\%) underwent endoscopic intervention \cite{Elghali_2016, Lee_2007}, 29 cases (32\%) were managed conservatively \cite{Elghali_2016, Karp_1991b}, 15 cases (17\%) underwent surgical intervention \cite{Elghali_2016, Karp_1991b, Lee_2007}, 6 cases (7\%) reported complications \cite{Lee_2007}, 1 case (1\%) died \cite{Elghali_2016}.
\paragraph*{Geographical Location}Cases were recorded in 33 countries: 13 cases from USA \cite{Alao_2006i, Ataya_2013, Bhumi_2024f, Fry_2010, Guinan_2019f, Hardy_2023g, Jehangir_2019h, Kerestes_2019, Kumar_2001, Liu_2005, Tammana_2012j, Tay_2004, Tupesis_2004f}; 7 cases from India \cite{Bhasin_2014, Bhattacharjee_2008, Kar_2015, Kariholu_2008, Kumar_2019f, Misra_2013, Wadhwa_2015e} and UK \cite{Beecroft_1998, Berry_2021e, Cauchi_2002, Cox_2007, Gardner_2017h, Qureshi_2016}; 6 cases from Bulgaria \cite{Losanoff_1996, Losanoff_1997e}; 5 cases from Iran \cite{DivsalarP._2023a, Emamhadi_2018, Farhadi_2024h}; 4 cases from Turkey \cite{Akay_2015f, Atayan_2016, Tanrikulu_2015e, Yildiz_2016e}; 2 cases from China \cite{Jin_2023, Li_2013}, Poland \cite{Kobiela_2015, Wnęk_2015f}, and Spain \cite{CamachoDorado_2018, fjbuilsRepeatedBehaviorDeliberate2024}; 1 case from Australia \cite{Apikotoa_2022f}, Bahrain \cite{Ali_2020f}, Croatia \cite{Trgo_2012f}, Ecuador \cite{DelgadoSalazar_2020c}, Egypt \cite{Ali_2022g}, Ethiopia \cite{Mesfin_2022a}, Germany \cite{teWildt_2010}, Greece \cite{Sakellaridis_2008f}, Hungary \cite{Csaky_1998e}, Iraq \cite{Al-Faham_2020k}, Israel \cite{Goldman_1998f}, Italy \cite{Riva_2018j}, Japan \cite{Ohno_2005}, Nepal \cite{Thapa_2019f}, Netherlands \cite{Benoist_2019e}, Oman \cite{AlShaaibi_2021b}, Pakistan \cite{Yasin_2009}, Portugal \cite{Peixoto_2017f}, Qatar \cite{Ali_2017}, Saudi Arabia \cite{Sultan_2024f}, South Africa \cite{Sobnach_2011f}, Sweden \cite{Naji_2012f}, Switzerland \cite{Wildhaber_2005}, and Taiwan \cite{Chang_2017f}. \paragraph*{Gender} 43 cases (60\%) were male \cite{Akay_2015f, Al-Faham_2020k, Alao_2006i, Ali_2017, Ali_2022g, Apikotoa_2022f, Atayan_2016, Benoist_2019e, Berry_2021e, Bhumi_2024f, CamachoDorado_2018, Csaky_1998e, Emamhadi_2018, Farhadi_2024h, Fry_2010, Gardner_2017h, Guinan_2019f, Jehangir_2019h, Jin_2023, Kobiela_2015, Kumar_2001, Kumar_2019f, Liu_2005, Losanoff_1996, Losanoff_1997e, Mesfin_2022a, Misra_2013, Qureshi_2016, Riva_2018j, Sobnach_2011f, Tammana_2012j, Tanrikulu_2015e, Tay_2004, Thapa_2019f, Trgo_2012f, Wadhwa_2015e, Yasin_2009, teWildt_2010}, 28 cases (39\%) were female \cite{AlShaaibi_2021b, Ali_2020f, Ataya_2013, Beecroft_1998, Bhasin_2014, Bhattacharjee_2008, Cauchi_2002, Chang_2017f, Cox_2007, DelgadoSalazar_2020c, DivsalarP._2023a, Goldman_1998f, Hardy_2023g, Kar_2015, Kariholu_2008, Kerestes_2019, Li_2013, Naji_2012f, Ohno_2005, Peixoto_2017f, Sakellaridis_2008f, Sultan_2024f, Tupesis_2004f, Wildhaber_2005, Wnęk_2015f, Yildiz_2016e}, 1 case (1\%) had no gender recorded \cite{fjbuilsRepeatedBehaviorDeliberate2024}. \paragraph*{Age Group} 25 cases (35\%) were between 26 and 40 years of age \cite{Alao_2006i, Ali_2022g, Apikotoa_2022f, Ataya_2013, Benoist_2019e, Bhasin_2014, Chang_2017f, Cox_2007, DelgadoSalazar_2020c, Farhadi_2024h, Fry_2010, Gardner_2017h, Guinan_2019f, Jin_2023, Kumar_2019f, Losanoff_1996, Misra_2013, Qureshi_2016, Riva_2018j, Sakellaridis_2008f, Tammana_2012j, Trgo_2012f, Wnęk_2015f, Yildiz_2016e, fjbuilsRepeatedBehaviorDeliberate2024}, 18 cases (25\%) were between 18 and 25 years of age \cite{Akay_2015f, Ali_2017, Atayan_2016, Bhattacharjee_2008, Csaky_1998e, Kar_2015, Kariholu_2008, Kobiela_2015, Losanoff_1996, Losanoff_1997e, Mesfin_2022a, Peixoto_2017f, Sobnach_2011f, Tupesis_2004f, Yasin_2009}, 13 cases (18\%) were under 18 years of age \cite{AlShaaibi_2021b, Ali_2020f, Cauchi_2002, DivsalarP._2023a, Goldman_1998f, Liu_2005, Naji_2012f, Ohno_2005, Tanrikulu_2015e, Tay_2004, Wildhaber_2005}, 11 cases (15\%) were between 41 and 60 years of age \cite{Al-Faham_2020k, Bhumi_2024f, CamachoDorado_2018, Emamhadi_2018, Hardy_2023g, Jehangir_2019h, Kumar_2001, Sultan_2024f, Thapa_2019f, Wadhwa_2015e, teWildt_2010}, 3 cases (4\%) were over 60 years of age \cite{Beecroft_1998, Kerestes_2019, Li_2013}, 2 cases (3\%) had no age documented \cite{Berry_2021e}. \paragraph*{Population} 36 cases (50\%) had a psychiatric history \cite{AlShaaibi_2021b, Alao_2006i, Ali_2020f, Apikotoa_2022f, Ataya_2013, Atayan_2016, Beecroft_1998, CamachoDorado_2018, Chang_2017f, DelgadoSalazar_2020c, DivsalarP._2023a, Farhadi_2024h, Fry_2010, Guinan_2019f, Hardy_2023g, Jehangir_2019h, Jin_2023, Kar_2015, Kerestes_2019, Kobiela_2015, Kumar_2001, Kumar_2019f, Liu_2005, Mesfin_2022a, Misra_2013, Ohno_2005, Peixoto_2017f, Sakellaridis_2008f, Sultan_2024f, Tammana_2012j, Tanrikulu_2015e, Yildiz_2016e, fjbuilsRepeatedBehaviorDeliberate2024, teWildt_2010}, 19 cases (26\%) had ingested previously \cite{Alao_2006i, Apikotoa_2022f, Berry_2021e, Bhattacharjee_2008, Csaky_1998e, DivsalarP._2023a, Emamhadi_2018, Guinan_2019f, Jehangir_2019h, Jin_2023, Liu_2005, Sakellaridis_2008f, Tanrikulu_2015e, Thapa_2019f, Yildiz_2016e, fjbuilsRepeatedBehaviorDeliberate2024, teWildt_2010}, 12 cases (17\%) were detained persons \cite{Alao_2006i, Ali_2022g, Apikotoa_2022f, Losanoff_1996, Losanoff_1997e, Qureshi_2016, Tammana_2012j, Trgo_2012f}, 7 cases (10\%) were severely disabled \cite{Atayan_2016, Kerestes_2019, Liu_2005, Ohno_2005, Peixoto_2017f, Yildiz_2016e, teWildt_2010}, 4 cases (6\%) were psychiatric inpatients \cite{DivsalarP._2023a, fjbuilsRepeatedBehaviorDeliberate2024, teWildt_2010}, 3 cases (4\%) were under the influence of alcohol \cite{Benoist_2019e, Csaky_1998e, Thapa_2019f}, 2 cases (3\%) were displaced people \cite{Akay_2015f, Gardner_2017h}. \paragraph*{Motivation} 34 cases (47\%) had a psychiatric motivation \cite{Al-Faham_2020k, Alao_2006i, Ali_2020f, Apikotoa_2022f, Ataya_2013, Atayan_2016, Bhasin_2014, Bhattacharjee_2008, DelgadoSalazar_2020c, DivsalarP._2023a, Emamhadi_2018, Farhadi_2024h, Guinan_2019f, Hardy_2023g, Jehangir_2019h, Jin_2023, Kar_2015, Kariholu_2008, Kerestes_2019, Kobiela_2015, Kumar_2001, Kumar_2019f, Li_2013, Liu_2005, Misra_2013, Ohno_2005, Sakellaridis_2008f, Sultan_2024f, Tammana_2012j, Tanrikulu_2015e, Yasin_2009, teWildt_2010}, 21 cases (29\%) were motivated by self-harm intention \cite{Al-Faham_2020k, AlShaaibi_2021b, Alao_2006i, Ali_2017, CamachoDorado_2018, Chang_2017f, Cox_2007, Csaky_1998e, Fry_2010, Li_2013, Losanoff_1996, Losanoff_1997e, Mesfin_2022a, Sakellaridis_2008f, Tammana_2012j, Tanrikulu_2015e, fjbuilsRepeatedBehaviorDeliberate2024}, 17 cases (24\%) had a psychosocial motivation \cite{Akay_2015f, Benoist_2019e, Bhattacharjee_2008, Cauchi_2002, Goldman_1998f, Hardy_2023g, Kobiela_2015, Li_2013, Naji_2012f, Qureshi_2016, Riva_2018j, Sobnach_2011f, Tay_2004, Thapa_2019f, Tupesis_2004f, Wildhaber_2005, Wnęk_2015f}, 9 cases (12\%) were motivated by protest \cite{Bhumi_2024f, Gardner_2017h, Losanoff_1996, Losanoff_1997e, Tupesis_2004f}, 9 cases (12\%) had another documented motivation \cite{Ali_2020f, Ali_2022g, Emamhadi_2018, Guinan_2019f, Peixoto_2017f, Sakellaridis_2008f, Trgo_2012f, Wadhwa_2015e, Yildiz_2016e}. \paragraph*{Object Characteristics} 51 cases (71\%) ingested a large diameter object (\textgreater{}2.5cm) \cite{Akay_2015f, Al-Faham_2020k, AlShaaibi_2021b, Alao_2006i, Ali_2017, Ali_2022g, Apikotoa_2022f, Atayan_2016, Berry_2021e, Bhasin_2014, CamachoDorado_2018, Cauchi_2002, Chang_2017f, Cox_2007, Csaky_1998e, DivsalarP._2023a, Emamhadi_2018, Gardner_2017h, Guinan_2019f, Jehangir_2019h, Jin_2023, Kariholu_2008, Kerestes_2019, Kobiela_2015, Kumar_2001, Kumar_2019f, Losanoff_1996, Losanoff_1997e, Mesfin_2022a, Misra_2013, Naji_2012f, Ohno_2005, Peixoto_2017f, Qureshi_2016, Riva_2018j, Sakellaridis_2008f, Sultan_2024f, Tanrikulu_2015e, Thapa_2019f, Trgo_2012f, Wnęk_2015f, Yildiz_2016e, fjbuilsRepeatedBehaviorDeliberate2024, teWildt_2010}, 44 cases (61\%) ingested multiple objects \cite{Ali_2020f, Apikotoa_2022f, Ataya_2013, Atayan_2016, Beecroft_1998, Bhattacharjee_2008, Bhumi_2024f, CamachoDorado_2018, Cauchi_2002, Emamhadi_2018, Farhadi_2024h, Fry_2010, Goldman_1998f, Guinan_2019f, Hardy_2023g, Jehangir_2019h, Jin_2023, Kar_2015, Kariholu_2008, Kobiela_2015, Kumar_2001, Kumar_2019f, Li_2013, Liu_2005, Losanoff_1996, Mesfin_2022a, Misra_2013, Naji_2012f, Ohno_2005, Sobnach_2011f, Sultan_2024f, Tammana_2012j, Tanrikulu_2015e, Tay_2004, Thapa_2019f, Wadhwa_2015e, Wildhaber_2005, Yasin_2009, fjbuilsRepeatedBehaviorDeliberate2024, teWildt_2010}, 34 cases (47\%) ingested a sharp object \cite{AlShaaibi_2021b, Alao_2006i, Apikotoa_2022f, Ataya_2013, Benoist_2019e, Bhasin_2014, Bhattacharjee_2008, CamachoDorado_2018, Csaky_1998e, DelgadoSalazar_2020c, DivsalarP._2023a, Emamhadi_2018, Farhadi_2024h, Fry_2010, Guinan_2019f, Hardy_2023g, Jehangir_2019h, Jin_2023, Kariholu_2008, Kobiela_2015, Kumar_2019f, Losanoff_1996, Losanoff_1997e, Mesfin_2022a, Misra_2013, Sobnach_2011f, Yasin_2009, teWildt_2010}, 32 cases (44\%) ingested a long object (\textgreater{}5cm) \cite{Al-Faham_2020k, AlShaaibi_2021b, Ali_2017, Ali_2022g, Atayan_2016, Bhasin_2014, CamachoDorado_2018, Chang_2017f, Cox_2007, Csaky_1998e, DivsalarP._2023a, Emamhadi_2018, Fry_2010, Gardner_2017h, Jin_2023, Kariholu_2008, Kerestes_2019, Kobiela_2015, Kumar_2019f, Mesfin_2022a, Misra_2013, Ohno_2005, Qureshi_2016, Sakellaridis_2008f, Sultan_2024f, Thapa_2019f, Trgo_2012f, Yasin_2009, Yildiz_2016e, teWildt_2010}, 9 cases (12\%) ingested a magnet \cite{Ali_2020f, Bhumi_2024f, Cauchi_2002, Liu_2005, Naji_2012f, Ohno_2005, Tanrikulu_2015e, Tay_2004, Wildhaber_2005}, 2 cases (3\%) ingested a button battery \cite{Berry_2021e, Bhumi_2024f}. \paragraph*{Outcomes} 48 cases (67\%) experienced a complication \cite{Ali_2017, Ali_2020f, Apikotoa_2022f, Atayan_2016, Beecroft_1998, Benoist_2019e, Berry_2021e, Bhasin_2014, Bhumi_2024f, CamachoDorado_2018, Cauchi_2002, Cox_2007, Csaky_1998e, DelgadoSalazar_2020c, DivsalarP._2023a, Emamhadi_2018, Farhadi_2024h, Fry_2010, Gardner_2017h, Goldman_1998f, Jin_2023, Kariholu_2008, Kerestes_2019, Kobiela_2015, Kumar_2001, Kumar_2019f, Liu_2005, Losanoff_1996, Mesfin_2022a, Misra_2013, Naji_2012f, Ohno_2005, Sakellaridis_2008f, Sobnach_2011f, Sultan_2024f, Tanrikulu_2015e, Tay_2004, Thapa_2019f, Trgo_2012f, Tupesis_2004f, Wildhaber_2005, Wnęk_2015f, Yasin_2009, Yildiz_2016e}, 44 cases (61\%) underwent surgery \cite{Al-Faham_2020k, AlShaaibi_2021b, Alao_2006i, Ali_2017, Ali_2020f, Atayan_2016, Beecroft_1998, Bhasin_2014, CamachoDorado_2018, Cauchi_2002, Chang_2017f, Cox_2007, Csaky_1998e, DelgadoSalazar_2020c, DivsalarP._2023a, Farhadi_2024h, Fry_2010, Gardner_2017h, Jin_2023, Kariholu_2008, Kerestes_2019, Kobiela_2015, Kumar_2019f, Liu_2005, Losanoff_1996, Losanoff_1997e, Mesfin_2022a, Misra_2013, Naji_2012f, Sobnach_2011f, Tanrikulu_2015e, Tay_2004, Thapa_2019f, Tupesis_2004f, Wildhaber_2005, Wnęk_2015f, Yasin_2009, Yildiz_2016e, fjbuilsRepeatedBehaviorDeliberate2024}, 31 cases (43\%) underwent endoscopy \cite{Akay_2015f, Ali_2022g, Apikotoa_2022f, Atayan_2016, Benoist_2019e, Berry_2021e, Bhasin_2014, Bhumi_2024f, CamachoDorado_2018, Chang_2017f, DelgadoSalazar_2020c, Gardner_2017h, Guinan_2019f, Hardy_2023g, Jehangir_2019h, Kariholu_2008, Li_2013, Liu_2005, Ohno_2005, Peixoto_2017f, Qureshi_2016, Riva_2018j, Sakellaridis_2008f, Sultan_2024f, Tammana_2012j, Tanrikulu_2015e, Trgo_2012f, Wadhwa_2015e, Wnęk_2015f, teWildt_2010}, 7 cases (10\%) were managed conservatively \cite{Ataya_2013, Bhattacharjee_2008, DivsalarP._2023a, Emamhadi_2018, Goldman_1998f, Kar_2015, Kumar_2001}, 2 cases (3\%) died \cite{Emamhadi_2018, Kumar_2001}. All 90 were male gender. 90 cases (100\%) were detained at the time of ingestion \cite{Elghali_2016, Karp_1991b, Lee_2007}, 88 cases (98\%) were intentional ingestions \cite{Elghali_2016, Karp_1991b, Lee_2007}, 30 cases (33\%) had a psychiatric history documented \cite{Elghali_2016, Karp_1991b, Lee_2007}, 2 cases (2\%) had a history of prior ingestion \cite{Elghali_2016}. No cases were reported for were psychiatric inpatients, were displaced people, were under the influence of alcohol at the time of ingestion, and had a severe disability history.
\paragraph*{Motivation}  70 cases (78\%) reported protest motivation \cite{Elghali_2016, Karp_1991b, Lee_2007}, 12 cases (13\%) reported psychiatric motivation \cite{Karp_1991b}, 6 cases (7\%) reported self-harm motivation \cite{Elghali_2016, Karp_1991b}. No cases were reported for psychosocial motivation and other motivation.
\paragraph*{Object Characteristics}  68 cases (76\%) involved sharp object ingestion \cite{Elghali_2016, Karp_1991b, Lee_2007}, 32 cases (36\%) involved long (\textgreater 5cm) object ingestion \cite{Lee_2007}, 25 cases (28\%) involved ingestion of multiple objects \cite{Elghali_2016, Lee_2007}. No cases were reported for button battery ingestion, magnet ingestion, and involved large diameter (\textgreater 2.5cm) object ingestion.
\paragraph*{Outcomes}  47 cases (52\%) underwent endoscopic intervention \cite{Elghali_2016, Lee_2007}, 29 cases (32\%) were managed conservatively \cite{Elghali_2016, Karp_1991b}, 15 cases (17\%) underwent surgical intervention \cite{Elghali_2016, Karp_1991b, Lee_2007}, 6 cases (7\%) reported complications \cite{Lee_2007}, 1 case (1\%) died \cite{Elghali_2016}.
\paragraph*{Geographical Location}Cases were recorded in 33 countries: 13 cases from USA \cite{Alao_2006i, Ataya_2013, Bhumi_2024f, Fry_2010, Guinan_2019f, Hardy_2023g, Jehangir_2019h, Kerestes_2019, Kumar_2001, Liu_2005, Tammana_2012j, Tay_2004, Tupesis_2004f}; 7 cases from India \cite{Bhasin_2014, Bhattacharjee_2008, Kar_2015, Kariholu_2008, Kumar_2019f, Misra_2013, Wadhwa_2015e} and UK \cite{Beecroft_1998, Berry_2021e, Cauchi_2002, Cox_2007, Gardner_2017h, Qureshi_2016}; 6 cases from Bulgaria \cite{Losanoff_1996, Losanoff_1997e}; 5 cases from Iran \cite{DivsalarP._2023a, Emamhadi_2018, Farhadi_2024h}; 4 cases from Turkey \cite{Akay_2015f, Atayan_2016, Tanrikulu_2015e, Yildiz_2016e}; 2 cases from China \cite{Jin_2023, Li_2013}, Poland \cite{Kobiela_2015, Wnęk_2015f}, and Spain \cite{CamachoDorado_2018, fjbuilsRepeatedBehaviorDeliberate2024}; 1 case from Australia \cite{Apikotoa_2022f}, Bahrain \cite{Ali_2020f}, Croatia \cite{Trgo_2012f}, Ecuador \cite{DelgadoSalazar_2020c}, Egypt \cite{Ali_2022g}, Ethiopia \cite{Mesfin_2022a}, Germany \cite{teWildt_2010}, Greece \cite{Sakellaridis_2008f}, Hungary \cite{Csaky_1998e}, Iraq \cite{Al-Faham_2020k}, Israel \cite{Goldman_1998f}, Italy \cite{Riva_2018j}, Japan \cite{Ohno_2005}, Nepal \cite{Thapa_2019f}, Netherlands \cite{Benoist_2019e}, Oman \cite{AlShaaibi_2021b}, Pakistan \cite{Yasin_2009}, Portugal \cite{Peixoto_2017f}, Qatar \cite{Ali_2017}, Saudi Arabia \cite{Sultan_2024f}, South Africa \cite{Sobnach_2011f}, Sweden \cite{Naji_2012f}, Switzerland \cite{Wildhaber_2005}, and Taiwan \cite{Chang_2017f}. \paragraph*{Gender} 43 cases (60\%) were male \cite{Akay_2015f, Al-Faham_2020k, Alao_2006i, Ali_2017, Ali_2022g, Apikotoa_2022f, Atayan_2016, Benoist_2019e, Berry_2021e, Bhumi_2024f, CamachoDorado_2018, Csaky_1998e, Emamhadi_2018, Farhadi_2024h, Fry_2010, Gardner_2017h, Guinan_2019f, Jehangir_2019h, Jin_2023, Kobiela_2015, Kumar_2001, Kumar_2019f, Liu_2005, Losanoff_1996, Losanoff_1997e, Mesfin_2022a, Misra_2013, Qureshi_2016, Riva_2018j, Sobnach_2011f, Tammana_2012j, Tanrikulu_2015e, Tay_2004, Thapa_2019f, Trgo_2012f, Wadhwa_2015e, Yasin_2009, teWildt_2010}, 28 cases (39\%) were female \cite{AlShaaibi_2021b, Ali_2020f, Ataya_2013, Beecroft_1998, Bhasin_2014, Bhattacharjee_2008, Cauchi_2002, Chang_2017f, Cox_2007, DelgadoSalazar_2020c, DivsalarP._2023a, Goldman_1998f, Hardy_2023g, Kar_2015, Kariholu_2008, Kerestes_2019, Li_2013, Naji_2012f, Ohno_2005, Peixoto_2017f, Sakellaridis_2008f, Sultan_2024f, Tupesis_2004f, Wildhaber_2005, Wnęk_2015f, Yildiz_2016e}, 1 case (1\%) had no gender recorded \cite{fjbuilsRepeatedBehaviorDeliberate2024}. \paragraph*{Age Group} 25 cases (35\%) were between 26 and 40 years of age \cite{Alao_2006i, Ali_2022g, Apikotoa_2022f, Ataya_2013, Benoist_2019e, Bhasin_2014, Chang_2017f, Cox_2007, DelgadoSalazar_2020c, Farhadi_2024h, Fry_2010, Gardner_2017h, Guinan_2019f, Jin_2023, Kumar_2019f, Losanoff_1996, Misra_2013, Qureshi_2016, Riva_2018j, Sakellaridis_2008f, Tammana_2012j, Trgo_2012f, Wnęk_2015f, Yildiz_2016e, fjbuilsRepeatedBehaviorDeliberate2024}, 18 cases (25\%) were between 18 and 25 years of age \cite{Akay_2015f, Ali_2017, Atayan_2016, Bhattacharjee_2008, Csaky_1998e, Kar_2015, Kariholu_2008, Kobiela_2015, Losanoff_1996, Losanoff_1997e, Mesfin_2022a, Peixoto_2017f, Sobnach_2011f, Tupesis_2004f, Yasin_2009}, 13 cases (18\%) were under 18 years of age \cite{AlShaaibi_2021b, Ali_2020f, Cauchi_2002, DivsalarP._2023a, Goldman_1998f, Liu_2005, Naji_2012f, Ohno_2005, Tanrikulu_2015e, Tay_2004, Wildhaber_2005}, 11 cases (15\%) were between 41 and 60 years of age \cite{Al-Faham_2020k, Bhumi_2024f, CamachoDorado_2018, Emamhadi_2018, Hardy_2023g, Jehangir_2019h, Kumar_2001, Sultan_2024f, Thapa_2019f, Wadhwa_2015e, teWildt_2010}, 3 cases (4\%) were over 60 years of age \cite{Beecroft_1998, Kerestes_2019, Li_2013}, 2 cases (3\%) had no age documented \cite{Berry_2021e}. \paragraph*{Population} 36 cases (50\%) had a psychiatric history \cite{AlShaaibi_2021b, Alao_2006i, Ali_2020f, Apikotoa_2022f, Ataya_2013, Atayan_2016, Beecroft_1998, CamachoDorado_2018, Chang_2017f, DelgadoSalazar_2020c, DivsalarP._2023a, Farhadi_2024h, Fry_2010, Guinan_2019f, Hardy_2023g, Jehangir_2019h, Jin_2023, Kar_2015, Kerestes_2019, Kobiela_2015, Kumar_2001, Kumar_2019f, Liu_2005, Mesfin_2022a, Misra_2013, Ohno_2005, Peixoto_2017f, Sakellaridis_2008f, Sultan_2024f, Tammana_2012j, Tanrikulu_2015e, Yildiz_2016e, fjbuilsRepeatedBehaviorDeliberate2024, teWildt_2010}, 19 cases (26\%) had ingested previously \cite{Alao_2006i, Apikotoa_2022f, Berry_2021e, Bhattacharjee_2008, Csaky_1998e, DivsalarP._2023a, Emamhadi_2018, Guinan_2019f, Jehangir_2019h, Jin_2023, Liu_2005, Sakellaridis_2008f, Tanrikulu_2015e, Thapa_2019f, Yildiz_2016e, fjbuilsRepeatedBehaviorDeliberate2024, teWildt_2010}, 12 cases (17\%) were detained persons \cite{Alao_2006i, Ali_2022g, Apikotoa_2022f, Losanoff_1996, Losanoff_1997e, Qureshi_2016, Tammana_2012j, Trgo_2012f}, 7 cases (10\%) were severely disabled \cite{Atayan_2016, Kerestes_2019, Liu_2005, Ohno_2005, Peixoto_2017f, Yildiz_2016e, teWildt_2010}, 4 cases (6\%) were psychiatric inpatients \cite{DivsalarP._2023a, fjbuilsRepeatedBehaviorDeliberate2024, teWildt_2010}, 3 cases (4\%) were under the influence of alcohol \cite{Benoist_2019e, Csaky_1998e, Thapa_2019f}, 2 cases (3\%) were displaced people \cite{Akay_2015f, Gardner_2017h}. \paragraph*{Motivation} 34 cases (47\%) had a psychiatric motivation \cite{Al-Faham_2020k, Alao_2006i, Ali_2020f, Apikotoa_2022f, Ataya_2013, Atayan_2016, Bhasin_2014, Bhattacharjee_2008, DelgadoSalazar_2020c, DivsalarP._2023a, Emamhadi_2018, Farhadi_2024h, Guinan_2019f, Hardy_2023g, Jehangir_2019h, Jin_2023, Kar_2015, Kariholu_2008, Kerestes_2019, Kobiela_2015, Kumar_2001, Kumar_2019f, Li_2013, Liu_2005, Misra_2013, Ohno_2005, Sakellaridis_2008f, Sultan_2024f, Tammana_2012j, Tanrikulu_2015e, Yasin_2009, teWildt_2010}, 21 cases (29\%) were motivated by self-harm intention \cite{Al-Faham_2020k, AlShaaibi_2021b, Alao_2006i, Ali_2017, CamachoDorado_2018, Chang_2017f, Cox_2007, Csaky_1998e, Fry_2010, Li_2013, Losanoff_1996, Losanoff_1997e, Mesfin_2022a, Sakellaridis_2008f, Tammana_2012j, Tanrikulu_2015e, fjbuilsRepeatedBehaviorDeliberate2024}, 17 cases (24\%) had a psychosocial motivation \cite{Akay_2015f, Benoist_2019e, Bhattacharjee_2008, Cauchi_2002, Goldman_1998f, Hardy_2023g, Kobiela_2015, Li_2013, Naji_2012f, Qureshi_2016, Riva_2018j, Sobnach_2011f, Tay_2004, Thapa_2019f, Tupesis_2004f, Wildhaber_2005, Wnęk_2015f}, 9 cases (12\%) were motivated by protest \cite{Bhumi_2024f, Gardner_2017h, Losanoff_1996, Losanoff_1997e, Tupesis_2004f}, 9 cases (12\%) had another documented motivation \cite{Ali_2020f, Ali_2022g, Emamhadi_2018, Guinan_2019f, Peixoto_2017f, Sakellaridis_2008f, Trgo_2012f, Wadhwa_2015e, Yildiz_2016e}. \paragraph*{Object Characteristics} 51 cases (71\%) ingested a large diameter object (\textgreater{}2.5cm) \cite{Akay_2015f, Al-Faham_2020k, AlShaaibi_2021b, Alao_2006i, Ali_2017, Ali_2022g, Apikotoa_2022f, Atayan_2016, Berry_2021e, Bhasin_2014, CamachoDorado_2018, Cauchi_2002, Chang_2017f, Cox_2007, Csaky_1998e, DivsalarP._2023a, Emamhadi_2018, Gardner_2017h, Guinan_2019f, Jehangir_2019h, Jin_2023, Kariholu_2008, Kerestes_2019, Kobiela_2015, Kumar_2001, Kumar_2019f, Losanoff_1996, Losanoff_1997e, Mesfin_2022a, Misra_2013, Naji_2012f, Ohno_2005, Peixoto_2017f, Qureshi_2016, Riva_2018j, Sakellaridis_2008f, Sultan_2024f, Tanrikulu_2015e, Thapa_2019f, Trgo_2012f, Wnęk_2015f, Yildiz_2016e, fjbuilsRepeatedBehaviorDeliberate2024, teWildt_2010}, 44 cases (61\%) ingested multiple objects \cite{Ali_2020f, Apikotoa_2022f, Ataya_2013, Atayan_2016, Beecroft_1998, Bhattacharjee_2008, Bhumi_2024f, CamachoDorado_2018, Cauchi_2002, Emamhadi_2018, Farhadi_2024h, Fry_2010, Goldman_1998f, Guinan_2019f, Hardy_2023g, Jehangir_2019h, Jin_2023, Kar_2015, Kariholu_2008, Kobiela_2015, Kumar_2001, Kumar_2019f, Li_2013, Liu_2005, Losanoff_1996, Mesfin_2022a, Misra_2013, Naji_2012f, Ohno_2005, Sobnach_2011f, Sultan_2024f, Tammana_2012j, Tanrikulu_2015e, Tay_2004, Thapa_2019f, Wadhwa_2015e, Wildhaber_2005, Yasin_2009, fjbuilsRepeatedBehaviorDeliberate2024, teWildt_2010}, 34 cases (47\%) ingested a sharp object \cite{AlShaaibi_2021b, Alao_2006i, Apikotoa_2022f, Ataya_2013, Benoist_2019e, Bhasin_2014, Bhattacharjee_2008, CamachoDorado_2018, Csaky_1998e, DelgadoSalazar_2020c, DivsalarP._2023a, Emamhadi_2018, Farhadi_2024h, Fry_2010, Guinan_2019f, Hardy_2023g, Jehangir_2019h, Jin_2023, Kariholu_2008, Kobiela_2015, Kumar_2019f, Losanoff_1996, Losanoff_1997e, Mesfin_2022a, Misra_2013, Sobnach_2011f, Yasin_2009, teWildt_2010}, 32 cases (44\%) ingested a long object (\textgreater{}5cm) \cite{Al-Faham_2020k, AlShaaibi_2021b, Ali_2017, Ali_2022g, Atayan_2016, Bhasin_2014, CamachoDorado_2018, Chang_2017f, Cox_2007, Csaky_1998e, DivsalarP._2023a, Emamhadi_2018, Fry_2010, Gardner_2017h, Jin_2023, Kariholu_2008, Kerestes_2019, Kobiela_2015, Kumar_2019f, Mesfin_2022a, Misra_2013, Ohno_2005, Qureshi_2016, Sakellaridis_2008f, Sultan_2024f, Thapa_2019f, Trgo_2012f, Yasin_2009, Yildiz_2016e, teWildt_2010}, 9 cases (12\%) ingested a magnet \cite{Ali_2020f, Bhumi_2024f, Cauchi_2002, Liu_2005, Naji_2012f, Ohno_2005, Tanrikulu_2015e, Tay_2004, Wildhaber_2005}, 2 cases (3\%) ingested a button battery \cite{Berry_2021e, Bhumi_2024f}. \paragraph*{Outcomes} 48 cases (67\%) experienced a complication \cite{Ali_2017, Ali_2020f, Apikotoa_2022f, Atayan_2016, Beecroft_1998, Benoist_2019e, Berry_2021e, Bhasin_2014, Bhumi_2024f, CamachoDorado_2018, Cauchi_2002, Cox_2007, Csaky_1998e, DelgadoSalazar_2020c, DivsalarP._2023a, Emamhadi_2018, Farhadi_2024h, Fry_2010, Gardner_2017h, Goldman_1998f, Jin_2023, Kariholu_2008, Kerestes_2019, Kobiela_2015, Kumar_2001, Kumar_2019f, Liu_2005, Losanoff_1996, Mesfin_2022a, Misra_2013, Naji_2012f, Ohno_2005, Sakellaridis_2008f, Sobnach_2011f, Sultan_2024f, Tanrikulu_2015e, Tay_2004, Thapa_2019f, Trgo_2012f, Tupesis_2004f, Wildhaber_2005, Wnęk_2015f, Yasin_2009, Yildiz_2016e}, 44 cases (61\%) underwent surgery \cite{Al-Faham_2020k, AlShaaibi_2021b, Alao_2006i, Ali_2017, Ali_2020f, Atayan_2016, Beecroft_1998, Bhasin_2014, CamachoDorado_2018, Cauchi_2002, Chang_2017f, Cox_2007, Csaky_1998e, DelgadoSalazar_2020c, DivsalarP._2023a, Farhadi_2024h, Fry_2010, Gardner_2017h, Jin_2023, Kariholu_2008, Kerestes_2019, Kobiela_2015, Kumar_2019f, Liu_2005, Losanoff_1996, Losanoff_1997e, Mesfin_2022a, Misra_2013, Naji_2012f, Sobnach_2011f, Tanrikulu_2015e, Tay_2004, Thapa_2019f, Tupesis_2004f, Wildhaber_2005, Wnęk_2015f, Yasin_2009, Yildiz_2016e, fjbuilsRepeatedBehaviorDeliberate2024}, 31 cases (43\%) underwent endoscopy \cite{Akay_2015f, Ali_2022g, Apikotoa_2022f, Atayan_2016, Benoist_2019e, Berry_2021e, Bhasin_2014, Bhumi_2024f, CamachoDorado_2018, Chang_2017f, DelgadoSalazar_2020c, Gardner_2017h, Guinan_2019f, Hardy_2023g, Jehangir_2019h, Kariholu_2008, Li_2013, Liu_2005, Ohno_2005, Peixoto_2017f, Qureshi_2016, Riva_2018j, Sakellaridis_2008f, Sultan_2024f, Tammana_2012j, Tanrikulu_2015e, Trgo_2012f, Wadhwa_2015e, Wnęk_2015f, teWildt_2010}, 7 cases (10\%) were managed conservatively \cite{Ataya_2013, Bhattacharjee_2008, DivsalarP._2023a, Emamhadi_2018, Goldman_1998f, Kar_2015, Kumar_2001}, 2 cases (3\%) died \cite{Emamhadi_2018, Kumar_2001}. All 90 were male gender. 90 cases (100\%) were detained at the time of ingestion \cite{Elghali_2016, Karp_1991b, Lee_2007}, 88 cases (98\%) were intentional ingestions \cite{Elghali_2016, Karp_1991b, Lee_2007}, 30 cases (33\%) had a psychiatric history documented \cite{Elghali_2016, Karp_1991b, Lee_2007}, 2 cases (2\%) had a history of prior ingestion \cite{Elghali_2016}. No cases were reported for were psychiatric inpatients, were displaced people, were under the influence of alcohol at the time of ingestion, and had a severe disability history.
\paragraph*{Motivation}  70 cases (78\%) reported protest motivation \cite{Elghali_2016, Karp_1991b, Lee_2007}, 12 cases (13\%) reported psychiatric motivation \cite{Karp_1991b}, 6 cases (7\%) reported self-harm motivation \cite{Elghali_2016, Karp_1991b}. No cases were reported for psychosocial motivation and other motivation.
\paragraph*{Object Characteristics}  68 cases (76\%) involved sharp object ingestion \cite{Elghali_2016, Karp_1991b, Lee_2007}, 32 cases (36\%) involved long (\textgreater 5cm) object ingestion \cite{Lee_2007}, 25 cases (28\%) involved ingestion of multiple objects \cite{Elghali_2016, Lee_2007}. No cases were reported for button battery ingestion, magnet ingestion, and involved large diameter (\textgreater 2.5cm) object ingestion.
\paragraph*{Outcomes}  47 cases (52\%) underwent endoscopic intervention \cite{Elghali_2016, Lee_2007}, 29 cases (32\%) were managed conservatively \cite{Elghali_2016, Karp_1991b}, 15 cases (17\%) underwent surgical intervention \cite{Elghali_2016, Karp_1991b, Lee_2007}, 6 cases (7\%) reported complications \cite{Lee_2007}, 1 case (1\%) died \cite{Elghali_2016}.
\paragraph*{Geographical Location}Cases were recorded in 33 countries: 13 cases from USA \cite{Alao_2006i, Ataya_2013, Bhumi_2024f, Fry_2010, Guinan_2019f, Hardy_2023g, Jehangir_2019h, Kerestes_2019, Kumar_2001, Liu_2005, Tammana_2012j, Tay_2004, Tupesis_2004f}; 7 cases from India \cite{Bhasin_2014, Bhattacharjee_2008, Kar_2015, Kariholu_2008, Kumar_2019f, Misra_2013, Wadhwa_2015e} and UK \cite{Beecroft_1998, Berry_2021e, Cauchi_2002, Cox_2007, Gardner_2017h, Qureshi_2016}; 6 cases from Bulgaria \cite{Losanoff_1996, Losanoff_1997e}; 5 cases from Iran \cite{DivsalarP._2023a, Emamhadi_2018, Farhadi_2024h}; 4 cases from Turkey \cite{Akay_2015f, Atayan_2016, Tanrikulu_2015e, Yildiz_2016e}; 2 cases from China \cite{Jin_2023, Li_2013}, Poland \cite{Kobiela_2015, Wnęk_2015f}, and Spain \cite{CamachoDorado_2018, fjbuilsRepeatedBehaviorDeliberate2024}; 1 case from Australia \cite{Apikotoa_2022f}, Bahrain \cite{Ali_2020f}, Croatia \cite{Trgo_2012f}, Ecuador \cite{DelgadoSalazar_2020c}, Egypt \cite{Ali_2022g}, Ethiopia \cite{Mesfin_2022a}, Germany \cite{teWildt_2010}, Greece \cite{Sakellaridis_2008f}, Hungary \cite{Csaky_1998e}, Iraq \cite{Al-Faham_2020k}, Israel \cite{Goldman_1998f}, Italy \cite{Riva_2018j}, Japan \cite{Ohno_2005}, Nepal \cite{Thapa_2019f}, Netherlands \cite{Benoist_2019e}, Oman \cite{AlShaaibi_2021b}, Pakistan \cite{Yasin_2009}, Portugal \cite{Peixoto_2017f}, Qatar \cite{Ali_2017}, Saudi Arabia \cite{Sultan_2024f}, South Africa \cite{Sobnach_2011f}, Sweden \cite{Naji_2012f}, Switzerland \cite{Wildhaber_2005}, and Taiwan \cite{Chang_2017f}. \paragraph*{Gender} 43 cases (60\%) were male \cite{Akay_2015f, Al-Faham_2020k, Alao_2006i, Ali_2017, Ali_2022g, Apikotoa_2022f, Atayan_2016, Benoist_2019e, Berry_2021e, Bhumi_2024f, CamachoDorado_2018, Csaky_1998e, Emamhadi_2018, Farhadi_2024h, Fry_2010, Gardner_2017h, Guinan_2019f, Jehangir_2019h, Jin_2023, Kobiela_2015, Kumar_2001, Kumar_2019f, Liu_2005, Losanoff_1996, Losanoff_1997e, Mesfin_2022a, Misra_2013, Qureshi_2016, Riva_2018j, Sobnach_2011f, Tammana_2012j, Tanrikulu_2015e, Tay_2004, Thapa_2019f, Trgo_2012f, Wadhwa_2015e, Yasin_2009, teWildt_2010}, 28 cases (39\%) were female \cite{AlShaaibi_2021b, Ali_2020f, Ataya_2013, Beecroft_1998, Bhasin_2014, Bhattacharjee_2008, Cauchi_2002, Chang_2017f, Cox_2007, DelgadoSalazar_2020c, DivsalarP._2023a, Goldman_1998f, Hardy_2023g, Kar_2015, Kariholu_2008, Kerestes_2019, Li_2013, Naji_2012f, Ohno_2005, Peixoto_2017f, Sakellaridis_2008f, Sultan_2024f, Tupesis_2004f, Wildhaber_2005, Wnęk_2015f, Yildiz_2016e}, 1 case (1\%) had no gender recorded \cite{fjbuilsRepeatedBehaviorDeliberate2024}. \paragraph*{Age Group} 25 cases (35\%) were between 26 and 40 years of age \cite{Alao_2006i, Ali_2022g, Apikotoa_2022f, Ataya_2013, Benoist_2019e, Bhasin_2014, Chang_2017f, Cox_2007, DelgadoSalazar_2020c, Farhadi_2024h, Fry_2010, Gardner_2017h, Guinan_2019f, Jin_2023, Kumar_2019f, Losanoff_1996, Misra_2013, Qureshi_2016, Riva_2018j, Sakellaridis_2008f, Tammana_2012j, Trgo_2012f, Wnęk_2015f, Yildiz_2016e, fjbuilsRepeatedBehaviorDeliberate2024}, 18 cases (25\%) were between 18 and 25 years of age \cite{Akay_2015f, Ali_2017, Atayan_2016, Bhattacharjee_2008, Csaky_1998e, Kar_2015, Kariholu_2008, Kobiela_2015, Losanoff_1996, Losanoff_1997e, Mesfin_2022a, Peixoto_2017f, Sobnach_2011f, Tupesis_2004f, Yasin_2009}, 13 cases (18\%) were under 18 years of age \cite{AlShaaibi_2021b, Ali_2020f, Cauchi_2002, DivsalarP._2023a, Goldman_1998f, Liu_2005, Naji_2012f, Ohno_2005, Tanrikulu_2015e, Tay_2004, Wildhaber_2005}, 11 cases (15\%) were between 41 and 60 years of age \cite{Al-Faham_2020k, Bhumi_2024f, CamachoDorado_2018, Emamhadi_2018, Hardy_2023g, Jehangir_2019h, Kumar_2001, Sultan_2024f, Thapa_2019f, Wadhwa_2015e, teWildt_2010}, 3 cases (4\%) were over 60 years of age \cite{Beecroft_1998, Kerestes_2019, Li_2013}, 2 cases (3\%) had no age documented \cite{Berry_2021e}. \paragraph*{Population} 36 cases (50\%) had a psychiatric history \cite{AlShaaibi_2021b, Alao_2006i, Ali_2020f, Apikotoa_2022f, Ataya_2013, Atayan_2016, Beecroft_1998, CamachoDorado_2018, Chang_2017f, DelgadoSalazar_2020c, DivsalarP._2023a, Farhadi_2024h, Fry_2010, Guinan_2019f, Hardy_2023g, Jehangir_2019h, Jin_2023, Kar_2015, Kerestes_2019, Kobiela_2015, Kumar_2001, Kumar_2019f, Liu_2005, Mesfin_2022a, Misra_2013, Ohno_2005, Peixoto_2017f, Sakellaridis_2008f, Sultan_2024f, Tammana_2012j, Tanrikulu_2015e, Yildiz_2016e, fjbuilsRepeatedBehaviorDeliberate2024, teWildt_2010}, 19 cases (26\%) had ingested previously \cite{Alao_2006i, Apikotoa_2022f, Berry_2021e, Bhattacharjee_2008, Csaky_1998e, DivsalarP._2023a, Emamhadi_2018, Guinan_2019f, Jehangir_2019h, Jin_2023, Liu_2005, Sakellaridis_2008f, Tanrikulu_2015e, Thapa_2019f, Yildiz_2016e, fjbuilsRepeatedBehaviorDeliberate2024, teWildt_2010}, 12 cases (17\%) were detained persons \cite{Alao_2006i, Ali_2022g, Apikotoa_2022f, Losanoff_1996, Losanoff_1997e, Qureshi_2016, Tammana_2012j, Trgo_2012f}, 7 cases (10\%) were severely disabled \cite{Atayan_2016, Kerestes_2019, Liu_2005, Ohno_2005, Peixoto_2017f, Yildiz_2016e, teWildt_2010}, 4 cases (6\%) were psychiatric inpatients \cite{DivsalarP._2023a, fjbuilsRepeatedBehaviorDeliberate2024, teWildt_2010}, 3 cases (4\%) were under the influence of alcohol \cite{Benoist_2019e, Csaky_1998e, Thapa_2019f}, 2 cases (3\%) were displaced people \cite{Akay_2015f, Gardner_2017h}. \paragraph*{Motivation} 34 cases (47\%) had a psychiatric motivation \cite{Al-Faham_2020k, Alao_2006i, Ali_2020f, Apikotoa_2022f, Ataya_2013, Atayan_2016, Bhasin_2014, Bhattacharjee_2008, DelgadoSalazar_2020c, DivsalarP._2023a, Emamhadi_2018, Farhadi_2024h, Guinan_2019f, Hardy_2023g, Jehangir_2019h, Jin_2023, Kar_2015, Kariholu_2008, Kerestes_2019, Kobiela_2015, Kumar_2001, Kumar_2019f, Li_2013, Liu_2005, Misra_2013, Ohno_2005, Sakellaridis_2008f, Sultan_2024f, Tammana_2012j, Tanrikulu_2015e, Yasin_2009, teWildt_2010}, 21 cases (29\%) were motivated by self-harm intention \cite{Al-Faham_2020k, AlShaaibi_2021b, Alao_2006i, Ali_2017, CamachoDorado_2018, Chang_2017f, Cox_2007, Csaky_1998e, Fry_2010, Li_2013, Losanoff_1996, Losanoff_1997e, Mesfin_2022a, Sakellaridis_2008f, Tammana_2012j, Tanrikulu_2015e, fjbuilsRepeatedBehaviorDeliberate2024}, 17 cases (24\%) had a psychosocial motivation \cite{Akay_2015f, Benoist_2019e, Bhattacharjee_2008, Cauchi_2002, Goldman_1998f, Hardy_2023g, Kobiela_2015, Li_2013, Naji_2012f, Qureshi_2016, Riva_2018j, Sobnach_2011f, Tay_2004, Thapa_2019f, Tupesis_2004f, Wildhaber_2005, Wnęk_2015f}, 9 cases (12\%) were motivated by protest \cite{Bhumi_2024f, Gardner_2017h, Losanoff_1996, Losanoff_1997e, Tupesis_2004f}, 9 cases (12\%) had another documented motivation \cite{Ali_2020f, Ali_2022g, Emamhadi_2018, Guinan_2019f, Peixoto_2017f, Sakellaridis_2008f, Trgo_2012f, Wadhwa_2015e, Yildiz_2016e}. \paragraph*{Object Characteristics} 51 cases (71\%) ingested a large diameter object (\textgreater{}2.5cm) \cite{Akay_2015f, Al-Faham_2020k, AlShaaibi_2021b, Alao_2006i, Ali_2017, Ali_2022g, Apikotoa_2022f, Atayan_2016, Berry_2021e, Bhasin_2014, CamachoDorado_2018, Cauchi_2002, Chang_2017f, Cox_2007, Csaky_1998e, DivsalarP._2023a, Emamhadi_2018, Gardner_2017h, Guinan_2019f, Jehangir_2019h, Jin_2023, Kariholu_2008, Kerestes_2019, Kobiela_2015, Kumar_2001, Kumar_2019f, Losanoff_1996, Losanoff_1997e, Mesfin_2022a, Misra_2013, Naji_2012f, Ohno_2005, Peixoto_2017f, Qureshi_2016, Riva_2018j, Sakellaridis_2008f, Sultan_2024f, Tanrikulu_2015e, Thapa_2019f, Trgo_2012f, Wnęk_2015f, Yildiz_2016e, fjbuilsRepeatedBehaviorDeliberate2024, teWildt_2010}, 44 cases (61\%) ingested multiple objects \cite{Ali_2020f, Apikotoa_2022f, Ataya_2013, Atayan_2016, Beecroft_1998, Bhattacharjee_2008, Bhumi_2024f, CamachoDorado_2018, Cauchi_2002, Emamhadi_2018, Farhadi_2024h, Fry_2010, Goldman_1998f, Guinan_2019f, Hardy_2023g, Jehangir_2019h, Jin_2023, Kar_2015, Kariholu_2008, Kobiela_2015, Kumar_2001, Kumar_2019f, Li_2013, Liu_2005, Losanoff_1996, Mesfin_2022a, Misra_2013, Naji_2012f, Ohno_2005, Sobnach_2011f, Sultan_2024f, Tammana_2012j, Tanrikulu_2015e, Tay_2004, Thapa_2019f, Wadhwa_2015e, Wildhaber_2005, Yasin_2009, fjbuilsRepeatedBehaviorDeliberate2024, teWildt_2010}, 34 cases (47\%) ingested a sharp object \cite{AlShaaibi_2021b, Alao_2006i, Apikotoa_2022f, Ataya_2013, Benoist_2019e, Bhasin_2014, Bhattacharjee_2008, CamachoDorado_2018, Csaky_1998e, DelgadoSalazar_2020c, DivsalarP._2023a, Emamhadi_2018, Farhadi_2024h, Fry_2010, Guinan_2019f, Hardy_2023g, Jehangir_2019h, Jin_2023, Kariholu_2008, Kobiela_2015, Kumar_2019f, Losanoff_1996, Losanoff_1997e, Mesfin_2022a, Misra_2013, Sobnach_2011f, Yasin_2009, teWildt_2010}, 32 cases (44\%) ingested a long object (\textgreater{}5cm) \cite{Al-Faham_2020k, AlShaaibi_2021b, Ali_2017, Ali_2022g, Atayan_2016, Bhasin_2014, CamachoDorado_2018, Chang_2017f, Cox_2007, Csaky_1998e, DivsalarP._2023a, Emamhadi_2018, Fry_2010, Gardner_2017h, Jin_2023, Kariholu_2008, Kerestes_2019, Kobiela_2015, Kumar_2019f, Mesfin_2022a, Misra_2013, Ohno_2005, Qureshi_2016, Sakellaridis_2008f, Sultan_2024f, Thapa_2019f, Trgo_2012f, Yasin_2009, Yildiz_2016e, teWildt_2010}, 9 cases (12\%) ingested a magnet \cite{Ali_2020f, Bhumi_2024f, Cauchi_2002, Liu_2005, Naji_2012f, Ohno_2005, Tanrikulu_2015e, Tay_2004, Wildhaber_2005}, 2 cases (3\%) ingested a button battery \cite{Berry_2021e, Bhumi_2024f}. \paragraph*{Outcomes} 48 cases (67\%) experienced a complication \cite{Ali_2017, Ali_2020f, Apikotoa_2022f, Atayan_2016, Beecroft_1998, Benoist_2019e, Berry_2021e, Bhasin_2014, Bhumi_2024f, CamachoDorado_2018, Cauchi_2002, Cox_2007, Csaky_1998e, DelgadoSalazar_2020c, DivsalarP._2023a, Emamhadi_2018, Farhadi_2024h, Fry_2010, Gardner_2017h, Goldman_1998f, Jin_2023, Kariholu_2008, Kerestes_2019, Kobiela_2015, Kumar_2001, Kumar_2019f, Liu_2005, Losanoff_1996, Mesfin_2022a, Misra_2013, Naji_2012f, Ohno_2005, Sakellaridis_2008f, Sobnach_2011f, Sultan_2024f, Tanrikulu_2015e, Tay_2004, Thapa_2019f, Trgo_2012f, Tupesis_2004f, Wildhaber_2005, Wnęk_2015f, Yasin_2009, Yildiz_2016e}, 44 cases (61\%) underwent surgery \cite{Al-Faham_2020k, AlShaaibi_2021b, Alao_2006i, Ali_2017, Ali_2020f, Atayan_2016, Beecroft_1998, Bhasin_2014, CamachoDorado_2018, Cauchi_2002, Chang_2017f, Cox_2007, Csaky_1998e, DelgadoSalazar_2020c, DivsalarP._2023a, Farhadi_2024h, Fry_2010, Gardner_2017h, Jin_2023, Kariholu_2008, Kerestes_2019, Kobiela_2015, Kumar_2019f, Liu_2005, Losanoff_1996, Losanoff_1997e, Mesfin_2022a, Misra_2013, Naji_2012f, Sobnach_2011f, Tanrikulu_2015e, Tay_2004, Thapa_2019f, Tupesis_2004f, Wildhaber_2005, Wnęk_2015f, Yasin_2009, Yildiz_2016e, fjbuilsRepeatedBehaviorDeliberate2024}, 31 cases (43\%) underwent endoscopy \cite{Akay_2015f, Ali_2022g, Apikotoa_2022f, Atayan_2016, Benoist_2019e, Berry_2021e, Bhasin_2014, Bhumi_2024f, CamachoDorado_2018, Chang_2017f, DelgadoSalazar_2020c, Gardner_2017h, Guinan_2019f, Hardy_2023g, Jehangir_2019h, Kariholu_2008, Li_2013, Liu_2005, Ohno_2005, Peixoto_2017f, Qureshi_2016, Riva_2018j, Sakellaridis_2008f, Sultan_2024f, Tammana_2012j, Tanrikulu_2015e, Trgo_2012f, Wadhwa_2015e, Wnęk_2015f, teWildt_2010}, 7 cases (10\%) were managed conservatively \cite{Ataya_2013, Bhattacharjee_2008, DivsalarP._2023a, Emamhadi_2018, Goldman_1998f, Kar_2015, Kumar_2001}, 2 cases (3\%) died \cite{Emamhadi_2018, Kumar_2001}. All 90 were male gender. 90 cases (100\%) were detained at the time of ingestion \cite{Elghali_2016, Karp_1991b, Lee_2007}, 88 cases (98\%) were intentional ingestions \cite{Elghali_2016, Karp_1991b, Lee_2007}, 30 cases (33\%) had a psychiatric history documented \cite{Elghali_2016, Karp_1991b, Lee_2007}, 2 cases (2\%) had a history of prior ingestion \cite{Elghali_2016}. No cases were reported for were psychiatric inpatients, were displaced people, were under the influence of alcohol at the time of ingestion, and had a severe disability history.
\paragraph*{Motivation}  70 cases (78\%) reported protest motivation \cite{Elghali_2016, Karp_1991b, Lee_2007}, 12 cases (13\%) reported psychiatric motivation \cite{Karp_1991b}, 6 cases (7\%) reported self-harm motivation \cite{Elghali_2016, Karp_1991b}. No cases were reported for psychosocial motivation and other motivation.
\paragraph*{Object Characteristics}  68 cases (76\%) involved sharp object ingestion \cite{Elghali_2016, Karp_1991b, Lee_2007}, 32 cases (36\%) involved long (\textgreater 5cm) object ingestion \cite{Lee_2007}, 25 cases (28\%) involved ingestion of multiple objects \cite{Elghali_2016, Lee_2007}. No cases were reported for button battery ingestion, magnet ingestion, and involved large diameter (\textgreater 2.5cm) object ingestion.
\paragraph*{Outcomes}  47 cases (52\%) underwent endoscopic intervention \cite{Elghali_2016, Lee_2007}, 29 cases (32\%) were managed conservatively \cite{Elghali_2016, Karp_1991b}, 15 cases (17\%) underwent surgical intervention \cite{Elghali_2016, Karp_1991b, Lee_2007}, 6 cases (7\%) reported complications \cite{Lee_2007}, 1 case (1\%) died \cite{Elghali_2016}.
\paragraph*{Geographical Location}Cases were recorded in 33 countries: 13 cases from USA \cite{Alao_2006i, Ataya_2013, Bhumi_2024f, Fry_2010, Guinan_2019f, Hardy_2023g, Jehangir_2019h, Kerestes_2019, Kumar_2001, Liu_2005, Tammana_2012j, Tay_2004, Tupesis_2004f}; 7 cases from India \cite{Bhasin_2014, Bhattacharjee_2008, Kar_2015, Kariholu_2008, Kumar_2019f, Misra_2013, Wadhwa_2015e} and UK \cite{Beecroft_1998, Berry_2021e, Cauchi_2002, Cox_2007, Gardner_2017h, Qureshi_2016}; 6 cases from Bulgaria \cite{Losanoff_1996, Losanoff_1997e}; 5 cases from Iran \cite{DivsalarP._2023a, Emamhadi_2018, Farhadi_2024h}; 4 cases from Turkey \cite{Akay_2015f, Atayan_2016, Tanrikulu_2015e, Yildiz_2016e}; 2 cases from China \cite{Jin_2023, Li_2013}, Poland \cite{Kobiela_2015, Wnęk_2015f}, and Spain \cite{CamachoDorado_2018, fjbuilsRepeatedBehaviorDeliberate2024}; 1 case from Australia \cite{Apikotoa_2022f}, Bahrain \cite{Ali_2020f}, Croatia \cite{Trgo_2012f}, Ecuador \cite{DelgadoSalazar_2020c}, Egypt \cite{Ali_2022g}, Ethiopia \cite{Mesfin_2022a}, Germany \cite{teWildt_2010}, Greece \cite{Sakellaridis_2008f}, Hungary \cite{Csaky_1998e}, Iraq \cite{Al-Faham_2020k}, Israel \cite{Goldman_1998f}, Italy \cite{Riva_2018j}, Japan \cite{Ohno_2005}, Nepal \cite{Thapa_2019f}, Netherlands \cite{Benoist_2019e}, Oman \cite{AlShaaibi_2021b}, Pakistan \cite{Yasin_2009}, Portugal \cite{Peixoto_2017f}, Qatar \cite{Ali_2017}, Saudi Arabia \cite{Sultan_2024f}, South Africa \cite{Sobnach_2011f}, Sweden \cite{Naji_2012f}, Switzerland \cite{Wildhaber_2005}, and Taiwan \cite{Chang_2017f}. \paragraph*{Gender} 43 cases (60\%) were male \cite{Akay_2015f, Al-Faham_2020k, Alao_2006i, Ali_2017, Ali_2022g, Apikotoa_2022f, Atayan_2016, Benoist_2019e, Berry_2021e, Bhumi_2024f, CamachoDorado_2018, Csaky_1998e, Emamhadi_2018, Farhadi_2024h, Fry_2010, Gardner_2017h, Guinan_2019f, Jehangir_2019h, Jin_2023, Kobiela_2015, Kumar_2001, Kumar_2019f, Liu_2005, Losanoff_1996, Losanoff_1997e, Mesfin_2022a, Misra_2013, Qureshi_2016, Riva_2018j, Sobnach_2011f, Tammana_2012j, Tanrikulu_2015e, Tay_2004, Thapa_2019f, Trgo_2012f, Wadhwa_2015e, Yasin_2009, teWildt_2010}, 28 cases (39\%) were female \cite{AlShaaibi_2021b, Ali_2020f, Ataya_2013, Beecroft_1998, Bhasin_2014, Bhattacharjee_2008, Cauchi_2002, Chang_2017f, Cox_2007, DelgadoSalazar_2020c, DivsalarP._2023a, Goldman_1998f, Hardy_2023g, Kar_2015, Kariholu_2008, Kerestes_2019, Li_2013, Naji_2012f, Ohno_2005, Peixoto_2017f, Sakellaridis_2008f, Sultan_2024f, Tupesis_2004f, Wildhaber_2005, Wnęk_2015f, Yildiz_2016e}, 1 case (1\%) had no gender recorded \cite{fjbuilsRepeatedBehaviorDeliberate2024}. \paragraph*{Age Group} 25 cases (35\%) were between 26 and 40 years of age \cite{Alao_2006i, Ali_2022g, Apikotoa_2022f, Ataya_2013, Benoist_2019e, Bhasin_2014, Chang_2017f, Cox_2007, DelgadoSalazar_2020c, Farhadi_2024h, Fry_2010, Gardner_2017h, Guinan_2019f, Jin_2023, Kumar_2019f, Losanoff_1996, Misra_2013, Qureshi_2016, Riva_2018j, Sakellaridis_2008f, Tammana_2012j, Trgo_2012f, Wnęk_2015f, Yildiz_2016e, fjbuilsRepeatedBehaviorDeliberate2024}, 18 cases (25\%) were between 18 and 25 years of age \cite{Akay_2015f, Ali_2017, Atayan_2016, Bhattacharjee_2008, Csaky_1998e, Kar_2015, Kariholu_2008, Kobiela_2015, Losanoff_1996, Losanoff_1997e, Mesfin_2022a, Peixoto_2017f, Sobnach_2011f, Tupesis_2004f, Yasin_2009}, 13 cases (18\%) were under 18 years of age \cite{AlShaaibi_2021b, Ali_2020f, Cauchi_2002, DivsalarP._2023a, Goldman_1998f, Liu_2005, Naji_2012f, Ohno_2005, Tanrikulu_2015e, Tay_2004, Wildhaber_2005}, 11 cases (15\%) were between 41 and 60 years of age \cite{Al-Faham_2020k, Bhumi_2024f, CamachoDorado_2018, Emamhadi_2018, Hardy_2023g, Jehangir_2019h, Kumar_2001, Sultan_2024f, Thapa_2019f, Wadhwa_2015e, teWildt_2010}, 3 cases (4\%) were over 60 years of age \cite{Beecroft_1998, Kerestes_2019, Li_2013}, 2 cases (3\%) had no age documented \cite{Berry_2021e}. \paragraph*{Population} 36 cases (50\%) had a psychiatric history \cite{AlShaaibi_2021b, Alao_2006i, Ali_2020f, Apikotoa_2022f, Ataya_2013, Atayan_2016, Beecroft_1998, CamachoDorado_2018, Chang_2017f, DelgadoSalazar_2020c, DivsalarP._2023a, Farhadi_2024h, Fry_2010, Guinan_2019f, Hardy_2023g, Jehangir_2019h, Jin_2023, Kar_2015, Kerestes_2019, Kobiela_2015, Kumar_2001, Kumar_2019f, Liu_2005, Mesfin_2022a, Misra_2013, Ohno_2005, Peixoto_2017f, Sakellaridis_2008f, Sultan_2024f, Tammana_2012j, Tanrikulu_2015e, Yildiz_2016e, fjbuilsRepeatedBehaviorDeliberate2024, teWildt_2010}, 19 cases (26\%) had ingested previously \cite{Alao_2006i, Apikotoa_2022f, Berry_2021e, Bhattacharjee_2008, Csaky_1998e, DivsalarP._2023a, Emamhadi_2018, Guinan_2019f, Jehangir_2019h, Jin_2023, Liu_2005, Sakellaridis_2008f, Tanrikulu_2015e, Thapa_2019f, Yildiz_2016e, fjbuilsRepeatedBehaviorDeliberate2024, teWildt_2010}, 12 cases (17\%) were detained persons \cite{Alao_2006i, Ali_2022g, Apikotoa_2022f, Losanoff_1996, Losanoff_1997e, Qureshi_2016, Tammana_2012j, Trgo_2012f}, 7 cases (10\%) were severely disabled \cite{Atayan_2016, Kerestes_2019, Liu_2005, Ohno_2005, Peixoto_2017f, Yildiz_2016e, teWildt_2010}, 4 cases (6\%) were psychiatric inpatients \cite{DivsalarP._2023a, fjbuilsRepeatedBehaviorDeliberate2024, teWildt_2010}, 3 cases (4\%) were under the influence of alcohol \cite{Benoist_2019e, Csaky_1998e, Thapa_2019f}, 2 cases (3\%) were displaced people \cite{Akay_2015f, Gardner_2017h}. \paragraph*{Motivation} 34 cases (47\%) had a psychiatric motivation \cite{Al-Faham_2020k, Alao_2006i, Ali_2020f, Apikotoa_2022f, Ataya_2013, Atayan_2016, Bhasin_2014, Bhattacharjee_2008, DelgadoSalazar_2020c, DivsalarP._2023a, Emamhadi_2018, Farhadi_2024h, Guinan_2019f, Hardy_2023g, Jehangir_2019h, Jin_2023, Kar_2015, Kariholu_2008, Kerestes_2019, Kobiela_2015, Kumar_2001, Kumar_2019f, Li_2013, Liu_2005, Misra_2013, Ohno_2005, Sakellaridis_2008f, Sultan_2024f, Tammana_2012j, Tanrikulu_2015e, Yasin_2009, teWildt_2010}, 21 cases (29\%) were motivated by self-harm intention \cite{Al-Faham_2020k, AlShaaibi_2021b, Alao_2006i, Ali_2017, CamachoDorado_2018, Chang_2017f, Cox_2007, Csaky_1998e, Fry_2010, Li_2013, Losanoff_1996, Losanoff_1997e, Mesfin_2022a, Sakellaridis_2008f, Tammana_2012j, Tanrikulu_2015e, fjbuilsRepeatedBehaviorDeliberate2024}, 17 cases (24\%) had a psychosocial motivation \cite{Akay_2015f, Benoist_2019e, Bhattacharjee_2008, Cauchi_2002, Goldman_1998f, Hardy_2023g, Kobiela_2015, Li_2013, Naji_2012f, Qureshi_2016, Riva_2018j, Sobnach_2011f, Tay_2004, Thapa_2019f, Tupesis_2004f, Wildhaber_2005, Wnęk_2015f}, 9 cases (12\%) were motivated by protest \cite{Bhumi_2024f, Gardner_2017h, Losanoff_1996, Losanoff_1997e, Tupesis_2004f}, 9 cases (12\%) had another documented motivation \cite{Ali_2020f, Ali_2022g, Emamhadi_2018, Guinan_2019f, Peixoto_2017f, Sakellaridis_2008f, Trgo_2012f, Wadhwa_2015e, Yildiz_2016e}. \paragraph*{Object Characteristics} 51 cases (71\%) ingested a large diameter object (\textgreater{}2.5cm) \cite{Akay_2015f, Al-Faham_2020k, AlShaaibi_2021b, Alao_2006i, Ali_2017, Ali_2022g, Apikotoa_2022f, Atayan_2016, Berry_2021e, Bhasin_2014, CamachoDorado_2018, Cauchi_2002, Chang_2017f, Cox_2007, Csaky_1998e, DivsalarP._2023a, Emamhadi_2018, Gardner_2017h, Guinan_2019f, Jehangir_2019h, Jin_2023, Kariholu_2008, Kerestes_2019, Kobiela_2015, Kumar_2001, Kumar_2019f, Losanoff_1996, Losanoff_1997e, Mesfin_2022a, Misra_2013, Naji_2012f, Ohno_2005, Peixoto_2017f, Qureshi_2016, Riva_2018j, Sakellaridis_2008f, Sultan_2024f, Tanrikulu_2015e, Thapa_2019f, Trgo_2012f, Wnęk_2015f, Yildiz_2016e, fjbuilsRepeatedBehaviorDeliberate2024, teWildt_2010}, 44 cases (61\%) ingested multiple objects \cite{Ali_2020f, Apikotoa_2022f, Ataya_2013, Atayan_2016, Beecroft_1998, Bhattacharjee_2008, Bhumi_2024f, CamachoDorado_2018, Cauchi_2002, Emamhadi_2018, Farhadi_2024h, Fry_2010, Goldman_1998f, Guinan_2019f, Hardy_2023g, Jehangir_2019h, Jin_2023, Kar_2015, Kariholu_2008, Kobiela_2015, Kumar_2001, Kumar_2019f, Li_2013, Liu_2005, Losanoff_1996, Mesfin_2022a, Misra_2013, Naji_2012f, Ohno_2005, Sobnach_2011f, Sultan_2024f, Tammana_2012j, Tanrikulu_2015e, Tay_2004, Thapa_2019f, Wadhwa_2015e, Wildhaber_2005, Yasin_2009, fjbuilsRepeatedBehaviorDeliberate2024, teWildt_2010}, 34 cases (47\%) ingested a sharp object \cite{AlShaaibi_2021b, Alao_2006i, Apikotoa_2022f, Ataya_2013, Benoist_2019e, Bhasin_2014, Bhattacharjee_2008, CamachoDorado_2018, Csaky_1998e, DelgadoSalazar_2020c, DivsalarP._2023a, Emamhadi_2018, Farhadi_2024h, Fry_2010, Guinan_2019f, Hardy_2023g, Jehangir_2019h, Jin_2023, Kariholu_2008, Kobiela_2015, Kumar_2019f, Losanoff_1996, Losanoff_1997e, Mesfin_2022a, Misra_2013, Sobnach_2011f, Yasin_2009, teWildt_2010}, 32 cases (44\%) ingested a long object (\textgreater{}5cm) \cite{Al-Faham_2020k, AlShaaibi_2021b, Ali_2017, Ali_2022g, Atayan_2016, Bhasin_2014, CamachoDorado_2018, Chang_2017f, Cox_2007, Csaky_1998e, DivsalarP._2023a, Emamhadi_2018, Fry_2010, Gardner_2017h, Jin_2023, Kariholu_2008, Kerestes_2019, Kobiela_2015, Kumar_2019f, Mesfin_2022a, Misra_2013, Ohno_2005, Qureshi_2016, Sakellaridis_2008f, Sultan_2024f, Thapa_2019f, Trgo_2012f, Yasin_2009, Yildiz_2016e, teWildt_2010}, 9 cases (12\%) ingested a magnet \cite{Ali_2020f, Bhumi_2024f, Cauchi_2002, Liu_2005, Naji_2012f, Ohno_2005, Tanrikulu_2015e, Tay_2004, Wildhaber_2005}, 2 cases (3\%) ingested a button battery \cite{Berry_2021e, Bhumi_2024f}. \paragraph*{Outcomes} 48 cases (67\%) experienced a complication \cite{Ali_2017, Ali_2020f, Apikotoa_2022f, Atayan_2016, Beecroft_1998, Benoist_2019e, Berry_2021e, Bhasin_2014, Bhumi_2024f, CamachoDorado_2018, Cauchi_2002, Cox_2007, Csaky_1998e, DelgadoSalazar_2020c, DivsalarP._2023a, Emamhadi_2018, Farhadi_2024h, Fry_2010, Gardner_2017h, Goldman_1998f, Jin_2023, Kariholu_2008, Kerestes_2019, Kobiela_2015, Kumar_2001, Kumar_2019f, Liu_2005, Losanoff_1996, Mesfin_2022a, Misra_2013, Naji_2012f, Ohno_2005, Sakellaridis_2008f, Sobnach_2011f, Sultan_2024f, Tanrikulu_2015e, Tay_2004, Thapa_2019f, Trgo_2012f, Tupesis_2004f, Wildhaber_2005, Wnęk_2015f, Yasin_2009, Yildiz_2016e}, 44 cases (61\%) underwent surgery \cite{Al-Faham_2020k, AlShaaibi_2021b, Alao_2006i, Ali_2017, Ali_2020f, Atayan_2016, Beecroft_1998, Bhasin_2014, CamachoDorado_2018, Cauchi_2002, Chang_2017f, Cox_2007, Csaky_1998e, DelgadoSalazar_2020c, DivsalarP._2023a, Farhadi_2024h, Fry_2010, Gardner_2017h, Jin_2023, Kariholu_2008, Kerestes_2019, Kobiela_2015, Kumar_2019f, Liu_2005, Losanoff_1996, Losanoff_1997e, Mesfin_2022a, Misra_2013, Naji_2012f, Sobnach_2011f, Tanrikulu_2015e, Tay_2004, Thapa_2019f, Tupesis_2004f, Wildhaber_2005, Wnęk_2015f, Yasin_2009, Yildiz_2016e, fjbuilsRepeatedBehaviorDeliberate2024}, 31 cases (43\%) underwent endoscopy \cite{Akay_2015f, Ali_2022g, Apikotoa_2022f, Atayan_2016, Benoist_2019e, Berry_2021e, Bhasin_2014, Bhumi_2024f, CamachoDorado_2018, Chang_2017f, DelgadoSalazar_2020c, Gardner_2017h, Guinan_2019f, Hardy_2023g, Jehangir_2019h, Kariholu_2008, Li_2013, Liu_2005, Ohno_2005, Peixoto_2017f, Qureshi_2016, Riva_2018j, Sakellaridis_2008f, Sultan_2024f, Tammana_2012j, Tanrikulu_2015e, Trgo_2012f, Wadhwa_2015e, Wnęk_2015f, teWildt_2010}, 7 cases (10\%) were managed conservatively \cite{Ataya_2013, Bhattacharjee_2008, DivsalarP._2023a, Emamhadi_2018, Goldman_1998f, Kar_2015, Kumar_2001}, 2 cases (3\%) died \cite{Emamhadi_2018, Kumar_2001}. All 90 were male gender. 90 cases (100\%) were detained at the time of ingestion \cite{Elghali_2016, Karp_1991b, Lee_2007}, 88 cases (98\%) were intentional ingestions \cite{Elghali_2016, Karp_1991b, Lee_2007}, 30 cases (33\%) had a psychiatric history documented \cite{Elghali_2016, Karp_1991b, Lee_2007}, 2 cases (2\%) had a history of prior ingestion \cite{Elghali_2016}. No cases were reported for were psychiatric inpatients, were displaced people, were under the influence of alcohol at the time of ingestion, and had a severe disability history.
\paragraph*{Motivation}  70 cases (78\%) reported protest motivation \cite{Elghali_2016, Karp_1991b, Lee_2007}, 12 cases (13\%) reported psychiatric motivation \cite{Karp_1991b}, 6 cases (7\%) reported self-harm motivation \cite{Elghali_2016, Karp_1991b}. No cases were reported for psychosocial motivation and other motivation.
\paragraph*{Object Characteristics}  68 cases (76\%) involved sharp object ingestion \cite{Elghali_2016, Karp_1991b, Lee_2007}, 32 cases (36\%) involved long (\textgreater 5cm) object ingestion \cite{Lee_2007}, 25 cases (28\%) involved ingestion of multiple objects \cite{Elghali_2016, Lee_2007}. No cases were reported for button battery ingestion, magnet ingestion, and involved large diameter (\textgreater 2.5cm) object ingestion.
\paragraph*{Outcomes}  47 cases (52\%) underwent endoscopic intervention \cite{Elghali_2016, Lee_2007}, 29 cases (32\%) were managed conservatively \cite{Elghali_2016, Karp_1991b}, 15 cases (17\%) underwent surgical intervention \cite{Elghali_2016, Karp_1991b, Lee_2007}, 6 cases (7\%) reported complications \cite{Lee_2007}, 1 case (1\%) died \cite{Elghali_2016}.
\paragraph*{Geographical Location}Cases were recorded in 33 countries: 13 cases from USA \cite{Alao_2006i, Ataya_2013, Bhumi_2024f, Fry_2010, Guinan_2019f, Hardy_2023g, Jehangir_2019h, Kerestes_2019, Kumar_2001, Liu_2005, Tammana_2012j, Tay_2004, Tupesis_2004f}; 7 cases from India \cite{Bhasin_2014, Bhattacharjee_2008, Kar_2015, Kariholu_2008, Kumar_2019f, Misra_2013, Wadhwa_2015e} and UK \cite{Beecroft_1998, Berry_2021e, Cauchi_2002, Cox_2007, Gardner_2017h, Qureshi_2016}; 6 cases from Bulgaria \cite{Losanoff_1996, Losanoff_1997e}; 5 cases from Iran \cite{DivsalarP._2023a, Emamhadi_2018, Farhadi_2024h}; 4 cases from Turkey \cite{Akay_2015f, Atayan_2016, Tanrikulu_2015e, Yildiz_2016e}; 2 cases from China \cite{Jin_2023, Li_2013}, Poland \cite{Kobiela_2015, Wnęk_2015f}, and Spain \cite{CamachoDorado_2018, fjbuilsRepeatedBehaviorDeliberate2024}; 1 case from Australia \cite{Apikotoa_2022f}, Bahrain \cite{Ali_2020f}, Croatia \cite{Trgo_2012f}, Ecuador \cite{DelgadoSalazar_2020c}, Egypt \cite{Ali_2022g}, Ethiopia \cite{Mesfin_2022a}, Germany \cite{teWildt_2010}, Greece \cite{Sakellaridis_2008f}, Hungary \cite{Csaky_1998e}, Iraq \cite{Al-Faham_2020k}, Israel \cite{Goldman_1998f}, Italy \cite{Riva_2018j}, Japan \cite{Ohno_2005}, Nepal \cite{Thapa_2019f}, Netherlands \cite{Benoist_2019e}, Oman \cite{AlShaaibi_2021b}, Pakistan \cite{Yasin_2009}, Portugal \cite{Peixoto_2017f}, Qatar \cite{Ali_2017}, Saudi Arabia \cite{Sultan_2024f}, South Africa \cite{Sobnach_2011f}, Sweden \cite{Naji_2012f}, Switzerland \cite{Wildhaber_2005}, and Taiwan \cite{Chang_2017f}. \paragraph*{Gender} 43 cases (60\%) were male \cite{Akay_2015f, Al-Faham_2020k, Alao_2006i, Ali_2017, Ali_2022g, Apikotoa_2022f, Atayan_2016, Benoist_2019e, Berry_2021e, Bhumi_2024f, CamachoDorado_2018, Csaky_1998e, Emamhadi_2018, Farhadi_2024h, Fry_2010, Gardner_2017h, Guinan_2019f, Jehangir_2019h, Jin_2023, Kobiela_2015, Kumar_2001, Kumar_2019f, Liu_2005, Losanoff_1996, Losanoff_1997e, Mesfin_2022a, Misra_2013, Qureshi_2016, Riva_2018j, Sobnach_2011f, Tammana_2012j, Tanrikulu_2015e, Tay_2004, Thapa_2019f, Trgo_2012f, Wadhwa_2015e, Yasin_2009, teWildt_2010}, 28 cases (39\%) were female \cite{AlShaaibi_2021b, Ali_2020f, Ataya_2013, Beecroft_1998, Bhasin_2014, Bhattacharjee_2008, Cauchi_2002, Chang_2017f, Cox_2007, DelgadoSalazar_2020c, DivsalarP._2023a, Goldman_1998f, Hardy_2023g, Kar_2015, Kariholu_2008, Kerestes_2019, Li_2013, Naji_2012f, Ohno_2005, Peixoto_2017f, Sakellaridis_2008f, Sultan_2024f, Tupesis_2004f, Wildhaber_2005, Wnęk_2015f, Yildiz_2016e}, 1 case (1\%) had no gender recorded \cite{fjbuilsRepeatedBehaviorDeliberate2024}. \paragraph*{Age Group} 25 cases (35\%) were between 26 and 40 years of age \cite{Alao_2006i, Ali_2022g, Apikotoa_2022f, Ataya_2013, Benoist_2019e, Bhasin_2014, Chang_2017f, Cox_2007, DelgadoSalazar_2020c, Farhadi_2024h, Fry_2010, Gardner_2017h, Guinan_2019f, Jin_2023, Kumar_2019f, Losanoff_1996, Misra_2013, Qureshi_2016, Riva_2018j, Sakellaridis_2008f, Tammana_2012j, Trgo_2012f, Wnęk_2015f, Yildiz_2016e, fjbuilsRepeatedBehaviorDeliberate2024}, 18 cases (25\%) were between 18 and 25 years of age \cite{Akay_2015f, Ali_2017, Atayan_2016, Bhattacharjee_2008, Csaky_1998e, Kar_2015, Kariholu_2008, Kobiela_2015, Losanoff_1996, Losanoff_1997e, Mesfin_2022a, Peixoto_2017f, Sobnach_2011f, Tupesis_2004f, Yasin_2009}, 13 cases (18\%) were under 18 years of age \cite{AlShaaibi_2021b, Ali_2020f, Cauchi_2002, DivsalarP._2023a, Goldman_1998f, Liu_2005, Naji_2012f, Ohno_2005, Tanrikulu_2015e, Tay_2004, Wildhaber_2005}, 11 cases (15\%) were between 41 and 60 years of age \cite{Al-Faham_2020k, Bhumi_2024f, CamachoDorado_2018, Emamhadi_2018, Hardy_2023g, Jehangir_2019h, Kumar_2001, Sultan_2024f, Thapa_2019f, Wadhwa_2015e, teWildt_2010}, 3 cases (4\%) were over 60 years of age \cite{Beecroft_1998, Kerestes_2019, Li_2013}, 2 cases (3\%) had no age documented \cite{Berry_2021e}. \paragraph*{Population} 36 cases (50\%) had a psychiatric history \cite{AlShaaibi_2021b, Alao_2006i, Ali_2020f, Apikotoa_2022f, Ataya_2013, Atayan_2016, Beecroft_1998, CamachoDorado_2018, Chang_2017f, DelgadoSalazar_2020c, DivsalarP._2023a, Farhadi_2024h, Fry_2010, Guinan_2019f, Hardy_2023g, Jehangir_2019h, Jin_2023, Kar_2015, Kerestes_2019, Kobiela_2015, Kumar_2001, Kumar_2019f, Liu_2005, Mesfin_2022a, Misra_2013, Ohno_2005, Peixoto_2017f, Sakellaridis_2008f, Sultan_2024f, Tammana_2012j, Tanrikulu_2015e, Yildiz_2016e, fjbuilsRepeatedBehaviorDeliberate2024, teWildt_2010}, 19 cases (26\%) had ingested previously \cite{Alao_2006i, Apikotoa_2022f, Berry_2021e, Bhattacharjee_2008, Csaky_1998e, DivsalarP._2023a, Emamhadi_2018, Guinan_2019f, Jehangir_2019h, Jin_2023, Liu_2005, Sakellaridis_2008f, Tanrikulu_2015e, Thapa_2019f, Yildiz_2016e, fjbuilsRepeatedBehaviorDeliberate2024, teWildt_2010}, 12 cases (17\%) were detained persons \cite{Alao_2006i, Ali_2022g, Apikotoa_2022f, Losanoff_1996, Losanoff_1997e, Qureshi_2016, Tammana_2012j, Trgo_2012f}, 7 cases (10\%) were severely disabled \cite{Atayan_2016, Kerestes_2019, Liu_2005, Ohno_2005, Peixoto_2017f, Yildiz_2016e, teWildt_2010}, 4 cases (6\%) were psychiatric inpatients \cite{DivsalarP._2023a, fjbuilsRepeatedBehaviorDeliberate2024, teWildt_2010}, 3 cases (4\%) were under the influence of alcohol \cite{Benoist_2019e, Csaky_1998e, Thapa_2019f}, 2 cases (3\%) were displaced people \cite{Akay_2015f, Gardner_2017h}. \paragraph*{Motivation} 34 cases (47\%) had a psychiatric motivation \cite{Al-Faham_2020k, Alao_2006i, Ali_2020f, Apikotoa_2022f, Ataya_2013, Atayan_2016, Bhasin_2014, Bhattacharjee_2008, DelgadoSalazar_2020c, DivsalarP._2023a, Emamhadi_2018, Farhadi_2024h, Guinan_2019f, Hardy_2023g, Jehangir_2019h, Jin_2023, Kar_2015, Kariholu_2008, Kerestes_2019, Kobiela_2015, Kumar_2001, Kumar_2019f, Li_2013, Liu_2005, Misra_2013, Ohno_2005, Sakellaridis_2008f, Sultan_2024f, Tammana_2012j, Tanrikulu_2015e, Yasin_2009, teWildt_2010}, 21 cases (29\%) were motivated by self-harm intention \cite{Al-Faham_2020k, AlShaaibi_2021b, Alao_2006i, Ali_2017, CamachoDorado_2018, Chang_2017f, Cox_2007, Csaky_1998e, Fry_2010, Li_2013, Losanoff_1996, Losanoff_1997e, Mesfin_2022a, Sakellaridis_2008f, Tammana_2012j, Tanrikulu_2015e, fjbuilsRepeatedBehaviorDeliberate2024}, 17 cases (24\%) had a psychosocial motivation \cite{Akay_2015f, Benoist_2019e, Bhattacharjee_2008, Cauchi_2002, Goldman_1998f, Hardy_2023g, Kobiela_2015, Li_2013, Naji_2012f, Qureshi_2016, Riva_2018j, Sobnach_2011f, Tay_2004, Thapa_2019f, Tupesis_2004f, Wildhaber_2005, Wnęk_2015f}, 9 cases (12\%) were motivated by protest \cite{Bhumi_2024f, Gardner_2017h, Losanoff_1996, Losanoff_1997e, Tupesis_2004f}, 9 cases (12\%) had another documented motivation \cite{Ali_2020f, Ali_2022g, Emamhadi_2018, Guinan_2019f, Peixoto_2017f, Sakellaridis_2008f, Trgo_2012f, Wadhwa_2015e, Yildiz_2016e}. \paragraph*{Object Characteristics} 51 cases (71\%) ingested a large diameter object (\textgreater{}2.5cm) \cite{Akay_2015f, Al-Faham_2020k, AlShaaibi_2021b, Alao_2006i, Ali_2017, Ali_2022g, Apikotoa_2022f, Atayan_2016, Berry_2021e, Bhasin_2014, CamachoDorado_2018, Cauchi_2002, Chang_2017f, Cox_2007, Csaky_1998e, DivsalarP._2023a, Emamhadi_2018, Gardner_2017h, Guinan_2019f, Jehangir_2019h, Jin_2023, Kariholu_2008, Kerestes_2019, Kobiela_2015, Kumar_2001, Kumar_2019f, Losanoff_1996, Losanoff_1997e, Mesfin_2022a, Misra_2013, Naji_2012f, Ohno_2005, Peixoto_2017f, Qureshi_2016, Riva_2018j, Sakellaridis_2008f, Sultan_2024f, Tanrikulu_2015e, Thapa_2019f, Trgo_2012f, Wnęk_2015f, Yildiz_2016e, fjbuilsRepeatedBehaviorDeliberate2024, teWildt_2010}, 44 cases (61\%) ingested multiple objects \cite{Ali_2020f, Apikotoa_2022f, Ataya_2013, Atayan_2016, Beecroft_1998, Bhattacharjee_2008, Bhumi_2024f, CamachoDorado_2018, Cauchi_2002, Emamhadi_2018, Farhadi_2024h, Fry_2010, Goldman_1998f, Guinan_2019f, Hardy_2023g, Jehangir_2019h, Jin_2023, Kar_2015, Kariholu_2008, Kobiela_2015, Kumar_2001, Kumar_2019f, Li_2013, Liu_2005, Losanoff_1996, Mesfin_2022a, Misra_2013, Naji_2012f, Ohno_2005, Sobnach_2011f, Sultan_2024f, Tammana_2012j, Tanrikulu_2015e, Tay_2004, Thapa_2019f, Wadhwa_2015e, Wildhaber_2005, Yasin_2009, fjbuilsRepeatedBehaviorDeliberate2024, teWildt_2010}, 34 cases (47\%) ingested a sharp object \cite{AlShaaibi_2021b, Alao_2006i, Apikotoa_2022f, Ataya_2013, Benoist_2019e, Bhasin_2014, Bhattacharjee_2008, CamachoDorado_2018, Csaky_1998e, DelgadoSalazar_2020c, DivsalarP._2023a, Emamhadi_2018, Farhadi_2024h, Fry_2010, Guinan_2019f, Hardy_2023g, Jehangir_2019h, Jin_2023, Kariholu_2008, Kobiela_2015, Kumar_2019f, Losanoff_1996, Losanoff_1997e, Mesfin_2022a, Misra_2013, Sobnach_2011f, Yasin_2009, teWildt_2010}, 32 cases (44\%) ingested a long object (\textgreater{}5cm) \cite{Al-Faham_2020k, AlShaaibi_2021b, Ali_2017, Ali_2022g, Atayan_2016, Bhasin_2014, CamachoDorado_2018, Chang_2017f, Cox_2007, Csaky_1998e, DivsalarP._2023a, Emamhadi_2018, Fry_2010, Gardner_2017h, Jin_2023, Kariholu_2008, Kerestes_2019, Kobiela_2015, Kumar_2019f, Mesfin_2022a, Misra_2013, Ohno_2005, Qureshi_2016, Sakellaridis_2008f, Sultan_2024f, Thapa_2019f, Trgo_2012f, Yasin_2009, Yildiz_2016e, teWildt_2010}, 9 cases (12\%) ingested a magnet \cite{Ali_2020f, Bhumi_2024f, Cauchi_2002, Liu_2005, Naji_2012f, Ohno_2005, Tanrikulu_2015e, Tay_2004, Wildhaber_2005}, 2 cases (3\%) ingested a button battery \cite{Berry_2021e, Bhumi_2024f}. \paragraph*{Outcomes} 48 cases (67\%) experienced a complication \cite{Ali_2017, Ali_2020f, Apikotoa_2022f, Atayan_2016, Beecroft_1998, Benoist_2019e, Berry_2021e, Bhasin_2014, Bhumi_2024f, CamachoDorado_2018, Cauchi_2002, Cox_2007, Csaky_1998e, DelgadoSalazar_2020c, DivsalarP._2023a, Emamhadi_2018, Farhadi_2024h, Fry_2010, Gardner_2017h, Goldman_1998f, Jin_2023, Kariholu_2008, Kerestes_2019, Kobiela_2015, Kumar_2001, Kumar_2019f, Liu_2005, Losanoff_1996, Mesfin_2022a, Misra_2013, Naji_2012f, Ohno_2005, Sakellaridis_2008f, Sobnach_2011f, Sultan_2024f, Tanrikulu_2015e, Tay_2004, Thapa_2019f, Trgo_2012f, Tupesis_2004f, Wildhaber_2005, Wnęk_2015f, Yasin_2009, Yildiz_2016e}, 44 cases (61\%) underwent surgery \cite{Al-Faham_2020k, AlShaaibi_2021b, Alao_2006i, Ali_2017, Ali_2020f, Atayan_2016, Beecroft_1998, Bhasin_2014, CamachoDorado_2018, Cauchi_2002, Chang_2017f, Cox_2007, Csaky_1998e, DelgadoSalazar_2020c, DivsalarP._2023a, Farhadi_2024h, Fry_2010, Gardner_2017h, Jin_2023, Kariholu_2008, Kerestes_2019, Kobiela_2015, Kumar_2019f, Liu_2005, Losanoff_1996, Losanoff_1997e, Mesfin_2022a, Misra_2013, Naji_2012f, Sobnach_2011f, Tanrikulu_2015e, Tay_2004, Thapa_2019f, Tupesis_2004f, Wildhaber_2005, Wnęk_2015f, Yasin_2009, Yildiz_2016e, fjbuilsRepeatedBehaviorDeliberate2024}, 31 cases (43\%) underwent endoscopy \cite{Akay_2015f, Ali_2022g, Apikotoa_2022f, Atayan_2016, Benoist_2019e, Berry_2021e, Bhasin_2014, Bhumi_2024f, CamachoDorado_2018, Chang_2017f, DelgadoSalazar_2020c, Gardner_2017h, Guinan_2019f, Hardy_2023g, Jehangir_2019h, Kariholu_2008, Li_2013, Liu_2005, Ohno_2005, Peixoto_2017f, Qureshi_2016, Riva_2018j, Sakellaridis_2008f, Sultan_2024f, Tammana_2012j, Tanrikulu_2015e, Trgo_2012f, Wadhwa_2015e, Wnęk_2015f, teWildt_2010}, 7 cases (10\%) were managed conservatively \cite{Ataya_2013, Bhattacharjee_2008, DivsalarP._2023a, Emamhadi_2018, Goldman_1998f, Kar_2015, Kumar_2001}, 2 cases (3\%) died \cite{Emamhadi_2018, Kumar_2001}. All 90 were male gender. 90 cases (100\%) were detained at the time of ingestion \cite{Elghali_2016, Karp_1991b, Lee_2007}, 88 cases (98\%) were intentional ingestions \cite{Elghali_2016, Karp_1991b, Lee_2007}, 30 cases (33\%) had a psychiatric history documented \cite{Elghali_2016, Karp_1991b, Lee_2007}, 2 cases (2\%) had a history of prior ingestion \cite{Elghali_2016}. No cases were reported for were psychiatric inpatients, were displaced people, were under the influence of alcohol at the time of ingestion, and had a severe disability history.
\paragraph*{Motivation}  70 cases (78\%) reported protest motivation \cite{Elghali_2016, Karp_1991b, Lee_2007}, 12 cases (13\%) reported psychiatric motivation \cite{Karp_1991b}, 6 cases (7\%) reported self-harm motivation \cite{Elghali_2016, Karp_1991b}. No cases were reported for psychosocial motivation and other motivation.
\paragraph*{Object Characteristics}  68 cases (76\%) involved sharp object ingestion \cite{Elghali_2016, Karp_1991b, Lee_2007}, 32 cases (36\%) involved long (\textgreater 5cm) object ingestion \cite{Lee_2007}, 25 cases (28\%) involved ingestion of multiple objects \cite{Elghali_2016, Lee_2007}. No cases were reported for button battery ingestion, magnet ingestion, and involved large diameter (\textgreater 2.5cm) object ingestion.
\paragraph*{Outcomes}  47 cases (52\%) underwent endoscopic intervention \cite{Elghali_2016, Lee_2007}, 29 cases (32\%) were managed conservatively \cite{Elghali_2016, Karp_1991b}, 15 cases (17\%) underwent surgical intervention \cite{Elghali_2016, Karp_1991b, Lee_2007}, 6 cases (7\%) reported complications \cite{Lee_2007}, 1 case (1\%) died \cite{Elghali_2016}.
\paragraph*{Geographical Location}Cases were recorded in 33 countries: 13 cases from USA \cite{Alao_2006i, Ataya_2013, Bhumi_2024f, Fry_2010, Guinan_2019f, Hardy_2023g, Jehangir_2019h, Kerestes_2019, Kumar_2001, Liu_2005, Tammana_2012j, Tay_2004, Tupesis_2004f}; 7 cases from India \cite{Bhasin_2014, Bhattacharjee_2008, Kar_2015, Kariholu_2008, Kumar_2019f, Misra_2013, Wadhwa_2015e} and UK \cite{Beecroft_1998, Berry_2021e, Cauchi_2002, Cox_2007, Gardner_2017h, Qureshi_2016}; 6 cases from Bulgaria \cite{Losanoff_1996, Losanoff_1997e}; 5 cases from Iran \cite{DivsalarP._2023a, Emamhadi_2018, Farhadi_2024h}; 4 cases from Turkey \cite{Akay_2015f, Atayan_2016, Tanrikulu_2015e, Yildiz_2016e}; 2 cases from China \cite{Jin_2023, Li_2013}, Poland \cite{Kobiela_2015, Wnęk_2015f}, and Spain \cite{CamachoDorado_2018, fjbuilsRepeatedBehaviorDeliberate2024}; 1 case from Australia \cite{Apikotoa_2022f}, Bahrain \cite{Ali_2020f}, Croatia \cite{Trgo_2012f}, Ecuador \cite{DelgadoSalazar_2020c}, Egypt \cite{Ali_2022g}, Ethiopia \cite{Mesfin_2022a}, Germany \cite{teWildt_2010}, Greece \cite{Sakellaridis_2008f}, Hungary \cite{Csaky_1998e}, Iraq \cite{Al-Faham_2020k}, Israel \cite{Goldman_1998f}, Italy \cite{Riva_2018j}, Japan \cite{Ohno_2005}, Nepal \cite{Thapa_2019f}, Netherlands \cite{Benoist_2019e}, Oman \cite{AlShaaibi_2021b}, Pakistan \cite{Yasin_2009}, Portugal \cite{Peixoto_2017f}, Qatar \cite{Ali_2017}, Saudi Arabia \cite{Sultan_2024f}, South Africa \cite{Sobnach_2011f}, Sweden \cite{Naji_2012f}, Switzerland \cite{Wildhaber_2005}, and Taiwan \cite{Chang_2017f}. \paragraph*{Gender} 43 cases (60\%) were male \cite{Akay_2015f, Al-Faham_2020k, Alao_2006i, Ali_2017, Ali_2022g, Apikotoa_2022f, Atayan_2016, Benoist_2019e, Berry_2021e, Bhumi_2024f, CamachoDorado_2018, Csaky_1998e, Emamhadi_2018, Farhadi_2024h, Fry_2010, Gardner_2017h, Guinan_2019f, Jehangir_2019h, Jin_2023, Kobiela_2015, Kumar_2001, Kumar_2019f, Liu_2005, Losanoff_1996, Losanoff_1997e, Mesfin_2022a, Misra_2013, Qureshi_2016, Riva_2018j, Sobnach_2011f, Tammana_2012j, Tanrikulu_2015e, Tay_2004, Thapa_2019f, Trgo_2012f, Wadhwa_2015e, Yasin_2009, teWildt_2010}, 28 cases (39\%) were female \cite{AlShaaibi_2021b, Ali_2020f, Ataya_2013, Beecroft_1998, Bhasin_2014, Bhattacharjee_2008, Cauchi_2002, Chang_2017f, Cox_2007, DelgadoSalazar_2020c, DivsalarP._2023a, Goldman_1998f, Hardy_2023g, Kar_2015, Kariholu_2008, Kerestes_2019, Li_2013, Naji_2012f, Ohno_2005, Peixoto_2017f, Sakellaridis_2008f, Sultan_2024f, Tupesis_2004f, Wildhaber_2005, Wnęk_2015f, Yildiz_2016e}, 1 case (1\%) had no gender recorded \cite{fjbuilsRepeatedBehaviorDeliberate2024}. \paragraph*{Age Group} 25 cases (35\%) were between 26 and 40 years of age \cite{Alao_2006i, Ali_2022g, Apikotoa_2022f, Ataya_2013, Benoist_2019e, Bhasin_2014, Chang_2017f, Cox_2007, DelgadoSalazar_2020c, Farhadi_2024h, Fry_2010, Gardner_2017h, Guinan_2019f, Jin_2023, Kumar_2019f, Losanoff_1996, Misra_2013, Qureshi_2016, Riva_2018j, Sakellaridis_2008f, Tammana_2012j, Trgo_2012f, Wnęk_2015f, Yildiz_2016e, fjbuilsRepeatedBehaviorDeliberate2024}, 18 cases (25\%) were between 18 and 25 years of age \cite{Akay_2015f, Ali_2017, Atayan_2016, Bhattacharjee_2008, Csaky_1998e, Kar_2015, Kariholu_2008, Kobiela_2015, Losanoff_1996, Losanoff_1997e, Mesfin_2022a, Peixoto_2017f, Sobnach_2011f, Tupesis_2004f, Yasin_2009}, 13 cases (18\%) were under 18 years of age \cite{AlShaaibi_2021b, Ali_2020f, Cauchi_2002, DivsalarP._2023a, Goldman_1998f, Liu_2005, Naji_2012f, Ohno_2005, Tanrikulu_2015e, Tay_2004, Wildhaber_2005}, 11 cases (15\%) were between 41 and 60 years of age \cite{Al-Faham_2020k, Bhumi_2024f, CamachoDorado_2018, Emamhadi_2018, Hardy_2023g, Jehangir_2019h, Kumar_2001, Sultan_2024f, Thapa_2019f, Wadhwa_2015e, teWildt_2010}, 3 cases (4\%) were over 60 years of age \cite{Beecroft_1998, Kerestes_2019, Li_2013}, 2 cases (3\%) had no age documented \cite{Berry_2021e}. \paragraph*{Population} 36 cases (50\%) had a psychiatric history \cite{AlShaaibi_2021b, Alao_2006i, Ali_2020f, Apikotoa_2022f, Ataya_2013, Atayan_2016, Beecroft_1998, CamachoDorado_2018, Chang_2017f, DelgadoSalazar_2020c, DivsalarP._2023a, Farhadi_2024h, Fry_2010, Guinan_2019f, Hardy_2023g, Jehangir_2019h, Jin_2023, Kar_2015, Kerestes_2019, Kobiela_2015, Kumar_2001, Kumar_2019f, Liu_2005, Mesfin_2022a, Misra_2013, Ohno_2005, Peixoto_2017f, Sakellaridis_2008f, Sultan_2024f, Tammana_2012j, Tanrikulu_2015e, Yildiz_2016e, fjbuilsRepeatedBehaviorDeliberate2024, teWildt_2010}, 19 cases (26\%) had ingested previously \cite{Alao_2006i, Apikotoa_2022f, Berry_2021e, Bhattacharjee_2008, Csaky_1998e, DivsalarP._2023a, Emamhadi_2018, Guinan_2019f, Jehangir_2019h, Jin_2023, Liu_2005, Sakellaridis_2008f, Tanrikulu_2015e, Thapa_2019f, Yildiz_2016e, fjbuilsRepeatedBehaviorDeliberate2024, teWildt_2010}, 12 cases (17\%) were detained persons \cite{Alao_2006i, Ali_2022g, Apikotoa_2022f, Losanoff_1996, Losanoff_1997e, Qureshi_2016, Tammana_2012j, Trgo_2012f}, 7 cases (10\%) were severely disabled \cite{Atayan_2016, Kerestes_2019, Liu_2005, Ohno_2005, Peixoto_2017f, Yildiz_2016e, teWildt_2010}, 4 cases (6\%) were psychiatric inpatients \cite{DivsalarP._2023a, fjbuilsRepeatedBehaviorDeliberate2024, teWildt_2010}, 3 cases (4\%) were under the influence of alcohol \cite{Benoist_2019e, Csaky_1998e, Thapa_2019f}, 2 cases (3\%) were displaced people \cite{Akay_2015f, Gardner_2017h}. \paragraph*{Motivation} 34 cases (47\%) had a psychiatric motivation \cite{Al-Faham_2020k, Alao_2006i, Ali_2020f, Apikotoa_2022f, Ataya_2013, Atayan_2016, Bhasin_2014, Bhattacharjee_2008, DelgadoSalazar_2020c, DivsalarP._2023a, Emamhadi_2018, Farhadi_2024h, Guinan_2019f, Hardy_2023g, Jehangir_2019h, Jin_2023, Kar_2015, Kariholu_2008, Kerestes_2019, Kobiela_2015, Kumar_2001, Kumar_2019f, Li_2013, Liu_2005, Misra_2013, Ohno_2005, Sakellaridis_2008f, Sultan_2024f, Tammana_2012j, Tanrikulu_2015e, Yasin_2009, teWildt_2010}, 21 cases (29\%) were motivated by self-harm intention \cite{Al-Faham_2020k, AlShaaibi_2021b, Alao_2006i, Ali_2017, CamachoDorado_2018, Chang_2017f, Cox_2007, Csaky_1998e, Fry_2010, Li_2013, Losanoff_1996, Losanoff_1997e, Mesfin_2022a, Sakellaridis_2008f, Tammana_2012j, Tanrikulu_2015e, fjbuilsRepeatedBehaviorDeliberate2024}, 17 cases (24\%) had a psychosocial motivation \cite{Akay_2015f, Benoist_2019e, Bhattacharjee_2008, Cauchi_2002, Goldman_1998f, Hardy_2023g, Kobiela_2015, Li_2013, Naji_2012f, Qureshi_2016, Riva_2018j, Sobnach_2011f, Tay_2004, Thapa_2019f, Tupesis_2004f, Wildhaber_2005, Wnęk_2015f}, 9 cases (12\%) were motivated by protest \cite{Bhumi_2024f, Gardner_2017h, Losanoff_1996, Losanoff_1997e, Tupesis_2004f}, 9 cases (12\%) had another documented motivation \cite{Ali_2020f, Ali_2022g, Emamhadi_2018, Guinan_2019f, Peixoto_2017f, Sakellaridis_2008f, Trgo_2012f, Wadhwa_2015e, Yildiz_2016e}. \paragraph*{Object Characteristics} 51 cases (71\%) ingested a large diameter object (\textgreater{}2.5cm) \cite{Akay_2015f, Al-Faham_2020k, AlShaaibi_2021b, Alao_2006i, Ali_2017, Ali_2022g, Apikotoa_2022f, Atayan_2016, Berry_2021e, Bhasin_2014, CamachoDorado_2018, Cauchi_2002, Chang_2017f, Cox_2007, Csaky_1998e, DivsalarP._2023a, Emamhadi_2018, Gardner_2017h, Guinan_2019f, Jehangir_2019h, Jin_2023, Kariholu_2008, Kerestes_2019, Kobiela_2015, Kumar_2001, Kumar_2019f, Losanoff_1996, Losanoff_1997e, Mesfin_2022a, Misra_2013, Naji_2012f, Ohno_2005, Peixoto_2017f, Qureshi_2016, Riva_2018j, Sakellaridis_2008f, Sultan_2024f, Tanrikulu_2015e, Thapa_2019f, Trgo_2012f, Wnęk_2015f, Yildiz_2016e, fjbuilsRepeatedBehaviorDeliberate2024, teWildt_2010}, 44 cases (61\%) ingested multiple objects \cite{Ali_2020f, Apikotoa_2022f, Ataya_2013, Atayan_2016, Beecroft_1998, Bhattacharjee_2008, Bhumi_2024f, CamachoDorado_2018, Cauchi_2002, Emamhadi_2018, Farhadi_2024h, Fry_2010, Goldman_1998f, Guinan_2019f, Hardy_2023g, Jehangir_2019h, Jin_2023, Kar_2015, Kariholu_2008, Kobiela_2015, Kumar_2001, Kumar_2019f, Li_2013, Liu_2005, Losanoff_1996, Mesfin_2022a, Misra_2013, Naji_2012f, Ohno_2005, Sobnach_2011f, Sultan_2024f, Tammana_2012j, Tanrikulu_2015e, Tay_2004, Thapa_2019f, Wadhwa_2015e, Wildhaber_2005, Yasin_2009, fjbuilsRepeatedBehaviorDeliberate2024, teWildt_2010}, 34 cases (47\%) ingested a sharp object \cite{AlShaaibi_2021b, Alao_2006i, Apikotoa_2022f, Ataya_2013, Benoist_2019e, Bhasin_2014, Bhattacharjee_2008, CamachoDorado_2018, Csaky_1998e, DelgadoSalazar_2020c, DivsalarP._2023a, Emamhadi_2018, Farhadi_2024h, Fry_2010, Guinan_2019f, Hardy_2023g, Jehangir_2019h, Jin_2023, Kariholu_2008, Kobiela_2015, Kumar_2019f, Losanoff_1996, Losanoff_1997e, Mesfin_2022a, Misra_2013, Sobnach_2011f, Yasin_2009, teWildt_2010}, 32 cases (44\%) ingested a long object (\textgreater{}5cm) \cite{Al-Faham_2020k, AlShaaibi_2021b, Ali_2017, Ali_2022g, Atayan_2016, Bhasin_2014, CamachoDorado_2018, Chang_2017f, Cox_2007, Csaky_1998e, DivsalarP._2023a, Emamhadi_2018, Fry_2010, Gardner_2017h, Jin_2023, Kariholu_2008, Kerestes_2019, Kobiela_2015, Kumar_2019f, Mesfin_2022a, Misra_2013, Ohno_2005, Qureshi_2016, Sakellaridis_2008f, Sultan_2024f, Thapa_2019f, Trgo_2012f, Yasin_2009, Yildiz_2016e, teWildt_2010}, 9 cases (12\%) ingested a magnet \cite{Ali_2020f, Bhumi_2024f, Cauchi_2002, Liu_2005, Naji_2012f, Ohno_2005, Tanrikulu_2015e, Tay_2004, Wildhaber_2005}, 2 cases (3\%) ingested a button battery \cite{Berry_2021e, Bhumi_2024f}. \paragraph*{Outcomes} 48 cases (67\%) experienced a complication \cite{Ali_2017, Ali_2020f, Apikotoa_2022f, Atayan_2016, Beecroft_1998, Benoist_2019e, Berry_2021e, Bhasin_2014, Bhumi_2024f, CamachoDorado_2018, Cauchi_2002, Cox_2007, Csaky_1998e, DelgadoSalazar_2020c, DivsalarP._2023a, Emamhadi_2018, Farhadi_2024h, Fry_2010, Gardner_2017h, Goldman_1998f, Jin_2023, Kariholu_2008, Kerestes_2019, Kobiela_2015, Kumar_2001, Kumar_2019f, Liu_2005, Losanoff_1996, Mesfin_2022a, Misra_2013, Naji_2012f, Ohno_2005, Sakellaridis_2008f, Sobnach_2011f, Sultan_2024f, Tanrikulu_2015e, Tay_2004, Thapa_2019f, Trgo_2012f, Tupesis_2004f, Wildhaber_2005, Wnęk_2015f, Yasin_2009, Yildiz_2016e}, 44 cases (61\%) underwent surgery \cite{Al-Faham_2020k, AlShaaibi_2021b, Alao_2006i, Ali_2017, Ali_2020f, Atayan_2016, Beecroft_1998, Bhasin_2014, CamachoDorado_2018, Cauchi_2002, Chang_2017f, Cox_2007, Csaky_1998e, DelgadoSalazar_2020c, DivsalarP._2023a, Farhadi_2024h, Fry_2010, Gardner_2017h, Jin_2023, Kariholu_2008, Kerestes_2019, Kobiela_2015, Kumar_2019f, Liu_2005, Losanoff_1996, Losanoff_1997e, Mesfin_2022a, Misra_2013, Naji_2012f, Sobnach_2011f, Tanrikulu_2015e, Tay_2004, Thapa_2019f, Tupesis_2004f, Wildhaber_2005, Wnęk_2015f, Yasin_2009, Yildiz_2016e, fjbuilsRepeatedBehaviorDeliberate2024}, 31 cases (43\%) underwent endoscopy \cite{Akay_2015f, Ali_2022g, Apikotoa_2022f, Atayan_2016, Benoist_2019e, Berry_2021e, Bhasin_2014, Bhumi_2024f, CamachoDorado_2018, Chang_2017f, DelgadoSalazar_2020c, Gardner_2017h, Guinan_2019f, Hardy_2023g, Jehangir_2019h, Kariholu_2008, Li_2013, Liu_2005, Ohno_2005, Peixoto_2017f, Qureshi_2016, Riva_2018j, Sakellaridis_2008f, Sultan_2024f, Tammana_2012j, Tanrikulu_2015e, Trgo_2012f, Wadhwa_2015e, Wnęk_2015f, teWildt_2010}, 7 cases (10\%) were managed conservatively \cite{Ataya_2013, Bhattacharjee_2008, DivsalarP._2023a, Emamhadi_2018, Goldman_1998f, Kar_2015, Kumar_2001}, 2 cases (3\%) died \cite{Emamhadi_2018, Kumar_2001}. All 90 were male gender. 90 cases (100\%) were detained at the time of ingestion \cite{Elghali_2016, Karp_1991b, Lee_2007}, 88 cases (98\%) were intentional ingestions \cite{Elghali_2016, Karp_1991b, Lee_2007}, 30 cases (33\%) had a psychiatric history documented \cite{Elghali_2016, Karp_1991b, Lee_2007}, 2 cases (2\%) had a history of prior ingestion \cite{Elghali_2016}. No cases were reported for were psychiatric inpatients, were displaced people, were under the influence of alcohol at the time of ingestion, and had a severe disability history.
\paragraph*{Motivation}  70 cases (78\%) reported protest motivation \cite{Elghali_2016, Karp_1991b, Lee_2007}, 12 cases (13\%) reported psychiatric motivation \cite{Karp_1991b}, 6 cases (7\%) reported self-harm motivation \cite{Elghali_2016, Karp_1991b}. No cases were reported for psychosocial motivation and other motivation.
\paragraph*{Object Characteristics}  68 cases (76\%) involved sharp object ingestion \cite{Elghali_2016, Karp_1991b, Lee_2007}, 32 cases (36\%) involved long (\textgreater 5cm) object ingestion \cite{Lee_2007}, 25 cases (28\%) involved ingestion of multiple objects \cite{Elghali_2016, Lee_2007}. No cases were reported for button battery ingestion, magnet ingestion, and involved large diameter (\textgreater 2.5cm) object ingestion.
\paragraph*{Outcomes}  47 cases (52\%) underwent endoscopic intervention \cite{Elghali_2016, Lee_2007}, 29 cases (32\%) were managed conservatively \cite{Elghali_2016, Karp_1991b}, 15 cases (17\%) underwent surgical intervention \cite{Elghali_2016, Karp_1991b, Lee_2007}, 6 cases (7\%) reported complications \cite{Lee_2007}, 1 case (1\%) died \cite{Elghali_2016}.
\paragraph*{Geographical Location}Cases were recorded in 33 countries: 13 cases from USA \cite{Alao_2006i, Ataya_2013, Bhumi_2024f, Fry_2010, Guinan_2019f, Hardy_2023g, Jehangir_2019h, Kerestes_2019, Kumar_2001, Liu_2005, Tammana_2012j, Tay_2004, Tupesis_2004f}; 7 cases from India \cite{Bhasin_2014, Bhattacharjee_2008, Kar_2015, Kariholu_2008, Kumar_2019f, Misra_2013, Wadhwa_2015e} and UK \cite{Beecroft_1998, Berry_2021e, Cauchi_2002, Cox_2007, Gardner_2017h, Qureshi_2016}; 6 cases from Bulgaria \cite{Losanoff_1996, Losanoff_1997e}; 5 cases from Iran \cite{DivsalarP._2023a, Emamhadi_2018, Farhadi_2024h}; 4 cases from Turkey \cite{Akay_2015f, Atayan_2016, Tanrikulu_2015e, Yildiz_2016e}; 2 cases from China \cite{Jin_2023, Li_2013}, Poland \cite{Kobiela_2015, Wnęk_2015f}, and Spain \cite{CamachoDorado_2018, fjbuilsRepeatedBehaviorDeliberate2024}; 1 case from Australia \cite{Apikotoa_2022f}, Bahrain \cite{Ali_2020f}, Croatia \cite{Trgo_2012f}, Ecuador \cite{DelgadoSalazar_2020c}, Egypt \cite{Ali_2022g}, Ethiopia \cite{Mesfin_2022a}, Germany \cite{teWildt_2010}, Greece \cite{Sakellaridis_2008f}, Hungary \cite{Csaky_1998e}, Iraq \cite{Al-Faham_2020k}, Israel \cite{Goldman_1998f}, Italy \cite{Riva_2018j}, Japan \cite{Ohno_2005}, Nepal \cite{Thapa_2019f}, Netherlands \cite{Benoist_2019e}, Oman \cite{AlShaaibi_2021b}, Pakistan \cite{Yasin_2009}, Portugal \cite{Peixoto_2017f}, Qatar \cite{Ali_2017}, Saudi Arabia \cite{Sultan_2024f}, South Africa \cite{Sobnach_2011f}, Sweden \cite{Naji_2012f}, Switzerland \cite{Wildhaber_2005}, and Taiwan \cite{Chang_2017f}. \paragraph*{Gender} 43 cases (60\%) were male \cite{Akay_2015f, Al-Faham_2020k, Alao_2006i, Ali_2017, Ali_2022g, Apikotoa_2022f, Atayan_2016, Benoist_2019e, Berry_2021e, Bhumi_2024f, CamachoDorado_2018, Csaky_1998e, Emamhadi_2018, Farhadi_2024h, Fry_2010, Gardner_2017h, Guinan_2019f, Jehangir_2019h, Jin_2023, Kobiela_2015, Kumar_2001, Kumar_2019f, Liu_2005, Losanoff_1996, Losanoff_1997e, Mesfin_2022a, Misra_2013, Qureshi_2016, Riva_2018j, Sobnach_2011f, Tammana_2012j, Tanrikulu_2015e, Tay_2004, Thapa_2019f, Trgo_2012f, Wadhwa_2015e, Yasin_2009, teWildt_2010}, 28 cases (39\%) were female \cite{AlShaaibi_2021b, Ali_2020f, Ataya_2013, Beecroft_1998, Bhasin_2014, Bhattacharjee_2008, Cauchi_2002, Chang_2017f, Cox_2007, DelgadoSalazar_2020c, DivsalarP._2023a, Goldman_1998f, Hardy_2023g, Kar_2015, Kariholu_2008, Kerestes_2019, Li_2013, Naji_2012f, Ohno_2005, Peixoto_2017f, Sakellaridis_2008f, Sultan_2024f, Tupesis_2004f, Wildhaber_2005, Wnęk_2015f, Yildiz_2016e}, 1 case (1\%) had no gender recorded \cite{fjbuilsRepeatedBehaviorDeliberate2024}. \paragraph*{Age Group} 25 cases (35\%) were between 26 and 40 years of age \cite{Alao_2006i, Ali_2022g, Apikotoa_2022f, Ataya_2013, Benoist_2019e, Bhasin_2014, Chang_2017f, Cox_2007, DelgadoSalazar_2020c, Farhadi_2024h, Fry_2010, Gardner_2017h, Guinan_2019f, Jin_2023, Kumar_2019f, Losanoff_1996, Misra_2013, Qureshi_2016, Riva_2018j, Sakellaridis_2008f, Tammana_2012j, Trgo_2012f, Wnęk_2015f, Yildiz_2016e, fjbuilsRepeatedBehaviorDeliberate2024}, 18 cases (25\%) were between 18 and 25 years of age \cite{Akay_2015f, Ali_2017, Atayan_2016, Bhattacharjee_2008, Csaky_1998e, Kar_2015, Kariholu_2008, Kobiela_2015, Losanoff_1996, Losanoff_1997e, Mesfin_2022a, Peixoto_2017f, Sobnach_2011f, Tupesis_2004f, Yasin_2009}, 13 cases (18\%) were under 18 years of age \cite{AlShaaibi_2021b, Ali_2020f, Cauchi_2002, DivsalarP._2023a, Goldman_1998f, Liu_2005, Naji_2012f, Ohno_2005, Tanrikulu_2015e, Tay_2004, Wildhaber_2005}, 11 cases (15\%) were between 41 and 60 years of age \cite{Al-Faham_2020k, Bhumi_2024f, CamachoDorado_2018, Emamhadi_2018, Hardy_2023g, Jehangir_2019h, Kumar_2001, Sultan_2024f, Thapa_2019f, Wadhwa_2015e, teWildt_2010}, 3 cases (4\%) were over 60 years of age \cite{Beecroft_1998, Kerestes_2019, Li_2013}, 2 cases (3\%) had no age documented \cite{Berry_2021e}. \paragraph*{Population} 36 cases (50\%) had a psychiatric history \cite{AlShaaibi_2021b, Alao_2006i, Ali_2020f, Apikotoa_2022f, Ataya_2013, Atayan_2016, Beecroft_1998, CamachoDorado_2018, Chang_2017f, DelgadoSalazar_2020c, DivsalarP._2023a, Farhadi_2024h, Fry_2010, Guinan_2019f, Hardy_2023g, Jehangir_2019h, Jin_2023, Kar_2015, Kerestes_2019, Kobiela_2015, Kumar_2001, Kumar_2019f, Liu_2005, Mesfin_2022a, Misra_2013, Ohno_2005, Peixoto_2017f, Sakellaridis_2008f, Sultan_2024f, Tammana_2012j, Tanrikulu_2015e, Yildiz_2016e, fjbuilsRepeatedBehaviorDeliberate2024, teWildt_2010}, 19 cases (26\%) had ingested previously \cite{Alao_2006i, Apikotoa_2022f, Berry_2021e, Bhattacharjee_2008, Csaky_1998e, DivsalarP._2023a, Emamhadi_2018, Guinan_2019f, Jehangir_2019h, Jin_2023, Liu_2005, Sakellaridis_2008f, Tanrikulu_2015e, Thapa_2019f, Yildiz_2016e, fjbuilsRepeatedBehaviorDeliberate2024, teWildt_2010}, 12 cases (17\%) were detained persons \cite{Alao_2006i, Ali_2022g, Apikotoa_2022f, Losanoff_1996, Losanoff_1997e, Qureshi_2016, Tammana_2012j, Trgo_2012f}, 7 cases (10\%) were severely disabled \cite{Atayan_2016, Kerestes_2019, Liu_2005, Ohno_2005, Peixoto_2017f, Yildiz_2016e, teWildt_2010}, 4 cases (6\%) were psychiatric inpatients \cite{DivsalarP._2023a, fjbuilsRepeatedBehaviorDeliberate2024, teWildt_2010}, 3 cases (4\%) were under the influence of alcohol \cite{Benoist_2019e, Csaky_1998e, Thapa_2019f}, 2 cases (3\%) were displaced people \cite{Akay_2015f, Gardner_2017h}. \paragraph*{Motivation} 34 cases (47\%) had a psychiatric motivation \cite{Al-Faham_2020k, Alao_2006i, Ali_2020f, Apikotoa_2022f, Ataya_2013, Atayan_2016, Bhasin_2014, Bhattacharjee_2008, DelgadoSalazar_2020c, DivsalarP._2023a, Emamhadi_2018, Farhadi_2024h, Guinan_2019f, Hardy_2023g, Jehangir_2019h, Jin_2023, Kar_2015, Kariholu_2008, Kerestes_2019, Kobiela_2015, Kumar_2001, Kumar_2019f, Li_2013, Liu_2005, Misra_2013, Ohno_2005, Sakellaridis_2008f, Sultan_2024f, Tammana_2012j, Tanrikulu_2015e, Yasin_2009, teWildt_2010}, 21 cases (29\%) were motivated by self-harm intention \cite{Al-Faham_2020k, AlShaaibi_2021b, Alao_2006i, Ali_2017, CamachoDorado_2018, Chang_2017f, Cox_2007, Csaky_1998e, Fry_2010, Li_2013, Losanoff_1996, Losanoff_1997e, Mesfin_2022a, Sakellaridis_2008f, Tammana_2012j, Tanrikulu_2015e, fjbuilsRepeatedBehaviorDeliberate2024}, 17 cases (24\%) had a psychosocial motivation \cite{Akay_2015f, Benoist_2019e, Bhattacharjee_2008, Cauchi_2002, Goldman_1998f, Hardy_2023g, Kobiela_2015, Li_2013, Naji_2012f, Qureshi_2016, Riva_2018j, Sobnach_2011f, Tay_2004, Thapa_2019f, Tupesis_2004f, Wildhaber_2005, Wnęk_2015f}, 9 cases (12\%) were motivated by protest \cite{Bhumi_2024f, Gardner_2017h, Losanoff_1996, Losanoff_1997e, Tupesis_2004f}, 9 cases (12\%) had another documented motivation \cite{Ali_2020f, Ali_2022g, Emamhadi_2018, Guinan_2019f, Peixoto_2017f, Sakellaridis_2008f, Trgo_2012f, Wadhwa_2015e, Yildiz_2016e}. \paragraph*{Object Characteristics} 51 cases (71\%) ingested a large diameter object (\textgreater{}2.5cm) \cite{Akay_2015f, Al-Faham_2020k, AlShaaibi_2021b, Alao_2006i, Ali_2017, Ali_2022g, Apikotoa_2022f, Atayan_2016, Berry_2021e, Bhasin_2014, CamachoDorado_2018, Cauchi_2002, Chang_2017f, Cox_2007, Csaky_1998e, DivsalarP._2023a, Emamhadi_2018, Gardner_2017h, Guinan_2019f, Jehangir_2019h, Jin_2023, Kariholu_2008, Kerestes_2019, Kobiela_2015, Kumar_2001, Kumar_2019f, Losanoff_1996, Losanoff_1997e, Mesfin_2022a, Misra_2013, Naji_2012f, Ohno_2005, Peixoto_2017f, Qureshi_2016, Riva_2018j, Sakellaridis_2008f, Sultan_2024f, Tanrikulu_2015e, Thapa_2019f, Trgo_2012f, Wnęk_2015f, Yildiz_2016e, fjbuilsRepeatedBehaviorDeliberate2024, teWildt_2010}, 44 cases (61\%) ingested multiple objects \cite{Ali_2020f, Apikotoa_2022f, Ataya_2013, Atayan_2016, Beecroft_1998, Bhattacharjee_2008, Bhumi_2024f, CamachoDorado_2018, Cauchi_2002, Emamhadi_2018, Farhadi_2024h, Fry_2010, Goldman_1998f, Guinan_2019f, Hardy_2023g, Jehangir_2019h, Jin_2023, Kar_2015, Kariholu_2008, Kobiela_2015, Kumar_2001, Kumar_2019f, Li_2013, Liu_2005, Losanoff_1996, Mesfin_2022a, Misra_2013, Naji_2012f, Ohno_2005, Sobnach_2011f, Sultan_2024f, Tammana_2012j, Tanrikulu_2015e, Tay_2004, Thapa_2019f, Wadhwa_2015e, Wildhaber_2005, Yasin_2009, fjbuilsRepeatedBehaviorDeliberate2024, teWildt_2010}, 34 cases (47\%) ingested a sharp object \cite{AlShaaibi_2021b, Alao_2006i, Apikotoa_2022f, Ataya_2013, Benoist_2019e, Bhasin_2014, Bhattacharjee_2008, CamachoDorado_2018, Csaky_1998e, DelgadoSalazar_2020c, DivsalarP._2023a, Emamhadi_2018, Farhadi_2024h, Fry_2010, Guinan_2019f, Hardy_2023g, Jehangir_2019h, Jin_2023, Kariholu_2008, Kobiela_2015, Kumar_2019f, Losanoff_1996, Losanoff_1997e, Mesfin_2022a, Misra_2013, Sobnach_2011f, Yasin_2009, teWildt_2010}, 32 cases (44\%) ingested a long object (\textgreater{}5cm) \cite{Al-Faham_2020k, AlShaaibi_2021b, Ali_2017, Ali_2022g, Atayan_2016, Bhasin_2014, CamachoDorado_2018, Chang_2017f, Cox_2007, Csaky_1998e, DivsalarP._2023a, Emamhadi_2018, Fry_2010, Gardner_2017h, Jin_2023, Kariholu_2008, Kerestes_2019, Kobiela_2015, Kumar_2019f, Mesfin_2022a, Misra_2013, Ohno_2005, Qureshi_2016, Sakellaridis_2008f, Sultan_2024f, Thapa_2019f, Trgo_2012f, Yasin_2009, Yildiz_2016e, teWildt_2010}, 9 cases (12\%) ingested a magnet \cite{Ali_2020f, Bhumi_2024f, Cauchi_2002, Liu_2005, Naji_2012f, Ohno_2005, Tanrikulu_2015e, Tay_2004, Wildhaber_2005}, 2 cases (3\%) ingested a button battery \cite{Berry_2021e, Bhumi_2024f}. \paragraph*{Outcomes} 48 cases (67\%) experienced a complication \cite{Ali_2017, Ali_2020f, Apikotoa_2022f, Atayan_2016, Beecroft_1998, Benoist_2019e, Berry_2021e, Bhasin_2014, Bhumi_2024f, CamachoDorado_2018, Cauchi_2002, Cox_2007, Csaky_1998e, DelgadoSalazar_2020c, DivsalarP._2023a, Emamhadi_2018, Farhadi_2024h, Fry_2010, Gardner_2017h, Goldman_1998f, Jin_2023, Kariholu_2008, Kerestes_2019, Kobiela_2015, Kumar_2001, Kumar_2019f, Liu_2005, Losanoff_1996, Mesfin_2022a, Misra_2013, Naji_2012f, Ohno_2005, Sakellaridis_2008f, Sobnach_2011f, Sultan_2024f, Tanrikulu_2015e, Tay_2004, Thapa_2019f, Trgo_2012f, Tupesis_2004f, Wildhaber_2005, Wnęk_2015f, Yasin_2009, Yildiz_2016e}, 44 cases (61\%) underwent surgery \cite{Al-Faham_2020k, AlShaaibi_2021b, Alao_2006i, Ali_2017, Ali_2020f, Atayan_2016, Beecroft_1998, Bhasin_2014, CamachoDorado_2018, Cauchi_2002, Chang_2017f, Cox_2007, Csaky_1998e, DelgadoSalazar_2020c, DivsalarP._2023a, Farhadi_2024h, Fry_2010, Gardner_2017h, Jin_2023, Kariholu_2008, Kerestes_2019, Kobiela_2015, Kumar_2019f, Liu_2005, Losanoff_1996, Losanoff_1997e, Mesfin_2022a, Misra_2013, Naji_2012f, Sobnach_2011f, Tanrikulu_2015e, Tay_2004, Thapa_2019f, Tupesis_2004f, Wildhaber_2005, Wnęk_2015f, Yasin_2009, Yildiz_2016e, fjbuilsRepeatedBehaviorDeliberate2024}, 31 cases (43\%) underwent endoscopy \cite{Akay_2015f, Ali_2022g, Apikotoa_2022f, Atayan_2016, Benoist_2019e, Berry_2021e, Bhasin_2014, Bhumi_2024f, CamachoDorado_2018, Chang_2017f, DelgadoSalazar_2020c, Gardner_2017h, Guinan_2019f, Hardy_2023g, Jehangir_2019h, Kariholu_2008, Li_2013, Liu_2005, Ohno_2005, Peixoto_2017f, Qureshi_2016, Riva_2018j, Sakellaridis_2008f, Sultan_2024f, Tammana_2012j, Tanrikulu_2015e, Trgo_2012f, Wadhwa_2015e, Wnęk_2015f, teWildt_2010}, 7 cases (10\%) were managed conservatively \cite{Ataya_2013, Bhattacharjee_2008, DivsalarP._2023a, Emamhadi_2018, Goldman_1998f, Kar_2015, Kumar_2001}, 2 cases (3\%) died \cite{Emamhadi_2018, Kumar_2001}. All 90 were male gender. 90 cases (100\%) were detained at the time of ingestion \cite{Elghali_2016, Karp_1991b, Lee_2007}, 88 cases (98\%) were intentional ingestions \cite{Elghali_2016, Karp_1991b, Lee_2007}, 30 cases (33\%) had a psychiatric history documented \cite{Elghali_2016, Karp_1991b, Lee_2007}, 2 cases (2\%) had a history of prior ingestion \cite{Elghali_2016}. No cases were reported for were psychiatric inpatients, were displaced people, were under the influence of alcohol at the time of ingestion, and had a severe disability history.
\paragraph*{Motivation}  70 cases (78\%) reported protest motivation \cite{Elghali_2016, Karp_1991b, Lee_2007}, 12 cases (13\%) reported psychiatric motivation \cite{Karp_1991b}, 6 cases (7\%) reported self-harm motivation \cite{Elghali_2016, Karp_1991b}. No cases were reported for psychosocial motivation and other motivation.
\paragraph*{Object Characteristics}  68 cases (76\%) involved sharp object ingestion \cite{Elghali_2016, Karp_1991b, Lee_2007}, 32 cases (36\%) involved long (\textgreater 5cm) object ingestion \cite{Lee_2007}, 25 cases (28\%) involved ingestion of multiple objects \cite{Elghali_2016, Lee_2007}. No cases were reported for button battery ingestion, magnet ingestion, and involved large diameter (\textgreater 2.5cm) object ingestion.
\paragraph*{Outcomes}  47 cases (52\%) underwent endoscopic intervention \cite{Elghali_2016, Lee_2007}, 29 cases (32\%) were managed conservatively \cite{Elghali_2016, Karp_1991b}, 15 cases (17\%) underwent surgical intervention \cite{Elghali_2016, Karp_1991b, Lee_2007}, 6 cases (7\%) reported complications \cite{Lee_2007}, 1 case (1\%) died \cite{Elghali_2016}.
\paragraph*{Geographical Location}Cases were recorded in 33 countries: 13 cases from USA \cite{Alao_2006i, Ataya_2013, Bhumi_2024f, Fry_2010, Guinan_2019f, Hardy_2023g, Jehangir_2019h, Kerestes_2019, Kumar_2001, Liu_2005, Tammana_2012j, Tay_2004, Tupesis_2004f}; 7 cases from India \cite{Bhasin_2014, Bhattacharjee_2008, Kar_2015, Kariholu_2008, Kumar_2019f, Misra_2013, Wadhwa_2015e} and UK \cite{Beecroft_1998, Berry_2021e, Cauchi_2002, Cox_2007, Gardner_2017h, Qureshi_2016}; 6 cases from Bulgaria \cite{Losanoff_1996, Losanoff_1997e}; 5 cases from Iran \cite{DivsalarP._2023a, Emamhadi_2018, Farhadi_2024h}; 4 cases from Turkey \cite{Akay_2015f, Atayan_2016, Tanrikulu_2015e, Yildiz_2016e}; 2 cases from China \cite{Jin_2023, Li_2013}, Poland \cite{Kobiela_2015, Wnęk_2015f}, and Spain \cite{CamachoDorado_2018, fjbuilsRepeatedBehaviorDeliberate2024}; 1 case from Australia \cite{Apikotoa_2022f}, Bahrain \cite{Ali_2020f}, Croatia \cite{Trgo_2012f}, Ecuador \cite{DelgadoSalazar_2020c}, Egypt \cite{Ali_2022g}, Ethiopia \cite{Mesfin_2022a}, Germany \cite{teWildt_2010}, Greece \cite{Sakellaridis_2008f}, Hungary \cite{Csaky_1998e}, Iraq \cite{Al-Faham_2020k}, Israel \cite{Goldman_1998f}, Italy \cite{Riva_2018j}, Japan \cite{Ohno_2005}, Nepal \cite{Thapa_2019f}, Netherlands \cite{Benoist_2019e}, Oman \cite{AlShaaibi_2021b}, Pakistan \cite{Yasin_2009}, Portugal \cite{Peixoto_2017f}, Qatar \cite{Ali_2017}, Saudi Arabia \cite{Sultan_2024f}, South Africa \cite{Sobnach_2011f}, Sweden \cite{Naji_2012f}, Switzerland \cite{Wildhaber_2005}, and Taiwan \cite{Chang_2017f}. \paragraph*{Gender} 43 cases (60\%) were male \cite{Akay_2015f, Al-Faham_2020k, Alao_2006i, Ali_2017, Ali_2022g, Apikotoa_2022f, Atayan_2016, Benoist_2019e, Berry_2021e, Bhumi_2024f, CamachoDorado_2018, Csaky_1998e, Emamhadi_2018, Farhadi_2024h, Fry_2010, Gardner_2017h, Guinan_2019f, Jehangir_2019h, Jin_2023, Kobiela_2015, Kumar_2001, Kumar_2019f, Liu_2005, Losanoff_1996, Losanoff_1997e, Mesfin_2022a, Misra_2013, Qureshi_2016, Riva_2018j, Sobnach_2011f, Tammana_2012j, Tanrikulu_2015e, Tay_2004, Thapa_2019f, Trgo_2012f, Wadhwa_2015e, Yasin_2009, teWildt_2010}, 28 cases (39\%) were female \cite{AlShaaibi_2021b, Ali_2020f, Ataya_2013, Beecroft_1998, Bhasin_2014, Bhattacharjee_2008, Cauchi_2002, Chang_2017f, Cox_2007, DelgadoSalazar_2020c, DivsalarP._2023a, Goldman_1998f, Hardy_2023g, Kar_2015, Kariholu_2008, Kerestes_2019, Li_2013, Naji_2012f, Ohno_2005, Peixoto_2017f, Sakellaridis_2008f, Sultan_2024f, Tupesis_2004f, Wildhaber_2005, Wnęk_2015f, Yildiz_2016e}, 1 case (1\%) had no gender recorded \cite{fjbuilsRepeatedBehaviorDeliberate2024}. \paragraph*{Age Group} 25 cases (35\%) were between 26 and 40 years of age \cite{Alao_2006i, Ali_2022g, Apikotoa_2022f, Ataya_2013, Benoist_2019e, Bhasin_2014, Chang_2017f, Cox_2007, DelgadoSalazar_2020c, Farhadi_2024h, Fry_2010, Gardner_2017h, Guinan_2019f, Jin_2023, Kumar_2019f, Losanoff_1996, Misra_2013, Qureshi_2016, Riva_2018j, Sakellaridis_2008f, Tammana_2012j, Trgo_2012f, Wnęk_2015f, Yildiz_2016e, fjbuilsRepeatedBehaviorDeliberate2024}, 18 cases (25\%) were between 18 and 25 years of age \cite{Akay_2015f, Ali_2017, Atayan_2016, Bhattacharjee_2008, Csaky_1998e, Kar_2015, Kariholu_2008, Kobiela_2015, Losanoff_1996, Losanoff_1997e, Mesfin_2022a, Peixoto_2017f, Sobnach_2011f, Tupesis_2004f, Yasin_2009}, 13 cases (18\%) were under 18 years of age \cite{AlShaaibi_2021b, Ali_2020f, Cauchi_2002, DivsalarP._2023a, Goldman_1998f, Liu_2005, Naji_2012f, Ohno_2005, Tanrikulu_2015e, Tay_2004, Wildhaber_2005}, 11 cases (15\%) were between 41 and 60 years of age \cite{Al-Faham_2020k, Bhumi_2024f, CamachoDorado_2018, Emamhadi_2018, Hardy_2023g, Jehangir_2019h, Kumar_2001, Sultan_2024f, Thapa_2019f, Wadhwa_2015e, teWildt_2010}, 3 cases (4\%) were over 60 years of age \cite{Beecroft_1998, Kerestes_2019, Li_2013}, 2 cases (3\%) had no age documented \cite{Berry_2021e}. \paragraph*{Population} 36 cases (50\%) had a psychiatric history \cite{AlShaaibi_2021b, Alao_2006i, Ali_2020f, Apikotoa_2022f, Ataya_2013, Atayan_2016, Beecroft_1998, CamachoDorado_2018, Chang_2017f, DelgadoSalazar_2020c, DivsalarP._2023a, Farhadi_2024h, Fry_2010, Guinan_2019f, Hardy_2023g, Jehangir_2019h, Jin_2023, Kar_2015, Kerestes_2019, Kobiela_2015, Kumar_2001, Kumar_2019f, Liu_2005, Mesfin_2022a, Misra_2013, Ohno_2005, Peixoto_2017f, Sakellaridis_2008f, Sultan_2024f, Tammana_2012j, Tanrikulu_2015e, Yildiz_2016e, fjbuilsRepeatedBehaviorDeliberate2024, teWildt_2010}, 19 cases (26\%) had ingested previously \cite{Alao_2006i, Apikotoa_2022f, Berry_2021e, Bhattacharjee_2008, Csaky_1998e, DivsalarP._2023a, Emamhadi_2018, Guinan_2019f, Jehangir_2019h, Jin_2023, Liu_2005, Sakellaridis_2008f, Tanrikulu_2015e, Thapa_2019f, Yildiz_2016e, fjbuilsRepeatedBehaviorDeliberate2024, teWildt_2010}, 12 cases (17\%) were detained persons \cite{Alao_2006i, Ali_2022g, Apikotoa_2022f, Losanoff_1996, Losanoff_1997e, Qureshi_2016, Tammana_2012j, Trgo_2012f}, 7 cases (10\%) were severely disabled \cite{Atayan_2016, Kerestes_2019, Liu_2005, Ohno_2005, Peixoto_2017f, Yildiz_2016e, teWildt_2010}, 4 cases (6\%) were psychiatric inpatients \cite{DivsalarP._2023a, fjbuilsRepeatedBehaviorDeliberate2024, teWildt_2010}, 3 cases (4\%) were under the influence of alcohol \cite{Benoist_2019e, Csaky_1998e, Thapa_2019f}, 2 cases (3\%) were displaced people \cite{Akay_2015f, Gardner_2017h}. \paragraph*{Motivation} 34 cases (47\%) had a psychiatric motivation \cite{Al-Faham_2020k, Alao_2006i, Ali_2020f, Apikotoa_2022f, Ataya_2013, Atayan_2016, Bhasin_2014, Bhattacharjee_2008, DelgadoSalazar_2020c, DivsalarP._2023a, Emamhadi_2018, Farhadi_2024h, Guinan_2019f, Hardy_2023g, Jehangir_2019h, Jin_2023, Kar_2015, Kariholu_2008, Kerestes_2019, Kobiela_2015, Kumar_2001, Kumar_2019f, Li_2013, Liu_2005, Misra_2013, Ohno_2005, Sakellaridis_2008f, Sultan_2024f, Tammana_2012j, Tanrikulu_2015e, Yasin_2009, teWildt_2010}, 21 cases (29\%) were motivated by self-harm intention \cite{Al-Faham_2020k, AlShaaibi_2021b, Alao_2006i, Ali_2017, CamachoDorado_2018, Chang_2017f, Cox_2007, Csaky_1998e, Fry_2010, Li_2013, Losanoff_1996, Losanoff_1997e, Mesfin_2022a, Sakellaridis_2008f, Tammana_2012j, Tanrikulu_2015e, fjbuilsRepeatedBehaviorDeliberate2024}, 17 cases (24\%) had a psychosocial motivation \cite{Akay_2015f, Benoist_2019e, Bhattacharjee_2008, Cauchi_2002, Goldman_1998f, Hardy_2023g, Kobiela_2015, Li_2013, Naji_2012f, Qureshi_2016, Riva_2018j, Sobnach_2011f, Tay_2004, Thapa_2019f, Tupesis_2004f, Wildhaber_2005, Wnęk_2015f}, 9 cases (12\%) were motivated by protest \cite{Bhumi_2024f, Gardner_2017h, Losanoff_1996, Losanoff_1997e, Tupesis_2004f}, 9 cases (12\%) had another documented motivation \cite{Ali_2020f, Ali_2022g, Emamhadi_2018, Guinan_2019f, Peixoto_2017f, Sakellaridis_2008f, Trgo_2012f, Wadhwa_2015e, Yildiz_2016e}. \paragraph*{Object Characteristics} 51 cases (71\%) ingested a large diameter object (\textgreater{}2.5cm) \cite{Akay_2015f, Al-Faham_2020k, AlShaaibi_2021b, Alao_2006i, Ali_2017, Ali_2022g, Apikotoa_2022f, Atayan_2016, Berry_2021e, Bhasin_2014, CamachoDorado_2018, Cauchi_2002, Chang_2017f, Cox_2007, Csaky_1998e, DivsalarP._2023a, Emamhadi_2018, Gardner_2017h, Guinan_2019f, Jehangir_2019h, Jin_2023, Kariholu_2008, Kerestes_2019, Kobiela_2015, Kumar_2001, Kumar_2019f, Losanoff_1996, Losanoff_1997e, Mesfin_2022a, Misra_2013, Naji_2012f, Ohno_2005, Peixoto_2017f, Qureshi_2016, Riva_2018j, Sakellaridis_2008f, Sultan_2024f, Tanrikulu_2015e, Thapa_2019f, Trgo_2012f, Wnęk_2015f, Yildiz_2016e, fjbuilsRepeatedBehaviorDeliberate2024, teWildt_2010}, 44 cases (61\%) ingested multiple objects \cite{Ali_2020f, Apikotoa_2022f, Ataya_2013, Atayan_2016, Beecroft_1998, Bhattacharjee_2008, Bhumi_2024f, CamachoDorado_2018, Cauchi_2002, Emamhadi_2018, Farhadi_2024h, Fry_2010, Goldman_1998f, Guinan_2019f, Hardy_2023g, Jehangir_2019h, Jin_2023, Kar_2015, Kariholu_2008, Kobiela_2015, Kumar_2001, Kumar_2019f, Li_2013, Liu_2005, Losanoff_1996, Mesfin_2022a, Misra_2013, Naji_2012f, Ohno_2005, Sobnach_2011f, Sultan_2024f, Tammana_2012j, Tanrikulu_2015e, Tay_2004, Thapa_2019f, Wadhwa_2015e, Wildhaber_2005, Yasin_2009, fjbuilsRepeatedBehaviorDeliberate2024, teWildt_2010}, 34 cases (47\%) ingested a sharp object \cite{AlShaaibi_2021b, Alao_2006i, Apikotoa_2022f, Ataya_2013, Benoist_2019e, Bhasin_2014, Bhattacharjee_2008, CamachoDorado_2018, Csaky_1998e, DelgadoSalazar_2020c, DivsalarP._2023a, Emamhadi_2018, Farhadi_2024h, Fry_2010, Guinan_2019f, Hardy_2023g, Jehangir_2019h, Jin_2023, Kariholu_2008, Kobiela_2015, Kumar_2019f, Losanoff_1996, Losanoff_1997e, Mesfin_2022a, Misra_2013, Sobnach_2011f, Yasin_2009, teWildt_2010}, 32 cases (44\%) ingested a long object (\textgreater{}5cm) \cite{Al-Faham_2020k, AlShaaibi_2021b, Ali_2017, Ali_2022g, Atayan_2016, Bhasin_2014, CamachoDorado_2018, Chang_2017f, Cox_2007, Csaky_1998e, DivsalarP._2023a, Emamhadi_2018, Fry_2010, Gardner_2017h, Jin_2023, Kariholu_2008, Kerestes_2019, Kobiela_2015, Kumar_2019f, Mesfin_2022a, Misra_2013, Ohno_2005, Qureshi_2016, Sakellaridis_2008f, Sultan_2024f, Thapa_2019f, Trgo_2012f, Yasin_2009, Yildiz_2016e, teWildt_2010}, 9 cases (12\%) ingested a magnet \cite{Ali_2020f, Bhumi_2024f, Cauchi_2002, Liu_2005, Naji_2012f, Ohno_2005, Tanrikulu_2015e, Tay_2004, Wildhaber_2005}, 2 cases (3\%) ingested a button battery \cite{Berry_2021e, Bhumi_2024f}. \paragraph*{Outcomes} 48 cases (67\%) experienced a complication \cite{Ali_2017, Ali_2020f, Apikotoa_2022f, Atayan_2016, Beecroft_1998, Benoist_2019e, Berry_2021e, Bhasin_2014, Bhumi_2024f, CamachoDorado_2018, Cauchi_2002, Cox_2007, Csaky_1998e, DelgadoSalazar_2020c, DivsalarP._2023a, Emamhadi_2018, Farhadi_2024h, Fry_2010, Gardner_2017h, Goldman_1998f, Jin_2023, Kariholu_2008, Kerestes_2019, Kobiela_2015, Kumar_2001, Kumar_2019f, Liu_2005, Losanoff_1996, Mesfin_2022a, Misra_2013, Naji_2012f, Ohno_2005, Sakellaridis_2008f, Sobnach_2011f, Sultan_2024f, Tanrikulu_2015e, Tay_2004, Thapa_2019f, Trgo_2012f, Tupesis_2004f, Wildhaber_2005, Wnęk_2015f, Yasin_2009, Yildiz_2016e}, 44 cases (61\%) underwent surgery \cite{Al-Faham_2020k, AlShaaibi_2021b, Alao_2006i, Ali_2017, Ali_2020f, Atayan_2016, Beecroft_1998, Bhasin_2014, CamachoDorado_2018, Cauchi_2002, Chang_2017f, Cox_2007, Csaky_1998e, DelgadoSalazar_2020c, DivsalarP._2023a, Farhadi_2024h, Fry_2010, Gardner_2017h, Jin_2023, Kariholu_2008, Kerestes_2019, Kobiela_2015, Kumar_2019f, Liu_2005, Losanoff_1996, Losanoff_1997e, Mesfin_2022a, Misra_2013, Naji_2012f, Sobnach_2011f, Tanrikulu_2015e, Tay_2004, Thapa_2019f, Tupesis_2004f, Wildhaber_2005, Wnęk_2015f, Yasin_2009, Yildiz_2016e, fjbuilsRepeatedBehaviorDeliberate2024}, 31 cases (43\%) underwent endoscopy \cite{Akay_2015f, Ali_2022g, Apikotoa_2022f, Atayan_2016, Benoist_2019e, Berry_2021e, Bhasin_2014, Bhumi_2024f, CamachoDorado_2018, Chang_2017f, DelgadoSalazar_2020c, Gardner_2017h, Guinan_2019f, Hardy_2023g, Jehangir_2019h, Kariholu_2008, Li_2013, Liu_2005, Ohno_2005, Peixoto_2017f, Qureshi_2016, Riva_2018j, Sakellaridis_2008f, Sultan_2024f, Tammana_2012j, Tanrikulu_2015e, Trgo_2012f, Wadhwa_2015e, Wnęk_2015f, teWildt_2010}, 7 cases (10\%) were managed conservatively \cite{Ataya_2013, Bhattacharjee_2008, DivsalarP._2023a, Emamhadi_2018, Goldman_1998f, Kar_2015, Kumar_2001}, 2 cases (3\%) died \cite{Emamhadi_2018, Kumar_2001}. All 90 were male gender. 90 cases (100\%) were detained at the time of ingestion \cite{Elghali_2016, Karp_1991b, Lee_2007}, 88 cases (98\%) were intentional ingestions \cite{Elghali_2016, Karp_1991b, Lee_2007}, 30 cases (33\%) had a psychiatric history documented \cite{Elghali_2016, Karp_1991b, Lee_2007}, 2 cases (2\%) had a history of prior ingestion \cite{Elghali_2016}. No cases were reported for were psychiatric inpatients, were displaced people, were under the influence of alcohol at the time of ingestion, and had a severe disability history.
\paragraph*{Motivation}  70 cases (78\%) reported protest motivation \cite{Elghali_2016, Karp_1991b, Lee_2007}, 12 cases (13\%) reported psychiatric motivation \cite{Karp_1991b}, 6 cases (7\%) reported self-harm motivation \cite{Elghali_2016, Karp_1991b}. No cases were reported for psychosocial motivation and other motivation.
\paragraph*{Object Characteristics}  68 cases (76\%) involved sharp object ingestion \cite{Elghali_2016, Karp_1991b, Lee_2007}, 32 cases (36\%) involved long (\textgreater 5cm) object ingestion \cite{Lee_2007}, 25 cases (28\%) involved ingestion of multiple objects \cite{Elghali_2016, Lee_2007}. No cases were reported for button battery ingestion, magnet ingestion, and involved large diameter (\textgreater 2.5cm) object ingestion.
\paragraph*{Outcomes}  47 cases (52\%) underwent endoscopic intervention \cite{Elghali_2016, Lee_2007}, 29 cases (32\%) were managed conservatively \cite{Elghali_2016, Karp_1991b}, 15 cases (17\%) underwent surgical intervention \cite{Elghali_2016, Karp_1991b, Lee_2007}, 6 cases (7\%) reported complications \cite{Lee_2007}, 1 case (1\%) died \cite{Elghali_2016}.
\paragraph*{Geographical Location}Cases were recorded in 33 countries: 13 cases from USA \cite{Alao_2006i, Ataya_2013, Bhumi_2024f, Fry_2010, Guinan_2019f, Hardy_2023g, Jehangir_2019h, Kerestes_2019, Kumar_2001, Liu_2005, Tammana_2012j, Tay_2004, Tupesis_2004f}; 7 cases from India \cite{Bhasin_2014, Bhattacharjee_2008, Kar_2015, Kariholu_2008, Kumar_2019f, Misra_2013, Wadhwa_2015e} and UK \cite{Beecroft_1998, Berry_2021e, Cauchi_2002, Cox_2007, Gardner_2017h, Qureshi_2016}; 6 cases from Bulgaria \cite{Losanoff_1996, Losanoff_1997e}; 5 cases from Iran \cite{DivsalarP._2023a, Emamhadi_2018, Farhadi_2024h}; 4 cases from Turkey \cite{Akay_2015f, Atayan_2016, Tanrikulu_2015e, Yildiz_2016e}; 2 cases from China \cite{Jin_2023, Li_2013}, Poland \cite{Kobiela_2015, Wnęk_2015f}, and Spain \cite{CamachoDorado_2018, fjbuilsRepeatedBehaviorDeliberate2024}; 1 case from Australia \cite{Apikotoa_2022f}, Bahrain \cite{Ali_2020f}, Croatia \cite{Trgo_2012f}, Ecuador \cite{DelgadoSalazar_2020c}, Egypt \cite{Ali_2022g}, Ethiopia \cite{Mesfin_2022a}, Germany \cite{teWildt_2010}, Greece \cite{Sakellaridis_2008f}, Hungary \cite{Csaky_1998e}, Iraq \cite{Al-Faham_2020k}, Israel \cite{Goldman_1998f}, Italy \cite{Riva_2018j}, Japan \cite{Ohno_2005}, Nepal \cite{Thapa_2019f}, Netherlands \cite{Benoist_2019e}, Oman \cite{AlShaaibi_2021b}, Pakistan \cite{Yasin_2009}, Portugal \cite{Peixoto_2017f}, Qatar \cite{Ali_2017}, Saudi Arabia \cite{Sultan_2024f}, South Africa \cite{Sobnach_2011f}, Sweden \cite{Naji_2012f}, Switzerland \cite{Wildhaber_2005}, and Taiwan \cite{Chang_2017f}. \paragraph*{Gender} 43 cases (60\%) were male \cite{Akay_2015f, Al-Faham_2020k, Alao_2006i, Ali_2017, Ali_2022g, Apikotoa_2022f, Atayan_2016, Benoist_2019e, Berry_2021e, Bhumi_2024f, CamachoDorado_2018, Csaky_1998e, Emamhadi_2018, Farhadi_2024h, Fry_2010, Gardner_2017h, Guinan_2019f, Jehangir_2019h, Jin_2023, Kobiela_2015, Kumar_2001, Kumar_2019f, Liu_2005, Losanoff_1996, Losanoff_1997e, Mesfin_2022a, Misra_2013, Qureshi_2016, Riva_2018j, Sobnach_2011f, Tammana_2012j, Tanrikulu_2015e, Tay_2004, Thapa_2019f, Trgo_2012f, Wadhwa_2015e, Yasin_2009, teWildt_2010}, 28 cases (39\%) were female \cite{AlShaaibi_2021b, Ali_2020f, Ataya_2013, Beecroft_1998, Bhasin_2014, Bhattacharjee_2008, Cauchi_2002, Chang_2017f, Cox_2007, DelgadoSalazar_2020c, DivsalarP._2023a, Goldman_1998f, Hardy_2023g, Kar_2015, Kariholu_2008, Kerestes_2019, Li_2013, Naji_2012f, Ohno_2005, Peixoto_2017f, Sakellaridis_2008f, Sultan_2024f, Tupesis_2004f, Wildhaber_2005, Wnęk_2015f, Yildiz_2016e}, 1 case (1\%) had no gender recorded \cite{fjbuilsRepeatedBehaviorDeliberate2024}. \paragraph*{Age Group} 25 cases (35\%) were between 26 and 40 years of age \cite{Alao_2006i, Ali_2022g, Apikotoa_2022f, Ataya_2013, Benoist_2019e, Bhasin_2014, Chang_2017f, Cox_2007, DelgadoSalazar_2020c, Farhadi_2024h, Fry_2010, Gardner_2017h, Guinan_2019f, Jin_2023, Kumar_2019f, Losanoff_1996, Misra_2013, Qureshi_2016, Riva_2018j, Sakellaridis_2008f, Tammana_2012j, Trgo_2012f, Wnęk_2015f, Yildiz_2016e, fjbuilsRepeatedBehaviorDeliberate2024}, 18 cases (25\%) were between 18 and 25 years of age \cite{Akay_2015f, Ali_2017, Atayan_2016, Bhattacharjee_2008, Csaky_1998e, Kar_2015, Kariholu_2008, Kobiela_2015, Losanoff_1996, Losanoff_1997e, Mesfin_2022a, Peixoto_2017f, Sobnach_2011f, Tupesis_2004f, Yasin_2009}, 13 cases (18\%) were under 18 years of age \cite{AlShaaibi_2021b, Ali_2020f, Cauchi_2002, DivsalarP._2023a, Goldman_1998f, Liu_2005, Naji_2012f, Ohno_2005, Tanrikulu_2015e, Tay_2004, Wildhaber_2005}, 11 cases (15\%) were between 41 and 60 years of age \cite{Al-Faham_2020k, Bhumi_2024f, CamachoDorado_2018, Emamhadi_2018, Hardy_2023g, Jehangir_2019h, Kumar_2001, Sultan_2024f, Thapa_2019f, Wadhwa_2015e, teWildt_2010}, 3 cases (4\%) were over 60 years of age \cite{Beecroft_1998, Kerestes_2019, Li_2013}, 2 cases (3\%) had no age documented \cite{Berry_2021e}. \paragraph*{Population} 36 cases (50\%) had a psychiatric history \cite{AlShaaibi_2021b, Alao_2006i, Ali_2020f, Apikotoa_2022f, Ataya_2013, Atayan_2016, Beecroft_1998, CamachoDorado_2018, Chang_2017f, DelgadoSalazar_2020c, DivsalarP._2023a, Farhadi_2024h, Fry_2010, Guinan_2019f, Hardy_2023g, Jehangir_2019h, Jin_2023, Kar_2015, Kerestes_2019, Kobiela_2015, Kumar_2001, Kumar_2019f, Liu_2005, Mesfin_2022a, Misra_2013, Ohno_2005, Peixoto_2017f, Sakellaridis_2008f, Sultan_2024f, Tammana_2012j, Tanrikulu_2015e, Yildiz_2016e, fjbuilsRepeatedBehaviorDeliberate2024, teWildt_2010}, 19 cases (26\%) had ingested previously \cite{Alao_2006i, Apikotoa_2022f, Berry_2021e, Bhattacharjee_2008, Csaky_1998e, DivsalarP._2023a, Emamhadi_2018, Guinan_2019f, Jehangir_2019h, Jin_2023, Liu_2005, Sakellaridis_2008f, Tanrikulu_2015e, Thapa_2019f, Yildiz_2016e, fjbuilsRepeatedBehaviorDeliberate2024, teWildt_2010}, 12 cases (17\%) were detained persons \cite{Alao_2006i, Ali_2022g, Apikotoa_2022f, Losanoff_1996, Losanoff_1997e, Qureshi_2016, Tammana_2012j, Trgo_2012f}, 7 cases (10\%) were severely disabled \cite{Atayan_2016, Kerestes_2019, Liu_2005, Ohno_2005, Peixoto_2017f, Yildiz_2016e, teWildt_2010}, 4 cases (6\%) were psychiatric inpatients \cite{DivsalarP._2023a, fjbuilsRepeatedBehaviorDeliberate2024, teWildt_2010}, 3 cases (4\%) were under the influence of alcohol \cite{Benoist_2019e, Csaky_1998e, Thapa_2019f}, 2 cases (3\%) were displaced people \cite{Akay_2015f, Gardner_2017h}. \paragraph*{Motivation} 34 cases (47\%) had a psychiatric motivation \cite{Al-Faham_2020k, Alao_2006i, Ali_2020f, Apikotoa_2022f, Ataya_2013, Atayan_2016, Bhasin_2014, Bhattacharjee_2008, DelgadoSalazar_2020c, DivsalarP._2023a, Emamhadi_2018, Farhadi_2024h, Guinan_2019f, Hardy_2023g, Jehangir_2019h, Jin_2023, Kar_2015, Kariholu_2008, Kerestes_2019, Kobiela_2015, Kumar_2001, Kumar_2019f, Li_2013, Liu_2005, Misra_2013, Ohno_2005, Sakellaridis_2008f, Sultan_2024f, Tammana_2012j, Tanrikulu_2015e, Yasin_2009, teWildt_2010}, 21 cases (29\%) were motivated by self-harm intention \cite{Al-Faham_2020k, AlShaaibi_2021b, Alao_2006i, Ali_2017, CamachoDorado_2018, Chang_2017f, Cox_2007, Csaky_1998e, Fry_2010, Li_2013, Losanoff_1996, Losanoff_1997e, Mesfin_2022a, Sakellaridis_2008f, Tammana_2012j, Tanrikulu_2015e, fjbuilsRepeatedBehaviorDeliberate2024}, 17 cases (24\%) had a psychosocial motivation \cite{Akay_2015f, Benoist_2019e, Bhattacharjee_2008, Cauchi_2002, Goldman_1998f, Hardy_2023g, Kobiela_2015, Li_2013, Naji_2012f, Qureshi_2016, Riva_2018j, Sobnach_2011f, Tay_2004, Thapa_2019f, Tupesis_2004f, Wildhaber_2005, Wnęk_2015f}, 9 cases (12\%) were motivated by protest \cite{Bhumi_2024f, Gardner_2017h, Losanoff_1996, Losanoff_1997e, Tupesis_2004f}, 9 cases (12\%) had another documented motivation \cite{Ali_2020f, Ali_2022g, Emamhadi_2018, Guinan_2019f, Peixoto_2017f, Sakellaridis_2008f, Trgo_2012f, Wadhwa_2015e, Yildiz_2016e}. \paragraph*{Object Characteristics} 51 cases (71\%) ingested a large diameter object (\textgreater{}2.5cm) \cite{Akay_2015f, Al-Faham_2020k, AlShaaibi_2021b, Alao_2006i, Ali_2017, Ali_2022g, Apikotoa_2022f, Atayan_2016, Berry_2021e, Bhasin_2014, CamachoDorado_2018, Cauchi_2002, Chang_2017f, Cox_2007, Csaky_1998e, DivsalarP._2023a, Emamhadi_2018, Gardner_2017h, Guinan_2019f, Jehangir_2019h, Jin_2023, Kariholu_2008, Kerestes_2019, Kobiela_2015, Kumar_2001, Kumar_2019f, Losanoff_1996, Losanoff_1997e, Mesfin_2022a, Misra_2013, Naji_2012f, Ohno_2005, Peixoto_2017f, Qureshi_2016, Riva_2018j, Sakellaridis_2008f, Sultan_2024f, Tanrikulu_2015e, Thapa_2019f, Trgo_2012f, Wnęk_2015f, Yildiz_2016e, fjbuilsRepeatedBehaviorDeliberate2024, teWildt_2010}, 44 cases (61\%) ingested multiple objects \cite{Ali_2020f, Apikotoa_2022f, Ataya_2013, Atayan_2016, Beecroft_1998, Bhattacharjee_2008, Bhumi_2024f, CamachoDorado_2018, Cauchi_2002, Emamhadi_2018, Farhadi_2024h, Fry_2010, Goldman_1998f, Guinan_2019f, Hardy_2023g, Jehangir_2019h, Jin_2023, Kar_2015, Kariholu_2008, Kobiela_2015, Kumar_2001, Kumar_2019f, Li_2013, Liu_2005, Losanoff_1996, Mesfin_2022a, Misra_2013, Naji_2012f, Ohno_2005, Sobnach_2011f, Sultan_2024f, Tammana_2012j, Tanrikulu_2015e, Tay_2004, Thapa_2019f, Wadhwa_2015e, Wildhaber_2005, Yasin_2009, fjbuilsRepeatedBehaviorDeliberate2024, teWildt_2010}, 34 cases (47\%) ingested a sharp object \cite{AlShaaibi_2021b, Alao_2006i, Apikotoa_2022f, Ataya_2013, Benoist_2019e, Bhasin_2014, Bhattacharjee_2008, CamachoDorado_2018, Csaky_1998e, DelgadoSalazar_2020c, DivsalarP._2023a, Emamhadi_2018, Farhadi_2024h, Fry_2010, Guinan_2019f, Hardy_2023g, Jehangir_2019h, Jin_2023, Kariholu_2008, Kobiela_2015, Kumar_2019f, Losanoff_1996, Losanoff_1997e, Mesfin_2022a, Misra_2013, Sobnach_2011f, Yasin_2009, teWildt_2010}, 32 cases (44\%) ingested a long object (\textgreater{}5cm) \cite{Al-Faham_2020k, AlShaaibi_2021b, Ali_2017, Ali_2022g, Atayan_2016, Bhasin_2014, CamachoDorado_2018, Chang_2017f, Cox_2007, Csaky_1998e, DivsalarP._2023a, Emamhadi_2018, Fry_2010, Gardner_2017h, Jin_2023, Kariholu_2008, Kerestes_2019, Kobiela_2015, Kumar_2019f, Mesfin_2022a, Misra_2013, Ohno_2005, Qureshi_2016, Sakellaridis_2008f, Sultan_2024f, Thapa_2019f, Trgo_2012f, Yasin_2009, Yildiz_2016e, teWildt_2010}, 9 cases (12\%) ingested a magnet \cite{Ali_2020f, Bhumi_2024f, Cauchi_2002, Liu_2005, Naji_2012f, Ohno_2005, Tanrikulu_2015e, Tay_2004, Wildhaber_2005}, 2 cases (3\%) ingested a button battery \cite{Berry_2021e, Bhumi_2024f}. \paragraph*{Outcomes} 48 cases (67\%) experienced a complication \cite{Ali_2017, Ali_2020f, Apikotoa_2022f, Atayan_2016, Beecroft_1998, Benoist_2019e, Berry_2021e, Bhasin_2014, Bhumi_2024f, CamachoDorado_2018, Cauchi_2002, Cox_2007, Csaky_1998e, DelgadoSalazar_2020c, DivsalarP._2023a, Emamhadi_2018, Farhadi_2024h, Fry_2010, Gardner_2017h, Goldman_1998f, Jin_2023, Kariholu_2008, Kerestes_2019, Kobiela_2015, Kumar_2001, Kumar_2019f, Liu_2005, Losanoff_1996, Mesfin_2022a, Misra_2013, Naji_2012f, Ohno_2005, Sakellaridis_2008f, Sobnach_2011f, Sultan_2024f, Tanrikulu_2015e, Tay_2004, Thapa_2019f, Trgo_2012f, Tupesis_2004f, Wildhaber_2005, Wnęk_2015f, Yasin_2009, Yildiz_2016e}, 44 cases (61\%) underwent surgery \cite{Al-Faham_2020k, AlShaaibi_2021b, Alao_2006i, Ali_2017, Ali_2020f, Atayan_2016, Beecroft_1998, Bhasin_2014, CamachoDorado_2018, Cauchi_2002, Chang_2017f, Cox_2007, Csaky_1998e, DelgadoSalazar_2020c, DivsalarP._2023a, Farhadi_2024h, Fry_2010, Gardner_2017h, Jin_2023, Kariholu_2008, Kerestes_2019, Kobiela_2015, Kumar_2019f, Liu_2005, Losanoff_1996, Losanoff_1997e, Mesfin_2022a, Misra_2013, Naji_2012f, Sobnach_2011f, Tanrikulu_2015e, Tay_2004, Thapa_2019f, Tupesis_2004f, Wildhaber_2005, Wnęk_2015f, Yasin_2009, Yildiz_2016e, fjbuilsRepeatedBehaviorDeliberate2024}, 31 cases (43\%) underwent endoscopy \cite{Akay_2015f, Ali_2022g, Apikotoa_2022f, Atayan_2016, Benoist_2019e, Berry_2021e, Bhasin_2014, Bhumi_2024f, CamachoDorado_2018, Chang_2017f, DelgadoSalazar_2020c, Gardner_2017h, Guinan_2019f, Hardy_2023g, Jehangir_2019h, Kariholu_2008, Li_2013, Liu_2005, Ohno_2005, Peixoto_2017f, Qureshi_2016, Riva_2018j, Sakellaridis_2008f, Sultan_2024f, Tammana_2012j, Tanrikulu_2015e, Trgo_2012f, Wadhwa_2015e, Wnęk_2015f, teWildt_2010}, 7 cases (10\%) were managed conservatively \cite{Ataya_2013, Bhattacharjee_2008, DivsalarP._2023a, Emamhadi_2018, Goldman_1998f, Kar_2015, Kumar_2001}, 2 cases (3\%) died \cite{Emamhadi_2018, Kumar_2001}. All 90 were male gender. 90 cases (100\%) were detained at the time of ingestion \cite{Elghali_2016, Karp_1991b, Lee_2007}, 88 cases (98\%) were intentional ingestions \cite{Elghali_2016, Karp_1991b, Lee_2007}, 30 cases (33\%) had a psychiatric history documented \cite{Elghali_2016, Karp_1991b, Lee_2007}, 2 cases (2\%) had a history of prior ingestion \cite{Elghali_2016}. No cases were reported for were psychiatric inpatients, were displaced people, were under the influence of alcohol at the time of ingestion, and had a severe disability history.
\paragraph*{Motivation}  70 cases (78\%) reported protest motivation \cite{Elghali_2016, Karp_1991b, Lee_2007}, 12 cases (13\%) reported psychiatric motivation \cite{Karp_1991b}, 6 cases (7\%) reported self-harm motivation \cite{Elghali_2016, Karp_1991b}. No cases were reported for psychosocial motivation and other motivation.
\paragraph*{Object Characteristics}  68 cases (76\%) involved sharp object ingestion \cite{Elghali_2016, Karp_1991b, Lee_2007}, 32 cases (36\%) involved long (\textgreater 5cm) object ingestion \cite{Lee_2007}, 25 cases (28\%) involved ingestion of multiple objects \cite{Elghali_2016, Lee_2007}. No cases were reported for button battery ingestion, magnet ingestion, and involved large diameter (\textgreater 2.5cm) object ingestion.
\paragraph*{Outcomes}  47 cases (52\%) underwent endoscopic intervention \cite{Elghali_2016, Lee_2007}, 29 cases (32\%) were managed conservatively \cite{Elghali_2016, Karp_1991b}, 15 cases (17\%) underwent surgical intervention \cite{Elghali_2016, Karp_1991b, Lee_2007}, 6 cases (7\%) reported complications \cite{Lee_2007}, 1 case (1\%) died \cite{Elghali_2016}.
\paragraph*{Geographical Location}Cases were recorded in 33 countries: 13 cases from USA \cite{Alao_2006i, Ataya_2013, Bhumi_2024f, Fry_2010, Guinan_2019f, Hardy_2023g, Jehangir_2019h, Kerestes_2019, Kumar_2001, Liu_2005, Tammana_2012j, Tay_2004, Tupesis_2004f}; 7 cases from India \cite{Bhasin_2014, Bhattacharjee_2008, Kar_2015, Kariholu_2008, Kumar_2019f, Misra_2013, Wadhwa_2015e} and UK \cite{Beecroft_1998, Berry_2021e, Cauchi_2002, Cox_2007, Gardner_2017h, Qureshi_2016}; 6 cases from Bulgaria \cite{Losanoff_1996, Losanoff_1997e}; 5 cases from Iran \cite{DivsalarP._2023a, Emamhadi_2018, Farhadi_2024h}; 4 cases from Turkey \cite{Akay_2015f, Atayan_2016, Tanrikulu_2015e, Yildiz_2016e}; 2 cases from China \cite{Jin_2023, Li_2013}, Poland \cite{Kobiela_2015, Wnęk_2015f}, and Spain \cite{CamachoDorado_2018, fjbuilsRepeatedBehaviorDeliberate2024}; 1 case from Australia \cite{Apikotoa_2022f}, Bahrain \cite{Ali_2020f}, Croatia \cite{Trgo_2012f}, Ecuador \cite{DelgadoSalazar_2020c}, Egypt \cite{Ali_2022g}, Ethiopia \cite{Mesfin_2022a}, Germany \cite{teWildt_2010}, Greece \cite{Sakellaridis_2008f}, Hungary \cite{Csaky_1998e}, Iraq \cite{Al-Faham_2020k}, Israel \cite{Goldman_1998f}, Italy \cite{Riva_2018j}, Japan \cite{Ohno_2005}, Nepal \cite{Thapa_2019f}, Netherlands \cite{Benoist_2019e}, Oman \cite{AlShaaibi_2021b}, Pakistan \cite{Yasin_2009}, Portugal \cite{Peixoto_2017f}, Qatar \cite{Ali_2017}, Saudi Arabia \cite{Sultan_2024f}, South Africa \cite{Sobnach_2011f}, Sweden \cite{Naji_2012f}, Switzerland \cite{Wildhaber_2005}, and Taiwan \cite{Chang_2017f}. \paragraph*{Gender} 43 cases (60\%) were male \cite{Akay_2015f, Al-Faham_2020k, Alao_2006i, Ali_2017, Ali_2022g, Apikotoa_2022f, Atayan_2016, Benoist_2019e, Berry_2021e, Bhumi_2024f, CamachoDorado_2018, Csaky_1998e, Emamhadi_2018, Farhadi_2024h, Fry_2010, Gardner_2017h, Guinan_2019f, Jehangir_2019h, Jin_2023, Kobiela_2015, Kumar_2001, Kumar_2019f, Liu_2005, Losanoff_1996, Losanoff_1997e, Mesfin_2022a, Misra_2013, Qureshi_2016, Riva_2018j, Sobnach_2011f, Tammana_2012j, Tanrikulu_2015e, Tay_2004, Thapa_2019f, Trgo_2012f, Wadhwa_2015e, Yasin_2009, teWildt_2010}, 28 cases (39\%) were female \cite{AlShaaibi_2021b, Ali_2020f, Ataya_2013, Beecroft_1998, Bhasin_2014, Bhattacharjee_2008, Cauchi_2002, Chang_2017f, Cox_2007, DelgadoSalazar_2020c, DivsalarP._2023a, Goldman_1998f, Hardy_2023g, Kar_2015, Kariholu_2008, Kerestes_2019, Li_2013, Naji_2012f, Ohno_2005, Peixoto_2017f, Sakellaridis_2008f, Sultan_2024f, Tupesis_2004f, Wildhaber_2005, Wnęk_2015f, Yildiz_2016e}, 1 case (1\%) had no gender recorded \cite{fjbuilsRepeatedBehaviorDeliberate2024}. \paragraph*{Age Group} 25 cases (35\%) were between 26 and 40 years of age \cite{Alao_2006i, Ali_2022g, Apikotoa_2022f, Ataya_2013, Benoist_2019e, Bhasin_2014, Chang_2017f, Cox_2007, DelgadoSalazar_2020c, Farhadi_2024h, Fry_2010, Gardner_2017h, Guinan_2019f, Jin_2023, Kumar_2019f, Losanoff_1996, Misra_2013, Qureshi_2016, Riva_2018j, Sakellaridis_2008f, Tammana_2012j, Trgo_2012f, Wnęk_2015f, Yildiz_2016e, fjbuilsRepeatedBehaviorDeliberate2024}, 18 cases (25\%) were between 18 and 25 years of age \cite{Akay_2015f, Ali_2017, Atayan_2016, Bhattacharjee_2008, Csaky_1998e, Kar_2015, Kariholu_2008, Kobiela_2015, Losanoff_1996, Losanoff_1997e, Mesfin_2022a, Peixoto_2017f, Sobnach_2011f, Tupesis_2004f, Yasin_2009}, 13 cases (18\%) were under 18 years of age \cite{AlShaaibi_2021b, Ali_2020f, Cauchi_2002, DivsalarP._2023a, Goldman_1998f, Liu_2005, Naji_2012f, Ohno_2005, Tanrikulu_2015e, Tay_2004, Wildhaber_2005}, 11 cases (15\%) were between 41 and 60 years of age \cite{Al-Faham_2020k, Bhumi_2024f, CamachoDorado_2018, Emamhadi_2018, Hardy_2023g, Jehangir_2019h, Kumar_2001, Sultan_2024f, Thapa_2019f, Wadhwa_2015e, teWildt_2010}, 3 cases (4\%) were over 60 years of age \cite{Beecroft_1998, Kerestes_2019, Li_2013}, 2 cases (3\%) had no age documented \cite{Berry_2021e}. \paragraph*{Population} 36 cases (50\%) had a psychiatric history \cite{AlShaaibi_2021b, Alao_2006i, Ali_2020f, Apikotoa_2022f, Ataya_2013, Atayan_2016, Beecroft_1998, CamachoDorado_2018, Chang_2017f, DelgadoSalazar_2020c, DivsalarP._2023a, Farhadi_2024h, Fry_2010, Guinan_2019f, Hardy_2023g, Jehangir_2019h, Jin_2023, Kar_2015, Kerestes_2019, Kobiela_2015, Kumar_2001, Kumar_2019f, Liu_2005, Mesfin_2022a, Misra_2013, Ohno_2005, Peixoto_2017f, Sakellaridis_2008f, Sultan_2024f, Tammana_2012j, Tanrikulu_2015e, Yildiz_2016e, fjbuilsRepeatedBehaviorDeliberate2024, teWildt_2010}, 19 cases (26\%) had ingested previously \cite{Alao_2006i, Apikotoa_2022f, Berry_2021e, Bhattacharjee_2008, Csaky_1998e, DivsalarP._2023a, Emamhadi_2018, Guinan_2019f, Jehangir_2019h, Jin_2023, Liu_2005, Sakellaridis_2008f, Tanrikulu_2015e, Thapa_2019f, Yildiz_2016e, fjbuilsRepeatedBehaviorDeliberate2024, teWildt_2010}, 12 cases (17\%) were detained persons \cite{Alao_2006i, Ali_2022g, Apikotoa_2022f, Losanoff_1996, Losanoff_1997e, Qureshi_2016, Tammana_2012j, Trgo_2012f}, 7 cases (10\%) were severely disabled \cite{Atayan_2016, Kerestes_2019, Liu_2005, Ohno_2005, Peixoto_2017f, Yildiz_2016e, teWildt_2010}, 4 cases (6\%) were psychiatric inpatients \cite{DivsalarP._2023a, fjbuilsRepeatedBehaviorDeliberate2024, teWildt_2010}, 3 cases (4\%) were under the influence of alcohol \cite{Benoist_2019e, Csaky_1998e, Thapa_2019f}, 2 cases (3\%) were displaced people \cite{Akay_2015f, Gardner_2017h}. \paragraph*{Motivation} 34 cases (47\%) had a psychiatric motivation \cite{Al-Faham_2020k, Alao_2006i, Ali_2020f, Apikotoa_2022f, Ataya_2013, Atayan_2016, Bhasin_2014, Bhattacharjee_2008, DelgadoSalazar_2020c, DivsalarP._2023a, Emamhadi_2018, Farhadi_2024h, Guinan_2019f, Hardy_2023g, Jehangir_2019h, Jin_2023, Kar_2015, Kariholu_2008, Kerestes_2019, Kobiela_2015, Kumar_2001, Kumar_2019f, Li_2013, Liu_2005, Misra_2013, Ohno_2005, Sakellaridis_2008f, Sultan_2024f, Tammana_2012j, Tanrikulu_2015e, Yasin_2009, teWildt_2010}, 21 cases (29\%) were motivated by self-harm intention \cite{Al-Faham_2020k, AlShaaibi_2021b, Alao_2006i, Ali_2017, CamachoDorado_2018, Chang_2017f, Cox_2007, Csaky_1998e, Fry_2010, Li_2013, Losanoff_1996, Losanoff_1997e, Mesfin_2022a, Sakellaridis_2008f, Tammana_2012j, Tanrikulu_2015e, fjbuilsRepeatedBehaviorDeliberate2024}, 17 cases (24\%) had a psychosocial motivation \cite{Akay_2015f, Benoist_2019e, Bhattacharjee_2008, Cauchi_2002, Goldman_1998f, Hardy_2023g, Kobiela_2015, Li_2013, Naji_2012f, Qureshi_2016, Riva_2018j, Sobnach_2011f, Tay_2004, Thapa_2019f, Tupesis_2004f, Wildhaber_2005, Wnęk_2015f}, 9 cases (12\%) were motivated by protest \cite{Bhumi_2024f, Gardner_2017h, Losanoff_1996, Losanoff_1997e, Tupesis_2004f}, 9 cases (12\%) had another documented motivation \cite{Ali_2020f, Ali_2022g, Emamhadi_2018, Guinan_2019f, Peixoto_2017f, Sakellaridis_2008f, Trgo_2012f, Wadhwa_2015e, Yildiz_2016e}. \paragraph*{Object Characteristics} 51 cases (71\%) ingested a large diameter object (\textgreater{}2.5cm) \cite{Akay_2015f, Al-Faham_2020k, AlShaaibi_2021b, Alao_2006i, Ali_2017, Ali_2022g, Apikotoa_2022f, Atayan_2016, Berry_2021e, Bhasin_2014, CamachoDorado_2018, Cauchi_2002, Chang_2017f, Cox_2007, Csaky_1998e, DivsalarP._2023a, Emamhadi_2018, Gardner_2017h, Guinan_2019f, Jehangir_2019h, Jin_2023, Kariholu_2008, Kerestes_2019, Kobiela_2015, Kumar_2001, Kumar_2019f, Losanoff_1996, Losanoff_1997e, Mesfin_2022a, Misra_2013, Naji_2012f, Ohno_2005, Peixoto_2017f, Qureshi_2016, Riva_2018j, Sakellaridis_2008f, Sultan_2024f, Tanrikulu_2015e, Thapa_2019f, Trgo_2012f, Wnęk_2015f, Yildiz_2016e, fjbuilsRepeatedBehaviorDeliberate2024, teWildt_2010}, 44 cases (61\%) ingested multiple objects \cite{Ali_2020f, Apikotoa_2022f, Ataya_2013, Atayan_2016, Beecroft_1998, Bhattacharjee_2008, Bhumi_2024f, CamachoDorado_2018, Cauchi_2002, Emamhadi_2018, Farhadi_2024h, Fry_2010, Goldman_1998f, Guinan_2019f, Hardy_2023g, Jehangir_2019h, Jin_2023, Kar_2015, Kariholu_2008, Kobiela_2015, Kumar_2001, Kumar_2019f, Li_2013, Liu_2005, Losanoff_1996, Mesfin_2022a, Misra_2013, Naji_2012f, Ohno_2005, Sobnach_2011f, Sultan_2024f, Tammana_2012j, Tanrikulu_2015e, Tay_2004, Thapa_2019f, Wadhwa_2015e, Wildhaber_2005, Yasin_2009, fjbuilsRepeatedBehaviorDeliberate2024, teWildt_2010}, 34 cases (47\%) ingested a sharp object \cite{AlShaaibi_2021b, Alao_2006i, Apikotoa_2022f, Ataya_2013, Benoist_2019e, Bhasin_2014, Bhattacharjee_2008, CamachoDorado_2018, Csaky_1998e, DelgadoSalazar_2020c, DivsalarP._2023a, Emamhadi_2018, Farhadi_2024h, Fry_2010, Guinan_2019f, Hardy_2023g, Jehangir_2019h, Jin_2023, Kariholu_2008, Kobiela_2015, Kumar_2019f, Losanoff_1996, Losanoff_1997e, Mesfin_2022a, Misra_2013, Sobnach_2011f, Yasin_2009, teWildt_2010}, 32 cases (44\%) ingested a long object (\textgreater{}5cm) \cite{Al-Faham_2020k, AlShaaibi_2021b, Ali_2017, Ali_2022g, Atayan_2016, Bhasin_2014, CamachoDorado_2018, Chang_2017f, Cox_2007, Csaky_1998e, DivsalarP._2023a, Emamhadi_2018, Fry_2010, Gardner_2017h, Jin_2023, Kariholu_2008, Kerestes_2019, Kobiela_2015, Kumar_2019f, Mesfin_2022a, Misra_2013, Ohno_2005, Qureshi_2016, Sakellaridis_2008f, Sultan_2024f, Thapa_2019f, Trgo_2012f, Yasin_2009, Yildiz_2016e, teWildt_2010}, 9 cases (12\%) ingested a magnet \cite{Ali_2020f, Bhumi_2024f, Cauchi_2002, Liu_2005, Naji_2012f, Ohno_2005, Tanrikulu_2015e, Tay_2004, Wildhaber_2005}, 2 cases (3\%) ingested a button battery \cite{Berry_2021e, Bhumi_2024f}. \paragraph*{Outcomes} 48 cases (67\%) experienced a complication \cite{Ali_2017, Ali_2020f, Apikotoa_2022f, Atayan_2016, Beecroft_1998, Benoist_2019e, Berry_2021e, Bhasin_2014, Bhumi_2024f, CamachoDorado_2018, Cauchi_2002, Cox_2007, Csaky_1998e, DelgadoSalazar_2020c, DivsalarP._2023a, Emamhadi_2018, Farhadi_2024h, Fry_2010, Gardner_2017h, Goldman_1998f, Jin_2023, Kariholu_2008, Kerestes_2019, Kobiela_2015, Kumar_2001, Kumar_2019f, Liu_2005, Losanoff_1996, Mesfin_2022a, Misra_2013, Naji_2012f, Ohno_2005, Sakellaridis_2008f, Sobnach_2011f, Sultan_2024f, Tanrikulu_2015e, Tay_2004, Thapa_2019f, Trgo_2012f, Tupesis_2004f, Wildhaber_2005, Wnęk_2015f, Yasin_2009, Yildiz_2016e}, 44 cases (61\%) underwent surgery \cite{Al-Faham_2020k, AlShaaibi_2021b, Alao_2006i, Ali_2017, Ali_2020f, Atayan_2016, Beecroft_1998, Bhasin_2014, CamachoDorado_2018, Cauchi_2002, Chang_2017f, Cox_2007, Csaky_1998e, DelgadoSalazar_2020c, DivsalarP._2023a, Farhadi_2024h, Fry_2010, Gardner_2017h, Jin_2023, Kariholu_2008, Kerestes_2019, Kobiela_2015, Kumar_2019f, Liu_2005, Losanoff_1996, Losanoff_1997e, Mesfin_2022a, Misra_2013, Naji_2012f, Sobnach_2011f, Tanrikulu_2015e, Tay_2004, Thapa_2019f, Tupesis_2004f, Wildhaber_2005, Wnęk_2015f, Yasin_2009, Yildiz_2016e, fjbuilsRepeatedBehaviorDeliberate2024}, 31 cases (43\%) underwent endoscopy \cite{Akay_2015f, Ali_2022g, Apikotoa_2022f, Atayan_2016, Benoist_2019e, Berry_2021e, Bhasin_2014, Bhumi_2024f, CamachoDorado_2018, Chang_2017f, DelgadoSalazar_2020c, Gardner_2017h, Guinan_2019f, Hardy_2023g, Jehangir_2019h, Kariholu_2008, Li_2013, Liu_2005, Ohno_2005, Peixoto_2017f, Qureshi_2016, Riva_2018j, Sakellaridis_2008f, Sultan_2024f, Tammana_2012j, Tanrikulu_2015e, Trgo_2012f, Wadhwa_2015e, Wnęk_2015f, teWildt_2010}, 7 cases (10\%) were managed conservatively \cite{Ataya_2013, Bhattacharjee_2008, DivsalarP._2023a, Emamhadi_2018, Goldman_1998f, Kar_2015, Kumar_2001}, 2 cases (3\%) died \cite{Emamhadi_2018, Kumar_2001}. All 90 were male gender. 90 cases (100\%) were detained at the time of ingestion \cite{Elghali_2016, Karp_1991b, Lee_2007}, 88 cases (98\%) were intentional ingestions \cite{Elghali_2016, Karp_1991b, Lee_2007}, 30 cases (33\%) had a psychiatric history documented \cite{Elghali_2016, Karp_1991b, Lee_2007}, 2 cases (2\%) had a history of prior ingestion \cite{Elghali_2016}. No cases were reported for were psychiatric inpatients, were displaced people, were under the influence of alcohol at the time of ingestion, and had a severe disability history.
\paragraph*{Motivation}  70 cases (78\%) reported protest motivation \cite{Elghali_2016, Karp_1991b, Lee_2007}, 12 cases (13\%) reported psychiatric motivation \cite{Karp_1991b}, 6 cases (7\%) reported self-harm motivation \cite{Elghali_2016, Karp_1991b}. No cases were reported for psychosocial motivation and other motivation.
\paragraph*{Object Characteristics}  68 cases (76\%) involved sharp object ingestion \cite{Elghali_2016, Karp_1991b, Lee_2007}, 32 cases (36\%) involved long (\textgreater 5cm) object ingestion \cite{Lee_2007}, 25 cases (28\%) involved ingestion of multiple objects \cite{Elghali_2016, Lee_2007}. No cases were reported for button battery ingestion, magnet ingestion, and involved large diameter (\textgreater 2.5cm) object ingestion.
\paragraph*{Outcomes}  47 cases (52\%) underwent endoscopic intervention \cite{Elghali_2016, Lee_2007}, 29 cases (32\%) were managed conservatively \cite{Elghali_2016, Karp_1991b}, 15 cases (17\%) underwent surgical intervention \cite{Elghali_2016, Karp_1991b, Lee_2007}, 6 cases (7\%) reported complications \cite{Lee_2007}, 1 case (1\%) died \cite{Elghali_2016}.
\paragraph*{Geographical Location}Cases were recorded in 33 countries: 13 cases from USA \cite{Alao_2006i, Ataya_2013, Bhumi_2024f, Fry_2010, Guinan_2019f, Hardy_2023g, Jehangir_2019h, Kerestes_2019, Kumar_2001, Liu_2005, Tammana_2012j, Tay_2004, Tupesis_2004f}; 7 cases from India \cite{Bhasin_2014, Bhattacharjee_2008, Kar_2015, Kariholu_2008, Kumar_2019f, Misra_2013, Wadhwa_2015e} and UK \cite{Beecroft_1998, Berry_2021e, Cauchi_2002, Cox_2007, Gardner_2017h, Qureshi_2016}; 6 cases from Bulgaria \cite{Losanoff_1996, Losanoff_1997e}; 5 cases from Iran \cite{DivsalarP._2023a, Emamhadi_2018, Farhadi_2024h}; 4 cases from Turkey \cite{Akay_2015f, Atayan_2016, Tanrikulu_2015e, Yildiz_2016e}; 2 cases from China \cite{Jin_2023, Li_2013}, Poland \cite{Kobiela_2015, Wnęk_2015f}, and Spain \cite{CamachoDorado_2018, fjbuilsRepeatedBehaviorDeliberate2024}; 1 case from Australia \cite{Apikotoa_2022f}, Bahrain \cite{Ali_2020f}, Croatia \cite{Trgo_2012f}, Ecuador \cite{DelgadoSalazar_2020c}, Egypt \cite{Ali_2022g}, Ethiopia \cite{Mesfin_2022a}, Germany \cite{teWildt_2010}, Greece \cite{Sakellaridis_2008f}, Hungary \cite{Csaky_1998e}, Iraq \cite{Al-Faham_2020k}, Israel \cite{Goldman_1998f}, Italy \cite{Riva_2018j}, Japan \cite{Ohno_2005}, Nepal \cite{Thapa_2019f}, Netherlands \cite{Benoist_2019e}, Oman \cite{AlShaaibi_2021b}, Pakistan \cite{Yasin_2009}, Portugal \cite{Peixoto_2017f}, Qatar \cite{Ali_2017}, Saudi Arabia \cite{Sultan_2024f}, South Africa \cite{Sobnach_2011f}, Sweden \cite{Naji_2012f}, Switzerland \cite{Wildhaber_2005}, and Taiwan \cite{Chang_2017f}. \paragraph*{Gender} 43 cases (60\%) were male \cite{Akay_2015f, Al-Faham_2020k, Alao_2006i, Ali_2017, Ali_2022g, Apikotoa_2022f, Atayan_2016, Benoist_2019e, Berry_2021e, Bhumi_2024f, CamachoDorado_2018, Csaky_1998e, Emamhadi_2018, Farhadi_2024h, Fry_2010, Gardner_2017h, Guinan_2019f, Jehangir_2019h, Jin_2023, Kobiela_2015, Kumar_2001, Kumar_2019f, Liu_2005, Losanoff_1996, Losanoff_1997e, Mesfin_2022a, Misra_2013, Qureshi_2016, Riva_2018j, Sobnach_2011f, Tammana_2012j, Tanrikulu_2015e, Tay_2004, Thapa_2019f, Trgo_2012f, Wadhwa_2015e, Yasin_2009, teWildt_2010}, 28 cases (39\%) were female \cite{AlShaaibi_2021b, Ali_2020f, Ataya_2013, Beecroft_1998, Bhasin_2014, Bhattacharjee_2008, Cauchi_2002, Chang_2017f, Cox_2007, DelgadoSalazar_2020c, DivsalarP._2023a, Goldman_1998f, Hardy_2023g, Kar_2015, Kariholu_2008, Kerestes_2019, Li_2013, Naji_2012f, Ohno_2005, Peixoto_2017f, Sakellaridis_2008f, Sultan_2024f, Tupesis_2004f, Wildhaber_2005, Wnęk_2015f, Yildiz_2016e}, 1 case (1\%) had no gender recorded \cite{fjbuilsRepeatedBehaviorDeliberate2024}. \paragraph*{Age Group} 25 cases (35\%) were between 26 and 40 years of age \cite{Alao_2006i, Ali_2022g, Apikotoa_2022f, Ataya_2013, Benoist_2019e, Bhasin_2014, Chang_2017f, Cox_2007, DelgadoSalazar_2020c, Farhadi_2024h, Fry_2010, Gardner_2017h, Guinan_2019f, Jin_2023, Kumar_2019f, Losanoff_1996, Misra_2013, Qureshi_2016, Riva_2018j, Sakellaridis_2008f, Tammana_2012j, Trgo_2012f, Wnęk_2015f, Yildiz_2016e, fjbuilsRepeatedBehaviorDeliberate2024}, 18 cases (25\%) were between 18 and 25 years of age \cite{Akay_2015f, Ali_2017, Atayan_2016, Bhattacharjee_2008, Csaky_1998e, Kar_2015, Kariholu_2008, Kobiela_2015, Losanoff_1996, Losanoff_1997e, Mesfin_2022a, Peixoto_2017f, Sobnach_2011f, Tupesis_2004f, Yasin_2009}, 13 cases (18\%) were under 18 years of age \cite{AlShaaibi_2021b, Ali_2020f, Cauchi_2002, DivsalarP._2023a, Goldman_1998f, Liu_2005, Naji_2012f, Ohno_2005, Tanrikulu_2015e, Tay_2004, Wildhaber_2005}, 11 cases (15\%) were between 41 and 60 years of age \cite{Al-Faham_2020k, Bhumi_2024f, CamachoDorado_2018, Emamhadi_2018, Hardy_2023g, Jehangir_2019h, Kumar_2001, Sultan_2024f, Thapa_2019f, Wadhwa_2015e, teWildt_2010}, 3 cases (4\%) were over 60 years of age \cite{Beecroft_1998, Kerestes_2019, Li_2013}, 2 cases (3\%) had no age documented \cite{Berry_2021e}. \paragraph*{Population} 36 cases (50\%) had a psychiatric history \cite{AlShaaibi_2021b, Alao_2006i, Ali_2020f, Apikotoa_2022f, Ataya_2013, Atayan_2016, Beecroft_1998, CamachoDorado_2018, Chang_2017f, DelgadoSalazar_2020c, DivsalarP._2023a, Farhadi_2024h, Fry_2010, Guinan_2019f, Hardy_2023g, Jehangir_2019h, Jin_2023, Kar_2015, Kerestes_2019, Kobiela_2015, Kumar_2001, Kumar_2019f, Liu_2005, Mesfin_2022a, Misra_2013, Ohno_2005, Peixoto_2017f, Sakellaridis_2008f, Sultan_2024f, Tammana_2012j, Tanrikulu_2015e, Yildiz_2016e, fjbuilsRepeatedBehaviorDeliberate2024, teWildt_2010}, 19 cases (26\%) had ingested previously \cite{Alao_2006i, Apikotoa_2022f, Berry_2021e, Bhattacharjee_2008, Csaky_1998e, DivsalarP._2023a, Emamhadi_2018, Guinan_2019f, Jehangir_2019h, Jin_2023, Liu_2005, Sakellaridis_2008f, Tanrikulu_2015e, Thapa_2019f, Yildiz_2016e, fjbuilsRepeatedBehaviorDeliberate2024, teWildt_2010}, 12 cases (17\%) were detained persons \cite{Alao_2006i, Ali_2022g, Apikotoa_2022f, Losanoff_1996, Losanoff_1997e, Qureshi_2016, Tammana_2012j, Trgo_2012f}, 7 cases (10\%) were severely disabled \cite{Atayan_2016, Kerestes_2019, Liu_2005, Ohno_2005, Peixoto_2017f, Yildiz_2016e, teWildt_2010}, 4 cases (6\%) were psychiatric inpatients \cite{DivsalarP._2023a, fjbuilsRepeatedBehaviorDeliberate2024, teWildt_2010}, 3 cases (4\%) were under the influence of alcohol \cite{Benoist_2019e, Csaky_1998e, Thapa_2019f}, 2 cases (3\%) were displaced people \cite{Akay_2015f, Gardner_2017h}. \paragraph*{Motivation} 34 cases (47\%) had a psychiatric motivation \cite{Al-Faham_2020k, Alao_2006i, Ali_2020f, Apikotoa_2022f, Ataya_2013, Atayan_2016, Bhasin_2014, Bhattacharjee_2008, DelgadoSalazar_2020c, DivsalarP._2023a, Emamhadi_2018, Farhadi_2024h, Guinan_2019f, Hardy_2023g, Jehangir_2019h, Jin_2023, Kar_2015, Kariholu_2008, Kerestes_2019, Kobiela_2015, Kumar_2001, Kumar_2019f, Li_2013, Liu_2005, Misra_2013, Ohno_2005, Sakellaridis_2008f, Sultan_2024f, Tammana_2012j, Tanrikulu_2015e, Yasin_2009, teWildt_2010}, 21 cases (29\%) were motivated by self-harm intention \cite{Al-Faham_2020k, AlShaaibi_2021b, Alao_2006i, Ali_2017, CamachoDorado_2018, Chang_2017f, Cox_2007, Csaky_1998e, Fry_2010, Li_2013, Losanoff_1996, Losanoff_1997e, Mesfin_2022a, Sakellaridis_2008f, Tammana_2012j, Tanrikulu_2015e, fjbuilsRepeatedBehaviorDeliberate2024}, 17 cases (24\%) had a psychosocial motivation \cite{Akay_2015f, Benoist_2019e, Bhattacharjee_2008, Cauchi_2002, Goldman_1998f, Hardy_2023g, Kobiela_2015, Li_2013, Naji_2012f, Qureshi_2016, Riva_2018j, Sobnach_2011f, Tay_2004, Thapa_2019f, Tupesis_2004f, Wildhaber_2005, Wnęk_2015f}, 9 cases (12\%) were motivated by protest \cite{Bhumi_2024f, Gardner_2017h, Losanoff_1996, Losanoff_1997e, Tupesis_2004f}, 9 cases (12\%) had another documented motivation \cite{Ali_2020f, Ali_2022g, Emamhadi_2018, Guinan_2019f, Peixoto_2017f, Sakellaridis_2008f, Trgo_2012f, Wadhwa_2015e, Yildiz_2016e}. \paragraph*{Object Characteristics} 51 cases (71\%) ingested a large diameter object (\textgreater{}2.5cm) \cite{Akay_2015f, Al-Faham_2020k, AlShaaibi_2021b, Alao_2006i, Ali_2017, Ali_2022g, Apikotoa_2022f, Atayan_2016, Berry_2021e, Bhasin_2014, CamachoDorado_2018, Cauchi_2002, Chang_2017f, Cox_2007, Csaky_1998e, DivsalarP._2023a, Emamhadi_2018, Gardner_2017h, Guinan_2019f, Jehangir_2019h, Jin_2023, Kariholu_2008, Kerestes_2019, Kobiela_2015, Kumar_2001, Kumar_2019f, Losanoff_1996, Losanoff_1997e, Mesfin_2022a, Misra_2013, Naji_2012f, Ohno_2005, Peixoto_2017f, Qureshi_2016, Riva_2018j, Sakellaridis_2008f, Sultan_2024f, Tanrikulu_2015e, Thapa_2019f, Trgo_2012f, Wnęk_2015f, Yildiz_2016e, fjbuilsRepeatedBehaviorDeliberate2024, teWildt_2010}, 44 cases (61\%) ingested multiple objects \cite{Ali_2020f, Apikotoa_2022f, Ataya_2013, Atayan_2016, Beecroft_1998, Bhattacharjee_2008, Bhumi_2024f, CamachoDorado_2018, Cauchi_2002, Emamhadi_2018, Farhadi_2024h, Fry_2010, Goldman_1998f, Guinan_2019f, Hardy_2023g, Jehangir_2019h, Jin_2023, Kar_2015, Kariholu_2008, Kobiela_2015, Kumar_2001, Kumar_2019f, Li_2013, Liu_2005, Losanoff_1996, Mesfin_2022a, Misra_2013, Naji_2012f, Ohno_2005, Sobnach_2011f, Sultan_2024f, Tammana_2012j, Tanrikulu_2015e, Tay_2004, Thapa_2019f, Wadhwa_2015e, Wildhaber_2005, Yasin_2009, fjbuilsRepeatedBehaviorDeliberate2024, teWildt_2010}, 34 cases (47\%) ingested a sharp object \cite{AlShaaibi_2021b, Alao_2006i, Apikotoa_2022f, Ataya_2013, Benoist_2019e, Bhasin_2014, Bhattacharjee_2008, CamachoDorado_2018, Csaky_1998e, DelgadoSalazar_2020c, DivsalarP._2023a, Emamhadi_2018, Farhadi_2024h, Fry_2010, Guinan_2019f, Hardy_2023g, Jehangir_2019h, Jin_2023, Kariholu_2008, Kobiela_2015, Kumar_2019f, Losanoff_1996, Losanoff_1997e, Mesfin_2022a, Misra_2013, Sobnach_2011f, Yasin_2009, teWildt_2010}, 32 cases (44\%) ingested a long object (\textgreater{}5cm) \cite{Al-Faham_2020k, AlShaaibi_2021b, Ali_2017, Ali_2022g, Atayan_2016, Bhasin_2014, CamachoDorado_2018, Chang_2017f, Cox_2007, Csaky_1998e, DivsalarP._2023a, Emamhadi_2018, Fry_2010, Gardner_2017h, Jin_2023, Kariholu_2008, Kerestes_2019, Kobiela_2015, Kumar_2019f, Mesfin_2022a, Misra_2013, Ohno_2005, Qureshi_2016, Sakellaridis_2008f, Sultan_2024f, Thapa_2019f, Trgo_2012f, Yasin_2009, Yildiz_2016e, teWildt_2010}, 9 cases (12\%) ingested a magnet \cite{Ali_2020f, Bhumi_2024f, Cauchi_2002, Liu_2005, Naji_2012f, Ohno_2005, Tanrikulu_2015e, Tay_2004, Wildhaber_2005}, 2 cases (3\%) ingested a button battery \cite{Berry_2021e, Bhumi_2024f}. \paragraph*{Outcomes} 48 cases (67\%) experienced a complication \cite{Ali_2017, Ali_2020f, Apikotoa_2022f, Atayan_2016, Beecroft_1998, Benoist_2019e, Berry_2021e, Bhasin_2014, Bhumi_2024f, CamachoDorado_2018, Cauchi_2002, Cox_2007, Csaky_1998e, DelgadoSalazar_2020c, DivsalarP._2023a, Emamhadi_2018, Farhadi_2024h, Fry_2010, Gardner_2017h, Goldman_1998f, Jin_2023, Kariholu_2008, Kerestes_2019, Kobiela_2015, Kumar_2001, Kumar_2019f, Liu_2005, Losanoff_1996, Mesfin_2022a, Misra_2013, Naji_2012f, Ohno_2005, Sakellaridis_2008f, Sobnach_2011f, Sultan_2024f, Tanrikulu_2015e, Tay_2004, Thapa_2019f, Trgo_2012f, Tupesis_2004f, Wildhaber_2005, Wnęk_2015f, Yasin_2009, Yildiz_2016e}, 44 cases (61\%) underwent surgery \cite{Al-Faham_2020k, AlShaaibi_2021b, Alao_2006i, Ali_2017, Ali_2020f, Atayan_2016, Beecroft_1998, Bhasin_2014, CamachoDorado_2018, Cauchi_2002, Chang_2017f, Cox_2007, Csaky_1998e, DelgadoSalazar_2020c, DivsalarP._2023a, Farhadi_2024h, Fry_2010, Gardner_2017h, Jin_2023, Kariholu_2008, Kerestes_2019, Kobiela_2015, Kumar_2019f, Liu_2005, Losanoff_1996, Losanoff_1997e, Mesfin_2022a, Misra_2013, Naji_2012f, Sobnach_2011f, Tanrikulu_2015e, Tay_2004, Thapa_2019f, Tupesis_2004f, Wildhaber_2005, Wnęk_2015f, Yasin_2009, Yildiz_2016e, fjbuilsRepeatedBehaviorDeliberate2024}, 31 cases (43\%) underwent endoscopy \cite{Akay_2015f, Ali_2022g, Apikotoa_2022f, Atayan_2016, Benoist_2019e, Berry_2021e, Bhasin_2014, Bhumi_2024f, CamachoDorado_2018, Chang_2017f, DelgadoSalazar_2020c, Gardner_2017h, Guinan_2019f, Hardy_2023g, Jehangir_2019h, Kariholu_2008, Li_2013, Liu_2005, Ohno_2005, Peixoto_2017f, Qureshi_2016, Riva_2018j, Sakellaridis_2008f, Sultan_2024f, Tammana_2012j, Tanrikulu_2015e, Trgo_2012f, Wadhwa_2015e, Wnęk_2015f, teWildt_2010}, 7 cases (10\%) were managed conservatively \cite{Ataya_2013, Bhattacharjee_2008, DivsalarP._2023a, Emamhadi_2018, Goldman_1998f, Kar_2015, Kumar_2001}, 2 cases (3\%) died \cite{Emamhadi_2018, Kumar_2001}. All 90 were male gender. 90 cases (100\%) were detained at the time of ingestion \cite{Elghali_2016, Karp_1991b, Lee_2007}, 88 cases (98\%) were intentional ingestions \cite{Elghali_2016, Karp_1991b, Lee_2007}, 30 cases (33\%) had a psychiatric history documented \cite{Elghali_2016, Karp_1991b, Lee_2007}, 2 cases (2\%) had a history of prior ingestion \cite{Elghali_2016}. No cases were reported for were psychiatric inpatients, were displaced people, were under the influence of alcohol at the time of ingestion, and had a severe disability history.
\paragraph*{Motivation}  70 cases (78\%) reported protest motivation \cite{Elghali_2016, Karp_1991b, Lee_2007}, 12 cases (13\%) reported psychiatric motivation \cite{Karp_1991b}, 6 cases (7\%) reported self-harm motivation \cite{Elghali_2016, Karp_1991b}. No cases were reported for psychosocial motivation and other motivation.
\paragraph*{Object Characteristics}  68 cases (76\%) involved sharp object ingestion \cite{Elghali_2016, Karp_1991b, Lee_2007}, 32 cases (36\%) involved long (\textgreater 5cm) object ingestion \cite{Lee_2007}, 25 cases (28\%) involved ingestion of multiple objects \cite{Elghali_2016, Lee_2007}. No cases were reported for button battery ingestion, magnet ingestion, and involved large diameter (\textgreater 2.5cm) object ingestion.
\paragraph*{Outcomes}  47 cases (52\%) underwent endoscopic intervention \cite{Elghali_2016, Lee_2007}, 29 cases (32\%) were managed conservatively \cite{Elghali_2016, Karp_1991b}, 15 cases (17\%) underwent surgical intervention \cite{Elghali_2016, Karp_1991b, Lee_2007}, 6 cases (7\%) reported complications \cite{Lee_2007}, 1 case (1\%) died \cite{Elghali_2016}.
\paragraph*{Geographical Location}Cases were recorded in 33 countries: 13 cases from USA \cite{Alao_2006i, Ataya_2013, Bhumi_2024f, Fry_2010, Guinan_2019f, Hardy_2023g, Jehangir_2019h, Kerestes_2019, Kumar_2001, Liu_2005, Tammana_2012j, Tay_2004, Tupesis_2004f}; 7 cases from India \cite{Bhasin_2014, Bhattacharjee_2008, Kar_2015, Kariholu_2008, Kumar_2019f, Misra_2013, Wadhwa_2015e} and UK \cite{Beecroft_1998, Berry_2021e, Cauchi_2002, Cox_2007, Gardner_2017h, Qureshi_2016}; 6 cases from Bulgaria \cite{Losanoff_1996, Losanoff_1997e}; 5 cases from Iran \cite{DivsalarP._2023a, Emamhadi_2018, Farhadi_2024h}; 4 cases from Turkey \cite{Akay_2015f, Atayan_2016, Tanrikulu_2015e, Yildiz_2016e}; 2 cases from China \cite{Jin_2023, Li_2013}, Poland \cite{Kobiela_2015, Wnęk_2015f}, and Spain \cite{CamachoDorado_2018, fjbuilsRepeatedBehaviorDeliberate2024}; 1 case from Australia \cite{Apikotoa_2022f}, Bahrain \cite{Ali_2020f}, Croatia \cite{Trgo_2012f}, Ecuador \cite{DelgadoSalazar_2020c}, Egypt \cite{Ali_2022g}, Ethiopia \cite{Mesfin_2022a}, Germany \cite{teWildt_2010}, Greece \cite{Sakellaridis_2008f}, Hungary \cite{Csaky_1998e}, Iraq \cite{Al-Faham_2020k}, Israel \cite{Goldman_1998f}, Italy \cite{Riva_2018j}, Japan \cite{Ohno_2005}, Nepal \cite{Thapa_2019f}, Netherlands \cite{Benoist_2019e}, Oman \cite{AlShaaibi_2021b}, Pakistan \cite{Yasin_2009}, Portugal \cite{Peixoto_2017f}, Qatar \cite{Ali_2017}, Saudi Arabia \cite{Sultan_2024f}, South Africa \cite{Sobnach_2011f}, Sweden \cite{Naji_2012f}, Switzerland \cite{Wildhaber_2005}, and Taiwan \cite{Chang_2017f}. \paragraph*{Gender} 43 cases (60\%) were male \cite{Akay_2015f, Al-Faham_2020k, Alao_2006i, Ali_2017, Ali_2022g, Apikotoa_2022f, Atayan_2016, Benoist_2019e, Berry_2021e, Bhumi_2024f, CamachoDorado_2018, Csaky_1998e, Emamhadi_2018, Farhadi_2024h, Fry_2010, Gardner_2017h, Guinan_2019f, Jehangir_2019h, Jin_2023, Kobiela_2015, Kumar_2001, Kumar_2019f, Liu_2005, Losanoff_1996, Losanoff_1997e, Mesfin_2022a, Misra_2013, Qureshi_2016, Riva_2018j, Sobnach_2011f, Tammana_2012j, Tanrikulu_2015e, Tay_2004, Thapa_2019f, Trgo_2012f, Wadhwa_2015e, Yasin_2009, teWildt_2010}, 28 cases (39\%) were female \cite{AlShaaibi_2021b, Ali_2020f, Ataya_2013, Beecroft_1998, Bhasin_2014, Bhattacharjee_2008, Cauchi_2002, Chang_2017f, Cox_2007, DelgadoSalazar_2020c, DivsalarP._2023a, Goldman_1998f, Hardy_2023g, Kar_2015, Kariholu_2008, Kerestes_2019, Li_2013, Naji_2012f, Ohno_2005, Peixoto_2017f, Sakellaridis_2008f, Sultan_2024f, Tupesis_2004f, Wildhaber_2005, Wnęk_2015f, Yildiz_2016e}, 1 case (1\%) had no gender recorded \cite{fjbuilsRepeatedBehaviorDeliberate2024}. \paragraph*{Age Group} 25 cases (35\%) were between 26 and 40 years of age \cite{Alao_2006i, Ali_2022g, Apikotoa_2022f, Ataya_2013, Benoist_2019e, Bhasin_2014, Chang_2017f, Cox_2007, DelgadoSalazar_2020c, Farhadi_2024h, Fry_2010, Gardner_2017h, Guinan_2019f, Jin_2023, Kumar_2019f, Losanoff_1996, Misra_2013, Qureshi_2016, Riva_2018j, Sakellaridis_2008f, Tammana_2012j, Trgo_2012f, Wnęk_2015f, Yildiz_2016e, fjbuilsRepeatedBehaviorDeliberate2024}, 18 cases (25\%) were between 18 and 25 years of age \cite{Akay_2015f, Ali_2017, Atayan_2016, Bhattacharjee_2008, Csaky_1998e, Kar_2015, Kariholu_2008, Kobiela_2015, Losanoff_1996, Losanoff_1997e, Mesfin_2022a, Peixoto_2017f, Sobnach_2011f, Tupesis_2004f, Yasin_2009}, 13 cases (18\%) were under 18 years of age \cite{AlShaaibi_2021b, Ali_2020f, Cauchi_2002, DivsalarP._2023a, Goldman_1998f, Liu_2005, Naji_2012f, Ohno_2005, Tanrikulu_2015e, Tay_2004, Wildhaber_2005}, 11 cases (15\%) were between 41 and 60 years of age \cite{Al-Faham_2020k, Bhumi_2024f, CamachoDorado_2018, Emamhadi_2018, Hardy_2023g, Jehangir_2019h, Kumar_2001, Sultan_2024f, Thapa_2019f, Wadhwa_2015e, teWildt_2010}, 3 cases (4\%) were over 60 years of age \cite{Beecroft_1998, Kerestes_2019, Li_2013}, 2 cases (3\%) had no age documented \cite{Berry_2021e}. \paragraph*{Population} 36 cases (50\%) had a psychiatric history \cite{AlShaaibi_2021b, Alao_2006i, Ali_2020f, Apikotoa_2022f, Ataya_2013, Atayan_2016, Beecroft_1998, CamachoDorado_2018, Chang_2017f, DelgadoSalazar_2020c, DivsalarP._2023a, Farhadi_2024h, Fry_2010, Guinan_2019f, Hardy_2023g, Jehangir_2019h, Jin_2023, Kar_2015, Kerestes_2019, Kobiela_2015, Kumar_2001, Kumar_2019f, Liu_2005, Mesfin_2022a, Misra_2013, Ohno_2005, Peixoto_2017f, Sakellaridis_2008f, Sultan_2024f, Tammana_2012j, Tanrikulu_2015e, Yildiz_2016e, fjbuilsRepeatedBehaviorDeliberate2024, teWildt_2010}, 19 cases (26\%) had ingested previously \cite{Alao_2006i, Apikotoa_2022f, Berry_2021e, Bhattacharjee_2008, Csaky_1998e, DivsalarP._2023a, Emamhadi_2018, Guinan_2019f, Jehangir_2019h, Jin_2023, Liu_2005, Sakellaridis_2008f, Tanrikulu_2015e, Thapa_2019f, Yildiz_2016e, fjbuilsRepeatedBehaviorDeliberate2024, teWildt_2010}, 12 cases (17\%) were detained persons \cite{Alao_2006i, Ali_2022g, Apikotoa_2022f, Losanoff_1996, Losanoff_1997e, Qureshi_2016, Tammana_2012j, Trgo_2012f}, 7 cases (10\%) were severely disabled \cite{Atayan_2016, Kerestes_2019, Liu_2005, Ohno_2005, Peixoto_2017f, Yildiz_2016e, teWildt_2010}, 4 cases (6\%) were psychiatric inpatients \cite{DivsalarP._2023a, fjbuilsRepeatedBehaviorDeliberate2024, teWildt_2010}, 3 cases (4\%) were under the influence of alcohol \cite{Benoist_2019e, Csaky_1998e, Thapa_2019f}, 2 cases (3\%) were displaced people \cite{Akay_2015f, Gardner_2017h}. \paragraph*{Motivation} 34 cases (47\%) had a psychiatric motivation \cite{Al-Faham_2020k, Alao_2006i, Ali_2020f, Apikotoa_2022f, Ataya_2013, Atayan_2016, Bhasin_2014, Bhattacharjee_2008, DelgadoSalazar_2020c, DivsalarP._2023a, Emamhadi_2018, Farhadi_2024h, Guinan_2019f, Hardy_2023g, Jehangir_2019h, Jin_2023, Kar_2015, Kariholu_2008, Kerestes_2019, Kobiela_2015, Kumar_2001, Kumar_2019f, Li_2013, Liu_2005, Misra_2013, Ohno_2005, Sakellaridis_2008f, Sultan_2024f, Tammana_2012j, Tanrikulu_2015e, Yasin_2009, teWildt_2010}, 21 cases (29\%) were motivated by self-harm intention \cite{Al-Faham_2020k, AlShaaibi_2021b, Alao_2006i, Ali_2017, CamachoDorado_2018, Chang_2017f, Cox_2007, Csaky_1998e, Fry_2010, Li_2013, Losanoff_1996, Losanoff_1997e, Mesfin_2022a, Sakellaridis_2008f, Tammana_2012j, Tanrikulu_2015e, fjbuilsRepeatedBehaviorDeliberate2024}, 17 cases (24\%) had a psychosocial motivation \cite{Akay_2015f, Benoist_2019e, Bhattacharjee_2008, Cauchi_2002, Goldman_1998f, Hardy_2023g, Kobiela_2015, Li_2013, Naji_2012f, Qureshi_2016, Riva_2018j, Sobnach_2011f, Tay_2004, Thapa_2019f, Tupesis_2004f, Wildhaber_2005, Wnęk_2015f}, 9 cases (12\%) were motivated by protest \cite{Bhumi_2024f, Gardner_2017h, Losanoff_1996, Losanoff_1997e, Tupesis_2004f}, 9 cases (12\%) had another documented motivation \cite{Ali_2020f, Ali_2022g, Emamhadi_2018, Guinan_2019f, Peixoto_2017f, Sakellaridis_2008f, Trgo_2012f, Wadhwa_2015e, Yildiz_2016e}. \paragraph*{Object Characteristics} 51 cases (71\%) ingested a large diameter object (\textgreater{}2.5cm) \cite{Akay_2015f, Al-Faham_2020k, AlShaaibi_2021b, Alao_2006i, Ali_2017, Ali_2022g, Apikotoa_2022f, Atayan_2016, Berry_2021e, Bhasin_2014, CamachoDorado_2018, Cauchi_2002, Chang_2017f, Cox_2007, Csaky_1998e, DivsalarP._2023a, Emamhadi_2018, Gardner_2017h, Guinan_2019f, Jehangir_2019h, Jin_2023, Kariholu_2008, Kerestes_2019, Kobiela_2015, Kumar_2001, Kumar_2019f, Losanoff_1996, Losanoff_1997e, Mesfin_2022a, Misra_2013, Naji_2012f, Ohno_2005, Peixoto_2017f, Qureshi_2016, Riva_2018j, Sakellaridis_2008f, Sultan_2024f, Tanrikulu_2015e, Thapa_2019f, Trgo_2012f, Wnęk_2015f, Yildiz_2016e, fjbuilsRepeatedBehaviorDeliberate2024, teWildt_2010}, 44 cases (61\%) ingested multiple objects \cite{Ali_2020f, Apikotoa_2022f, Ataya_2013, Atayan_2016, Beecroft_1998, Bhattacharjee_2008, Bhumi_2024f, CamachoDorado_2018, Cauchi_2002, Emamhadi_2018, Farhadi_2024h, Fry_2010, Goldman_1998f, Guinan_2019f, Hardy_2023g, Jehangir_2019h, Jin_2023, Kar_2015, Kariholu_2008, Kobiela_2015, Kumar_2001, Kumar_2019f, Li_2013, Liu_2005, Losanoff_1996, Mesfin_2022a, Misra_2013, Naji_2012f, Ohno_2005, Sobnach_2011f, Sultan_2024f, Tammana_2012j, Tanrikulu_2015e, Tay_2004, Thapa_2019f, Wadhwa_2015e, Wildhaber_2005, Yasin_2009, fjbuilsRepeatedBehaviorDeliberate2024, teWildt_2010}, 34 cases (47\%) ingested a sharp object \cite{AlShaaibi_2021b, Alao_2006i, Apikotoa_2022f, Ataya_2013, Benoist_2019e, Bhasin_2014, Bhattacharjee_2008, CamachoDorado_2018, Csaky_1998e, DelgadoSalazar_2020c, DivsalarP._2023a, Emamhadi_2018, Farhadi_2024h, Fry_2010, Guinan_2019f, Hardy_2023g, Jehangir_2019h, Jin_2023, Kariholu_2008, Kobiela_2015, Kumar_2019f, Losanoff_1996, Losanoff_1997e, Mesfin_2022a, Misra_2013, Sobnach_2011f, Yasin_2009, teWildt_2010}, 32 cases (44\%) ingested a long object (\textgreater{}5cm) \cite{Al-Faham_2020k, AlShaaibi_2021b, Ali_2017, Ali_2022g, Atayan_2016, Bhasin_2014, CamachoDorado_2018, Chang_2017f, Cox_2007, Csaky_1998e, DivsalarP._2023a, Emamhadi_2018, Fry_2010, Gardner_2017h, Jin_2023, Kariholu_2008, Kerestes_2019, Kobiela_2015, Kumar_2019f, Mesfin_2022a, Misra_2013, Ohno_2005, Qureshi_2016, Sakellaridis_2008f, Sultan_2024f, Thapa_2019f, Trgo_2012f, Yasin_2009, Yildiz_2016e, teWildt_2010}, 9 cases (12\%) ingested a magnet \cite{Ali_2020f, Bhumi_2024f, Cauchi_2002, Liu_2005, Naji_2012f, Ohno_2005, Tanrikulu_2015e, Tay_2004, Wildhaber_2005}, 2 cases (3\%) ingested a button battery \cite{Berry_2021e, Bhumi_2024f}. \paragraph*{Outcomes} 48 cases (67\%) experienced a complication \cite{Ali_2017, Ali_2020f, Apikotoa_2022f, Atayan_2016, Beecroft_1998, Benoist_2019e, Berry_2021e, Bhasin_2014, Bhumi_2024f, CamachoDorado_2018, Cauchi_2002, Cox_2007, Csaky_1998e, DelgadoSalazar_2020c, DivsalarP._2023a, Emamhadi_2018, Farhadi_2024h, Fry_2010, Gardner_2017h, Goldman_1998f, Jin_2023, Kariholu_2008, Kerestes_2019, Kobiela_2015, Kumar_2001, Kumar_2019f, Liu_2005, Losanoff_1996, Mesfin_2022a, Misra_2013, Naji_2012f, Ohno_2005, Sakellaridis_2008f, Sobnach_2011f, Sultan_2024f, Tanrikulu_2015e, Tay_2004, Thapa_2019f, Trgo_2012f, Tupesis_2004f, Wildhaber_2005, Wnęk_2015f, Yasin_2009, Yildiz_2016e}, 44 cases (61\%) underwent surgery \cite{Al-Faham_2020k, AlShaaibi_2021b, Alao_2006i, Ali_2017, Ali_2020f, Atayan_2016, Beecroft_1998, Bhasin_2014, CamachoDorado_2018, Cauchi_2002, Chang_2017f, Cox_2007, Csaky_1998e, DelgadoSalazar_2020c, DivsalarP._2023a, Farhadi_2024h, Fry_2010, Gardner_2017h, Jin_2023, Kariholu_2008, Kerestes_2019, Kobiela_2015, Kumar_2019f, Liu_2005, Losanoff_1996, Losanoff_1997e, Mesfin_2022a, Misra_2013, Naji_2012f, Sobnach_2011f, Tanrikulu_2015e, Tay_2004, Thapa_2019f, Tupesis_2004f, Wildhaber_2005, Wnęk_2015f, Yasin_2009, Yildiz_2016e, fjbuilsRepeatedBehaviorDeliberate2024}, 31 cases (43\%) underwent endoscopy \cite{Akay_2015f, Ali_2022g, Apikotoa_2022f, Atayan_2016, Benoist_2019e, Berry_2021e, Bhasin_2014, Bhumi_2024f, CamachoDorado_2018, Chang_2017f, DelgadoSalazar_2020c, Gardner_2017h, Guinan_2019f, Hardy_2023g, Jehangir_2019h, Kariholu_2008, Li_2013, Liu_2005, Ohno_2005, Peixoto_2017f, Qureshi_2016, Riva_2018j, Sakellaridis_2008f, Sultan_2024f, Tammana_2012j, Tanrikulu_2015e, Trgo_2012f, Wadhwa_2015e, Wnęk_2015f, teWildt_2010}, 7 cases (10\%) were managed conservatively \cite{Ataya_2013, Bhattacharjee_2008, DivsalarP._2023a, Emamhadi_2018, Goldman_1998f, Kar_2015, Kumar_2001}, 2 cases (3\%) died \cite{Emamhadi_2018, Kumar_2001}. All 90 were male gender. 90 cases (100\%) were detained at the time of ingestion \cite{Elghali_2016, Karp_1991b, Lee_2007}, 88 cases (98\%) were intentional ingestions \cite{Elghali_2016, Karp_1991b, Lee_2007}, 30 cases (33\%) had a psychiatric history documented \cite{Elghali_2016, Karp_1991b, Lee_2007}, 2 cases (2\%) had a history of prior ingestion \cite{Elghali_2016}. No cases were reported for were psychiatric inpatients, were displaced people, were under the influence of alcohol at the time of ingestion, and had a severe disability history.
\paragraph*{Motivation}  70 cases (78\%) reported protest motivation \cite{Elghali_2016, Karp_1991b, Lee_2007}, 12 cases (13\%) reported psychiatric motivation \cite{Karp_1991b}, 6 cases (7\%) reported self-harm motivation \cite{Elghali_2016, Karp_1991b}. No cases were reported for psychosocial motivation and other motivation.
\paragraph*{Object Characteristics}  68 cases (76\%) involved sharp object ingestion \cite{Elghali_2016, Karp_1991b, Lee_2007}, 32 cases (36\%) involved long (\textgreater 5cm) object ingestion \cite{Lee_2007}, 25 cases (28\%) involved ingestion of multiple objects \cite{Elghali_2016, Lee_2007}. No cases were reported for button battery ingestion, magnet ingestion, and involved large diameter (\textgreater 2.5cm) object ingestion.
\paragraph*{Outcomes}  47 cases (52\%) underwent endoscopic intervention \cite{Elghali_2016, Lee_2007}, 29 cases (32\%) were managed conservatively \cite{Elghali_2016, Karp_1991b}, 15 cases (17\%) underwent surgical intervention \cite{Elghali_2016, Karp_1991b, Lee_2007}, 6 cases (7\%) reported complications \cite{Lee_2007}, 1 case (1\%) died \cite{Elghali_2016}.
\paragraph*{Geographical Location}Cases were recorded in 33 countries: 13 cases from USA \cite{Alao_2006i, Ataya_2013, Bhumi_2024f, Fry_2010, Guinan_2019f, Hardy_2023g, Jehangir_2019h, Kerestes_2019, Kumar_2001, Liu_2005, Tammana_2012j, Tay_2004, Tupesis_2004f}; 7 cases from India \cite{Bhasin_2014, Bhattacharjee_2008, Kar_2015, Kariholu_2008, Kumar_2019f, Misra_2013, Wadhwa_2015e} and UK \cite{Beecroft_1998, Berry_2021e, Cauchi_2002, Cox_2007, Gardner_2017h, Qureshi_2016}; 6 cases from Bulgaria \cite{Losanoff_1996, Losanoff_1997e}; 5 cases from Iran \cite{DivsalarP._2023a, Emamhadi_2018, Farhadi_2024h}; 4 cases from Turkey \cite{Akay_2015f, Atayan_2016, Tanrikulu_2015e, Yildiz_2016e}; 2 cases from China \cite{Jin_2023, Li_2013}, Poland \cite{Kobiela_2015, Wnęk_2015f}, and Spain \cite{CamachoDorado_2018, fjbuilsRepeatedBehaviorDeliberate2024}; 1 case from Australia \cite{Apikotoa_2022f}, Bahrain \cite{Ali_2020f}, Croatia \cite{Trgo_2012f}, Ecuador \cite{DelgadoSalazar_2020c}, Egypt \cite{Ali_2022g}, Ethiopia \cite{Mesfin_2022a}, Germany \cite{teWildt_2010}, Greece \cite{Sakellaridis_2008f}, Hungary \cite{Csaky_1998e}, Iraq \cite{Al-Faham_2020k}, Israel \cite{Goldman_1998f}, Italy \cite{Riva_2018j}, Japan \cite{Ohno_2005}, Nepal \cite{Thapa_2019f}, Netherlands \cite{Benoist_2019e}, Oman \cite{AlShaaibi_2021b}, Pakistan \cite{Yasin_2009}, Portugal \cite{Peixoto_2017f}, Qatar \cite{Ali_2017}, Saudi Arabia \cite{Sultan_2024f}, South Africa \cite{Sobnach_2011f}, Sweden \cite{Naji_2012f}, Switzerland \cite{Wildhaber_2005}, and Taiwan \cite{Chang_2017f}. \paragraph*{Gender} 43 cases (60\%) were male \cite{Akay_2015f, Al-Faham_2020k, Alao_2006i, Ali_2017, Ali_2022g, Apikotoa_2022f, Atayan_2016, Benoist_2019e, Berry_2021e, Bhumi_2024f, CamachoDorado_2018, Csaky_1998e, Emamhadi_2018, Farhadi_2024h, Fry_2010, Gardner_2017h, Guinan_2019f, Jehangir_2019h, Jin_2023, Kobiela_2015, Kumar_2001, Kumar_2019f, Liu_2005, Losanoff_1996, Losanoff_1997e, Mesfin_2022a, Misra_2013, Qureshi_2016, Riva_2018j, Sobnach_2011f, Tammana_2012j, Tanrikulu_2015e, Tay_2004, Thapa_2019f, Trgo_2012f, Wadhwa_2015e, Yasin_2009, teWildt_2010}, 28 cases (39\%) were female \cite{AlShaaibi_2021b, Ali_2020f, Ataya_2013, Beecroft_1998, Bhasin_2014, Bhattacharjee_2008, Cauchi_2002, Chang_2017f, Cox_2007, DelgadoSalazar_2020c, DivsalarP._2023a, Goldman_1998f, Hardy_2023g, Kar_2015, Kariholu_2008, Kerestes_2019, Li_2013, Naji_2012f, Ohno_2005, Peixoto_2017f, Sakellaridis_2008f, Sultan_2024f, Tupesis_2004f, Wildhaber_2005, Wnęk_2015f, Yildiz_2016e}, 1 case (1\%) had no gender recorded \cite{fjbuilsRepeatedBehaviorDeliberate2024}. \paragraph*{Age Group} 25 cases (35\%) were between 26 and 40 years of age \cite{Alao_2006i, Ali_2022g, Apikotoa_2022f, Ataya_2013, Benoist_2019e, Bhasin_2014, Chang_2017f, Cox_2007, DelgadoSalazar_2020c, Farhadi_2024h, Fry_2010, Gardner_2017h, Guinan_2019f, Jin_2023, Kumar_2019f, Losanoff_1996, Misra_2013, Qureshi_2016, Riva_2018j, Sakellaridis_2008f, Tammana_2012j, Trgo_2012f, Wnęk_2015f, Yildiz_2016e, fjbuilsRepeatedBehaviorDeliberate2024}, 18 cases (25\%) were between 18 and 25 years of age \cite{Akay_2015f, Ali_2017, Atayan_2016, Bhattacharjee_2008, Csaky_1998e, Kar_2015, Kariholu_2008, Kobiela_2015, Losanoff_1996, Losanoff_1997e, Mesfin_2022a, Peixoto_2017f, Sobnach_2011f, Tupesis_2004f, Yasin_2009}, 13 cases (18\%) were under 18 years of age \cite{AlShaaibi_2021b, Ali_2020f, Cauchi_2002, DivsalarP._2023a, Goldman_1998f, Liu_2005, Naji_2012f, Ohno_2005, Tanrikulu_2015e, Tay_2004, Wildhaber_2005}, 11 cases (15\%) were between 41 and 60 years of age \cite{Al-Faham_2020k, Bhumi_2024f, CamachoDorado_2018, Emamhadi_2018, Hardy_2023g, Jehangir_2019h, Kumar_2001, Sultan_2024f, Thapa_2019f, Wadhwa_2015e, teWildt_2010}, 3 cases (4\%) were over 60 years of age \cite{Beecroft_1998, Kerestes_2019, Li_2013}, 2 cases (3\%) had no age documented \cite{Berry_2021e}. \paragraph*{Population} 36 cases (50\%) had a psychiatric history \cite{AlShaaibi_2021b, Alao_2006i, Ali_2020f, Apikotoa_2022f, Ataya_2013, Atayan_2016, Beecroft_1998, CamachoDorado_2018, Chang_2017f, DelgadoSalazar_2020c, DivsalarP._2023a, Farhadi_2024h, Fry_2010, Guinan_2019f, Hardy_2023g, Jehangir_2019h, Jin_2023, Kar_2015, Kerestes_2019, Kobiela_2015, Kumar_2001, Kumar_2019f, Liu_2005, Mesfin_2022a, Misra_2013, Ohno_2005, Peixoto_2017f, Sakellaridis_2008f, Sultan_2024f, Tammana_2012j, Tanrikulu_2015e, Yildiz_2016e, fjbuilsRepeatedBehaviorDeliberate2024, teWildt_2010}, 19 cases (26\%) had ingested previously \cite{Alao_2006i, Apikotoa_2022f, Berry_2021e, Bhattacharjee_2008, Csaky_1998e, DivsalarP._2023a, Emamhadi_2018, Guinan_2019f, Jehangir_2019h, Jin_2023, Liu_2005, Sakellaridis_2008f, Tanrikulu_2015e, Thapa_2019f, Yildiz_2016e, fjbuilsRepeatedBehaviorDeliberate2024, teWildt_2010}, 12 cases (17\%) were detained persons \cite{Alao_2006i, Ali_2022g, Apikotoa_2022f, Losanoff_1996, Losanoff_1997e, Qureshi_2016, Tammana_2012j, Trgo_2012f}, 7 cases (10\%) were severely disabled \cite{Atayan_2016, Kerestes_2019, Liu_2005, Ohno_2005, Peixoto_2017f, Yildiz_2016e, teWildt_2010}, 4 cases (6\%) were psychiatric inpatients \cite{DivsalarP._2023a, fjbuilsRepeatedBehaviorDeliberate2024, teWildt_2010}, 3 cases (4\%) were under the influence of alcohol \cite{Benoist_2019e, Csaky_1998e, Thapa_2019f}, 2 cases (3\%) were displaced people \cite{Akay_2015f, Gardner_2017h}. \paragraph*{Motivation} 34 cases (47\%) had a psychiatric motivation \cite{Al-Faham_2020k, Alao_2006i, Ali_2020f, Apikotoa_2022f, Ataya_2013, Atayan_2016, Bhasin_2014, Bhattacharjee_2008, DelgadoSalazar_2020c, DivsalarP._2023a, Emamhadi_2018, Farhadi_2024h, Guinan_2019f, Hardy_2023g, Jehangir_2019h, Jin_2023, Kar_2015, Kariholu_2008, Kerestes_2019, Kobiela_2015, Kumar_2001, Kumar_2019f, Li_2013, Liu_2005, Misra_2013, Ohno_2005, Sakellaridis_2008f, Sultan_2024f, Tammana_2012j, Tanrikulu_2015e, Yasin_2009, teWildt_2010}, 21 cases (29\%) were motivated by self-harm intention \cite{Al-Faham_2020k, AlShaaibi_2021b, Alao_2006i, Ali_2017, CamachoDorado_2018, Chang_2017f, Cox_2007, Csaky_1998e, Fry_2010, Li_2013, Losanoff_1996, Losanoff_1997e, Mesfin_2022a, Sakellaridis_2008f, Tammana_2012j, Tanrikulu_2015e, fjbuilsRepeatedBehaviorDeliberate2024}, 17 cases (24\%) had a psychosocial motivation \cite{Akay_2015f, Benoist_2019e, Bhattacharjee_2008, Cauchi_2002, Goldman_1998f, Hardy_2023g, Kobiela_2015, Li_2013, Naji_2012f, Qureshi_2016, Riva_2018j, Sobnach_2011f, Tay_2004, Thapa_2019f, Tupesis_2004f, Wildhaber_2005, Wnęk_2015f}, 9 cases (12\%) were motivated by protest \cite{Bhumi_2024f, Gardner_2017h, Losanoff_1996, Losanoff_1997e, Tupesis_2004f}, 9 cases (12\%) had another documented motivation \cite{Ali_2020f, Ali_2022g, Emamhadi_2018, Guinan_2019f, Peixoto_2017f, Sakellaridis_2008f, Trgo_2012f, Wadhwa_2015e, Yildiz_2016e}. \paragraph*{Object Characteristics} 51 cases (71\%) ingested a large diameter object (\textgreater{}2.5cm) \cite{Akay_2015f, Al-Faham_2020k, AlShaaibi_2021b, Alao_2006i, Ali_2017, Ali_2022g, Apikotoa_2022f, Atayan_2016, Berry_2021e, Bhasin_2014, CamachoDorado_2018, Cauchi_2002, Chang_2017f, Cox_2007, Csaky_1998e, DivsalarP._2023a, Emamhadi_2018, Gardner_2017h, Guinan_2019f, Jehangir_2019h, Jin_2023, Kariholu_2008, Kerestes_2019, Kobiela_2015, Kumar_2001, Kumar_2019f, Losanoff_1996, Losanoff_1997e, Mesfin_2022a, Misra_2013, Naji_2012f, Ohno_2005, Peixoto_2017f, Qureshi_2016, Riva_2018j, Sakellaridis_2008f, Sultan_2024f, Tanrikulu_2015e, Thapa_2019f, Trgo_2012f, Wnęk_2015f, Yildiz_2016e, fjbuilsRepeatedBehaviorDeliberate2024, teWildt_2010}, 44 cases (61\%) ingested multiple objects \cite{Ali_2020f, Apikotoa_2022f, Ataya_2013, Atayan_2016, Beecroft_1998, Bhattacharjee_2008, Bhumi_2024f, CamachoDorado_2018, Cauchi_2002, Emamhadi_2018, Farhadi_2024h, Fry_2010, Goldman_1998f, Guinan_2019f, Hardy_2023g, Jehangir_2019h, Jin_2023, Kar_2015, Kariholu_2008, Kobiela_2015, Kumar_2001, Kumar_2019f, Li_2013, Liu_2005, Losanoff_1996, Mesfin_2022a, Misra_2013, Naji_2012f, Ohno_2005, Sobnach_2011f, Sultan_2024f, Tammana_2012j, Tanrikulu_2015e, Tay_2004, Thapa_2019f, Wadhwa_2015e, Wildhaber_2005, Yasin_2009, fjbuilsRepeatedBehaviorDeliberate2024, teWildt_2010}, 34 cases (47\%) ingested a sharp object \cite{AlShaaibi_2021b, Alao_2006i, Apikotoa_2022f, Ataya_2013, Benoist_2019e, Bhasin_2014, Bhattacharjee_2008, CamachoDorado_2018, Csaky_1998e, DelgadoSalazar_2020c, DivsalarP._2023a, Emamhadi_2018, Farhadi_2024h, Fry_2010, Guinan_2019f, Hardy_2023g, Jehangir_2019h, Jin_2023, Kariholu_2008, Kobiela_2015, Kumar_2019f, Losanoff_1996, Losanoff_1997e, Mesfin_2022a, Misra_2013, Sobnach_2011f, Yasin_2009, teWildt_2010}, 32 cases (44\%) ingested a long object (\textgreater{}5cm) \cite{Al-Faham_2020k, AlShaaibi_2021b, Ali_2017, Ali_2022g, Atayan_2016, Bhasin_2014, CamachoDorado_2018, Chang_2017f, Cox_2007, Csaky_1998e, DivsalarP._2023a, Emamhadi_2018, Fry_2010, Gardner_2017h, Jin_2023, Kariholu_2008, Kerestes_2019, Kobiela_2015, Kumar_2019f, Mesfin_2022a, Misra_2013, Ohno_2005, Qureshi_2016, Sakellaridis_2008f, Sultan_2024f, Thapa_2019f, Trgo_2012f, Yasin_2009, Yildiz_2016e, teWildt_2010}, 9 cases (12\%) ingested a magnet \cite{Ali_2020f, Bhumi_2024f, Cauchi_2002, Liu_2005, Naji_2012f, Ohno_2005, Tanrikulu_2015e, Tay_2004, Wildhaber_2005}, 2 cases (3\%) ingested a button battery \cite{Berry_2021e, Bhumi_2024f}. \paragraph*{Outcomes} 48 cases (67\%) experienced a complication \cite{Ali_2017, Ali_2020f, Apikotoa_2022f, Atayan_2016, Beecroft_1998, Benoist_2019e, Berry_2021e, Bhasin_2014, Bhumi_2024f, CamachoDorado_2018, Cauchi_2002, Cox_2007, Csaky_1998e, DelgadoSalazar_2020c, DivsalarP._2023a, Emamhadi_2018, Farhadi_2024h, Fry_2010, Gardner_2017h, Goldman_1998f, Jin_2023, Kariholu_2008, Kerestes_2019, Kobiela_2015, Kumar_2001, Kumar_2019f, Liu_2005, Losanoff_1996, Mesfin_2022a, Misra_2013, Naji_2012f, Ohno_2005, Sakellaridis_2008f, Sobnach_2011f, Sultan_2024f, Tanrikulu_2015e, Tay_2004, Thapa_2019f, Trgo_2012f, Tupesis_2004f, Wildhaber_2005, Wnęk_2015f, Yasin_2009, Yildiz_2016e}, 44 cases (61\%) underwent surgery \cite{Al-Faham_2020k, AlShaaibi_2021b, Alao_2006i, Ali_2017, Ali_2020f, Atayan_2016, Beecroft_1998, Bhasin_2014, CamachoDorado_2018, Cauchi_2002, Chang_2017f, Cox_2007, Csaky_1998e, DelgadoSalazar_2020c, DivsalarP._2023a, Farhadi_2024h, Fry_2010, Gardner_2017h, Jin_2023, Kariholu_2008, Kerestes_2019, Kobiela_2015, Kumar_2019f, Liu_2005, Losanoff_1996, Losanoff_1997e, Mesfin_2022a, Misra_2013, Naji_2012f, Sobnach_2011f, Tanrikulu_2015e, Tay_2004, Thapa_2019f, Tupesis_2004f, Wildhaber_2005, Wnęk_2015f, Yasin_2009, Yildiz_2016e, fjbuilsRepeatedBehaviorDeliberate2024}, 31 cases (43\%) underwent endoscopy \cite{Akay_2015f, Ali_2022g, Apikotoa_2022f, Atayan_2016, Benoist_2019e, Berry_2021e, Bhasin_2014, Bhumi_2024f, CamachoDorado_2018, Chang_2017f, DelgadoSalazar_2020c, Gardner_2017h, Guinan_2019f, Hardy_2023g, Jehangir_2019h, Kariholu_2008, Li_2013, Liu_2005, Ohno_2005, Peixoto_2017f, Qureshi_2016, Riva_2018j, Sakellaridis_2008f, Sultan_2024f, Tammana_2012j, Tanrikulu_2015e, Trgo_2012f, Wadhwa_2015e, Wnęk_2015f, teWildt_2010}, 7 cases (10\%) were managed conservatively \cite{Ataya_2013, Bhattacharjee_2008, DivsalarP._2023a, Emamhadi_2018, Goldman_1998f, Kar_2015, Kumar_2001}, 2 cases (3\%) died \cite{Emamhadi_2018, Kumar_2001}. All 90 were male gender. 90 cases (100\%) were detained at the time of ingestion \cite{Elghali_2016, Karp_1991b, Lee_2007}, 88 cases (98\%) were intentional ingestions \cite{Elghali_2016, Karp_1991b, Lee_2007}, 30 cases (33\%) had a psychiatric history documented \cite{Elghali_2016, Karp_1991b, Lee_2007}, 2 cases (2\%) had a history of prior ingestion \cite{Elghali_2016}. No cases were reported for were psychiatric inpatients, were displaced people, were under the influence of alcohol at the time of ingestion, and had a severe disability history.
\paragraph*{Motivation}  70 cases (78\%) reported protest motivation \cite{Elghali_2016, Karp_1991b, Lee_2007}, 12 cases (13\%) reported psychiatric motivation \cite{Karp_1991b}, 6 cases (7\%) reported self-harm motivation \cite{Elghali_2016, Karp_1991b}. No cases were reported for psychosocial motivation and other motivation.
\paragraph*{Object Characteristics}  68 cases (76\%) involved sharp object ingestion \cite{Elghali_2016, Karp_1991b, Lee_2007}, 32 cases (36\%) involved long (\textgreater 5cm) object ingestion \cite{Lee_2007}, 25 cases (28\%) involved ingestion of multiple objects \cite{Elghali_2016, Lee_2007}. No cases were reported for button battery ingestion, magnet ingestion, and involved large diameter (\textgreater 2.5cm) object ingestion.
\paragraph*{Outcomes}  47 cases (52\%) underwent endoscopic intervention \cite{Elghali_2016, Lee_2007}, 29 cases (32\%) were managed conservatively \cite{Elghali_2016, Karp_1991b}, 15 cases (17\%) underwent surgical intervention \cite{Elghali_2016, Karp_1991b, Lee_2007}, 6 cases (7\%) reported complications \cite{Lee_2007}, 1 case (1\%) died \cite{Elghali_2016}.
\paragraph*{Geographical Location}Cases were recorded in 33 countries: 13 cases from USA \cite{Alao_2006i, Ataya_2013, Bhumi_2024f, Fry_2010, Guinan_2019f, Hardy_2023g, Jehangir_2019h, Kerestes_2019, Kumar_2001, Liu_2005, Tammana_2012j, Tay_2004, Tupesis_2004f}; 7 cases from India \cite{Bhasin_2014, Bhattacharjee_2008, Kar_2015, Kariholu_2008, Kumar_2019f, Misra_2013, Wadhwa_2015e} and UK \cite{Beecroft_1998, Berry_2021e, Cauchi_2002, Cox_2007, Gardner_2017h, Qureshi_2016}; 6 cases from Bulgaria \cite{Losanoff_1996, Losanoff_1997e}; 5 cases from Iran \cite{DivsalarP._2023a, Emamhadi_2018, Farhadi_2024h}; 4 cases from Turkey \cite{Akay_2015f, Atayan_2016, Tanrikulu_2015e, Yildiz_2016e}; 2 cases from China \cite{Jin_2023, Li_2013}, Poland \cite{Kobiela_2015, Wnęk_2015f}, and Spain \cite{CamachoDorado_2018, fjbuilsRepeatedBehaviorDeliberate2024}; 1 case from Australia \cite{Apikotoa_2022f}, Bahrain \cite{Ali_2020f}, Croatia \cite{Trgo_2012f}, Ecuador \cite{DelgadoSalazar_2020c}, Egypt \cite{Ali_2022g}, Ethiopia \cite{Mesfin_2022a}, Germany \cite{teWildt_2010}, Greece \cite{Sakellaridis_2008f}, Hungary \cite{Csaky_1998e}, Iraq \cite{Al-Faham_2020k}, Israel \cite{Goldman_1998f}, Italy \cite{Riva_2018j}, Japan \cite{Ohno_2005}, Nepal \cite{Thapa_2019f}, Netherlands \cite{Benoist_2019e}, Oman \cite{AlShaaibi_2021b}, Pakistan \cite{Yasin_2009}, Portugal \cite{Peixoto_2017f}, Qatar \cite{Ali_2017}, Saudi Arabia \cite{Sultan_2024f}, South Africa \cite{Sobnach_2011f}, Sweden \cite{Naji_2012f}, Switzerland \cite{Wildhaber_2005}, and Taiwan \cite{Chang_2017f}. \paragraph*{Gender} 43 cases (60\%) were male \cite{Akay_2015f, Al-Faham_2020k, Alao_2006i, Ali_2017, Ali_2022g, Apikotoa_2022f, Atayan_2016, Benoist_2019e, Berry_2021e, Bhumi_2024f, CamachoDorado_2018, Csaky_1998e, Emamhadi_2018, Farhadi_2024h, Fry_2010, Gardner_2017h, Guinan_2019f, Jehangir_2019h, Jin_2023, Kobiela_2015, Kumar_2001, Kumar_2019f, Liu_2005, Losanoff_1996, Losanoff_1997e, Mesfin_2022a, Misra_2013, Qureshi_2016, Riva_2018j, Sobnach_2011f, Tammana_2012j, Tanrikulu_2015e, Tay_2004, Thapa_2019f, Trgo_2012f, Wadhwa_2015e, Yasin_2009, teWildt_2010}, 28 cases (39\%) were female \cite{AlShaaibi_2021b, Ali_2020f, Ataya_2013, Beecroft_1998, Bhasin_2014, Bhattacharjee_2008, Cauchi_2002, Chang_2017f, Cox_2007, DelgadoSalazar_2020c, DivsalarP._2023a, Goldman_1998f, Hardy_2023g, Kar_2015, Kariholu_2008, Kerestes_2019, Li_2013, Naji_2012f, Ohno_2005, Peixoto_2017f, Sakellaridis_2008f, Sultan_2024f, Tupesis_2004f, Wildhaber_2005, Wnęk_2015f, Yildiz_2016e}, 1 case (1\%) had no gender recorded \cite{fjbuilsRepeatedBehaviorDeliberate2024}. \paragraph*{Age Group} 25 cases (35\%) were between 26 and 40 years of age \cite{Alao_2006i, Ali_2022g, Apikotoa_2022f, Ataya_2013, Benoist_2019e, Bhasin_2014, Chang_2017f, Cox_2007, DelgadoSalazar_2020c, Farhadi_2024h, Fry_2010, Gardner_2017h, Guinan_2019f, Jin_2023, Kumar_2019f, Losanoff_1996, Misra_2013, Qureshi_2016, Riva_2018j, Sakellaridis_2008f, Tammana_2012j, Trgo_2012f, Wnęk_2015f, Yildiz_2016e, fjbuilsRepeatedBehaviorDeliberate2024}, 18 cases (25\%) were between 18 and 25 years of age \cite{Akay_2015f, Ali_2017, Atayan_2016, Bhattacharjee_2008, Csaky_1998e, Kar_2015, Kariholu_2008, Kobiela_2015, Losanoff_1996, Losanoff_1997e, Mesfin_2022a, Peixoto_2017f, Sobnach_2011f, Tupesis_2004f, Yasin_2009}, 13 cases (18\%) were under 18 years of age \cite{AlShaaibi_2021b, Ali_2020f, Cauchi_2002, DivsalarP._2023a, Goldman_1998f, Liu_2005, Naji_2012f, Ohno_2005, Tanrikulu_2015e, Tay_2004, Wildhaber_2005}, 11 cases (15\%) were between 41 and 60 years of age \cite{Al-Faham_2020k, Bhumi_2024f, CamachoDorado_2018, Emamhadi_2018, Hardy_2023g, Jehangir_2019h, Kumar_2001, Sultan_2024f, Thapa_2019f, Wadhwa_2015e, teWildt_2010}, 3 cases (4\%) were over 60 years of age \cite{Beecroft_1998, Kerestes_2019, Li_2013}, 2 cases (3\%) had no age documented \cite{Berry_2021e}. \paragraph*{Population} 36 cases (50\%) had a psychiatric history \cite{AlShaaibi_2021b, Alao_2006i, Ali_2020f, Apikotoa_2022f, Ataya_2013, Atayan_2016, Beecroft_1998, CamachoDorado_2018, Chang_2017f, DelgadoSalazar_2020c, DivsalarP._2023a, Farhadi_2024h, Fry_2010, Guinan_2019f, Hardy_2023g, Jehangir_2019h, Jin_2023, Kar_2015, Kerestes_2019, Kobiela_2015, Kumar_2001, Kumar_2019f, Liu_2005, Mesfin_2022a, Misra_2013, Ohno_2005, Peixoto_2017f, Sakellaridis_2008f, Sultan_2024f, Tammana_2012j, Tanrikulu_2015e, Yildiz_2016e, fjbuilsRepeatedBehaviorDeliberate2024, teWildt_2010}, 19 cases (26\%) had ingested previously \cite{Alao_2006i, Apikotoa_2022f, Berry_2021e, Bhattacharjee_2008, Csaky_1998e, DivsalarP._2023a, Emamhadi_2018, Guinan_2019f, Jehangir_2019h, Jin_2023, Liu_2005, Sakellaridis_2008f, Tanrikulu_2015e, Thapa_2019f, Yildiz_2016e, fjbuilsRepeatedBehaviorDeliberate2024, teWildt_2010}, 12 cases (17\%) were detained persons \cite{Alao_2006i, Ali_2022g, Apikotoa_2022f, Losanoff_1996, Losanoff_1997e, Qureshi_2016, Tammana_2012j, Trgo_2012f}, 7 cases (10\%) were severely disabled \cite{Atayan_2016, Kerestes_2019, Liu_2005, Ohno_2005, Peixoto_2017f, Yildiz_2016e, teWildt_2010}, 4 cases (6\%) were psychiatric inpatients \cite{DivsalarP._2023a, fjbuilsRepeatedBehaviorDeliberate2024, teWildt_2010}, 3 cases (4\%) were under the influence of alcohol \cite{Benoist_2019e, Csaky_1998e, Thapa_2019f}, 2 cases (3\%) were displaced people \cite{Akay_2015f, Gardner_2017h}. \paragraph*{Motivation} 34 cases (47\%) had a psychiatric motivation \cite{Al-Faham_2020k, Alao_2006i, Ali_2020f, Apikotoa_2022f, Ataya_2013, Atayan_2016, Bhasin_2014, Bhattacharjee_2008, DelgadoSalazar_2020c, DivsalarP._2023a, Emamhadi_2018, Farhadi_2024h, Guinan_2019f, Hardy_2023g, Jehangir_2019h, Jin_2023, Kar_2015, Kariholu_2008, Kerestes_2019, Kobiela_2015, Kumar_2001, Kumar_2019f, Li_2013, Liu_2005, Misra_2013, Ohno_2005, Sakellaridis_2008f, Sultan_2024f, Tammana_2012j, Tanrikulu_2015e, Yasin_2009, teWildt_2010}, 21 cases (29\%) were motivated by self-harm intention \cite{Al-Faham_2020k, AlShaaibi_2021b, Alao_2006i, Ali_2017, CamachoDorado_2018, Chang_2017f, Cox_2007, Csaky_1998e, Fry_2010, Li_2013, Losanoff_1996, Losanoff_1997e, Mesfin_2022a, Sakellaridis_2008f, Tammana_2012j, Tanrikulu_2015e, fjbuilsRepeatedBehaviorDeliberate2024}, 17 cases (24\%) had a psychosocial motivation \cite{Akay_2015f, Benoist_2019e, Bhattacharjee_2008, Cauchi_2002, Goldman_1998f, Hardy_2023g, Kobiela_2015, Li_2013, Naji_2012f, Qureshi_2016, Riva_2018j, Sobnach_2011f, Tay_2004, Thapa_2019f, Tupesis_2004f, Wildhaber_2005, Wnęk_2015f}, 9 cases (12\%) were motivated by protest \cite{Bhumi_2024f, Gardner_2017h, Losanoff_1996, Losanoff_1997e, Tupesis_2004f}, 9 cases (12\%) had another documented motivation \cite{Ali_2020f, Ali_2022g, Emamhadi_2018, Guinan_2019f, Peixoto_2017f, Sakellaridis_2008f, Trgo_2012f, Wadhwa_2015e, Yildiz_2016e}. \paragraph*{Object Characteristics} 51 cases (71\%) ingested a large diameter object (\textgreater{}2.5cm) \cite{Akay_2015f, Al-Faham_2020k, AlShaaibi_2021b, Alao_2006i, Ali_2017, Ali_2022g, Apikotoa_2022f, Atayan_2016, Berry_2021e, Bhasin_2014, CamachoDorado_2018, Cauchi_2002, Chang_2017f, Cox_2007, Csaky_1998e, DivsalarP._2023a, Emamhadi_2018, Gardner_2017h, Guinan_2019f, Jehangir_2019h, Jin_2023, Kariholu_2008, Kerestes_2019, Kobiela_2015, Kumar_2001, Kumar_2019f, Losanoff_1996, Losanoff_1997e, Mesfin_2022a, Misra_2013, Naji_2012f, Ohno_2005, Peixoto_2017f, Qureshi_2016, Riva_2018j, Sakellaridis_2008f, Sultan_2024f, Tanrikulu_2015e, Thapa_2019f, Trgo_2012f, Wnęk_2015f, Yildiz_2016e, fjbuilsRepeatedBehaviorDeliberate2024, teWildt_2010}, 44 cases (61\%) ingested multiple objects \cite{Ali_2020f, Apikotoa_2022f, Ataya_2013, Atayan_2016, Beecroft_1998, Bhattacharjee_2008, Bhumi_2024f, CamachoDorado_2018, Cauchi_2002, Emamhadi_2018, Farhadi_2024h, Fry_2010, Goldman_1998f, Guinan_2019f, Hardy_2023g, Jehangir_2019h, Jin_2023, Kar_2015, Kariholu_2008, Kobiela_2015, Kumar_2001, Kumar_2019f, Li_2013, Liu_2005, Losanoff_1996, Mesfin_2022a, Misra_2013, Naji_2012f, Ohno_2005, Sobnach_2011f, Sultan_2024f, Tammana_2012j, Tanrikulu_2015e, Tay_2004, Thapa_2019f, Wadhwa_2015e, Wildhaber_2005, Yasin_2009, fjbuilsRepeatedBehaviorDeliberate2024, teWildt_2010}, 34 cases (47\%) ingested a sharp object \cite{AlShaaibi_2021b, Alao_2006i, Apikotoa_2022f, Ataya_2013, Benoist_2019e, Bhasin_2014, Bhattacharjee_2008, CamachoDorado_2018, Csaky_1998e, DelgadoSalazar_2020c, DivsalarP._2023a, Emamhadi_2018, Farhadi_2024h, Fry_2010, Guinan_2019f, Hardy_2023g, Jehangir_2019h, Jin_2023, Kariholu_2008, Kobiela_2015, Kumar_2019f, Losanoff_1996, Losanoff_1997e, Mesfin_2022a, Misra_2013, Sobnach_2011f, Yasin_2009, teWildt_2010}, 32 cases (44\%) ingested a long object (\textgreater{}5cm) \cite{Al-Faham_2020k, AlShaaibi_2021b, Ali_2017, Ali_2022g, Atayan_2016, Bhasin_2014, CamachoDorado_2018, Chang_2017f, Cox_2007, Csaky_1998e, DivsalarP._2023a, Emamhadi_2018, Fry_2010, Gardner_2017h, Jin_2023, Kariholu_2008, Kerestes_2019, Kobiela_2015, Kumar_2019f, Mesfin_2022a, Misra_2013, Ohno_2005, Qureshi_2016, Sakellaridis_2008f, Sultan_2024f, Thapa_2019f, Trgo_2012f, Yasin_2009, Yildiz_2016e, teWildt_2010}, 9 cases (12\%) ingested a magnet \cite{Ali_2020f, Bhumi_2024f, Cauchi_2002, Liu_2005, Naji_2012f, Ohno_2005, Tanrikulu_2015e, Tay_2004, Wildhaber_2005}, 2 cases (3\%) ingested a button battery \cite{Berry_2021e, Bhumi_2024f}. \paragraph*{Outcomes} 48 cases (67\%) experienced a complication \cite{Ali_2017, Ali_2020f, Apikotoa_2022f, Atayan_2016, Beecroft_1998, Benoist_2019e, Berry_2021e, Bhasin_2014, Bhumi_2024f, CamachoDorado_2018, Cauchi_2002, Cox_2007, Csaky_1998e, DelgadoSalazar_2020c, DivsalarP._2023a, Emamhadi_2018, Farhadi_2024h, Fry_2010, Gardner_2017h, Goldman_1998f, Jin_2023, Kariholu_2008, Kerestes_2019, Kobiela_2015, Kumar_2001, Kumar_2019f, Liu_2005, Losanoff_1996, Mesfin_2022a, Misra_2013, Naji_2012f, Ohno_2005, Sakellaridis_2008f, Sobnach_2011f, Sultan_2024f, Tanrikulu_2015e, Tay_2004, Thapa_2019f, Trgo_2012f, Tupesis_2004f, Wildhaber_2005, Wnęk_2015f, Yasin_2009, Yildiz_2016e}, 44 cases (61\%) underwent surgery \cite{Al-Faham_2020k, AlShaaibi_2021b, Alao_2006i, Ali_2017, Ali_2020f, Atayan_2016, Beecroft_1998, Bhasin_2014, CamachoDorado_2018, Cauchi_2002, Chang_2017f, Cox_2007, Csaky_1998e, DelgadoSalazar_2020c, DivsalarP._2023a, Farhadi_2024h, Fry_2010, Gardner_2017h, Jin_2023, Kariholu_2008, Kerestes_2019, Kobiela_2015, Kumar_2019f, Liu_2005, Losanoff_1996, Losanoff_1997e, Mesfin_2022a, Misra_2013, Naji_2012f, Sobnach_2011f, Tanrikulu_2015e, Tay_2004, Thapa_2019f, Tupesis_2004f, Wildhaber_2005, Wnęk_2015f, Yasin_2009, Yildiz_2016e, fjbuilsRepeatedBehaviorDeliberate2024}, 31 cases (43\%) underwent endoscopy \cite{Akay_2015f, Ali_2022g, Apikotoa_2022f, Atayan_2016, Benoist_2019e, Berry_2021e, Bhasin_2014, Bhumi_2024f, CamachoDorado_2018, Chang_2017f, DelgadoSalazar_2020c, Gardner_2017h, Guinan_2019f, Hardy_2023g, Jehangir_2019h, Kariholu_2008, Li_2013, Liu_2005, Ohno_2005, Peixoto_2017f, Qureshi_2016, Riva_2018j, Sakellaridis_2008f, Sultan_2024f, Tammana_2012j, Tanrikulu_2015e, Trgo_2012f, Wadhwa_2015e, Wnęk_2015f, teWildt_2010}, 7 cases (10\%) were managed conservatively \cite{Ataya_2013, Bhattacharjee_2008, DivsalarP._2023a, Emamhadi_2018, Goldman_1998f, Kar_2015, Kumar_2001}, 2 cases (3\%) died \cite{Emamhadi_2018, Kumar_2001}. All 90 were male gender. 90 cases (100\%) were detained at the time of ingestion \cite{Elghali_2016, Karp_1991b, Lee_2007}, 88 cases (98\%) were intentional ingestions \cite{Elghali_2016, Karp_1991b, Lee_2007}, 30 cases (33\%) had a psychiatric history documented \cite{Elghali_2016, Karp_1991b, Lee_2007}, 2 cases (2\%) had a history of prior ingestion \cite{Elghali_2016}. No cases were reported for were psychiatric inpatients, were displaced people, were under the influence of alcohol at the time of ingestion, and had a severe disability history.
\paragraph*{Motivation}  70 cases (78\%) reported protest motivation \cite{Elghali_2016, Karp_1991b, Lee_2007}, 12 cases (13\%) reported psychiatric motivation \cite{Karp_1991b}, 6 cases (7\%) reported self-harm motivation \cite{Elghali_2016, Karp_1991b}. No cases were reported for psychosocial motivation and other motivation.
\paragraph*{Object Characteristics}  68 cases (76\%) involved sharp object ingestion \cite{Elghali_2016, Karp_1991b, Lee_2007}, 32 cases (36\%) involved long (\textgreater 5cm) object ingestion \cite{Lee_2007}, 25 cases (28\%) involved ingestion of multiple objects \cite{Elghali_2016, Lee_2007}. No cases were reported for button battery ingestion, magnet ingestion, and involved large diameter (\textgreater 2.5cm) object ingestion.
\paragraph*{Outcomes}  47 cases (52\%) underwent endoscopic intervention \cite{Elghali_2016, Lee_2007}, 29 cases (32\%) were managed conservatively \cite{Elghali_2016, Karp_1991b}, 15 cases (17\%) underwent surgical intervention \cite{Elghali_2016, Karp_1991b, Lee_2007}, 6 cases (7\%) reported complications \cite{Lee_2007}, 1 case (1\%) died \cite{Elghali_2016}.
\paragraph*{Geographical Location}Cases were recorded in 33 countries: 13 cases from USA \cite{Alao_2006i, Ataya_2013, Bhumi_2024f, Fry_2010, Guinan_2019f, Hardy_2023g, Jehangir_2019h, Kerestes_2019, Kumar_2001, Liu_2005, Tammana_2012j, Tay_2004, Tupesis_2004f}; 7 cases from India \cite{Bhasin_2014, Bhattacharjee_2008, Kar_2015, Kariholu_2008, Kumar_2019f, Misra_2013, Wadhwa_2015e} and UK \cite{Beecroft_1998, Berry_2021e, Cauchi_2002, Cox_2007, Gardner_2017h, Qureshi_2016}; 6 cases from Bulgaria \cite{Losanoff_1996, Losanoff_1997e}; 5 cases from Iran \cite{DivsalarP._2023a, Emamhadi_2018, Farhadi_2024h}; 4 cases from Turkey \cite{Akay_2015f, Atayan_2016, Tanrikulu_2015e, Yildiz_2016e}; 2 cases from China \cite{Jin_2023, Li_2013}, Poland \cite{Kobiela_2015, Wnęk_2015f}, and Spain \cite{CamachoDorado_2018, fjbuilsRepeatedBehaviorDeliberate2024}; 1 case from Australia \cite{Apikotoa_2022f}, Bahrain \cite{Ali_2020f}, Croatia \cite{Trgo_2012f}, Ecuador \cite{DelgadoSalazar_2020c}, Egypt \cite{Ali_2022g}, Ethiopia \cite{Mesfin_2022a}, Germany \cite{teWildt_2010}, Greece \cite{Sakellaridis_2008f}, Hungary \cite{Csaky_1998e}, Iraq \cite{Al-Faham_2020k}, Israel \cite{Goldman_1998f}, Italy \cite{Riva_2018j}, Japan \cite{Ohno_2005}, Nepal \cite{Thapa_2019f}, Netherlands \cite{Benoist_2019e}, Oman \cite{AlShaaibi_2021b}, Pakistan \cite{Yasin_2009}, Portugal \cite{Peixoto_2017f}, Qatar \cite{Ali_2017}, Saudi Arabia \cite{Sultan_2024f}, South Africa \cite{Sobnach_2011f}, Sweden \cite{Naji_2012f}, Switzerland \cite{Wildhaber_2005}, and Taiwan \cite{Chang_2017f}. \paragraph*{Gender} 43 cases (60\%) were male \cite{Akay_2015f, Al-Faham_2020k, Alao_2006i, Ali_2017, Ali_2022g, Apikotoa_2022f, Atayan_2016, Benoist_2019e, Berry_2021e, Bhumi_2024f, CamachoDorado_2018, Csaky_1998e, Emamhadi_2018, Farhadi_2024h, Fry_2010, Gardner_2017h, Guinan_2019f, Jehangir_2019h, Jin_2023, Kobiela_2015, Kumar_2001, Kumar_2019f, Liu_2005, Losanoff_1996, Losanoff_1997e, Mesfin_2022a, Misra_2013, Qureshi_2016, Riva_2018j, Sobnach_2011f, Tammana_2012j, Tanrikulu_2015e, Tay_2004, Thapa_2019f, Trgo_2012f, Wadhwa_2015e, Yasin_2009, teWildt_2010}, 28 cases (39\%) were female \cite{AlShaaibi_2021b, Ali_2020f, Ataya_2013, Beecroft_1998, Bhasin_2014, Bhattacharjee_2008, Cauchi_2002, Chang_2017f, Cox_2007, DelgadoSalazar_2020c, DivsalarP._2023a, Goldman_1998f, Hardy_2023g, Kar_2015, Kariholu_2008, Kerestes_2019, Li_2013, Naji_2012f, Ohno_2005, Peixoto_2017f, Sakellaridis_2008f, Sultan_2024f, Tupesis_2004f, Wildhaber_2005, Wnęk_2015f, Yildiz_2016e}, 1 case (1\%) had no gender recorded \cite{fjbuilsRepeatedBehaviorDeliberate2024}. \paragraph*{Age Group} 25 cases (35\%) were between 26 and 40 years of age \cite{Alao_2006i, Ali_2022g, Apikotoa_2022f, Ataya_2013, Benoist_2019e, Bhasin_2014, Chang_2017f, Cox_2007, DelgadoSalazar_2020c, Farhadi_2024h, Fry_2010, Gardner_2017h, Guinan_2019f, Jin_2023, Kumar_2019f, Losanoff_1996, Misra_2013, Qureshi_2016, Riva_2018j, Sakellaridis_2008f, Tammana_2012j, Trgo_2012f, Wnęk_2015f, Yildiz_2016e, fjbuilsRepeatedBehaviorDeliberate2024}, 18 cases (25\%) were between 18 and 25 years of age \cite{Akay_2015f, Ali_2017, Atayan_2016, Bhattacharjee_2008, Csaky_1998e, Kar_2015, Kariholu_2008, Kobiela_2015, Losanoff_1996, Losanoff_1997e, Mesfin_2022a, Peixoto_2017f, Sobnach_2011f, Tupesis_2004f, Yasin_2009}, 13 cases (18\%) were under 18 years of age \cite{AlShaaibi_2021b, Ali_2020f, Cauchi_2002, DivsalarP._2023a, Goldman_1998f, Liu_2005, Naji_2012f, Ohno_2005, Tanrikulu_2015e, Tay_2004, Wildhaber_2005}, 11 cases (15\%) were between 41 and 60 years of age \cite{Al-Faham_2020k, Bhumi_2024f, CamachoDorado_2018, Emamhadi_2018, Hardy_2023g, Jehangir_2019h, Kumar_2001, Sultan_2024f, Thapa_2019f, Wadhwa_2015e, teWildt_2010}, 3 cases (4\%) were over 60 years of age \cite{Beecroft_1998, Kerestes_2019, Li_2013}, 2 cases (3\%) had no age documented \cite{Berry_2021e}. \paragraph*{Population} 36 cases (50\%) had a psychiatric history \cite{AlShaaibi_2021b, Alao_2006i, Ali_2020f, Apikotoa_2022f, Ataya_2013, Atayan_2016, Beecroft_1998, CamachoDorado_2018, Chang_2017f, DelgadoSalazar_2020c, DivsalarP._2023a, Farhadi_2024h, Fry_2010, Guinan_2019f, Hardy_2023g, Jehangir_2019h, Jin_2023, Kar_2015, Kerestes_2019, Kobiela_2015, Kumar_2001, Kumar_2019f, Liu_2005, Mesfin_2022a, Misra_2013, Ohno_2005, Peixoto_2017f, Sakellaridis_2008f, Sultan_2024f, Tammana_2012j, Tanrikulu_2015e, Yildiz_2016e, fjbuilsRepeatedBehaviorDeliberate2024, teWildt_2010}, 19 cases (26\%) had ingested previously \cite{Alao_2006i, Apikotoa_2022f, Berry_2021e, Bhattacharjee_2008, Csaky_1998e, DivsalarP._2023a, Emamhadi_2018, Guinan_2019f, Jehangir_2019h, Jin_2023, Liu_2005, Sakellaridis_2008f, Tanrikulu_2015e, Thapa_2019f, Yildiz_2016e, fjbuilsRepeatedBehaviorDeliberate2024, teWildt_2010}, 12 cases (17\%) were detained persons \cite{Alao_2006i, Ali_2022g, Apikotoa_2022f, Losanoff_1996, Losanoff_1997e, Qureshi_2016, Tammana_2012j, Trgo_2012f}, 7 cases (10\%) were severely disabled \cite{Atayan_2016, Kerestes_2019, Liu_2005, Ohno_2005, Peixoto_2017f, Yildiz_2016e, teWildt_2010}, 4 cases (6\%) were psychiatric inpatients \cite{DivsalarP._2023a, fjbuilsRepeatedBehaviorDeliberate2024, teWildt_2010}, 3 cases (4\%) were under the influence of alcohol \cite{Benoist_2019e, Csaky_1998e, Thapa_2019f}, 2 cases (3\%) were displaced people \cite{Akay_2015f, Gardner_2017h}. \paragraph*{Motivation} 34 cases (47\%) had a psychiatric motivation \cite{Al-Faham_2020k, Alao_2006i, Ali_2020f, Apikotoa_2022f, Ataya_2013, Atayan_2016, Bhasin_2014, Bhattacharjee_2008, DelgadoSalazar_2020c, DivsalarP._2023a, Emamhadi_2018, Farhadi_2024h, Guinan_2019f, Hardy_2023g, Jehangir_2019h, Jin_2023, Kar_2015, Kariholu_2008, Kerestes_2019, Kobiela_2015, Kumar_2001, Kumar_2019f, Li_2013, Liu_2005, Misra_2013, Ohno_2005, Sakellaridis_2008f, Sultan_2024f, Tammana_2012j, Tanrikulu_2015e, Yasin_2009, teWildt_2010}, 21 cases (29\%) were motivated by self-harm intention \cite{Al-Faham_2020k, AlShaaibi_2021b, Alao_2006i, Ali_2017, CamachoDorado_2018, Chang_2017f, Cox_2007, Csaky_1998e, Fry_2010, Li_2013, Losanoff_1996, Losanoff_1997e, Mesfin_2022a, Sakellaridis_2008f, Tammana_2012j, Tanrikulu_2015e, fjbuilsRepeatedBehaviorDeliberate2024}, 17 cases (24\%) had a psychosocial motivation \cite{Akay_2015f, Benoist_2019e, Bhattacharjee_2008, Cauchi_2002, Goldman_1998f, Hardy_2023g, Kobiela_2015, Li_2013, Naji_2012f, Qureshi_2016, Riva_2018j, Sobnach_2011f, Tay_2004, Thapa_2019f, Tupesis_2004f, Wildhaber_2005, Wnęk_2015f}, 9 cases (12\%) were motivated by protest \cite{Bhumi_2024f, Gardner_2017h, Losanoff_1996, Losanoff_1997e, Tupesis_2004f}, 9 cases (12\%) had another documented motivation \cite{Ali_2020f, Ali_2022g, Emamhadi_2018, Guinan_2019f, Peixoto_2017f, Sakellaridis_2008f, Trgo_2012f, Wadhwa_2015e, Yildiz_2016e}. \paragraph*{Object Characteristics} 51 cases (71\%) ingested a large diameter object (\textgreater{}2.5cm) \cite{Akay_2015f, Al-Faham_2020k, AlShaaibi_2021b, Alao_2006i, Ali_2017, Ali_2022g, Apikotoa_2022f, Atayan_2016, Berry_2021e, Bhasin_2014, CamachoDorado_2018, Cauchi_2002, Chang_2017f, Cox_2007, Csaky_1998e, DivsalarP._2023a, Emamhadi_2018, Gardner_2017h, Guinan_2019f, Jehangir_2019h, Jin_2023, Kariholu_2008, Kerestes_2019, Kobiela_2015, Kumar_2001, Kumar_2019f, Losanoff_1996, Losanoff_1997e, Mesfin_2022a, Misra_2013, Naji_2012f, Ohno_2005, Peixoto_2017f, Qureshi_2016, Riva_2018j, Sakellaridis_2008f, Sultan_2024f, Tanrikulu_2015e, Thapa_2019f, Trgo_2012f, Wnęk_2015f, Yildiz_2016e, fjbuilsRepeatedBehaviorDeliberate2024, teWildt_2010}, 44 cases (61\%) ingested multiple objects \cite{Ali_2020f, Apikotoa_2022f, Ataya_2013, Atayan_2016, Beecroft_1998, Bhattacharjee_2008, Bhumi_2024f, CamachoDorado_2018, Cauchi_2002, Emamhadi_2018, Farhadi_2024h, Fry_2010, Goldman_1998f, Guinan_2019f, Hardy_2023g, Jehangir_2019h, Jin_2023, Kar_2015, Kariholu_2008, Kobiela_2015, Kumar_2001, Kumar_2019f, Li_2013, Liu_2005, Losanoff_1996, Mesfin_2022a, Misra_2013, Naji_2012f, Ohno_2005, Sobnach_2011f, Sultan_2024f, Tammana_2012j, Tanrikulu_2015e, Tay_2004, Thapa_2019f, Wadhwa_2015e, Wildhaber_2005, Yasin_2009, fjbuilsRepeatedBehaviorDeliberate2024, teWildt_2010}, 34 cases (47\%) ingested a sharp object \cite{AlShaaibi_2021b, Alao_2006i, Apikotoa_2022f, Ataya_2013, Benoist_2019e, Bhasin_2014, Bhattacharjee_2008, CamachoDorado_2018, Csaky_1998e, DelgadoSalazar_2020c, DivsalarP._2023a, Emamhadi_2018, Farhadi_2024h, Fry_2010, Guinan_2019f, Hardy_2023g, Jehangir_2019h, Jin_2023, Kariholu_2008, Kobiela_2015, Kumar_2019f, Losanoff_1996, Losanoff_1997e, Mesfin_2022a, Misra_2013, Sobnach_2011f, Yasin_2009, teWildt_2010}, 32 cases (44\%) ingested a long object (\textgreater{}5cm) \cite{Al-Faham_2020k, AlShaaibi_2021b, Ali_2017, Ali_2022g, Atayan_2016, Bhasin_2014, CamachoDorado_2018, Chang_2017f, Cox_2007, Csaky_1998e, DivsalarP._2023a, Emamhadi_2018, Fry_2010, Gardner_2017h, Jin_2023, Kariholu_2008, Kerestes_2019, Kobiela_2015, Kumar_2019f, Mesfin_2022a, Misra_2013, Ohno_2005, Qureshi_2016, Sakellaridis_2008f, Sultan_2024f, Thapa_2019f, Trgo_2012f, Yasin_2009, Yildiz_2016e, teWildt_2010}, 9 cases (12\%) ingested a magnet \cite{Ali_2020f, Bhumi_2024f, Cauchi_2002, Liu_2005, Naji_2012f, Ohno_2005, Tanrikulu_2015e, Tay_2004, Wildhaber_2005}, 2 cases (3\%) ingested a button battery \cite{Berry_2021e, Bhumi_2024f}. \paragraph*{Outcomes} 48 cases (67\%) experienced a complication \cite{Ali_2017, Ali_2020f, Apikotoa_2022f, Atayan_2016, Beecroft_1998, Benoist_2019e, Berry_2021e, Bhasin_2014, Bhumi_2024f, CamachoDorado_2018, Cauchi_2002, Cox_2007, Csaky_1998e, DelgadoSalazar_2020c, DivsalarP._2023a, Emamhadi_2018, Farhadi_2024h, Fry_2010, Gardner_2017h, Goldman_1998f, Jin_2023, Kariholu_2008, Kerestes_2019, Kobiela_2015, Kumar_2001, Kumar_2019f, Liu_2005, Losanoff_1996, Mesfin_2022a, Misra_2013, Naji_2012f, Ohno_2005, Sakellaridis_2008f, Sobnach_2011f, Sultan_2024f, Tanrikulu_2015e, Tay_2004, Thapa_2019f, Trgo_2012f, Tupesis_2004f, Wildhaber_2005, Wnęk_2015f, Yasin_2009, Yildiz_2016e}, 44 cases (61\%) underwent surgery \cite{Al-Faham_2020k, AlShaaibi_2021b, Alao_2006i, Ali_2017, Ali_2020f, Atayan_2016, Beecroft_1998, Bhasin_2014, CamachoDorado_2018, Cauchi_2002, Chang_2017f, Cox_2007, Csaky_1998e, DelgadoSalazar_2020c, DivsalarP._2023a, Farhadi_2024h, Fry_2010, Gardner_2017h, Jin_2023, Kariholu_2008, Kerestes_2019, Kobiela_2015, Kumar_2019f, Liu_2005, Losanoff_1996, Losanoff_1997e, Mesfin_2022a, Misra_2013, Naji_2012f, Sobnach_2011f, Tanrikulu_2015e, Tay_2004, Thapa_2019f, Tupesis_2004f, Wildhaber_2005, Wnęk_2015f, Yasin_2009, Yildiz_2016e, fjbuilsRepeatedBehaviorDeliberate2024}, 31 cases (43\%) underwent endoscopy \cite{Akay_2015f, Ali_2022g, Apikotoa_2022f, Atayan_2016, Benoist_2019e, Berry_2021e, Bhasin_2014, Bhumi_2024f, CamachoDorado_2018, Chang_2017f, DelgadoSalazar_2020c, Gardner_2017h, Guinan_2019f, Hardy_2023g, Jehangir_2019h, Kariholu_2008, Li_2013, Liu_2005, Ohno_2005, Peixoto_2017f, Qureshi_2016, Riva_2018j, Sakellaridis_2008f, Sultan_2024f, Tammana_2012j, Tanrikulu_2015e, Trgo_2012f, Wadhwa_2015e, Wnęk_2015f, teWildt_2010}, 7 cases (10\%) were managed conservatively \cite{Ataya_2013, Bhattacharjee_2008, DivsalarP._2023a, Emamhadi_2018, Goldman_1998f, Kar_2015, Kumar_2001}, 2 cases (3\%) died \cite{Emamhadi_2018, Kumar_2001}. All 90 were male gender. 90 cases (100\%) were detained at the time of ingestion \cite{Elghali_2016, Karp_1991b, Lee_2007}, 88 cases (98\%) were intentional ingestions \cite{Elghali_2016, Karp_1991b, Lee_2007}, 30 cases (33\%) had a psychiatric history documented \cite{Elghali_2016, Karp_1991b, Lee_2007}, 2 cases (2\%) had a history of prior ingestion \cite{Elghali_2016}. No cases were reported for were psychiatric inpatients, were displaced people, were under the influence of alcohol at the time of ingestion, and had a severe disability history.
\paragraph*{Motivation}  70 cases (78\%) reported protest motivation \cite{Elghali_2016, Karp_1991b, Lee_2007}, 12 cases (13\%) reported psychiatric motivation \cite{Karp_1991b}, 6 cases (7\%) reported self-harm motivation \cite{Elghali_2016, Karp_1991b}. No cases were reported for psychosocial motivation and other motivation.
\paragraph*{Object Characteristics}  68 cases (76\%) involved sharp object ingestion \cite{Elghali_2016, Karp_1991b, Lee_2007}, 32 cases (36\%) involved long (\textgreater 5cm) object ingestion \cite{Lee_2007}, 25 cases (28\%) involved ingestion of multiple objects \cite{Elghali_2016, Lee_2007}. No cases were reported for button battery ingestion, magnet ingestion, and involved large diameter (\textgreater 2.5cm) object ingestion.
\paragraph*{Outcomes}  47 cases (52\%) underwent endoscopic intervention \cite{Elghali_2016, Lee_2007}, 29 cases (32\%) were managed conservatively \cite{Elghali_2016, Karp_1991b}, 15 cases (17\%) underwent surgical intervention \cite{Elghali_2016, Karp_1991b, Lee_2007}, 6 cases (7\%) reported complications \cite{Lee_2007}, 1 case (1\%) died \cite{Elghali_2016}.
\paragraph*{Geographical Location}Cases were recorded in 33 countries: 13 cases from USA \cite{Alao_2006i, Ataya_2013, Bhumi_2024f, Fry_2010, Guinan_2019f, Hardy_2023g, Jehangir_2019h, Kerestes_2019, Kumar_2001, Liu_2005, Tammana_2012j, Tay_2004, Tupesis_2004f}; 7 cases from India \cite{Bhasin_2014, Bhattacharjee_2008, Kar_2015, Kariholu_2008, Kumar_2019f, Misra_2013, Wadhwa_2015e} and UK \cite{Beecroft_1998, Berry_2021e, Cauchi_2002, Cox_2007, Gardner_2017h, Qureshi_2016}; 6 cases from Bulgaria \cite{Losanoff_1996, Losanoff_1997e}; 5 cases from Iran \cite{DivsalarP._2023a, Emamhadi_2018, Farhadi_2024h}; 4 cases from Turkey \cite{Akay_2015f, Atayan_2016, Tanrikulu_2015e, Yildiz_2016e}; 2 cases from China \cite{Jin_2023, Li_2013}, Poland \cite{Kobiela_2015, Wnęk_2015f}, and Spain \cite{CamachoDorado_2018, fjbuilsRepeatedBehaviorDeliberate2024}; 1 case from Australia \cite{Apikotoa_2022f}, Bahrain \cite{Ali_2020f}, Croatia \cite{Trgo_2012f}, Ecuador \cite{DelgadoSalazar_2020c}, Egypt \cite{Ali_2022g}, Ethiopia \cite{Mesfin_2022a}, Germany \cite{teWildt_2010}, Greece \cite{Sakellaridis_2008f}, Hungary \cite{Csaky_1998e}, Iraq \cite{Al-Faham_2020k}, Israel \cite{Goldman_1998f}, Italy \cite{Riva_2018j}, Japan \cite{Ohno_2005}, Nepal \cite{Thapa_2019f}, Netherlands \cite{Benoist_2019e}, Oman \cite{AlShaaibi_2021b}, Pakistan \cite{Yasin_2009}, Portugal \cite{Peixoto_2017f}, Qatar \cite{Ali_2017}, Saudi Arabia \cite{Sultan_2024f}, South Africa \cite{Sobnach_2011f}, Sweden \cite{Naji_2012f}, Switzerland \cite{Wildhaber_2005}, and Taiwan \cite{Chang_2017f}. \paragraph*{Gender} 43 cases (60\%) were male \cite{Akay_2015f, Al-Faham_2020k, Alao_2006i, Ali_2017, Ali_2022g, Apikotoa_2022f, Atayan_2016, Benoist_2019e, Berry_2021e, Bhumi_2024f, CamachoDorado_2018, Csaky_1998e, Emamhadi_2018, Farhadi_2024h, Fry_2010, Gardner_2017h, Guinan_2019f, Jehangir_2019h, Jin_2023, Kobiela_2015, Kumar_2001, Kumar_2019f, Liu_2005, Losanoff_1996, Losanoff_1997e, Mesfin_2022a, Misra_2013, Qureshi_2016, Riva_2018j, Sobnach_2011f, Tammana_2012j, Tanrikulu_2015e, Tay_2004, Thapa_2019f, Trgo_2012f, Wadhwa_2015e, Yasin_2009, teWildt_2010}, 28 cases (39\%) were female \cite{AlShaaibi_2021b, Ali_2020f, Ataya_2013, Beecroft_1998, Bhasin_2014, Bhattacharjee_2008, Cauchi_2002, Chang_2017f, Cox_2007, DelgadoSalazar_2020c, DivsalarP._2023a, Goldman_1998f, Hardy_2023g, Kar_2015, Kariholu_2008, Kerestes_2019, Li_2013, Naji_2012f, Ohno_2005, Peixoto_2017f, Sakellaridis_2008f, Sultan_2024f, Tupesis_2004f, Wildhaber_2005, Wnęk_2015f, Yildiz_2016e}, 1 case (1\%) had no gender recorded \cite{fjbuilsRepeatedBehaviorDeliberate2024}. \paragraph*{Age Group} 25 cases (35\%) were between 26 and 40 years of age \cite{Alao_2006i, Ali_2022g, Apikotoa_2022f, Ataya_2013, Benoist_2019e, Bhasin_2014, Chang_2017f, Cox_2007, DelgadoSalazar_2020c, Farhadi_2024h, Fry_2010, Gardner_2017h, Guinan_2019f, Jin_2023, Kumar_2019f, Losanoff_1996, Misra_2013, Qureshi_2016, Riva_2018j, Sakellaridis_2008f, Tammana_2012j, Trgo_2012f, Wnęk_2015f, Yildiz_2016e, fjbuilsRepeatedBehaviorDeliberate2024}, 18 cases (25\%) were between 18 and 25 years of age \cite{Akay_2015f, Ali_2017, Atayan_2016, Bhattacharjee_2008, Csaky_1998e, Kar_2015, Kariholu_2008, Kobiela_2015, Losanoff_1996, Losanoff_1997e, Mesfin_2022a, Peixoto_2017f, Sobnach_2011f, Tupesis_2004f, Yasin_2009}, 13 cases (18\%) were under 18 years of age \cite{AlShaaibi_2021b, Ali_2020f, Cauchi_2002, DivsalarP._2023a, Goldman_1998f, Liu_2005, Naji_2012f, Ohno_2005, Tanrikulu_2015e, Tay_2004, Wildhaber_2005}, 11 cases (15\%) were between 41 and 60 years of age \cite{Al-Faham_2020k, Bhumi_2024f, CamachoDorado_2018, Emamhadi_2018, Hardy_2023g, Jehangir_2019h, Kumar_2001, Sultan_2024f, Thapa_2019f, Wadhwa_2015e, teWildt_2010}, 3 cases (4\%) were over 60 years of age \cite{Beecroft_1998, Kerestes_2019, Li_2013}, 2 cases (3\%) had no age documented \cite{Berry_2021e}. \paragraph*{Population} 36 cases (50\%) had a psychiatric history \cite{AlShaaibi_2021b, Alao_2006i, Ali_2020f, Apikotoa_2022f, Ataya_2013, Atayan_2016, Beecroft_1998, CamachoDorado_2018, Chang_2017f, DelgadoSalazar_2020c, DivsalarP._2023a, Farhadi_2024h, Fry_2010, Guinan_2019f, Hardy_2023g, Jehangir_2019h, Jin_2023, Kar_2015, Kerestes_2019, Kobiela_2015, Kumar_2001, Kumar_2019f, Liu_2005, Mesfin_2022a, Misra_2013, Ohno_2005, Peixoto_2017f, Sakellaridis_2008f, Sultan_2024f, Tammana_2012j, Tanrikulu_2015e, Yildiz_2016e, fjbuilsRepeatedBehaviorDeliberate2024, teWildt_2010}, 19 cases (26\%) had ingested previously \cite{Alao_2006i, Apikotoa_2022f, Berry_2021e, Bhattacharjee_2008, Csaky_1998e, DivsalarP._2023a, Emamhadi_2018, Guinan_2019f, Jehangir_2019h, Jin_2023, Liu_2005, Sakellaridis_2008f, Tanrikulu_2015e, Thapa_2019f, Yildiz_2016e, fjbuilsRepeatedBehaviorDeliberate2024, teWildt_2010}, 12 cases (17\%) were detained persons \cite{Alao_2006i, Ali_2022g, Apikotoa_2022f, Losanoff_1996, Losanoff_1997e, Qureshi_2016, Tammana_2012j, Trgo_2012f}, 7 cases (10\%) were severely disabled \cite{Atayan_2016, Kerestes_2019, Liu_2005, Ohno_2005, Peixoto_2017f, Yildiz_2016e, teWildt_2010}, 4 cases (6\%) were psychiatric inpatients \cite{DivsalarP._2023a, fjbuilsRepeatedBehaviorDeliberate2024, teWildt_2010}, 3 cases (4\%) were under the influence of alcohol \cite{Benoist_2019e, Csaky_1998e, Thapa_2019f}, 2 cases (3\%) were displaced people \cite{Akay_2015f, Gardner_2017h}. \paragraph*{Motivation} 34 cases (47\%) had a psychiatric motivation \cite{Al-Faham_2020k, Alao_2006i, Ali_2020f, Apikotoa_2022f, Ataya_2013, Atayan_2016, Bhasin_2014, Bhattacharjee_2008, DelgadoSalazar_2020c, DivsalarP._2023a, Emamhadi_2018, Farhadi_2024h, Guinan_2019f, Hardy_2023g, Jehangir_2019h, Jin_2023, Kar_2015, Kariholu_2008, Kerestes_2019, Kobiela_2015, Kumar_2001, Kumar_2019f, Li_2013, Liu_2005, Misra_2013, Ohno_2005, Sakellaridis_2008f, Sultan_2024f, Tammana_2012j, Tanrikulu_2015e, Yasin_2009, teWildt_2010}, 21 cases (29\%) were motivated by self-harm intention \cite{Al-Faham_2020k, AlShaaibi_2021b, Alao_2006i, Ali_2017, CamachoDorado_2018, Chang_2017f, Cox_2007, Csaky_1998e, Fry_2010, Li_2013, Losanoff_1996, Losanoff_1997e, Mesfin_2022a, Sakellaridis_2008f, Tammana_2012j, Tanrikulu_2015e, fjbuilsRepeatedBehaviorDeliberate2024}, 17 cases (24\%) had a psychosocial motivation \cite{Akay_2015f, Benoist_2019e, Bhattacharjee_2008, Cauchi_2002, Goldman_1998f, Hardy_2023g, Kobiela_2015, Li_2013, Naji_2012f, Qureshi_2016, Riva_2018j, Sobnach_2011f, Tay_2004, Thapa_2019f, Tupesis_2004f, Wildhaber_2005, Wnęk_2015f}, 9 cases (12\%) were motivated by protest \cite{Bhumi_2024f, Gardner_2017h, Losanoff_1996, Losanoff_1997e, Tupesis_2004f}, 9 cases (12\%) had another documented motivation \cite{Ali_2020f, Ali_2022g, Emamhadi_2018, Guinan_2019f, Peixoto_2017f, Sakellaridis_2008f, Trgo_2012f, Wadhwa_2015e, Yildiz_2016e}. \paragraph*{Object Characteristics} 51 cases (71\%) ingested a large diameter object (\textgreater{}2.5cm) \cite{Akay_2015f, Al-Faham_2020k, AlShaaibi_2021b, Alao_2006i, Ali_2017, Ali_2022g, Apikotoa_2022f, Atayan_2016, Berry_2021e, Bhasin_2014, CamachoDorado_2018, Cauchi_2002, Chang_2017f, Cox_2007, Csaky_1998e, DivsalarP._2023a, Emamhadi_2018, Gardner_2017h, Guinan_2019f, Jehangir_2019h, Jin_2023, Kariholu_2008, Kerestes_2019, Kobiela_2015, Kumar_2001, Kumar_2019f, Losanoff_1996, Losanoff_1997e, Mesfin_2022a, Misra_2013, Naji_2012f, Ohno_2005, Peixoto_2017f, Qureshi_2016, Riva_2018j, Sakellaridis_2008f, Sultan_2024f, Tanrikulu_2015e, Thapa_2019f, Trgo_2012f, Wnęk_2015f, Yildiz_2016e, fjbuilsRepeatedBehaviorDeliberate2024, teWildt_2010}, 44 cases (61\%) ingested multiple objects \cite{Ali_2020f, Apikotoa_2022f, Ataya_2013, Atayan_2016, Beecroft_1998, Bhattacharjee_2008, Bhumi_2024f, CamachoDorado_2018, Cauchi_2002, Emamhadi_2018, Farhadi_2024h, Fry_2010, Goldman_1998f, Guinan_2019f, Hardy_2023g, Jehangir_2019h, Jin_2023, Kar_2015, Kariholu_2008, Kobiela_2015, Kumar_2001, Kumar_2019f, Li_2013, Liu_2005, Losanoff_1996, Mesfin_2022a, Misra_2013, Naji_2012f, Ohno_2005, Sobnach_2011f, Sultan_2024f, Tammana_2012j, Tanrikulu_2015e, Tay_2004, Thapa_2019f, Wadhwa_2015e, Wildhaber_2005, Yasin_2009, fjbuilsRepeatedBehaviorDeliberate2024, teWildt_2010}, 34 cases (47\%) ingested a sharp object \cite{AlShaaibi_2021b, Alao_2006i, Apikotoa_2022f, Ataya_2013, Benoist_2019e, Bhasin_2014, Bhattacharjee_2008, CamachoDorado_2018, Csaky_1998e, DelgadoSalazar_2020c, DivsalarP._2023a, Emamhadi_2018, Farhadi_2024h, Fry_2010, Guinan_2019f, Hardy_2023g, Jehangir_2019h, Jin_2023, Kariholu_2008, Kobiela_2015, Kumar_2019f, Losanoff_1996, Losanoff_1997e, Mesfin_2022a, Misra_2013, Sobnach_2011f, Yasin_2009, teWildt_2010}, 32 cases (44\%) ingested a long object (\textgreater{}5cm) \cite{Al-Faham_2020k, AlShaaibi_2021b, Ali_2017, Ali_2022g, Atayan_2016, Bhasin_2014, CamachoDorado_2018, Chang_2017f, Cox_2007, Csaky_1998e, DivsalarP._2023a, Emamhadi_2018, Fry_2010, Gardner_2017h, Jin_2023, Kariholu_2008, Kerestes_2019, Kobiela_2015, Kumar_2019f, Mesfin_2022a, Misra_2013, Ohno_2005, Qureshi_2016, Sakellaridis_2008f, Sultan_2024f, Thapa_2019f, Trgo_2012f, Yasin_2009, Yildiz_2016e, teWildt_2010}, 9 cases (12\%) ingested a magnet \cite{Ali_2020f, Bhumi_2024f, Cauchi_2002, Liu_2005, Naji_2012f, Ohno_2005, Tanrikulu_2015e, Tay_2004, Wildhaber_2005}, 2 cases (3\%) ingested a button battery \cite{Berry_2021e, Bhumi_2024f}. \paragraph*{Outcomes} 48 cases (67\%) experienced a complication \cite{Ali_2017, Ali_2020f, Apikotoa_2022f, Atayan_2016, Beecroft_1998, Benoist_2019e, Berry_2021e, Bhasin_2014, Bhumi_2024f, CamachoDorado_2018, Cauchi_2002, Cox_2007, Csaky_1998e, DelgadoSalazar_2020c, DivsalarP._2023a, Emamhadi_2018, Farhadi_2024h, Fry_2010, Gardner_2017h, Goldman_1998f, Jin_2023, Kariholu_2008, Kerestes_2019, Kobiela_2015, Kumar_2001, Kumar_2019f, Liu_2005, Losanoff_1996, Mesfin_2022a, Misra_2013, Naji_2012f, Ohno_2005, Sakellaridis_2008f, Sobnach_2011f, Sultan_2024f, Tanrikulu_2015e, Tay_2004, Thapa_2019f, Trgo_2012f, Tupesis_2004f, Wildhaber_2005, Wnęk_2015f, Yasin_2009, Yildiz_2016e}, 44 cases (61\%) underwent surgery \cite{Al-Faham_2020k, AlShaaibi_2021b, Alao_2006i, Ali_2017, Ali_2020f, Atayan_2016, Beecroft_1998, Bhasin_2014, CamachoDorado_2018, Cauchi_2002, Chang_2017f, Cox_2007, Csaky_1998e, DelgadoSalazar_2020c, DivsalarP._2023a, Farhadi_2024h, Fry_2010, Gardner_2017h, Jin_2023, Kariholu_2008, Kerestes_2019, Kobiela_2015, Kumar_2019f, Liu_2005, Losanoff_1996, Losanoff_1997e, Mesfin_2022a, Misra_2013, Naji_2012f, Sobnach_2011f, Tanrikulu_2015e, Tay_2004, Thapa_2019f, Tupesis_2004f, Wildhaber_2005, Wnęk_2015f, Yasin_2009, Yildiz_2016e, fjbuilsRepeatedBehaviorDeliberate2024}, 31 cases (43\%) underwent endoscopy \cite{Akay_2015f, Ali_2022g, Apikotoa_2022f, Atayan_2016, Benoist_2019e, Berry_2021e, Bhasin_2014, Bhumi_2024f, CamachoDorado_2018, Chang_2017f, DelgadoSalazar_2020c, Gardner_2017h, Guinan_2019f, Hardy_2023g, Jehangir_2019h, Kariholu_2008, Li_2013, Liu_2005, Ohno_2005, Peixoto_2017f, Qureshi_2016, Riva_2018j, Sakellaridis_2008f, Sultan_2024f, Tammana_2012j, Tanrikulu_2015e, Trgo_2012f, Wadhwa_2015e, Wnęk_2015f, teWildt_2010}, 7 cases (10\%) were managed conservatively \cite{Ataya_2013, Bhattacharjee_2008, DivsalarP._2023a, Emamhadi_2018, Goldman_1998f, Kar_2015, Kumar_2001}, 2 cases (3\%) died \cite{Emamhadi_2018, Kumar_2001}. All 90 were male gender. 90 cases (100\%) were detained at the time of ingestion \cite{Elghali_2016, Karp_1991b, Lee_2007}, 88 cases (98\%) were intentional ingestions \cite{Elghali_2016, Karp_1991b, Lee_2007}, 30 cases (33\%) had a psychiatric history documented \cite{Elghali_2016, Karp_1991b, Lee_2007}, 2 cases (2\%) had a history of prior ingestion \cite{Elghali_2016}. No cases were reported for were psychiatric inpatients, were displaced people, were under the influence of alcohol at the time of ingestion, and had a severe disability history.
\paragraph*{Motivation}  70 cases (78\%) reported protest motivation \cite{Elghali_2016, Karp_1991b, Lee_2007}, 12 cases (13\%) reported psychiatric motivation \cite{Karp_1991b}, 6 cases (7\%) reported self-harm motivation \cite{Elghali_2016, Karp_1991b}. No cases were reported for psychosocial motivation and other motivation.
\paragraph*{Object Characteristics}  68 cases (76\%) involved sharp object ingestion \cite{Elghali_2016, Karp_1991b, Lee_2007}, 32 cases (36\%) involved long (\textgreater 5cm) object ingestion \cite{Lee_2007}, 25 cases (28\%) involved ingestion of multiple objects \cite{Elghali_2016, Lee_2007}. No cases were reported for button battery ingestion, magnet ingestion, and involved large diameter (\textgreater 2.5cm) object ingestion.
\paragraph*{Outcomes}  47 cases (52\%) underwent endoscopic intervention \cite{Elghali_2016, Lee_2007}, 29 cases (32\%) were managed conservatively \cite{Elghali_2016, Karp_1991b}, 15 cases (17\%) underwent surgical intervention \cite{Elghali_2016, Karp_1991b, Lee_2007}, 6 cases (7\%) reported complications \cite{Lee_2007}, 1 case (1\%) died \cite{Elghali_2016}.
\paragraph*{Geographical Location}Cases were recorded in 33 countries: 13 cases from USA \cite{Alao_2006i, Ataya_2013, Bhumi_2024f, Fry_2010, Guinan_2019f, Hardy_2023g, Jehangir_2019h, Kerestes_2019, Kumar_2001, Liu_2005, Tammana_2012j, Tay_2004, Tupesis_2004f}; 7 cases from India \cite{Bhasin_2014, Bhattacharjee_2008, Kar_2015, Kariholu_2008, Kumar_2019f, Misra_2013, Wadhwa_2015e} and UK \cite{Beecroft_1998, Berry_2021e, Cauchi_2002, Cox_2007, Gardner_2017h, Qureshi_2016}; 6 cases from Bulgaria \cite{Losanoff_1996, Losanoff_1997e}; 5 cases from Iran \cite{DivsalarP._2023a, Emamhadi_2018, Farhadi_2024h}; 4 cases from Turkey \cite{Akay_2015f, Atayan_2016, Tanrikulu_2015e, Yildiz_2016e}; 2 cases from China \cite{Jin_2023, Li_2013}, Poland \cite{Kobiela_2015, Wnęk_2015f}, and Spain \cite{CamachoDorado_2018, fjbuilsRepeatedBehaviorDeliberate2024}; 1 case from Australia \cite{Apikotoa_2022f}, Bahrain \cite{Ali_2020f}, Croatia \cite{Trgo_2012f}, Ecuador \cite{DelgadoSalazar_2020c}, Egypt \cite{Ali_2022g}, Ethiopia \cite{Mesfin_2022a}, Germany \cite{teWildt_2010}, Greece \cite{Sakellaridis_2008f}, Hungary \cite{Csaky_1998e}, Iraq \cite{Al-Faham_2020k}, Israel \cite{Goldman_1998f}, Italy \cite{Riva_2018j}, Japan \cite{Ohno_2005}, Nepal \cite{Thapa_2019f}, Netherlands \cite{Benoist_2019e}, Oman \cite{AlShaaibi_2021b}, Pakistan \cite{Yasin_2009}, Portugal \cite{Peixoto_2017f}, Qatar \cite{Ali_2017}, Saudi Arabia \cite{Sultan_2024f}, South Africa \cite{Sobnach_2011f}, Sweden \cite{Naji_2012f}, Switzerland \cite{Wildhaber_2005}, and Taiwan \cite{Chang_2017f}. \paragraph*{Gender} 43 cases (60\%) were male \cite{Akay_2015f, Al-Faham_2020k, Alao_2006i, Ali_2017, Ali_2022g, Apikotoa_2022f, Atayan_2016, Benoist_2019e, Berry_2021e, Bhumi_2024f, CamachoDorado_2018, Csaky_1998e, Emamhadi_2018, Farhadi_2024h, Fry_2010, Gardner_2017h, Guinan_2019f, Jehangir_2019h, Jin_2023, Kobiela_2015, Kumar_2001, Kumar_2019f, Liu_2005, Losanoff_1996, Losanoff_1997e, Mesfin_2022a, Misra_2013, Qureshi_2016, Riva_2018j, Sobnach_2011f, Tammana_2012j, Tanrikulu_2015e, Tay_2004, Thapa_2019f, Trgo_2012f, Wadhwa_2015e, Yasin_2009, teWildt_2010}, 28 cases (39\%) were female \cite{AlShaaibi_2021b, Ali_2020f, Ataya_2013, Beecroft_1998, Bhasin_2014, Bhattacharjee_2008, Cauchi_2002, Chang_2017f, Cox_2007, DelgadoSalazar_2020c, DivsalarP._2023a, Goldman_1998f, Hardy_2023g, Kar_2015, Kariholu_2008, Kerestes_2019, Li_2013, Naji_2012f, Ohno_2005, Peixoto_2017f, Sakellaridis_2008f, Sultan_2024f, Tupesis_2004f, Wildhaber_2005, Wnęk_2015f, Yildiz_2016e}, 1 case (1\%) had no gender recorded \cite{fjbuilsRepeatedBehaviorDeliberate2024}. \paragraph*{Age Group} 25 cases (35\%) were between 26 and 40 years of age \cite{Alao_2006i, Ali_2022g, Apikotoa_2022f, Ataya_2013, Benoist_2019e, Bhasin_2014, Chang_2017f, Cox_2007, DelgadoSalazar_2020c, Farhadi_2024h, Fry_2010, Gardner_2017h, Guinan_2019f, Jin_2023, Kumar_2019f, Losanoff_1996, Misra_2013, Qureshi_2016, Riva_2018j, Sakellaridis_2008f, Tammana_2012j, Trgo_2012f, Wnęk_2015f, Yildiz_2016e, fjbuilsRepeatedBehaviorDeliberate2024}, 18 cases (25\%) were between 18 and 25 years of age \cite{Akay_2015f, Ali_2017, Atayan_2016, Bhattacharjee_2008, Csaky_1998e, Kar_2015, Kariholu_2008, Kobiela_2015, Losanoff_1996, Losanoff_1997e, Mesfin_2022a, Peixoto_2017f, Sobnach_2011f, Tupesis_2004f, Yasin_2009}, 13 cases (18\%) were under 18 years of age \cite{AlShaaibi_2021b, Ali_2020f, Cauchi_2002, DivsalarP._2023a, Goldman_1998f, Liu_2005, Naji_2012f, Ohno_2005, Tanrikulu_2015e, Tay_2004, Wildhaber_2005}, 11 cases (15\%) were between 41 and 60 years of age \cite{Al-Faham_2020k, Bhumi_2024f, CamachoDorado_2018, Emamhadi_2018, Hardy_2023g, Jehangir_2019h, Kumar_2001, Sultan_2024f, Thapa_2019f, Wadhwa_2015e, teWildt_2010}, 3 cases (4\%) were over 60 years of age \cite{Beecroft_1998, Kerestes_2019, Li_2013}, 2 cases (3\%) had no age documented \cite{Berry_2021e}. \paragraph*{Population} 36 cases (50\%) had a psychiatric history \cite{AlShaaibi_2021b, Alao_2006i, Ali_2020f, Apikotoa_2022f, Ataya_2013, Atayan_2016, Beecroft_1998, CamachoDorado_2018, Chang_2017f, DelgadoSalazar_2020c, DivsalarP._2023a, Farhadi_2024h, Fry_2010, Guinan_2019f, Hardy_2023g, Jehangir_2019h, Jin_2023, Kar_2015, Kerestes_2019, Kobiela_2015, Kumar_2001, Kumar_2019f, Liu_2005, Mesfin_2022a, Misra_2013, Ohno_2005, Peixoto_2017f, Sakellaridis_2008f, Sultan_2024f, Tammana_2012j, Tanrikulu_2015e, Yildiz_2016e, fjbuilsRepeatedBehaviorDeliberate2024, teWildt_2010}, 19 cases (26\%) had ingested previously \cite{Alao_2006i, Apikotoa_2022f, Berry_2021e, Bhattacharjee_2008, Csaky_1998e, DivsalarP._2023a, Emamhadi_2018, Guinan_2019f, Jehangir_2019h, Jin_2023, Liu_2005, Sakellaridis_2008f, Tanrikulu_2015e, Thapa_2019f, Yildiz_2016e, fjbuilsRepeatedBehaviorDeliberate2024, teWildt_2010}, 12 cases (17\%) were detained persons \cite{Alao_2006i, Ali_2022g, Apikotoa_2022f, Losanoff_1996, Losanoff_1997e, Qureshi_2016, Tammana_2012j, Trgo_2012f}, 7 cases (10\%) were severely disabled \cite{Atayan_2016, Kerestes_2019, Liu_2005, Ohno_2005, Peixoto_2017f, Yildiz_2016e, teWildt_2010}, 4 cases (6\%) were psychiatric inpatients \cite{DivsalarP._2023a, fjbuilsRepeatedBehaviorDeliberate2024, teWildt_2010}, 3 cases (4\%) were under the influence of alcohol \cite{Benoist_2019e, Csaky_1998e, Thapa_2019f}, 2 cases (3\%) were displaced people \cite{Akay_2015f, Gardner_2017h}. \paragraph*{Motivation} 34 cases (47\%) had a psychiatric motivation \cite{Al-Faham_2020k, Alao_2006i, Ali_2020f, Apikotoa_2022f, Ataya_2013, Atayan_2016, Bhasin_2014, Bhattacharjee_2008, DelgadoSalazar_2020c, DivsalarP._2023a, Emamhadi_2018, Farhadi_2024h, Guinan_2019f, Hardy_2023g, Jehangir_2019h, Jin_2023, Kar_2015, Kariholu_2008, Kerestes_2019, Kobiela_2015, Kumar_2001, Kumar_2019f, Li_2013, Liu_2005, Misra_2013, Ohno_2005, Sakellaridis_2008f, Sultan_2024f, Tammana_2012j, Tanrikulu_2015e, Yasin_2009, teWildt_2010}, 21 cases (29\%) were motivated by self-harm intention \cite{Al-Faham_2020k, AlShaaibi_2021b, Alao_2006i, Ali_2017, CamachoDorado_2018, Chang_2017f, Cox_2007, Csaky_1998e, Fry_2010, Li_2013, Losanoff_1996, Losanoff_1997e, Mesfin_2022a, Sakellaridis_2008f, Tammana_2012j, Tanrikulu_2015e, fjbuilsRepeatedBehaviorDeliberate2024}, 17 cases (24\%) had a psychosocial motivation \cite{Akay_2015f, Benoist_2019e, Bhattacharjee_2008, Cauchi_2002, Goldman_1998f, Hardy_2023g, Kobiela_2015, Li_2013, Naji_2012f, Qureshi_2016, Riva_2018j, Sobnach_2011f, Tay_2004, Thapa_2019f, Tupesis_2004f, Wildhaber_2005, Wnęk_2015f}, 9 cases (12\%) were motivated by protest \cite{Bhumi_2024f, Gardner_2017h, Losanoff_1996, Losanoff_1997e, Tupesis_2004f}, 9 cases (12\%) had another documented motivation \cite{Ali_2020f, Ali_2022g, Emamhadi_2018, Guinan_2019f, Peixoto_2017f, Sakellaridis_2008f, Trgo_2012f, Wadhwa_2015e, Yildiz_2016e}. \paragraph*{Object Characteristics} 51 cases (71\%) ingested a large diameter object (\textgreater{}2.5cm) \cite{Akay_2015f, Al-Faham_2020k, AlShaaibi_2021b, Alao_2006i, Ali_2017, Ali_2022g, Apikotoa_2022f, Atayan_2016, Berry_2021e, Bhasin_2014, CamachoDorado_2018, Cauchi_2002, Chang_2017f, Cox_2007, Csaky_1998e, DivsalarP._2023a, Emamhadi_2018, Gardner_2017h, Guinan_2019f, Jehangir_2019h, Jin_2023, Kariholu_2008, Kerestes_2019, Kobiela_2015, Kumar_2001, Kumar_2019f, Losanoff_1996, Losanoff_1997e, Mesfin_2022a, Misra_2013, Naji_2012f, Ohno_2005, Peixoto_2017f, Qureshi_2016, Riva_2018j, Sakellaridis_2008f, Sultan_2024f, Tanrikulu_2015e, Thapa_2019f, Trgo_2012f, Wnęk_2015f, Yildiz_2016e, fjbuilsRepeatedBehaviorDeliberate2024, teWildt_2010}, 44 cases (61\%) ingested multiple objects \cite{Ali_2020f, Apikotoa_2022f, Ataya_2013, Atayan_2016, Beecroft_1998, Bhattacharjee_2008, Bhumi_2024f, CamachoDorado_2018, Cauchi_2002, Emamhadi_2018, Farhadi_2024h, Fry_2010, Goldman_1998f, Guinan_2019f, Hardy_2023g, Jehangir_2019h, Jin_2023, Kar_2015, Kariholu_2008, Kobiela_2015, Kumar_2001, Kumar_2019f, Li_2013, Liu_2005, Losanoff_1996, Mesfin_2022a, Misra_2013, Naji_2012f, Ohno_2005, Sobnach_2011f, Sultan_2024f, Tammana_2012j, Tanrikulu_2015e, Tay_2004, Thapa_2019f, Wadhwa_2015e, Wildhaber_2005, Yasin_2009, fjbuilsRepeatedBehaviorDeliberate2024, teWildt_2010}, 34 cases (47\%) ingested a sharp object \cite{AlShaaibi_2021b, Alao_2006i, Apikotoa_2022f, Ataya_2013, Benoist_2019e, Bhasin_2014, Bhattacharjee_2008, CamachoDorado_2018, Csaky_1998e, DelgadoSalazar_2020c, DivsalarP._2023a, Emamhadi_2018, Farhadi_2024h, Fry_2010, Guinan_2019f, Hardy_2023g, Jehangir_2019h, Jin_2023, Kariholu_2008, Kobiela_2015, Kumar_2019f, Losanoff_1996, Losanoff_1997e, Mesfin_2022a, Misra_2013, Sobnach_2011f, Yasin_2009, teWildt_2010}, 32 cases (44\%) ingested a long object (\textgreater{}5cm) \cite{Al-Faham_2020k, AlShaaibi_2021b, Ali_2017, Ali_2022g, Atayan_2016, Bhasin_2014, CamachoDorado_2018, Chang_2017f, Cox_2007, Csaky_1998e, DivsalarP._2023a, Emamhadi_2018, Fry_2010, Gardner_2017h, Jin_2023, Kariholu_2008, Kerestes_2019, Kobiela_2015, Kumar_2019f, Mesfin_2022a, Misra_2013, Ohno_2005, Qureshi_2016, Sakellaridis_2008f, Sultan_2024f, Thapa_2019f, Trgo_2012f, Yasin_2009, Yildiz_2016e, teWildt_2010}, 9 cases (12\%) ingested a magnet \cite{Ali_2020f, Bhumi_2024f, Cauchi_2002, Liu_2005, Naji_2012f, Ohno_2005, Tanrikulu_2015e, Tay_2004, Wildhaber_2005}, 2 cases (3\%) ingested a button battery \cite{Berry_2021e, Bhumi_2024f}. \paragraph*{Outcomes} 48 cases (67\%) experienced a complication \cite{Ali_2017, Ali_2020f, Apikotoa_2022f, Atayan_2016, Beecroft_1998, Benoist_2019e, Berry_2021e, Bhasin_2014, Bhumi_2024f, CamachoDorado_2018, Cauchi_2002, Cox_2007, Csaky_1998e, DelgadoSalazar_2020c, DivsalarP._2023a, Emamhadi_2018, Farhadi_2024h, Fry_2010, Gardner_2017h, Goldman_1998f, Jin_2023, Kariholu_2008, Kerestes_2019, Kobiela_2015, Kumar_2001, Kumar_2019f, Liu_2005, Losanoff_1996, Mesfin_2022a, Misra_2013, Naji_2012f, Ohno_2005, Sakellaridis_2008f, Sobnach_2011f, Sultan_2024f, Tanrikulu_2015e, Tay_2004, Thapa_2019f, Trgo_2012f, Tupesis_2004f, Wildhaber_2005, Wnęk_2015f, Yasin_2009, Yildiz_2016e}, 44 cases (61\%) underwent surgery \cite{Al-Faham_2020k, AlShaaibi_2021b, Alao_2006i, Ali_2017, Ali_2020f, Atayan_2016, Beecroft_1998, Bhasin_2014, CamachoDorado_2018, Cauchi_2002, Chang_2017f, Cox_2007, Csaky_1998e, DelgadoSalazar_2020c, DivsalarP._2023a, Farhadi_2024h, Fry_2010, Gardner_2017h, Jin_2023, Kariholu_2008, Kerestes_2019, Kobiela_2015, Kumar_2019f, Liu_2005, Losanoff_1996, Losanoff_1997e, Mesfin_2022a, Misra_2013, Naji_2012f, Sobnach_2011f, Tanrikulu_2015e, Tay_2004, Thapa_2019f, Tupesis_2004f, Wildhaber_2005, Wnęk_2015f, Yasin_2009, Yildiz_2016e, fjbuilsRepeatedBehaviorDeliberate2024}, 31 cases (43\%) underwent endoscopy \cite{Akay_2015f, Ali_2022g, Apikotoa_2022f, Atayan_2016, Benoist_2019e, Berry_2021e, Bhasin_2014, Bhumi_2024f, CamachoDorado_2018, Chang_2017f, DelgadoSalazar_2020c, Gardner_2017h, Guinan_2019f, Hardy_2023g, Jehangir_2019h, Kariholu_2008, Li_2013, Liu_2005, Ohno_2005, Peixoto_2017f, Qureshi_2016, Riva_2018j, Sakellaridis_2008f, Sultan_2024f, Tammana_2012j, Tanrikulu_2015e, Trgo_2012f, Wadhwa_2015e, Wnęk_2015f, teWildt_2010}, 7 cases (10\%) were managed conservatively \cite{Ataya_2013, Bhattacharjee_2008, DivsalarP._2023a, Emamhadi_2018, Goldman_1998f, Kar_2015, Kumar_2001}, 2 cases (3\%) died \cite{Emamhadi_2018, Kumar_2001}. All 90 were male gender. 90 cases (100\%) were detained at the time of ingestion \cite{Elghali_2016, Karp_1991b, Lee_2007}, 88 cases (98\%) were intentional ingestions \cite{Elghali_2016, Karp_1991b, Lee_2007}, 30 cases (33\%) had a psychiatric history documented \cite{Elghali_2016, Karp_1991b, Lee_2007}, 2 cases (2\%) had a history of prior ingestion \cite{Elghali_2016}. No cases were reported for were psychiatric inpatients, were displaced people, were under the influence of alcohol at the time of ingestion, and had a severe disability history.
\paragraph*{Motivation}  70 cases (78\%) reported protest motivation \cite{Elghali_2016, Karp_1991b, Lee_2007}, 12 cases (13\%) reported psychiatric motivation \cite{Karp_1991b}, 6 cases (7\%) reported self-harm motivation \cite{Elghali_2016, Karp_1991b}. No cases were reported for psychosocial motivation and other motivation.
\paragraph*{Object Characteristics}  68 cases (76\%) involved sharp object ingestion \cite{Elghali_2016, Karp_1991b, Lee_2007}, 32 cases (36\%) involved long (\textgreater 5cm) object ingestion \cite{Lee_2007}, 25 cases (28\%) involved ingestion of multiple objects \cite{Elghali_2016, Lee_2007}. No cases were reported for button battery ingestion, magnet ingestion, and involved large diameter (\textgreater 2.5cm) object ingestion.
\paragraph*{Outcomes}  47 cases (52\%) underwent endoscopic intervention \cite{Elghali_2016, Lee_2007}, 29 cases (32\%) were managed conservatively \cite{Elghali_2016, Karp_1991b}, 15 cases (17\%) underwent surgical intervention \cite{Elghali_2016, Karp_1991b, Lee_2007}, 6 cases (7\%) reported complications \cite{Lee_2007}, 1 case (1\%) died \cite{Elghali_2016}.
\paragraph*{Geographical Location}Cases were recorded in 33 countries: 13 cases from USA \cite{Alao_2006i, Ataya_2013, Bhumi_2024f, Fry_2010, Guinan_2019f, Hardy_2023g, Jehangir_2019h, Kerestes_2019, Kumar_2001, Liu_2005, Tammana_2012j, Tay_2004, Tupesis_2004f}; 7 cases from India \cite{Bhasin_2014, Bhattacharjee_2008, Kar_2015, Kariholu_2008, Kumar_2019f, Misra_2013, Wadhwa_2015e} and UK \cite{Beecroft_1998, Berry_2021e, Cauchi_2002, Cox_2007, Gardner_2017h, Qureshi_2016}; 6 cases from Bulgaria \cite{Losanoff_1996, Losanoff_1997e}; 5 cases from Iran \cite{DivsalarP._2023a, Emamhadi_2018, Farhadi_2024h}; 4 cases from Turkey \cite{Akay_2015f, Atayan_2016, Tanrikulu_2015e, Yildiz_2016e}; 2 cases from China \cite{Jin_2023, Li_2013}, Poland \cite{Kobiela_2015, Wnęk_2015f}, and Spain \cite{CamachoDorado_2018, fjbuilsRepeatedBehaviorDeliberate2024}; 1 case from Australia \cite{Apikotoa_2022f}, Bahrain \cite{Ali_2020f}, Croatia \cite{Trgo_2012f}, Ecuador \cite{DelgadoSalazar_2020c}, Egypt \cite{Ali_2022g}, Ethiopia \cite{Mesfin_2022a}, Germany \cite{teWildt_2010}, Greece \cite{Sakellaridis_2008f}, Hungary \cite{Csaky_1998e}, Iraq \cite{Al-Faham_2020k}, Israel \cite{Goldman_1998f}, Italy \cite{Riva_2018j}, Japan \cite{Ohno_2005}, Nepal \cite{Thapa_2019f}, Netherlands \cite{Benoist_2019e}, Oman \cite{AlShaaibi_2021b}, Pakistan \cite{Yasin_2009}, Portugal \cite{Peixoto_2017f}, Qatar \cite{Ali_2017}, Saudi Arabia \cite{Sultan_2024f}, South Africa \cite{Sobnach_2011f}, Sweden \cite{Naji_2012f}, Switzerland \cite{Wildhaber_2005}, and Taiwan \cite{Chang_2017f}. \paragraph*{Gender} 43 cases (60\%) were male \cite{Akay_2015f, Al-Faham_2020k, Alao_2006i, Ali_2017, Ali_2022g, Apikotoa_2022f, Atayan_2016, Benoist_2019e, Berry_2021e, Bhumi_2024f, CamachoDorado_2018, Csaky_1998e, Emamhadi_2018, Farhadi_2024h, Fry_2010, Gardner_2017h, Guinan_2019f, Jehangir_2019h, Jin_2023, Kobiela_2015, Kumar_2001, Kumar_2019f, Liu_2005, Losanoff_1996, Losanoff_1997e, Mesfin_2022a, Misra_2013, Qureshi_2016, Riva_2018j, Sobnach_2011f, Tammana_2012j, Tanrikulu_2015e, Tay_2004, Thapa_2019f, Trgo_2012f, Wadhwa_2015e, Yasin_2009, teWildt_2010}, 28 cases (39\%) were female \cite{AlShaaibi_2021b, Ali_2020f, Ataya_2013, Beecroft_1998, Bhasin_2014, Bhattacharjee_2008, Cauchi_2002, Chang_2017f, Cox_2007, DelgadoSalazar_2020c, DivsalarP._2023a, Goldman_1998f, Hardy_2023g, Kar_2015, Kariholu_2008, Kerestes_2019, Li_2013, Naji_2012f, Ohno_2005, Peixoto_2017f, Sakellaridis_2008f, Sultan_2024f, Tupesis_2004f, Wildhaber_2005, Wnęk_2015f, Yildiz_2016e}, 1 case (1\%) had no gender recorded \cite{fjbuilsRepeatedBehaviorDeliberate2024}. \paragraph*{Age Group} 25 cases (35\%) were between 26 and 40 years of age \cite{Alao_2006i, Ali_2022g, Apikotoa_2022f, Ataya_2013, Benoist_2019e, Bhasin_2014, Chang_2017f, Cox_2007, DelgadoSalazar_2020c, Farhadi_2024h, Fry_2010, Gardner_2017h, Guinan_2019f, Jin_2023, Kumar_2019f, Losanoff_1996, Misra_2013, Qureshi_2016, Riva_2018j, Sakellaridis_2008f, Tammana_2012j, Trgo_2012f, Wnęk_2015f, Yildiz_2016e, fjbuilsRepeatedBehaviorDeliberate2024}, 18 cases (25\%) were between 18 and 25 years of age \cite{Akay_2015f, Ali_2017, Atayan_2016, Bhattacharjee_2008, Csaky_1998e, Kar_2015, Kariholu_2008, Kobiela_2015, Losanoff_1996, Losanoff_1997e, Mesfin_2022a, Peixoto_2017f, Sobnach_2011f, Tupesis_2004f, Yasin_2009}, 13 cases (18\%) were under 18 years of age \cite{AlShaaibi_2021b, Ali_2020f, Cauchi_2002, DivsalarP._2023a, Goldman_1998f, Liu_2005, Naji_2012f, Ohno_2005, Tanrikulu_2015e, Tay_2004, Wildhaber_2005}, 11 cases (15\%) were between 41 and 60 years of age \cite{Al-Faham_2020k, Bhumi_2024f, CamachoDorado_2018, Emamhadi_2018, Hardy_2023g, Jehangir_2019h, Kumar_2001, Sultan_2024f, Thapa_2019f, Wadhwa_2015e, teWildt_2010}, 3 cases (4\%) were over 60 years of age \cite{Beecroft_1998, Kerestes_2019, Li_2013}, 2 cases (3\%) had no age documented \cite{Berry_2021e}. \paragraph*{Population} 36 cases (50\%) had a psychiatric history \cite{AlShaaibi_2021b, Alao_2006i, Ali_2020f, Apikotoa_2022f, Ataya_2013, Atayan_2016, Beecroft_1998, CamachoDorado_2018, Chang_2017f, DelgadoSalazar_2020c, DivsalarP._2023a, Farhadi_2024h, Fry_2010, Guinan_2019f, Hardy_2023g, Jehangir_2019h, Jin_2023, Kar_2015, Kerestes_2019, Kobiela_2015, Kumar_2001, Kumar_2019f, Liu_2005, Mesfin_2022a, Misra_2013, Ohno_2005, Peixoto_2017f, Sakellaridis_2008f, Sultan_2024f, Tammana_2012j, Tanrikulu_2015e, Yildiz_2016e, fjbuilsRepeatedBehaviorDeliberate2024, teWildt_2010}, 19 cases (26\%) had ingested previously \cite{Alao_2006i, Apikotoa_2022f, Berry_2021e, Bhattacharjee_2008, Csaky_1998e, DivsalarP._2023a, Emamhadi_2018, Guinan_2019f, Jehangir_2019h, Jin_2023, Liu_2005, Sakellaridis_2008f, Tanrikulu_2015e, Thapa_2019f, Yildiz_2016e, fjbuilsRepeatedBehaviorDeliberate2024, teWildt_2010}, 12 cases (17\%) were detained persons \cite{Alao_2006i, Ali_2022g, Apikotoa_2022f, Losanoff_1996, Losanoff_1997e, Qureshi_2016, Tammana_2012j, Trgo_2012f}, 7 cases (10\%) were severely disabled \cite{Atayan_2016, Kerestes_2019, Liu_2005, Ohno_2005, Peixoto_2017f, Yildiz_2016e, teWildt_2010}, 4 cases (6\%) were psychiatric inpatients \cite{DivsalarP._2023a, fjbuilsRepeatedBehaviorDeliberate2024, teWildt_2010}, 3 cases (4\%) were under the influence of alcohol \cite{Benoist_2019e, Csaky_1998e, Thapa_2019f}, 2 cases (3\%) were displaced people \cite{Akay_2015f, Gardner_2017h}. \paragraph*{Motivation} 34 cases (47\%) had a psychiatric motivation \cite{Al-Faham_2020k, Alao_2006i, Ali_2020f, Apikotoa_2022f, Ataya_2013, Atayan_2016, Bhasin_2014, Bhattacharjee_2008, DelgadoSalazar_2020c, DivsalarP._2023a, Emamhadi_2018, Farhadi_2024h, Guinan_2019f, Hardy_2023g, Jehangir_2019h, Jin_2023, Kar_2015, Kariholu_2008, Kerestes_2019, Kobiela_2015, Kumar_2001, Kumar_2019f, Li_2013, Liu_2005, Misra_2013, Ohno_2005, Sakellaridis_2008f, Sultan_2024f, Tammana_2012j, Tanrikulu_2015e, Yasin_2009, teWildt_2010}, 21 cases (29\%) were motivated by self-harm intention \cite{Al-Faham_2020k, AlShaaibi_2021b, Alao_2006i, Ali_2017, CamachoDorado_2018, Chang_2017f, Cox_2007, Csaky_1998e, Fry_2010, Li_2013, Losanoff_1996, Losanoff_1997e, Mesfin_2022a, Sakellaridis_2008f, Tammana_2012j, Tanrikulu_2015e, fjbuilsRepeatedBehaviorDeliberate2024}, 17 cases (24\%) had a psychosocial motivation \cite{Akay_2015f, Benoist_2019e, Bhattacharjee_2008, Cauchi_2002, Goldman_1998f, Hardy_2023g, Kobiela_2015, Li_2013, Naji_2012f, Qureshi_2016, Riva_2018j, Sobnach_2011f, Tay_2004, Thapa_2019f, Tupesis_2004f, Wildhaber_2005, Wnęk_2015f}, 9 cases (12\%) were motivated by protest \cite{Bhumi_2024f, Gardner_2017h, Losanoff_1996, Losanoff_1997e, Tupesis_2004f}, 9 cases (12\%) had another documented motivation \cite{Ali_2020f, Ali_2022g, Emamhadi_2018, Guinan_2019f, Peixoto_2017f, Sakellaridis_2008f, Trgo_2012f, Wadhwa_2015e, Yildiz_2016e}. \paragraph*{Object Characteristics} 51 cases (71\%) ingested a large diameter object (\textgreater{}2.5cm) \cite{Akay_2015f, Al-Faham_2020k, AlShaaibi_2021b, Alao_2006i, Ali_2017, Ali_2022g, Apikotoa_2022f, Atayan_2016, Berry_2021e, Bhasin_2014, CamachoDorado_2018, Cauchi_2002, Chang_2017f, Cox_2007, Csaky_1998e, DivsalarP._2023a, Emamhadi_2018, Gardner_2017h, Guinan_2019f, Jehangir_2019h, Jin_2023, Kariholu_2008, Kerestes_2019, Kobiela_2015, Kumar_2001, Kumar_2019f, Losanoff_1996, Losanoff_1997e, Mesfin_2022a, Misra_2013, Naji_2012f, Ohno_2005, Peixoto_2017f, Qureshi_2016, Riva_2018j, Sakellaridis_2008f, Sultan_2024f, Tanrikulu_2015e, Thapa_2019f, Trgo_2012f, Wnęk_2015f, Yildiz_2016e, fjbuilsRepeatedBehaviorDeliberate2024, teWildt_2010}, 44 cases (61\%) ingested multiple objects \cite{Ali_2020f, Apikotoa_2022f, Ataya_2013, Atayan_2016, Beecroft_1998, Bhattacharjee_2008, Bhumi_2024f, CamachoDorado_2018, Cauchi_2002, Emamhadi_2018, Farhadi_2024h, Fry_2010, Goldman_1998f, Guinan_2019f, Hardy_2023g, Jehangir_2019h, Jin_2023, Kar_2015, Kariholu_2008, Kobiela_2015, Kumar_2001, Kumar_2019f, Li_2013, Liu_2005, Losanoff_1996, Mesfin_2022a, Misra_2013, Naji_2012f, Ohno_2005, Sobnach_2011f, Sultan_2024f, Tammana_2012j, Tanrikulu_2015e, Tay_2004, Thapa_2019f, Wadhwa_2015e, Wildhaber_2005, Yasin_2009, fjbuilsRepeatedBehaviorDeliberate2024, teWildt_2010}, 34 cases (47\%) ingested a sharp object \cite{AlShaaibi_2021b, Alao_2006i, Apikotoa_2022f, Ataya_2013, Benoist_2019e, Bhasin_2014, Bhattacharjee_2008, CamachoDorado_2018, Csaky_1998e, DelgadoSalazar_2020c, DivsalarP._2023a, Emamhadi_2018, Farhadi_2024h, Fry_2010, Guinan_2019f, Hardy_2023g, Jehangir_2019h, Jin_2023, Kariholu_2008, Kobiela_2015, Kumar_2019f, Losanoff_1996, Losanoff_1997e, Mesfin_2022a, Misra_2013, Sobnach_2011f, Yasin_2009, teWildt_2010}, 32 cases (44\%) ingested a long object (\textgreater{}5cm) \cite{Al-Faham_2020k, AlShaaibi_2021b, Ali_2017, Ali_2022g, Atayan_2016, Bhasin_2014, CamachoDorado_2018, Chang_2017f, Cox_2007, Csaky_1998e, DivsalarP._2023a, Emamhadi_2018, Fry_2010, Gardner_2017h, Jin_2023, Kariholu_2008, Kerestes_2019, Kobiela_2015, Kumar_2019f, Mesfin_2022a, Misra_2013, Ohno_2005, Qureshi_2016, Sakellaridis_2008f, Sultan_2024f, Thapa_2019f, Trgo_2012f, Yasin_2009, Yildiz_2016e, teWildt_2010}, 9 cases (12\%) ingested a magnet \cite{Ali_2020f, Bhumi_2024f, Cauchi_2002, Liu_2005, Naji_2012f, Ohno_2005, Tanrikulu_2015e, Tay_2004, Wildhaber_2005}, 2 cases (3\%) ingested a button battery \cite{Berry_2021e, Bhumi_2024f}. \paragraph*{Outcomes} 48 cases (67\%) experienced a complication \cite{Ali_2017, Ali_2020f, Apikotoa_2022f, Atayan_2016, Beecroft_1998, Benoist_2019e, Berry_2021e, Bhasin_2014, Bhumi_2024f, CamachoDorado_2018, Cauchi_2002, Cox_2007, Csaky_1998e, DelgadoSalazar_2020c, DivsalarP._2023a, Emamhadi_2018, Farhadi_2024h, Fry_2010, Gardner_2017h, Goldman_1998f, Jin_2023, Kariholu_2008, Kerestes_2019, Kobiela_2015, Kumar_2001, Kumar_2019f, Liu_2005, Losanoff_1996, Mesfin_2022a, Misra_2013, Naji_2012f, Ohno_2005, Sakellaridis_2008f, Sobnach_2011f, Sultan_2024f, Tanrikulu_2015e, Tay_2004, Thapa_2019f, Trgo_2012f, Tupesis_2004f, Wildhaber_2005, Wnęk_2015f, Yasin_2009, Yildiz_2016e}, 44 cases (61\%) underwent surgery \cite{Al-Faham_2020k, AlShaaibi_2021b, Alao_2006i, Ali_2017, Ali_2020f, Atayan_2016, Beecroft_1998, Bhasin_2014, CamachoDorado_2018, Cauchi_2002, Chang_2017f, Cox_2007, Csaky_1998e, DelgadoSalazar_2020c, DivsalarP._2023a, Farhadi_2024h, Fry_2010, Gardner_2017h, Jin_2023, Kariholu_2008, Kerestes_2019, Kobiela_2015, Kumar_2019f, Liu_2005, Losanoff_1996, Losanoff_1997e, Mesfin_2022a, Misra_2013, Naji_2012f, Sobnach_2011f, Tanrikulu_2015e, Tay_2004, Thapa_2019f, Tupesis_2004f, Wildhaber_2005, Wnęk_2015f, Yasin_2009, Yildiz_2016e, fjbuilsRepeatedBehaviorDeliberate2024}, 31 cases (43\%) underwent endoscopy \cite{Akay_2015f, Ali_2022g, Apikotoa_2022f, Atayan_2016, Benoist_2019e, Berry_2021e, Bhasin_2014, Bhumi_2024f, CamachoDorado_2018, Chang_2017f, DelgadoSalazar_2020c, Gardner_2017h, Guinan_2019f, Hardy_2023g, Jehangir_2019h, Kariholu_2008, Li_2013, Liu_2005, Ohno_2005, Peixoto_2017f, Qureshi_2016, Riva_2018j, Sakellaridis_2008f, Sultan_2024f, Tammana_2012j, Tanrikulu_2015e, Trgo_2012f, Wadhwa_2015e, Wnęk_2015f, teWildt_2010}, 7 cases (10\%) were managed conservatively \cite{Ataya_2013, Bhattacharjee_2008, DivsalarP._2023a, Emamhadi_2018, Goldman_1998f, Kar_2015, Kumar_2001}, 2 cases (3\%) died \cite{Emamhadi_2018, Kumar_2001}. All 90 were male gender. 90 cases (100\%) were detained at the time of ingestion \cite{Elghali_2016, Karp_1991b, Lee_2007}, 88 cases (98\%) were intentional ingestions \cite{Elghali_2016, Karp_1991b, Lee_2007}, 30 cases (33\%) had a psychiatric history documented \cite{Elghali_2016, Karp_1991b, Lee_2007}, 2 cases (2\%) had a history of prior ingestion \cite{Elghali_2016}. No cases were reported for were psychiatric inpatients, were displaced people, were under the influence of alcohol at the time of ingestion, and had a severe disability history.
\paragraph*{Motivation}  70 cases (78\%) reported protest motivation \cite{Elghali_2016, Karp_1991b, Lee_2007}, 12 cases (13\%) reported psychiatric motivation \cite{Karp_1991b}, 6 cases (7\%) reported self-harm motivation \cite{Elghali_2016, Karp_1991b}. No cases were reported for psychosocial motivation and other motivation.
\paragraph*{Object Characteristics}  68 cases (76\%) involved sharp object ingestion \cite{Elghali_2016, Karp_1991b, Lee_2007}, 32 cases (36\%) involved long (\textgreater 5cm) object ingestion \cite{Lee_2007}, 25 cases (28\%) involved ingestion of multiple objects \cite{Elghali_2016, Lee_2007}. No cases were reported for button battery ingestion, magnet ingestion, and involved large diameter (\textgreater 2.5cm) object ingestion.
\paragraph*{Outcomes}  47 cases (52\%) underwent endoscopic intervention \cite{Elghali_2016, Lee_2007}, 29 cases (32\%) were managed conservatively \cite{Elghali_2016, Karp_1991b}, 15 cases (17\%) underwent surgical intervention \cite{Elghali_2016, Karp_1991b, Lee_2007}, 6 cases (7\%) reported complications \cite{Lee_2007}, 1 case (1\%) died \cite{Elghali_2016}.
\paragraph*{Geographical Location}Cases were recorded in 33 countries: 13 cases from USA \cite{Alao_2006i, Ataya_2013, Bhumi_2024f, Fry_2010, Guinan_2019f, Hardy_2023g, Jehangir_2019h, Kerestes_2019, Kumar_2001, Liu_2005, Tammana_2012j, Tay_2004, Tupesis_2004f}; 7 cases from India \cite{Bhasin_2014, Bhattacharjee_2008, Kar_2015, Kariholu_2008, Kumar_2019f, Misra_2013, Wadhwa_2015e} and UK \cite{Beecroft_1998, Berry_2021e, Cauchi_2002, Cox_2007, Gardner_2017h, Qureshi_2016}; 6 cases from Bulgaria \cite{Losanoff_1996, Losanoff_1997e}; 5 cases from Iran \cite{DivsalarP._2023a, Emamhadi_2018, Farhadi_2024h}; 4 cases from Turkey \cite{Akay_2015f, Atayan_2016, Tanrikulu_2015e, Yildiz_2016e}; 2 cases from China \cite{Jin_2023, Li_2013}, Poland \cite{Kobiela_2015, Wnęk_2015f}, and Spain \cite{CamachoDorado_2018, fjbuilsRepeatedBehaviorDeliberate2024}; 1 case from Australia \cite{Apikotoa_2022f}, Bahrain \cite{Ali_2020f}, Croatia \cite{Trgo_2012f}, Ecuador \cite{DelgadoSalazar_2020c}, Egypt \cite{Ali_2022g}, Ethiopia \cite{Mesfin_2022a}, Germany \cite{teWildt_2010}, Greece \cite{Sakellaridis_2008f}, Hungary \cite{Csaky_1998e}, Iraq \cite{Al-Faham_2020k}, Israel \cite{Goldman_1998f}, Italy \cite{Riva_2018j}, Japan \cite{Ohno_2005}, Nepal \cite{Thapa_2019f}, Netherlands \cite{Benoist_2019e}, Oman \cite{AlShaaibi_2021b}, Pakistan \cite{Yasin_2009}, Portugal \cite{Peixoto_2017f}, Qatar \cite{Ali_2017}, Saudi Arabia \cite{Sultan_2024f}, South Africa \cite{Sobnach_2011f}, Sweden \cite{Naji_2012f}, Switzerland \cite{Wildhaber_2005}, and Taiwan \cite{Chang_2017f}. \paragraph*{Gender} 43 cases (60\%) were male \cite{Akay_2015f, Al-Faham_2020k, Alao_2006i, Ali_2017, Ali_2022g, Apikotoa_2022f, Atayan_2016, Benoist_2019e, Berry_2021e, Bhumi_2024f, CamachoDorado_2018, Csaky_1998e, Emamhadi_2018, Farhadi_2024h, Fry_2010, Gardner_2017h, Guinan_2019f, Jehangir_2019h, Jin_2023, Kobiela_2015, Kumar_2001, Kumar_2019f, Liu_2005, Losanoff_1996, Losanoff_1997e, Mesfin_2022a, Misra_2013, Qureshi_2016, Riva_2018j, Sobnach_2011f, Tammana_2012j, Tanrikulu_2015e, Tay_2004, Thapa_2019f, Trgo_2012f, Wadhwa_2015e, Yasin_2009, teWildt_2010}, 28 cases (39\%) were female \cite{AlShaaibi_2021b, Ali_2020f, Ataya_2013, Beecroft_1998, Bhasin_2014, Bhattacharjee_2008, Cauchi_2002, Chang_2017f, Cox_2007, DelgadoSalazar_2020c, DivsalarP._2023a, Goldman_1998f, Hardy_2023g, Kar_2015, Kariholu_2008, Kerestes_2019, Li_2013, Naji_2012f, Ohno_2005, Peixoto_2017f, Sakellaridis_2008f, Sultan_2024f, Tupesis_2004f, Wildhaber_2005, Wnęk_2015f, Yildiz_2016e}, 1 case (1\%) had no gender recorded \cite{fjbuilsRepeatedBehaviorDeliberate2024}. \paragraph*{Age Group} 25 cases (35\%) were between 26 and 40 years of age \cite{Alao_2006i, Ali_2022g, Apikotoa_2022f, Ataya_2013, Benoist_2019e, Bhasin_2014, Chang_2017f, Cox_2007, DelgadoSalazar_2020c, Farhadi_2024h, Fry_2010, Gardner_2017h, Guinan_2019f, Jin_2023, Kumar_2019f, Losanoff_1996, Misra_2013, Qureshi_2016, Riva_2018j, Sakellaridis_2008f, Tammana_2012j, Trgo_2012f, Wnęk_2015f, Yildiz_2016e, fjbuilsRepeatedBehaviorDeliberate2024}, 18 cases (25\%) were between 18 and 25 years of age \cite{Akay_2015f, Ali_2017, Atayan_2016, Bhattacharjee_2008, Csaky_1998e, Kar_2015, Kariholu_2008, Kobiela_2015, Losanoff_1996, Losanoff_1997e, Mesfin_2022a, Peixoto_2017f, Sobnach_2011f, Tupesis_2004f, Yasin_2009}, 13 cases (18\%) were under 18 years of age \cite{AlShaaibi_2021b, Ali_2020f, Cauchi_2002, DivsalarP._2023a, Goldman_1998f, Liu_2005, Naji_2012f, Ohno_2005, Tanrikulu_2015e, Tay_2004, Wildhaber_2005}, 11 cases (15\%) were between 41 and 60 years of age \cite{Al-Faham_2020k, Bhumi_2024f, CamachoDorado_2018, Emamhadi_2018, Hardy_2023g, Jehangir_2019h, Kumar_2001, Sultan_2024f, Thapa_2019f, Wadhwa_2015e, teWildt_2010}, 3 cases (4\%) were over 60 years of age \cite{Beecroft_1998, Kerestes_2019, Li_2013}, 2 cases (3\%) had no age documented \cite{Berry_2021e}. \paragraph*{Population} 36 cases (50\%) had a psychiatric history \cite{AlShaaibi_2021b, Alao_2006i, Ali_2020f, Apikotoa_2022f, Ataya_2013, Atayan_2016, Beecroft_1998, CamachoDorado_2018, Chang_2017f, DelgadoSalazar_2020c, DivsalarP._2023a, Farhadi_2024h, Fry_2010, Guinan_2019f, Hardy_2023g, Jehangir_2019h, Jin_2023, Kar_2015, Kerestes_2019, Kobiela_2015, Kumar_2001, Kumar_2019f, Liu_2005, Mesfin_2022a, Misra_2013, Ohno_2005, Peixoto_2017f, Sakellaridis_2008f, Sultan_2024f, Tammana_2012j, Tanrikulu_2015e, Yildiz_2016e, fjbuilsRepeatedBehaviorDeliberate2024, teWildt_2010}, 19 cases (26\%) had ingested previously \cite{Alao_2006i, Apikotoa_2022f, Berry_2021e, Bhattacharjee_2008, Csaky_1998e, DivsalarP._2023a, Emamhadi_2018, Guinan_2019f, Jehangir_2019h, Jin_2023, Liu_2005, Sakellaridis_2008f, Tanrikulu_2015e, Thapa_2019f, Yildiz_2016e, fjbuilsRepeatedBehaviorDeliberate2024, teWildt_2010}, 12 cases (17\%) were detained persons \cite{Alao_2006i, Ali_2022g, Apikotoa_2022f, Losanoff_1996, Losanoff_1997e, Qureshi_2016, Tammana_2012j, Trgo_2012f}, 7 cases (10\%) were severely disabled \cite{Atayan_2016, Kerestes_2019, Liu_2005, Ohno_2005, Peixoto_2017f, Yildiz_2016e, teWildt_2010}, 4 cases (6\%) were psychiatric inpatients \cite{DivsalarP._2023a, fjbuilsRepeatedBehaviorDeliberate2024, teWildt_2010}, 3 cases (4\%) were under the influence of alcohol \cite{Benoist_2019e, Csaky_1998e, Thapa_2019f}, 2 cases (3\%) were displaced people \cite{Akay_2015f, Gardner_2017h}. \paragraph*{Motivation} 34 cases (47\%) had a psychiatric motivation \cite{Al-Faham_2020k, Alao_2006i, Ali_2020f, Apikotoa_2022f, Ataya_2013, Atayan_2016, Bhasin_2014, Bhattacharjee_2008, DelgadoSalazar_2020c, DivsalarP._2023a, Emamhadi_2018, Farhadi_2024h, Guinan_2019f, Hardy_2023g, Jehangir_2019h, Jin_2023, Kar_2015, Kariholu_2008, Kerestes_2019, Kobiela_2015, Kumar_2001, Kumar_2019f, Li_2013, Liu_2005, Misra_2013, Ohno_2005, Sakellaridis_2008f, Sultan_2024f, Tammana_2012j, Tanrikulu_2015e, Yasin_2009, teWildt_2010}, 21 cases (29\%) were motivated by self-harm intention \cite{Al-Faham_2020k, AlShaaibi_2021b, Alao_2006i, Ali_2017, CamachoDorado_2018, Chang_2017f, Cox_2007, Csaky_1998e, Fry_2010, Li_2013, Losanoff_1996, Losanoff_1997e, Mesfin_2022a, Sakellaridis_2008f, Tammana_2012j, Tanrikulu_2015e, fjbuilsRepeatedBehaviorDeliberate2024}, 17 cases (24\%) had a psychosocial motivation \cite{Akay_2015f, Benoist_2019e, Bhattacharjee_2008, Cauchi_2002, Goldman_1998f, Hardy_2023g, Kobiela_2015, Li_2013, Naji_2012f, Qureshi_2016, Riva_2018j, Sobnach_2011f, Tay_2004, Thapa_2019f, Tupesis_2004f, Wildhaber_2005, Wnęk_2015f}, 9 cases (12\%) were motivated by protest \cite{Bhumi_2024f, Gardner_2017h, Losanoff_1996, Losanoff_1997e, Tupesis_2004f}, 9 cases (12\%) had another documented motivation \cite{Ali_2020f, Ali_2022g, Emamhadi_2018, Guinan_2019f, Peixoto_2017f, Sakellaridis_2008f, Trgo_2012f, Wadhwa_2015e, Yildiz_2016e}. \paragraph*{Object Characteristics} 51 cases (71\%) ingested a large diameter object (\textgreater{}2.5cm) \cite{Akay_2015f, Al-Faham_2020k, AlShaaibi_2021b, Alao_2006i, Ali_2017, Ali_2022g, Apikotoa_2022f, Atayan_2016, Berry_2021e, Bhasin_2014, CamachoDorado_2018, Cauchi_2002, Chang_2017f, Cox_2007, Csaky_1998e, DivsalarP._2023a, Emamhadi_2018, Gardner_2017h, Guinan_2019f, Jehangir_2019h, Jin_2023, Kariholu_2008, Kerestes_2019, Kobiela_2015, Kumar_2001, Kumar_2019f, Losanoff_1996, Losanoff_1997e, Mesfin_2022a, Misra_2013, Naji_2012f, Ohno_2005, Peixoto_2017f, Qureshi_2016, Riva_2018j, Sakellaridis_2008f, Sultan_2024f, Tanrikulu_2015e, Thapa_2019f, Trgo_2012f, Wnęk_2015f, Yildiz_2016e, fjbuilsRepeatedBehaviorDeliberate2024, teWildt_2010}, 44 cases (61\%) ingested multiple objects \cite{Ali_2020f, Apikotoa_2022f, Ataya_2013, Atayan_2016, Beecroft_1998, Bhattacharjee_2008, Bhumi_2024f, CamachoDorado_2018, Cauchi_2002, Emamhadi_2018, Farhadi_2024h, Fry_2010, Goldman_1998f, Guinan_2019f, Hardy_2023g, Jehangir_2019h, Jin_2023, Kar_2015, Kariholu_2008, Kobiela_2015, Kumar_2001, Kumar_2019f, Li_2013, Liu_2005, Losanoff_1996, Mesfin_2022a, Misra_2013, Naji_2012f, Ohno_2005, Sobnach_2011f, Sultan_2024f, Tammana_2012j, Tanrikulu_2015e, Tay_2004, Thapa_2019f, Wadhwa_2015e, Wildhaber_2005, Yasin_2009, fjbuilsRepeatedBehaviorDeliberate2024, teWildt_2010}, 34 cases (47\%) ingested a sharp object \cite{AlShaaibi_2021b, Alao_2006i, Apikotoa_2022f, Ataya_2013, Benoist_2019e, Bhasin_2014, Bhattacharjee_2008, CamachoDorado_2018, Csaky_1998e, DelgadoSalazar_2020c, DivsalarP._2023a, Emamhadi_2018, Farhadi_2024h, Fry_2010, Guinan_2019f, Hardy_2023g, Jehangir_2019h, Jin_2023, Kariholu_2008, Kobiela_2015, Kumar_2019f, Losanoff_1996, Losanoff_1997e, Mesfin_2022a, Misra_2013, Sobnach_2011f, Yasin_2009, teWildt_2010}, 32 cases (44\%) ingested a long object (\textgreater{}5cm) \cite{Al-Faham_2020k, AlShaaibi_2021b, Ali_2017, Ali_2022g, Atayan_2016, Bhasin_2014, CamachoDorado_2018, Chang_2017f, Cox_2007, Csaky_1998e, DivsalarP._2023a, Emamhadi_2018, Fry_2010, Gardner_2017h, Jin_2023, Kariholu_2008, Kerestes_2019, Kobiela_2015, Kumar_2019f, Mesfin_2022a, Misra_2013, Ohno_2005, Qureshi_2016, Sakellaridis_2008f, Sultan_2024f, Thapa_2019f, Trgo_2012f, Yasin_2009, Yildiz_2016e, teWildt_2010}, 9 cases (12\%) ingested a magnet \cite{Ali_2020f, Bhumi_2024f, Cauchi_2002, Liu_2005, Naji_2012f, Ohno_2005, Tanrikulu_2015e, Tay_2004, Wildhaber_2005}, 2 cases (3\%) ingested a button battery \cite{Berry_2021e, Bhumi_2024f}. \paragraph*{Outcomes} 48 cases (67\%) experienced a complication \cite{Ali_2017, Ali_2020f, Apikotoa_2022f, Atayan_2016, Beecroft_1998, Benoist_2019e, Berry_2021e, Bhasin_2014, Bhumi_2024f, CamachoDorado_2018, Cauchi_2002, Cox_2007, Csaky_1998e, DelgadoSalazar_2020c, DivsalarP._2023a, Emamhadi_2018, Farhadi_2024h, Fry_2010, Gardner_2017h, Goldman_1998f, Jin_2023, Kariholu_2008, Kerestes_2019, Kobiela_2015, Kumar_2001, Kumar_2019f, Liu_2005, Losanoff_1996, Mesfin_2022a, Misra_2013, Naji_2012f, Ohno_2005, Sakellaridis_2008f, Sobnach_2011f, Sultan_2024f, Tanrikulu_2015e, Tay_2004, Thapa_2019f, Trgo_2012f, Tupesis_2004f, Wildhaber_2005, Wnęk_2015f, Yasin_2009, Yildiz_2016e}, 44 cases (61\%) underwent surgery \cite{Al-Faham_2020k, AlShaaibi_2021b, Alao_2006i, Ali_2017, Ali_2020f, Atayan_2016, Beecroft_1998, Bhasin_2014, CamachoDorado_2018, Cauchi_2002, Chang_2017f, Cox_2007, Csaky_1998e, DelgadoSalazar_2020c, DivsalarP._2023a, Farhadi_2024h, Fry_2010, Gardner_2017h, Jin_2023, Kariholu_2008, Kerestes_2019, Kobiela_2015, Kumar_2019f, Liu_2005, Losanoff_1996, Losanoff_1997e, Mesfin_2022a, Misra_2013, Naji_2012f, Sobnach_2011f, Tanrikulu_2015e, Tay_2004, Thapa_2019f, Tupesis_2004f, Wildhaber_2005, Wnęk_2015f, Yasin_2009, Yildiz_2016e, fjbuilsRepeatedBehaviorDeliberate2024}, 31 cases (43\%) underwent endoscopy \cite{Akay_2015f, Ali_2022g, Apikotoa_2022f, Atayan_2016, Benoist_2019e, Berry_2021e, Bhasin_2014, Bhumi_2024f, CamachoDorado_2018, Chang_2017f, DelgadoSalazar_2020c, Gardner_2017h, Guinan_2019f, Hardy_2023g, Jehangir_2019h, Kariholu_2008, Li_2013, Liu_2005, Ohno_2005, Peixoto_2017f, Qureshi_2016, Riva_2018j, Sakellaridis_2008f, Sultan_2024f, Tammana_2012j, Tanrikulu_2015e, Trgo_2012f, Wadhwa_2015e, Wnęk_2015f, teWildt_2010}, 7 cases (10\%) were managed conservatively \cite{Ataya_2013, Bhattacharjee_2008, DivsalarP._2023a, Emamhadi_2018, Goldman_1998f, Kar_2015, Kumar_2001}, 2 cases (3\%) died \cite{Emamhadi_2018, Kumar_2001}. All 90 were male gender. 90 cases (100\%) were detained at the time of ingestion \cite{Elghali_2016, Karp_1991b, Lee_2007}, 88 cases (98\%) were intentional ingestions \cite{Elghali_2016, Karp_1991b, Lee_2007}, 30 cases (33\%) had a psychiatric history documented \cite{Elghali_2016, Karp_1991b, Lee_2007}, 2 cases (2\%) had a history of prior ingestion \cite{Elghali_2016}. No cases were reported for were psychiatric inpatients, were displaced people, were under the influence of alcohol at the time of ingestion, and had a severe disability history.
\paragraph*{Motivation}  70 cases (78\%) reported protest motivation \cite{Elghali_2016, Karp_1991b, Lee_2007}, 12 cases (13\%) reported psychiatric motivation \cite{Karp_1991b}, 6 cases (7\%) reported self-harm motivation \cite{Elghali_2016, Karp_1991b}. No cases were reported for psychosocial motivation and other motivation.
\paragraph*{Object Characteristics}  68 cases (76\%) involved sharp object ingestion \cite{Elghali_2016, Karp_1991b, Lee_2007}, 32 cases (36\%) involved long (\textgreater 5cm) object ingestion \cite{Lee_2007}, 25 cases (28\%) involved ingestion of multiple objects \cite{Elghali_2016, Lee_2007}. No cases were reported for button battery ingestion, magnet ingestion, and involved large diameter (\textgreater 2.5cm) object ingestion.
\paragraph*{Outcomes}  47 cases (52\%) underwent endoscopic intervention \cite{Elghali_2016, Lee_2007}, 29 cases (32\%) were managed conservatively \cite{Elghali_2016, Karp_1991b}, 15 cases (17\%) underwent surgical intervention \cite{Elghali_2016, Karp_1991b, Lee_2007}, 6 cases (7\%) reported complications \cite{Lee_2007}, 1 case (1\%) died \cite{Elghali_2016}.
\paragraph*{Geographical Location}Cases were recorded in 33 countries: 13 cases from USA \cite{Alao_2006i, Ataya_2013, Bhumi_2024f, Fry_2010, Guinan_2019f, Hardy_2023g, Jehangir_2019h, Kerestes_2019, Kumar_2001, Liu_2005, Tammana_2012j, Tay_2004, Tupesis_2004f}; 7 cases from India \cite{Bhasin_2014, Bhattacharjee_2008, Kar_2015, Kariholu_2008, Kumar_2019f, Misra_2013, Wadhwa_2015e} and UK \cite{Beecroft_1998, Berry_2021e, Cauchi_2002, Cox_2007, Gardner_2017h, Qureshi_2016}; 6 cases from Bulgaria \cite{Losanoff_1996, Losanoff_1997e}; 5 cases from Iran \cite{DivsalarP._2023a, Emamhadi_2018, Farhadi_2024h}; 4 cases from Turkey \cite{Akay_2015f, Atayan_2016, Tanrikulu_2015e, Yildiz_2016e}; 2 cases from China \cite{Jin_2023, Li_2013}, Poland \cite{Kobiela_2015, Wnęk_2015f}, and Spain \cite{CamachoDorado_2018, fjbuilsRepeatedBehaviorDeliberate2024}; 1 case from Australia \cite{Apikotoa_2022f}, Bahrain \cite{Ali_2020f}, Croatia \cite{Trgo_2012f}, Ecuador \cite{DelgadoSalazar_2020c}, Egypt \cite{Ali_2022g}, Ethiopia \cite{Mesfin_2022a}, Germany \cite{teWildt_2010}, Greece \cite{Sakellaridis_2008f}, Hungary \cite{Csaky_1998e}, Iraq \cite{Al-Faham_2020k}, Israel \cite{Goldman_1998f}, Italy \cite{Riva_2018j}, Japan \cite{Ohno_2005}, Nepal \cite{Thapa_2019f}, Netherlands \cite{Benoist_2019e}, Oman \cite{AlShaaibi_2021b}, Pakistan \cite{Yasin_2009}, Portugal \cite{Peixoto_2017f}, Qatar \cite{Ali_2017}, Saudi Arabia \cite{Sultan_2024f}, South Africa \cite{Sobnach_2011f}, Sweden \cite{Naji_2012f}, Switzerland \cite{Wildhaber_2005}, and Taiwan \cite{Chang_2017f}. \paragraph*{Gender} 43 cases (60\%) were male \cite{Akay_2015f, Al-Faham_2020k, Alao_2006i, Ali_2017, Ali_2022g, Apikotoa_2022f, Atayan_2016, Benoist_2019e, Berry_2021e, Bhumi_2024f, CamachoDorado_2018, Csaky_1998e, Emamhadi_2018, Farhadi_2024h, Fry_2010, Gardner_2017h, Guinan_2019f, Jehangir_2019h, Jin_2023, Kobiela_2015, Kumar_2001, Kumar_2019f, Liu_2005, Losanoff_1996, Losanoff_1997e, Mesfin_2022a, Misra_2013, Qureshi_2016, Riva_2018j, Sobnach_2011f, Tammana_2012j, Tanrikulu_2015e, Tay_2004, Thapa_2019f, Trgo_2012f, Wadhwa_2015e, Yasin_2009, teWildt_2010}, 28 cases (39\%) were female \cite{AlShaaibi_2021b, Ali_2020f, Ataya_2013, Beecroft_1998, Bhasin_2014, Bhattacharjee_2008, Cauchi_2002, Chang_2017f, Cox_2007, DelgadoSalazar_2020c, DivsalarP._2023a, Goldman_1998f, Hardy_2023g, Kar_2015, Kariholu_2008, Kerestes_2019, Li_2013, Naji_2012f, Ohno_2005, Peixoto_2017f, Sakellaridis_2008f, Sultan_2024f, Tupesis_2004f, Wildhaber_2005, Wnęk_2015f, Yildiz_2016e}, 1 case (1\%) had no gender recorded \cite{fjbuilsRepeatedBehaviorDeliberate2024}. \paragraph*{Age Group} 25 cases (35\%) were between 26 and 40 years of age \cite{Alao_2006i, Ali_2022g, Apikotoa_2022f, Ataya_2013, Benoist_2019e, Bhasin_2014, Chang_2017f, Cox_2007, DelgadoSalazar_2020c, Farhadi_2024h, Fry_2010, Gardner_2017h, Guinan_2019f, Jin_2023, Kumar_2019f, Losanoff_1996, Misra_2013, Qureshi_2016, Riva_2018j, Sakellaridis_2008f, Tammana_2012j, Trgo_2012f, Wnęk_2015f, Yildiz_2016e, fjbuilsRepeatedBehaviorDeliberate2024}, 18 cases (25\%) were between 18 and 25 years of age \cite{Akay_2015f, Ali_2017, Atayan_2016, Bhattacharjee_2008, Csaky_1998e, Kar_2015, Kariholu_2008, Kobiela_2015, Losanoff_1996, Losanoff_1997e, Mesfin_2022a, Peixoto_2017f, Sobnach_2011f, Tupesis_2004f, Yasin_2009}, 13 cases (18\%) were under 18 years of age \cite{AlShaaibi_2021b, Ali_2020f, Cauchi_2002, DivsalarP._2023a, Goldman_1998f, Liu_2005, Naji_2012f, Ohno_2005, Tanrikulu_2015e, Tay_2004, Wildhaber_2005}, 11 cases (15\%) were between 41 and 60 years of age \cite{Al-Faham_2020k, Bhumi_2024f, CamachoDorado_2018, Emamhadi_2018, Hardy_2023g, Jehangir_2019h, Kumar_2001, Sultan_2024f, Thapa_2019f, Wadhwa_2015e, teWildt_2010}, 3 cases (4\%) were over 60 years of age \cite{Beecroft_1998, Kerestes_2019, Li_2013}, 2 cases (3\%) had no age documented \cite{Berry_2021e}. \paragraph*{Population} 36 cases (50\%) had a psychiatric history \cite{AlShaaibi_2021b, Alao_2006i, Ali_2020f, Apikotoa_2022f, Ataya_2013, Atayan_2016, Beecroft_1998, CamachoDorado_2018, Chang_2017f, DelgadoSalazar_2020c, DivsalarP._2023a, Farhadi_2024h, Fry_2010, Guinan_2019f, Hardy_2023g, Jehangir_2019h, Jin_2023, Kar_2015, Kerestes_2019, Kobiela_2015, Kumar_2001, Kumar_2019f, Liu_2005, Mesfin_2022a, Misra_2013, Ohno_2005, Peixoto_2017f, Sakellaridis_2008f, Sultan_2024f, Tammana_2012j, Tanrikulu_2015e, Yildiz_2016e, fjbuilsRepeatedBehaviorDeliberate2024, teWildt_2010}, 19 cases (26\%) had ingested previously \cite{Alao_2006i, Apikotoa_2022f, Berry_2021e, Bhattacharjee_2008, Csaky_1998e, DivsalarP._2023a, Emamhadi_2018, Guinan_2019f, Jehangir_2019h, Jin_2023, Liu_2005, Sakellaridis_2008f, Tanrikulu_2015e, Thapa_2019f, Yildiz_2016e, fjbuilsRepeatedBehaviorDeliberate2024, teWildt_2010}, 12 cases (17\%) were detained persons \cite{Alao_2006i, Ali_2022g, Apikotoa_2022f, Losanoff_1996, Losanoff_1997e, Qureshi_2016, Tammana_2012j, Trgo_2012f}, 7 cases (10\%) were severely disabled \cite{Atayan_2016, Kerestes_2019, Liu_2005, Ohno_2005, Peixoto_2017f, Yildiz_2016e, teWildt_2010}, 4 cases (6\%) were psychiatric inpatients \cite{DivsalarP._2023a, fjbuilsRepeatedBehaviorDeliberate2024, teWildt_2010}, 3 cases (4\%) were under the influence of alcohol \cite{Benoist_2019e, Csaky_1998e, Thapa_2019f}, 2 cases (3\%) were displaced people \cite{Akay_2015f, Gardner_2017h}. \paragraph*{Motivation} 34 cases (47\%) had a psychiatric motivation \cite{Al-Faham_2020k, Alao_2006i, Ali_2020f, Apikotoa_2022f, Ataya_2013, Atayan_2016, Bhasin_2014, Bhattacharjee_2008, DelgadoSalazar_2020c, DivsalarP._2023a, Emamhadi_2018, Farhadi_2024h, Guinan_2019f, Hardy_2023g, Jehangir_2019h, Jin_2023, Kar_2015, Kariholu_2008, Kerestes_2019, Kobiela_2015, Kumar_2001, Kumar_2019f, Li_2013, Liu_2005, Misra_2013, Ohno_2005, Sakellaridis_2008f, Sultan_2024f, Tammana_2012j, Tanrikulu_2015e, Yasin_2009, teWildt_2010}, 21 cases (29\%) were motivated by self-harm intention \cite{Al-Faham_2020k, AlShaaibi_2021b, Alao_2006i, Ali_2017, CamachoDorado_2018, Chang_2017f, Cox_2007, Csaky_1998e, Fry_2010, Li_2013, Losanoff_1996, Losanoff_1997e, Mesfin_2022a, Sakellaridis_2008f, Tammana_2012j, Tanrikulu_2015e, fjbuilsRepeatedBehaviorDeliberate2024}, 17 cases (24\%) had a psychosocial motivation \cite{Akay_2015f, Benoist_2019e, Bhattacharjee_2008, Cauchi_2002, Goldman_1998f, Hardy_2023g, Kobiela_2015, Li_2013, Naji_2012f, Qureshi_2016, Riva_2018j, Sobnach_2011f, Tay_2004, Thapa_2019f, Tupesis_2004f, Wildhaber_2005, Wnęk_2015f}, 9 cases (12\%) were motivated by protest \cite{Bhumi_2024f, Gardner_2017h, Losanoff_1996, Losanoff_1997e, Tupesis_2004f}, 9 cases (12\%) had another documented motivation \cite{Ali_2020f, Ali_2022g, Emamhadi_2018, Guinan_2019f, Peixoto_2017f, Sakellaridis_2008f, Trgo_2012f, Wadhwa_2015e, Yildiz_2016e}. \paragraph*{Object Characteristics} 51 cases (71\%) ingested a large diameter object (\textgreater{}2.5cm) \cite{Akay_2015f, Al-Faham_2020k, AlShaaibi_2021b, Alao_2006i, Ali_2017, Ali_2022g, Apikotoa_2022f, Atayan_2016, Berry_2021e, Bhasin_2014, CamachoDorado_2018, Cauchi_2002, Chang_2017f, Cox_2007, Csaky_1998e, DivsalarP._2023a, Emamhadi_2018, Gardner_2017h, Guinan_2019f, Jehangir_2019h, Jin_2023, Kariholu_2008, Kerestes_2019, Kobiela_2015, Kumar_2001, Kumar_2019f, Losanoff_1996, Losanoff_1997e, Mesfin_2022a, Misra_2013, Naji_2012f, Ohno_2005, Peixoto_2017f, Qureshi_2016, Riva_2018j, Sakellaridis_2008f, Sultan_2024f, Tanrikulu_2015e, Thapa_2019f, Trgo_2012f, Wnęk_2015f, Yildiz_2016e, fjbuilsRepeatedBehaviorDeliberate2024, teWildt_2010}, 44 cases (61\%) ingested multiple objects \cite{Ali_2020f, Apikotoa_2022f, Ataya_2013, Atayan_2016, Beecroft_1998, Bhattacharjee_2008, Bhumi_2024f, CamachoDorado_2018, Cauchi_2002, Emamhadi_2018, Farhadi_2024h, Fry_2010, Goldman_1998f, Guinan_2019f, Hardy_2023g, Jehangir_2019h, Jin_2023, Kar_2015, Kariholu_2008, Kobiela_2015, Kumar_2001, Kumar_2019f, Li_2013, Liu_2005, Losanoff_1996, Mesfin_2022a, Misra_2013, Naji_2012f, Ohno_2005, Sobnach_2011f, Sultan_2024f, Tammana_2012j, Tanrikulu_2015e, Tay_2004, Thapa_2019f, Wadhwa_2015e, Wildhaber_2005, Yasin_2009, fjbuilsRepeatedBehaviorDeliberate2024, teWildt_2010}, 34 cases (47\%) ingested a sharp object \cite{AlShaaibi_2021b, Alao_2006i, Apikotoa_2022f, Ataya_2013, Benoist_2019e, Bhasin_2014, Bhattacharjee_2008, CamachoDorado_2018, Csaky_1998e, DelgadoSalazar_2020c, DivsalarP._2023a, Emamhadi_2018, Farhadi_2024h, Fry_2010, Guinan_2019f, Hardy_2023g, Jehangir_2019h, Jin_2023, Kariholu_2008, Kobiela_2015, Kumar_2019f, Losanoff_1996, Losanoff_1997e, Mesfin_2022a, Misra_2013, Sobnach_2011f, Yasin_2009, teWildt_2010}, 32 cases (44\%) ingested a long object (\textgreater{}5cm) \cite{Al-Faham_2020k, AlShaaibi_2021b, Ali_2017, Ali_2022g, Atayan_2016, Bhasin_2014, CamachoDorado_2018, Chang_2017f, Cox_2007, Csaky_1998e, DivsalarP._2023a, Emamhadi_2018, Fry_2010, Gardner_2017h, Jin_2023, Kariholu_2008, Kerestes_2019, Kobiela_2015, Kumar_2019f, Mesfin_2022a, Misra_2013, Ohno_2005, Qureshi_2016, Sakellaridis_2008f, Sultan_2024f, Thapa_2019f, Trgo_2012f, Yasin_2009, Yildiz_2016e, teWildt_2010}, 9 cases (12\%) ingested a magnet \cite{Ali_2020f, Bhumi_2024f, Cauchi_2002, Liu_2005, Naji_2012f, Ohno_2005, Tanrikulu_2015e, Tay_2004, Wildhaber_2005}, 2 cases (3\%) ingested a button battery \cite{Berry_2021e, Bhumi_2024f}. \paragraph*{Outcomes} 48 cases (67\%) experienced a complication \cite{Ali_2017, Ali_2020f, Apikotoa_2022f, Atayan_2016, Beecroft_1998, Benoist_2019e, Berry_2021e, Bhasin_2014, Bhumi_2024f, CamachoDorado_2018, Cauchi_2002, Cox_2007, Csaky_1998e, DelgadoSalazar_2020c, DivsalarP._2023a, Emamhadi_2018, Farhadi_2024h, Fry_2010, Gardner_2017h, Goldman_1998f, Jin_2023, Kariholu_2008, Kerestes_2019, Kobiela_2015, Kumar_2001, Kumar_2019f, Liu_2005, Losanoff_1996, Mesfin_2022a, Misra_2013, Naji_2012f, Ohno_2005, Sakellaridis_2008f, Sobnach_2011f, Sultan_2024f, Tanrikulu_2015e, Tay_2004, Thapa_2019f, Trgo_2012f, Tupesis_2004f, Wildhaber_2005, Wnęk_2015f, Yasin_2009, Yildiz_2016e}, 44 cases (61\%) underwent surgery \cite{Al-Faham_2020k, AlShaaibi_2021b, Alao_2006i, Ali_2017, Ali_2020f, Atayan_2016, Beecroft_1998, Bhasin_2014, CamachoDorado_2018, Cauchi_2002, Chang_2017f, Cox_2007, Csaky_1998e, DelgadoSalazar_2020c, DivsalarP._2023a, Farhadi_2024h, Fry_2010, Gardner_2017h, Jin_2023, Kariholu_2008, Kerestes_2019, Kobiela_2015, Kumar_2019f, Liu_2005, Losanoff_1996, Losanoff_1997e, Mesfin_2022a, Misra_2013, Naji_2012f, Sobnach_2011f, Tanrikulu_2015e, Tay_2004, Thapa_2019f, Tupesis_2004f, Wildhaber_2005, Wnęk_2015f, Yasin_2009, Yildiz_2016e, fjbuilsRepeatedBehaviorDeliberate2024}, 31 cases (43\%) underwent endoscopy \cite{Akay_2015f, Ali_2022g, Apikotoa_2022f, Atayan_2016, Benoist_2019e, Berry_2021e, Bhasin_2014, Bhumi_2024f, CamachoDorado_2018, Chang_2017f, DelgadoSalazar_2020c, Gardner_2017h, Guinan_2019f, Hardy_2023g, Jehangir_2019h, Kariholu_2008, Li_2013, Liu_2005, Ohno_2005, Peixoto_2017f, Qureshi_2016, Riva_2018j, Sakellaridis_2008f, Sultan_2024f, Tammana_2012j, Tanrikulu_2015e, Trgo_2012f, Wadhwa_2015e, Wnęk_2015f, teWildt_2010}, 7 cases (10\%) were managed conservatively \cite{Ataya_2013, Bhattacharjee_2008, DivsalarP._2023a, Emamhadi_2018, Goldman_1998f, Kar_2015, Kumar_2001}, 2 cases (3\%) died \cite{Emamhadi_2018, Kumar_2001}. All 90 were male gender. 90 cases (100\%) were detained at the time of ingestion \cite{Elghali_2016, Karp_1991b, Lee_2007}, 88 cases (98\%) were intentional ingestions \cite{Elghali_2016, Karp_1991b, Lee_2007}, 30 cases (33\%) had a psychiatric history documented \cite{Elghali_2016, Karp_1991b, Lee_2007}, 2 cases (2\%) had a history of prior ingestion \cite{Elghali_2016}. No cases were reported for were psychiatric inpatients, were displaced people, were under the influence of alcohol at the time of ingestion, and had a severe disability history.
\paragraph*{Motivation}  70 cases (78\%) reported protest motivation \cite{Elghali_2016, Karp_1991b, Lee_2007}, 12 cases (13\%) reported psychiatric motivation \cite{Karp_1991b}, 6 cases (7\%) reported self-harm motivation \cite{Elghali_2016, Karp_1991b}. No cases were reported for psychosocial motivation and other motivation.
\paragraph*{Object Characteristics}  68 cases (76\%) involved sharp object ingestion \cite{Elghali_2016, Karp_1991b, Lee_2007}, 32 cases (36\%) involved long (\textgreater 5cm) object ingestion \cite{Lee_2007}, 25 cases (28\%) involved ingestion of multiple objects \cite{Elghali_2016, Lee_2007}. No cases were reported for button battery ingestion, magnet ingestion, and involved large diameter (\textgreater 2.5cm) object ingestion.
\paragraph*{Outcomes}  47 cases (52\%) underwent endoscopic intervention \cite{Elghali_2016, Lee_2007}, 29 cases (32\%) were managed conservatively \cite{Elghali_2016, Karp_1991b}, 15 cases (17\%) underwent surgical intervention \cite{Elghali_2016, Karp_1991b, Lee_2007}, 6 cases (7\%) reported complications \cite{Lee_2007}, 1 case (1\%) died \cite{Elghali_2016}.
\paragraph*{Geographical Location}Cases were recorded in 33 countries: 13 cases from USA \cite{Alao_2006i, Ataya_2013, Bhumi_2024f, Fry_2010, Guinan_2019f, Hardy_2023g, Jehangir_2019h, Kerestes_2019, Kumar_2001, Liu_2005, Tammana_2012j, Tay_2004, Tupesis_2004f}; 7 cases from India \cite{Bhasin_2014, Bhattacharjee_2008, Kar_2015, Kariholu_2008, Kumar_2019f, Misra_2013, Wadhwa_2015e} and UK \cite{Beecroft_1998, Berry_2021e, Cauchi_2002, Cox_2007, Gardner_2017h, Qureshi_2016}; 6 cases from Bulgaria \cite{Losanoff_1996, Losanoff_1997e}; 5 cases from Iran \cite{DivsalarP._2023a, Emamhadi_2018, Farhadi_2024h}; 4 cases from Turkey \cite{Akay_2015f, Atayan_2016, Tanrikulu_2015e, Yildiz_2016e}; 2 cases from China \cite{Jin_2023, Li_2013}, Poland \cite{Kobiela_2015, Wnęk_2015f}, and Spain \cite{CamachoDorado_2018, fjbuilsRepeatedBehaviorDeliberate2024}; 1 case from Australia \cite{Apikotoa_2022f}, Bahrain \cite{Ali_2020f}, Croatia \cite{Trgo_2012f}, Ecuador \cite{DelgadoSalazar_2020c}, Egypt \cite{Ali_2022g}, Ethiopia \cite{Mesfin_2022a}, Germany \cite{teWildt_2010}, Greece \cite{Sakellaridis_2008f}, Hungary \cite{Csaky_1998e}, Iraq \cite{Al-Faham_2020k}, Israel \cite{Goldman_1998f}, Italy \cite{Riva_2018j}, Japan \cite{Ohno_2005}, Nepal \cite{Thapa_2019f}, Netherlands \cite{Benoist_2019e}, Oman \cite{AlShaaibi_2021b}, Pakistan \cite{Yasin_2009}, Portugal \cite{Peixoto_2017f}, Qatar \cite{Ali_2017}, Saudi Arabia \cite{Sultan_2024f}, South Africa \cite{Sobnach_2011f}, Sweden \cite{Naji_2012f}, Switzerland \cite{Wildhaber_2005}, and Taiwan \cite{Chang_2017f}. \paragraph*{Gender} 43 cases (60\%) were male \cite{Akay_2015f, Al-Faham_2020k, Alao_2006i, Ali_2017, Ali_2022g, Apikotoa_2022f, Atayan_2016, Benoist_2019e, Berry_2021e, Bhumi_2024f, CamachoDorado_2018, Csaky_1998e, Emamhadi_2018, Farhadi_2024h, Fry_2010, Gardner_2017h, Guinan_2019f, Jehangir_2019h, Jin_2023, Kobiela_2015, Kumar_2001, Kumar_2019f, Liu_2005, Losanoff_1996, Losanoff_1997e, Mesfin_2022a, Misra_2013, Qureshi_2016, Riva_2018j, Sobnach_2011f, Tammana_2012j, Tanrikulu_2015e, Tay_2004, Thapa_2019f, Trgo_2012f, Wadhwa_2015e, Yasin_2009, teWildt_2010}, 28 cases (39\%) were female \cite{AlShaaibi_2021b, Ali_2020f, Ataya_2013, Beecroft_1998, Bhasin_2014, Bhattacharjee_2008, Cauchi_2002, Chang_2017f, Cox_2007, DelgadoSalazar_2020c, DivsalarP._2023a, Goldman_1998f, Hardy_2023g, Kar_2015, Kariholu_2008, Kerestes_2019, Li_2013, Naji_2012f, Ohno_2005, Peixoto_2017f, Sakellaridis_2008f, Sultan_2024f, Tupesis_2004f, Wildhaber_2005, Wnęk_2015f, Yildiz_2016e}, 1 case (1\%) had no gender recorded \cite{fjbuilsRepeatedBehaviorDeliberate2024}. \paragraph*{Age Group} 25 cases (35\%) were between 26 and 40 years of age \cite{Alao_2006i, Ali_2022g, Apikotoa_2022f, Ataya_2013, Benoist_2019e, Bhasin_2014, Chang_2017f, Cox_2007, DelgadoSalazar_2020c, Farhadi_2024h, Fry_2010, Gardner_2017h, Guinan_2019f, Jin_2023, Kumar_2019f, Losanoff_1996, Misra_2013, Qureshi_2016, Riva_2018j, Sakellaridis_2008f, Tammana_2012j, Trgo_2012f, Wnęk_2015f, Yildiz_2016e, fjbuilsRepeatedBehaviorDeliberate2024}, 18 cases (25\%) were between 18 and 25 years of age \cite{Akay_2015f, Ali_2017, Atayan_2016, Bhattacharjee_2008, Csaky_1998e, Kar_2015, Kariholu_2008, Kobiela_2015, Losanoff_1996, Losanoff_1997e, Mesfin_2022a, Peixoto_2017f, Sobnach_2011f, Tupesis_2004f, Yasin_2009}, 13 cases (18\%) were under 18 years of age \cite{AlShaaibi_2021b, Ali_2020f, Cauchi_2002, DivsalarP._2023a, Goldman_1998f, Liu_2005, Naji_2012f, Ohno_2005, Tanrikulu_2015e, Tay_2004, Wildhaber_2005}, 11 cases (15\%) were between 41 and 60 years of age \cite{Al-Faham_2020k, Bhumi_2024f, CamachoDorado_2018, Emamhadi_2018, Hardy_2023g, Jehangir_2019h, Kumar_2001, Sultan_2024f, Thapa_2019f, Wadhwa_2015e, teWildt_2010}, 3 cases (4\%) were over 60 years of age \cite{Beecroft_1998, Kerestes_2019, Li_2013}, 2 cases (3\%) had no age documented \cite{Berry_2021e}. \paragraph*{Population} 36 cases (50\%) had a psychiatric history \cite{AlShaaibi_2021b, Alao_2006i, Ali_2020f, Apikotoa_2022f, Ataya_2013, Atayan_2016, Beecroft_1998, CamachoDorado_2018, Chang_2017f, DelgadoSalazar_2020c, DivsalarP._2023a, Farhadi_2024h, Fry_2010, Guinan_2019f, Hardy_2023g, Jehangir_2019h, Jin_2023, Kar_2015, Kerestes_2019, Kobiela_2015, Kumar_2001, Kumar_2019f, Liu_2005, Mesfin_2022a, Misra_2013, Ohno_2005, Peixoto_2017f, Sakellaridis_2008f, Sultan_2024f, Tammana_2012j, Tanrikulu_2015e, Yildiz_2016e, fjbuilsRepeatedBehaviorDeliberate2024, teWildt_2010}, 19 cases (26\%) had ingested previously \cite{Alao_2006i, Apikotoa_2022f, Berry_2021e, Bhattacharjee_2008, Csaky_1998e, DivsalarP._2023a, Emamhadi_2018, Guinan_2019f, Jehangir_2019h, Jin_2023, Liu_2005, Sakellaridis_2008f, Tanrikulu_2015e, Thapa_2019f, Yildiz_2016e, fjbuilsRepeatedBehaviorDeliberate2024, teWildt_2010}, 12 cases (17\%) were detained persons \cite{Alao_2006i, Ali_2022g, Apikotoa_2022f, Losanoff_1996, Losanoff_1997e, Qureshi_2016, Tammana_2012j, Trgo_2012f}, 7 cases (10\%) were severely disabled \cite{Atayan_2016, Kerestes_2019, Liu_2005, Ohno_2005, Peixoto_2017f, Yildiz_2016e, teWildt_2010}, 4 cases (6\%) were psychiatric inpatients \cite{DivsalarP._2023a, fjbuilsRepeatedBehaviorDeliberate2024, teWildt_2010}, 3 cases (4\%) were under the influence of alcohol \cite{Benoist_2019e, Csaky_1998e, Thapa_2019f}, 2 cases (3\%) were displaced people \cite{Akay_2015f, Gardner_2017h}. \paragraph*{Motivation} 34 cases (47\%) had a psychiatric motivation \cite{Al-Faham_2020k, Alao_2006i, Ali_2020f, Apikotoa_2022f, Ataya_2013, Atayan_2016, Bhasin_2014, Bhattacharjee_2008, DelgadoSalazar_2020c, DivsalarP._2023a, Emamhadi_2018, Farhadi_2024h, Guinan_2019f, Hardy_2023g, Jehangir_2019h, Jin_2023, Kar_2015, Kariholu_2008, Kerestes_2019, Kobiela_2015, Kumar_2001, Kumar_2019f, Li_2013, Liu_2005, Misra_2013, Ohno_2005, Sakellaridis_2008f, Sultan_2024f, Tammana_2012j, Tanrikulu_2015e, Yasin_2009, teWildt_2010}, 21 cases (29\%) were motivated by self-harm intention \cite{Al-Faham_2020k, AlShaaibi_2021b, Alao_2006i, Ali_2017, CamachoDorado_2018, Chang_2017f, Cox_2007, Csaky_1998e, Fry_2010, Li_2013, Losanoff_1996, Losanoff_1997e, Mesfin_2022a, Sakellaridis_2008f, Tammana_2012j, Tanrikulu_2015e, fjbuilsRepeatedBehaviorDeliberate2024}, 17 cases (24\%) had a psychosocial motivation \cite{Akay_2015f, Benoist_2019e, Bhattacharjee_2008, Cauchi_2002, Goldman_1998f, Hardy_2023g, Kobiela_2015, Li_2013, Naji_2012f, Qureshi_2016, Riva_2018j, Sobnach_2011f, Tay_2004, Thapa_2019f, Tupesis_2004f, Wildhaber_2005, Wnęk_2015f}, 9 cases (12\%) were motivated by protest \cite{Bhumi_2024f, Gardner_2017h, Losanoff_1996, Losanoff_1997e, Tupesis_2004f}, 9 cases (12\%) had another documented motivation \cite{Ali_2020f, Ali_2022g, Emamhadi_2018, Guinan_2019f, Peixoto_2017f, Sakellaridis_2008f, Trgo_2012f, Wadhwa_2015e, Yildiz_2016e}. \paragraph*{Object Characteristics} 51 cases (71\%) ingested a large diameter object (\textgreater{}2.5cm) \cite{Akay_2015f, Al-Faham_2020k, AlShaaibi_2021b, Alao_2006i, Ali_2017, Ali_2022g, Apikotoa_2022f, Atayan_2016, Berry_2021e, Bhasin_2014, CamachoDorado_2018, Cauchi_2002, Chang_2017f, Cox_2007, Csaky_1998e, DivsalarP._2023a, Emamhadi_2018, Gardner_2017h, Guinan_2019f, Jehangir_2019h, Jin_2023, Kariholu_2008, Kerestes_2019, Kobiela_2015, Kumar_2001, Kumar_2019f, Losanoff_1996, Losanoff_1997e, Mesfin_2022a, Misra_2013, Naji_2012f, Ohno_2005, Peixoto_2017f, Qureshi_2016, Riva_2018j, Sakellaridis_2008f, Sultan_2024f, Tanrikulu_2015e, Thapa_2019f, Trgo_2012f, Wnęk_2015f, Yildiz_2016e, fjbuilsRepeatedBehaviorDeliberate2024, teWildt_2010}, 44 cases (61\%) ingested multiple objects \cite{Ali_2020f, Apikotoa_2022f, Ataya_2013, Atayan_2016, Beecroft_1998, Bhattacharjee_2008, Bhumi_2024f, CamachoDorado_2018, Cauchi_2002, Emamhadi_2018, Farhadi_2024h, Fry_2010, Goldman_1998f, Guinan_2019f, Hardy_2023g, Jehangir_2019h, Jin_2023, Kar_2015, Kariholu_2008, Kobiela_2015, Kumar_2001, Kumar_2019f, Li_2013, Liu_2005, Losanoff_1996, Mesfin_2022a, Misra_2013, Naji_2012f, Ohno_2005, Sobnach_2011f, Sultan_2024f, Tammana_2012j, Tanrikulu_2015e, Tay_2004, Thapa_2019f, Wadhwa_2015e, Wildhaber_2005, Yasin_2009, fjbuilsRepeatedBehaviorDeliberate2024, teWildt_2010}, 34 cases (47\%) ingested a sharp object \cite{AlShaaibi_2021b, Alao_2006i, Apikotoa_2022f, Ataya_2013, Benoist_2019e, Bhasin_2014, Bhattacharjee_2008, CamachoDorado_2018, Csaky_1998e, DelgadoSalazar_2020c, DivsalarP._2023a, Emamhadi_2018, Farhadi_2024h, Fry_2010, Guinan_2019f, Hardy_2023g, Jehangir_2019h, Jin_2023, Kariholu_2008, Kobiela_2015, Kumar_2019f, Losanoff_1996, Losanoff_1997e, Mesfin_2022a, Misra_2013, Sobnach_2011f, Yasin_2009, teWildt_2010}, 32 cases (44\%) ingested a long object (\textgreater{}5cm) \cite{Al-Faham_2020k, AlShaaibi_2021b, Ali_2017, Ali_2022g, Atayan_2016, Bhasin_2014, CamachoDorado_2018, Chang_2017f, Cox_2007, Csaky_1998e, DivsalarP._2023a, Emamhadi_2018, Fry_2010, Gardner_2017h, Jin_2023, Kariholu_2008, Kerestes_2019, Kobiela_2015, Kumar_2019f, Mesfin_2022a, Misra_2013, Ohno_2005, Qureshi_2016, Sakellaridis_2008f, Sultan_2024f, Thapa_2019f, Trgo_2012f, Yasin_2009, Yildiz_2016e, teWildt_2010}, 9 cases (12\%) ingested a magnet \cite{Ali_2020f, Bhumi_2024f, Cauchi_2002, Liu_2005, Naji_2012f, Ohno_2005, Tanrikulu_2015e, Tay_2004, Wildhaber_2005}, 2 cases (3\%) ingested a button battery \cite{Berry_2021e, Bhumi_2024f}. \paragraph*{Outcomes} 48 cases (67\%) experienced a complication \cite{Ali_2017, Ali_2020f, Apikotoa_2022f, Atayan_2016, Beecroft_1998, Benoist_2019e, Berry_2021e, Bhasin_2014, Bhumi_2024f, CamachoDorado_2018, Cauchi_2002, Cox_2007, Csaky_1998e, DelgadoSalazar_2020c, DivsalarP._2023a, Emamhadi_2018, Farhadi_2024h, Fry_2010, Gardner_2017h, Goldman_1998f, Jin_2023, Kariholu_2008, Kerestes_2019, Kobiela_2015, Kumar_2001, Kumar_2019f, Liu_2005, Losanoff_1996, Mesfin_2022a, Misra_2013, Naji_2012f, Ohno_2005, Sakellaridis_2008f, Sobnach_2011f, Sultan_2024f, Tanrikulu_2015e, Tay_2004, Thapa_2019f, Trgo_2012f, Tupesis_2004f, Wildhaber_2005, Wnęk_2015f, Yasin_2009, Yildiz_2016e}, 44 cases (61\%) underwent surgery \cite{Al-Faham_2020k, AlShaaibi_2021b, Alao_2006i, Ali_2017, Ali_2020f, Atayan_2016, Beecroft_1998, Bhasin_2014, CamachoDorado_2018, Cauchi_2002, Chang_2017f, Cox_2007, Csaky_1998e, DelgadoSalazar_2020c, DivsalarP._2023a, Farhadi_2024h, Fry_2010, Gardner_2017h, Jin_2023, Kariholu_2008, Kerestes_2019, Kobiela_2015, Kumar_2019f, Liu_2005, Losanoff_1996, Losanoff_1997e, Mesfin_2022a, Misra_2013, Naji_2012f, Sobnach_2011f, Tanrikulu_2015e, Tay_2004, Thapa_2019f, Tupesis_2004f, Wildhaber_2005, Wnęk_2015f, Yasin_2009, Yildiz_2016e, fjbuilsRepeatedBehaviorDeliberate2024}, 31 cases (43\%) underwent endoscopy \cite{Akay_2015f, Ali_2022g, Apikotoa_2022f, Atayan_2016, Benoist_2019e, Berry_2021e, Bhasin_2014, Bhumi_2024f, CamachoDorado_2018, Chang_2017f, DelgadoSalazar_2020c, Gardner_2017h, Guinan_2019f, Hardy_2023g, Jehangir_2019h, Kariholu_2008, Li_2013, Liu_2005, Ohno_2005, Peixoto_2017f, Qureshi_2016, Riva_2018j, Sakellaridis_2008f, Sultan_2024f, Tammana_2012j, Tanrikulu_2015e, Trgo_2012f, Wadhwa_2015e, Wnęk_2015f, teWildt_2010}, 7 cases (10\%) were managed conservatively \cite{Ataya_2013, Bhattacharjee_2008, DivsalarP._2023a, Emamhadi_2018, Goldman_1998f, Kar_2015, Kumar_2001}, 2 cases (3\%) died \cite{Emamhadi_2018, Kumar_2001}. All 90 were male gender. 90 cases (100\%) were detained at the time of ingestion \cite{Elghali_2016, Karp_1991b, Lee_2007}, 88 cases (98\%) were intentional ingestions \cite{Elghali_2016, Karp_1991b, Lee_2007}, 30 cases (33\%) had a psychiatric history documented \cite{Elghali_2016, Karp_1991b, Lee_2007}, 2 cases (2\%) had a history of prior ingestion \cite{Elghali_2016}. No cases were reported for were psychiatric inpatients, were displaced people, were under the influence of alcohol at the time of ingestion, and had a severe disability history.
\paragraph*{Motivation}  70 cases (78\%) reported protest motivation \cite{Elghali_2016, Karp_1991b, Lee_2007}, 12 cases (13\%) reported psychiatric motivation \cite{Karp_1991b}, 6 cases (7\%) reported self-harm motivation \cite{Elghali_2016, Karp_1991b}. No cases were reported for psychosocial motivation and other motivation.
\paragraph*{Object Characteristics}  68 cases (76\%) involved sharp object ingestion \cite{Elghali_2016, Karp_1991b, Lee_2007}, 32 cases (36\%) involved long (\textgreater 5cm) object ingestion \cite{Lee_2007}, 25 cases (28\%) involved ingestion of multiple objects \cite{Elghali_2016, Lee_2007}. No cases were reported for button battery ingestion, magnet ingestion, and involved large diameter (\textgreater 2.5cm) object ingestion.
\paragraph*{Outcomes}  47 cases (52\%) underwent endoscopic intervention \cite{Elghali_2016, Lee_2007}, 29 cases (32\%) were managed conservatively \cite{Elghali_2016, Karp_1991b}, 15 cases (17\%) underwent surgical intervention \cite{Elghali_2016, Karp_1991b, Lee_2007}, 6 cases (7\%) reported complications \cite{Lee_2007}, 1 case (1\%) died \cite{Elghali_2016}.
\paragraph*{Geographical Location}Cases were recorded in 33 countries: 13 cases from USA \cite{Alao_2006i, Ataya_2013, Bhumi_2024f, Fry_2010, Guinan_2019f, Hardy_2023g, Jehangir_2019h, Kerestes_2019, Kumar_2001, Liu_2005, Tammana_2012j, Tay_2004, Tupesis_2004f}; 7 cases from India \cite{Bhasin_2014, Bhattacharjee_2008, Kar_2015, Kariholu_2008, Kumar_2019f, Misra_2013, Wadhwa_2015e} and UK \cite{Beecroft_1998, Berry_2021e, Cauchi_2002, Cox_2007, Gardner_2017h, Qureshi_2016}; 6 cases from Bulgaria \cite{Losanoff_1996, Losanoff_1997e}; 5 cases from Iran \cite{DivsalarP._2023a, Emamhadi_2018, Farhadi_2024h}; 4 cases from Turkey \cite{Akay_2015f, Atayan_2016, Tanrikulu_2015e, Yildiz_2016e}; 2 cases from China \cite{Jin_2023, Li_2013}, Poland \cite{Kobiela_2015, Wnęk_2015f}, and Spain \cite{CamachoDorado_2018, fjbuilsRepeatedBehaviorDeliberate2024}; 1 case from Australia \cite{Apikotoa_2022f}, Bahrain \cite{Ali_2020f}, Croatia \cite{Trgo_2012f}, Ecuador \cite{DelgadoSalazar_2020c}, Egypt \cite{Ali_2022g}, Ethiopia \cite{Mesfin_2022a}, Germany \cite{teWildt_2010}, Greece \cite{Sakellaridis_2008f}, Hungary \cite{Csaky_1998e}, Iraq \cite{Al-Faham_2020k}, Israel \cite{Goldman_1998f}, Italy \cite{Riva_2018j}, Japan \cite{Ohno_2005}, Nepal \cite{Thapa_2019f}, Netherlands \cite{Benoist_2019e}, Oman \cite{AlShaaibi_2021b}, Pakistan \cite{Yasin_2009}, Portugal \cite{Peixoto_2017f}, Qatar \cite{Ali_2017}, Saudi Arabia \cite{Sultan_2024f}, South Africa \cite{Sobnach_2011f}, Sweden \cite{Naji_2012f}, Switzerland \cite{Wildhaber_2005}, and Taiwan \cite{Chang_2017f}. \paragraph*{Gender} 43 cases (60\%) were male \cite{Akay_2015f, Al-Faham_2020k, Alao_2006i, Ali_2017, Ali_2022g, Apikotoa_2022f, Atayan_2016, Benoist_2019e, Berry_2021e, Bhumi_2024f, CamachoDorado_2018, Csaky_1998e, Emamhadi_2018, Farhadi_2024h, Fry_2010, Gardner_2017h, Guinan_2019f, Jehangir_2019h, Jin_2023, Kobiela_2015, Kumar_2001, Kumar_2019f, Liu_2005, Losanoff_1996, Losanoff_1997e, Mesfin_2022a, Misra_2013, Qureshi_2016, Riva_2018j, Sobnach_2011f, Tammana_2012j, Tanrikulu_2015e, Tay_2004, Thapa_2019f, Trgo_2012f, Wadhwa_2015e, Yasin_2009, teWildt_2010}, 28 cases (39\%) were female \cite{AlShaaibi_2021b, Ali_2020f, Ataya_2013, Beecroft_1998, Bhasin_2014, Bhattacharjee_2008, Cauchi_2002, Chang_2017f, Cox_2007, DelgadoSalazar_2020c, DivsalarP._2023a, Goldman_1998f, Hardy_2023g, Kar_2015, Kariholu_2008, Kerestes_2019, Li_2013, Naji_2012f, Ohno_2005, Peixoto_2017f, Sakellaridis_2008f, Sultan_2024f, Tupesis_2004f, Wildhaber_2005, Wnęk_2015f, Yildiz_2016e}, 1 case (1\%) had no gender recorded \cite{fjbuilsRepeatedBehaviorDeliberate2024}. \paragraph*{Age Group} 25 cases (35\%) were between 26 and 40 years of age \cite{Alao_2006i, Ali_2022g, Apikotoa_2022f, Ataya_2013, Benoist_2019e, Bhasin_2014, Chang_2017f, Cox_2007, DelgadoSalazar_2020c, Farhadi_2024h, Fry_2010, Gardner_2017h, Guinan_2019f, Jin_2023, Kumar_2019f, Losanoff_1996, Misra_2013, Qureshi_2016, Riva_2018j, Sakellaridis_2008f, Tammana_2012j, Trgo_2012f, Wnęk_2015f, Yildiz_2016e, fjbuilsRepeatedBehaviorDeliberate2024}, 18 cases (25\%) were between 18 and 25 years of age \cite{Akay_2015f, Ali_2017, Atayan_2016, Bhattacharjee_2008, Csaky_1998e, Kar_2015, Kariholu_2008, Kobiela_2015, Losanoff_1996, Losanoff_1997e, Mesfin_2022a, Peixoto_2017f, Sobnach_2011f, Tupesis_2004f, Yasin_2009}, 13 cases (18\%) were under 18 years of age \cite{AlShaaibi_2021b, Ali_2020f, Cauchi_2002, DivsalarP._2023a, Goldman_1998f, Liu_2005, Naji_2012f, Ohno_2005, Tanrikulu_2015e, Tay_2004, Wildhaber_2005}, 11 cases (15\%) were between 41 and 60 years of age \cite{Al-Faham_2020k, Bhumi_2024f, CamachoDorado_2018, Emamhadi_2018, Hardy_2023g, Jehangir_2019h, Kumar_2001, Sultan_2024f, Thapa_2019f, Wadhwa_2015e, teWildt_2010}, 3 cases (4\%) were over 60 years of age \cite{Beecroft_1998, Kerestes_2019, Li_2013}, 2 cases (3\%) had no age documented \cite{Berry_2021e}. \paragraph*{Population} 36 cases (50\%) had a psychiatric history \cite{AlShaaibi_2021b, Alao_2006i, Ali_2020f, Apikotoa_2022f, Ataya_2013, Atayan_2016, Beecroft_1998, CamachoDorado_2018, Chang_2017f, DelgadoSalazar_2020c, DivsalarP._2023a, Farhadi_2024h, Fry_2010, Guinan_2019f, Hardy_2023g, Jehangir_2019h, Jin_2023, Kar_2015, Kerestes_2019, Kobiela_2015, Kumar_2001, Kumar_2019f, Liu_2005, Mesfin_2022a, Misra_2013, Ohno_2005, Peixoto_2017f, Sakellaridis_2008f, Sultan_2024f, Tammana_2012j, Tanrikulu_2015e, Yildiz_2016e, fjbuilsRepeatedBehaviorDeliberate2024, teWildt_2010}, 19 cases (26\%) had ingested previously \cite{Alao_2006i, Apikotoa_2022f, Berry_2021e, Bhattacharjee_2008, Csaky_1998e, DivsalarP._2023a, Emamhadi_2018, Guinan_2019f, Jehangir_2019h, Jin_2023, Liu_2005, Sakellaridis_2008f, Tanrikulu_2015e, Thapa_2019f, Yildiz_2016e, fjbuilsRepeatedBehaviorDeliberate2024, teWildt_2010}, 12 cases (17\%) were detained persons \cite{Alao_2006i, Ali_2022g, Apikotoa_2022f, Losanoff_1996, Losanoff_1997e, Qureshi_2016, Tammana_2012j, Trgo_2012f}, 7 cases (10\%) were severely disabled \cite{Atayan_2016, Kerestes_2019, Liu_2005, Ohno_2005, Peixoto_2017f, Yildiz_2016e, teWildt_2010}, 4 cases (6\%) were psychiatric inpatients \cite{DivsalarP._2023a, fjbuilsRepeatedBehaviorDeliberate2024, teWildt_2010}, 3 cases (4\%) were under the influence of alcohol \cite{Benoist_2019e, Csaky_1998e, Thapa_2019f}, 2 cases (3\%) were displaced people \cite{Akay_2015f, Gardner_2017h}. \paragraph*{Motivation} 34 cases (47\%) had a psychiatric motivation \cite{Al-Faham_2020k, Alao_2006i, Ali_2020f, Apikotoa_2022f, Ataya_2013, Atayan_2016, Bhasin_2014, Bhattacharjee_2008, DelgadoSalazar_2020c, DivsalarP._2023a, Emamhadi_2018, Farhadi_2024h, Guinan_2019f, Hardy_2023g, Jehangir_2019h, Jin_2023, Kar_2015, Kariholu_2008, Kerestes_2019, Kobiela_2015, Kumar_2001, Kumar_2019f, Li_2013, Liu_2005, Misra_2013, Ohno_2005, Sakellaridis_2008f, Sultan_2024f, Tammana_2012j, Tanrikulu_2015e, Yasin_2009, teWildt_2010}, 21 cases (29\%) were motivated by self-harm intention \cite{Al-Faham_2020k, AlShaaibi_2021b, Alao_2006i, Ali_2017, CamachoDorado_2018, Chang_2017f, Cox_2007, Csaky_1998e, Fry_2010, Li_2013, Losanoff_1996, Losanoff_1997e, Mesfin_2022a, Sakellaridis_2008f, Tammana_2012j, Tanrikulu_2015e, fjbuilsRepeatedBehaviorDeliberate2024}, 17 cases (24\%) had a psychosocial motivation \cite{Akay_2015f, Benoist_2019e, Bhattacharjee_2008, Cauchi_2002, Goldman_1998f, Hardy_2023g, Kobiela_2015, Li_2013, Naji_2012f, Qureshi_2016, Riva_2018j, Sobnach_2011f, Tay_2004, Thapa_2019f, Tupesis_2004f, Wildhaber_2005, Wnęk_2015f}, 9 cases (12\%) were motivated by protest \cite{Bhumi_2024f, Gardner_2017h, Losanoff_1996, Losanoff_1997e, Tupesis_2004f}, 9 cases (12\%) had another documented motivation \cite{Ali_2020f, Ali_2022g, Emamhadi_2018, Guinan_2019f, Peixoto_2017f, Sakellaridis_2008f, Trgo_2012f, Wadhwa_2015e, Yildiz_2016e}. \paragraph*{Object Characteristics} 51 cases (71\%) ingested a large diameter object (\textgreater{}2.5cm) \cite{Akay_2015f, Al-Faham_2020k, AlShaaibi_2021b, Alao_2006i, Ali_2017, Ali_2022g, Apikotoa_2022f, Atayan_2016, Berry_2021e, Bhasin_2014, CamachoDorado_2018, Cauchi_2002, Chang_2017f, Cox_2007, Csaky_1998e, DivsalarP._2023a, Emamhadi_2018, Gardner_2017h, Guinan_2019f, Jehangir_2019h, Jin_2023, Kariholu_2008, Kerestes_2019, Kobiela_2015, Kumar_2001, Kumar_2019f, Losanoff_1996, Losanoff_1997e, Mesfin_2022a, Misra_2013, Naji_2012f, Ohno_2005, Peixoto_2017f, Qureshi_2016, Riva_2018j, Sakellaridis_2008f, Sultan_2024f, Tanrikulu_2015e, Thapa_2019f, Trgo_2012f, Wnęk_2015f, Yildiz_2016e, fjbuilsRepeatedBehaviorDeliberate2024, teWildt_2010}, 44 cases (61\%) ingested multiple objects \cite{Ali_2020f, Apikotoa_2022f, Ataya_2013, Atayan_2016, Beecroft_1998, Bhattacharjee_2008, Bhumi_2024f, CamachoDorado_2018, Cauchi_2002, Emamhadi_2018, Farhadi_2024h, Fry_2010, Goldman_1998f, Guinan_2019f, Hardy_2023g, Jehangir_2019h, Jin_2023, Kar_2015, Kariholu_2008, Kobiela_2015, Kumar_2001, Kumar_2019f, Li_2013, Liu_2005, Losanoff_1996, Mesfin_2022a, Misra_2013, Naji_2012f, Ohno_2005, Sobnach_2011f, Sultan_2024f, Tammana_2012j, Tanrikulu_2015e, Tay_2004, Thapa_2019f, Wadhwa_2015e, Wildhaber_2005, Yasin_2009, fjbuilsRepeatedBehaviorDeliberate2024, teWildt_2010}, 34 cases (47\%) ingested a sharp object \cite{AlShaaibi_2021b, Alao_2006i, Apikotoa_2022f, Ataya_2013, Benoist_2019e, Bhasin_2014, Bhattacharjee_2008, CamachoDorado_2018, Csaky_1998e, DelgadoSalazar_2020c, DivsalarP._2023a, Emamhadi_2018, Farhadi_2024h, Fry_2010, Guinan_2019f, Hardy_2023g, Jehangir_2019h, Jin_2023, Kariholu_2008, Kobiela_2015, Kumar_2019f, Losanoff_1996, Losanoff_1997e, Mesfin_2022a, Misra_2013, Sobnach_2011f, Yasin_2009, teWildt_2010}, 32 cases (44\%) ingested a long object (\textgreater{}5cm) \cite{Al-Faham_2020k, AlShaaibi_2021b, Ali_2017, Ali_2022g, Atayan_2016, Bhasin_2014, CamachoDorado_2018, Chang_2017f, Cox_2007, Csaky_1998e, DivsalarP._2023a, Emamhadi_2018, Fry_2010, Gardner_2017h, Jin_2023, Kariholu_2008, Kerestes_2019, Kobiela_2015, Kumar_2019f, Mesfin_2022a, Misra_2013, Ohno_2005, Qureshi_2016, Sakellaridis_2008f, Sultan_2024f, Thapa_2019f, Trgo_2012f, Yasin_2009, Yildiz_2016e, teWildt_2010}, 9 cases (12\%) ingested a magnet \cite{Ali_2020f, Bhumi_2024f, Cauchi_2002, Liu_2005, Naji_2012f, Ohno_2005, Tanrikulu_2015e, Tay_2004, Wildhaber_2005}, 2 cases (3\%) ingested a button battery \cite{Berry_2021e, Bhumi_2024f}. \paragraph*{Outcomes} 48 cases (67\%) experienced a complication \cite{Ali_2017, Ali_2020f, Apikotoa_2022f, Atayan_2016, Beecroft_1998, Benoist_2019e, Berry_2021e, Bhasin_2014, Bhumi_2024f, CamachoDorado_2018, Cauchi_2002, Cox_2007, Csaky_1998e, DelgadoSalazar_2020c, DivsalarP._2023a, Emamhadi_2018, Farhadi_2024h, Fry_2010, Gardner_2017h, Goldman_1998f, Jin_2023, Kariholu_2008, Kerestes_2019, Kobiela_2015, Kumar_2001, Kumar_2019f, Liu_2005, Losanoff_1996, Mesfin_2022a, Misra_2013, Naji_2012f, Ohno_2005, Sakellaridis_2008f, Sobnach_2011f, Sultan_2024f, Tanrikulu_2015e, Tay_2004, Thapa_2019f, Trgo_2012f, Tupesis_2004f, Wildhaber_2005, Wnęk_2015f, Yasin_2009, Yildiz_2016e}, 44 cases (61\%) underwent surgery \cite{Al-Faham_2020k, AlShaaibi_2021b, Alao_2006i, Ali_2017, Ali_2020f, Atayan_2016, Beecroft_1998, Bhasin_2014, CamachoDorado_2018, Cauchi_2002, Chang_2017f, Cox_2007, Csaky_1998e, DelgadoSalazar_2020c, DivsalarP._2023a, Farhadi_2024h, Fry_2010, Gardner_2017h, Jin_2023, Kariholu_2008, Kerestes_2019, Kobiela_2015, Kumar_2019f, Liu_2005, Losanoff_1996, Losanoff_1997e, Mesfin_2022a, Misra_2013, Naji_2012f, Sobnach_2011f, Tanrikulu_2015e, Tay_2004, Thapa_2019f, Tupesis_2004f, Wildhaber_2005, Wnęk_2015f, Yasin_2009, Yildiz_2016e, fjbuilsRepeatedBehaviorDeliberate2024}, 31 cases (43\%) underwent endoscopy \cite{Akay_2015f, Ali_2022g, Apikotoa_2022f, Atayan_2016, Benoist_2019e, Berry_2021e, Bhasin_2014, Bhumi_2024f, CamachoDorado_2018, Chang_2017f, DelgadoSalazar_2020c, Gardner_2017h, Guinan_2019f, Hardy_2023g, Jehangir_2019h, Kariholu_2008, Li_2013, Liu_2005, Ohno_2005, Peixoto_2017f, Qureshi_2016, Riva_2018j, Sakellaridis_2008f, Sultan_2024f, Tammana_2012j, Tanrikulu_2015e, Trgo_2012f, Wadhwa_2015e, Wnęk_2015f, teWildt_2010}, 7 cases (10\%) were managed conservatively \cite{Ataya_2013, Bhattacharjee_2008, DivsalarP._2023a, Emamhadi_2018, Goldman_1998f, Kar_2015, Kumar_2001}, 2 cases (3\%) died \cite{Emamhadi_2018, Kumar_2001}. All 90 were male gender. 90 cases (100\%) were detained at the time of ingestion \cite{Elghali_2016, Karp_1991b, Lee_2007}, 88 cases (98\%) were intentional ingestions \cite{Elghali_2016, Karp_1991b, Lee_2007}, 30 cases (33\%) had a psychiatric history documented \cite{Elghali_2016, Karp_1991b, Lee_2007}, 2 cases (2\%) had a history of prior ingestion \cite{Elghali_2016}. No cases were reported for were psychiatric inpatients, were displaced people, were under the influence of alcohol at the time of ingestion, and had a severe disability history.
\paragraph*{Motivation}  70 cases (78\%) reported protest motivation \cite{Elghali_2016, Karp_1991b, Lee_2007}, 12 cases (13\%) reported psychiatric motivation \cite{Karp_1991b}, 6 cases (7\%) reported self-harm motivation \cite{Elghali_2016, Karp_1991b}. No cases were reported for psychosocial motivation and other motivation.
\paragraph*{Object Characteristics}  68 cases (76\%) involved sharp object ingestion \cite{Elghali_2016, Karp_1991b, Lee_2007}, 32 cases (36\%) involved long (\textgreater 5cm) object ingestion \cite{Lee_2007}, 25 cases (28\%) involved ingestion of multiple objects \cite{Elghali_2016, Lee_2007}. No cases were reported for button battery ingestion, magnet ingestion, and involved large diameter (\textgreater 2.5cm) object ingestion.
\paragraph*{Outcomes}  47 cases (52\%) underwent endoscopic intervention \cite{Elghali_2016, Lee_2007}, 29 cases (32\%) were managed conservatively \cite{Elghali_2016, Karp_1991b}, 15 cases (17\%) underwent surgical intervention \cite{Elghali_2016, Karp_1991b, Lee_2007}, 6 cases (7\%) reported complications \cite{Lee_2007}, 1 case (1\%) died \cite{Elghali_2016}.
\paragraph*{Geographical Location}Cases were recorded in 33 countries: 13 cases from USA \cite{Alao_2006i, Ataya_2013, Bhumi_2024f, Fry_2010, Guinan_2019f, Hardy_2023g, Jehangir_2019h, Kerestes_2019, Kumar_2001, Liu_2005, Tammana_2012j, Tay_2004, Tupesis_2004f}; 7 cases from India \cite{Bhasin_2014, Bhattacharjee_2008, Kar_2015, Kariholu_2008, Kumar_2019f, Misra_2013, Wadhwa_2015e} and UK \cite{Beecroft_1998, Berry_2021e, Cauchi_2002, Cox_2007, Gardner_2017h, Qureshi_2016}; 6 cases from Bulgaria \cite{Losanoff_1996, Losanoff_1997e}; 5 cases from Iran \cite{DivsalarP._2023a, Emamhadi_2018, Farhadi_2024h}; 4 cases from Turkey \cite{Akay_2015f, Atayan_2016, Tanrikulu_2015e, Yildiz_2016e}; 2 cases from China \cite{Jin_2023, Li_2013}, Poland \cite{Kobiela_2015, Wnęk_2015f}, and Spain \cite{CamachoDorado_2018, fjbuilsRepeatedBehaviorDeliberate2024}; 1 case from Australia \cite{Apikotoa_2022f}, Bahrain \cite{Ali_2020f}, Croatia \cite{Trgo_2012f}, Ecuador \cite{DelgadoSalazar_2020c}, Egypt \cite{Ali_2022g}, Ethiopia \cite{Mesfin_2022a}, Germany \cite{teWildt_2010}, Greece \cite{Sakellaridis_2008f}, Hungary \cite{Csaky_1998e}, Iraq \cite{Al-Faham_2020k}, Israel \cite{Goldman_1998f}, Italy \cite{Riva_2018j}, Japan \cite{Ohno_2005}, Nepal \cite{Thapa_2019f}, Netherlands \cite{Benoist_2019e}, Oman \cite{AlShaaibi_2021b}, Pakistan \cite{Yasin_2009}, Portugal \cite{Peixoto_2017f}, Qatar \cite{Ali_2017}, Saudi Arabia \cite{Sultan_2024f}, South Africa \cite{Sobnach_2011f}, Sweden \cite{Naji_2012f}, Switzerland \cite{Wildhaber_2005}, and Taiwan \cite{Chang_2017f}. \paragraph*{Gender} 43 cases (60\%) were male \cite{Akay_2015f, Al-Faham_2020k, Alao_2006i, Ali_2017, Ali_2022g, Apikotoa_2022f, Atayan_2016, Benoist_2019e, Berry_2021e, Bhumi_2024f, CamachoDorado_2018, Csaky_1998e, Emamhadi_2018, Farhadi_2024h, Fry_2010, Gardner_2017h, Guinan_2019f, Jehangir_2019h, Jin_2023, Kobiela_2015, Kumar_2001, Kumar_2019f, Liu_2005, Losanoff_1996, Losanoff_1997e, Mesfin_2022a, Misra_2013, Qureshi_2016, Riva_2018j, Sobnach_2011f, Tammana_2012j, Tanrikulu_2015e, Tay_2004, Thapa_2019f, Trgo_2012f, Wadhwa_2015e, Yasin_2009, teWildt_2010}, 28 cases (39\%) were female \cite{AlShaaibi_2021b, Ali_2020f, Ataya_2013, Beecroft_1998, Bhasin_2014, Bhattacharjee_2008, Cauchi_2002, Chang_2017f, Cox_2007, DelgadoSalazar_2020c, DivsalarP._2023a, Goldman_1998f, Hardy_2023g, Kar_2015, Kariholu_2008, Kerestes_2019, Li_2013, Naji_2012f, Ohno_2005, Peixoto_2017f, Sakellaridis_2008f, Sultan_2024f, Tupesis_2004f, Wildhaber_2005, Wnęk_2015f, Yildiz_2016e}, 1 case (1\%) had no gender recorded \cite{fjbuilsRepeatedBehaviorDeliberate2024}. \paragraph*{Age Group} 25 cases (35\%) were between 26 and 40 years of age \cite{Alao_2006i, Ali_2022g, Apikotoa_2022f, Ataya_2013, Benoist_2019e, Bhasin_2014, Chang_2017f, Cox_2007, DelgadoSalazar_2020c, Farhadi_2024h, Fry_2010, Gardner_2017h, Guinan_2019f, Jin_2023, Kumar_2019f, Losanoff_1996, Misra_2013, Qureshi_2016, Riva_2018j, Sakellaridis_2008f, Tammana_2012j, Trgo_2012f, Wnęk_2015f, Yildiz_2016e, fjbuilsRepeatedBehaviorDeliberate2024}, 18 cases (25\%) were between 18 and 25 years of age \cite{Akay_2015f, Ali_2017, Atayan_2016, Bhattacharjee_2008, Csaky_1998e, Kar_2015, Kariholu_2008, Kobiela_2015, Losanoff_1996, Losanoff_1997e, Mesfin_2022a, Peixoto_2017f, Sobnach_2011f, Tupesis_2004f, Yasin_2009}, 13 cases (18\%) were under 18 years of age \cite{AlShaaibi_2021b, Ali_2020f, Cauchi_2002, DivsalarP._2023a, Goldman_1998f, Liu_2005, Naji_2012f, Ohno_2005, Tanrikulu_2015e, Tay_2004, Wildhaber_2005}, 11 cases (15\%) were between 41 and 60 years of age \cite{Al-Faham_2020k, Bhumi_2024f, CamachoDorado_2018, Emamhadi_2018, Hardy_2023g, Jehangir_2019h, Kumar_2001, Sultan_2024f, Thapa_2019f, Wadhwa_2015e, teWildt_2010}, 3 cases (4\%) were over 60 years of age \cite{Beecroft_1998, Kerestes_2019, Li_2013}, 2 cases (3\%) had no age documented \cite{Berry_2021e}. \paragraph*{Population} 36 cases (50\%) had a psychiatric history \cite{AlShaaibi_2021b, Alao_2006i, Ali_2020f, Apikotoa_2022f, Ataya_2013, Atayan_2016, Beecroft_1998, CamachoDorado_2018, Chang_2017f, DelgadoSalazar_2020c, DivsalarP._2023a, Farhadi_2024h, Fry_2010, Guinan_2019f, Hardy_2023g, Jehangir_2019h, Jin_2023, Kar_2015, Kerestes_2019, Kobiela_2015, Kumar_2001, Kumar_2019f, Liu_2005, Mesfin_2022a, Misra_2013, Ohno_2005, Peixoto_2017f, Sakellaridis_2008f, Sultan_2024f, Tammana_2012j, Tanrikulu_2015e, Yildiz_2016e, fjbuilsRepeatedBehaviorDeliberate2024, teWildt_2010}, 19 cases (26\%) had ingested previously \cite{Alao_2006i, Apikotoa_2022f, Berry_2021e, Bhattacharjee_2008, Csaky_1998e, DivsalarP._2023a, Emamhadi_2018, Guinan_2019f, Jehangir_2019h, Jin_2023, Liu_2005, Sakellaridis_2008f, Tanrikulu_2015e, Thapa_2019f, Yildiz_2016e, fjbuilsRepeatedBehaviorDeliberate2024, teWildt_2010}, 12 cases (17\%) were detained persons \cite{Alao_2006i, Ali_2022g, Apikotoa_2022f, Losanoff_1996, Losanoff_1997e, Qureshi_2016, Tammana_2012j, Trgo_2012f}, 7 cases (10\%) were severely disabled \cite{Atayan_2016, Kerestes_2019, Liu_2005, Ohno_2005, Peixoto_2017f, Yildiz_2016e, teWildt_2010}, 4 cases (6\%) were psychiatric inpatients \cite{DivsalarP._2023a, fjbuilsRepeatedBehaviorDeliberate2024, teWildt_2010}, 3 cases (4\%) were under the influence of alcohol \cite{Benoist_2019e, Csaky_1998e, Thapa_2019f}, 2 cases (3\%) were displaced people \cite{Akay_2015f, Gardner_2017h}. \paragraph*{Motivation} 34 cases (47\%) had a psychiatric motivation \cite{Al-Faham_2020k, Alao_2006i, Ali_2020f, Apikotoa_2022f, Ataya_2013, Atayan_2016, Bhasin_2014, Bhattacharjee_2008, DelgadoSalazar_2020c, DivsalarP._2023a, Emamhadi_2018, Farhadi_2024h, Guinan_2019f, Hardy_2023g, Jehangir_2019h, Jin_2023, Kar_2015, Kariholu_2008, Kerestes_2019, Kobiela_2015, Kumar_2001, Kumar_2019f, Li_2013, Liu_2005, Misra_2013, Ohno_2005, Sakellaridis_2008f, Sultan_2024f, Tammana_2012j, Tanrikulu_2015e, Yasin_2009, teWildt_2010}, 21 cases (29\%) were motivated by self-harm intention \cite{Al-Faham_2020k, AlShaaibi_2021b, Alao_2006i, Ali_2017, CamachoDorado_2018, Chang_2017f, Cox_2007, Csaky_1998e, Fry_2010, Li_2013, Losanoff_1996, Losanoff_1997e, Mesfin_2022a, Sakellaridis_2008f, Tammana_2012j, Tanrikulu_2015e, fjbuilsRepeatedBehaviorDeliberate2024}, 17 cases (24\%) had a psychosocial motivation \cite{Akay_2015f, Benoist_2019e, Bhattacharjee_2008, Cauchi_2002, Goldman_1998f, Hardy_2023g, Kobiela_2015, Li_2013, Naji_2012f, Qureshi_2016, Riva_2018j, Sobnach_2011f, Tay_2004, Thapa_2019f, Tupesis_2004f, Wildhaber_2005, Wnęk_2015f}, 9 cases (12\%) were motivated by protest \cite{Bhumi_2024f, Gardner_2017h, Losanoff_1996, Losanoff_1997e, Tupesis_2004f}, 9 cases (12\%) had another documented motivation \cite{Ali_2020f, Ali_2022g, Emamhadi_2018, Guinan_2019f, Peixoto_2017f, Sakellaridis_2008f, Trgo_2012f, Wadhwa_2015e, Yildiz_2016e}. \paragraph*{Object Characteristics} 51 cases (71\%) ingested a large diameter object (\textgreater{}2.5cm) \cite{Akay_2015f, Al-Faham_2020k, AlShaaibi_2021b, Alao_2006i, Ali_2017, Ali_2022g, Apikotoa_2022f, Atayan_2016, Berry_2021e, Bhasin_2014, CamachoDorado_2018, Cauchi_2002, Chang_2017f, Cox_2007, Csaky_1998e, DivsalarP._2023a, Emamhadi_2018, Gardner_2017h, Guinan_2019f, Jehangir_2019h, Jin_2023, Kariholu_2008, Kerestes_2019, Kobiela_2015, Kumar_2001, Kumar_2019f, Losanoff_1996, Losanoff_1997e, Mesfin_2022a, Misra_2013, Naji_2012f, Ohno_2005, Peixoto_2017f, Qureshi_2016, Riva_2018j, Sakellaridis_2008f, Sultan_2024f, Tanrikulu_2015e, Thapa_2019f, Trgo_2012f, Wnęk_2015f, Yildiz_2016e, fjbuilsRepeatedBehaviorDeliberate2024, teWildt_2010}, 44 cases (61\%) ingested multiple objects \cite{Ali_2020f, Apikotoa_2022f, Ataya_2013, Atayan_2016, Beecroft_1998, Bhattacharjee_2008, Bhumi_2024f, CamachoDorado_2018, Cauchi_2002, Emamhadi_2018, Farhadi_2024h, Fry_2010, Goldman_1998f, Guinan_2019f, Hardy_2023g, Jehangir_2019h, Jin_2023, Kar_2015, Kariholu_2008, Kobiela_2015, Kumar_2001, Kumar_2019f, Li_2013, Liu_2005, Losanoff_1996, Mesfin_2022a, Misra_2013, Naji_2012f, Ohno_2005, Sobnach_2011f, Sultan_2024f, Tammana_2012j, Tanrikulu_2015e, Tay_2004, Thapa_2019f, Wadhwa_2015e, Wildhaber_2005, Yasin_2009, fjbuilsRepeatedBehaviorDeliberate2024, teWildt_2010}, 34 cases (47\%) ingested a sharp object \cite{AlShaaibi_2021b, Alao_2006i, Apikotoa_2022f, Ataya_2013, Benoist_2019e, Bhasin_2014, Bhattacharjee_2008, CamachoDorado_2018, Csaky_1998e, DelgadoSalazar_2020c, DivsalarP._2023a, Emamhadi_2018, Farhadi_2024h, Fry_2010, Guinan_2019f, Hardy_2023g, Jehangir_2019h, Jin_2023, Kariholu_2008, Kobiela_2015, Kumar_2019f, Losanoff_1996, Losanoff_1997e, Mesfin_2022a, Misra_2013, Sobnach_2011f, Yasin_2009, teWildt_2010}, 32 cases (44\%) ingested a long object (\textgreater{}5cm) \cite{Al-Faham_2020k, AlShaaibi_2021b, Ali_2017, Ali_2022g, Atayan_2016, Bhasin_2014, CamachoDorado_2018, Chang_2017f, Cox_2007, Csaky_1998e, DivsalarP._2023a, Emamhadi_2018, Fry_2010, Gardner_2017h, Jin_2023, Kariholu_2008, Kerestes_2019, Kobiela_2015, Kumar_2019f, Mesfin_2022a, Misra_2013, Ohno_2005, Qureshi_2016, Sakellaridis_2008f, Sultan_2024f, Thapa_2019f, Trgo_2012f, Yasin_2009, Yildiz_2016e, teWildt_2010}, 9 cases (12\%) ingested a magnet \cite{Ali_2020f, Bhumi_2024f, Cauchi_2002, Liu_2005, Naji_2012f, Ohno_2005, Tanrikulu_2015e, Tay_2004, Wildhaber_2005}, 2 cases (3\%) ingested a button battery \cite{Berry_2021e, Bhumi_2024f}. \paragraph*{Outcomes} 48 cases (67\%) experienced a complication \cite{Ali_2017, Ali_2020f, Apikotoa_2022f, Atayan_2016, Beecroft_1998, Benoist_2019e, Berry_2021e, Bhasin_2014, Bhumi_2024f, CamachoDorado_2018, Cauchi_2002, Cox_2007, Csaky_1998e, DelgadoSalazar_2020c, DivsalarP._2023a, Emamhadi_2018, Farhadi_2024h, Fry_2010, Gardner_2017h, Goldman_1998f, Jin_2023, Kariholu_2008, Kerestes_2019, Kobiela_2015, Kumar_2001, Kumar_2019f, Liu_2005, Losanoff_1996, Mesfin_2022a, Misra_2013, Naji_2012f, Ohno_2005, Sakellaridis_2008f, Sobnach_2011f, Sultan_2024f, Tanrikulu_2015e, Tay_2004, Thapa_2019f, Trgo_2012f, Tupesis_2004f, Wildhaber_2005, Wnęk_2015f, Yasin_2009, Yildiz_2016e}, 44 cases (61\%) underwent surgery \cite{Al-Faham_2020k, AlShaaibi_2021b, Alao_2006i, Ali_2017, Ali_2020f, Atayan_2016, Beecroft_1998, Bhasin_2014, CamachoDorado_2018, Cauchi_2002, Chang_2017f, Cox_2007, Csaky_1998e, DelgadoSalazar_2020c, DivsalarP._2023a, Farhadi_2024h, Fry_2010, Gardner_2017h, Jin_2023, Kariholu_2008, Kerestes_2019, Kobiela_2015, Kumar_2019f, Liu_2005, Losanoff_1996, Losanoff_1997e, Mesfin_2022a, Misra_2013, Naji_2012f, Sobnach_2011f, Tanrikulu_2015e, Tay_2004, Thapa_2019f, Tupesis_2004f, Wildhaber_2005, Wnęk_2015f, Yasin_2009, Yildiz_2016e, fjbuilsRepeatedBehaviorDeliberate2024}, 31 cases (43\%) underwent endoscopy \cite{Akay_2015f, Ali_2022g, Apikotoa_2022f, Atayan_2016, Benoist_2019e, Berry_2021e, Bhasin_2014, Bhumi_2024f, CamachoDorado_2018, Chang_2017f, DelgadoSalazar_2020c, Gardner_2017h, Guinan_2019f, Hardy_2023g, Jehangir_2019h, Kariholu_2008, Li_2013, Liu_2005, Ohno_2005, Peixoto_2017f, Qureshi_2016, Riva_2018j, Sakellaridis_2008f, Sultan_2024f, Tammana_2012j, Tanrikulu_2015e, Trgo_2012f, Wadhwa_2015e, Wnęk_2015f, teWildt_2010}, 7 cases (10\%) were managed conservatively \cite{Ataya_2013, Bhattacharjee_2008, DivsalarP._2023a, Emamhadi_2018, Goldman_1998f, Kar_2015, Kumar_2001}, 2 cases (3\%) died \cite{Emamhadi_2018, Kumar_2001}. All 90 were male gender. 90 cases (100\%) were detained at the time of ingestion \cite{Elghali_2016, Karp_1991b, Lee_2007}, 88 cases (98\%) were intentional ingestions \cite{Elghali_2016, Karp_1991b, Lee_2007}, 30 cases (33\%) had a psychiatric history documented \cite{Elghali_2016, Karp_1991b, Lee_2007}, 2 cases (2\%) had a history of prior ingestion \cite{Elghali_2016}. No cases were reported for were psychiatric inpatients, were displaced people, were under the influence of alcohol at the time of ingestion, and had a severe disability history.
\paragraph*{Motivation}  70 cases (78\%) reported protest motivation \cite{Elghali_2016, Karp_1991b, Lee_2007}, 12 cases (13\%) reported psychiatric motivation \cite{Karp_1991b}, 6 cases (7\%) reported self-harm motivation \cite{Elghali_2016, Karp_1991b}. No cases were reported for psychosocial motivation and other motivation.
\paragraph*{Object Characteristics}  68 cases (76\%) involved sharp object ingestion \cite{Elghali_2016, Karp_1991b, Lee_2007}, 32 cases (36\%) involved long (\textgreater 5cm) object ingestion \cite{Lee_2007}, 25 cases (28\%) involved ingestion of multiple objects \cite{Elghali_2016, Lee_2007}. No cases were reported for button battery ingestion, magnet ingestion, and involved large diameter (\textgreater 2.5cm) object ingestion.
\paragraph*{Outcomes}  47 cases (52\%) underwent endoscopic intervention \cite{Elghali_2016, Lee_2007}, 29 cases (32\%) were managed conservatively \cite{Elghali_2016, Karp_1991b}, 15 cases (17\%) underwent surgical intervention \cite{Elghali_2016, Karp_1991b, Lee_2007}, 6 cases (7\%) reported complications \cite{Lee_2007}, 1 case (1\%) died \cite{Elghali_2016}.
\paragraph*{Geographical Location}Cases were recorded in 33 countries: 13 cases from USA \cite{Alao_2006i, Ataya_2013, Bhumi_2024f, Fry_2010, Guinan_2019f, Hardy_2023g, Jehangir_2019h, Kerestes_2019, Kumar_2001, Liu_2005, Tammana_2012j, Tay_2004, Tupesis_2004f}; 7 cases from India \cite{Bhasin_2014, Bhattacharjee_2008, Kar_2015, Kariholu_2008, Kumar_2019f, Misra_2013, Wadhwa_2015e} and UK \cite{Beecroft_1998, Berry_2021e, Cauchi_2002, Cox_2007, Gardner_2017h, Qureshi_2016}; 6 cases from Bulgaria \cite{Losanoff_1996, Losanoff_1997e}; 5 cases from Iran \cite{DivsalarP._2023a, Emamhadi_2018, Farhadi_2024h}; 4 cases from Turkey \cite{Akay_2015f, Atayan_2016, Tanrikulu_2015e, Yildiz_2016e}; 2 cases from China \cite{Jin_2023, Li_2013}, Poland \cite{Kobiela_2015, Wnęk_2015f}, and Spain \cite{CamachoDorado_2018, fjbuilsRepeatedBehaviorDeliberate2024}; 1 case from Australia \cite{Apikotoa_2022f}, Bahrain \cite{Ali_2020f}, Croatia \cite{Trgo_2012f}, Ecuador \cite{DelgadoSalazar_2020c}, Egypt \cite{Ali_2022g}, Ethiopia \cite{Mesfin_2022a}, Germany \cite{teWildt_2010}, Greece \cite{Sakellaridis_2008f}, Hungary \cite{Csaky_1998e}, Iraq \cite{Al-Faham_2020k}, Israel \cite{Goldman_1998f}, Italy \cite{Riva_2018j}, Japan \cite{Ohno_2005}, Nepal \cite{Thapa_2019f}, Netherlands \cite{Benoist_2019e}, Oman \cite{AlShaaibi_2021b}, Pakistan \cite{Yasin_2009}, Portugal \cite{Peixoto_2017f}, Qatar \cite{Ali_2017}, Saudi Arabia \cite{Sultan_2024f}, South Africa \cite{Sobnach_2011f}, Sweden \cite{Naji_2012f}, Switzerland \cite{Wildhaber_2005}, and Taiwan \cite{Chang_2017f}. \paragraph*{Gender} 43 cases (60\%) were male \cite{Akay_2015f, Al-Faham_2020k, Alao_2006i, Ali_2017, Ali_2022g, Apikotoa_2022f, Atayan_2016, Benoist_2019e, Berry_2021e, Bhumi_2024f, CamachoDorado_2018, Csaky_1998e, Emamhadi_2018, Farhadi_2024h, Fry_2010, Gardner_2017h, Guinan_2019f, Jehangir_2019h, Jin_2023, Kobiela_2015, Kumar_2001, Kumar_2019f, Liu_2005, Losanoff_1996, Losanoff_1997e, Mesfin_2022a, Misra_2013, Qureshi_2016, Riva_2018j, Sobnach_2011f, Tammana_2012j, Tanrikulu_2015e, Tay_2004, Thapa_2019f, Trgo_2012f, Wadhwa_2015e, Yasin_2009, teWildt_2010}, 28 cases (39\%) were female \cite{AlShaaibi_2021b, Ali_2020f, Ataya_2013, Beecroft_1998, Bhasin_2014, Bhattacharjee_2008, Cauchi_2002, Chang_2017f, Cox_2007, DelgadoSalazar_2020c, DivsalarP._2023a, Goldman_1998f, Hardy_2023g, Kar_2015, Kariholu_2008, Kerestes_2019, Li_2013, Naji_2012f, Ohno_2005, Peixoto_2017f, Sakellaridis_2008f, Sultan_2024f, Tupesis_2004f, Wildhaber_2005, Wnęk_2015f, Yildiz_2016e}, 1 case (1\%) had no gender recorded \cite{fjbuilsRepeatedBehaviorDeliberate2024}. \paragraph*{Age Group} 25 cases (35\%) were between 26 and 40 years of age \cite{Alao_2006i, Ali_2022g, Apikotoa_2022f, Ataya_2013, Benoist_2019e, Bhasin_2014, Chang_2017f, Cox_2007, DelgadoSalazar_2020c, Farhadi_2024h, Fry_2010, Gardner_2017h, Guinan_2019f, Jin_2023, Kumar_2019f, Losanoff_1996, Misra_2013, Qureshi_2016, Riva_2018j, Sakellaridis_2008f, Tammana_2012j, Trgo_2012f, Wnęk_2015f, Yildiz_2016e, fjbuilsRepeatedBehaviorDeliberate2024}, 18 cases (25\%) were between 18 and 25 years of age \cite{Akay_2015f, Ali_2017, Atayan_2016, Bhattacharjee_2008, Csaky_1998e, Kar_2015, Kariholu_2008, Kobiela_2015, Losanoff_1996, Losanoff_1997e, Mesfin_2022a, Peixoto_2017f, Sobnach_2011f, Tupesis_2004f, Yasin_2009}, 13 cases (18\%) were under 18 years of age \cite{AlShaaibi_2021b, Ali_2020f, Cauchi_2002, DivsalarP._2023a, Goldman_1998f, Liu_2005, Naji_2012f, Ohno_2005, Tanrikulu_2015e, Tay_2004, Wildhaber_2005}, 11 cases (15\%) were between 41 and 60 years of age \cite{Al-Faham_2020k, Bhumi_2024f, CamachoDorado_2018, Emamhadi_2018, Hardy_2023g, Jehangir_2019h, Kumar_2001, Sultan_2024f, Thapa_2019f, Wadhwa_2015e, teWildt_2010}, 3 cases (4\%) were over 60 years of age \cite{Beecroft_1998, Kerestes_2019, Li_2013}, 2 cases (3\%) had no age documented \cite{Berry_2021e}. \paragraph*{Population} 36 cases (50\%) had a psychiatric history \cite{AlShaaibi_2021b, Alao_2006i, Ali_2020f, Apikotoa_2022f, Ataya_2013, Atayan_2016, Beecroft_1998, CamachoDorado_2018, Chang_2017f, DelgadoSalazar_2020c, DivsalarP._2023a, Farhadi_2024h, Fry_2010, Guinan_2019f, Hardy_2023g, Jehangir_2019h, Jin_2023, Kar_2015, Kerestes_2019, Kobiela_2015, Kumar_2001, Kumar_2019f, Liu_2005, Mesfin_2022a, Misra_2013, Ohno_2005, Peixoto_2017f, Sakellaridis_2008f, Sultan_2024f, Tammana_2012j, Tanrikulu_2015e, Yildiz_2016e, fjbuilsRepeatedBehaviorDeliberate2024, teWildt_2010}, 19 cases (26\%) had ingested previously \cite{Alao_2006i, Apikotoa_2022f, Berry_2021e, Bhattacharjee_2008, Csaky_1998e, DivsalarP._2023a, Emamhadi_2018, Guinan_2019f, Jehangir_2019h, Jin_2023, Liu_2005, Sakellaridis_2008f, Tanrikulu_2015e, Thapa_2019f, Yildiz_2016e, fjbuilsRepeatedBehaviorDeliberate2024, teWildt_2010}, 12 cases (17\%) were detained persons \cite{Alao_2006i, Ali_2022g, Apikotoa_2022f, Losanoff_1996, Losanoff_1997e, Qureshi_2016, Tammana_2012j, Trgo_2012f}, 7 cases (10\%) were severely disabled \cite{Atayan_2016, Kerestes_2019, Liu_2005, Ohno_2005, Peixoto_2017f, Yildiz_2016e, teWildt_2010}, 4 cases (6\%) were psychiatric inpatients \cite{DivsalarP._2023a, fjbuilsRepeatedBehaviorDeliberate2024, teWildt_2010}, 3 cases (4\%) were under the influence of alcohol \cite{Benoist_2019e, Csaky_1998e, Thapa_2019f}, 2 cases (3\%) were displaced people \cite{Akay_2015f, Gardner_2017h}. \paragraph*{Motivation} 34 cases (47\%) had a psychiatric motivation \cite{Al-Faham_2020k, Alao_2006i, Ali_2020f, Apikotoa_2022f, Ataya_2013, Atayan_2016, Bhasin_2014, Bhattacharjee_2008, DelgadoSalazar_2020c, DivsalarP._2023a, Emamhadi_2018, Farhadi_2024h, Guinan_2019f, Hardy_2023g, Jehangir_2019h, Jin_2023, Kar_2015, Kariholu_2008, Kerestes_2019, Kobiela_2015, Kumar_2001, Kumar_2019f, Li_2013, Liu_2005, Misra_2013, Ohno_2005, Sakellaridis_2008f, Sultan_2024f, Tammana_2012j, Tanrikulu_2015e, Yasin_2009, teWildt_2010}, 21 cases (29\%) were motivated by self-harm intention \cite{Al-Faham_2020k, AlShaaibi_2021b, Alao_2006i, Ali_2017, CamachoDorado_2018, Chang_2017f, Cox_2007, Csaky_1998e, Fry_2010, Li_2013, Losanoff_1996, Losanoff_1997e, Mesfin_2022a, Sakellaridis_2008f, Tammana_2012j, Tanrikulu_2015e, fjbuilsRepeatedBehaviorDeliberate2024}, 17 cases (24\%) had a psychosocial motivation \cite{Akay_2015f, Benoist_2019e, Bhattacharjee_2008, Cauchi_2002, Goldman_1998f, Hardy_2023g, Kobiela_2015, Li_2013, Naji_2012f, Qureshi_2016, Riva_2018j, Sobnach_2011f, Tay_2004, Thapa_2019f, Tupesis_2004f, Wildhaber_2005, Wnęk_2015f}, 9 cases (12\%) were motivated by protest \cite{Bhumi_2024f, Gardner_2017h, Losanoff_1996, Losanoff_1997e, Tupesis_2004f}, 9 cases (12\%) had another documented motivation \cite{Ali_2020f, Ali_2022g, Emamhadi_2018, Guinan_2019f, Peixoto_2017f, Sakellaridis_2008f, Trgo_2012f, Wadhwa_2015e, Yildiz_2016e}. \paragraph*{Object Characteristics} 51 cases (71\%) ingested a large diameter object (\textgreater{}2.5cm) \cite{Akay_2015f, Al-Faham_2020k, AlShaaibi_2021b, Alao_2006i, Ali_2017, Ali_2022g, Apikotoa_2022f, Atayan_2016, Berry_2021e, Bhasin_2014, CamachoDorado_2018, Cauchi_2002, Chang_2017f, Cox_2007, Csaky_1998e, DivsalarP._2023a, Emamhadi_2018, Gardner_2017h, Guinan_2019f, Jehangir_2019h, Jin_2023, Kariholu_2008, Kerestes_2019, Kobiela_2015, Kumar_2001, Kumar_2019f, Losanoff_1996, Losanoff_1997e, Mesfin_2022a, Misra_2013, Naji_2012f, Ohno_2005, Peixoto_2017f, Qureshi_2016, Riva_2018j, Sakellaridis_2008f, Sultan_2024f, Tanrikulu_2015e, Thapa_2019f, Trgo_2012f, Wnęk_2015f, Yildiz_2016e, fjbuilsRepeatedBehaviorDeliberate2024, teWildt_2010}, 44 cases (61\%) ingested multiple objects \cite{Ali_2020f, Apikotoa_2022f, Ataya_2013, Atayan_2016, Beecroft_1998, Bhattacharjee_2008, Bhumi_2024f, CamachoDorado_2018, Cauchi_2002, Emamhadi_2018, Farhadi_2024h, Fry_2010, Goldman_1998f, Guinan_2019f, Hardy_2023g, Jehangir_2019h, Jin_2023, Kar_2015, Kariholu_2008, Kobiela_2015, Kumar_2001, Kumar_2019f, Li_2013, Liu_2005, Losanoff_1996, Mesfin_2022a, Misra_2013, Naji_2012f, Ohno_2005, Sobnach_2011f, Sultan_2024f, Tammana_2012j, Tanrikulu_2015e, Tay_2004, Thapa_2019f, Wadhwa_2015e, Wildhaber_2005, Yasin_2009, fjbuilsRepeatedBehaviorDeliberate2024, teWildt_2010}, 34 cases (47\%) ingested a sharp object \cite{AlShaaibi_2021b, Alao_2006i, Apikotoa_2022f, Ataya_2013, Benoist_2019e, Bhasin_2014, Bhattacharjee_2008, CamachoDorado_2018, Csaky_1998e, DelgadoSalazar_2020c, DivsalarP._2023a, Emamhadi_2018, Farhadi_2024h, Fry_2010, Guinan_2019f, Hardy_2023g, Jehangir_2019h, Jin_2023, Kariholu_2008, Kobiela_2015, Kumar_2019f, Losanoff_1996, Losanoff_1997e, Mesfin_2022a, Misra_2013, Sobnach_2011f, Yasin_2009, teWildt_2010}, 32 cases (44\%) ingested a long object (\textgreater{}5cm) \cite{Al-Faham_2020k, AlShaaibi_2021b, Ali_2017, Ali_2022g, Atayan_2016, Bhasin_2014, CamachoDorado_2018, Chang_2017f, Cox_2007, Csaky_1998e, DivsalarP._2023a, Emamhadi_2018, Fry_2010, Gardner_2017h, Jin_2023, Kariholu_2008, Kerestes_2019, Kobiela_2015, Kumar_2019f, Mesfin_2022a, Misra_2013, Ohno_2005, Qureshi_2016, Sakellaridis_2008f, Sultan_2024f, Thapa_2019f, Trgo_2012f, Yasin_2009, Yildiz_2016e, teWildt_2010}, 9 cases (12\%) ingested a magnet \cite{Ali_2020f, Bhumi_2024f, Cauchi_2002, Liu_2005, Naji_2012f, Ohno_2005, Tanrikulu_2015e, Tay_2004, Wildhaber_2005}, 2 cases (3\%) ingested a button battery \cite{Berry_2021e, Bhumi_2024f}. \paragraph*{Outcomes} 48 cases (67\%) experienced a complication \cite{Ali_2017, Ali_2020f, Apikotoa_2022f, Atayan_2016, Beecroft_1998, Benoist_2019e, Berry_2021e, Bhasin_2014, Bhumi_2024f, CamachoDorado_2018, Cauchi_2002, Cox_2007, Csaky_1998e, DelgadoSalazar_2020c, DivsalarP._2023a, Emamhadi_2018, Farhadi_2024h, Fry_2010, Gardner_2017h, Goldman_1998f, Jin_2023, Kariholu_2008, Kerestes_2019, Kobiela_2015, Kumar_2001, Kumar_2019f, Liu_2005, Losanoff_1996, Mesfin_2022a, Misra_2013, Naji_2012f, Ohno_2005, Sakellaridis_2008f, Sobnach_2011f, Sultan_2024f, Tanrikulu_2015e, Tay_2004, Thapa_2019f, Trgo_2012f, Tupesis_2004f, Wildhaber_2005, Wnęk_2015f, Yasin_2009, Yildiz_2016e}, 44 cases (61\%) underwent surgery \cite{Al-Faham_2020k, AlShaaibi_2021b, Alao_2006i, Ali_2017, Ali_2020f, Atayan_2016, Beecroft_1998, Bhasin_2014, CamachoDorado_2018, Cauchi_2002, Chang_2017f, Cox_2007, Csaky_1998e, DelgadoSalazar_2020c, DivsalarP._2023a, Farhadi_2024h, Fry_2010, Gardner_2017h, Jin_2023, Kariholu_2008, Kerestes_2019, Kobiela_2015, Kumar_2019f, Liu_2005, Losanoff_1996, Losanoff_1997e, Mesfin_2022a, Misra_2013, Naji_2012f, Sobnach_2011f, Tanrikulu_2015e, Tay_2004, Thapa_2019f, Tupesis_2004f, Wildhaber_2005, Wnęk_2015f, Yasin_2009, Yildiz_2016e, fjbuilsRepeatedBehaviorDeliberate2024}, 31 cases (43\%) underwent endoscopy \cite{Akay_2015f, Ali_2022g, Apikotoa_2022f, Atayan_2016, Benoist_2019e, Berry_2021e, Bhasin_2014, Bhumi_2024f, CamachoDorado_2018, Chang_2017f, DelgadoSalazar_2020c, Gardner_2017h, Guinan_2019f, Hardy_2023g, Jehangir_2019h, Kariholu_2008, Li_2013, Liu_2005, Ohno_2005, Peixoto_2017f, Qureshi_2016, Riva_2018j, Sakellaridis_2008f, Sultan_2024f, Tammana_2012j, Tanrikulu_2015e, Trgo_2012f, Wadhwa_2015e, Wnęk_2015f, teWildt_2010}, 7 cases (10\%) were managed conservatively \cite{Ataya_2013, Bhattacharjee_2008, DivsalarP._2023a, Emamhadi_2018, Goldman_1998f, Kar_2015, Kumar_2001}, 2 cases (3\%) died \cite{Emamhadi_2018, Kumar_2001}. All 90 were male gender. 90 cases (100\%) were detained at the time of ingestion \cite{Elghali_2016, Karp_1991b, Lee_2007}, 88 cases (98\%) were intentional ingestions \cite{Elghali_2016, Karp_1991b, Lee_2007}, 30 cases (33\%) had a psychiatric history documented \cite{Elghali_2016, Karp_1991b, Lee_2007}, 2 cases (2\%) had a history of prior ingestion \cite{Elghali_2016}. No cases were reported for were psychiatric inpatients, were displaced people, were under the influence of alcohol at the time of ingestion, and had a severe disability history.
\paragraph*{Motivation}  70 cases (78\%) reported protest motivation \cite{Elghali_2016, Karp_1991b, Lee_2007}, 12 cases (13\%) reported psychiatric motivation \cite{Karp_1991b}, 6 cases (7\%) reported self-harm motivation \cite{Elghali_2016, Karp_1991b}. No cases were reported for psychosocial motivation and other motivation.
\paragraph*{Object Characteristics}  68 cases (76\%) involved sharp object ingestion \cite{Elghali_2016, Karp_1991b, Lee_2007}, 32 cases (36\%) involved long (\textgreater 5cm) object ingestion \cite{Lee_2007}, 25 cases (28\%) involved ingestion of multiple objects \cite{Elghali_2016, Lee_2007}. No cases were reported for button battery ingestion, magnet ingestion, and involved large diameter (\textgreater 2.5cm) object ingestion.
\paragraph*{Outcomes}  47 cases (52\%) underwent endoscopic intervention \cite{Elghali_2016, Lee_2007}, 29 cases (32\%) were managed conservatively \cite{Elghali_2016, Karp_1991b}, 15 cases (17\%) underwent surgical intervention \cite{Elghali_2016, Karp_1991b, Lee_2007}, 6 cases (7\%) reported complications \cite{Lee_2007}, 1 case (1\%) died \cite{Elghali_2016}.
\paragraph*{Geographical Location}Cases were recorded in 33 countries: 13 cases from USA \cite{Alao_2006i, Ataya_2013, Bhumi_2024f, Fry_2010, Guinan_2019f, Hardy_2023g, Jehangir_2019h, Kerestes_2019, Kumar_2001, Liu_2005, Tammana_2012j, Tay_2004, Tupesis_2004f}; 7 cases from India \cite{Bhasin_2014, Bhattacharjee_2008, Kar_2015, Kariholu_2008, Kumar_2019f, Misra_2013, Wadhwa_2015e} and UK \cite{Beecroft_1998, Berry_2021e, Cauchi_2002, Cox_2007, Gardner_2017h, Qureshi_2016}; 6 cases from Bulgaria \cite{Losanoff_1996, Losanoff_1997e}; 5 cases from Iran \cite{DivsalarP._2023a, Emamhadi_2018, Farhadi_2024h}; 4 cases from Turkey \cite{Akay_2015f, Atayan_2016, Tanrikulu_2015e, Yildiz_2016e}; 2 cases from China \cite{Jin_2023, Li_2013}, Poland \cite{Kobiela_2015, Wnęk_2015f}, and Spain \cite{CamachoDorado_2018, fjbuilsRepeatedBehaviorDeliberate2024}; 1 case from Australia \cite{Apikotoa_2022f}, Bahrain \cite{Ali_2020f}, Croatia \cite{Trgo_2012f}, Ecuador \cite{DelgadoSalazar_2020c}, Egypt \cite{Ali_2022g}, Ethiopia \cite{Mesfin_2022a}, Germany \cite{teWildt_2010}, Greece \cite{Sakellaridis_2008f}, Hungary \cite{Csaky_1998e}, Iraq \cite{Al-Faham_2020k}, Israel \cite{Goldman_1998f}, Italy \cite{Riva_2018j}, Japan \cite{Ohno_2005}, Nepal \cite{Thapa_2019f}, Netherlands \cite{Benoist_2019e}, Oman \cite{AlShaaibi_2021b}, Pakistan \cite{Yasin_2009}, Portugal \cite{Peixoto_2017f}, Qatar \cite{Ali_2017}, Saudi Arabia \cite{Sultan_2024f}, South Africa \cite{Sobnach_2011f}, Sweden \cite{Naji_2012f}, Switzerland \cite{Wildhaber_2005}, and Taiwan \cite{Chang_2017f}. \paragraph*{Gender} 43 cases (60\%) were male \cite{Akay_2015f, Al-Faham_2020k, Alao_2006i, Ali_2017, Ali_2022g, Apikotoa_2022f, Atayan_2016, Benoist_2019e, Berry_2021e, Bhumi_2024f, CamachoDorado_2018, Csaky_1998e, Emamhadi_2018, Farhadi_2024h, Fry_2010, Gardner_2017h, Guinan_2019f, Jehangir_2019h, Jin_2023, Kobiela_2015, Kumar_2001, Kumar_2019f, Liu_2005, Losanoff_1996, Losanoff_1997e, Mesfin_2022a, Misra_2013, Qureshi_2016, Riva_2018j, Sobnach_2011f, Tammana_2012j, Tanrikulu_2015e, Tay_2004, Thapa_2019f, Trgo_2012f, Wadhwa_2015e, Yasin_2009, teWildt_2010}, 28 cases (39\%) were female \cite{AlShaaibi_2021b, Ali_2020f, Ataya_2013, Beecroft_1998, Bhasin_2014, Bhattacharjee_2008, Cauchi_2002, Chang_2017f, Cox_2007, DelgadoSalazar_2020c, DivsalarP._2023a, Goldman_1998f, Hardy_2023g, Kar_2015, Kariholu_2008, Kerestes_2019, Li_2013, Naji_2012f, Ohno_2005, Peixoto_2017f, Sakellaridis_2008f, Sultan_2024f, Tupesis_2004f, Wildhaber_2005, Wnęk_2015f, Yildiz_2016e}, 1 case (1\%) had no gender recorded \cite{fjbuilsRepeatedBehaviorDeliberate2024}. \paragraph*{Age Group} 25 cases (35\%) were between 26 and 40 years of age \cite{Alao_2006i, Ali_2022g, Apikotoa_2022f, Ataya_2013, Benoist_2019e, Bhasin_2014, Chang_2017f, Cox_2007, DelgadoSalazar_2020c, Farhadi_2024h, Fry_2010, Gardner_2017h, Guinan_2019f, Jin_2023, Kumar_2019f, Losanoff_1996, Misra_2013, Qureshi_2016, Riva_2018j, Sakellaridis_2008f, Tammana_2012j, Trgo_2012f, Wnęk_2015f, Yildiz_2016e, fjbuilsRepeatedBehaviorDeliberate2024}, 18 cases (25\%) were between 18 and 25 years of age \cite{Akay_2015f, Ali_2017, Atayan_2016, Bhattacharjee_2008, Csaky_1998e, Kar_2015, Kariholu_2008, Kobiela_2015, Losanoff_1996, Losanoff_1997e, Mesfin_2022a, Peixoto_2017f, Sobnach_2011f, Tupesis_2004f, Yasin_2009}, 13 cases (18\%) were under 18 years of age \cite{AlShaaibi_2021b, Ali_2020f, Cauchi_2002, DivsalarP._2023a, Goldman_1998f, Liu_2005, Naji_2012f, Ohno_2005, Tanrikulu_2015e, Tay_2004, Wildhaber_2005}, 11 cases (15\%) were between 41 and 60 years of age \cite{Al-Faham_2020k, Bhumi_2024f, CamachoDorado_2018, Emamhadi_2018, Hardy_2023g, Jehangir_2019h, Kumar_2001, Sultan_2024f, Thapa_2019f, Wadhwa_2015e, teWildt_2010}, 3 cases (4\%) were over 60 years of age \cite{Beecroft_1998, Kerestes_2019, Li_2013}, 2 cases (3\%) had no age documented \cite{Berry_2021e}. \paragraph*{Population} 36 cases (50\%) had a psychiatric history \cite{AlShaaibi_2021b, Alao_2006i, Ali_2020f, Apikotoa_2022f, Ataya_2013, Atayan_2016, Beecroft_1998, CamachoDorado_2018, Chang_2017f, DelgadoSalazar_2020c, DivsalarP._2023a, Farhadi_2024h, Fry_2010, Guinan_2019f, Hardy_2023g, Jehangir_2019h, Jin_2023, Kar_2015, Kerestes_2019, Kobiela_2015, Kumar_2001, Kumar_2019f, Liu_2005, Mesfin_2022a, Misra_2013, Ohno_2005, Peixoto_2017f, Sakellaridis_2008f, Sultan_2024f, Tammana_2012j, Tanrikulu_2015e, Yildiz_2016e, fjbuilsRepeatedBehaviorDeliberate2024, teWildt_2010}, 19 cases (26\%) had ingested previously \cite{Alao_2006i, Apikotoa_2022f, Berry_2021e, Bhattacharjee_2008, Csaky_1998e, DivsalarP._2023a, Emamhadi_2018, Guinan_2019f, Jehangir_2019h, Jin_2023, Liu_2005, Sakellaridis_2008f, Tanrikulu_2015e, Thapa_2019f, Yildiz_2016e, fjbuilsRepeatedBehaviorDeliberate2024, teWildt_2010}, 12 cases (17\%) were detained persons \cite{Alao_2006i, Ali_2022g, Apikotoa_2022f, Losanoff_1996, Losanoff_1997e, Qureshi_2016, Tammana_2012j, Trgo_2012f}, 7 cases (10\%) were severely disabled \cite{Atayan_2016, Kerestes_2019, Liu_2005, Ohno_2005, Peixoto_2017f, Yildiz_2016e, teWildt_2010}, 4 cases (6\%) were psychiatric inpatients \cite{DivsalarP._2023a, fjbuilsRepeatedBehaviorDeliberate2024, teWildt_2010}, 3 cases (4\%) were under the influence of alcohol \cite{Benoist_2019e, Csaky_1998e, Thapa_2019f}, 2 cases (3\%) were displaced people \cite{Akay_2015f, Gardner_2017h}. \paragraph*{Motivation} 34 cases (47\%) had a psychiatric motivation \cite{Al-Faham_2020k, Alao_2006i, Ali_2020f, Apikotoa_2022f, Ataya_2013, Atayan_2016, Bhasin_2014, Bhattacharjee_2008, DelgadoSalazar_2020c, DivsalarP._2023a, Emamhadi_2018, Farhadi_2024h, Guinan_2019f, Hardy_2023g, Jehangir_2019h, Jin_2023, Kar_2015, Kariholu_2008, Kerestes_2019, Kobiela_2015, Kumar_2001, Kumar_2019f, Li_2013, Liu_2005, Misra_2013, Ohno_2005, Sakellaridis_2008f, Sultan_2024f, Tammana_2012j, Tanrikulu_2015e, Yasin_2009, teWildt_2010}, 21 cases (29\%) were motivated by self-harm intention \cite{Al-Faham_2020k, AlShaaibi_2021b, Alao_2006i, Ali_2017, CamachoDorado_2018, Chang_2017f, Cox_2007, Csaky_1998e, Fry_2010, Li_2013, Losanoff_1996, Losanoff_1997e, Mesfin_2022a, Sakellaridis_2008f, Tammana_2012j, Tanrikulu_2015e, fjbuilsRepeatedBehaviorDeliberate2024}, 17 cases (24\%) had a psychosocial motivation \cite{Akay_2015f, Benoist_2019e, Bhattacharjee_2008, Cauchi_2002, Goldman_1998f, Hardy_2023g, Kobiela_2015, Li_2013, Naji_2012f, Qureshi_2016, Riva_2018j, Sobnach_2011f, Tay_2004, Thapa_2019f, Tupesis_2004f, Wildhaber_2005, Wnęk_2015f}, 9 cases (12\%) were motivated by protest \cite{Bhumi_2024f, Gardner_2017h, Losanoff_1996, Losanoff_1997e, Tupesis_2004f}, 9 cases (12\%) had another documented motivation \cite{Ali_2020f, Ali_2022g, Emamhadi_2018, Guinan_2019f, Peixoto_2017f, Sakellaridis_2008f, Trgo_2012f, Wadhwa_2015e, Yildiz_2016e}. \paragraph*{Object Characteristics} 51 cases (71\%) ingested a large diameter object (\textgreater{}2.5cm) \cite{Akay_2015f, Al-Faham_2020k, AlShaaibi_2021b, Alao_2006i, Ali_2017, Ali_2022g, Apikotoa_2022f, Atayan_2016, Berry_2021e, Bhasin_2014, CamachoDorado_2018, Cauchi_2002, Chang_2017f, Cox_2007, Csaky_1998e, DivsalarP._2023a, Emamhadi_2018, Gardner_2017h, Guinan_2019f, Jehangir_2019h, Jin_2023, Kariholu_2008, Kerestes_2019, Kobiela_2015, Kumar_2001, Kumar_2019f, Losanoff_1996, Losanoff_1997e, Mesfin_2022a, Misra_2013, Naji_2012f, Ohno_2005, Peixoto_2017f, Qureshi_2016, Riva_2018j, Sakellaridis_2008f, Sultan_2024f, Tanrikulu_2015e, Thapa_2019f, Trgo_2012f, Wnęk_2015f, Yildiz_2016e, fjbuilsRepeatedBehaviorDeliberate2024, teWildt_2010}, 44 cases (61\%) ingested multiple objects \cite{Ali_2020f, Apikotoa_2022f, Ataya_2013, Atayan_2016, Beecroft_1998, Bhattacharjee_2008, Bhumi_2024f, CamachoDorado_2018, Cauchi_2002, Emamhadi_2018, Farhadi_2024h, Fry_2010, Goldman_1998f, Guinan_2019f, Hardy_2023g, Jehangir_2019h, Jin_2023, Kar_2015, Kariholu_2008, Kobiela_2015, Kumar_2001, Kumar_2019f, Li_2013, Liu_2005, Losanoff_1996, Mesfin_2022a, Misra_2013, Naji_2012f, Ohno_2005, Sobnach_2011f, Sultan_2024f, Tammana_2012j, Tanrikulu_2015e, Tay_2004, Thapa_2019f, Wadhwa_2015e, Wildhaber_2005, Yasin_2009, fjbuilsRepeatedBehaviorDeliberate2024, teWildt_2010}, 34 cases (47\%) ingested a sharp object \cite{AlShaaibi_2021b, Alao_2006i, Apikotoa_2022f, Ataya_2013, Benoist_2019e, Bhasin_2014, Bhattacharjee_2008, CamachoDorado_2018, Csaky_1998e, DelgadoSalazar_2020c, DivsalarP._2023a, Emamhadi_2018, Farhadi_2024h, Fry_2010, Guinan_2019f, Hardy_2023g, Jehangir_2019h, Jin_2023, Kariholu_2008, Kobiela_2015, Kumar_2019f, Losanoff_1996, Losanoff_1997e, Mesfin_2022a, Misra_2013, Sobnach_2011f, Yasin_2009, teWildt_2010}, 32 cases (44\%) ingested a long object (\textgreater{}5cm) \cite{Al-Faham_2020k, AlShaaibi_2021b, Ali_2017, Ali_2022g, Atayan_2016, Bhasin_2014, CamachoDorado_2018, Chang_2017f, Cox_2007, Csaky_1998e, DivsalarP._2023a, Emamhadi_2018, Fry_2010, Gardner_2017h, Jin_2023, Kariholu_2008, Kerestes_2019, Kobiela_2015, Kumar_2019f, Mesfin_2022a, Misra_2013, Ohno_2005, Qureshi_2016, Sakellaridis_2008f, Sultan_2024f, Thapa_2019f, Trgo_2012f, Yasin_2009, Yildiz_2016e, teWildt_2010}, 9 cases (12\%) ingested a magnet \cite{Ali_2020f, Bhumi_2024f, Cauchi_2002, Liu_2005, Naji_2012f, Ohno_2005, Tanrikulu_2015e, Tay_2004, Wildhaber_2005}, 2 cases (3\%) ingested a button battery \cite{Berry_2021e, Bhumi_2024f}. \paragraph*{Outcomes} 48 cases (67\%) experienced a complication \cite{Ali_2017, Ali_2020f, Apikotoa_2022f, Atayan_2016, Beecroft_1998, Benoist_2019e, Berry_2021e, Bhasin_2014, Bhumi_2024f, CamachoDorado_2018, Cauchi_2002, Cox_2007, Csaky_1998e, DelgadoSalazar_2020c, DivsalarP._2023a, Emamhadi_2018, Farhadi_2024h, Fry_2010, Gardner_2017h, Goldman_1998f, Jin_2023, Kariholu_2008, Kerestes_2019, Kobiela_2015, Kumar_2001, Kumar_2019f, Liu_2005, Losanoff_1996, Mesfin_2022a, Misra_2013, Naji_2012f, Ohno_2005, Sakellaridis_2008f, Sobnach_2011f, Sultan_2024f, Tanrikulu_2015e, Tay_2004, Thapa_2019f, Trgo_2012f, Tupesis_2004f, Wildhaber_2005, Wnęk_2015f, Yasin_2009, Yildiz_2016e}, 44 cases (61\%) underwent surgery \cite{Al-Faham_2020k, AlShaaibi_2021b, Alao_2006i, Ali_2017, Ali_2020f, Atayan_2016, Beecroft_1998, Bhasin_2014, CamachoDorado_2018, Cauchi_2002, Chang_2017f, Cox_2007, Csaky_1998e, DelgadoSalazar_2020c, DivsalarP._2023a, Farhadi_2024h, Fry_2010, Gardner_2017h, Jin_2023, Kariholu_2008, Kerestes_2019, Kobiela_2015, Kumar_2019f, Liu_2005, Losanoff_1996, Losanoff_1997e, Mesfin_2022a, Misra_2013, Naji_2012f, Sobnach_2011f, Tanrikulu_2015e, Tay_2004, Thapa_2019f, Tupesis_2004f, Wildhaber_2005, Wnęk_2015f, Yasin_2009, Yildiz_2016e, fjbuilsRepeatedBehaviorDeliberate2024}, 31 cases (43\%) underwent endoscopy \cite{Akay_2015f, Ali_2022g, Apikotoa_2022f, Atayan_2016, Benoist_2019e, Berry_2021e, Bhasin_2014, Bhumi_2024f, CamachoDorado_2018, Chang_2017f, DelgadoSalazar_2020c, Gardner_2017h, Guinan_2019f, Hardy_2023g, Jehangir_2019h, Kariholu_2008, Li_2013, Liu_2005, Ohno_2005, Peixoto_2017f, Qureshi_2016, Riva_2018j, Sakellaridis_2008f, Sultan_2024f, Tammana_2012j, Tanrikulu_2015e, Trgo_2012f, Wadhwa_2015e, Wnęk_2015f, teWildt_2010}, 7 cases (10\%) were managed conservatively \cite{Ataya_2013, Bhattacharjee_2008, DivsalarP._2023a, Emamhadi_2018, Goldman_1998f, Kar_2015, Kumar_2001}, 2 cases (3\%) died \cite{Emamhadi_2018, Kumar_2001}. All 90 were male gender. 90 cases (100\%) were detained at the time of ingestion \cite{Elghali_2016, Karp_1991b, Lee_2007}, 88 cases (98\%) were intentional ingestions \cite{Elghali_2016, Karp_1991b, Lee_2007}, 30 cases (33\%) had a psychiatric history documented \cite{Elghali_2016, Karp_1991b, Lee_2007}, 2 cases (2\%) had a history of prior ingestion \cite{Elghali_2016}. No cases were reported for were psychiatric inpatients, were displaced people, were under the influence of alcohol at the time of ingestion, and had a severe disability history.
\paragraph*{Motivation}  70 cases (78\%) reported protest motivation \cite{Elghali_2016, Karp_1991b, Lee_2007}, 12 cases (13\%) reported psychiatric motivation \cite{Karp_1991b}, 6 cases (7\%) reported self-harm motivation \cite{Elghali_2016, Karp_1991b}. No cases were reported for psychosocial motivation and other motivation.
\paragraph*{Object Characteristics}  68 cases (76\%) involved sharp object ingestion \cite{Elghali_2016, Karp_1991b, Lee_2007}, 32 cases (36\%) involved long (\textgreater 5cm) object ingestion \cite{Lee_2007}, 25 cases (28\%) involved ingestion of multiple objects \cite{Elghali_2016, Lee_2007}. No cases were reported for button battery ingestion, magnet ingestion, and involved large diameter (\textgreater 2.5cm) object ingestion.
\paragraph*{Outcomes}  47 cases (52\%) underwent endoscopic intervention \cite{Elghali_2016, Lee_2007}, 29 cases (32\%) were managed conservatively \cite{Elghali_2016, Karp_1991b}, 15 cases (17\%) underwent surgical intervention \cite{Elghali_2016, Karp_1991b, Lee_2007}, 6 cases (7\%) reported complications \cite{Lee_2007}, 1 case (1\%) died \cite{Elghali_2016}.
\paragraph*{Geographical Location}Cases were recorded in 33 countries: 13 cases from USA \cite{Alao_2006i, Ataya_2013, Bhumi_2024f, Fry_2010, Guinan_2019f, Hardy_2023g, Jehangir_2019h, Kerestes_2019, Kumar_2001, Liu_2005, Tammana_2012j, Tay_2004, Tupesis_2004f}; 7 cases from India \cite{Bhasin_2014, Bhattacharjee_2008, Kar_2015, Kariholu_2008, Kumar_2019f, Misra_2013, Wadhwa_2015e} and UK \cite{Beecroft_1998, Berry_2021e, Cauchi_2002, Cox_2007, Gardner_2017h, Qureshi_2016}; 6 cases from Bulgaria \cite{Losanoff_1996, Losanoff_1997e}; 5 cases from Iran \cite{DivsalarP._2023a, Emamhadi_2018, Farhadi_2024h}; 4 cases from Turkey \cite{Akay_2015f, Atayan_2016, Tanrikulu_2015e, Yildiz_2016e}; 2 cases from China \cite{Jin_2023, Li_2013}, Poland \cite{Kobiela_2015, Wnęk_2015f}, and Spain \cite{CamachoDorado_2018, fjbuilsRepeatedBehaviorDeliberate2024}; 1 case from Australia \cite{Apikotoa_2022f}, Bahrain \cite{Ali_2020f}, Croatia \cite{Trgo_2012f}, Ecuador \cite{DelgadoSalazar_2020c}, Egypt \cite{Ali_2022g}, Ethiopia \cite{Mesfin_2022a}, Germany \cite{teWildt_2010}, Greece \cite{Sakellaridis_2008f}, Hungary \cite{Csaky_1998e}, Iraq \cite{Al-Faham_2020k}, Israel \cite{Goldman_1998f}, Italy \cite{Riva_2018j}, Japan \cite{Ohno_2005}, Nepal \cite{Thapa_2019f}, Netherlands \cite{Benoist_2019e}, Oman \cite{AlShaaibi_2021b}, Pakistan \cite{Yasin_2009}, Portugal \cite{Peixoto_2017f}, Qatar \cite{Ali_2017}, Saudi Arabia \cite{Sultan_2024f}, South Africa \cite{Sobnach_2011f}, Sweden \cite{Naji_2012f}, Switzerland \cite{Wildhaber_2005}, and Taiwan \cite{Chang_2017f}. \paragraph*{Gender} 43 cases (60\%) were male \cite{Akay_2015f, Al-Faham_2020k, Alao_2006i, Ali_2017, Ali_2022g, Apikotoa_2022f, Atayan_2016, Benoist_2019e, Berry_2021e, Bhumi_2024f, CamachoDorado_2018, Csaky_1998e, Emamhadi_2018, Farhadi_2024h, Fry_2010, Gardner_2017h, Guinan_2019f, Jehangir_2019h, Jin_2023, Kobiela_2015, Kumar_2001, Kumar_2019f, Liu_2005, Losanoff_1996, Losanoff_1997e, Mesfin_2022a, Misra_2013, Qureshi_2016, Riva_2018j, Sobnach_2011f, Tammana_2012j, Tanrikulu_2015e, Tay_2004, Thapa_2019f, Trgo_2012f, Wadhwa_2015e, Yasin_2009, teWildt_2010}, 28 cases (39\%) were female \cite{AlShaaibi_2021b, Ali_2020f, Ataya_2013, Beecroft_1998, Bhasin_2014, Bhattacharjee_2008, Cauchi_2002, Chang_2017f, Cox_2007, DelgadoSalazar_2020c, DivsalarP._2023a, Goldman_1998f, Hardy_2023g, Kar_2015, Kariholu_2008, Kerestes_2019, Li_2013, Naji_2012f, Ohno_2005, Peixoto_2017f, Sakellaridis_2008f, Sultan_2024f, Tupesis_2004f, Wildhaber_2005, Wnęk_2015f, Yildiz_2016e}, 1 case (1\%) had no gender recorded \cite{fjbuilsRepeatedBehaviorDeliberate2024}. \paragraph*{Age Group} 25 cases (35\%) were between 26 and 40 years of age \cite{Alao_2006i, Ali_2022g, Apikotoa_2022f, Ataya_2013, Benoist_2019e, Bhasin_2014, Chang_2017f, Cox_2007, DelgadoSalazar_2020c, Farhadi_2024h, Fry_2010, Gardner_2017h, Guinan_2019f, Jin_2023, Kumar_2019f, Losanoff_1996, Misra_2013, Qureshi_2016, Riva_2018j, Sakellaridis_2008f, Tammana_2012j, Trgo_2012f, Wnęk_2015f, Yildiz_2016e, fjbuilsRepeatedBehaviorDeliberate2024}, 18 cases (25\%) were between 18 and 25 years of age \cite{Akay_2015f, Ali_2017, Atayan_2016, Bhattacharjee_2008, Csaky_1998e, Kar_2015, Kariholu_2008, Kobiela_2015, Losanoff_1996, Losanoff_1997e, Mesfin_2022a, Peixoto_2017f, Sobnach_2011f, Tupesis_2004f, Yasin_2009}, 13 cases (18\%) were under 18 years of age \cite{AlShaaibi_2021b, Ali_2020f, Cauchi_2002, DivsalarP._2023a, Goldman_1998f, Liu_2005, Naji_2012f, Ohno_2005, Tanrikulu_2015e, Tay_2004, Wildhaber_2005}, 11 cases (15\%) were between 41 and 60 years of age \cite{Al-Faham_2020k, Bhumi_2024f, CamachoDorado_2018, Emamhadi_2018, Hardy_2023g, Jehangir_2019h, Kumar_2001, Sultan_2024f, Thapa_2019f, Wadhwa_2015e, teWildt_2010}, 3 cases (4\%) were over 60 years of age \cite{Beecroft_1998, Kerestes_2019, Li_2013}, 2 cases (3\%) had no age documented \cite{Berry_2021e}. \paragraph*{Population} 36 cases (50\%) had a psychiatric history \cite{AlShaaibi_2021b, Alao_2006i, Ali_2020f, Apikotoa_2022f, Ataya_2013, Atayan_2016, Beecroft_1998, CamachoDorado_2018, Chang_2017f, DelgadoSalazar_2020c, DivsalarP._2023a, Farhadi_2024h, Fry_2010, Guinan_2019f, Hardy_2023g, Jehangir_2019h, Jin_2023, Kar_2015, Kerestes_2019, Kobiela_2015, Kumar_2001, Kumar_2019f, Liu_2005, Mesfin_2022a, Misra_2013, Ohno_2005, Peixoto_2017f, Sakellaridis_2008f, Sultan_2024f, Tammana_2012j, Tanrikulu_2015e, Yildiz_2016e, fjbuilsRepeatedBehaviorDeliberate2024, teWildt_2010}, 19 cases (26\%) had ingested previously \cite{Alao_2006i, Apikotoa_2022f, Berry_2021e, Bhattacharjee_2008, Csaky_1998e, DivsalarP._2023a, Emamhadi_2018, Guinan_2019f, Jehangir_2019h, Jin_2023, Liu_2005, Sakellaridis_2008f, Tanrikulu_2015e, Thapa_2019f, Yildiz_2016e, fjbuilsRepeatedBehaviorDeliberate2024, teWildt_2010}, 12 cases (17\%) were detained persons \cite{Alao_2006i, Ali_2022g, Apikotoa_2022f, Losanoff_1996, Losanoff_1997e, Qureshi_2016, Tammana_2012j, Trgo_2012f}, 7 cases (10\%) were severely disabled \cite{Atayan_2016, Kerestes_2019, Liu_2005, Ohno_2005, Peixoto_2017f, Yildiz_2016e, teWildt_2010}, 4 cases (6\%) were psychiatric inpatients \cite{DivsalarP._2023a, fjbuilsRepeatedBehaviorDeliberate2024, teWildt_2010}, 3 cases (4\%) were under the influence of alcohol \cite{Benoist_2019e, Csaky_1998e, Thapa_2019f}, 2 cases (3\%) were displaced people \cite{Akay_2015f, Gardner_2017h}. \paragraph*{Motivation} 34 cases (47\%) had a psychiatric motivation \cite{Al-Faham_2020k, Alao_2006i, Ali_2020f, Apikotoa_2022f, Ataya_2013, Atayan_2016, Bhasin_2014, Bhattacharjee_2008, DelgadoSalazar_2020c, DivsalarP._2023a, Emamhadi_2018, Farhadi_2024h, Guinan_2019f, Hardy_2023g, Jehangir_2019h, Jin_2023, Kar_2015, Kariholu_2008, Kerestes_2019, Kobiela_2015, Kumar_2001, Kumar_2019f, Li_2013, Liu_2005, Misra_2013, Ohno_2005, Sakellaridis_2008f, Sultan_2024f, Tammana_2012j, Tanrikulu_2015e, Yasin_2009, teWildt_2010}, 21 cases (29\%) were motivated by self-harm intention \cite{Al-Faham_2020k, AlShaaibi_2021b, Alao_2006i, Ali_2017, CamachoDorado_2018, Chang_2017f, Cox_2007, Csaky_1998e, Fry_2010, Li_2013, Losanoff_1996, Losanoff_1997e, Mesfin_2022a, Sakellaridis_2008f, Tammana_2012j, Tanrikulu_2015e, fjbuilsRepeatedBehaviorDeliberate2024}, 17 cases (24\%) had a psychosocial motivation \cite{Akay_2015f, Benoist_2019e, Bhattacharjee_2008, Cauchi_2002, Goldman_1998f, Hardy_2023g, Kobiela_2015, Li_2013, Naji_2012f, Qureshi_2016, Riva_2018j, Sobnach_2011f, Tay_2004, Thapa_2019f, Tupesis_2004f, Wildhaber_2005, Wnęk_2015f}, 9 cases (12\%) were motivated by protest \cite{Bhumi_2024f, Gardner_2017h, Losanoff_1996, Losanoff_1997e, Tupesis_2004f}, 9 cases (12\%) had another documented motivation \cite{Ali_2020f, Ali_2022g, Emamhadi_2018, Guinan_2019f, Peixoto_2017f, Sakellaridis_2008f, Trgo_2012f, Wadhwa_2015e, Yildiz_2016e}. \paragraph*{Object Characteristics} 51 cases (71\%) ingested a large diameter object (\textgreater{}2.5cm) \cite{Akay_2015f, Al-Faham_2020k, AlShaaibi_2021b, Alao_2006i, Ali_2017, Ali_2022g, Apikotoa_2022f, Atayan_2016, Berry_2021e, Bhasin_2014, CamachoDorado_2018, Cauchi_2002, Chang_2017f, Cox_2007, Csaky_1998e, DivsalarP._2023a, Emamhadi_2018, Gardner_2017h, Guinan_2019f, Jehangir_2019h, Jin_2023, Kariholu_2008, Kerestes_2019, Kobiela_2015, Kumar_2001, Kumar_2019f, Losanoff_1996, Losanoff_1997e, Mesfin_2022a, Misra_2013, Naji_2012f, Ohno_2005, Peixoto_2017f, Qureshi_2016, Riva_2018j, Sakellaridis_2008f, Sultan_2024f, Tanrikulu_2015e, Thapa_2019f, Trgo_2012f, Wnęk_2015f, Yildiz_2016e, fjbuilsRepeatedBehaviorDeliberate2024, teWildt_2010}, 44 cases (61\%) ingested multiple objects \cite{Ali_2020f, Apikotoa_2022f, Ataya_2013, Atayan_2016, Beecroft_1998, Bhattacharjee_2008, Bhumi_2024f, CamachoDorado_2018, Cauchi_2002, Emamhadi_2018, Farhadi_2024h, Fry_2010, Goldman_1998f, Guinan_2019f, Hardy_2023g, Jehangir_2019h, Jin_2023, Kar_2015, Kariholu_2008, Kobiela_2015, Kumar_2001, Kumar_2019f, Li_2013, Liu_2005, Losanoff_1996, Mesfin_2022a, Misra_2013, Naji_2012f, Ohno_2005, Sobnach_2011f, Sultan_2024f, Tammana_2012j, Tanrikulu_2015e, Tay_2004, Thapa_2019f, Wadhwa_2015e, Wildhaber_2005, Yasin_2009, fjbuilsRepeatedBehaviorDeliberate2024, teWildt_2010}, 34 cases (47\%) ingested a sharp object \cite{AlShaaibi_2021b, Alao_2006i, Apikotoa_2022f, Ataya_2013, Benoist_2019e, Bhasin_2014, Bhattacharjee_2008, CamachoDorado_2018, Csaky_1998e, DelgadoSalazar_2020c, DivsalarP._2023a, Emamhadi_2018, Farhadi_2024h, Fry_2010, Guinan_2019f, Hardy_2023g, Jehangir_2019h, Jin_2023, Kariholu_2008, Kobiela_2015, Kumar_2019f, Losanoff_1996, Losanoff_1997e, Mesfin_2022a, Misra_2013, Sobnach_2011f, Yasin_2009, teWildt_2010}, 32 cases (44\%) ingested a long object (\textgreater{}5cm) \cite{Al-Faham_2020k, AlShaaibi_2021b, Ali_2017, Ali_2022g, Atayan_2016, Bhasin_2014, CamachoDorado_2018, Chang_2017f, Cox_2007, Csaky_1998e, DivsalarP._2023a, Emamhadi_2018, Fry_2010, Gardner_2017h, Jin_2023, Kariholu_2008, Kerestes_2019, Kobiela_2015, Kumar_2019f, Mesfin_2022a, Misra_2013, Ohno_2005, Qureshi_2016, Sakellaridis_2008f, Sultan_2024f, Thapa_2019f, Trgo_2012f, Yasin_2009, Yildiz_2016e, teWildt_2010}, 9 cases (12\%) ingested a magnet \cite{Ali_2020f, Bhumi_2024f, Cauchi_2002, Liu_2005, Naji_2012f, Ohno_2005, Tanrikulu_2015e, Tay_2004, Wildhaber_2005}, 2 cases (3\%) ingested a button battery \cite{Berry_2021e, Bhumi_2024f}. \paragraph*{Outcomes} 48 cases (67\%) experienced a complication \cite{Ali_2017, Ali_2020f, Apikotoa_2022f, Atayan_2016, Beecroft_1998, Benoist_2019e, Berry_2021e, Bhasin_2014, Bhumi_2024f, CamachoDorado_2018, Cauchi_2002, Cox_2007, Csaky_1998e, DelgadoSalazar_2020c, DivsalarP._2023a, Emamhadi_2018, Farhadi_2024h, Fry_2010, Gardner_2017h, Goldman_1998f, Jin_2023, Kariholu_2008, Kerestes_2019, Kobiela_2015, Kumar_2001, Kumar_2019f, Liu_2005, Losanoff_1996, Mesfin_2022a, Misra_2013, Naji_2012f, Ohno_2005, Sakellaridis_2008f, Sobnach_2011f, Sultan_2024f, Tanrikulu_2015e, Tay_2004, Thapa_2019f, Trgo_2012f, Tupesis_2004f, Wildhaber_2005, Wnęk_2015f, Yasin_2009, Yildiz_2016e}, 44 cases (61\%) underwent surgery \cite{Al-Faham_2020k, AlShaaibi_2021b, Alao_2006i, Ali_2017, Ali_2020f, Atayan_2016, Beecroft_1998, Bhasin_2014, CamachoDorado_2018, Cauchi_2002, Chang_2017f, Cox_2007, Csaky_1998e, DelgadoSalazar_2020c, DivsalarP._2023a, Farhadi_2024h, Fry_2010, Gardner_2017h, Jin_2023, Kariholu_2008, Kerestes_2019, Kobiela_2015, Kumar_2019f, Liu_2005, Losanoff_1996, Losanoff_1997e, Mesfin_2022a, Misra_2013, Naji_2012f, Sobnach_2011f, Tanrikulu_2015e, Tay_2004, Thapa_2019f, Tupesis_2004f, Wildhaber_2005, Wnęk_2015f, Yasin_2009, Yildiz_2016e, fjbuilsRepeatedBehaviorDeliberate2024}, 31 cases (43\%) underwent endoscopy \cite{Akay_2015f, Ali_2022g, Apikotoa_2022f, Atayan_2016, Benoist_2019e, Berry_2021e, Bhasin_2014, Bhumi_2024f, CamachoDorado_2018, Chang_2017f, DelgadoSalazar_2020c, Gardner_2017h, Guinan_2019f, Hardy_2023g, Jehangir_2019h, Kariholu_2008, Li_2013, Liu_2005, Ohno_2005, Peixoto_2017f, Qureshi_2016, Riva_2018j, Sakellaridis_2008f, Sultan_2024f, Tammana_2012j, Tanrikulu_2015e, Trgo_2012f, Wadhwa_2015e, Wnęk_2015f, teWildt_2010}, 7 cases (10\%) were managed conservatively \cite{Ataya_2013, Bhattacharjee_2008, DivsalarP._2023a, Emamhadi_2018, Goldman_1998f, Kar_2015, Kumar_2001}, 2 cases (3\%) died \cite{Emamhadi_2018, Kumar_2001}. All 90 were male gender. 90 cases (100\%) were detained at the time of ingestion \cite{Elghali_2016, Karp_1991b, Lee_2007}, 88 cases (98\%) were intentional ingestions \cite{Elghali_2016, Karp_1991b, Lee_2007}, 30 cases (33\%) had a psychiatric history documented \cite{Elghali_2016, Karp_1991b, Lee_2007}, 2 cases (2\%) had a history of prior ingestion \cite{Elghali_2016}. No cases were reported for were psychiatric inpatients, were displaced people, were under the influence of alcohol at the time of ingestion, and had a severe disability history.
\paragraph*{Motivation}  70 cases (78\%) reported protest motivation \cite{Elghali_2016, Karp_1991b, Lee_2007}, 12 cases (13\%) reported psychiatric motivation \cite{Karp_1991b}, 6 cases (7\%) reported self-harm motivation \cite{Elghali_2016, Karp_1991b}. No cases were reported for psychosocial motivation and other motivation.
\paragraph*{Object Characteristics}  68 cases (76\%) involved sharp object ingestion \cite{Elghali_2016, Karp_1991b, Lee_2007}, 32 cases (36\%) involved long (\textgreater 5cm) object ingestion \cite{Lee_2007}, 25 cases (28\%) involved ingestion of multiple objects \cite{Elghali_2016, Lee_2007}. No cases were reported for button battery ingestion, magnet ingestion, and involved large diameter (\textgreater 2.5cm) object ingestion.
\paragraph*{Outcomes}  47 cases (52\%) underwent endoscopic intervention \cite{Elghali_2016, Lee_2007}, 29 cases (32\%) were managed conservatively \cite{Elghali_2016, Karp_1991b}, 15 cases (17\%) underwent surgical intervention \cite{Elghali_2016, Karp_1991b, Lee_2007}, 6 cases (7\%) reported complications \cite{Lee_2007}, 1 case (1\%) died \cite{Elghali_2016}.
\paragraph*{Geographical Location}Cases were recorded in 33 countries: 13 cases from USA \cite{Alao_2006i, Ataya_2013, Bhumi_2024f, Fry_2010, Guinan_2019f, Hardy_2023g, Jehangir_2019h, Kerestes_2019, Kumar_2001, Liu_2005, Tammana_2012j, Tay_2004, Tupesis_2004f}; 7 cases from India \cite{Bhasin_2014, Bhattacharjee_2008, Kar_2015, Kariholu_2008, Kumar_2019f, Misra_2013, Wadhwa_2015e} and UK \cite{Beecroft_1998, Berry_2021e, Cauchi_2002, Cox_2007, Gardner_2017h, Qureshi_2016}; 6 cases from Bulgaria \cite{Losanoff_1996, Losanoff_1997e}; 5 cases from Iran \cite{DivsalarP._2023a, Emamhadi_2018, Farhadi_2024h}; 4 cases from Turkey \cite{Akay_2015f, Atayan_2016, Tanrikulu_2015e, Yildiz_2016e}; 2 cases from China \cite{Jin_2023, Li_2013}, Poland \cite{Kobiela_2015, Wnęk_2015f}, and Spain \cite{CamachoDorado_2018, fjbuilsRepeatedBehaviorDeliberate2024}; 1 case from Australia \cite{Apikotoa_2022f}, Bahrain \cite{Ali_2020f}, Croatia \cite{Trgo_2012f}, Ecuador \cite{DelgadoSalazar_2020c}, Egypt \cite{Ali_2022g}, Ethiopia \cite{Mesfin_2022a}, Germany \cite{teWildt_2010}, Greece \cite{Sakellaridis_2008f}, Hungary \cite{Csaky_1998e}, Iraq \cite{Al-Faham_2020k}, Israel \cite{Goldman_1998f}, Italy \cite{Riva_2018j}, Japan \cite{Ohno_2005}, Nepal \cite{Thapa_2019f}, Netherlands \cite{Benoist_2019e}, Oman \cite{AlShaaibi_2021b}, Pakistan \cite{Yasin_2009}, Portugal \cite{Peixoto_2017f}, Qatar \cite{Ali_2017}, Saudi Arabia \cite{Sultan_2024f}, South Africa \cite{Sobnach_2011f}, Sweden \cite{Naji_2012f}, Switzerland \cite{Wildhaber_2005}, and Taiwan \cite{Chang_2017f}. \paragraph*{Gender} 43 cases (60\%) were male \cite{Akay_2015f, Al-Faham_2020k, Alao_2006i, Ali_2017, Ali_2022g, Apikotoa_2022f, Atayan_2016, Benoist_2019e, Berry_2021e, Bhumi_2024f, CamachoDorado_2018, Csaky_1998e, Emamhadi_2018, Farhadi_2024h, Fry_2010, Gardner_2017h, Guinan_2019f, Jehangir_2019h, Jin_2023, Kobiela_2015, Kumar_2001, Kumar_2019f, Liu_2005, Losanoff_1996, Losanoff_1997e, Mesfin_2022a, Misra_2013, Qureshi_2016, Riva_2018j, Sobnach_2011f, Tammana_2012j, Tanrikulu_2015e, Tay_2004, Thapa_2019f, Trgo_2012f, Wadhwa_2015e, Yasin_2009, teWildt_2010}, 28 cases (39\%) were female \cite{AlShaaibi_2021b, Ali_2020f, Ataya_2013, Beecroft_1998, Bhasin_2014, Bhattacharjee_2008, Cauchi_2002, Chang_2017f, Cox_2007, DelgadoSalazar_2020c, DivsalarP._2023a, Goldman_1998f, Hardy_2023g, Kar_2015, Kariholu_2008, Kerestes_2019, Li_2013, Naji_2012f, Ohno_2005, Peixoto_2017f, Sakellaridis_2008f, Sultan_2024f, Tupesis_2004f, Wildhaber_2005, Wnęk_2015f, Yildiz_2016e}, 1 case (1\%) had no gender recorded \cite{fjbuilsRepeatedBehaviorDeliberate2024}. \paragraph*{Age Group} 25 cases (35\%) were between 26 and 40 years of age \cite{Alao_2006i, Ali_2022g, Apikotoa_2022f, Ataya_2013, Benoist_2019e, Bhasin_2014, Chang_2017f, Cox_2007, DelgadoSalazar_2020c, Farhadi_2024h, Fry_2010, Gardner_2017h, Guinan_2019f, Jin_2023, Kumar_2019f, Losanoff_1996, Misra_2013, Qureshi_2016, Riva_2018j, Sakellaridis_2008f, Tammana_2012j, Trgo_2012f, Wnęk_2015f, Yildiz_2016e, fjbuilsRepeatedBehaviorDeliberate2024}, 18 cases (25\%) were between 18 and 25 years of age \cite{Akay_2015f, Ali_2017, Atayan_2016, Bhattacharjee_2008, Csaky_1998e, Kar_2015, Kariholu_2008, Kobiela_2015, Losanoff_1996, Losanoff_1997e, Mesfin_2022a, Peixoto_2017f, Sobnach_2011f, Tupesis_2004f, Yasin_2009}, 13 cases (18\%) were under 18 years of age \cite{AlShaaibi_2021b, Ali_2020f, Cauchi_2002, DivsalarP._2023a, Goldman_1998f, Liu_2005, Naji_2012f, Ohno_2005, Tanrikulu_2015e, Tay_2004, Wildhaber_2005}, 11 cases (15\%) were between 41 and 60 years of age \cite{Al-Faham_2020k, Bhumi_2024f, CamachoDorado_2018, Emamhadi_2018, Hardy_2023g, Jehangir_2019h, Kumar_2001, Sultan_2024f, Thapa_2019f, Wadhwa_2015e, teWildt_2010}, 3 cases (4\%) were over 60 years of age \cite{Beecroft_1998, Kerestes_2019, Li_2013}, 2 cases (3\%) had no age documented \cite{Berry_2021e}. \paragraph*{Population} 36 cases (50\%) had a psychiatric history \cite{AlShaaibi_2021b, Alao_2006i, Ali_2020f, Apikotoa_2022f, Ataya_2013, Atayan_2016, Beecroft_1998, CamachoDorado_2018, Chang_2017f, DelgadoSalazar_2020c, DivsalarP._2023a, Farhadi_2024h, Fry_2010, Guinan_2019f, Hardy_2023g, Jehangir_2019h, Jin_2023, Kar_2015, Kerestes_2019, Kobiela_2015, Kumar_2001, Kumar_2019f, Liu_2005, Mesfin_2022a, Misra_2013, Ohno_2005, Peixoto_2017f, Sakellaridis_2008f, Sultan_2024f, Tammana_2012j, Tanrikulu_2015e, Yildiz_2016e, fjbuilsRepeatedBehaviorDeliberate2024, teWildt_2010}, 19 cases (26\%) had ingested previously \cite{Alao_2006i, Apikotoa_2022f, Berry_2021e, Bhattacharjee_2008, Csaky_1998e, DivsalarP._2023a, Emamhadi_2018, Guinan_2019f, Jehangir_2019h, Jin_2023, Liu_2005, Sakellaridis_2008f, Tanrikulu_2015e, Thapa_2019f, Yildiz_2016e, fjbuilsRepeatedBehaviorDeliberate2024, teWildt_2010}, 12 cases (17\%) were detained persons \cite{Alao_2006i, Ali_2022g, Apikotoa_2022f, Losanoff_1996, Losanoff_1997e, Qureshi_2016, Tammana_2012j, Trgo_2012f}, 7 cases (10\%) were severely disabled \cite{Atayan_2016, Kerestes_2019, Liu_2005, Ohno_2005, Peixoto_2017f, Yildiz_2016e, teWildt_2010}, 4 cases (6\%) were psychiatric inpatients \cite{DivsalarP._2023a, fjbuilsRepeatedBehaviorDeliberate2024, teWildt_2010}, 3 cases (4\%) were under the influence of alcohol \cite{Benoist_2019e, Csaky_1998e, Thapa_2019f}, 2 cases (3\%) were displaced people \cite{Akay_2015f, Gardner_2017h}. \paragraph*{Motivation} 34 cases (47\%) had a psychiatric motivation \cite{Al-Faham_2020k, Alao_2006i, Ali_2020f, Apikotoa_2022f, Ataya_2013, Atayan_2016, Bhasin_2014, Bhattacharjee_2008, DelgadoSalazar_2020c, DivsalarP._2023a, Emamhadi_2018, Farhadi_2024h, Guinan_2019f, Hardy_2023g, Jehangir_2019h, Jin_2023, Kar_2015, Kariholu_2008, Kerestes_2019, Kobiela_2015, Kumar_2001, Kumar_2019f, Li_2013, Liu_2005, Misra_2013, Ohno_2005, Sakellaridis_2008f, Sultan_2024f, Tammana_2012j, Tanrikulu_2015e, Yasin_2009, teWildt_2010}, 21 cases (29\%) were motivated by self-harm intention \cite{Al-Faham_2020k, AlShaaibi_2021b, Alao_2006i, Ali_2017, CamachoDorado_2018, Chang_2017f, Cox_2007, Csaky_1998e, Fry_2010, Li_2013, Losanoff_1996, Losanoff_1997e, Mesfin_2022a, Sakellaridis_2008f, Tammana_2012j, Tanrikulu_2015e, fjbuilsRepeatedBehaviorDeliberate2024}, 17 cases (24\%) had a psychosocial motivation \cite{Akay_2015f, Benoist_2019e, Bhattacharjee_2008, Cauchi_2002, Goldman_1998f, Hardy_2023g, Kobiela_2015, Li_2013, Naji_2012f, Qureshi_2016, Riva_2018j, Sobnach_2011f, Tay_2004, Thapa_2019f, Tupesis_2004f, Wildhaber_2005, Wnęk_2015f}, 9 cases (12\%) were motivated by protest \cite{Bhumi_2024f, Gardner_2017h, Losanoff_1996, Losanoff_1997e, Tupesis_2004f}, 9 cases (12\%) had another documented motivation \cite{Ali_2020f, Ali_2022g, Emamhadi_2018, Guinan_2019f, Peixoto_2017f, Sakellaridis_2008f, Trgo_2012f, Wadhwa_2015e, Yildiz_2016e}. \paragraph*{Object Characteristics} 51 cases (71\%) ingested a large diameter object (\textgreater{}2.5cm) \cite{Akay_2015f, Al-Faham_2020k, AlShaaibi_2021b, Alao_2006i, Ali_2017, Ali_2022g, Apikotoa_2022f, Atayan_2016, Berry_2021e, Bhasin_2014, CamachoDorado_2018, Cauchi_2002, Chang_2017f, Cox_2007, Csaky_1998e, DivsalarP._2023a, Emamhadi_2018, Gardner_2017h, Guinan_2019f, Jehangir_2019h, Jin_2023, Kariholu_2008, Kerestes_2019, Kobiela_2015, Kumar_2001, Kumar_2019f, Losanoff_1996, Losanoff_1997e, Mesfin_2022a, Misra_2013, Naji_2012f, Ohno_2005, Peixoto_2017f, Qureshi_2016, Riva_2018j, Sakellaridis_2008f, Sultan_2024f, Tanrikulu_2015e, Thapa_2019f, Trgo_2012f, Wnęk_2015f, Yildiz_2016e, fjbuilsRepeatedBehaviorDeliberate2024, teWildt_2010}, 44 cases (61\%) ingested multiple objects \cite{Ali_2020f, Apikotoa_2022f, Ataya_2013, Atayan_2016, Beecroft_1998, Bhattacharjee_2008, Bhumi_2024f, CamachoDorado_2018, Cauchi_2002, Emamhadi_2018, Farhadi_2024h, Fry_2010, Goldman_1998f, Guinan_2019f, Hardy_2023g, Jehangir_2019h, Jin_2023, Kar_2015, Kariholu_2008, Kobiela_2015, Kumar_2001, Kumar_2019f, Li_2013, Liu_2005, Losanoff_1996, Mesfin_2022a, Misra_2013, Naji_2012f, Ohno_2005, Sobnach_2011f, Sultan_2024f, Tammana_2012j, Tanrikulu_2015e, Tay_2004, Thapa_2019f, Wadhwa_2015e, Wildhaber_2005, Yasin_2009, fjbuilsRepeatedBehaviorDeliberate2024, teWildt_2010}, 34 cases (47\%) ingested a sharp object \cite{AlShaaibi_2021b, Alao_2006i, Apikotoa_2022f, Ataya_2013, Benoist_2019e, Bhasin_2014, Bhattacharjee_2008, CamachoDorado_2018, Csaky_1998e, DelgadoSalazar_2020c, DivsalarP._2023a, Emamhadi_2018, Farhadi_2024h, Fry_2010, Guinan_2019f, Hardy_2023g, Jehangir_2019h, Jin_2023, Kariholu_2008, Kobiela_2015, Kumar_2019f, Losanoff_1996, Losanoff_1997e, Mesfin_2022a, Misra_2013, Sobnach_2011f, Yasin_2009, teWildt_2010}, 32 cases (44\%) ingested a long object (\textgreater{}5cm) \cite{Al-Faham_2020k, AlShaaibi_2021b, Ali_2017, Ali_2022g, Atayan_2016, Bhasin_2014, CamachoDorado_2018, Chang_2017f, Cox_2007, Csaky_1998e, DivsalarP._2023a, Emamhadi_2018, Fry_2010, Gardner_2017h, Jin_2023, Kariholu_2008, Kerestes_2019, Kobiela_2015, Kumar_2019f, Mesfin_2022a, Misra_2013, Ohno_2005, Qureshi_2016, Sakellaridis_2008f, Sultan_2024f, Thapa_2019f, Trgo_2012f, Yasin_2009, Yildiz_2016e, teWildt_2010}, 9 cases (12\%) ingested a magnet \cite{Ali_2020f, Bhumi_2024f, Cauchi_2002, Liu_2005, Naji_2012f, Ohno_2005, Tanrikulu_2015e, Tay_2004, Wildhaber_2005}, 2 cases (3\%) ingested a button battery \cite{Berry_2021e, Bhumi_2024f}. \paragraph*{Outcomes} 48 cases (67\%) experienced a complication \cite{Ali_2017, Ali_2020f, Apikotoa_2022f, Atayan_2016, Beecroft_1998, Benoist_2019e, Berry_2021e, Bhasin_2014, Bhumi_2024f, CamachoDorado_2018, Cauchi_2002, Cox_2007, Csaky_1998e, DelgadoSalazar_2020c, DivsalarP._2023a, Emamhadi_2018, Farhadi_2024h, Fry_2010, Gardner_2017h, Goldman_1998f, Jin_2023, Kariholu_2008, Kerestes_2019, Kobiela_2015, Kumar_2001, Kumar_2019f, Liu_2005, Losanoff_1996, Mesfin_2022a, Misra_2013, Naji_2012f, Ohno_2005, Sakellaridis_2008f, Sobnach_2011f, Sultan_2024f, Tanrikulu_2015e, Tay_2004, Thapa_2019f, Trgo_2012f, Tupesis_2004f, Wildhaber_2005, Wnęk_2015f, Yasin_2009, Yildiz_2016e}, 44 cases (61\%) underwent surgery \cite{Al-Faham_2020k, AlShaaibi_2021b, Alao_2006i, Ali_2017, Ali_2020f, Atayan_2016, Beecroft_1998, Bhasin_2014, CamachoDorado_2018, Cauchi_2002, Chang_2017f, Cox_2007, Csaky_1998e, DelgadoSalazar_2020c, DivsalarP._2023a, Farhadi_2024h, Fry_2010, Gardner_2017h, Jin_2023, Kariholu_2008, Kerestes_2019, Kobiela_2015, Kumar_2019f, Liu_2005, Losanoff_1996, Losanoff_1997e, Mesfin_2022a, Misra_2013, Naji_2012f, Sobnach_2011f, Tanrikulu_2015e, Tay_2004, Thapa_2019f, Tupesis_2004f, Wildhaber_2005, Wnęk_2015f, Yasin_2009, Yildiz_2016e, fjbuilsRepeatedBehaviorDeliberate2024}, 31 cases (43\%) underwent endoscopy \cite{Akay_2015f, Ali_2022g, Apikotoa_2022f, Atayan_2016, Benoist_2019e, Berry_2021e, Bhasin_2014, Bhumi_2024f, CamachoDorado_2018, Chang_2017f, DelgadoSalazar_2020c, Gardner_2017h, Guinan_2019f, Hardy_2023g, Jehangir_2019h, Kariholu_2008, Li_2013, Liu_2005, Ohno_2005, Peixoto_2017f, Qureshi_2016, Riva_2018j, Sakellaridis_2008f, Sultan_2024f, Tammana_2012j, Tanrikulu_2015e, Trgo_2012f, Wadhwa_2015e, Wnęk_2015f, teWildt_2010}, 7 cases (10\%) were managed conservatively \cite{Ataya_2013, Bhattacharjee_2008, DivsalarP._2023a, Emamhadi_2018, Goldman_1998f, Kar_2015, Kumar_2001}, 2 cases (3\%) died \cite{Emamhadi_2018, Kumar_2001}. All 90 were male gender. 90 cases (100\%) were detained at the time of ingestion \cite{Elghali_2016, Karp_1991b, Lee_2007}, 88 cases (98\%) were intentional ingestions \cite{Elghali_2016, Karp_1991b, Lee_2007}, 30 cases (33\%) had a psychiatric history documented \cite{Elghali_2016, Karp_1991b, Lee_2007}, 2 cases (2\%) had a history of prior ingestion \cite{Elghali_2016}. No cases were reported for were psychiatric inpatients, were displaced people, were under the influence of alcohol at the time of ingestion, and had a severe disability history.
\paragraph*{Motivation}  70 cases (78\%) reported protest motivation \cite{Elghali_2016, Karp_1991b, Lee_2007}, 12 cases (13\%) reported psychiatric motivation \cite{Karp_1991b}, 6 cases (7\%) reported self-harm motivation \cite{Elghali_2016, Karp_1991b}. No cases were reported for psychosocial motivation and other motivation.
\paragraph*{Object Characteristics}  68 cases (76\%) involved sharp object ingestion \cite{Elghali_2016, Karp_1991b, Lee_2007}, 32 cases (36\%) involved long (\textgreater 5cm) object ingestion \cite{Lee_2007}, 25 cases (28\%) involved ingestion of multiple objects \cite{Elghali_2016, Lee_2007}. No cases were reported for button battery ingestion, magnet ingestion, and involved large diameter (\textgreater 2.5cm) object ingestion.
\paragraph*{Outcomes}  47 cases (52\%) underwent endoscopic intervention \cite{Elghali_2016, Lee_2007}, 29 cases (32\%) were managed conservatively \cite{Elghali_2016, Karp_1991b}, 15 cases (17\%) underwent surgical intervention \cite{Elghali_2016, Karp_1991b, Lee_2007}, 6 cases (7\%) reported complications \cite{Lee_2007}, 1 case (1\%) died \cite{Elghali_2016}.
\paragraph*{Geographical Location}Cases were recorded in 33 countries: 13 cases from USA \cite{Alao_2006i, Ataya_2013, Bhumi_2024f, Fry_2010, Guinan_2019f, Hardy_2023g, Jehangir_2019h, Kerestes_2019, Kumar_2001, Liu_2005, Tammana_2012j, Tay_2004, Tupesis_2004f}; 7 cases from India \cite{Bhasin_2014, Bhattacharjee_2008, Kar_2015, Kariholu_2008, Kumar_2019f, Misra_2013, Wadhwa_2015e} and UK \cite{Beecroft_1998, Berry_2021e, Cauchi_2002, Cox_2007, Gardner_2017h, Qureshi_2016}; 6 cases from Bulgaria \cite{Losanoff_1996, Losanoff_1997e}; 5 cases from Iran \cite{DivsalarP._2023a, Emamhadi_2018, Farhadi_2024h}; 4 cases from Turkey \cite{Akay_2015f, Atayan_2016, Tanrikulu_2015e, Yildiz_2016e}; 2 cases from China \cite{Jin_2023, Li_2013}, Poland \cite{Kobiela_2015, Wnęk_2015f}, and Spain \cite{CamachoDorado_2018, fjbuilsRepeatedBehaviorDeliberate2024}; 1 case from Australia \cite{Apikotoa_2022f}, Bahrain \cite{Ali_2020f}, Croatia \cite{Trgo_2012f}, Ecuador \cite{DelgadoSalazar_2020c}, Egypt \cite{Ali_2022g}, Ethiopia \cite{Mesfin_2022a}, Germany \cite{teWildt_2010}, Greece \cite{Sakellaridis_2008f}, Hungary \cite{Csaky_1998e}, Iraq \cite{Al-Faham_2020k}, Israel \cite{Goldman_1998f}, Italy \cite{Riva_2018j}, Japan \cite{Ohno_2005}, Nepal \cite{Thapa_2019f}, Netherlands \cite{Benoist_2019e}, Oman \cite{AlShaaibi_2021b}, Pakistan \cite{Yasin_2009}, Portugal \cite{Peixoto_2017f}, Qatar \cite{Ali_2017}, Saudi Arabia \cite{Sultan_2024f}, South Africa \cite{Sobnach_2011f}, Sweden \cite{Naji_2012f}, Switzerland \cite{Wildhaber_2005}, and Taiwan \cite{Chang_2017f}. \paragraph*{Gender} 43 cases (60\%) were male \cite{Akay_2015f, Al-Faham_2020k, Alao_2006i, Ali_2017, Ali_2022g, Apikotoa_2022f, Atayan_2016, Benoist_2019e, Berry_2021e, Bhumi_2024f, CamachoDorado_2018, Csaky_1998e, Emamhadi_2018, Farhadi_2024h, Fry_2010, Gardner_2017h, Guinan_2019f, Jehangir_2019h, Jin_2023, Kobiela_2015, Kumar_2001, Kumar_2019f, Liu_2005, Losanoff_1996, Losanoff_1997e, Mesfin_2022a, Misra_2013, Qureshi_2016, Riva_2018j, Sobnach_2011f, Tammana_2012j, Tanrikulu_2015e, Tay_2004, Thapa_2019f, Trgo_2012f, Wadhwa_2015e, Yasin_2009, teWildt_2010}, 28 cases (39\%) were female \cite{AlShaaibi_2021b, Ali_2020f, Ataya_2013, Beecroft_1998, Bhasin_2014, Bhattacharjee_2008, Cauchi_2002, Chang_2017f, Cox_2007, DelgadoSalazar_2020c, DivsalarP._2023a, Goldman_1998f, Hardy_2023g, Kar_2015, Kariholu_2008, Kerestes_2019, Li_2013, Naji_2012f, Ohno_2005, Peixoto_2017f, Sakellaridis_2008f, Sultan_2024f, Tupesis_2004f, Wildhaber_2005, Wnęk_2015f, Yildiz_2016e}, 1 case (1\%) had no gender recorded \cite{fjbuilsRepeatedBehaviorDeliberate2024}. \paragraph*{Age Group} 25 cases (35\%) were between 26 and 40 years of age \cite{Alao_2006i, Ali_2022g, Apikotoa_2022f, Ataya_2013, Benoist_2019e, Bhasin_2014, Chang_2017f, Cox_2007, DelgadoSalazar_2020c, Farhadi_2024h, Fry_2010, Gardner_2017h, Guinan_2019f, Jin_2023, Kumar_2019f, Losanoff_1996, Misra_2013, Qureshi_2016, Riva_2018j, Sakellaridis_2008f, Tammana_2012j, Trgo_2012f, Wnęk_2015f, Yildiz_2016e, fjbuilsRepeatedBehaviorDeliberate2024}, 18 cases (25\%) were between 18 and 25 years of age \cite{Akay_2015f, Ali_2017, Atayan_2016, Bhattacharjee_2008, Csaky_1998e, Kar_2015, Kariholu_2008, Kobiela_2015, Losanoff_1996, Losanoff_1997e, Mesfin_2022a, Peixoto_2017f, Sobnach_2011f, Tupesis_2004f, Yasin_2009}, 13 cases (18\%) were under 18 years of age \cite{AlShaaibi_2021b, Ali_2020f, Cauchi_2002, DivsalarP._2023a, Goldman_1998f, Liu_2005, Naji_2012f, Ohno_2005, Tanrikulu_2015e, Tay_2004, Wildhaber_2005}, 11 cases (15\%) were between 41 and 60 years of age \cite{Al-Faham_2020k, Bhumi_2024f, CamachoDorado_2018, Emamhadi_2018, Hardy_2023g, Jehangir_2019h, Kumar_2001, Sultan_2024f, Thapa_2019f, Wadhwa_2015e, teWildt_2010}, 3 cases (4\%) were over 60 years of age \cite{Beecroft_1998, Kerestes_2019, Li_2013}, 2 cases (3\%) had no age documented \cite{Berry_2021e}. \paragraph*{Population} 36 cases (50\%) had a psychiatric history \cite{AlShaaibi_2021b, Alao_2006i, Ali_2020f, Apikotoa_2022f, Ataya_2013, Atayan_2016, Beecroft_1998, CamachoDorado_2018, Chang_2017f, DelgadoSalazar_2020c, DivsalarP._2023a, Farhadi_2024h, Fry_2010, Guinan_2019f, Hardy_2023g, Jehangir_2019h, Jin_2023, Kar_2015, Kerestes_2019, Kobiela_2015, Kumar_2001, Kumar_2019f, Liu_2005, Mesfin_2022a, Misra_2013, Ohno_2005, Peixoto_2017f, Sakellaridis_2008f, Sultan_2024f, Tammana_2012j, Tanrikulu_2015e, Yildiz_2016e, fjbuilsRepeatedBehaviorDeliberate2024, teWildt_2010}, 19 cases (26\%) had ingested previously \cite{Alao_2006i, Apikotoa_2022f, Berry_2021e, Bhattacharjee_2008, Csaky_1998e, DivsalarP._2023a, Emamhadi_2018, Guinan_2019f, Jehangir_2019h, Jin_2023, Liu_2005, Sakellaridis_2008f, Tanrikulu_2015e, Thapa_2019f, Yildiz_2016e, fjbuilsRepeatedBehaviorDeliberate2024, teWildt_2010}, 12 cases (17\%) were detained persons \cite{Alao_2006i, Ali_2022g, Apikotoa_2022f, Losanoff_1996, Losanoff_1997e, Qureshi_2016, Tammana_2012j, Trgo_2012f}, 7 cases (10\%) were severely disabled \cite{Atayan_2016, Kerestes_2019, Liu_2005, Ohno_2005, Peixoto_2017f, Yildiz_2016e, teWildt_2010}, 4 cases (6\%) were psychiatric inpatients \cite{DivsalarP._2023a, fjbuilsRepeatedBehaviorDeliberate2024, teWildt_2010}, 3 cases (4\%) were under the influence of alcohol \cite{Benoist_2019e, Csaky_1998e, Thapa_2019f}, 2 cases (3\%) were displaced people \cite{Akay_2015f, Gardner_2017h}. \paragraph*{Motivation} 34 cases (47\%) had a psychiatric motivation \cite{Al-Faham_2020k, Alao_2006i, Ali_2020f, Apikotoa_2022f, Ataya_2013, Atayan_2016, Bhasin_2014, Bhattacharjee_2008, DelgadoSalazar_2020c, DivsalarP._2023a, Emamhadi_2018, Farhadi_2024h, Guinan_2019f, Hardy_2023g, Jehangir_2019h, Jin_2023, Kar_2015, Kariholu_2008, Kerestes_2019, Kobiela_2015, Kumar_2001, Kumar_2019f, Li_2013, Liu_2005, Misra_2013, Ohno_2005, Sakellaridis_2008f, Sultan_2024f, Tammana_2012j, Tanrikulu_2015e, Yasin_2009, teWildt_2010}, 21 cases (29\%) were motivated by self-harm intention \cite{Al-Faham_2020k, AlShaaibi_2021b, Alao_2006i, Ali_2017, CamachoDorado_2018, Chang_2017f, Cox_2007, Csaky_1998e, Fry_2010, Li_2013, Losanoff_1996, Losanoff_1997e, Mesfin_2022a, Sakellaridis_2008f, Tammana_2012j, Tanrikulu_2015e, fjbuilsRepeatedBehaviorDeliberate2024}, 17 cases (24\%) had a psychosocial motivation \cite{Akay_2015f, Benoist_2019e, Bhattacharjee_2008, Cauchi_2002, Goldman_1998f, Hardy_2023g, Kobiela_2015, Li_2013, Naji_2012f, Qureshi_2016, Riva_2018j, Sobnach_2011f, Tay_2004, Thapa_2019f, Tupesis_2004f, Wildhaber_2005, Wnęk_2015f}, 9 cases (12\%) were motivated by protest \cite{Bhumi_2024f, Gardner_2017h, Losanoff_1996, Losanoff_1997e, Tupesis_2004f}, 9 cases (12\%) had another documented motivation \cite{Ali_2020f, Ali_2022g, Emamhadi_2018, Guinan_2019f, Peixoto_2017f, Sakellaridis_2008f, Trgo_2012f, Wadhwa_2015e, Yildiz_2016e}. \paragraph*{Object Characteristics} 51 cases (71\%) ingested a large diameter object (\textgreater{}2.5cm) \cite{Akay_2015f, Al-Faham_2020k, AlShaaibi_2021b, Alao_2006i, Ali_2017, Ali_2022g, Apikotoa_2022f, Atayan_2016, Berry_2021e, Bhasin_2014, CamachoDorado_2018, Cauchi_2002, Chang_2017f, Cox_2007, Csaky_1998e, DivsalarP._2023a, Emamhadi_2018, Gardner_2017h, Guinan_2019f, Jehangir_2019h, Jin_2023, Kariholu_2008, Kerestes_2019, Kobiela_2015, Kumar_2001, Kumar_2019f, Losanoff_1996, Losanoff_1997e, Mesfin_2022a, Misra_2013, Naji_2012f, Ohno_2005, Peixoto_2017f, Qureshi_2016, Riva_2018j, Sakellaridis_2008f, Sultan_2024f, Tanrikulu_2015e, Thapa_2019f, Trgo_2012f, Wnęk_2015f, Yildiz_2016e, fjbuilsRepeatedBehaviorDeliberate2024, teWildt_2010}, 44 cases (61\%) ingested multiple objects \cite{Ali_2020f, Apikotoa_2022f, Ataya_2013, Atayan_2016, Beecroft_1998, Bhattacharjee_2008, Bhumi_2024f, CamachoDorado_2018, Cauchi_2002, Emamhadi_2018, Farhadi_2024h, Fry_2010, Goldman_1998f, Guinan_2019f, Hardy_2023g, Jehangir_2019h, Jin_2023, Kar_2015, Kariholu_2008, Kobiela_2015, Kumar_2001, Kumar_2019f, Li_2013, Liu_2005, Losanoff_1996, Mesfin_2022a, Misra_2013, Naji_2012f, Ohno_2005, Sobnach_2011f, Sultan_2024f, Tammana_2012j, Tanrikulu_2015e, Tay_2004, Thapa_2019f, Wadhwa_2015e, Wildhaber_2005, Yasin_2009, fjbuilsRepeatedBehaviorDeliberate2024, teWildt_2010}, 34 cases (47\%) ingested a sharp object \cite{AlShaaibi_2021b, Alao_2006i, Apikotoa_2022f, Ataya_2013, Benoist_2019e, Bhasin_2014, Bhattacharjee_2008, CamachoDorado_2018, Csaky_1998e, DelgadoSalazar_2020c, DivsalarP._2023a, Emamhadi_2018, Farhadi_2024h, Fry_2010, Guinan_2019f, Hardy_2023g, Jehangir_2019h, Jin_2023, Kariholu_2008, Kobiela_2015, Kumar_2019f, Losanoff_1996, Losanoff_1997e, Mesfin_2022a, Misra_2013, Sobnach_2011f, Yasin_2009, teWildt_2010}, 32 cases (44\%) ingested a long object (\textgreater{}5cm) \cite{Al-Faham_2020k, AlShaaibi_2021b, Ali_2017, Ali_2022g, Atayan_2016, Bhasin_2014, CamachoDorado_2018, Chang_2017f, Cox_2007, Csaky_1998e, DivsalarP._2023a, Emamhadi_2018, Fry_2010, Gardner_2017h, Jin_2023, Kariholu_2008, Kerestes_2019, Kobiela_2015, Kumar_2019f, Mesfin_2022a, Misra_2013, Ohno_2005, Qureshi_2016, Sakellaridis_2008f, Sultan_2024f, Thapa_2019f, Trgo_2012f, Yasin_2009, Yildiz_2016e, teWildt_2010}, 9 cases (12\%) ingested a magnet \cite{Ali_2020f, Bhumi_2024f, Cauchi_2002, Liu_2005, Naji_2012f, Ohno_2005, Tanrikulu_2015e, Tay_2004, Wildhaber_2005}, 2 cases (3\%) ingested a button battery \cite{Berry_2021e, Bhumi_2024f}. \paragraph*{Outcomes} 48 cases (67\%) experienced a complication \cite{Ali_2017, Ali_2020f, Apikotoa_2022f, Atayan_2016, Beecroft_1998, Benoist_2019e, Berry_2021e, Bhasin_2014, Bhumi_2024f, CamachoDorado_2018, Cauchi_2002, Cox_2007, Csaky_1998e, DelgadoSalazar_2020c, DivsalarP._2023a, Emamhadi_2018, Farhadi_2024h, Fry_2010, Gardner_2017h, Goldman_1998f, Jin_2023, Kariholu_2008, Kerestes_2019, Kobiela_2015, Kumar_2001, Kumar_2019f, Liu_2005, Losanoff_1996, Mesfin_2022a, Misra_2013, Naji_2012f, Ohno_2005, Sakellaridis_2008f, Sobnach_2011f, Sultan_2024f, Tanrikulu_2015e, Tay_2004, Thapa_2019f, Trgo_2012f, Tupesis_2004f, Wildhaber_2005, Wnęk_2015f, Yasin_2009, Yildiz_2016e}, 44 cases (61\%) underwent surgery \cite{Al-Faham_2020k, AlShaaibi_2021b, Alao_2006i, Ali_2017, Ali_2020f, Atayan_2016, Beecroft_1998, Bhasin_2014, CamachoDorado_2018, Cauchi_2002, Chang_2017f, Cox_2007, Csaky_1998e, DelgadoSalazar_2020c, DivsalarP._2023a, Farhadi_2024h, Fry_2010, Gardner_2017h, Jin_2023, Kariholu_2008, Kerestes_2019, Kobiela_2015, Kumar_2019f, Liu_2005, Losanoff_1996, Losanoff_1997e, Mesfin_2022a, Misra_2013, Naji_2012f, Sobnach_2011f, Tanrikulu_2015e, Tay_2004, Thapa_2019f, Tupesis_2004f, Wildhaber_2005, Wnęk_2015f, Yasin_2009, Yildiz_2016e, fjbuilsRepeatedBehaviorDeliberate2024}, 31 cases (43\%) underwent endoscopy \cite{Akay_2015f, Ali_2022g, Apikotoa_2022f, Atayan_2016, Benoist_2019e, Berry_2021e, Bhasin_2014, Bhumi_2024f, CamachoDorado_2018, Chang_2017f, DelgadoSalazar_2020c, Gardner_2017h, Guinan_2019f, Hardy_2023g, Jehangir_2019h, Kariholu_2008, Li_2013, Liu_2005, Ohno_2005, Peixoto_2017f, Qureshi_2016, Riva_2018j, Sakellaridis_2008f, Sultan_2024f, Tammana_2012j, Tanrikulu_2015e, Trgo_2012f, Wadhwa_2015e, Wnęk_2015f, teWildt_2010}, 7 cases (10\%) were managed conservatively \cite{Ataya_2013, Bhattacharjee_2008, DivsalarP._2023a, Emamhadi_2018, Goldman_1998f, Kar_2015, Kumar_2001}, 2 cases (3\%) died \cite{Emamhadi_2018, Kumar_2001}. All 90 were male gender. 90 cases (100\%) were detained at the time of ingestion \cite{Elghali_2016, Karp_1991b, Lee_2007}, 88 cases (98\%) were intentional ingestions \cite{Elghali_2016, Karp_1991b, Lee_2007}, 30 cases (33\%) had a psychiatric history documented \cite{Elghali_2016, Karp_1991b, Lee_2007}, 2 cases (2\%) had a history of prior ingestion \cite{Elghali_2016}. No cases were reported for were psychiatric inpatients, were displaced people, were under the influence of alcohol at the time of ingestion, and had a severe disability history.
\paragraph*{Motivation}  70 cases (78\%) reported protest motivation \cite{Elghali_2016, Karp_1991b, Lee_2007}, 12 cases (13\%) reported psychiatric motivation \cite{Karp_1991b}, 6 cases (7\%) reported self-harm motivation \cite{Elghali_2016, Karp_1991b}. No cases were reported for psychosocial motivation and other motivation.
\paragraph*{Object Characteristics}  68 cases (76\%) involved sharp object ingestion \cite{Elghali_2016, Karp_1991b, Lee_2007}, 32 cases (36\%) involved long (\textgreater 5cm) object ingestion \cite{Lee_2007}, 25 cases (28\%) involved ingestion of multiple objects \cite{Elghali_2016, Lee_2007}. No cases were reported for button battery ingestion, magnet ingestion, and involved large diameter (\textgreater 2.5cm) object ingestion.
\paragraph*{Outcomes}  47 cases (52\%) underwent endoscopic intervention \cite{Elghali_2016, Lee_2007}, 29 cases (32\%) were managed conservatively \cite{Elghali_2016, Karp_1991b}, 15 cases (17\%) underwent surgical intervention \cite{Elghali_2016, Karp_1991b, Lee_2007}, 6 cases (7\%) reported complications \cite{Lee_2007}, 1 case (1\%) died \cite{Elghali_2016}.
\paragraph*{Geographical Location}Cases were recorded in 33 countries: 13 cases from USA \cite{Alao_2006i, Ataya_2013, Bhumi_2024f, Fry_2010, Guinan_2019f, Hardy_2023g, Jehangir_2019h, Kerestes_2019, Kumar_2001, Liu_2005, Tammana_2012j, Tay_2004, Tupesis_2004f}; 7 cases from India \cite{Bhasin_2014, Bhattacharjee_2008, Kar_2015, Kariholu_2008, Kumar_2019f, Misra_2013, Wadhwa_2015e} and UK \cite{Beecroft_1998, Berry_2021e, Cauchi_2002, Cox_2007, Gardner_2017h, Qureshi_2016}; 6 cases from Bulgaria \cite{Losanoff_1996, Losanoff_1997e}; 5 cases from Iran \cite{DivsalarP._2023a, Emamhadi_2018, Farhadi_2024h}; 4 cases from Turkey \cite{Akay_2015f, Atayan_2016, Tanrikulu_2015e, Yildiz_2016e}; 2 cases from China \cite{Jin_2023, Li_2013}, Poland \cite{Kobiela_2015, Wnęk_2015f}, and Spain \cite{CamachoDorado_2018, fjbuilsRepeatedBehaviorDeliberate2024}; 1 case from Australia \cite{Apikotoa_2022f}, Bahrain \cite{Ali_2020f}, Croatia \cite{Trgo_2012f}, Ecuador \cite{DelgadoSalazar_2020c}, Egypt \cite{Ali_2022g}, Ethiopia \cite{Mesfin_2022a}, Germany \cite{teWildt_2010}, Greece \cite{Sakellaridis_2008f}, Hungary \cite{Csaky_1998e}, Iraq \cite{Al-Faham_2020k}, Israel \cite{Goldman_1998f}, Italy \cite{Riva_2018j}, Japan \cite{Ohno_2005}, Nepal \cite{Thapa_2019f}, Netherlands \cite{Benoist_2019e}, Oman \cite{AlShaaibi_2021b}, Pakistan \cite{Yasin_2009}, Portugal \cite{Peixoto_2017f}, Qatar \cite{Ali_2017}, Saudi Arabia \cite{Sultan_2024f}, South Africa \cite{Sobnach_2011f}, Sweden \cite{Naji_2012f}, Switzerland \cite{Wildhaber_2005}, and Taiwan \cite{Chang_2017f}. \paragraph*{Gender} 43 cases (60\%) were male \cite{Akay_2015f, Al-Faham_2020k, Alao_2006i, Ali_2017, Ali_2022g, Apikotoa_2022f, Atayan_2016, Benoist_2019e, Berry_2021e, Bhumi_2024f, CamachoDorado_2018, Csaky_1998e, Emamhadi_2018, Farhadi_2024h, Fry_2010, Gardner_2017h, Guinan_2019f, Jehangir_2019h, Jin_2023, Kobiela_2015, Kumar_2001, Kumar_2019f, Liu_2005, Losanoff_1996, Losanoff_1997e, Mesfin_2022a, Misra_2013, Qureshi_2016, Riva_2018j, Sobnach_2011f, Tammana_2012j, Tanrikulu_2015e, Tay_2004, Thapa_2019f, Trgo_2012f, Wadhwa_2015e, Yasin_2009, teWildt_2010}, 28 cases (39\%) were female \cite{AlShaaibi_2021b, Ali_2020f, Ataya_2013, Beecroft_1998, Bhasin_2014, Bhattacharjee_2008, Cauchi_2002, Chang_2017f, Cox_2007, DelgadoSalazar_2020c, DivsalarP._2023a, Goldman_1998f, Hardy_2023g, Kar_2015, Kariholu_2008, Kerestes_2019, Li_2013, Naji_2012f, Ohno_2005, Peixoto_2017f, Sakellaridis_2008f, Sultan_2024f, Tupesis_2004f, Wildhaber_2005, Wnęk_2015f, Yildiz_2016e}, 1 case (1\%) had no gender recorded \cite{fjbuilsRepeatedBehaviorDeliberate2024}. \paragraph*{Age Group} 25 cases (35\%) were between 26 and 40 years of age \cite{Alao_2006i, Ali_2022g, Apikotoa_2022f, Ataya_2013, Benoist_2019e, Bhasin_2014, Chang_2017f, Cox_2007, DelgadoSalazar_2020c, Farhadi_2024h, Fry_2010, Gardner_2017h, Guinan_2019f, Jin_2023, Kumar_2019f, Losanoff_1996, Misra_2013, Qureshi_2016, Riva_2018j, Sakellaridis_2008f, Tammana_2012j, Trgo_2012f, Wnęk_2015f, Yildiz_2016e, fjbuilsRepeatedBehaviorDeliberate2024}, 18 cases (25\%) were between 18 and 25 years of age \cite{Akay_2015f, Ali_2017, Atayan_2016, Bhattacharjee_2008, Csaky_1998e, Kar_2015, Kariholu_2008, Kobiela_2015, Losanoff_1996, Losanoff_1997e, Mesfin_2022a, Peixoto_2017f, Sobnach_2011f, Tupesis_2004f, Yasin_2009}, 13 cases (18\%) were under 18 years of age \cite{AlShaaibi_2021b, Ali_2020f, Cauchi_2002, DivsalarP._2023a, Goldman_1998f, Liu_2005, Naji_2012f, Ohno_2005, Tanrikulu_2015e, Tay_2004, Wildhaber_2005}, 11 cases (15\%) were between 41 and 60 years of age \cite{Al-Faham_2020k, Bhumi_2024f, CamachoDorado_2018, Emamhadi_2018, Hardy_2023g, Jehangir_2019h, Kumar_2001, Sultan_2024f, Thapa_2019f, Wadhwa_2015e, teWildt_2010}, 3 cases (4\%) were over 60 years of age \cite{Beecroft_1998, Kerestes_2019, Li_2013}, 2 cases (3\%) had no age documented \cite{Berry_2021e}. \paragraph*{Population} 36 cases (50\%) had a psychiatric history \cite{AlShaaibi_2021b, Alao_2006i, Ali_2020f, Apikotoa_2022f, Ataya_2013, Atayan_2016, Beecroft_1998, CamachoDorado_2018, Chang_2017f, DelgadoSalazar_2020c, DivsalarP._2023a, Farhadi_2024h, Fry_2010, Guinan_2019f, Hardy_2023g, Jehangir_2019h, Jin_2023, Kar_2015, Kerestes_2019, Kobiela_2015, Kumar_2001, Kumar_2019f, Liu_2005, Mesfin_2022a, Misra_2013, Ohno_2005, Peixoto_2017f, Sakellaridis_2008f, Sultan_2024f, Tammana_2012j, Tanrikulu_2015e, Yildiz_2016e, fjbuilsRepeatedBehaviorDeliberate2024, teWildt_2010}, 19 cases (26\%) had ingested previously \cite{Alao_2006i, Apikotoa_2022f, Berry_2021e, Bhattacharjee_2008, Csaky_1998e, DivsalarP._2023a, Emamhadi_2018, Guinan_2019f, Jehangir_2019h, Jin_2023, Liu_2005, Sakellaridis_2008f, Tanrikulu_2015e, Thapa_2019f, Yildiz_2016e, fjbuilsRepeatedBehaviorDeliberate2024, teWildt_2010}, 12 cases (17\%) were detained persons \cite{Alao_2006i, Ali_2022g, Apikotoa_2022f, Losanoff_1996, Losanoff_1997e, Qureshi_2016, Tammana_2012j, Trgo_2012f}, 7 cases (10\%) were severely disabled \cite{Atayan_2016, Kerestes_2019, Liu_2005, Ohno_2005, Peixoto_2017f, Yildiz_2016e, teWildt_2010}, 4 cases (6\%) were psychiatric inpatients \cite{DivsalarP._2023a, fjbuilsRepeatedBehaviorDeliberate2024, teWildt_2010}, 3 cases (4\%) were under the influence of alcohol \cite{Benoist_2019e, Csaky_1998e, Thapa_2019f}, 2 cases (3\%) were displaced people \cite{Akay_2015f, Gardner_2017h}. \paragraph*{Motivation} 34 cases (47\%) had a psychiatric motivation \cite{Al-Faham_2020k, Alao_2006i, Ali_2020f, Apikotoa_2022f, Ataya_2013, Atayan_2016, Bhasin_2014, Bhattacharjee_2008, DelgadoSalazar_2020c, DivsalarP._2023a, Emamhadi_2018, Farhadi_2024h, Guinan_2019f, Hardy_2023g, Jehangir_2019h, Jin_2023, Kar_2015, Kariholu_2008, Kerestes_2019, Kobiela_2015, Kumar_2001, Kumar_2019f, Li_2013, Liu_2005, Misra_2013, Ohno_2005, Sakellaridis_2008f, Sultan_2024f, Tammana_2012j, Tanrikulu_2015e, Yasin_2009, teWildt_2010}, 21 cases (29\%) were motivated by self-harm intention \cite{Al-Faham_2020k, AlShaaibi_2021b, Alao_2006i, Ali_2017, CamachoDorado_2018, Chang_2017f, Cox_2007, Csaky_1998e, Fry_2010, Li_2013, Losanoff_1996, Losanoff_1997e, Mesfin_2022a, Sakellaridis_2008f, Tammana_2012j, Tanrikulu_2015e, fjbuilsRepeatedBehaviorDeliberate2024}, 17 cases (24\%) had a psychosocial motivation \cite{Akay_2015f, Benoist_2019e, Bhattacharjee_2008, Cauchi_2002, Goldman_1998f, Hardy_2023g, Kobiela_2015, Li_2013, Naji_2012f, Qureshi_2016, Riva_2018j, Sobnach_2011f, Tay_2004, Thapa_2019f, Tupesis_2004f, Wildhaber_2005, Wnęk_2015f}, 9 cases (12\%) were motivated by protest \cite{Bhumi_2024f, Gardner_2017h, Losanoff_1996, Losanoff_1997e, Tupesis_2004f}, 9 cases (12\%) had another documented motivation \cite{Ali_2020f, Ali_2022g, Emamhadi_2018, Guinan_2019f, Peixoto_2017f, Sakellaridis_2008f, Trgo_2012f, Wadhwa_2015e, Yildiz_2016e}. \paragraph*{Object Characteristics} 51 cases (71\%) ingested a large diameter object (\textgreater{}2.5cm) \cite{Akay_2015f, Al-Faham_2020k, AlShaaibi_2021b, Alao_2006i, Ali_2017, Ali_2022g, Apikotoa_2022f, Atayan_2016, Berry_2021e, Bhasin_2014, CamachoDorado_2018, Cauchi_2002, Chang_2017f, Cox_2007, Csaky_1998e, DivsalarP._2023a, Emamhadi_2018, Gardner_2017h, Guinan_2019f, Jehangir_2019h, Jin_2023, Kariholu_2008, Kerestes_2019, Kobiela_2015, Kumar_2001, Kumar_2019f, Losanoff_1996, Losanoff_1997e, Mesfin_2022a, Misra_2013, Naji_2012f, Ohno_2005, Peixoto_2017f, Qureshi_2016, Riva_2018j, Sakellaridis_2008f, Sultan_2024f, Tanrikulu_2015e, Thapa_2019f, Trgo_2012f, Wnęk_2015f, Yildiz_2016e, fjbuilsRepeatedBehaviorDeliberate2024, teWildt_2010}, 44 cases (61\%) ingested multiple objects \cite{Ali_2020f, Apikotoa_2022f, Ataya_2013, Atayan_2016, Beecroft_1998, Bhattacharjee_2008, Bhumi_2024f, CamachoDorado_2018, Cauchi_2002, Emamhadi_2018, Farhadi_2024h, Fry_2010, Goldman_1998f, Guinan_2019f, Hardy_2023g, Jehangir_2019h, Jin_2023, Kar_2015, Kariholu_2008, Kobiela_2015, Kumar_2001, Kumar_2019f, Li_2013, Liu_2005, Losanoff_1996, Mesfin_2022a, Misra_2013, Naji_2012f, Ohno_2005, Sobnach_2011f, Sultan_2024f, Tammana_2012j, Tanrikulu_2015e, Tay_2004, Thapa_2019f, Wadhwa_2015e, Wildhaber_2005, Yasin_2009, fjbuilsRepeatedBehaviorDeliberate2024, teWildt_2010}, 34 cases (47\%) ingested a sharp object \cite{AlShaaibi_2021b, Alao_2006i, Apikotoa_2022f, Ataya_2013, Benoist_2019e, Bhasin_2014, Bhattacharjee_2008, CamachoDorado_2018, Csaky_1998e, DelgadoSalazar_2020c, DivsalarP._2023a, Emamhadi_2018, Farhadi_2024h, Fry_2010, Guinan_2019f, Hardy_2023g, Jehangir_2019h, Jin_2023, Kariholu_2008, Kobiela_2015, Kumar_2019f, Losanoff_1996, Losanoff_1997e, Mesfin_2022a, Misra_2013, Sobnach_2011f, Yasin_2009, teWildt_2010}, 32 cases (44\%) ingested a long object (\textgreater{}5cm) \cite{Al-Faham_2020k, AlShaaibi_2021b, Ali_2017, Ali_2022g, Atayan_2016, Bhasin_2014, CamachoDorado_2018, Chang_2017f, Cox_2007, Csaky_1998e, DivsalarP._2023a, Emamhadi_2018, Fry_2010, Gardner_2017h, Jin_2023, Kariholu_2008, Kerestes_2019, Kobiela_2015, Kumar_2019f, Mesfin_2022a, Misra_2013, Ohno_2005, Qureshi_2016, Sakellaridis_2008f, Sultan_2024f, Thapa_2019f, Trgo_2012f, Yasin_2009, Yildiz_2016e, teWildt_2010}, 9 cases (12\%) ingested a magnet \cite{Ali_2020f, Bhumi_2024f, Cauchi_2002, Liu_2005, Naji_2012f, Ohno_2005, Tanrikulu_2015e, Tay_2004, Wildhaber_2005}, 2 cases (3\%) ingested a button battery \cite{Berry_2021e, Bhumi_2024f}. \paragraph*{Outcomes} 48 cases (67\%) experienced a complication \cite{Ali_2017, Ali_2020f, Apikotoa_2022f, Atayan_2016, Beecroft_1998, Benoist_2019e, Berry_2021e, Bhasin_2014, Bhumi_2024f, CamachoDorado_2018, Cauchi_2002, Cox_2007, Csaky_1998e, DelgadoSalazar_2020c, DivsalarP._2023a, Emamhadi_2018, Farhadi_2024h, Fry_2010, Gardner_2017h, Goldman_1998f, Jin_2023, Kariholu_2008, Kerestes_2019, Kobiela_2015, Kumar_2001, Kumar_2019f, Liu_2005, Losanoff_1996, Mesfin_2022a, Misra_2013, Naji_2012f, Ohno_2005, Sakellaridis_2008f, Sobnach_2011f, Sultan_2024f, Tanrikulu_2015e, Tay_2004, Thapa_2019f, Trgo_2012f, Tupesis_2004f, Wildhaber_2005, Wnęk_2015f, Yasin_2009, Yildiz_2016e}, 44 cases (61\%) underwent surgery \cite{Al-Faham_2020k, AlShaaibi_2021b, Alao_2006i, Ali_2017, Ali_2020f, Atayan_2016, Beecroft_1998, Bhasin_2014, CamachoDorado_2018, Cauchi_2002, Chang_2017f, Cox_2007, Csaky_1998e, DelgadoSalazar_2020c, DivsalarP._2023a, Farhadi_2024h, Fry_2010, Gardner_2017h, Jin_2023, Kariholu_2008, Kerestes_2019, Kobiela_2015, Kumar_2019f, Liu_2005, Losanoff_1996, Losanoff_1997e, Mesfin_2022a, Misra_2013, Naji_2012f, Sobnach_2011f, Tanrikulu_2015e, Tay_2004, Thapa_2019f, Tupesis_2004f, Wildhaber_2005, Wnęk_2015f, Yasin_2009, Yildiz_2016e, fjbuilsRepeatedBehaviorDeliberate2024}, 31 cases (43\%) underwent endoscopy \cite{Akay_2015f, Ali_2022g, Apikotoa_2022f, Atayan_2016, Benoist_2019e, Berry_2021e, Bhasin_2014, Bhumi_2024f, CamachoDorado_2018, Chang_2017f, DelgadoSalazar_2020c, Gardner_2017h, Guinan_2019f, Hardy_2023g, Jehangir_2019h, Kariholu_2008, Li_2013, Liu_2005, Ohno_2005, Peixoto_2017f, Qureshi_2016, Riva_2018j, Sakellaridis_2008f, Sultan_2024f, Tammana_2012j, Tanrikulu_2015e, Trgo_2012f, Wadhwa_2015e, Wnęk_2015f, teWildt_2010}, 7 cases (10\%) were managed conservatively \cite{Ataya_2013, Bhattacharjee_2008, DivsalarP._2023a, Emamhadi_2018, Goldman_1998f, Kar_2015, Kumar_2001}, 2 cases (3\%) died \cite{Emamhadi_2018, Kumar_2001}. All 90 were male gender. 90 cases (100\%) were detained at the time of ingestion \cite{Elghali_2016, Karp_1991b, Lee_2007}, 88 cases (98\%) were intentional ingestions \cite{Elghali_2016, Karp_1991b, Lee_2007}, 30 cases (33\%) had a psychiatric history documented \cite{Elghali_2016, Karp_1991b, Lee_2007}, 2 cases (2\%) had a history of prior ingestion \cite{Elghali_2016}. No cases were reported for were psychiatric inpatients, were displaced people, were under the influence of alcohol at the time of ingestion, and had a severe disability history.
\paragraph*{Motivation}  70 cases (78\%) reported protest motivation \cite{Elghali_2016, Karp_1991b, Lee_2007}, 12 cases (13\%) reported psychiatric motivation \cite{Karp_1991b}, 6 cases (7\%) reported self-harm motivation \cite{Elghali_2016, Karp_1991b}. No cases were reported for psychosocial motivation and other motivation.
\paragraph*{Object Characteristics}  68 cases (76\%) involved sharp object ingestion \cite{Elghali_2016, Karp_1991b, Lee_2007}, 32 cases (36\%) involved long (\textgreater 5cm) object ingestion \cite{Lee_2007}, 25 cases (28\%) involved ingestion of multiple objects \cite{Elghali_2016, Lee_2007}. No cases were reported for button battery ingestion, magnet ingestion, and involved large diameter (\textgreater 2.5cm) object ingestion.
\paragraph*{Outcomes}  47 cases (52\%) underwent endoscopic intervention \cite{Elghali_2016, Lee_2007}, 29 cases (32\%) were managed conservatively \cite{Elghali_2016, Karp_1991b}, 15 cases (17\%) underwent surgical intervention \cite{Elghali_2016, Karp_1991b, Lee_2007}, 6 cases (7\%) reported complications \cite{Lee_2007}, 1 case (1\%) died \cite{Elghali_2016}.
\paragraph*{Geographical Location}Cases were recorded in 33 countries: 13 cases from USA \cite{Alao_2006i, Ataya_2013, Bhumi_2024f, Fry_2010, Guinan_2019f, Hardy_2023g, Jehangir_2019h, Kerestes_2019, Kumar_2001, Liu_2005, Tammana_2012j, Tay_2004, Tupesis_2004f}; 7 cases from India \cite{Bhasin_2014, Bhattacharjee_2008, Kar_2015, Kariholu_2008, Kumar_2019f, Misra_2013, Wadhwa_2015e} and UK \cite{Beecroft_1998, Berry_2021e, Cauchi_2002, Cox_2007, Gardner_2017h, Qureshi_2016}; 6 cases from Bulgaria \cite{Losanoff_1996, Losanoff_1997e}; 5 cases from Iran \cite{DivsalarP._2023a, Emamhadi_2018, Farhadi_2024h}; 4 cases from Turkey \cite{Akay_2015f, Atayan_2016, Tanrikulu_2015e, Yildiz_2016e}; 2 cases from China \cite{Jin_2023, Li_2013}, Poland \cite{Kobiela_2015, Wnęk_2015f}, and Spain \cite{CamachoDorado_2018, fjbuilsRepeatedBehaviorDeliberate2024}; 1 case from Australia \cite{Apikotoa_2022f}, Bahrain \cite{Ali_2020f}, Croatia \cite{Trgo_2012f}, Ecuador \cite{DelgadoSalazar_2020c}, Egypt \cite{Ali_2022g}, Ethiopia \cite{Mesfin_2022a}, Germany \cite{teWildt_2010}, Greece \cite{Sakellaridis_2008f}, Hungary \cite{Csaky_1998e}, Iraq \cite{Al-Faham_2020k}, Israel \cite{Goldman_1998f}, Italy \cite{Riva_2018j}, Japan \cite{Ohno_2005}, Nepal \cite{Thapa_2019f}, Netherlands \cite{Benoist_2019e}, Oman \cite{AlShaaibi_2021b}, Pakistan \cite{Yasin_2009}, Portugal \cite{Peixoto_2017f}, Qatar \cite{Ali_2017}, Saudi Arabia \cite{Sultan_2024f}, South Africa \cite{Sobnach_2011f}, Sweden \cite{Naji_2012f}, Switzerland \cite{Wildhaber_2005}, and Taiwan \cite{Chang_2017f}. \paragraph*{Gender} 43 cases (60\%) were male \cite{Akay_2015f, Al-Faham_2020k, Alao_2006i, Ali_2017, Ali_2022g, Apikotoa_2022f, Atayan_2016, Benoist_2019e, Berry_2021e, Bhumi_2024f, CamachoDorado_2018, Csaky_1998e, Emamhadi_2018, Farhadi_2024h, Fry_2010, Gardner_2017h, Guinan_2019f, Jehangir_2019h, Jin_2023, Kobiela_2015, Kumar_2001, Kumar_2019f, Liu_2005, Losanoff_1996, Losanoff_1997e, Mesfin_2022a, Misra_2013, Qureshi_2016, Riva_2018j, Sobnach_2011f, Tammana_2012j, Tanrikulu_2015e, Tay_2004, Thapa_2019f, Trgo_2012f, Wadhwa_2015e, Yasin_2009, teWildt_2010}, 28 cases (39\%) were female \cite{AlShaaibi_2021b, Ali_2020f, Ataya_2013, Beecroft_1998, Bhasin_2014, Bhattacharjee_2008, Cauchi_2002, Chang_2017f, Cox_2007, DelgadoSalazar_2020c, DivsalarP._2023a, Goldman_1998f, Hardy_2023g, Kar_2015, Kariholu_2008, Kerestes_2019, Li_2013, Naji_2012f, Ohno_2005, Peixoto_2017f, Sakellaridis_2008f, Sultan_2024f, Tupesis_2004f, Wildhaber_2005, Wnęk_2015f, Yildiz_2016e}, 1 case (1\%) had no gender recorded \cite{fjbuilsRepeatedBehaviorDeliberate2024}. \paragraph*{Age Group} 25 cases (35\%) were between 26 and 40 years of age \cite{Alao_2006i, Ali_2022g, Apikotoa_2022f, Ataya_2013, Benoist_2019e, Bhasin_2014, Chang_2017f, Cox_2007, DelgadoSalazar_2020c, Farhadi_2024h, Fry_2010, Gardner_2017h, Guinan_2019f, Jin_2023, Kumar_2019f, Losanoff_1996, Misra_2013, Qureshi_2016, Riva_2018j, Sakellaridis_2008f, Tammana_2012j, Trgo_2012f, Wnęk_2015f, Yildiz_2016e, fjbuilsRepeatedBehaviorDeliberate2024}, 18 cases (25\%) were between 18 and 25 years of age \cite{Akay_2015f, Ali_2017, Atayan_2016, Bhattacharjee_2008, Csaky_1998e, Kar_2015, Kariholu_2008, Kobiela_2015, Losanoff_1996, Losanoff_1997e, Mesfin_2022a, Peixoto_2017f, Sobnach_2011f, Tupesis_2004f, Yasin_2009}, 13 cases (18\%) were under 18 years of age \cite{AlShaaibi_2021b, Ali_2020f, Cauchi_2002, DivsalarP._2023a, Goldman_1998f, Liu_2005, Naji_2012f, Ohno_2005, Tanrikulu_2015e, Tay_2004, Wildhaber_2005}, 11 cases (15\%) were between 41 and 60 years of age \cite{Al-Faham_2020k, Bhumi_2024f, CamachoDorado_2018, Emamhadi_2018, Hardy_2023g, Jehangir_2019h, Kumar_2001, Sultan_2024f, Thapa_2019f, Wadhwa_2015e, teWildt_2010}, 3 cases (4\%) were over 60 years of age \cite{Beecroft_1998, Kerestes_2019, Li_2013}, 2 cases (3\%) had no age documented \cite{Berry_2021e}. \paragraph*{Population} 36 cases (50\%) had a psychiatric history \cite{AlShaaibi_2021b, Alao_2006i, Ali_2020f, Apikotoa_2022f, Ataya_2013, Atayan_2016, Beecroft_1998, CamachoDorado_2018, Chang_2017f, DelgadoSalazar_2020c, DivsalarP._2023a, Farhadi_2024h, Fry_2010, Guinan_2019f, Hardy_2023g, Jehangir_2019h, Jin_2023, Kar_2015, Kerestes_2019, Kobiela_2015, Kumar_2001, Kumar_2019f, Liu_2005, Mesfin_2022a, Misra_2013, Ohno_2005, Peixoto_2017f, Sakellaridis_2008f, Sultan_2024f, Tammana_2012j, Tanrikulu_2015e, Yildiz_2016e, fjbuilsRepeatedBehaviorDeliberate2024, teWildt_2010}, 19 cases (26\%) had ingested previously \cite{Alao_2006i, Apikotoa_2022f, Berry_2021e, Bhattacharjee_2008, Csaky_1998e, DivsalarP._2023a, Emamhadi_2018, Guinan_2019f, Jehangir_2019h, Jin_2023, Liu_2005, Sakellaridis_2008f, Tanrikulu_2015e, Thapa_2019f, Yildiz_2016e, fjbuilsRepeatedBehaviorDeliberate2024, teWildt_2010}, 12 cases (17\%) were detained persons \cite{Alao_2006i, Ali_2022g, Apikotoa_2022f, Losanoff_1996, Losanoff_1997e, Qureshi_2016, Tammana_2012j, Trgo_2012f}, 7 cases (10\%) were severely disabled \cite{Atayan_2016, Kerestes_2019, Liu_2005, Ohno_2005, Peixoto_2017f, Yildiz_2016e, teWildt_2010}, 4 cases (6\%) were psychiatric inpatients \cite{DivsalarP._2023a, fjbuilsRepeatedBehaviorDeliberate2024, teWildt_2010}, 3 cases (4\%) were under the influence of alcohol \cite{Benoist_2019e, Csaky_1998e, Thapa_2019f}, 2 cases (3\%) were displaced people \cite{Akay_2015f, Gardner_2017h}. \paragraph*{Motivation} 34 cases (47\%) had a psychiatric motivation \cite{Al-Faham_2020k, Alao_2006i, Ali_2020f, Apikotoa_2022f, Ataya_2013, Atayan_2016, Bhasin_2014, Bhattacharjee_2008, DelgadoSalazar_2020c, DivsalarP._2023a, Emamhadi_2018, Farhadi_2024h, Guinan_2019f, Hardy_2023g, Jehangir_2019h, Jin_2023, Kar_2015, Kariholu_2008, Kerestes_2019, Kobiela_2015, Kumar_2001, Kumar_2019f, Li_2013, Liu_2005, Misra_2013, Ohno_2005, Sakellaridis_2008f, Sultan_2024f, Tammana_2012j, Tanrikulu_2015e, Yasin_2009, teWildt_2010}, 21 cases (29\%) were motivated by self-harm intention \cite{Al-Faham_2020k, AlShaaibi_2021b, Alao_2006i, Ali_2017, CamachoDorado_2018, Chang_2017f, Cox_2007, Csaky_1998e, Fry_2010, Li_2013, Losanoff_1996, Losanoff_1997e, Mesfin_2022a, Sakellaridis_2008f, Tammana_2012j, Tanrikulu_2015e, fjbuilsRepeatedBehaviorDeliberate2024}, 17 cases (24\%) had a psychosocial motivation \cite{Akay_2015f, Benoist_2019e, Bhattacharjee_2008, Cauchi_2002, Goldman_1998f, Hardy_2023g, Kobiela_2015, Li_2013, Naji_2012f, Qureshi_2016, Riva_2018j, Sobnach_2011f, Tay_2004, Thapa_2019f, Tupesis_2004f, Wildhaber_2005, Wnęk_2015f}, 9 cases (12\%) were motivated by protest \cite{Bhumi_2024f, Gardner_2017h, Losanoff_1996, Losanoff_1997e, Tupesis_2004f}, 9 cases (12\%) had another documented motivation \cite{Ali_2020f, Ali_2022g, Emamhadi_2018, Guinan_2019f, Peixoto_2017f, Sakellaridis_2008f, Trgo_2012f, Wadhwa_2015e, Yildiz_2016e}. \paragraph*{Object Characteristics} 51 cases (71\%) ingested a large diameter object (\textgreater{}2.5cm) \cite{Akay_2015f, Al-Faham_2020k, AlShaaibi_2021b, Alao_2006i, Ali_2017, Ali_2022g, Apikotoa_2022f, Atayan_2016, Berry_2021e, Bhasin_2014, CamachoDorado_2018, Cauchi_2002, Chang_2017f, Cox_2007, Csaky_1998e, DivsalarP._2023a, Emamhadi_2018, Gardner_2017h, Guinan_2019f, Jehangir_2019h, Jin_2023, Kariholu_2008, Kerestes_2019, Kobiela_2015, Kumar_2001, Kumar_2019f, Losanoff_1996, Losanoff_1997e, Mesfin_2022a, Misra_2013, Naji_2012f, Ohno_2005, Peixoto_2017f, Qureshi_2016, Riva_2018j, Sakellaridis_2008f, Sultan_2024f, Tanrikulu_2015e, Thapa_2019f, Trgo_2012f, Wnęk_2015f, Yildiz_2016e, fjbuilsRepeatedBehaviorDeliberate2024, teWildt_2010}, 44 cases (61\%) ingested multiple objects \cite{Ali_2020f, Apikotoa_2022f, Ataya_2013, Atayan_2016, Beecroft_1998, Bhattacharjee_2008, Bhumi_2024f, CamachoDorado_2018, Cauchi_2002, Emamhadi_2018, Farhadi_2024h, Fry_2010, Goldman_1998f, Guinan_2019f, Hardy_2023g, Jehangir_2019h, Jin_2023, Kar_2015, Kariholu_2008, Kobiela_2015, Kumar_2001, Kumar_2019f, Li_2013, Liu_2005, Losanoff_1996, Mesfin_2022a, Misra_2013, Naji_2012f, Ohno_2005, Sobnach_2011f, Sultan_2024f, Tammana_2012j, Tanrikulu_2015e, Tay_2004, Thapa_2019f, Wadhwa_2015e, Wildhaber_2005, Yasin_2009, fjbuilsRepeatedBehaviorDeliberate2024, teWildt_2010}, 34 cases (47\%) ingested a sharp object \cite{AlShaaibi_2021b, Alao_2006i, Apikotoa_2022f, Ataya_2013, Benoist_2019e, Bhasin_2014, Bhattacharjee_2008, CamachoDorado_2018, Csaky_1998e, DelgadoSalazar_2020c, DivsalarP._2023a, Emamhadi_2018, Farhadi_2024h, Fry_2010, Guinan_2019f, Hardy_2023g, Jehangir_2019h, Jin_2023, Kariholu_2008, Kobiela_2015, Kumar_2019f, Losanoff_1996, Losanoff_1997e, Mesfin_2022a, Misra_2013, Sobnach_2011f, Yasin_2009, teWildt_2010}, 32 cases (44\%) ingested a long object (\textgreater{}5cm) \cite{Al-Faham_2020k, AlShaaibi_2021b, Ali_2017, Ali_2022g, Atayan_2016, Bhasin_2014, CamachoDorado_2018, Chang_2017f, Cox_2007, Csaky_1998e, DivsalarP._2023a, Emamhadi_2018, Fry_2010, Gardner_2017h, Jin_2023, Kariholu_2008, Kerestes_2019, Kobiela_2015, Kumar_2019f, Mesfin_2022a, Misra_2013, Ohno_2005, Qureshi_2016, Sakellaridis_2008f, Sultan_2024f, Thapa_2019f, Trgo_2012f, Yasin_2009, Yildiz_2016e, teWildt_2010}, 9 cases (12\%) ingested a magnet \cite{Ali_2020f, Bhumi_2024f, Cauchi_2002, Liu_2005, Naji_2012f, Ohno_2005, Tanrikulu_2015e, Tay_2004, Wildhaber_2005}, 2 cases (3\%) ingested a button battery \cite{Berry_2021e, Bhumi_2024f}. \paragraph*{Outcomes} 48 cases (67\%) experienced a complication \cite{Ali_2017, Ali_2020f, Apikotoa_2022f, Atayan_2016, Beecroft_1998, Benoist_2019e, Berry_2021e, Bhasin_2014, Bhumi_2024f, CamachoDorado_2018, Cauchi_2002, Cox_2007, Csaky_1998e, DelgadoSalazar_2020c, DivsalarP._2023a, Emamhadi_2018, Farhadi_2024h, Fry_2010, Gardner_2017h, Goldman_1998f, Jin_2023, Kariholu_2008, Kerestes_2019, Kobiela_2015, Kumar_2001, Kumar_2019f, Liu_2005, Losanoff_1996, Mesfin_2022a, Misra_2013, Naji_2012f, Ohno_2005, Sakellaridis_2008f, Sobnach_2011f, Sultan_2024f, Tanrikulu_2015e, Tay_2004, Thapa_2019f, Trgo_2012f, Tupesis_2004f, Wildhaber_2005, Wnęk_2015f, Yasin_2009, Yildiz_2016e}, 44 cases (61\%) underwent surgery \cite{Al-Faham_2020k, AlShaaibi_2021b, Alao_2006i, Ali_2017, Ali_2020f, Atayan_2016, Beecroft_1998, Bhasin_2014, CamachoDorado_2018, Cauchi_2002, Chang_2017f, Cox_2007, Csaky_1998e, DelgadoSalazar_2020c, DivsalarP._2023a, Farhadi_2024h, Fry_2010, Gardner_2017h, Jin_2023, Kariholu_2008, Kerestes_2019, Kobiela_2015, Kumar_2019f, Liu_2005, Losanoff_1996, Losanoff_1997e, Mesfin_2022a, Misra_2013, Naji_2012f, Sobnach_2011f, Tanrikulu_2015e, Tay_2004, Thapa_2019f, Tupesis_2004f, Wildhaber_2005, Wnęk_2015f, Yasin_2009, Yildiz_2016e, fjbuilsRepeatedBehaviorDeliberate2024}, 31 cases (43\%) underwent endoscopy \cite{Akay_2015f, Ali_2022g, Apikotoa_2022f, Atayan_2016, Benoist_2019e, Berry_2021e, Bhasin_2014, Bhumi_2024f, CamachoDorado_2018, Chang_2017f, DelgadoSalazar_2020c, Gardner_2017h, Guinan_2019f, Hardy_2023g, Jehangir_2019h, Kariholu_2008, Li_2013, Liu_2005, Ohno_2005, Peixoto_2017f, Qureshi_2016, Riva_2018j, Sakellaridis_2008f, Sultan_2024f, Tammana_2012j, Tanrikulu_2015e, Trgo_2012f, Wadhwa_2015e, Wnęk_2015f, teWildt_2010}, 7 cases (10\%) were managed conservatively \cite{Ataya_2013, Bhattacharjee_2008, DivsalarP._2023a, Emamhadi_2018, Goldman_1998f, Kar_2015, Kumar_2001}, 2 cases (3\%) died \cite{Emamhadi_2018, Kumar_2001}. All 90 were male gender. 90 cases (100\%) were detained at the time of ingestion \cite{Elghali_2016, Karp_1991b, Lee_2007}, 88 cases (98\%) were intentional ingestions \cite{Elghali_2016, Karp_1991b, Lee_2007}, 30 cases (33\%) had a psychiatric history documented \cite{Elghali_2016, Karp_1991b, Lee_2007}, 2 cases (2\%) had a history of prior ingestion \cite{Elghali_2016}. No cases were reported for were psychiatric inpatients, were displaced people, were under the influence of alcohol at the time of ingestion, and had a severe disability history.
\paragraph*{Motivation}  70 cases (78\%) reported protest motivation \cite{Elghali_2016, Karp_1991b, Lee_2007}, 12 cases (13\%) reported psychiatric motivation \cite{Karp_1991b}, 6 cases (7\%) reported self-harm motivation \cite{Elghali_2016, Karp_1991b}. No cases were reported for psychosocial motivation and other motivation.
\paragraph*{Object Characteristics}  68 cases (76\%) involved sharp object ingestion \cite{Elghali_2016, Karp_1991b, Lee_2007}, 32 cases (36\%) involved long (\textgreater 5cm) object ingestion \cite{Lee_2007}, 25 cases (28\%) involved ingestion of multiple objects \cite{Elghali_2016, Lee_2007}. No cases were reported for button battery ingestion, magnet ingestion, and involved large diameter (\textgreater 2.5cm) object ingestion.
\paragraph*{Outcomes}  47 cases (52\%) underwent endoscopic intervention \cite{Elghali_2016, Lee_2007}, 29 cases (32\%) were managed conservatively \cite{Elghali_2016, Karp_1991b}, 15 cases (17\%) underwent surgical intervention \cite{Elghali_2016, Karp_1991b, Lee_2007}, 6 cases (7\%) reported complications \cite{Lee_2007}, 1 case (1\%) died \cite{Elghali_2016}.
\paragraph*{Geographical Location}Cases were recorded in 33 countries: 13 cases from USA \cite{Alao_2006i, Ataya_2013, Bhumi_2024f, Fry_2010, Guinan_2019f, Hardy_2023g, Jehangir_2019h, Kerestes_2019, Kumar_2001, Liu_2005, Tammana_2012j, Tay_2004, Tupesis_2004f}; 7 cases from India \cite{Bhasin_2014, Bhattacharjee_2008, Kar_2015, Kariholu_2008, Kumar_2019f, Misra_2013, Wadhwa_2015e} and UK \cite{Beecroft_1998, Berry_2021e, Cauchi_2002, Cox_2007, Gardner_2017h, Qureshi_2016}; 6 cases from Bulgaria \cite{Losanoff_1996, Losanoff_1997e}; 5 cases from Iran \cite{DivsalarP._2023a, Emamhadi_2018, Farhadi_2024h}; 4 cases from Turkey \cite{Akay_2015f, Atayan_2016, Tanrikulu_2015e, Yildiz_2016e}; 2 cases from China \cite{Jin_2023, Li_2013}, Poland \cite{Kobiela_2015, Wnęk_2015f}, and Spain \cite{CamachoDorado_2018, fjbuilsRepeatedBehaviorDeliberate2024}; 1 case from Australia \cite{Apikotoa_2022f}, Bahrain \cite{Ali_2020f}, Croatia \cite{Trgo_2012f}, Ecuador \cite{DelgadoSalazar_2020c}, Egypt \cite{Ali_2022g}, Ethiopia \cite{Mesfin_2022a}, Germany \cite{teWildt_2010}, Greece \cite{Sakellaridis_2008f}, Hungary \cite{Csaky_1998e}, Iraq \cite{Al-Faham_2020k}, Israel \cite{Goldman_1998f}, Italy \cite{Riva_2018j}, Japan \cite{Ohno_2005}, Nepal \cite{Thapa_2019f}, Netherlands \cite{Benoist_2019e}, Oman \cite{AlShaaibi_2021b}, Pakistan \cite{Yasin_2009}, Portugal \cite{Peixoto_2017f}, Qatar \cite{Ali_2017}, Saudi Arabia \cite{Sultan_2024f}, South Africa \cite{Sobnach_2011f}, Sweden \cite{Naji_2012f}, Switzerland \cite{Wildhaber_2005}, and Taiwan \cite{Chang_2017f}. \paragraph*{Gender} 43 cases (60\%) were male \cite{Akay_2015f, Al-Faham_2020k, Alao_2006i, Ali_2017, Ali_2022g, Apikotoa_2022f, Atayan_2016, Benoist_2019e, Berry_2021e, Bhumi_2024f, CamachoDorado_2018, Csaky_1998e, Emamhadi_2018, Farhadi_2024h, Fry_2010, Gardner_2017h, Guinan_2019f, Jehangir_2019h, Jin_2023, Kobiela_2015, Kumar_2001, Kumar_2019f, Liu_2005, Losanoff_1996, Losanoff_1997e, Mesfin_2022a, Misra_2013, Qureshi_2016, Riva_2018j, Sobnach_2011f, Tammana_2012j, Tanrikulu_2015e, Tay_2004, Thapa_2019f, Trgo_2012f, Wadhwa_2015e, Yasin_2009, teWildt_2010}, 28 cases (39\%) were female \cite{AlShaaibi_2021b, Ali_2020f, Ataya_2013, Beecroft_1998, Bhasin_2014, Bhattacharjee_2008, Cauchi_2002, Chang_2017f, Cox_2007, DelgadoSalazar_2020c, DivsalarP._2023a, Goldman_1998f, Hardy_2023g, Kar_2015, Kariholu_2008, Kerestes_2019, Li_2013, Naji_2012f, Ohno_2005, Peixoto_2017f, Sakellaridis_2008f, Sultan_2024f, Tupesis_2004f, Wildhaber_2005, Wnęk_2015f, Yildiz_2016e}, 1 case (1\%) had no gender recorded \cite{fjbuilsRepeatedBehaviorDeliberate2024}. \paragraph*{Age Group} 25 cases (35\%) were between 26 and 40 years of age \cite{Alao_2006i, Ali_2022g, Apikotoa_2022f, Ataya_2013, Benoist_2019e, Bhasin_2014, Chang_2017f, Cox_2007, DelgadoSalazar_2020c, Farhadi_2024h, Fry_2010, Gardner_2017h, Guinan_2019f, Jin_2023, Kumar_2019f, Losanoff_1996, Misra_2013, Qureshi_2016, Riva_2018j, Sakellaridis_2008f, Tammana_2012j, Trgo_2012f, Wnęk_2015f, Yildiz_2016e, fjbuilsRepeatedBehaviorDeliberate2024}, 18 cases (25\%) were between 18 and 25 years of age \cite{Akay_2015f, Ali_2017, Atayan_2016, Bhattacharjee_2008, Csaky_1998e, Kar_2015, Kariholu_2008, Kobiela_2015, Losanoff_1996, Losanoff_1997e, Mesfin_2022a, Peixoto_2017f, Sobnach_2011f, Tupesis_2004f, Yasin_2009}, 13 cases (18\%) were under 18 years of age \cite{AlShaaibi_2021b, Ali_2020f, Cauchi_2002, DivsalarP._2023a, Goldman_1998f, Liu_2005, Naji_2012f, Ohno_2005, Tanrikulu_2015e, Tay_2004, Wildhaber_2005}, 11 cases (15\%) were between 41 and 60 years of age \cite{Al-Faham_2020k, Bhumi_2024f, CamachoDorado_2018, Emamhadi_2018, Hardy_2023g, Jehangir_2019h, Kumar_2001, Sultan_2024f, Thapa_2019f, Wadhwa_2015e, teWildt_2010}, 3 cases (4\%) were over 60 years of age \cite{Beecroft_1998, Kerestes_2019, Li_2013}, 2 cases (3\%) had no age documented \cite{Berry_2021e}. \paragraph*{Population} 36 cases (50\%) had a psychiatric history \cite{AlShaaibi_2021b, Alao_2006i, Ali_2020f, Apikotoa_2022f, Ataya_2013, Atayan_2016, Beecroft_1998, CamachoDorado_2018, Chang_2017f, DelgadoSalazar_2020c, DivsalarP._2023a, Farhadi_2024h, Fry_2010, Guinan_2019f, Hardy_2023g, Jehangir_2019h, Jin_2023, Kar_2015, Kerestes_2019, Kobiela_2015, Kumar_2001, Kumar_2019f, Liu_2005, Mesfin_2022a, Misra_2013, Ohno_2005, Peixoto_2017f, Sakellaridis_2008f, Sultan_2024f, Tammana_2012j, Tanrikulu_2015e, Yildiz_2016e, fjbuilsRepeatedBehaviorDeliberate2024, teWildt_2010}, 19 cases (26\%) had ingested previously \cite{Alao_2006i, Apikotoa_2022f, Berry_2021e, Bhattacharjee_2008, Csaky_1998e, DivsalarP._2023a, Emamhadi_2018, Guinan_2019f, Jehangir_2019h, Jin_2023, Liu_2005, Sakellaridis_2008f, Tanrikulu_2015e, Thapa_2019f, Yildiz_2016e, fjbuilsRepeatedBehaviorDeliberate2024, teWildt_2010}, 12 cases (17\%) were detained persons \cite{Alao_2006i, Ali_2022g, Apikotoa_2022f, Losanoff_1996, Losanoff_1997e, Qureshi_2016, Tammana_2012j, Trgo_2012f}, 7 cases (10\%) were severely disabled \cite{Atayan_2016, Kerestes_2019, Liu_2005, Ohno_2005, Peixoto_2017f, Yildiz_2016e, teWildt_2010}, 4 cases (6\%) were psychiatric inpatients \cite{DivsalarP._2023a, fjbuilsRepeatedBehaviorDeliberate2024, teWildt_2010}, 3 cases (4\%) were under the influence of alcohol \cite{Benoist_2019e, Csaky_1998e, Thapa_2019f}, 2 cases (3\%) were displaced people \cite{Akay_2015f, Gardner_2017h}. \paragraph*{Motivation} 34 cases (47\%) had a psychiatric motivation \cite{Al-Faham_2020k, Alao_2006i, Ali_2020f, Apikotoa_2022f, Ataya_2013, Atayan_2016, Bhasin_2014, Bhattacharjee_2008, DelgadoSalazar_2020c, DivsalarP._2023a, Emamhadi_2018, Farhadi_2024h, Guinan_2019f, Hardy_2023g, Jehangir_2019h, Jin_2023, Kar_2015, Kariholu_2008, Kerestes_2019, Kobiela_2015, Kumar_2001, Kumar_2019f, Li_2013, Liu_2005, Misra_2013, Ohno_2005, Sakellaridis_2008f, Sultan_2024f, Tammana_2012j, Tanrikulu_2015e, Yasin_2009, teWildt_2010}, 21 cases (29\%) were motivated by self-harm intention \cite{Al-Faham_2020k, AlShaaibi_2021b, Alao_2006i, Ali_2017, CamachoDorado_2018, Chang_2017f, Cox_2007, Csaky_1998e, Fry_2010, Li_2013, Losanoff_1996, Losanoff_1997e, Mesfin_2022a, Sakellaridis_2008f, Tammana_2012j, Tanrikulu_2015e, fjbuilsRepeatedBehaviorDeliberate2024}, 17 cases (24\%) had a psychosocial motivation \cite{Akay_2015f, Benoist_2019e, Bhattacharjee_2008, Cauchi_2002, Goldman_1998f, Hardy_2023g, Kobiela_2015, Li_2013, Naji_2012f, Qureshi_2016, Riva_2018j, Sobnach_2011f, Tay_2004, Thapa_2019f, Tupesis_2004f, Wildhaber_2005, Wnęk_2015f}, 9 cases (12\%) were motivated by protest \cite{Bhumi_2024f, Gardner_2017h, Losanoff_1996, Losanoff_1997e, Tupesis_2004f}, 9 cases (12\%) had another documented motivation \cite{Ali_2020f, Ali_2022g, Emamhadi_2018, Guinan_2019f, Peixoto_2017f, Sakellaridis_2008f, Trgo_2012f, Wadhwa_2015e, Yildiz_2016e}. \paragraph*{Object Characteristics} 51 cases (71\%) ingested a large diameter object (\textgreater{}2.5cm) \cite{Akay_2015f, Al-Faham_2020k, AlShaaibi_2021b, Alao_2006i, Ali_2017, Ali_2022g, Apikotoa_2022f, Atayan_2016, Berry_2021e, Bhasin_2014, CamachoDorado_2018, Cauchi_2002, Chang_2017f, Cox_2007, Csaky_1998e, DivsalarP._2023a, Emamhadi_2018, Gardner_2017h, Guinan_2019f, Jehangir_2019h, Jin_2023, Kariholu_2008, Kerestes_2019, Kobiela_2015, Kumar_2001, Kumar_2019f, Losanoff_1996, Losanoff_1997e, Mesfin_2022a, Misra_2013, Naji_2012f, Ohno_2005, Peixoto_2017f, Qureshi_2016, Riva_2018j, Sakellaridis_2008f, Sultan_2024f, Tanrikulu_2015e, Thapa_2019f, Trgo_2012f, Wnęk_2015f, Yildiz_2016e, fjbuilsRepeatedBehaviorDeliberate2024, teWildt_2010}, 44 cases (61\%) ingested multiple objects \cite{Ali_2020f, Apikotoa_2022f, Ataya_2013, Atayan_2016, Beecroft_1998, Bhattacharjee_2008, Bhumi_2024f, CamachoDorado_2018, Cauchi_2002, Emamhadi_2018, Farhadi_2024h, Fry_2010, Goldman_1998f, Guinan_2019f, Hardy_2023g, Jehangir_2019h, Jin_2023, Kar_2015, Kariholu_2008, Kobiela_2015, Kumar_2001, Kumar_2019f, Li_2013, Liu_2005, Losanoff_1996, Mesfin_2022a, Misra_2013, Naji_2012f, Ohno_2005, Sobnach_2011f, Sultan_2024f, Tammana_2012j, Tanrikulu_2015e, Tay_2004, Thapa_2019f, Wadhwa_2015e, Wildhaber_2005, Yasin_2009, fjbuilsRepeatedBehaviorDeliberate2024, teWildt_2010}, 34 cases (47\%) ingested a sharp object \cite{AlShaaibi_2021b, Alao_2006i, Apikotoa_2022f, Ataya_2013, Benoist_2019e, Bhasin_2014, Bhattacharjee_2008, CamachoDorado_2018, Csaky_1998e, DelgadoSalazar_2020c, DivsalarP._2023a, Emamhadi_2018, Farhadi_2024h, Fry_2010, Guinan_2019f, Hardy_2023g, Jehangir_2019h, Jin_2023, Kariholu_2008, Kobiela_2015, Kumar_2019f, Losanoff_1996, Losanoff_1997e, Mesfin_2022a, Misra_2013, Sobnach_2011f, Yasin_2009, teWildt_2010}, 32 cases (44\%) ingested a long object (\textgreater{}5cm) \cite{Al-Faham_2020k, AlShaaibi_2021b, Ali_2017, Ali_2022g, Atayan_2016, Bhasin_2014, CamachoDorado_2018, Chang_2017f, Cox_2007, Csaky_1998e, DivsalarP._2023a, Emamhadi_2018, Fry_2010, Gardner_2017h, Jin_2023, Kariholu_2008, Kerestes_2019, Kobiela_2015, Kumar_2019f, Mesfin_2022a, Misra_2013, Ohno_2005, Qureshi_2016, Sakellaridis_2008f, Sultan_2024f, Thapa_2019f, Trgo_2012f, Yasin_2009, Yildiz_2016e, teWildt_2010}, 9 cases (12\%) ingested a magnet \cite{Ali_2020f, Bhumi_2024f, Cauchi_2002, Liu_2005, Naji_2012f, Ohno_2005, Tanrikulu_2015e, Tay_2004, Wildhaber_2005}, 2 cases (3\%) ingested a button battery \cite{Berry_2021e, Bhumi_2024f}. \paragraph*{Outcomes} 48 cases (67\%) experienced a complication \cite{Ali_2017, Ali_2020f, Apikotoa_2022f, Atayan_2016, Beecroft_1998, Benoist_2019e, Berry_2021e, Bhasin_2014, Bhumi_2024f, CamachoDorado_2018, Cauchi_2002, Cox_2007, Csaky_1998e, DelgadoSalazar_2020c, DivsalarP._2023a, Emamhadi_2018, Farhadi_2024h, Fry_2010, Gardner_2017h, Goldman_1998f, Jin_2023, Kariholu_2008, Kerestes_2019, Kobiela_2015, Kumar_2001, Kumar_2019f, Liu_2005, Losanoff_1996, Mesfin_2022a, Misra_2013, Naji_2012f, Ohno_2005, Sakellaridis_2008f, Sobnach_2011f, Sultan_2024f, Tanrikulu_2015e, Tay_2004, Thapa_2019f, Trgo_2012f, Tupesis_2004f, Wildhaber_2005, Wnęk_2015f, Yasin_2009, Yildiz_2016e}, 44 cases (61\%) underwent surgery \cite{Al-Faham_2020k, AlShaaibi_2021b, Alao_2006i, Ali_2017, Ali_2020f, Atayan_2016, Beecroft_1998, Bhasin_2014, CamachoDorado_2018, Cauchi_2002, Chang_2017f, Cox_2007, Csaky_1998e, DelgadoSalazar_2020c, DivsalarP._2023a, Farhadi_2024h, Fry_2010, Gardner_2017h, Jin_2023, Kariholu_2008, Kerestes_2019, Kobiela_2015, Kumar_2019f, Liu_2005, Losanoff_1996, Losanoff_1997e, Mesfin_2022a, Misra_2013, Naji_2012f, Sobnach_2011f, Tanrikulu_2015e, Tay_2004, Thapa_2019f, Tupesis_2004f, Wildhaber_2005, Wnęk_2015f, Yasin_2009, Yildiz_2016e, fjbuilsRepeatedBehaviorDeliberate2024}, 31 cases (43\%) underwent endoscopy \cite{Akay_2015f, Ali_2022g, Apikotoa_2022f, Atayan_2016, Benoist_2019e, Berry_2021e, Bhasin_2014, Bhumi_2024f, CamachoDorado_2018, Chang_2017f, DelgadoSalazar_2020c, Gardner_2017h, Guinan_2019f, Hardy_2023g, Jehangir_2019h, Kariholu_2008, Li_2013, Liu_2005, Ohno_2005, Peixoto_2017f, Qureshi_2016, Riva_2018j, Sakellaridis_2008f, Sultan_2024f, Tammana_2012j, Tanrikulu_2015e, Trgo_2012f, Wadhwa_2015e, Wnęk_2015f, teWildt_2010}, 7 cases (10\%) were managed conservatively \cite{Ataya_2013, Bhattacharjee_2008, DivsalarP._2023a, Emamhadi_2018, Goldman_1998f, Kar_2015, Kumar_2001}, 2 cases (3\%) died \cite{Emamhadi_2018, Kumar_2001}. All 90 were male gender. 90 cases (100\%) were detained at the time of ingestion \cite{Elghali_2016, Karp_1991b, Lee_2007}, 88 cases (98\%) were intentional ingestions \cite{Elghali_2016, Karp_1991b, Lee_2007}, 30 cases (33\%) had a psychiatric history documented \cite{Elghali_2016, Karp_1991b, Lee_2007}, 2 cases (2\%) had a history of prior ingestion \cite{Elghali_2016}. No cases were reported for were psychiatric inpatients, were displaced people, were under the influence of alcohol at the time of ingestion, and had a severe disability history.
\paragraph*{Motivation}  70 cases (78\%) reported protest motivation \cite{Elghali_2016, Karp_1991b, Lee_2007}, 12 cases (13\%) reported psychiatric motivation \cite{Karp_1991b}, 6 cases (7\%) reported self-harm motivation \cite{Elghali_2016, Karp_1991b}. No cases were reported for psychosocial motivation and other motivation.
\paragraph*{Object Characteristics}  68 cases (76\%) involved sharp object ingestion \cite{Elghali_2016, Karp_1991b, Lee_2007}, 32 cases (36\%) involved long (\textgreater 5cm) object ingestion \cite{Lee_2007}, 25 cases (28\%) involved ingestion of multiple objects \cite{Elghali_2016, Lee_2007}. No cases were reported for button battery ingestion, magnet ingestion, and involved large diameter (\textgreater 2.5cm) object ingestion.
\paragraph*{Outcomes}  47 cases (52\%) underwent endoscopic intervention \cite{Elghali_2016, Lee_2007}, 29 cases (32\%) were managed conservatively \cite{Elghali_2016, Karp_1991b}, 15 cases (17\%) underwent surgical intervention \cite{Elghali_2016, Karp_1991b, Lee_2007}, 6 cases (7\%) reported complications \cite{Lee_2007}, 1 case (1\%) died \cite{Elghali_2016}.
\paragraph*{Geographical Location}Cases were recorded in 33 countries: 13 cases from USA \cite{Alao_2006i, Ataya_2013, Bhumi_2024f, Fry_2010, Guinan_2019f, Hardy_2023g, Jehangir_2019h, Kerestes_2019, Kumar_2001, Liu_2005, Tammana_2012j, Tay_2004, Tupesis_2004f}; 7 cases from India \cite{Bhasin_2014, Bhattacharjee_2008, Kar_2015, Kariholu_2008, Kumar_2019f, Misra_2013, Wadhwa_2015e} and UK \cite{Beecroft_1998, Berry_2021e, Cauchi_2002, Cox_2007, Gardner_2017h, Qureshi_2016}; 6 cases from Bulgaria \cite{Losanoff_1996, Losanoff_1997e}; 5 cases from Iran \cite{DivsalarP._2023a, Emamhadi_2018, Farhadi_2024h}; 4 cases from Turkey \cite{Akay_2015f, Atayan_2016, Tanrikulu_2015e, Yildiz_2016e}; 2 cases from China \cite{Jin_2023, Li_2013}, Poland \cite{Kobiela_2015, Wnęk_2015f}, and Spain \cite{CamachoDorado_2018, fjbuilsRepeatedBehaviorDeliberate2024}; 1 case from Australia \cite{Apikotoa_2022f}, Bahrain \cite{Ali_2020f}, Croatia \cite{Trgo_2012f}, Ecuador \cite{DelgadoSalazar_2020c}, Egypt \cite{Ali_2022g}, Ethiopia \cite{Mesfin_2022a}, Germany \cite{teWildt_2010}, Greece \cite{Sakellaridis_2008f}, Hungary \cite{Csaky_1998e}, Iraq \cite{Al-Faham_2020k}, Israel \cite{Goldman_1998f}, Italy \cite{Riva_2018j}, Japan \cite{Ohno_2005}, Nepal \cite{Thapa_2019f}, Netherlands \cite{Benoist_2019e}, Oman \cite{AlShaaibi_2021b}, Pakistan \cite{Yasin_2009}, Portugal \cite{Peixoto_2017f}, Qatar \cite{Ali_2017}, Saudi Arabia \cite{Sultan_2024f}, South Africa \cite{Sobnach_2011f}, Sweden \cite{Naji_2012f}, Switzerland \cite{Wildhaber_2005}, and Taiwan \cite{Chang_2017f}. \paragraph*{Gender} 43 cases (60\%) were male \cite{Akay_2015f, Al-Faham_2020k, Alao_2006i, Ali_2017, Ali_2022g, Apikotoa_2022f, Atayan_2016, Benoist_2019e, Berry_2021e, Bhumi_2024f, CamachoDorado_2018, Csaky_1998e, Emamhadi_2018, Farhadi_2024h, Fry_2010, Gardner_2017h, Guinan_2019f, Jehangir_2019h, Jin_2023, Kobiela_2015, Kumar_2001, Kumar_2019f, Liu_2005, Losanoff_1996, Losanoff_1997e, Mesfin_2022a, Misra_2013, Qureshi_2016, Riva_2018j, Sobnach_2011f, Tammana_2012j, Tanrikulu_2015e, Tay_2004, Thapa_2019f, Trgo_2012f, Wadhwa_2015e, Yasin_2009, teWildt_2010}, 28 cases (39\%) were female \cite{AlShaaibi_2021b, Ali_2020f, Ataya_2013, Beecroft_1998, Bhasin_2014, Bhattacharjee_2008, Cauchi_2002, Chang_2017f, Cox_2007, DelgadoSalazar_2020c, DivsalarP._2023a, Goldman_1998f, Hardy_2023g, Kar_2015, Kariholu_2008, Kerestes_2019, Li_2013, Naji_2012f, Ohno_2005, Peixoto_2017f, Sakellaridis_2008f, Sultan_2024f, Tupesis_2004f, Wildhaber_2005, Wnęk_2015f, Yildiz_2016e}, 1 case (1\%) had no gender recorded \cite{fjbuilsRepeatedBehaviorDeliberate2024}. \paragraph*{Age Group} 25 cases (35\%) were between 26 and 40 years of age \cite{Alao_2006i, Ali_2022g, Apikotoa_2022f, Ataya_2013, Benoist_2019e, Bhasin_2014, Chang_2017f, Cox_2007, DelgadoSalazar_2020c, Farhadi_2024h, Fry_2010, Gardner_2017h, Guinan_2019f, Jin_2023, Kumar_2019f, Losanoff_1996, Misra_2013, Qureshi_2016, Riva_2018j, Sakellaridis_2008f, Tammana_2012j, Trgo_2012f, Wnęk_2015f, Yildiz_2016e, fjbuilsRepeatedBehaviorDeliberate2024}, 18 cases (25\%) were between 18 and 25 years of age \cite{Akay_2015f, Ali_2017, Atayan_2016, Bhattacharjee_2008, Csaky_1998e, Kar_2015, Kariholu_2008, Kobiela_2015, Losanoff_1996, Losanoff_1997e, Mesfin_2022a, Peixoto_2017f, Sobnach_2011f, Tupesis_2004f, Yasin_2009}, 13 cases (18\%) were under 18 years of age \cite{AlShaaibi_2021b, Ali_2020f, Cauchi_2002, DivsalarP._2023a, Goldman_1998f, Liu_2005, Naji_2012f, Ohno_2005, Tanrikulu_2015e, Tay_2004, Wildhaber_2005}, 11 cases (15\%) were between 41 and 60 years of age \cite{Al-Faham_2020k, Bhumi_2024f, CamachoDorado_2018, Emamhadi_2018, Hardy_2023g, Jehangir_2019h, Kumar_2001, Sultan_2024f, Thapa_2019f, Wadhwa_2015e, teWildt_2010}, 3 cases (4\%) were over 60 years of age \cite{Beecroft_1998, Kerestes_2019, Li_2013}, 2 cases (3\%) had no age documented \cite{Berry_2021e}. \paragraph*{Population} 36 cases (50\%) had a psychiatric history \cite{AlShaaibi_2021b, Alao_2006i, Ali_2020f, Apikotoa_2022f, Ataya_2013, Atayan_2016, Beecroft_1998, CamachoDorado_2018, Chang_2017f, DelgadoSalazar_2020c, DivsalarP._2023a, Farhadi_2024h, Fry_2010, Guinan_2019f, Hardy_2023g, Jehangir_2019h, Jin_2023, Kar_2015, Kerestes_2019, Kobiela_2015, Kumar_2001, Kumar_2019f, Liu_2005, Mesfin_2022a, Misra_2013, Ohno_2005, Peixoto_2017f, Sakellaridis_2008f, Sultan_2024f, Tammana_2012j, Tanrikulu_2015e, Yildiz_2016e, fjbuilsRepeatedBehaviorDeliberate2024, teWildt_2010}, 19 cases (26\%) had ingested previously \cite{Alao_2006i, Apikotoa_2022f, Berry_2021e, Bhattacharjee_2008, Csaky_1998e, DivsalarP._2023a, Emamhadi_2018, Guinan_2019f, Jehangir_2019h, Jin_2023, Liu_2005, Sakellaridis_2008f, Tanrikulu_2015e, Thapa_2019f, Yildiz_2016e, fjbuilsRepeatedBehaviorDeliberate2024, teWildt_2010}, 12 cases (17\%) were detained persons \cite{Alao_2006i, Ali_2022g, Apikotoa_2022f, Losanoff_1996, Losanoff_1997e, Qureshi_2016, Tammana_2012j, Trgo_2012f}, 7 cases (10\%) were severely disabled \cite{Atayan_2016, Kerestes_2019, Liu_2005, Ohno_2005, Peixoto_2017f, Yildiz_2016e, teWildt_2010}, 4 cases (6\%) were psychiatric inpatients \cite{DivsalarP._2023a, fjbuilsRepeatedBehaviorDeliberate2024, teWildt_2010}, 3 cases (4\%) were under the influence of alcohol \cite{Benoist_2019e, Csaky_1998e, Thapa_2019f}, 2 cases (3\%) were displaced people \cite{Akay_2015f, Gardner_2017h}. \paragraph*{Motivation} 34 cases (47\%) had a psychiatric motivation \cite{Al-Faham_2020k, Alao_2006i, Ali_2020f, Apikotoa_2022f, Ataya_2013, Atayan_2016, Bhasin_2014, Bhattacharjee_2008, DelgadoSalazar_2020c, DivsalarP._2023a, Emamhadi_2018, Farhadi_2024h, Guinan_2019f, Hardy_2023g, Jehangir_2019h, Jin_2023, Kar_2015, Kariholu_2008, Kerestes_2019, Kobiela_2015, Kumar_2001, Kumar_2019f, Li_2013, Liu_2005, Misra_2013, Ohno_2005, Sakellaridis_2008f, Sultan_2024f, Tammana_2012j, Tanrikulu_2015e, Yasin_2009, teWildt_2010}, 21 cases (29\%) were motivated by self-harm intention \cite{Al-Faham_2020k, AlShaaibi_2021b, Alao_2006i, Ali_2017, CamachoDorado_2018, Chang_2017f, Cox_2007, Csaky_1998e, Fry_2010, Li_2013, Losanoff_1996, Losanoff_1997e, Mesfin_2022a, Sakellaridis_2008f, Tammana_2012j, Tanrikulu_2015e, fjbuilsRepeatedBehaviorDeliberate2024}, 17 cases (24\%) had a psychosocial motivation \cite{Akay_2015f, Benoist_2019e, Bhattacharjee_2008, Cauchi_2002, Goldman_1998f, Hardy_2023g, Kobiela_2015, Li_2013, Naji_2012f, Qureshi_2016, Riva_2018j, Sobnach_2011f, Tay_2004, Thapa_2019f, Tupesis_2004f, Wildhaber_2005, Wnęk_2015f}, 9 cases (12\%) were motivated by protest \cite{Bhumi_2024f, Gardner_2017h, Losanoff_1996, Losanoff_1997e, Tupesis_2004f}, 9 cases (12\%) had another documented motivation \cite{Ali_2020f, Ali_2022g, Emamhadi_2018, Guinan_2019f, Peixoto_2017f, Sakellaridis_2008f, Trgo_2012f, Wadhwa_2015e, Yildiz_2016e}. \paragraph*{Object Characteristics} 51 cases (71\%) ingested a large diameter object (\textgreater{}2.5cm) \cite{Akay_2015f, Al-Faham_2020k, AlShaaibi_2021b, Alao_2006i, Ali_2017, Ali_2022g, Apikotoa_2022f, Atayan_2016, Berry_2021e, Bhasin_2014, CamachoDorado_2018, Cauchi_2002, Chang_2017f, Cox_2007, Csaky_1998e, DivsalarP._2023a, Emamhadi_2018, Gardner_2017h, Guinan_2019f, Jehangir_2019h, Jin_2023, Kariholu_2008, Kerestes_2019, Kobiela_2015, Kumar_2001, Kumar_2019f, Losanoff_1996, Losanoff_1997e, Mesfin_2022a, Misra_2013, Naji_2012f, Ohno_2005, Peixoto_2017f, Qureshi_2016, Riva_2018j, Sakellaridis_2008f, Sultan_2024f, Tanrikulu_2015e, Thapa_2019f, Trgo_2012f, Wnęk_2015f, Yildiz_2016e, fjbuilsRepeatedBehaviorDeliberate2024, teWildt_2010}, 44 cases (61\%) ingested multiple objects \cite{Ali_2020f, Apikotoa_2022f, Ataya_2013, Atayan_2016, Beecroft_1998, Bhattacharjee_2008, Bhumi_2024f, CamachoDorado_2018, Cauchi_2002, Emamhadi_2018, Farhadi_2024h, Fry_2010, Goldman_1998f, Guinan_2019f, Hardy_2023g, Jehangir_2019h, Jin_2023, Kar_2015, Kariholu_2008, Kobiela_2015, Kumar_2001, Kumar_2019f, Li_2013, Liu_2005, Losanoff_1996, Mesfin_2022a, Misra_2013, Naji_2012f, Ohno_2005, Sobnach_2011f, Sultan_2024f, Tammana_2012j, Tanrikulu_2015e, Tay_2004, Thapa_2019f, Wadhwa_2015e, Wildhaber_2005, Yasin_2009, fjbuilsRepeatedBehaviorDeliberate2024, teWildt_2010}, 34 cases (47\%) ingested a sharp object \cite{AlShaaibi_2021b, Alao_2006i, Apikotoa_2022f, Ataya_2013, Benoist_2019e, Bhasin_2014, Bhattacharjee_2008, CamachoDorado_2018, Csaky_1998e, DelgadoSalazar_2020c, DivsalarP._2023a, Emamhadi_2018, Farhadi_2024h, Fry_2010, Guinan_2019f, Hardy_2023g, Jehangir_2019h, Jin_2023, Kariholu_2008, Kobiela_2015, Kumar_2019f, Losanoff_1996, Losanoff_1997e, Mesfin_2022a, Misra_2013, Sobnach_2011f, Yasin_2009, teWildt_2010}, 32 cases (44\%) ingested a long object (\textgreater{}5cm) \cite{Al-Faham_2020k, AlShaaibi_2021b, Ali_2017, Ali_2022g, Atayan_2016, Bhasin_2014, CamachoDorado_2018, Chang_2017f, Cox_2007, Csaky_1998e, DivsalarP._2023a, Emamhadi_2018, Fry_2010, Gardner_2017h, Jin_2023, Kariholu_2008, Kerestes_2019, Kobiela_2015, Kumar_2019f, Mesfin_2022a, Misra_2013, Ohno_2005, Qureshi_2016, Sakellaridis_2008f, Sultan_2024f, Thapa_2019f, Trgo_2012f, Yasin_2009, Yildiz_2016e, teWildt_2010}, 9 cases (12\%) ingested a magnet \cite{Ali_2020f, Bhumi_2024f, Cauchi_2002, Liu_2005, Naji_2012f, Ohno_2005, Tanrikulu_2015e, Tay_2004, Wildhaber_2005}, 2 cases (3\%) ingested a button battery \cite{Berry_2021e, Bhumi_2024f}. \paragraph*{Outcomes} 48 cases (67\%) experienced a complication \cite{Ali_2017, Ali_2020f, Apikotoa_2022f, Atayan_2016, Beecroft_1998, Benoist_2019e, Berry_2021e, Bhasin_2014, Bhumi_2024f, CamachoDorado_2018, Cauchi_2002, Cox_2007, Csaky_1998e, DelgadoSalazar_2020c, DivsalarP._2023a, Emamhadi_2018, Farhadi_2024h, Fry_2010, Gardner_2017h, Goldman_1998f, Jin_2023, Kariholu_2008, Kerestes_2019, Kobiela_2015, Kumar_2001, Kumar_2019f, Liu_2005, Losanoff_1996, Mesfin_2022a, Misra_2013, Naji_2012f, Ohno_2005, Sakellaridis_2008f, Sobnach_2011f, Sultan_2024f, Tanrikulu_2015e, Tay_2004, Thapa_2019f, Trgo_2012f, Tupesis_2004f, Wildhaber_2005, Wnęk_2015f, Yasin_2009, Yildiz_2016e}, 44 cases (61\%) underwent surgery \cite{Al-Faham_2020k, AlShaaibi_2021b, Alao_2006i, Ali_2017, Ali_2020f, Atayan_2016, Beecroft_1998, Bhasin_2014, CamachoDorado_2018, Cauchi_2002, Chang_2017f, Cox_2007, Csaky_1998e, DelgadoSalazar_2020c, DivsalarP._2023a, Farhadi_2024h, Fry_2010, Gardner_2017h, Jin_2023, Kariholu_2008, Kerestes_2019, Kobiela_2015, Kumar_2019f, Liu_2005, Losanoff_1996, Losanoff_1997e, Mesfin_2022a, Misra_2013, Naji_2012f, Sobnach_2011f, Tanrikulu_2015e, Tay_2004, Thapa_2019f, Tupesis_2004f, Wildhaber_2005, Wnęk_2015f, Yasin_2009, Yildiz_2016e, fjbuilsRepeatedBehaviorDeliberate2024}, 31 cases (43\%) underwent endoscopy \cite{Akay_2015f, Ali_2022g, Apikotoa_2022f, Atayan_2016, Benoist_2019e, Berry_2021e, Bhasin_2014, Bhumi_2024f, CamachoDorado_2018, Chang_2017f, DelgadoSalazar_2020c, Gardner_2017h, Guinan_2019f, Hardy_2023g, Jehangir_2019h, Kariholu_2008, Li_2013, Liu_2005, Ohno_2005, Peixoto_2017f, Qureshi_2016, Riva_2018j, Sakellaridis_2008f, Sultan_2024f, Tammana_2012j, Tanrikulu_2015e, Trgo_2012f, Wadhwa_2015e, Wnęk_2015f, teWildt_2010}, 7 cases (10\%) were managed conservatively \cite{Ataya_2013, Bhattacharjee_2008, DivsalarP._2023a, Emamhadi_2018, Goldman_1998f, Kar_2015, Kumar_2001}, 2 cases (3\%) died \cite{Emamhadi_2018, Kumar_2001}. All 90 were male gender. 90 cases (100\%) were detained at the time of ingestion \cite{Elghali_2016, Karp_1991b, Lee_2007}, 88 cases (98\%) were intentional ingestions \cite{Elghali_2016, Karp_1991b, Lee_2007}, 30 cases (33\%) had a psychiatric history documented \cite{Elghali_2016, Karp_1991b, Lee_2007}, 2 cases (2\%) had a history of prior ingestion \cite{Elghali_2016}. No cases were reported for were psychiatric inpatients, were displaced people, were under the influence of alcohol at the time of ingestion, and had a severe disability history.
\paragraph*{Motivation}  70 cases (78\%) reported protest motivation \cite{Elghali_2016, Karp_1991b, Lee_2007}, 12 cases (13\%) reported psychiatric motivation \cite{Karp_1991b}, 6 cases (7\%) reported self-harm motivation \cite{Elghali_2016, Karp_1991b}. No cases were reported for psychosocial motivation and other motivation.
\paragraph*{Object Characteristics}  68 cases (76\%) involved sharp object ingestion \cite{Elghali_2016, Karp_1991b, Lee_2007}, 32 cases (36\%) involved long (\textgreater 5cm) object ingestion \cite{Lee_2007}, 25 cases (28\%) involved ingestion of multiple objects \cite{Elghali_2016, Lee_2007}. No cases were reported for button battery ingestion, magnet ingestion, and involved large diameter (\textgreater 2.5cm) object ingestion.
\paragraph*{Outcomes}  47 cases (52\%) underwent endoscopic intervention \cite{Elghali_2016, Lee_2007}, 29 cases (32\%) were managed conservatively \cite{Elghali_2016, Karp_1991b}, 15 cases (17\%) underwent surgical intervention \cite{Elghali_2016, Karp_1991b, Lee_2007}, 6 cases (7\%) reported complications \cite{Lee_2007}, 1 case (1\%) died \cite{Elghali_2016}.
\paragraph*{Geographical Location}Cases were recorded in 33 countries: 13 cases from USA \cite{Alao_2006i, Ataya_2013, Bhumi_2024f, Fry_2010, Guinan_2019f, Hardy_2023g, Jehangir_2019h, Kerestes_2019, Kumar_2001, Liu_2005, Tammana_2012j, Tay_2004, Tupesis_2004f}; 7 cases from India \cite{Bhasin_2014, Bhattacharjee_2008, Kar_2015, Kariholu_2008, Kumar_2019f, Misra_2013, Wadhwa_2015e} and UK \cite{Beecroft_1998, Berry_2021e, Cauchi_2002, Cox_2007, Gardner_2017h, Qureshi_2016}; 6 cases from Bulgaria \cite{Losanoff_1996, Losanoff_1997e}; 5 cases from Iran \cite{DivsalarP._2023a, Emamhadi_2018, Farhadi_2024h}; 4 cases from Turkey \cite{Akay_2015f, Atayan_2016, Tanrikulu_2015e, Yildiz_2016e}; 2 cases from China \cite{Jin_2023, Li_2013}, Poland \cite{Kobiela_2015, Wnęk_2015f}, and Spain \cite{CamachoDorado_2018, fjbuilsRepeatedBehaviorDeliberate2024}; 1 case from Australia \cite{Apikotoa_2022f}, Bahrain \cite{Ali_2020f}, Croatia \cite{Trgo_2012f}, Ecuador \cite{DelgadoSalazar_2020c}, Egypt \cite{Ali_2022g}, Ethiopia \cite{Mesfin_2022a}, Germany \cite{teWildt_2010}, Greece \cite{Sakellaridis_2008f}, Hungary \cite{Csaky_1998e}, Iraq \cite{Al-Faham_2020k}, Israel \cite{Goldman_1998f}, Italy \cite{Riva_2018j}, Japan \cite{Ohno_2005}, Nepal \cite{Thapa_2019f}, Netherlands \cite{Benoist_2019e}, Oman \cite{AlShaaibi_2021b}, Pakistan \cite{Yasin_2009}, Portugal \cite{Peixoto_2017f}, Qatar \cite{Ali_2017}, Saudi Arabia \cite{Sultan_2024f}, South Africa \cite{Sobnach_2011f}, Sweden \cite{Naji_2012f}, Switzerland \cite{Wildhaber_2005}, and Taiwan \cite{Chang_2017f}. \paragraph*{Gender} 43 cases (60\%) were male \cite{Akay_2015f, Al-Faham_2020k, Alao_2006i, Ali_2017, Ali_2022g, Apikotoa_2022f, Atayan_2016, Benoist_2019e, Berry_2021e, Bhumi_2024f, CamachoDorado_2018, Csaky_1998e, Emamhadi_2018, Farhadi_2024h, Fry_2010, Gardner_2017h, Guinan_2019f, Jehangir_2019h, Jin_2023, Kobiela_2015, Kumar_2001, Kumar_2019f, Liu_2005, Losanoff_1996, Losanoff_1997e, Mesfin_2022a, Misra_2013, Qureshi_2016, Riva_2018j, Sobnach_2011f, Tammana_2012j, Tanrikulu_2015e, Tay_2004, Thapa_2019f, Trgo_2012f, Wadhwa_2015e, Yasin_2009, teWildt_2010}, 28 cases (39\%) were female \cite{AlShaaibi_2021b, Ali_2020f, Ataya_2013, Beecroft_1998, Bhasin_2014, Bhattacharjee_2008, Cauchi_2002, Chang_2017f, Cox_2007, DelgadoSalazar_2020c, DivsalarP._2023a, Goldman_1998f, Hardy_2023g, Kar_2015, Kariholu_2008, Kerestes_2019, Li_2013, Naji_2012f, Ohno_2005, Peixoto_2017f, Sakellaridis_2008f, Sultan_2024f, Tupesis_2004f, Wildhaber_2005, Wnęk_2015f, Yildiz_2016e}, 1 case (1\%) had no gender recorded \cite{fjbuilsRepeatedBehaviorDeliberate2024}. \paragraph*{Age Group} 25 cases (35\%) were between 26 and 40 years of age \cite{Alao_2006i, Ali_2022g, Apikotoa_2022f, Ataya_2013, Benoist_2019e, Bhasin_2014, Chang_2017f, Cox_2007, DelgadoSalazar_2020c, Farhadi_2024h, Fry_2010, Gardner_2017h, Guinan_2019f, Jin_2023, Kumar_2019f, Losanoff_1996, Misra_2013, Qureshi_2016, Riva_2018j, Sakellaridis_2008f, Tammana_2012j, Trgo_2012f, Wnęk_2015f, Yildiz_2016e, fjbuilsRepeatedBehaviorDeliberate2024}, 18 cases (25\%) were between 18 and 25 years of age \cite{Akay_2015f, Ali_2017, Atayan_2016, Bhattacharjee_2008, Csaky_1998e, Kar_2015, Kariholu_2008, Kobiela_2015, Losanoff_1996, Losanoff_1997e, Mesfin_2022a, Peixoto_2017f, Sobnach_2011f, Tupesis_2004f, Yasin_2009}, 13 cases (18\%) were under 18 years of age \cite{AlShaaibi_2021b, Ali_2020f, Cauchi_2002, DivsalarP._2023a, Goldman_1998f, Liu_2005, Naji_2012f, Ohno_2005, Tanrikulu_2015e, Tay_2004, Wildhaber_2005}, 11 cases (15\%) were between 41 and 60 years of age \cite{Al-Faham_2020k, Bhumi_2024f, CamachoDorado_2018, Emamhadi_2018, Hardy_2023g, Jehangir_2019h, Kumar_2001, Sultan_2024f, Thapa_2019f, Wadhwa_2015e, teWildt_2010}, 3 cases (4\%) were over 60 years of age \cite{Beecroft_1998, Kerestes_2019, Li_2013}, 2 cases (3\%) had no age documented \cite{Berry_2021e}. \paragraph*{Population} 36 cases (50\%) had a psychiatric history \cite{AlShaaibi_2021b, Alao_2006i, Ali_2020f, Apikotoa_2022f, Ataya_2013, Atayan_2016, Beecroft_1998, CamachoDorado_2018, Chang_2017f, DelgadoSalazar_2020c, DivsalarP._2023a, Farhadi_2024h, Fry_2010, Guinan_2019f, Hardy_2023g, Jehangir_2019h, Jin_2023, Kar_2015, Kerestes_2019, Kobiela_2015, Kumar_2001, Kumar_2019f, Liu_2005, Mesfin_2022a, Misra_2013, Ohno_2005, Peixoto_2017f, Sakellaridis_2008f, Sultan_2024f, Tammana_2012j, Tanrikulu_2015e, Yildiz_2016e, fjbuilsRepeatedBehaviorDeliberate2024, teWildt_2010}, 19 cases (26\%) had ingested previously \cite{Alao_2006i, Apikotoa_2022f, Berry_2021e, Bhattacharjee_2008, Csaky_1998e, DivsalarP._2023a, Emamhadi_2018, Guinan_2019f, Jehangir_2019h, Jin_2023, Liu_2005, Sakellaridis_2008f, Tanrikulu_2015e, Thapa_2019f, Yildiz_2016e, fjbuilsRepeatedBehaviorDeliberate2024, teWildt_2010}, 12 cases (17\%) were detained persons \cite{Alao_2006i, Ali_2022g, Apikotoa_2022f, Losanoff_1996, Losanoff_1997e, Qureshi_2016, Tammana_2012j, Trgo_2012f}, 7 cases (10\%) were severely disabled \cite{Atayan_2016, Kerestes_2019, Liu_2005, Ohno_2005, Peixoto_2017f, Yildiz_2016e, teWildt_2010}, 4 cases (6\%) were psychiatric inpatients \cite{DivsalarP._2023a, fjbuilsRepeatedBehaviorDeliberate2024, teWildt_2010}, 3 cases (4\%) were under the influence of alcohol \cite{Benoist_2019e, Csaky_1998e, Thapa_2019f}, 2 cases (3\%) were displaced people \cite{Akay_2015f, Gardner_2017h}. \paragraph*{Motivation} 34 cases (47\%) had a psychiatric motivation \cite{Al-Faham_2020k, Alao_2006i, Ali_2020f, Apikotoa_2022f, Ataya_2013, Atayan_2016, Bhasin_2014, Bhattacharjee_2008, DelgadoSalazar_2020c, DivsalarP._2023a, Emamhadi_2018, Farhadi_2024h, Guinan_2019f, Hardy_2023g, Jehangir_2019h, Jin_2023, Kar_2015, Kariholu_2008, Kerestes_2019, Kobiela_2015, Kumar_2001, Kumar_2019f, Li_2013, Liu_2005, Misra_2013, Ohno_2005, Sakellaridis_2008f, Sultan_2024f, Tammana_2012j, Tanrikulu_2015e, Yasin_2009, teWildt_2010}, 21 cases (29\%) were motivated by self-harm intention \cite{Al-Faham_2020k, AlShaaibi_2021b, Alao_2006i, Ali_2017, CamachoDorado_2018, Chang_2017f, Cox_2007, Csaky_1998e, Fry_2010, Li_2013, Losanoff_1996, Losanoff_1997e, Mesfin_2022a, Sakellaridis_2008f, Tammana_2012j, Tanrikulu_2015e, fjbuilsRepeatedBehaviorDeliberate2024}, 17 cases (24\%) had a psychosocial motivation \cite{Akay_2015f, Benoist_2019e, Bhattacharjee_2008, Cauchi_2002, Goldman_1998f, Hardy_2023g, Kobiela_2015, Li_2013, Naji_2012f, Qureshi_2016, Riva_2018j, Sobnach_2011f, Tay_2004, Thapa_2019f, Tupesis_2004f, Wildhaber_2005, Wnęk_2015f}, 9 cases (12\%) were motivated by protest \cite{Bhumi_2024f, Gardner_2017h, Losanoff_1996, Losanoff_1997e, Tupesis_2004f}, 9 cases (12\%) had another documented motivation \cite{Ali_2020f, Ali_2022g, Emamhadi_2018, Guinan_2019f, Peixoto_2017f, Sakellaridis_2008f, Trgo_2012f, Wadhwa_2015e, Yildiz_2016e}. \paragraph*{Object Characteristics} 51 cases (71\%) ingested a large diameter object (\textgreater{}2.5cm) \cite{Akay_2015f, Al-Faham_2020k, AlShaaibi_2021b, Alao_2006i, Ali_2017, Ali_2022g, Apikotoa_2022f, Atayan_2016, Berry_2021e, Bhasin_2014, CamachoDorado_2018, Cauchi_2002, Chang_2017f, Cox_2007, Csaky_1998e, DivsalarP._2023a, Emamhadi_2018, Gardner_2017h, Guinan_2019f, Jehangir_2019h, Jin_2023, Kariholu_2008, Kerestes_2019, Kobiela_2015, Kumar_2001, Kumar_2019f, Losanoff_1996, Losanoff_1997e, Mesfin_2022a, Misra_2013, Naji_2012f, Ohno_2005, Peixoto_2017f, Qureshi_2016, Riva_2018j, Sakellaridis_2008f, Sultan_2024f, Tanrikulu_2015e, Thapa_2019f, Trgo_2012f, Wnęk_2015f, Yildiz_2016e, fjbuilsRepeatedBehaviorDeliberate2024, teWildt_2010}, 44 cases (61\%) ingested multiple objects \cite{Ali_2020f, Apikotoa_2022f, Ataya_2013, Atayan_2016, Beecroft_1998, Bhattacharjee_2008, Bhumi_2024f, CamachoDorado_2018, Cauchi_2002, Emamhadi_2018, Farhadi_2024h, Fry_2010, Goldman_1998f, Guinan_2019f, Hardy_2023g, Jehangir_2019h, Jin_2023, Kar_2015, Kariholu_2008, Kobiela_2015, Kumar_2001, Kumar_2019f, Li_2013, Liu_2005, Losanoff_1996, Mesfin_2022a, Misra_2013, Naji_2012f, Ohno_2005, Sobnach_2011f, Sultan_2024f, Tammana_2012j, Tanrikulu_2015e, Tay_2004, Thapa_2019f, Wadhwa_2015e, Wildhaber_2005, Yasin_2009, fjbuilsRepeatedBehaviorDeliberate2024, teWildt_2010}, 34 cases (47\%) ingested a sharp object \cite{AlShaaibi_2021b, Alao_2006i, Apikotoa_2022f, Ataya_2013, Benoist_2019e, Bhasin_2014, Bhattacharjee_2008, CamachoDorado_2018, Csaky_1998e, DelgadoSalazar_2020c, DivsalarP._2023a, Emamhadi_2018, Farhadi_2024h, Fry_2010, Guinan_2019f, Hardy_2023g, Jehangir_2019h, Jin_2023, Kariholu_2008, Kobiela_2015, Kumar_2019f, Losanoff_1996, Losanoff_1997e, Mesfin_2022a, Misra_2013, Sobnach_2011f, Yasin_2009, teWildt_2010}, 32 cases (44\%) ingested a long object (\textgreater{}5cm) \cite{Al-Faham_2020k, AlShaaibi_2021b, Ali_2017, Ali_2022g, Atayan_2016, Bhasin_2014, CamachoDorado_2018, Chang_2017f, Cox_2007, Csaky_1998e, DivsalarP._2023a, Emamhadi_2018, Fry_2010, Gardner_2017h, Jin_2023, Kariholu_2008, Kerestes_2019, Kobiela_2015, Kumar_2019f, Mesfin_2022a, Misra_2013, Ohno_2005, Qureshi_2016, Sakellaridis_2008f, Sultan_2024f, Thapa_2019f, Trgo_2012f, Yasin_2009, Yildiz_2016e, teWildt_2010}, 9 cases (12\%) ingested a magnet \cite{Ali_2020f, Bhumi_2024f, Cauchi_2002, Liu_2005, Naji_2012f, Ohno_2005, Tanrikulu_2015e, Tay_2004, Wildhaber_2005}, 2 cases (3\%) ingested a button battery \cite{Berry_2021e, Bhumi_2024f}. \paragraph*{Outcomes} 48 cases (67\%) experienced a complication \cite{Ali_2017, Ali_2020f, Apikotoa_2022f, Atayan_2016, Beecroft_1998, Benoist_2019e, Berry_2021e, Bhasin_2014, Bhumi_2024f, CamachoDorado_2018, Cauchi_2002, Cox_2007, Csaky_1998e, DelgadoSalazar_2020c, DivsalarP._2023a, Emamhadi_2018, Farhadi_2024h, Fry_2010, Gardner_2017h, Goldman_1998f, Jin_2023, Kariholu_2008, Kerestes_2019, Kobiela_2015, Kumar_2001, Kumar_2019f, Liu_2005, Losanoff_1996, Mesfin_2022a, Misra_2013, Naji_2012f, Ohno_2005, Sakellaridis_2008f, Sobnach_2011f, Sultan_2024f, Tanrikulu_2015e, Tay_2004, Thapa_2019f, Trgo_2012f, Tupesis_2004f, Wildhaber_2005, Wnęk_2015f, Yasin_2009, Yildiz_2016e}, 44 cases (61\%) underwent surgery \cite{Al-Faham_2020k, AlShaaibi_2021b, Alao_2006i, Ali_2017, Ali_2020f, Atayan_2016, Beecroft_1998, Bhasin_2014, CamachoDorado_2018, Cauchi_2002, Chang_2017f, Cox_2007, Csaky_1998e, DelgadoSalazar_2020c, DivsalarP._2023a, Farhadi_2024h, Fry_2010, Gardner_2017h, Jin_2023, Kariholu_2008, Kerestes_2019, Kobiela_2015, Kumar_2019f, Liu_2005, Losanoff_1996, Losanoff_1997e, Mesfin_2022a, Misra_2013, Naji_2012f, Sobnach_2011f, Tanrikulu_2015e, Tay_2004, Thapa_2019f, Tupesis_2004f, Wildhaber_2005, Wnęk_2015f, Yasin_2009, Yildiz_2016e, fjbuilsRepeatedBehaviorDeliberate2024}, 31 cases (43\%) underwent endoscopy \cite{Akay_2015f, Ali_2022g, Apikotoa_2022f, Atayan_2016, Benoist_2019e, Berry_2021e, Bhasin_2014, Bhumi_2024f, CamachoDorado_2018, Chang_2017f, DelgadoSalazar_2020c, Gardner_2017h, Guinan_2019f, Hardy_2023g, Jehangir_2019h, Kariholu_2008, Li_2013, Liu_2005, Ohno_2005, Peixoto_2017f, Qureshi_2016, Riva_2018j, Sakellaridis_2008f, Sultan_2024f, Tammana_2012j, Tanrikulu_2015e, Trgo_2012f, Wadhwa_2015e, Wnęk_2015f, teWildt_2010}, 7 cases (10\%) were managed conservatively \cite{Ataya_2013, Bhattacharjee_2008, DivsalarP._2023a, Emamhadi_2018, Goldman_1998f, Kar_2015, Kumar_2001}, 2 cases (3\%) died \cite{Emamhadi_2018, Kumar_2001}. All 90 were male gender. 90 cases (100\%) were detained at the time of ingestion \cite{Elghali_2016, Karp_1991b, Lee_2007}, 88 cases (98\%) were intentional ingestions \cite{Elghali_2016, Karp_1991b, Lee_2007}, 30 cases (33\%) had a psychiatric history documented \cite{Elghali_2016, Karp_1991b, Lee_2007}, 2 cases (2\%) had a history of prior ingestion \cite{Elghali_2016}. No cases were reported for were psychiatric inpatients, were displaced people, were under the influence of alcohol at the time of ingestion, and had a severe disability history.
\paragraph*{Motivation}  70 cases (78\%) reported protest motivation \cite{Elghali_2016, Karp_1991b, Lee_2007}, 12 cases (13\%) reported psychiatric motivation \cite{Karp_1991b}, 6 cases (7\%) reported self-harm motivation \cite{Elghali_2016, Karp_1991b}. No cases were reported for psychosocial motivation and other motivation.
\paragraph*{Object Characteristics}  68 cases (76\%) involved sharp object ingestion \cite{Elghali_2016, Karp_1991b, Lee_2007}, 32 cases (36\%) involved long (\textgreater 5cm) object ingestion \cite{Lee_2007}, 25 cases (28\%) involved ingestion of multiple objects \cite{Elghali_2016, Lee_2007}. No cases were reported for button battery ingestion, magnet ingestion, and involved large diameter (\textgreater 2.5cm) object ingestion.
\paragraph*{Outcomes}  47 cases (52\%) underwent endoscopic intervention \cite{Elghali_2016, Lee_2007}, 29 cases (32\%) were managed conservatively \cite{Elghali_2016, Karp_1991b}, 15 cases (17\%) underwent surgical intervention \cite{Elghali_2016, Karp_1991b, Lee_2007}, 6 cases (7\%) reported complications \cite{Lee_2007}, 1 case (1\%) died \cite{Elghali_2016}.
\paragraph*{Geographical Location}Cases were recorded in 33 countries: 13 cases from USA \cite{Alao_2006i, Ataya_2013, Bhumi_2024f, Fry_2010, Guinan_2019f, Hardy_2023g, Jehangir_2019h, Kerestes_2019, Kumar_2001, Liu_2005, Tammana_2012j, Tay_2004, Tupesis_2004f}; 7 cases from India \cite{Bhasin_2014, Bhattacharjee_2008, Kar_2015, Kariholu_2008, Kumar_2019f, Misra_2013, Wadhwa_2015e} and UK \cite{Beecroft_1998, Berry_2021e, Cauchi_2002, Cox_2007, Gardner_2017h, Qureshi_2016}; 6 cases from Bulgaria \cite{Losanoff_1996, Losanoff_1997e}; 5 cases from Iran \cite{DivsalarP._2023a, Emamhadi_2018, Farhadi_2024h}; 4 cases from Turkey \cite{Akay_2015f, Atayan_2016, Tanrikulu_2015e, Yildiz_2016e}; 2 cases from China \cite{Jin_2023, Li_2013}, Poland \cite{Kobiela_2015, Wnęk_2015f}, and Spain \cite{CamachoDorado_2018, fjbuilsRepeatedBehaviorDeliberate2024}; 1 case from Australia \cite{Apikotoa_2022f}, Bahrain \cite{Ali_2020f}, Croatia \cite{Trgo_2012f}, Ecuador \cite{DelgadoSalazar_2020c}, Egypt \cite{Ali_2022g}, Ethiopia \cite{Mesfin_2022a}, Germany \cite{teWildt_2010}, Greece \cite{Sakellaridis_2008f}, Hungary \cite{Csaky_1998e}, Iraq \cite{Al-Faham_2020k}, Israel \cite{Goldman_1998f}, Italy \cite{Riva_2018j}, Japan \cite{Ohno_2005}, Nepal \cite{Thapa_2019f}, Netherlands \cite{Benoist_2019e}, Oman \cite{AlShaaibi_2021b}, Pakistan \cite{Yasin_2009}, Portugal \cite{Peixoto_2017f}, Qatar \cite{Ali_2017}, Saudi Arabia \cite{Sultan_2024f}, South Africa \cite{Sobnach_2011f}, Sweden \cite{Naji_2012f}, Switzerland \cite{Wildhaber_2005}, and Taiwan \cite{Chang_2017f}. \paragraph*{Gender} 43 cases (60\%) were male \cite{Akay_2015f, Al-Faham_2020k, Alao_2006i, Ali_2017, Ali_2022g, Apikotoa_2022f, Atayan_2016, Benoist_2019e, Berry_2021e, Bhumi_2024f, CamachoDorado_2018, Csaky_1998e, Emamhadi_2018, Farhadi_2024h, Fry_2010, Gardner_2017h, Guinan_2019f, Jehangir_2019h, Jin_2023, Kobiela_2015, Kumar_2001, Kumar_2019f, Liu_2005, Losanoff_1996, Losanoff_1997e, Mesfin_2022a, Misra_2013, Qureshi_2016, Riva_2018j, Sobnach_2011f, Tammana_2012j, Tanrikulu_2015e, Tay_2004, Thapa_2019f, Trgo_2012f, Wadhwa_2015e, Yasin_2009, teWildt_2010}, 28 cases (39\%) were female \cite{AlShaaibi_2021b, Ali_2020f, Ataya_2013, Beecroft_1998, Bhasin_2014, Bhattacharjee_2008, Cauchi_2002, Chang_2017f, Cox_2007, DelgadoSalazar_2020c, DivsalarP._2023a, Goldman_1998f, Hardy_2023g, Kar_2015, Kariholu_2008, Kerestes_2019, Li_2013, Naji_2012f, Ohno_2005, Peixoto_2017f, Sakellaridis_2008f, Sultan_2024f, Tupesis_2004f, Wildhaber_2005, Wnęk_2015f, Yildiz_2016e}, 1 case (1\%) had no gender recorded \cite{fjbuilsRepeatedBehaviorDeliberate2024}. \paragraph*{Age Group} 25 cases (35\%) were between 26 and 40 years of age \cite{Alao_2006i, Ali_2022g, Apikotoa_2022f, Ataya_2013, Benoist_2019e, Bhasin_2014, Chang_2017f, Cox_2007, DelgadoSalazar_2020c, Farhadi_2024h, Fry_2010, Gardner_2017h, Guinan_2019f, Jin_2023, Kumar_2019f, Losanoff_1996, Misra_2013, Qureshi_2016, Riva_2018j, Sakellaridis_2008f, Tammana_2012j, Trgo_2012f, Wnęk_2015f, Yildiz_2016e, fjbuilsRepeatedBehaviorDeliberate2024}, 18 cases (25\%) were between 18 and 25 years of age \cite{Akay_2015f, Ali_2017, Atayan_2016, Bhattacharjee_2008, Csaky_1998e, Kar_2015, Kariholu_2008, Kobiela_2015, Losanoff_1996, Losanoff_1997e, Mesfin_2022a, Peixoto_2017f, Sobnach_2011f, Tupesis_2004f, Yasin_2009}, 13 cases (18\%) were under 18 years of age \cite{AlShaaibi_2021b, Ali_2020f, Cauchi_2002, DivsalarP._2023a, Goldman_1998f, Liu_2005, Naji_2012f, Ohno_2005, Tanrikulu_2015e, Tay_2004, Wildhaber_2005}, 11 cases (15\%) were between 41 and 60 years of age \cite{Al-Faham_2020k, Bhumi_2024f, CamachoDorado_2018, Emamhadi_2018, Hardy_2023g, Jehangir_2019h, Kumar_2001, Sultan_2024f, Thapa_2019f, Wadhwa_2015e, teWildt_2010}, 3 cases (4\%) were over 60 years of age \cite{Beecroft_1998, Kerestes_2019, Li_2013}, 2 cases (3\%) had no age documented \cite{Berry_2021e}. \paragraph*{Population} 36 cases (50\%) had a psychiatric history \cite{AlShaaibi_2021b, Alao_2006i, Ali_2020f, Apikotoa_2022f, Ataya_2013, Atayan_2016, Beecroft_1998, CamachoDorado_2018, Chang_2017f, DelgadoSalazar_2020c, DivsalarP._2023a, Farhadi_2024h, Fry_2010, Guinan_2019f, Hardy_2023g, Jehangir_2019h, Jin_2023, Kar_2015, Kerestes_2019, Kobiela_2015, Kumar_2001, Kumar_2019f, Liu_2005, Mesfin_2022a, Misra_2013, Ohno_2005, Peixoto_2017f, Sakellaridis_2008f, Sultan_2024f, Tammana_2012j, Tanrikulu_2015e, Yildiz_2016e, fjbuilsRepeatedBehaviorDeliberate2024, teWildt_2010}, 19 cases (26\%) had ingested previously \cite{Alao_2006i, Apikotoa_2022f, Berry_2021e, Bhattacharjee_2008, Csaky_1998e, DivsalarP._2023a, Emamhadi_2018, Guinan_2019f, Jehangir_2019h, Jin_2023, Liu_2005, Sakellaridis_2008f, Tanrikulu_2015e, Thapa_2019f, Yildiz_2016e, fjbuilsRepeatedBehaviorDeliberate2024, teWildt_2010}, 12 cases (17\%) were detained persons \cite{Alao_2006i, Ali_2022g, Apikotoa_2022f, Losanoff_1996, Losanoff_1997e, Qureshi_2016, Tammana_2012j, Trgo_2012f}, 7 cases (10\%) were severely disabled \cite{Atayan_2016, Kerestes_2019, Liu_2005, Ohno_2005, Peixoto_2017f, Yildiz_2016e, teWildt_2010}, 4 cases (6\%) were psychiatric inpatients \cite{DivsalarP._2023a, fjbuilsRepeatedBehaviorDeliberate2024, teWildt_2010}, 3 cases (4\%) were under the influence of alcohol \cite{Benoist_2019e, Csaky_1998e, Thapa_2019f}, 2 cases (3\%) were displaced people \cite{Akay_2015f, Gardner_2017h}. \paragraph*{Motivation} 34 cases (47\%) had a psychiatric motivation \cite{Al-Faham_2020k, Alao_2006i, Ali_2020f, Apikotoa_2022f, Ataya_2013, Atayan_2016, Bhasin_2014, Bhattacharjee_2008, DelgadoSalazar_2020c, DivsalarP._2023a, Emamhadi_2018, Farhadi_2024h, Guinan_2019f, Hardy_2023g, Jehangir_2019h, Jin_2023, Kar_2015, Kariholu_2008, Kerestes_2019, Kobiela_2015, Kumar_2001, Kumar_2019f, Li_2013, Liu_2005, Misra_2013, Ohno_2005, Sakellaridis_2008f, Sultan_2024f, Tammana_2012j, Tanrikulu_2015e, Yasin_2009, teWildt_2010}, 21 cases (29\%) were motivated by self-harm intention \cite{Al-Faham_2020k, AlShaaibi_2021b, Alao_2006i, Ali_2017, CamachoDorado_2018, Chang_2017f, Cox_2007, Csaky_1998e, Fry_2010, Li_2013, Losanoff_1996, Losanoff_1997e, Mesfin_2022a, Sakellaridis_2008f, Tammana_2012j, Tanrikulu_2015e, fjbuilsRepeatedBehaviorDeliberate2024}, 17 cases (24\%) had a psychosocial motivation \cite{Akay_2015f, Benoist_2019e, Bhattacharjee_2008, Cauchi_2002, Goldman_1998f, Hardy_2023g, Kobiela_2015, Li_2013, Naji_2012f, Qureshi_2016, Riva_2018j, Sobnach_2011f, Tay_2004, Thapa_2019f, Tupesis_2004f, Wildhaber_2005, Wnęk_2015f}, 9 cases (12\%) were motivated by protest \cite{Bhumi_2024f, Gardner_2017h, Losanoff_1996, Losanoff_1997e, Tupesis_2004f}, 9 cases (12\%) had another documented motivation \cite{Ali_2020f, Ali_2022g, Emamhadi_2018, Guinan_2019f, Peixoto_2017f, Sakellaridis_2008f, Trgo_2012f, Wadhwa_2015e, Yildiz_2016e}. \paragraph*{Object Characteristics} 51 cases (71\%) ingested a large diameter object (\textgreater{}2.5cm) \cite{Akay_2015f, Al-Faham_2020k, AlShaaibi_2021b, Alao_2006i, Ali_2017, Ali_2022g, Apikotoa_2022f, Atayan_2016, Berry_2021e, Bhasin_2014, CamachoDorado_2018, Cauchi_2002, Chang_2017f, Cox_2007, Csaky_1998e, DivsalarP._2023a, Emamhadi_2018, Gardner_2017h, Guinan_2019f, Jehangir_2019h, Jin_2023, Kariholu_2008, Kerestes_2019, Kobiela_2015, Kumar_2001, Kumar_2019f, Losanoff_1996, Losanoff_1997e, Mesfin_2022a, Misra_2013, Naji_2012f, Ohno_2005, Peixoto_2017f, Qureshi_2016, Riva_2018j, Sakellaridis_2008f, Sultan_2024f, Tanrikulu_2015e, Thapa_2019f, Trgo_2012f, Wnęk_2015f, Yildiz_2016e, fjbuilsRepeatedBehaviorDeliberate2024, teWildt_2010}, 44 cases (61\%) ingested multiple objects \cite{Ali_2020f, Apikotoa_2022f, Ataya_2013, Atayan_2016, Beecroft_1998, Bhattacharjee_2008, Bhumi_2024f, CamachoDorado_2018, Cauchi_2002, Emamhadi_2018, Farhadi_2024h, Fry_2010, Goldman_1998f, Guinan_2019f, Hardy_2023g, Jehangir_2019h, Jin_2023, Kar_2015, Kariholu_2008, Kobiela_2015, Kumar_2001, Kumar_2019f, Li_2013, Liu_2005, Losanoff_1996, Mesfin_2022a, Misra_2013, Naji_2012f, Ohno_2005, Sobnach_2011f, Sultan_2024f, Tammana_2012j, Tanrikulu_2015e, Tay_2004, Thapa_2019f, Wadhwa_2015e, Wildhaber_2005, Yasin_2009, fjbuilsRepeatedBehaviorDeliberate2024, teWildt_2010}, 34 cases (47\%) ingested a sharp object \cite{AlShaaibi_2021b, Alao_2006i, Apikotoa_2022f, Ataya_2013, Benoist_2019e, Bhasin_2014, Bhattacharjee_2008, CamachoDorado_2018, Csaky_1998e, DelgadoSalazar_2020c, DivsalarP._2023a, Emamhadi_2018, Farhadi_2024h, Fry_2010, Guinan_2019f, Hardy_2023g, Jehangir_2019h, Jin_2023, Kariholu_2008, Kobiela_2015, Kumar_2019f, Losanoff_1996, Losanoff_1997e, Mesfin_2022a, Misra_2013, Sobnach_2011f, Yasin_2009, teWildt_2010}, 32 cases (44\%) ingested a long object (\textgreater{}5cm) \cite{Al-Faham_2020k, AlShaaibi_2021b, Ali_2017, Ali_2022g, Atayan_2016, Bhasin_2014, CamachoDorado_2018, Chang_2017f, Cox_2007, Csaky_1998e, DivsalarP._2023a, Emamhadi_2018, Fry_2010, Gardner_2017h, Jin_2023, Kariholu_2008, Kerestes_2019, Kobiela_2015, Kumar_2019f, Mesfin_2022a, Misra_2013, Ohno_2005, Qureshi_2016, Sakellaridis_2008f, Sultan_2024f, Thapa_2019f, Trgo_2012f, Yasin_2009, Yildiz_2016e, teWildt_2010}, 9 cases (12\%) ingested a magnet \cite{Ali_2020f, Bhumi_2024f, Cauchi_2002, Liu_2005, Naji_2012f, Ohno_2005, Tanrikulu_2015e, Tay_2004, Wildhaber_2005}, 2 cases (3\%) ingested a button battery \cite{Berry_2021e, Bhumi_2024f}. \paragraph*{Outcomes} 48 cases (67\%) experienced a complication \cite{Ali_2017, Ali_2020f, Apikotoa_2022f, Atayan_2016, Beecroft_1998, Benoist_2019e, Berry_2021e, Bhasin_2014, Bhumi_2024f, CamachoDorado_2018, Cauchi_2002, Cox_2007, Csaky_1998e, DelgadoSalazar_2020c, DivsalarP._2023a, Emamhadi_2018, Farhadi_2024h, Fry_2010, Gardner_2017h, Goldman_1998f, Jin_2023, Kariholu_2008, Kerestes_2019, Kobiela_2015, Kumar_2001, Kumar_2019f, Liu_2005, Losanoff_1996, Mesfin_2022a, Misra_2013, Naji_2012f, Ohno_2005, Sakellaridis_2008f, Sobnach_2011f, Sultan_2024f, Tanrikulu_2015e, Tay_2004, Thapa_2019f, Trgo_2012f, Tupesis_2004f, Wildhaber_2005, Wnęk_2015f, Yasin_2009, Yildiz_2016e}, 44 cases (61\%) underwent surgery \cite{Al-Faham_2020k, AlShaaibi_2021b, Alao_2006i, Ali_2017, Ali_2020f, Atayan_2016, Beecroft_1998, Bhasin_2014, CamachoDorado_2018, Cauchi_2002, Chang_2017f, Cox_2007, Csaky_1998e, DelgadoSalazar_2020c, DivsalarP._2023a, Farhadi_2024h, Fry_2010, Gardner_2017h, Jin_2023, Kariholu_2008, Kerestes_2019, Kobiela_2015, Kumar_2019f, Liu_2005, Losanoff_1996, Losanoff_1997e, Mesfin_2022a, Misra_2013, Naji_2012f, Sobnach_2011f, Tanrikulu_2015e, Tay_2004, Thapa_2019f, Tupesis_2004f, Wildhaber_2005, Wnęk_2015f, Yasin_2009, Yildiz_2016e, fjbuilsRepeatedBehaviorDeliberate2024}, 31 cases (43\%) underwent endoscopy \cite{Akay_2015f, Ali_2022g, Apikotoa_2022f, Atayan_2016, Benoist_2019e, Berry_2021e, Bhasin_2014, Bhumi_2024f, CamachoDorado_2018, Chang_2017f, DelgadoSalazar_2020c, Gardner_2017h, Guinan_2019f, Hardy_2023g, Jehangir_2019h, Kariholu_2008, Li_2013, Liu_2005, Ohno_2005, Peixoto_2017f, Qureshi_2016, Riva_2018j, Sakellaridis_2008f, Sultan_2024f, Tammana_2012j, Tanrikulu_2015e, Trgo_2012f, Wadhwa_2015e, Wnęk_2015f, teWildt_2010}, 7 cases (10\%) were managed conservatively \cite{Ataya_2013, Bhattacharjee_2008, DivsalarP._2023a, Emamhadi_2018, Goldman_1998f, Kar_2015, Kumar_2001}, 2 cases (3\%) died \cite{Emamhadi_2018, Kumar_2001}. All 90 were male gender. 90 cases (100\%) were detained at the time of ingestion \cite{Elghali_2016, Karp_1991b, Lee_2007}, 88 cases (98\%) were intentional ingestions \cite{Elghali_2016, Karp_1991b, Lee_2007}, 30 cases (33\%) had a psychiatric history documented \cite{Elghali_2016, Karp_1991b, Lee_2007}, 2 cases (2\%) had a history of prior ingestion \cite{Elghali_2016}. No cases were reported for were psychiatric inpatients, were displaced people, were under the influence of alcohol at the time of ingestion, and had a severe disability history.
\paragraph*{Motivation}  70 cases (78\%) reported protest motivation \cite{Elghali_2016, Karp_1991b, Lee_2007}, 12 cases (13\%) reported psychiatric motivation \cite{Karp_1991b}, 6 cases (7\%) reported self-harm motivation \cite{Elghali_2016, Karp_1991b}. No cases were reported for psychosocial motivation and other motivation.
\paragraph*{Object Characteristics}  68 cases (76\%) involved sharp object ingestion \cite{Elghali_2016, Karp_1991b, Lee_2007}, 32 cases (36\%) involved long (\textgreater 5cm) object ingestion \cite{Lee_2007}, 25 cases (28\%) involved ingestion of multiple objects \cite{Elghali_2016, Lee_2007}. No cases were reported for button battery ingestion, magnet ingestion, and involved large diameter (\textgreater 2.5cm) object ingestion.
\paragraph*{Outcomes}  47 cases (52\%) underwent endoscopic intervention \cite{Elghali_2016, Lee_2007}, 29 cases (32\%) were managed conservatively \cite{Elghali_2016, Karp_1991b}, 15 cases (17\%) underwent surgical intervention \cite{Elghali_2016, Karp_1991b, Lee_2007}, 6 cases (7\%) reported complications \cite{Lee_2007}, 1 case (1\%) died \cite{Elghali_2016}.
\paragraph*{Geographical Location}Cases were recorded in 33 countries: 13 cases from USA \cite{Alao_2006i, Ataya_2013, Bhumi_2024f, Fry_2010, Guinan_2019f, Hardy_2023g, Jehangir_2019h, Kerestes_2019, Kumar_2001, Liu_2005, Tammana_2012j, Tay_2004, Tupesis_2004f}; 7 cases from India \cite{Bhasin_2014, Bhattacharjee_2008, Kar_2015, Kariholu_2008, Kumar_2019f, Misra_2013, Wadhwa_2015e} and UK \cite{Beecroft_1998, Berry_2021e, Cauchi_2002, Cox_2007, Gardner_2017h, Qureshi_2016}; 6 cases from Bulgaria \cite{Losanoff_1996, Losanoff_1997e}; 5 cases from Iran \cite{DivsalarP._2023a, Emamhadi_2018, Farhadi_2024h}; 4 cases from Turkey \cite{Akay_2015f, Atayan_2016, Tanrikulu_2015e, Yildiz_2016e}; 2 cases from China \cite{Jin_2023, Li_2013}, Poland \cite{Kobiela_2015, Wnęk_2015f}, and Spain \cite{CamachoDorado_2018, fjbuilsRepeatedBehaviorDeliberate2024}; 1 case from Australia \cite{Apikotoa_2022f}, Bahrain \cite{Ali_2020f}, Croatia \cite{Trgo_2012f}, Ecuador \cite{DelgadoSalazar_2020c}, Egypt \cite{Ali_2022g}, Ethiopia \cite{Mesfin_2022a}, Germany \cite{teWildt_2010}, Greece \cite{Sakellaridis_2008f}, Hungary \cite{Csaky_1998e}, Iraq \cite{Al-Faham_2020k}, Israel \cite{Goldman_1998f}, Italy \cite{Riva_2018j}, Japan \cite{Ohno_2005}, Nepal \cite{Thapa_2019f}, Netherlands \cite{Benoist_2019e}, Oman \cite{AlShaaibi_2021b}, Pakistan \cite{Yasin_2009}, Portugal \cite{Peixoto_2017f}, Qatar \cite{Ali_2017}, Saudi Arabia \cite{Sultan_2024f}, South Africa \cite{Sobnach_2011f}, Sweden \cite{Naji_2012f}, Switzerland \cite{Wildhaber_2005}, and Taiwan \cite{Chang_2017f}. \paragraph*{Gender} 43 cases (60\%) were male \cite{Akay_2015f, Al-Faham_2020k, Alao_2006i, Ali_2017, Ali_2022g, Apikotoa_2022f, Atayan_2016, Benoist_2019e, Berry_2021e, Bhumi_2024f, CamachoDorado_2018, Csaky_1998e, Emamhadi_2018, Farhadi_2024h, Fry_2010, Gardner_2017h, Guinan_2019f, Jehangir_2019h, Jin_2023, Kobiela_2015, Kumar_2001, Kumar_2019f, Liu_2005, Losanoff_1996, Losanoff_1997e, Mesfin_2022a, Misra_2013, Qureshi_2016, Riva_2018j, Sobnach_2011f, Tammana_2012j, Tanrikulu_2015e, Tay_2004, Thapa_2019f, Trgo_2012f, Wadhwa_2015e, Yasin_2009, teWildt_2010}, 28 cases (39\%) were female \cite{AlShaaibi_2021b, Ali_2020f, Ataya_2013, Beecroft_1998, Bhasin_2014, Bhattacharjee_2008, Cauchi_2002, Chang_2017f, Cox_2007, DelgadoSalazar_2020c, DivsalarP._2023a, Goldman_1998f, Hardy_2023g, Kar_2015, Kariholu_2008, Kerestes_2019, Li_2013, Naji_2012f, Ohno_2005, Peixoto_2017f, Sakellaridis_2008f, Sultan_2024f, Tupesis_2004f, Wildhaber_2005, Wnęk_2015f, Yildiz_2016e}, 1 case (1\%) had no gender recorded \cite{fjbuilsRepeatedBehaviorDeliberate2024}. \paragraph*{Age Group} 25 cases (35\%) were between 26 and 40 years of age \cite{Alao_2006i, Ali_2022g, Apikotoa_2022f, Ataya_2013, Benoist_2019e, Bhasin_2014, Chang_2017f, Cox_2007, DelgadoSalazar_2020c, Farhadi_2024h, Fry_2010, Gardner_2017h, Guinan_2019f, Jin_2023, Kumar_2019f, Losanoff_1996, Misra_2013, Qureshi_2016, Riva_2018j, Sakellaridis_2008f, Tammana_2012j, Trgo_2012f, Wnęk_2015f, Yildiz_2016e, fjbuilsRepeatedBehaviorDeliberate2024}, 18 cases (25\%) were between 18 and 25 years of age \cite{Akay_2015f, Ali_2017, Atayan_2016, Bhattacharjee_2008, Csaky_1998e, Kar_2015, Kariholu_2008, Kobiela_2015, Losanoff_1996, Losanoff_1997e, Mesfin_2022a, Peixoto_2017f, Sobnach_2011f, Tupesis_2004f, Yasin_2009}, 13 cases (18\%) were under 18 years of age \cite{AlShaaibi_2021b, Ali_2020f, Cauchi_2002, DivsalarP._2023a, Goldman_1998f, Liu_2005, Naji_2012f, Ohno_2005, Tanrikulu_2015e, Tay_2004, Wildhaber_2005}, 11 cases (15\%) were between 41 and 60 years of age \cite{Al-Faham_2020k, Bhumi_2024f, CamachoDorado_2018, Emamhadi_2018, Hardy_2023g, Jehangir_2019h, Kumar_2001, Sultan_2024f, Thapa_2019f, Wadhwa_2015e, teWildt_2010}, 3 cases (4\%) were over 60 years of age \cite{Beecroft_1998, Kerestes_2019, Li_2013}, 2 cases (3\%) had no age documented \cite{Berry_2021e}. \paragraph*{Population} 36 cases (50\%) had a psychiatric history \cite{AlShaaibi_2021b, Alao_2006i, Ali_2020f, Apikotoa_2022f, Ataya_2013, Atayan_2016, Beecroft_1998, CamachoDorado_2018, Chang_2017f, DelgadoSalazar_2020c, DivsalarP._2023a, Farhadi_2024h, Fry_2010, Guinan_2019f, Hardy_2023g, Jehangir_2019h, Jin_2023, Kar_2015, Kerestes_2019, Kobiela_2015, Kumar_2001, Kumar_2019f, Liu_2005, Mesfin_2022a, Misra_2013, Ohno_2005, Peixoto_2017f, Sakellaridis_2008f, Sultan_2024f, Tammana_2012j, Tanrikulu_2015e, Yildiz_2016e, fjbuilsRepeatedBehaviorDeliberate2024, teWildt_2010}, 19 cases (26\%) had ingested previously \cite{Alao_2006i, Apikotoa_2022f, Berry_2021e, Bhattacharjee_2008, Csaky_1998e, DivsalarP._2023a, Emamhadi_2018, Guinan_2019f, Jehangir_2019h, Jin_2023, Liu_2005, Sakellaridis_2008f, Tanrikulu_2015e, Thapa_2019f, Yildiz_2016e, fjbuilsRepeatedBehaviorDeliberate2024, teWildt_2010}, 12 cases (17\%) were detained persons \cite{Alao_2006i, Ali_2022g, Apikotoa_2022f, Losanoff_1996, Losanoff_1997e, Qureshi_2016, Tammana_2012j, Trgo_2012f}, 7 cases (10\%) were severely disabled \cite{Atayan_2016, Kerestes_2019, Liu_2005, Ohno_2005, Peixoto_2017f, Yildiz_2016e, teWildt_2010}, 4 cases (6\%) were psychiatric inpatients \cite{DivsalarP._2023a, fjbuilsRepeatedBehaviorDeliberate2024, teWildt_2010}, 3 cases (4\%) were under the influence of alcohol \cite{Benoist_2019e, Csaky_1998e, Thapa_2019f}, 2 cases (3\%) were displaced people \cite{Akay_2015f, Gardner_2017h}. \paragraph*{Motivation} 34 cases (47\%) had a psychiatric motivation \cite{Al-Faham_2020k, Alao_2006i, Ali_2020f, Apikotoa_2022f, Ataya_2013, Atayan_2016, Bhasin_2014, Bhattacharjee_2008, DelgadoSalazar_2020c, DivsalarP._2023a, Emamhadi_2018, Farhadi_2024h, Guinan_2019f, Hardy_2023g, Jehangir_2019h, Jin_2023, Kar_2015, Kariholu_2008, Kerestes_2019, Kobiela_2015, Kumar_2001, Kumar_2019f, Li_2013, Liu_2005, Misra_2013, Ohno_2005, Sakellaridis_2008f, Sultan_2024f, Tammana_2012j, Tanrikulu_2015e, Yasin_2009, teWildt_2010}, 21 cases (29\%) were motivated by self-harm intention \cite{Al-Faham_2020k, AlShaaibi_2021b, Alao_2006i, Ali_2017, CamachoDorado_2018, Chang_2017f, Cox_2007, Csaky_1998e, Fry_2010, Li_2013, Losanoff_1996, Losanoff_1997e, Mesfin_2022a, Sakellaridis_2008f, Tammana_2012j, Tanrikulu_2015e, fjbuilsRepeatedBehaviorDeliberate2024}, 17 cases (24\%) had a psychosocial motivation \cite{Akay_2015f, Benoist_2019e, Bhattacharjee_2008, Cauchi_2002, Goldman_1998f, Hardy_2023g, Kobiela_2015, Li_2013, Naji_2012f, Qureshi_2016, Riva_2018j, Sobnach_2011f, Tay_2004, Thapa_2019f, Tupesis_2004f, Wildhaber_2005, Wnęk_2015f}, 9 cases (12\%) were motivated by protest \cite{Bhumi_2024f, Gardner_2017h, Losanoff_1996, Losanoff_1997e, Tupesis_2004f}, 9 cases (12\%) had another documented motivation \cite{Ali_2020f, Ali_2022g, Emamhadi_2018, Guinan_2019f, Peixoto_2017f, Sakellaridis_2008f, Trgo_2012f, Wadhwa_2015e, Yildiz_2016e}. \paragraph*{Object Characteristics} 51 cases (71\%) ingested a large diameter object (\textgreater{}2.5cm) \cite{Akay_2015f, Al-Faham_2020k, AlShaaibi_2021b, Alao_2006i, Ali_2017, Ali_2022g, Apikotoa_2022f, Atayan_2016, Berry_2021e, Bhasin_2014, CamachoDorado_2018, Cauchi_2002, Chang_2017f, Cox_2007, Csaky_1998e, DivsalarP._2023a, Emamhadi_2018, Gardner_2017h, Guinan_2019f, Jehangir_2019h, Jin_2023, Kariholu_2008, Kerestes_2019, Kobiela_2015, Kumar_2001, Kumar_2019f, Losanoff_1996, Losanoff_1997e, Mesfin_2022a, Misra_2013, Naji_2012f, Ohno_2005, Peixoto_2017f, Qureshi_2016, Riva_2018j, Sakellaridis_2008f, Sultan_2024f, Tanrikulu_2015e, Thapa_2019f, Trgo_2012f, Wnęk_2015f, Yildiz_2016e, fjbuilsRepeatedBehaviorDeliberate2024, teWildt_2010}, 44 cases (61\%) ingested multiple objects \cite{Ali_2020f, Apikotoa_2022f, Ataya_2013, Atayan_2016, Beecroft_1998, Bhattacharjee_2008, Bhumi_2024f, CamachoDorado_2018, Cauchi_2002, Emamhadi_2018, Farhadi_2024h, Fry_2010, Goldman_1998f, Guinan_2019f, Hardy_2023g, Jehangir_2019h, Jin_2023, Kar_2015, Kariholu_2008, Kobiela_2015, Kumar_2001, Kumar_2019f, Li_2013, Liu_2005, Losanoff_1996, Mesfin_2022a, Misra_2013, Naji_2012f, Ohno_2005, Sobnach_2011f, Sultan_2024f, Tammana_2012j, Tanrikulu_2015e, Tay_2004, Thapa_2019f, Wadhwa_2015e, Wildhaber_2005, Yasin_2009, fjbuilsRepeatedBehaviorDeliberate2024, teWildt_2010}, 34 cases (47\%) ingested a sharp object \cite{AlShaaibi_2021b, Alao_2006i, Apikotoa_2022f, Ataya_2013, Benoist_2019e, Bhasin_2014, Bhattacharjee_2008, CamachoDorado_2018, Csaky_1998e, DelgadoSalazar_2020c, DivsalarP._2023a, Emamhadi_2018, Farhadi_2024h, Fry_2010, Guinan_2019f, Hardy_2023g, Jehangir_2019h, Jin_2023, Kariholu_2008, Kobiela_2015, Kumar_2019f, Losanoff_1996, Losanoff_1997e, Mesfin_2022a, Misra_2013, Sobnach_2011f, Yasin_2009, teWildt_2010}, 32 cases (44\%) ingested a long object (\textgreater{}5cm) \cite{Al-Faham_2020k, AlShaaibi_2021b, Ali_2017, Ali_2022g, Atayan_2016, Bhasin_2014, CamachoDorado_2018, Chang_2017f, Cox_2007, Csaky_1998e, DivsalarP._2023a, Emamhadi_2018, Fry_2010, Gardner_2017h, Jin_2023, Kariholu_2008, Kerestes_2019, Kobiela_2015, Kumar_2019f, Mesfin_2022a, Misra_2013, Ohno_2005, Qureshi_2016, Sakellaridis_2008f, Sultan_2024f, Thapa_2019f, Trgo_2012f, Yasin_2009, Yildiz_2016e, teWildt_2010}, 9 cases (12\%) ingested a magnet \cite{Ali_2020f, Bhumi_2024f, Cauchi_2002, Liu_2005, Naji_2012f, Ohno_2005, Tanrikulu_2015e, Tay_2004, Wildhaber_2005}, 2 cases (3\%) ingested a button battery \cite{Berry_2021e, Bhumi_2024f}. \paragraph*{Outcomes} 48 cases (67\%) experienced a complication \cite{Ali_2017, Ali_2020f, Apikotoa_2022f, Atayan_2016, Beecroft_1998, Benoist_2019e, Berry_2021e, Bhasin_2014, Bhumi_2024f, CamachoDorado_2018, Cauchi_2002, Cox_2007, Csaky_1998e, DelgadoSalazar_2020c, DivsalarP._2023a, Emamhadi_2018, Farhadi_2024h, Fry_2010, Gardner_2017h, Goldman_1998f, Jin_2023, Kariholu_2008, Kerestes_2019, Kobiela_2015, Kumar_2001, Kumar_2019f, Liu_2005, Losanoff_1996, Mesfin_2022a, Misra_2013, Naji_2012f, Ohno_2005, Sakellaridis_2008f, Sobnach_2011f, Sultan_2024f, Tanrikulu_2015e, Tay_2004, Thapa_2019f, Trgo_2012f, Tupesis_2004f, Wildhaber_2005, Wnęk_2015f, Yasin_2009, Yildiz_2016e}, 44 cases (61\%) underwent surgery \cite{Al-Faham_2020k, AlShaaibi_2021b, Alao_2006i, Ali_2017, Ali_2020f, Atayan_2016, Beecroft_1998, Bhasin_2014, CamachoDorado_2018, Cauchi_2002, Chang_2017f, Cox_2007, Csaky_1998e, DelgadoSalazar_2020c, DivsalarP._2023a, Farhadi_2024h, Fry_2010, Gardner_2017h, Jin_2023, Kariholu_2008, Kerestes_2019, Kobiela_2015, Kumar_2019f, Liu_2005, Losanoff_1996, Losanoff_1997e, Mesfin_2022a, Misra_2013, Naji_2012f, Sobnach_2011f, Tanrikulu_2015e, Tay_2004, Thapa_2019f, Tupesis_2004f, Wildhaber_2005, Wnęk_2015f, Yasin_2009, Yildiz_2016e, fjbuilsRepeatedBehaviorDeliberate2024}, 31 cases (43\%) underwent endoscopy \cite{Akay_2015f, Ali_2022g, Apikotoa_2022f, Atayan_2016, Benoist_2019e, Berry_2021e, Bhasin_2014, Bhumi_2024f, CamachoDorado_2018, Chang_2017f, DelgadoSalazar_2020c, Gardner_2017h, Guinan_2019f, Hardy_2023g, Jehangir_2019h, Kariholu_2008, Li_2013, Liu_2005, Ohno_2005, Peixoto_2017f, Qureshi_2016, Riva_2018j, Sakellaridis_2008f, Sultan_2024f, Tammana_2012j, Tanrikulu_2015e, Trgo_2012f, Wadhwa_2015e, Wnęk_2015f, teWildt_2010}, 7 cases (10\%) were managed conservatively \cite{Ataya_2013, Bhattacharjee_2008, DivsalarP._2023a, Emamhadi_2018, Goldman_1998f, Kar_2015, Kumar_2001}, 2 cases (3\%) died \cite{Emamhadi_2018, Kumar_2001}. All 90 were male gender. 90 cases (100\%) were detained at the time of ingestion \cite{Elghali_2016, Karp_1991b, Lee_2007}, 88 cases (98\%) were intentional ingestions \cite{Elghali_2016, Karp_1991b, Lee_2007}, 30 cases (33\%) had a psychiatric history documented \cite{Elghali_2016, Karp_1991b, Lee_2007}, 2 cases (2\%) had a history of prior ingestion \cite{Elghali_2016}. No cases were reported for were psychiatric inpatients, were displaced people, were under the influence of alcohol at the time of ingestion, and had a severe disability history.
\paragraph*{Motivation}  70 cases (78\%) reported protest motivation \cite{Elghali_2016, Karp_1991b, Lee_2007}, 12 cases (13\%) reported psychiatric motivation \cite{Karp_1991b}, 6 cases (7\%) reported self-harm motivation \cite{Elghali_2016, Karp_1991b}. No cases were reported for psychosocial motivation and other motivation.
\paragraph*{Object Characteristics}  68 cases (76\%) involved sharp object ingestion \cite{Elghali_2016, Karp_1991b, Lee_2007}, 32 cases (36\%) involved long (\textgreater 5cm) object ingestion \cite{Lee_2007}, 25 cases (28\%) involved ingestion of multiple objects \cite{Elghali_2016, Lee_2007}. No cases were reported for button battery ingestion, magnet ingestion, and involved large diameter (\textgreater 2.5cm) object ingestion.
\paragraph*{Outcomes}  47 cases (52\%) underwent endoscopic intervention \cite{Elghali_2016, Lee_2007}, 29 cases (32\%) were managed conservatively \cite{Elghali_2016, Karp_1991b}, 15 cases (17\%) underwent surgical intervention \cite{Elghali_2016, Karp_1991b, Lee_2007}, 6 cases (7\%) reported complications \cite{Lee_2007}, 1 case (1\%) died \cite{Elghali_2016}.
\paragraph*{Geographical Location}Cases were recorded in 33 countries: 13 cases from USA \cite{Alao_2006i, Ataya_2013, Bhumi_2024f, Fry_2010, Guinan_2019f, Hardy_2023g, Jehangir_2019h, Kerestes_2019, Kumar_2001, Liu_2005, Tammana_2012j, Tay_2004, Tupesis_2004f}; 7 cases from India \cite{Bhasin_2014, Bhattacharjee_2008, Kar_2015, Kariholu_2008, Kumar_2019f, Misra_2013, Wadhwa_2015e} and UK \cite{Beecroft_1998, Berry_2021e, Cauchi_2002, Cox_2007, Gardner_2017h, Qureshi_2016}; 6 cases from Bulgaria \cite{Losanoff_1996, Losanoff_1997e}; 5 cases from Iran \cite{DivsalarP._2023a, Emamhadi_2018, Farhadi_2024h}; 4 cases from Turkey \cite{Akay_2015f, Atayan_2016, Tanrikulu_2015e, Yildiz_2016e}; 2 cases from China \cite{Jin_2023, Li_2013}, Poland \cite{Kobiela_2015, Wnęk_2015f}, and Spain \cite{CamachoDorado_2018, fjbuilsRepeatedBehaviorDeliberate2024}; 1 case from Australia \cite{Apikotoa_2022f}, Bahrain \cite{Ali_2020f}, Croatia \cite{Trgo_2012f}, Ecuador \cite{DelgadoSalazar_2020c}, Egypt \cite{Ali_2022g}, Ethiopia \cite{Mesfin_2022a}, Germany \cite{teWildt_2010}, Greece \cite{Sakellaridis_2008f}, Hungary \cite{Csaky_1998e}, Iraq \cite{Al-Faham_2020k}, Israel \cite{Goldman_1998f}, Italy \cite{Riva_2018j}, Japan \cite{Ohno_2005}, Nepal \cite{Thapa_2019f}, Netherlands \cite{Benoist_2019e}, Oman \cite{AlShaaibi_2021b}, Pakistan \cite{Yasin_2009}, Portugal \cite{Peixoto_2017f}, Qatar \cite{Ali_2017}, Saudi Arabia \cite{Sultan_2024f}, South Africa \cite{Sobnach_2011f}, Sweden \cite{Naji_2012f}, Switzerland \cite{Wildhaber_2005}, and Taiwan \cite{Chang_2017f}. \paragraph*{Gender} 43 cases (60\%) were male \cite{Akay_2015f, Al-Faham_2020k, Alao_2006i, Ali_2017, Ali_2022g, Apikotoa_2022f, Atayan_2016, Benoist_2019e, Berry_2021e, Bhumi_2024f, CamachoDorado_2018, Csaky_1998e, Emamhadi_2018, Farhadi_2024h, Fry_2010, Gardner_2017h, Guinan_2019f, Jehangir_2019h, Jin_2023, Kobiela_2015, Kumar_2001, Kumar_2019f, Liu_2005, Losanoff_1996, Losanoff_1997e, Mesfin_2022a, Misra_2013, Qureshi_2016, Riva_2018j, Sobnach_2011f, Tammana_2012j, Tanrikulu_2015e, Tay_2004, Thapa_2019f, Trgo_2012f, Wadhwa_2015e, Yasin_2009, teWildt_2010}, 28 cases (39\%) were female \cite{AlShaaibi_2021b, Ali_2020f, Ataya_2013, Beecroft_1998, Bhasin_2014, Bhattacharjee_2008, Cauchi_2002, Chang_2017f, Cox_2007, DelgadoSalazar_2020c, DivsalarP._2023a, Goldman_1998f, Hardy_2023g, Kar_2015, Kariholu_2008, Kerestes_2019, Li_2013, Naji_2012f, Ohno_2005, Peixoto_2017f, Sakellaridis_2008f, Sultan_2024f, Tupesis_2004f, Wildhaber_2005, Wnęk_2015f, Yildiz_2016e}, 1 case (1\%) had no gender recorded \cite{fjbuilsRepeatedBehaviorDeliberate2024}. \paragraph*{Age Group} 25 cases (35\%) were between 26 and 40 years of age \cite{Alao_2006i, Ali_2022g, Apikotoa_2022f, Ataya_2013, Benoist_2019e, Bhasin_2014, Chang_2017f, Cox_2007, DelgadoSalazar_2020c, Farhadi_2024h, Fry_2010, Gardner_2017h, Guinan_2019f, Jin_2023, Kumar_2019f, Losanoff_1996, Misra_2013, Qureshi_2016, Riva_2018j, Sakellaridis_2008f, Tammana_2012j, Trgo_2012f, Wnęk_2015f, Yildiz_2016e, fjbuilsRepeatedBehaviorDeliberate2024}, 18 cases (25\%) were between 18 and 25 years of age \cite{Akay_2015f, Ali_2017, Atayan_2016, Bhattacharjee_2008, Csaky_1998e, Kar_2015, Kariholu_2008, Kobiela_2015, Losanoff_1996, Losanoff_1997e, Mesfin_2022a, Peixoto_2017f, Sobnach_2011f, Tupesis_2004f, Yasin_2009}, 13 cases (18\%) were under 18 years of age \cite{AlShaaibi_2021b, Ali_2020f, Cauchi_2002, DivsalarP._2023a, Goldman_1998f, Liu_2005, Naji_2012f, Ohno_2005, Tanrikulu_2015e, Tay_2004, Wildhaber_2005}, 11 cases (15\%) were between 41 and 60 years of age \cite{Al-Faham_2020k, Bhumi_2024f, CamachoDorado_2018, Emamhadi_2018, Hardy_2023g, Jehangir_2019h, Kumar_2001, Sultan_2024f, Thapa_2019f, Wadhwa_2015e, teWildt_2010}, 3 cases (4\%) were over 60 years of age \cite{Beecroft_1998, Kerestes_2019, Li_2013}, 2 cases (3\%) had no age documented \cite{Berry_2021e}. \paragraph*{Population} 36 cases (50\%) had a psychiatric history \cite{AlShaaibi_2021b, Alao_2006i, Ali_2020f, Apikotoa_2022f, Ataya_2013, Atayan_2016, Beecroft_1998, CamachoDorado_2018, Chang_2017f, DelgadoSalazar_2020c, DivsalarP._2023a, Farhadi_2024h, Fry_2010, Guinan_2019f, Hardy_2023g, Jehangir_2019h, Jin_2023, Kar_2015, Kerestes_2019, Kobiela_2015, Kumar_2001, Kumar_2019f, Liu_2005, Mesfin_2022a, Misra_2013, Ohno_2005, Peixoto_2017f, Sakellaridis_2008f, Sultan_2024f, Tammana_2012j, Tanrikulu_2015e, Yildiz_2016e, fjbuilsRepeatedBehaviorDeliberate2024, teWildt_2010}, 19 cases (26\%) had ingested previously \cite{Alao_2006i, Apikotoa_2022f, Berry_2021e, Bhattacharjee_2008, Csaky_1998e, DivsalarP._2023a, Emamhadi_2018, Guinan_2019f, Jehangir_2019h, Jin_2023, Liu_2005, Sakellaridis_2008f, Tanrikulu_2015e, Thapa_2019f, Yildiz_2016e, fjbuilsRepeatedBehaviorDeliberate2024, teWildt_2010}, 12 cases (17\%) were detained persons \cite{Alao_2006i, Ali_2022g, Apikotoa_2022f, Losanoff_1996, Losanoff_1997e, Qureshi_2016, Tammana_2012j, Trgo_2012f}, 7 cases (10\%) were severely disabled \cite{Atayan_2016, Kerestes_2019, Liu_2005, Ohno_2005, Peixoto_2017f, Yildiz_2016e, teWildt_2010}, 4 cases (6\%) were psychiatric inpatients \cite{DivsalarP._2023a, fjbuilsRepeatedBehaviorDeliberate2024, teWildt_2010}, 3 cases (4\%) were under the influence of alcohol \cite{Benoist_2019e, Csaky_1998e, Thapa_2019f}, 2 cases (3\%) were displaced people \cite{Akay_2015f, Gardner_2017h}. \paragraph*{Motivation} 34 cases (47\%) had a psychiatric motivation \cite{Al-Faham_2020k, Alao_2006i, Ali_2020f, Apikotoa_2022f, Ataya_2013, Atayan_2016, Bhasin_2014, Bhattacharjee_2008, DelgadoSalazar_2020c, DivsalarP._2023a, Emamhadi_2018, Farhadi_2024h, Guinan_2019f, Hardy_2023g, Jehangir_2019h, Jin_2023, Kar_2015, Kariholu_2008, Kerestes_2019, Kobiela_2015, Kumar_2001, Kumar_2019f, Li_2013, Liu_2005, Misra_2013, Ohno_2005, Sakellaridis_2008f, Sultan_2024f, Tammana_2012j, Tanrikulu_2015e, Yasin_2009, teWildt_2010}, 21 cases (29\%) were motivated by self-harm intention \cite{Al-Faham_2020k, AlShaaibi_2021b, Alao_2006i, Ali_2017, CamachoDorado_2018, Chang_2017f, Cox_2007, Csaky_1998e, Fry_2010, Li_2013, Losanoff_1996, Losanoff_1997e, Mesfin_2022a, Sakellaridis_2008f, Tammana_2012j, Tanrikulu_2015e, fjbuilsRepeatedBehaviorDeliberate2024}, 17 cases (24\%) had a psychosocial motivation \cite{Akay_2015f, Benoist_2019e, Bhattacharjee_2008, Cauchi_2002, Goldman_1998f, Hardy_2023g, Kobiela_2015, Li_2013, Naji_2012f, Qureshi_2016, Riva_2018j, Sobnach_2011f, Tay_2004, Thapa_2019f, Tupesis_2004f, Wildhaber_2005, Wnęk_2015f}, 9 cases (12\%) were motivated by protest \cite{Bhumi_2024f, Gardner_2017h, Losanoff_1996, Losanoff_1997e, Tupesis_2004f}, 9 cases (12\%) had another documented motivation \cite{Ali_2020f, Ali_2022g, Emamhadi_2018, Guinan_2019f, Peixoto_2017f, Sakellaridis_2008f, Trgo_2012f, Wadhwa_2015e, Yildiz_2016e}. \paragraph*{Object Characteristics} 51 cases (71\%) ingested a large diameter object (\textgreater{}2.5cm) \cite{Akay_2015f, Al-Faham_2020k, AlShaaibi_2021b, Alao_2006i, Ali_2017, Ali_2022g, Apikotoa_2022f, Atayan_2016, Berry_2021e, Bhasin_2014, CamachoDorado_2018, Cauchi_2002, Chang_2017f, Cox_2007, Csaky_1998e, DivsalarP._2023a, Emamhadi_2018, Gardner_2017h, Guinan_2019f, Jehangir_2019h, Jin_2023, Kariholu_2008, Kerestes_2019, Kobiela_2015, Kumar_2001, Kumar_2019f, Losanoff_1996, Losanoff_1997e, Mesfin_2022a, Misra_2013, Naji_2012f, Ohno_2005, Peixoto_2017f, Qureshi_2016, Riva_2018j, Sakellaridis_2008f, Sultan_2024f, Tanrikulu_2015e, Thapa_2019f, Trgo_2012f, Wnęk_2015f, Yildiz_2016e, fjbuilsRepeatedBehaviorDeliberate2024, teWildt_2010}, 44 cases (61\%) ingested multiple objects \cite{Ali_2020f, Apikotoa_2022f, Ataya_2013, Atayan_2016, Beecroft_1998, Bhattacharjee_2008, Bhumi_2024f, CamachoDorado_2018, Cauchi_2002, Emamhadi_2018, Farhadi_2024h, Fry_2010, Goldman_1998f, Guinan_2019f, Hardy_2023g, Jehangir_2019h, Jin_2023, Kar_2015, Kariholu_2008, Kobiela_2015, Kumar_2001, Kumar_2019f, Li_2013, Liu_2005, Losanoff_1996, Mesfin_2022a, Misra_2013, Naji_2012f, Ohno_2005, Sobnach_2011f, Sultan_2024f, Tammana_2012j, Tanrikulu_2015e, Tay_2004, Thapa_2019f, Wadhwa_2015e, Wildhaber_2005, Yasin_2009, fjbuilsRepeatedBehaviorDeliberate2024, teWildt_2010}, 34 cases (47\%) ingested a sharp object \cite{AlShaaibi_2021b, Alao_2006i, Apikotoa_2022f, Ataya_2013, Benoist_2019e, Bhasin_2014, Bhattacharjee_2008, CamachoDorado_2018, Csaky_1998e, DelgadoSalazar_2020c, DivsalarP._2023a, Emamhadi_2018, Farhadi_2024h, Fry_2010, Guinan_2019f, Hardy_2023g, Jehangir_2019h, Jin_2023, Kariholu_2008, Kobiela_2015, Kumar_2019f, Losanoff_1996, Losanoff_1997e, Mesfin_2022a, Misra_2013, Sobnach_2011f, Yasin_2009, teWildt_2010}, 32 cases (44\%) ingested a long object (\textgreater{}5cm) \cite{Al-Faham_2020k, AlShaaibi_2021b, Ali_2017, Ali_2022g, Atayan_2016, Bhasin_2014, CamachoDorado_2018, Chang_2017f, Cox_2007, Csaky_1998e, DivsalarP._2023a, Emamhadi_2018, Fry_2010, Gardner_2017h, Jin_2023, Kariholu_2008, Kerestes_2019, Kobiela_2015, Kumar_2019f, Mesfin_2022a, Misra_2013, Ohno_2005, Qureshi_2016, Sakellaridis_2008f, Sultan_2024f, Thapa_2019f, Trgo_2012f, Yasin_2009, Yildiz_2016e, teWildt_2010}, 9 cases (12\%) ingested a magnet \cite{Ali_2020f, Bhumi_2024f, Cauchi_2002, Liu_2005, Naji_2012f, Ohno_2005, Tanrikulu_2015e, Tay_2004, Wildhaber_2005}, 2 cases (3\%) ingested a button battery \cite{Berry_2021e, Bhumi_2024f}. \paragraph*{Outcomes} 48 cases (67\%) experienced a complication \cite{Ali_2017, Ali_2020f, Apikotoa_2022f, Atayan_2016, Beecroft_1998, Benoist_2019e, Berry_2021e, Bhasin_2014, Bhumi_2024f, CamachoDorado_2018, Cauchi_2002, Cox_2007, Csaky_1998e, DelgadoSalazar_2020c, DivsalarP._2023a, Emamhadi_2018, Farhadi_2024h, Fry_2010, Gardner_2017h, Goldman_1998f, Jin_2023, Kariholu_2008, Kerestes_2019, Kobiela_2015, Kumar_2001, Kumar_2019f, Liu_2005, Losanoff_1996, Mesfin_2022a, Misra_2013, Naji_2012f, Ohno_2005, Sakellaridis_2008f, Sobnach_2011f, Sultan_2024f, Tanrikulu_2015e, Tay_2004, Thapa_2019f, Trgo_2012f, Tupesis_2004f, Wildhaber_2005, Wnęk_2015f, Yasin_2009, Yildiz_2016e}, 44 cases (61\%) underwent surgery \cite{Al-Faham_2020k, AlShaaibi_2021b, Alao_2006i, Ali_2017, Ali_2020f, Atayan_2016, Beecroft_1998, Bhasin_2014, CamachoDorado_2018, Cauchi_2002, Chang_2017f, Cox_2007, Csaky_1998e, DelgadoSalazar_2020c, DivsalarP._2023a, Farhadi_2024h, Fry_2010, Gardner_2017h, Jin_2023, Kariholu_2008, Kerestes_2019, Kobiela_2015, Kumar_2019f, Liu_2005, Losanoff_1996, Losanoff_1997e, Mesfin_2022a, Misra_2013, Naji_2012f, Sobnach_2011f, Tanrikulu_2015e, Tay_2004, Thapa_2019f, Tupesis_2004f, Wildhaber_2005, Wnęk_2015f, Yasin_2009, Yildiz_2016e, fjbuilsRepeatedBehaviorDeliberate2024}, 31 cases (43\%) underwent endoscopy \cite{Akay_2015f, Ali_2022g, Apikotoa_2022f, Atayan_2016, Benoist_2019e, Berry_2021e, Bhasin_2014, Bhumi_2024f, CamachoDorado_2018, Chang_2017f, DelgadoSalazar_2020c, Gardner_2017h, Guinan_2019f, Hardy_2023g, Jehangir_2019h, Kariholu_2008, Li_2013, Liu_2005, Ohno_2005, Peixoto_2017f, Qureshi_2016, Riva_2018j, Sakellaridis_2008f, Sultan_2024f, Tammana_2012j, Tanrikulu_2015e, Trgo_2012f, Wadhwa_2015e, Wnęk_2015f, teWildt_2010}, 7 cases (10\%) were managed conservatively \cite{Ataya_2013, Bhattacharjee_2008, DivsalarP._2023a, Emamhadi_2018, Goldman_1998f, Kar_2015, Kumar_2001}, 2 cases (3\%) died \cite{Emamhadi_2018, Kumar_2001}. All 90 were male gender. 90 cases (100\%) were detained at the time of ingestion \cite{Elghali_2016, Karp_1991b, Lee_2007}, 88 cases (98\%) were intentional ingestions \cite{Elghali_2016, Karp_1991b, Lee_2007}, 30 cases (33\%) had a psychiatric history documented \cite{Elghali_2016, Karp_1991b, Lee_2007}, 2 cases (2\%) had a history of prior ingestion \cite{Elghali_2016}. No cases were reported for were psychiatric inpatients, were displaced people, were under the influence of alcohol at the time of ingestion, and had a severe disability history.
\paragraph*{Motivation}  70 cases (78\%) reported protest motivation \cite{Elghali_2016, Karp_1991b, Lee_2007}, 12 cases (13\%) reported psychiatric motivation \cite{Karp_1991b}, 6 cases (7\%) reported self-harm motivation \cite{Elghali_2016, Karp_1991b}. No cases were reported for psychosocial motivation and other motivation.
\paragraph*{Object Characteristics}  68 cases (76\%) involved sharp object ingestion \cite{Elghali_2016, Karp_1991b, Lee_2007}, 32 cases (36\%) involved long (\textgreater 5cm) object ingestion \cite{Lee_2007}, 25 cases (28\%) involved ingestion of multiple objects \cite{Elghali_2016, Lee_2007}. No cases were reported for button battery ingestion, magnet ingestion, and involved large diameter (\textgreater 2.5cm) object ingestion.
\paragraph*{Outcomes}  47 cases (52\%) underwent endoscopic intervention \cite{Elghali_2016, Lee_2007}, 29 cases (32\%) were managed conservatively \cite{Elghali_2016, Karp_1991b}, 15 cases (17\%) underwent surgical intervention \cite{Elghali_2016, Karp_1991b, Lee_2007}, 6 cases (7\%) reported complications \cite{Lee_2007}, 1 case (1\%) died \cite{Elghali_2016}.
\paragraph*{Geographical Location}Cases were recorded in 33 countries: 13 cases from USA \cite{Alao_2006i, Ataya_2013, Bhumi_2024f, Fry_2010, Guinan_2019f, Hardy_2023g, Jehangir_2019h, Kerestes_2019, Kumar_2001, Liu_2005, Tammana_2012j, Tay_2004, Tupesis_2004f}; 7 cases from India \cite{Bhasin_2014, Bhattacharjee_2008, Kar_2015, Kariholu_2008, Kumar_2019f, Misra_2013, Wadhwa_2015e} and UK \cite{Beecroft_1998, Berry_2021e, Cauchi_2002, Cox_2007, Gardner_2017h, Qureshi_2016}; 6 cases from Bulgaria \cite{Losanoff_1996, Losanoff_1997e}; 5 cases from Iran \cite{DivsalarP._2023a, Emamhadi_2018, Farhadi_2024h}; 4 cases from Turkey \cite{Akay_2015f, Atayan_2016, Tanrikulu_2015e, Yildiz_2016e}; 2 cases from China \cite{Jin_2023, Li_2013}, Poland \cite{Kobiela_2015, Wnęk_2015f}, and Spain \cite{CamachoDorado_2018, fjbuilsRepeatedBehaviorDeliberate2024}; 1 case from Australia \cite{Apikotoa_2022f}, Bahrain \cite{Ali_2020f}, Croatia \cite{Trgo_2012f}, Ecuador \cite{DelgadoSalazar_2020c}, Egypt \cite{Ali_2022g}, Ethiopia \cite{Mesfin_2022a}, Germany \cite{teWildt_2010}, Greece \cite{Sakellaridis_2008f}, Hungary \cite{Csaky_1998e}, Iraq \cite{Al-Faham_2020k}, Israel \cite{Goldman_1998f}, Italy \cite{Riva_2018j}, Japan \cite{Ohno_2005}, Nepal \cite{Thapa_2019f}, Netherlands \cite{Benoist_2019e}, Oman \cite{AlShaaibi_2021b}, Pakistan \cite{Yasin_2009}, Portugal \cite{Peixoto_2017f}, Qatar \cite{Ali_2017}, Saudi Arabia \cite{Sultan_2024f}, South Africa \cite{Sobnach_2011f}, Sweden \cite{Naji_2012f}, Switzerland \cite{Wildhaber_2005}, and Taiwan \cite{Chang_2017f}. \paragraph*{Gender} 43 cases (60\%) were male \cite{Akay_2015f, Al-Faham_2020k, Alao_2006i, Ali_2017, Ali_2022g, Apikotoa_2022f, Atayan_2016, Benoist_2019e, Berry_2021e, Bhumi_2024f, CamachoDorado_2018, Csaky_1998e, Emamhadi_2018, Farhadi_2024h, Fry_2010, Gardner_2017h, Guinan_2019f, Jehangir_2019h, Jin_2023, Kobiela_2015, Kumar_2001, Kumar_2019f, Liu_2005, Losanoff_1996, Losanoff_1997e, Mesfin_2022a, Misra_2013, Qureshi_2016, Riva_2018j, Sobnach_2011f, Tammana_2012j, Tanrikulu_2015e, Tay_2004, Thapa_2019f, Trgo_2012f, Wadhwa_2015e, Yasin_2009, teWildt_2010}, 28 cases (39\%) were female \cite{AlShaaibi_2021b, Ali_2020f, Ataya_2013, Beecroft_1998, Bhasin_2014, Bhattacharjee_2008, Cauchi_2002, Chang_2017f, Cox_2007, DelgadoSalazar_2020c, DivsalarP._2023a, Goldman_1998f, Hardy_2023g, Kar_2015, Kariholu_2008, Kerestes_2019, Li_2013, Naji_2012f, Ohno_2005, Peixoto_2017f, Sakellaridis_2008f, Sultan_2024f, Tupesis_2004f, Wildhaber_2005, Wnęk_2015f, Yildiz_2016e}, 1 case (1\%) had no gender recorded \cite{fjbuilsRepeatedBehaviorDeliberate2024}. \paragraph*{Age Group} 25 cases (35\%) were between 26 and 40 years of age \cite{Alao_2006i, Ali_2022g, Apikotoa_2022f, Ataya_2013, Benoist_2019e, Bhasin_2014, Chang_2017f, Cox_2007, DelgadoSalazar_2020c, Farhadi_2024h, Fry_2010, Gardner_2017h, Guinan_2019f, Jin_2023, Kumar_2019f, Losanoff_1996, Misra_2013, Qureshi_2016, Riva_2018j, Sakellaridis_2008f, Tammana_2012j, Trgo_2012f, Wnęk_2015f, Yildiz_2016e, fjbuilsRepeatedBehaviorDeliberate2024}, 18 cases (25\%) were between 18 and 25 years of age \cite{Akay_2015f, Ali_2017, Atayan_2016, Bhattacharjee_2008, Csaky_1998e, Kar_2015, Kariholu_2008, Kobiela_2015, Losanoff_1996, Losanoff_1997e, Mesfin_2022a, Peixoto_2017f, Sobnach_2011f, Tupesis_2004f, Yasin_2009}, 13 cases (18\%) were under 18 years of age \cite{AlShaaibi_2021b, Ali_2020f, Cauchi_2002, DivsalarP._2023a, Goldman_1998f, Liu_2005, Naji_2012f, Ohno_2005, Tanrikulu_2015e, Tay_2004, Wildhaber_2005}, 11 cases (15\%) were between 41 and 60 years of age \cite{Al-Faham_2020k, Bhumi_2024f, CamachoDorado_2018, Emamhadi_2018, Hardy_2023g, Jehangir_2019h, Kumar_2001, Sultan_2024f, Thapa_2019f, Wadhwa_2015e, teWildt_2010}, 3 cases (4\%) were over 60 years of age \cite{Beecroft_1998, Kerestes_2019, Li_2013}, 2 cases (3\%) had no age documented \cite{Berry_2021e}. \paragraph*{Population} 36 cases (50\%) had a psychiatric history \cite{AlShaaibi_2021b, Alao_2006i, Ali_2020f, Apikotoa_2022f, Ataya_2013, Atayan_2016, Beecroft_1998, CamachoDorado_2018, Chang_2017f, DelgadoSalazar_2020c, DivsalarP._2023a, Farhadi_2024h, Fry_2010, Guinan_2019f, Hardy_2023g, Jehangir_2019h, Jin_2023, Kar_2015, Kerestes_2019, Kobiela_2015, Kumar_2001, Kumar_2019f, Liu_2005, Mesfin_2022a, Misra_2013, Ohno_2005, Peixoto_2017f, Sakellaridis_2008f, Sultan_2024f, Tammana_2012j, Tanrikulu_2015e, Yildiz_2016e, fjbuilsRepeatedBehaviorDeliberate2024, teWildt_2010}, 19 cases (26\%) had ingested previously \cite{Alao_2006i, Apikotoa_2022f, Berry_2021e, Bhattacharjee_2008, Csaky_1998e, DivsalarP._2023a, Emamhadi_2018, Guinan_2019f, Jehangir_2019h, Jin_2023, Liu_2005, Sakellaridis_2008f, Tanrikulu_2015e, Thapa_2019f, Yildiz_2016e, fjbuilsRepeatedBehaviorDeliberate2024, teWildt_2010}, 12 cases (17\%) were detained persons \cite{Alao_2006i, Ali_2022g, Apikotoa_2022f, Losanoff_1996, Losanoff_1997e, Qureshi_2016, Tammana_2012j, Trgo_2012f}, 7 cases (10\%) were severely disabled \cite{Atayan_2016, Kerestes_2019, Liu_2005, Ohno_2005, Peixoto_2017f, Yildiz_2016e, teWildt_2010}, 4 cases (6\%) were psychiatric inpatients \cite{DivsalarP._2023a, fjbuilsRepeatedBehaviorDeliberate2024, teWildt_2010}, 3 cases (4\%) were under the influence of alcohol \cite{Benoist_2019e, Csaky_1998e, Thapa_2019f}, 2 cases (3\%) were displaced people \cite{Akay_2015f, Gardner_2017h}. \paragraph*{Motivation} 34 cases (47\%) had a psychiatric motivation \cite{Al-Faham_2020k, Alao_2006i, Ali_2020f, Apikotoa_2022f, Ataya_2013, Atayan_2016, Bhasin_2014, Bhattacharjee_2008, DelgadoSalazar_2020c, DivsalarP._2023a, Emamhadi_2018, Farhadi_2024h, Guinan_2019f, Hardy_2023g, Jehangir_2019h, Jin_2023, Kar_2015, Kariholu_2008, Kerestes_2019, Kobiela_2015, Kumar_2001, Kumar_2019f, Li_2013, Liu_2005, Misra_2013, Ohno_2005, Sakellaridis_2008f, Sultan_2024f, Tammana_2012j, Tanrikulu_2015e, Yasin_2009, teWildt_2010}, 21 cases (29\%) were motivated by self-harm intention \cite{Al-Faham_2020k, AlShaaibi_2021b, Alao_2006i, Ali_2017, CamachoDorado_2018, Chang_2017f, Cox_2007, Csaky_1998e, Fry_2010, Li_2013, Losanoff_1996, Losanoff_1997e, Mesfin_2022a, Sakellaridis_2008f, Tammana_2012j, Tanrikulu_2015e, fjbuilsRepeatedBehaviorDeliberate2024}, 17 cases (24\%) had a psychosocial motivation \cite{Akay_2015f, Benoist_2019e, Bhattacharjee_2008, Cauchi_2002, Goldman_1998f, Hardy_2023g, Kobiela_2015, Li_2013, Naji_2012f, Qureshi_2016, Riva_2018j, Sobnach_2011f, Tay_2004, Thapa_2019f, Tupesis_2004f, Wildhaber_2005, Wnęk_2015f}, 9 cases (12\%) were motivated by protest \cite{Bhumi_2024f, Gardner_2017h, Losanoff_1996, Losanoff_1997e, Tupesis_2004f}, 9 cases (12\%) had another documented motivation \cite{Ali_2020f, Ali_2022g, Emamhadi_2018, Guinan_2019f, Peixoto_2017f, Sakellaridis_2008f, Trgo_2012f, Wadhwa_2015e, Yildiz_2016e}. \paragraph*{Object Characteristics} 51 cases (71\%) ingested a large diameter object (\textgreater{}2.5cm) \cite{Akay_2015f, Al-Faham_2020k, AlShaaibi_2021b, Alao_2006i, Ali_2017, Ali_2022g, Apikotoa_2022f, Atayan_2016, Berry_2021e, Bhasin_2014, CamachoDorado_2018, Cauchi_2002, Chang_2017f, Cox_2007, Csaky_1998e, DivsalarP._2023a, Emamhadi_2018, Gardner_2017h, Guinan_2019f, Jehangir_2019h, Jin_2023, Kariholu_2008, Kerestes_2019, Kobiela_2015, Kumar_2001, Kumar_2019f, Losanoff_1996, Losanoff_1997e, Mesfin_2022a, Misra_2013, Naji_2012f, Ohno_2005, Peixoto_2017f, Qureshi_2016, Riva_2018j, Sakellaridis_2008f, Sultan_2024f, Tanrikulu_2015e, Thapa_2019f, Trgo_2012f, Wnęk_2015f, Yildiz_2016e, fjbuilsRepeatedBehaviorDeliberate2024, teWildt_2010}, 44 cases (61\%) ingested multiple objects \cite{Ali_2020f, Apikotoa_2022f, Ataya_2013, Atayan_2016, Beecroft_1998, Bhattacharjee_2008, Bhumi_2024f, CamachoDorado_2018, Cauchi_2002, Emamhadi_2018, Farhadi_2024h, Fry_2010, Goldman_1998f, Guinan_2019f, Hardy_2023g, Jehangir_2019h, Jin_2023, Kar_2015, Kariholu_2008, Kobiela_2015, Kumar_2001, Kumar_2019f, Li_2013, Liu_2005, Losanoff_1996, Mesfin_2022a, Misra_2013, Naji_2012f, Ohno_2005, Sobnach_2011f, Sultan_2024f, Tammana_2012j, Tanrikulu_2015e, Tay_2004, Thapa_2019f, Wadhwa_2015e, Wildhaber_2005, Yasin_2009, fjbuilsRepeatedBehaviorDeliberate2024, teWildt_2010}, 34 cases (47\%) ingested a sharp object \cite{AlShaaibi_2021b, Alao_2006i, Apikotoa_2022f, Ataya_2013, Benoist_2019e, Bhasin_2014, Bhattacharjee_2008, CamachoDorado_2018, Csaky_1998e, DelgadoSalazar_2020c, DivsalarP._2023a, Emamhadi_2018, Farhadi_2024h, Fry_2010, Guinan_2019f, Hardy_2023g, Jehangir_2019h, Jin_2023, Kariholu_2008, Kobiela_2015, Kumar_2019f, Losanoff_1996, Losanoff_1997e, Mesfin_2022a, Misra_2013, Sobnach_2011f, Yasin_2009, teWildt_2010}, 32 cases (44\%) ingested a long object (\textgreater{}5cm) \cite{Al-Faham_2020k, AlShaaibi_2021b, Ali_2017, Ali_2022g, Atayan_2016, Bhasin_2014, CamachoDorado_2018, Chang_2017f, Cox_2007, Csaky_1998e, DivsalarP._2023a, Emamhadi_2018, Fry_2010, Gardner_2017h, Jin_2023, Kariholu_2008, Kerestes_2019, Kobiela_2015, Kumar_2019f, Mesfin_2022a, Misra_2013, Ohno_2005, Qureshi_2016, Sakellaridis_2008f, Sultan_2024f, Thapa_2019f, Trgo_2012f, Yasin_2009, Yildiz_2016e, teWildt_2010}, 9 cases (12\%) ingested a magnet \cite{Ali_2020f, Bhumi_2024f, Cauchi_2002, Liu_2005, Naji_2012f, Ohno_2005, Tanrikulu_2015e, Tay_2004, Wildhaber_2005}, 2 cases (3\%) ingested a button battery \cite{Berry_2021e, Bhumi_2024f}. \paragraph*{Outcomes} 48 cases (67\%) experienced a complication \cite{Ali_2017, Ali_2020f, Apikotoa_2022f, Atayan_2016, Beecroft_1998, Benoist_2019e, Berry_2021e, Bhasin_2014, Bhumi_2024f, CamachoDorado_2018, Cauchi_2002, Cox_2007, Csaky_1998e, DelgadoSalazar_2020c, DivsalarP._2023a, Emamhadi_2018, Farhadi_2024h, Fry_2010, Gardner_2017h, Goldman_1998f, Jin_2023, Kariholu_2008, Kerestes_2019, Kobiela_2015, Kumar_2001, Kumar_2019f, Liu_2005, Losanoff_1996, Mesfin_2022a, Misra_2013, Naji_2012f, Ohno_2005, Sakellaridis_2008f, Sobnach_2011f, Sultan_2024f, Tanrikulu_2015e, Tay_2004, Thapa_2019f, Trgo_2012f, Tupesis_2004f, Wildhaber_2005, Wnęk_2015f, Yasin_2009, Yildiz_2016e}, 44 cases (61\%) underwent surgery \cite{Al-Faham_2020k, AlShaaibi_2021b, Alao_2006i, Ali_2017, Ali_2020f, Atayan_2016, Beecroft_1998, Bhasin_2014, CamachoDorado_2018, Cauchi_2002, Chang_2017f, Cox_2007, Csaky_1998e, DelgadoSalazar_2020c, DivsalarP._2023a, Farhadi_2024h, Fry_2010, Gardner_2017h, Jin_2023, Kariholu_2008, Kerestes_2019, Kobiela_2015, Kumar_2019f, Liu_2005, Losanoff_1996, Losanoff_1997e, Mesfin_2022a, Misra_2013, Naji_2012f, Sobnach_2011f, Tanrikulu_2015e, Tay_2004, Thapa_2019f, Tupesis_2004f, Wildhaber_2005, Wnęk_2015f, Yasin_2009, Yildiz_2016e, fjbuilsRepeatedBehaviorDeliberate2024}, 31 cases (43\%) underwent endoscopy \cite{Akay_2015f, Ali_2022g, Apikotoa_2022f, Atayan_2016, Benoist_2019e, Berry_2021e, Bhasin_2014, Bhumi_2024f, CamachoDorado_2018, Chang_2017f, DelgadoSalazar_2020c, Gardner_2017h, Guinan_2019f, Hardy_2023g, Jehangir_2019h, Kariholu_2008, Li_2013, Liu_2005, Ohno_2005, Peixoto_2017f, Qureshi_2016, Riva_2018j, Sakellaridis_2008f, Sultan_2024f, Tammana_2012j, Tanrikulu_2015e, Trgo_2012f, Wadhwa_2015e, Wnęk_2015f, teWildt_2010}, 7 cases (10\%) were managed conservatively \cite{Ataya_2013, Bhattacharjee_2008, DivsalarP._2023a, Emamhadi_2018, Goldman_1998f, Kar_2015, Kumar_2001}, 2 cases (3\%) died \cite{Emamhadi_2018, Kumar_2001}. All 90 were male gender. 90 cases (100\%) were detained at the time of ingestion \cite{Elghali_2016, Karp_1991b, Lee_2007}, 88 cases (98\%) were intentional ingestions \cite{Elghali_2016, Karp_1991b, Lee_2007}, 30 cases (33\%) had a psychiatric history documented \cite{Elghali_2016, Karp_1991b, Lee_2007}, 2 cases (2\%) had a history of prior ingestion \cite{Elghali_2016}. No cases were reported for were psychiatric inpatients, were displaced people, were under the influence of alcohol at the time of ingestion, and had a severe disability history.
\paragraph*{Motivation}  70 cases (78\%) reported protest motivation \cite{Elghali_2016, Karp_1991b, Lee_2007}, 12 cases (13\%) reported psychiatric motivation \cite{Karp_1991b}, 6 cases (7\%) reported self-harm motivation \cite{Elghali_2016, Karp_1991b}. No cases were reported for psychosocial motivation and other motivation.
\paragraph*{Object Characteristics}  68 cases (76\%) involved sharp object ingestion \cite{Elghali_2016, Karp_1991b, Lee_2007}, 32 cases (36\%) involved long (\textgreater 5cm) object ingestion \cite{Lee_2007}, 25 cases (28\%) involved ingestion of multiple objects \cite{Elghali_2016, Lee_2007}. No cases were reported for button battery ingestion, magnet ingestion, and involved large diameter (\textgreater 2.5cm) object ingestion.
\paragraph*{Outcomes}  47 cases (52\%) underwent endoscopic intervention \cite{Elghali_2016, Lee_2007}, 29 cases (32\%) were managed conservatively \cite{Elghali_2016, Karp_1991b}, 15 cases (17\%) underwent surgical intervention \cite{Elghali_2016, Karp_1991b, Lee_2007}, 6 cases (7\%) reported complications \cite{Lee_2007}, 1 case (1\%) died \cite{Elghali_2016}.
\paragraph*{Geographical Location}Cases were recorded in 33 countries: 13 cases from USA \cite{Alao_2006i, Ataya_2013, Bhumi_2024f, Fry_2010, Guinan_2019f, Hardy_2023g, Jehangir_2019h, Kerestes_2019, Kumar_2001, Liu_2005, Tammana_2012j, Tay_2004, Tupesis_2004f}; 7 cases from India \cite{Bhasin_2014, Bhattacharjee_2008, Kar_2015, Kariholu_2008, Kumar_2019f, Misra_2013, Wadhwa_2015e} and UK \cite{Beecroft_1998, Berry_2021e, Cauchi_2002, Cox_2007, Gardner_2017h, Qureshi_2016}; 6 cases from Bulgaria \cite{Losanoff_1996, Losanoff_1997e}; 5 cases from Iran \cite{DivsalarP._2023a, Emamhadi_2018, Farhadi_2024h}; 4 cases from Turkey \cite{Akay_2015f, Atayan_2016, Tanrikulu_2015e, Yildiz_2016e}; 2 cases from China \cite{Jin_2023, Li_2013}, Poland \cite{Kobiela_2015, Wnęk_2015f}, and Spain \cite{CamachoDorado_2018, fjbuilsRepeatedBehaviorDeliberate2024}; 1 case from Australia \cite{Apikotoa_2022f}, Bahrain \cite{Ali_2020f}, Croatia \cite{Trgo_2012f}, Ecuador \cite{DelgadoSalazar_2020c}, Egypt \cite{Ali_2022g}, Ethiopia \cite{Mesfin_2022a}, Germany \cite{teWildt_2010}, Greece \cite{Sakellaridis_2008f}, Hungary \cite{Csaky_1998e}, Iraq \cite{Al-Faham_2020k}, Israel \cite{Goldman_1998f}, Italy \cite{Riva_2018j}, Japan \cite{Ohno_2005}, Nepal \cite{Thapa_2019f}, Netherlands \cite{Benoist_2019e}, Oman \cite{AlShaaibi_2021b}, Pakistan \cite{Yasin_2009}, Portugal \cite{Peixoto_2017f}, Qatar \cite{Ali_2017}, Saudi Arabia \cite{Sultan_2024f}, South Africa \cite{Sobnach_2011f}, Sweden \cite{Naji_2012f}, Switzerland \cite{Wildhaber_2005}, and Taiwan \cite{Chang_2017f}. \paragraph*{Gender} 43 cases (60\%) were male \cite{Akay_2015f, Al-Faham_2020k, Alao_2006i, Ali_2017, Ali_2022g, Apikotoa_2022f, Atayan_2016, Benoist_2019e, Berry_2021e, Bhumi_2024f, CamachoDorado_2018, Csaky_1998e, Emamhadi_2018, Farhadi_2024h, Fry_2010, Gardner_2017h, Guinan_2019f, Jehangir_2019h, Jin_2023, Kobiela_2015, Kumar_2001, Kumar_2019f, Liu_2005, Losanoff_1996, Losanoff_1997e, Mesfin_2022a, Misra_2013, Qureshi_2016, Riva_2018j, Sobnach_2011f, Tammana_2012j, Tanrikulu_2015e, Tay_2004, Thapa_2019f, Trgo_2012f, Wadhwa_2015e, Yasin_2009, teWildt_2010}, 28 cases (39\%) were female \cite{AlShaaibi_2021b, Ali_2020f, Ataya_2013, Beecroft_1998, Bhasin_2014, Bhattacharjee_2008, Cauchi_2002, Chang_2017f, Cox_2007, DelgadoSalazar_2020c, DivsalarP._2023a, Goldman_1998f, Hardy_2023g, Kar_2015, Kariholu_2008, Kerestes_2019, Li_2013, Naji_2012f, Ohno_2005, Peixoto_2017f, Sakellaridis_2008f, Sultan_2024f, Tupesis_2004f, Wildhaber_2005, Wnęk_2015f, Yildiz_2016e}, 1 case (1\%) had no gender recorded \cite{fjbuilsRepeatedBehaviorDeliberate2024}. \paragraph*{Age Group} 25 cases (35\%) were between 26 and 40 years of age \cite{Alao_2006i, Ali_2022g, Apikotoa_2022f, Ataya_2013, Benoist_2019e, Bhasin_2014, Chang_2017f, Cox_2007, DelgadoSalazar_2020c, Farhadi_2024h, Fry_2010, Gardner_2017h, Guinan_2019f, Jin_2023, Kumar_2019f, Losanoff_1996, Misra_2013, Qureshi_2016, Riva_2018j, Sakellaridis_2008f, Tammana_2012j, Trgo_2012f, Wnęk_2015f, Yildiz_2016e, fjbuilsRepeatedBehaviorDeliberate2024}, 18 cases (25\%) were between 18 and 25 years of age \cite{Akay_2015f, Ali_2017, Atayan_2016, Bhattacharjee_2008, Csaky_1998e, Kar_2015, Kariholu_2008, Kobiela_2015, Losanoff_1996, Losanoff_1997e, Mesfin_2022a, Peixoto_2017f, Sobnach_2011f, Tupesis_2004f, Yasin_2009}, 13 cases (18\%) were under 18 years of age \cite{AlShaaibi_2021b, Ali_2020f, Cauchi_2002, DivsalarP._2023a, Goldman_1998f, Liu_2005, Naji_2012f, Ohno_2005, Tanrikulu_2015e, Tay_2004, Wildhaber_2005}, 11 cases (15\%) were between 41 and 60 years of age \cite{Al-Faham_2020k, Bhumi_2024f, CamachoDorado_2018, Emamhadi_2018, Hardy_2023g, Jehangir_2019h, Kumar_2001, Sultan_2024f, Thapa_2019f, Wadhwa_2015e, teWildt_2010}, 3 cases (4\%) were over 60 years of age \cite{Beecroft_1998, Kerestes_2019, Li_2013}, 2 cases (3\%) had no age documented \cite{Berry_2021e}. \paragraph*{Population} 36 cases (50\%) had a psychiatric history \cite{AlShaaibi_2021b, Alao_2006i, Ali_2020f, Apikotoa_2022f, Ataya_2013, Atayan_2016, Beecroft_1998, CamachoDorado_2018, Chang_2017f, DelgadoSalazar_2020c, DivsalarP._2023a, Farhadi_2024h, Fry_2010, Guinan_2019f, Hardy_2023g, Jehangir_2019h, Jin_2023, Kar_2015, Kerestes_2019, Kobiela_2015, Kumar_2001, Kumar_2019f, Liu_2005, Mesfin_2022a, Misra_2013, Ohno_2005, Peixoto_2017f, Sakellaridis_2008f, Sultan_2024f, Tammana_2012j, Tanrikulu_2015e, Yildiz_2016e, fjbuilsRepeatedBehaviorDeliberate2024, teWildt_2010}, 19 cases (26\%) had ingested previously \cite{Alao_2006i, Apikotoa_2022f, Berry_2021e, Bhattacharjee_2008, Csaky_1998e, DivsalarP._2023a, Emamhadi_2018, Guinan_2019f, Jehangir_2019h, Jin_2023, Liu_2005, Sakellaridis_2008f, Tanrikulu_2015e, Thapa_2019f, Yildiz_2016e, fjbuilsRepeatedBehaviorDeliberate2024, teWildt_2010}, 12 cases (17\%) were detained persons \cite{Alao_2006i, Ali_2022g, Apikotoa_2022f, Losanoff_1996, Losanoff_1997e, Qureshi_2016, Tammana_2012j, Trgo_2012f}, 7 cases (10\%) were severely disabled \cite{Atayan_2016, Kerestes_2019, Liu_2005, Ohno_2005, Peixoto_2017f, Yildiz_2016e, teWildt_2010}, 4 cases (6\%) were psychiatric inpatients \cite{DivsalarP._2023a, fjbuilsRepeatedBehaviorDeliberate2024, teWildt_2010}, 3 cases (4\%) were under the influence of alcohol \cite{Benoist_2019e, Csaky_1998e, Thapa_2019f}, 2 cases (3\%) were displaced people \cite{Akay_2015f, Gardner_2017h}. \paragraph*{Motivation} 34 cases (47\%) had a psychiatric motivation \cite{Al-Faham_2020k, Alao_2006i, Ali_2020f, Apikotoa_2022f, Ataya_2013, Atayan_2016, Bhasin_2014, Bhattacharjee_2008, DelgadoSalazar_2020c, DivsalarP._2023a, Emamhadi_2018, Farhadi_2024h, Guinan_2019f, Hardy_2023g, Jehangir_2019h, Jin_2023, Kar_2015, Kariholu_2008, Kerestes_2019, Kobiela_2015, Kumar_2001, Kumar_2019f, Li_2013, Liu_2005, Misra_2013, Ohno_2005, Sakellaridis_2008f, Sultan_2024f, Tammana_2012j, Tanrikulu_2015e, Yasin_2009, teWildt_2010}, 21 cases (29\%) were motivated by self-harm intention \cite{Al-Faham_2020k, AlShaaibi_2021b, Alao_2006i, Ali_2017, CamachoDorado_2018, Chang_2017f, Cox_2007, Csaky_1998e, Fry_2010, Li_2013, Losanoff_1996, Losanoff_1997e, Mesfin_2022a, Sakellaridis_2008f, Tammana_2012j, Tanrikulu_2015e, fjbuilsRepeatedBehaviorDeliberate2024}, 17 cases (24\%) had a psychosocial motivation \cite{Akay_2015f, Benoist_2019e, Bhattacharjee_2008, Cauchi_2002, Goldman_1998f, Hardy_2023g, Kobiela_2015, Li_2013, Naji_2012f, Qureshi_2016, Riva_2018j, Sobnach_2011f, Tay_2004, Thapa_2019f, Tupesis_2004f, Wildhaber_2005, Wnęk_2015f}, 9 cases (12\%) were motivated by protest \cite{Bhumi_2024f, Gardner_2017h, Losanoff_1996, Losanoff_1997e, Tupesis_2004f}, 9 cases (12\%) had another documented motivation \cite{Ali_2020f, Ali_2022g, Emamhadi_2018, Guinan_2019f, Peixoto_2017f, Sakellaridis_2008f, Trgo_2012f, Wadhwa_2015e, Yildiz_2016e}. \paragraph*{Object Characteristics} 51 cases (71\%) ingested a large diameter object (\textgreater{}2.5cm) \cite{Akay_2015f, Al-Faham_2020k, AlShaaibi_2021b, Alao_2006i, Ali_2017, Ali_2022g, Apikotoa_2022f, Atayan_2016, Berry_2021e, Bhasin_2014, CamachoDorado_2018, Cauchi_2002, Chang_2017f, Cox_2007, Csaky_1998e, DivsalarP._2023a, Emamhadi_2018, Gardner_2017h, Guinan_2019f, Jehangir_2019h, Jin_2023, Kariholu_2008, Kerestes_2019, Kobiela_2015, Kumar_2001, Kumar_2019f, Losanoff_1996, Losanoff_1997e, Mesfin_2022a, Misra_2013, Naji_2012f, Ohno_2005, Peixoto_2017f, Qureshi_2016, Riva_2018j, Sakellaridis_2008f, Sultan_2024f, Tanrikulu_2015e, Thapa_2019f, Trgo_2012f, Wnęk_2015f, Yildiz_2016e, fjbuilsRepeatedBehaviorDeliberate2024, teWildt_2010}, 44 cases (61\%) ingested multiple objects \cite{Ali_2020f, Apikotoa_2022f, Ataya_2013, Atayan_2016, Beecroft_1998, Bhattacharjee_2008, Bhumi_2024f, CamachoDorado_2018, Cauchi_2002, Emamhadi_2018, Farhadi_2024h, Fry_2010, Goldman_1998f, Guinan_2019f, Hardy_2023g, Jehangir_2019h, Jin_2023, Kar_2015, Kariholu_2008, Kobiela_2015, Kumar_2001, Kumar_2019f, Li_2013, Liu_2005, Losanoff_1996, Mesfin_2022a, Misra_2013, Naji_2012f, Ohno_2005, Sobnach_2011f, Sultan_2024f, Tammana_2012j, Tanrikulu_2015e, Tay_2004, Thapa_2019f, Wadhwa_2015e, Wildhaber_2005, Yasin_2009, fjbuilsRepeatedBehaviorDeliberate2024, teWildt_2010}, 34 cases (47\%) ingested a sharp object \cite{AlShaaibi_2021b, Alao_2006i, Apikotoa_2022f, Ataya_2013, Benoist_2019e, Bhasin_2014, Bhattacharjee_2008, CamachoDorado_2018, Csaky_1998e, DelgadoSalazar_2020c, DivsalarP._2023a, Emamhadi_2018, Farhadi_2024h, Fry_2010, Guinan_2019f, Hardy_2023g, Jehangir_2019h, Jin_2023, Kariholu_2008, Kobiela_2015, Kumar_2019f, Losanoff_1996, Losanoff_1997e, Mesfin_2022a, Misra_2013, Sobnach_2011f, Yasin_2009, teWildt_2010}, 32 cases (44\%) ingested a long object (\textgreater{}5cm) \cite{Al-Faham_2020k, AlShaaibi_2021b, Ali_2017, Ali_2022g, Atayan_2016, Bhasin_2014, CamachoDorado_2018, Chang_2017f, Cox_2007, Csaky_1998e, DivsalarP._2023a, Emamhadi_2018, Fry_2010, Gardner_2017h, Jin_2023, Kariholu_2008, Kerestes_2019, Kobiela_2015, Kumar_2019f, Mesfin_2022a, Misra_2013, Ohno_2005, Qureshi_2016, Sakellaridis_2008f, Sultan_2024f, Thapa_2019f, Trgo_2012f, Yasin_2009, Yildiz_2016e, teWildt_2010}, 9 cases (12\%) ingested a magnet \cite{Ali_2020f, Bhumi_2024f, Cauchi_2002, Liu_2005, Naji_2012f, Ohno_2005, Tanrikulu_2015e, Tay_2004, Wildhaber_2005}, 2 cases (3\%) ingested a button battery \cite{Berry_2021e, Bhumi_2024f}. \paragraph*{Outcomes} 48 cases (67\%) experienced a complication \cite{Ali_2017, Ali_2020f, Apikotoa_2022f, Atayan_2016, Beecroft_1998, Benoist_2019e, Berry_2021e, Bhasin_2014, Bhumi_2024f, CamachoDorado_2018, Cauchi_2002, Cox_2007, Csaky_1998e, DelgadoSalazar_2020c, DivsalarP._2023a, Emamhadi_2018, Farhadi_2024h, Fry_2010, Gardner_2017h, Goldman_1998f, Jin_2023, Kariholu_2008, Kerestes_2019, Kobiela_2015, Kumar_2001, Kumar_2019f, Liu_2005, Losanoff_1996, Mesfin_2022a, Misra_2013, Naji_2012f, Ohno_2005, Sakellaridis_2008f, Sobnach_2011f, Sultan_2024f, Tanrikulu_2015e, Tay_2004, Thapa_2019f, Trgo_2012f, Tupesis_2004f, Wildhaber_2005, Wnęk_2015f, Yasin_2009, Yildiz_2016e}, 44 cases (61\%) underwent surgery \cite{Al-Faham_2020k, AlShaaibi_2021b, Alao_2006i, Ali_2017, Ali_2020f, Atayan_2016, Beecroft_1998, Bhasin_2014, CamachoDorado_2018, Cauchi_2002, Chang_2017f, Cox_2007, Csaky_1998e, DelgadoSalazar_2020c, DivsalarP._2023a, Farhadi_2024h, Fry_2010, Gardner_2017h, Jin_2023, Kariholu_2008, Kerestes_2019, Kobiela_2015, Kumar_2019f, Liu_2005, Losanoff_1996, Losanoff_1997e, Mesfin_2022a, Misra_2013, Naji_2012f, Sobnach_2011f, Tanrikulu_2015e, Tay_2004, Thapa_2019f, Tupesis_2004f, Wildhaber_2005, Wnęk_2015f, Yasin_2009, Yildiz_2016e, fjbuilsRepeatedBehaviorDeliberate2024}, 31 cases (43\%) underwent endoscopy \cite{Akay_2015f, Ali_2022g, Apikotoa_2022f, Atayan_2016, Benoist_2019e, Berry_2021e, Bhasin_2014, Bhumi_2024f, CamachoDorado_2018, Chang_2017f, DelgadoSalazar_2020c, Gardner_2017h, Guinan_2019f, Hardy_2023g, Jehangir_2019h, Kariholu_2008, Li_2013, Liu_2005, Ohno_2005, Peixoto_2017f, Qureshi_2016, Riva_2018j, Sakellaridis_2008f, Sultan_2024f, Tammana_2012j, Tanrikulu_2015e, Trgo_2012f, Wadhwa_2015e, Wnęk_2015f, teWildt_2010}, 7 cases (10\%) were managed conservatively \cite{Ataya_2013, Bhattacharjee_2008, DivsalarP._2023a, Emamhadi_2018, Goldman_1998f, Kar_2015, Kumar_2001}, 2 cases (3\%) died \cite{Emamhadi_2018, Kumar_2001}. All 90 were male gender. 90 cases (100\%) were detained at the time of ingestion \cite{Elghali_2016, Karp_1991b, Lee_2007}, 88 cases (98\%) were intentional ingestions \cite{Elghali_2016, Karp_1991b, Lee_2007}, 30 cases (33\%) had a psychiatric history documented \cite{Elghali_2016, Karp_1991b, Lee_2007}, 2 cases (2\%) had a history of prior ingestion \cite{Elghali_2016}. No cases were reported for were psychiatric inpatients, were displaced people, were under the influence of alcohol at the time of ingestion, and had a severe disability history.
\paragraph*{Motivation}  70 cases (78\%) reported protest motivation \cite{Elghali_2016, Karp_1991b, Lee_2007}, 12 cases (13\%) reported psychiatric motivation \cite{Karp_1991b}, 6 cases (7\%) reported self-harm motivation \cite{Elghali_2016, Karp_1991b}. No cases were reported for psychosocial motivation and other motivation.
\paragraph*{Object Characteristics}  68 cases (76\%) involved sharp object ingestion \cite{Elghali_2016, Karp_1991b, Lee_2007}, 32 cases (36\%) involved long (\textgreater 5cm) object ingestion \cite{Lee_2007}, 25 cases (28\%) involved ingestion of multiple objects \cite{Elghali_2016, Lee_2007}. No cases were reported for button battery ingestion, magnet ingestion, and involved large diameter (\textgreater 2.5cm) object ingestion.
\paragraph*{Outcomes}  47 cases (52\%) underwent endoscopic intervention \cite{Elghali_2016, Lee_2007}, 29 cases (32\%) were managed conservatively \cite{Elghali_2016, Karp_1991b}, 15 cases (17\%) underwent surgical intervention \cite{Elghali_2016, Karp_1991b, Lee_2007}, 6 cases (7\%) reported complications \cite{Lee_2007}, 1 case (1\%) died \cite{Elghali_2016}.
\paragraph*{Geographical Location}Cases were recorded in 33 countries: 13 cases from USA \cite{Alao_2006i, Ataya_2013, Bhumi_2024f, Fry_2010, Guinan_2019f, Hardy_2023g, Jehangir_2019h, Kerestes_2019, Kumar_2001, Liu_2005, Tammana_2012j, Tay_2004, Tupesis_2004f}; 7 cases from India \cite{Bhasin_2014, Bhattacharjee_2008, Kar_2015, Kariholu_2008, Kumar_2019f, Misra_2013, Wadhwa_2015e} and UK \cite{Beecroft_1998, Berry_2021e, Cauchi_2002, Cox_2007, Gardner_2017h, Qureshi_2016}; 6 cases from Bulgaria \cite{Losanoff_1996, Losanoff_1997e}; 5 cases from Iran \cite{DivsalarP._2023a, Emamhadi_2018, Farhadi_2024h}; 4 cases from Turkey \cite{Akay_2015f, Atayan_2016, Tanrikulu_2015e, Yildiz_2016e}; 2 cases from China \cite{Jin_2023, Li_2013}, Poland \cite{Kobiela_2015, Wnęk_2015f}, and Spain \cite{CamachoDorado_2018, fjbuilsRepeatedBehaviorDeliberate2024}; 1 case from Australia \cite{Apikotoa_2022f}, Bahrain \cite{Ali_2020f}, Croatia \cite{Trgo_2012f}, Ecuador \cite{DelgadoSalazar_2020c}, Egypt \cite{Ali_2022g}, Ethiopia \cite{Mesfin_2022a}, Germany \cite{teWildt_2010}, Greece \cite{Sakellaridis_2008f}, Hungary \cite{Csaky_1998e}, Iraq \cite{Al-Faham_2020k}, Israel \cite{Goldman_1998f}, Italy \cite{Riva_2018j}, Japan \cite{Ohno_2005}, Nepal \cite{Thapa_2019f}, Netherlands \cite{Benoist_2019e}, Oman \cite{AlShaaibi_2021b}, Pakistan \cite{Yasin_2009}, Portugal \cite{Peixoto_2017f}, Qatar \cite{Ali_2017}, Saudi Arabia \cite{Sultan_2024f}, South Africa \cite{Sobnach_2011f}, Sweden \cite{Naji_2012f}, Switzerland \cite{Wildhaber_2005}, and Taiwan \cite{Chang_2017f}. \paragraph*{Gender} 43 cases (60\%) were male \cite{Akay_2015f, Al-Faham_2020k, Alao_2006i, Ali_2017, Ali_2022g, Apikotoa_2022f, Atayan_2016, Benoist_2019e, Berry_2021e, Bhumi_2024f, CamachoDorado_2018, Csaky_1998e, Emamhadi_2018, Farhadi_2024h, Fry_2010, Gardner_2017h, Guinan_2019f, Jehangir_2019h, Jin_2023, Kobiela_2015, Kumar_2001, Kumar_2019f, Liu_2005, Losanoff_1996, Losanoff_1997e, Mesfin_2022a, Misra_2013, Qureshi_2016, Riva_2018j, Sobnach_2011f, Tammana_2012j, Tanrikulu_2015e, Tay_2004, Thapa_2019f, Trgo_2012f, Wadhwa_2015e, Yasin_2009, teWildt_2010}, 28 cases (39\%) were female \cite{AlShaaibi_2021b, Ali_2020f, Ataya_2013, Beecroft_1998, Bhasin_2014, Bhattacharjee_2008, Cauchi_2002, Chang_2017f, Cox_2007, DelgadoSalazar_2020c, DivsalarP._2023a, Goldman_1998f, Hardy_2023g, Kar_2015, Kariholu_2008, Kerestes_2019, Li_2013, Naji_2012f, Ohno_2005, Peixoto_2017f, Sakellaridis_2008f, Sultan_2024f, Tupesis_2004f, Wildhaber_2005, Wnęk_2015f, Yildiz_2016e}, 1 case (1\%) had no gender recorded \cite{fjbuilsRepeatedBehaviorDeliberate2024}. \paragraph*{Age Group} 25 cases (35\%) were between 26 and 40 years of age \cite{Alao_2006i, Ali_2022g, Apikotoa_2022f, Ataya_2013, Benoist_2019e, Bhasin_2014, Chang_2017f, Cox_2007, DelgadoSalazar_2020c, Farhadi_2024h, Fry_2010, Gardner_2017h, Guinan_2019f, Jin_2023, Kumar_2019f, Losanoff_1996, Misra_2013, Qureshi_2016, Riva_2018j, Sakellaridis_2008f, Tammana_2012j, Trgo_2012f, Wnęk_2015f, Yildiz_2016e, fjbuilsRepeatedBehaviorDeliberate2024}, 18 cases (25\%) were between 18 and 25 years of age \cite{Akay_2015f, Ali_2017, Atayan_2016, Bhattacharjee_2008, Csaky_1998e, Kar_2015, Kariholu_2008, Kobiela_2015, Losanoff_1996, Losanoff_1997e, Mesfin_2022a, Peixoto_2017f, Sobnach_2011f, Tupesis_2004f, Yasin_2009}, 13 cases (18\%) were under 18 years of age \cite{AlShaaibi_2021b, Ali_2020f, Cauchi_2002, DivsalarP._2023a, Goldman_1998f, Liu_2005, Naji_2012f, Ohno_2005, Tanrikulu_2015e, Tay_2004, Wildhaber_2005}, 11 cases (15\%) were between 41 and 60 years of age \cite{Al-Faham_2020k, Bhumi_2024f, CamachoDorado_2018, Emamhadi_2018, Hardy_2023g, Jehangir_2019h, Kumar_2001, Sultan_2024f, Thapa_2019f, Wadhwa_2015e, teWildt_2010}, 3 cases (4\%) were over 60 years of age \cite{Beecroft_1998, Kerestes_2019, Li_2013}, 2 cases (3\%) had no age documented \cite{Berry_2021e}. \paragraph*{Population} 36 cases (50\%) had a psychiatric history \cite{AlShaaibi_2021b, Alao_2006i, Ali_2020f, Apikotoa_2022f, Ataya_2013, Atayan_2016, Beecroft_1998, CamachoDorado_2018, Chang_2017f, DelgadoSalazar_2020c, DivsalarP._2023a, Farhadi_2024h, Fry_2010, Guinan_2019f, Hardy_2023g, Jehangir_2019h, Jin_2023, Kar_2015, Kerestes_2019, Kobiela_2015, Kumar_2001, Kumar_2019f, Liu_2005, Mesfin_2022a, Misra_2013, Ohno_2005, Peixoto_2017f, Sakellaridis_2008f, Sultan_2024f, Tammana_2012j, Tanrikulu_2015e, Yildiz_2016e, fjbuilsRepeatedBehaviorDeliberate2024, teWildt_2010}, 19 cases (26\%) had ingested previously \cite{Alao_2006i, Apikotoa_2022f, Berry_2021e, Bhattacharjee_2008, Csaky_1998e, DivsalarP._2023a, Emamhadi_2018, Guinan_2019f, Jehangir_2019h, Jin_2023, Liu_2005, Sakellaridis_2008f, Tanrikulu_2015e, Thapa_2019f, Yildiz_2016e, fjbuilsRepeatedBehaviorDeliberate2024, teWildt_2010}, 12 cases (17\%) were detained persons \cite{Alao_2006i, Ali_2022g, Apikotoa_2022f, Losanoff_1996, Losanoff_1997e, Qureshi_2016, Tammana_2012j, Trgo_2012f}, 7 cases (10\%) were severely disabled \cite{Atayan_2016, Kerestes_2019, Liu_2005, Ohno_2005, Peixoto_2017f, Yildiz_2016e, teWildt_2010}, 4 cases (6\%) were psychiatric inpatients \cite{DivsalarP._2023a, fjbuilsRepeatedBehaviorDeliberate2024, teWildt_2010}, 3 cases (4\%) were under the influence of alcohol \cite{Benoist_2019e, Csaky_1998e, Thapa_2019f}, 2 cases (3\%) were displaced people \cite{Akay_2015f, Gardner_2017h}. \paragraph*{Motivation} 34 cases (47\%) had a psychiatric motivation \cite{Al-Faham_2020k, Alao_2006i, Ali_2020f, Apikotoa_2022f, Ataya_2013, Atayan_2016, Bhasin_2014, Bhattacharjee_2008, DelgadoSalazar_2020c, DivsalarP._2023a, Emamhadi_2018, Farhadi_2024h, Guinan_2019f, Hardy_2023g, Jehangir_2019h, Jin_2023, Kar_2015, Kariholu_2008, Kerestes_2019, Kobiela_2015, Kumar_2001, Kumar_2019f, Li_2013, Liu_2005, Misra_2013, Ohno_2005, Sakellaridis_2008f, Sultan_2024f, Tammana_2012j, Tanrikulu_2015e, Yasin_2009, teWildt_2010}, 21 cases (29\%) were motivated by self-harm intention \cite{Al-Faham_2020k, AlShaaibi_2021b, Alao_2006i, Ali_2017, CamachoDorado_2018, Chang_2017f, Cox_2007, Csaky_1998e, Fry_2010, Li_2013, Losanoff_1996, Losanoff_1997e, Mesfin_2022a, Sakellaridis_2008f, Tammana_2012j, Tanrikulu_2015e, fjbuilsRepeatedBehaviorDeliberate2024}, 17 cases (24\%) had a psychosocial motivation \cite{Akay_2015f, Benoist_2019e, Bhattacharjee_2008, Cauchi_2002, Goldman_1998f, Hardy_2023g, Kobiela_2015, Li_2013, Naji_2012f, Qureshi_2016, Riva_2018j, Sobnach_2011f, Tay_2004, Thapa_2019f, Tupesis_2004f, Wildhaber_2005, Wnęk_2015f}, 9 cases (12\%) were motivated by protest \cite{Bhumi_2024f, Gardner_2017h, Losanoff_1996, Losanoff_1997e, Tupesis_2004f}, 9 cases (12\%) had another documented motivation \cite{Ali_2020f, Ali_2022g, Emamhadi_2018, Guinan_2019f, Peixoto_2017f, Sakellaridis_2008f, Trgo_2012f, Wadhwa_2015e, Yildiz_2016e}. \paragraph*{Object Characteristics} 51 cases (71\%) ingested a large diameter object (\textgreater{}2.5cm) \cite{Akay_2015f, Al-Faham_2020k, AlShaaibi_2021b, Alao_2006i, Ali_2017, Ali_2022g, Apikotoa_2022f, Atayan_2016, Berry_2021e, Bhasin_2014, CamachoDorado_2018, Cauchi_2002, Chang_2017f, Cox_2007, Csaky_1998e, DivsalarP._2023a, Emamhadi_2018, Gardner_2017h, Guinan_2019f, Jehangir_2019h, Jin_2023, Kariholu_2008, Kerestes_2019, Kobiela_2015, Kumar_2001, Kumar_2019f, Losanoff_1996, Losanoff_1997e, Mesfin_2022a, Misra_2013, Naji_2012f, Ohno_2005, Peixoto_2017f, Qureshi_2016, Riva_2018j, Sakellaridis_2008f, Sultan_2024f, Tanrikulu_2015e, Thapa_2019f, Trgo_2012f, Wnęk_2015f, Yildiz_2016e, fjbuilsRepeatedBehaviorDeliberate2024, teWildt_2010}, 44 cases (61\%) ingested multiple objects \cite{Ali_2020f, Apikotoa_2022f, Ataya_2013, Atayan_2016, Beecroft_1998, Bhattacharjee_2008, Bhumi_2024f, CamachoDorado_2018, Cauchi_2002, Emamhadi_2018, Farhadi_2024h, Fry_2010, Goldman_1998f, Guinan_2019f, Hardy_2023g, Jehangir_2019h, Jin_2023, Kar_2015, Kariholu_2008, Kobiela_2015, Kumar_2001, Kumar_2019f, Li_2013, Liu_2005, Losanoff_1996, Mesfin_2022a, Misra_2013, Naji_2012f, Ohno_2005, Sobnach_2011f, Sultan_2024f, Tammana_2012j, Tanrikulu_2015e, Tay_2004, Thapa_2019f, Wadhwa_2015e, Wildhaber_2005, Yasin_2009, fjbuilsRepeatedBehaviorDeliberate2024, teWildt_2010}, 34 cases (47\%) ingested a sharp object \cite{AlShaaibi_2021b, Alao_2006i, Apikotoa_2022f, Ataya_2013, Benoist_2019e, Bhasin_2014, Bhattacharjee_2008, CamachoDorado_2018, Csaky_1998e, DelgadoSalazar_2020c, DivsalarP._2023a, Emamhadi_2018, Farhadi_2024h, Fry_2010, Guinan_2019f, Hardy_2023g, Jehangir_2019h, Jin_2023, Kariholu_2008, Kobiela_2015, Kumar_2019f, Losanoff_1996, Losanoff_1997e, Mesfin_2022a, Misra_2013, Sobnach_2011f, Yasin_2009, teWildt_2010}, 32 cases (44\%) ingested a long object (\textgreater{}5cm) \cite{Al-Faham_2020k, AlShaaibi_2021b, Ali_2017, Ali_2022g, Atayan_2016, Bhasin_2014, CamachoDorado_2018, Chang_2017f, Cox_2007, Csaky_1998e, DivsalarP._2023a, Emamhadi_2018, Fry_2010, Gardner_2017h, Jin_2023, Kariholu_2008, Kerestes_2019, Kobiela_2015, Kumar_2019f, Mesfin_2022a, Misra_2013, Ohno_2005, Qureshi_2016, Sakellaridis_2008f, Sultan_2024f, Thapa_2019f, Trgo_2012f, Yasin_2009, Yildiz_2016e, teWildt_2010}, 9 cases (12\%) ingested a magnet \cite{Ali_2020f, Bhumi_2024f, Cauchi_2002, Liu_2005, Naji_2012f, Ohno_2005, Tanrikulu_2015e, Tay_2004, Wildhaber_2005}, 2 cases (3\%) ingested a button battery \cite{Berry_2021e, Bhumi_2024f}. \paragraph*{Outcomes} 48 cases (67\%) experienced a complication \cite{Ali_2017, Ali_2020f, Apikotoa_2022f, Atayan_2016, Beecroft_1998, Benoist_2019e, Berry_2021e, Bhasin_2014, Bhumi_2024f, CamachoDorado_2018, Cauchi_2002, Cox_2007, Csaky_1998e, DelgadoSalazar_2020c, DivsalarP._2023a, Emamhadi_2018, Farhadi_2024h, Fry_2010, Gardner_2017h, Goldman_1998f, Jin_2023, Kariholu_2008, Kerestes_2019, Kobiela_2015, Kumar_2001, Kumar_2019f, Liu_2005, Losanoff_1996, Mesfin_2022a, Misra_2013, Naji_2012f, Ohno_2005, Sakellaridis_2008f, Sobnach_2011f, Sultan_2024f, Tanrikulu_2015e, Tay_2004, Thapa_2019f, Trgo_2012f, Tupesis_2004f, Wildhaber_2005, Wnęk_2015f, Yasin_2009, Yildiz_2016e}, 44 cases (61\%) underwent surgery \cite{Al-Faham_2020k, AlShaaibi_2021b, Alao_2006i, Ali_2017, Ali_2020f, Atayan_2016, Beecroft_1998, Bhasin_2014, CamachoDorado_2018, Cauchi_2002, Chang_2017f, Cox_2007, Csaky_1998e, DelgadoSalazar_2020c, DivsalarP._2023a, Farhadi_2024h, Fry_2010, Gardner_2017h, Jin_2023, Kariholu_2008, Kerestes_2019, Kobiela_2015, Kumar_2019f, Liu_2005, Losanoff_1996, Losanoff_1997e, Mesfin_2022a, Misra_2013, Naji_2012f, Sobnach_2011f, Tanrikulu_2015e, Tay_2004, Thapa_2019f, Tupesis_2004f, Wildhaber_2005, Wnęk_2015f, Yasin_2009, Yildiz_2016e, fjbuilsRepeatedBehaviorDeliberate2024}, 31 cases (43\%) underwent endoscopy \cite{Akay_2015f, Ali_2022g, Apikotoa_2022f, Atayan_2016, Benoist_2019e, Berry_2021e, Bhasin_2014, Bhumi_2024f, CamachoDorado_2018, Chang_2017f, DelgadoSalazar_2020c, Gardner_2017h, Guinan_2019f, Hardy_2023g, Jehangir_2019h, Kariholu_2008, Li_2013, Liu_2005, Ohno_2005, Peixoto_2017f, Qureshi_2016, Riva_2018j, Sakellaridis_2008f, Sultan_2024f, Tammana_2012j, Tanrikulu_2015e, Trgo_2012f, Wadhwa_2015e, Wnęk_2015f, teWildt_2010}, 7 cases (10\%) were managed conservatively \cite{Ataya_2013, Bhattacharjee_2008, DivsalarP._2023a, Emamhadi_2018, Goldman_1998f, Kar_2015, Kumar_2001}, 2 cases (3\%) died \cite{Emamhadi_2018, Kumar_2001}. All 90 were male gender. 90 cases (100\%) were detained at the time of ingestion \cite{Elghali_2016, Karp_1991b, Lee_2007}, 88 cases (98\%) were intentional ingestions \cite{Elghali_2016, Karp_1991b, Lee_2007}, 30 cases (33\%) had a psychiatric history documented \cite{Elghali_2016, Karp_1991b, Lee_2007}, 2 cases (2\%) had a history of prior ingestion \cite{Elghali_2016}. No cases were reported for were psychiatric inpatients, were displaced people, were under the influence of alcohol at the time of ingestion, and had a severe disability history.
\paragraph*{Motivation}  70 cases (78\%) reported protest motivation \cite{Elghali_2016, Karp_1991b, Lee_2007}, 12 cases (13\%) reported psychiatric motivation \cite{Karp_1991b}, 6 cases (7\%) reported self-harm motivation \cite{Elghali_2016, Karp_1991b}. No cases were reported for psychosocial motivation and other motivation.
\paragraph*{Object Characteristics}  68 cases (76\%) involved sharp object ingestion \cite{Elghali_2016, Karp_1991b, Lee_2007}, 32 cases (36\%) involved long (\textgreater 5cm) object ingestion \cite{Lee_2007}, 25 cases (28\%) involved ingestion of multiple objects \cite{Elghali_2016, Lee_2007}. No cases were reported for button battery ingestion, magnet ingestion, and involved large diameter (\textgreater 2.5cm) object ingestion.
\paragraph*{Outcomes}  47 cases (52\%) underwent endoscopic intervention \cite{Elghali_2016, Lee_2007}, 29 cases (32\%) were managed conservatively \cite{Elghali_2016, Karp_1991b}, 15 cases (17\%) underwent surgical intervention \cite{Elghali_2016, Karp_1991b, Lee_2007}, 6 cases (7\%) reported complications \cite{Lee_2007}, 1 case (1\%) died \cite{Elghali_2016}.
\paragraph*{Geographical Location}Cases were recorded in 33 countries: 13 cases from USA \cite{Alao_2006i, Ataya_2013, Bhumi_2024f, Fry_2010, Guinan_2019f, Hardy_2023g, Jehangir_2019h, Kerestes_2019, Kumar_2001, Liu_2005, Tammana_2012j, Tay_2004, Tupesis_2004f}; 7 cases from India \cite{Bhasin_2014, Bhattacharjee_2008, Kar_2015, Kariholu_2008, Kumar_2019f, Misra_2013, Wadhwa_2015e} and UK \cite{Beecroft_1998, Berry_2021e, Cauchi_2002, Cox_2007, Gardner_2017h, Qureshi_2016}; 6 cases from Bulgaria \cite{Losanoff_1996, Losanoff_1997e}; 5 cases from Iran \cite{DivsalarP._2023a, Emamhadi_2018, Farhadi_2024h}; 4 cases from Turkey \cite{Akay_2015f, Atayan_2016, Tanrikulu_2015e, Yildiz_2016e}; 2 cases from China \cite{Jin_2023, Li_2013}, Poland \cite{Kobiela_2015, Wnęk_2015f}, and Spain \cite{CamachoDorado_2018, fjbuilsRepeatedBehaviorDeliberate2024}; 1 case from Australia \cite{Apikotoa_2022f}, Bahrain \cite{Ali_2020f}, Croatia \cite{Trgo_2012f}, Ecuador \cite{DelgadoSalazar_2020c}, Egypt \cite{Ali_2022g}, Ethiopia \cite{Mesfin_2022a}, Germany \cite{teWildt_2010}, Greece \cite{Sakellaridis_2008f}, Hungary \cite{Csaky_1998e}, Iraq \cite{Al-Faham_2020k}, Israel \cite{Goldman_1998f}, Italy \cite{Riva_2018j}, Japan \cite{Ohno_2005}, Nepal \cite{Thapa_2019f}, Netherlands \cite{Benoist_2019e}, Oman \cite{AlShaaibi_2021b}, Pakistan \cite{Yasin_2009}, Portugal \cite{Peixoto_2017f}, Qatar \cite{Ali_2017}, Saudi Arabia \cite{Sultan_2024f}, South Africa \cite{Sobnach_2011f}, Sweden \cite{Naji_2012f}, Switzerland \cite{Wildhaber_2005}, and Taiwan \cite{Chang_2017f}. \paragraph*{Gender} 43 cases (60\%) were male \cite{Akay_2015f, Al-Faham_2020k, Alao_2006i, Ali_2017, Ali_2022g, Apikotoa_2022f, Atayan_2016, Benoist_2019e, Berry_2021e, Bhumi_2024f, CamachoDorado_2018, Csaky_1998e, Emamhadi_2018, Farhadi_2024h, Fry_2010, Gardner_2017h, Guinan_2019f, Jehangir_2019h, Jin_2023, Kobiela_2015, Kumar_2001, Kumar_2019f, Liu_2005, Losanoff_1996, Losanoff_1997e, Mesfin_2022a, Misra_2013, Qureshi_2016, Riva_2018j, Sobnach_2011f, Tammana_2012j, Tanrikulu_2015e, Tay_2004, Thapa_2019f, Trgo_2012f, Wadhwa_2015e, Yasin_2009, teWildt_2010}, 28 cases (39\%) were female \cite{AlShaaibi_2021b, Ali_2020f, Ataya_2013, Beecroft_1998, Bhasin_2014, Bhattacharjee_2008, Cauchi_2002, Chang_2017f, Cox_2007, DelgadoSalazar_2020c, DivsalarP._2023a, Goldman_1998f, Hardy_2023g, Kar_2015, Kariholu_2008, Kerestes_2019, Li_2013, Naji_2012f, Ohno_2005, Peixoto_2017f, Sakellaridis_2008f, Sultan_2024f, Tupesis_2004f, Wildhaber_2005, Wnęk_2015f, Yildiz_2016e}, 1 case (1\%) had no gender recorded \cite{fjbuilsRepeatedBehaviorDeliberate2024}. \paragraph*{Age Group} 25 cases (35\%) were between 26 and 40 years of age \cite{Alao_2006i, Ali_2022g, Apikotoa_2022f, Ataya_2013, Benoist_2019e, Bhasin_2014, Chang_2017f, Cox_2007, DelgadoSalazar_2020c, Farhadi_2024h, Fry_2010, Gardner_2017h, Guinan_2019f, Jin_2023, Kumar_2019f, Losanoff_1996, Misra_2013, Qureshi_2016, Riva_2018j, Sakellaridis_2008f, Tammana_2012j, Trgo_2012f, Wnęk_2015f, Yildiz_2016e, fjbuilsRepeatedBehaviorDeliberate2024}, 18 cases (25\%) were between 18 and 25 years of age \cite{Akay_2015f, Ali_2017, Atayan_2016, Bhattacharjee_2008, Csaky_1998e, Kar_2015, Kariholu_2008, Kobiela_2015, Losanoff_1996, Losanoff_1997e, Mesfin_2022a, Peixoto_2017f, Sobnach_2011f, Tupesis_2004f, Yasin_2009}, 13 cases (18\%) were under 18 years of age \cite{AlShaaibi_2021b, Ali_2020f, Cauchi_2002, DivsalarP._2023a, Goldman_1998f, Liu_2005, Naji_2012f, Ohno_2005, Tanrikulu_2015e, Tay_2004, Wildhaber_2005}, 11 cases (15\%) were between 41 and 60 years of age \cite{Al-Faham_2020k, Bhumi_2024f, CamachoDorado_2018, Emamhadi_2018, Hardy_2023g, Jehangir_2019h, Kumar_2001, Sultan_2024f, Thapa_2019f, Wadhwa_2015e, teWildt_2010}, 3 cases (4\%) were over 60 years of age \cite{Beecroft_1998, Kerestes_2019, Li_2013}, 2 cases (3\%) had no age documented \cite{Berry_2021e}. \paragraph*{Population} 36 cases (50\%) had a psychiatric history \cite{AlShaaibi_2021b, Alao_2006i, Ali_2020f, Apikotoa_2022f, Ataya_2013, Atayan_2016, Beecroft_1998, CamachoDorado_2018, Chang_2017f, DelgadoSalazar_2020c, DivsalarP._2023a, Farhadi_2024h, Fry_2010, Guinan_2019f, Hardy_2023g, Jehangir_2019h, Jin_2023, Kar_2015, Kerestes_2019, Kobiela_2015, Kumar_2001, Kumar_2019f, Liu_2005, Mesfin_2022a, Misra_2013, Ohno_2005, Peixoto_2017f, Sakellaridis_2008f, Sultan_2024f, Tammana_2012j, Tanrikulu_2015e, Yildiz_2016e, fjbuilsRepeatedBehaviorDeliberate2024, teWildt_2010}, 19 cases (26\%) had ingested previously \cite{Alao_2006i, Apikotoa_2022f, Berry_2021e, Bhattacharjee_2008, Csaky_1998e, DivsalarP._2023a, Emamhadi_2018, Guinan_2019f, Jehangir_2019h, Jin_2023, Liu_2005, Sakellaridis_2008f, Tanrikulu_2015e, Thapa_2019f, Yildiz_2016e, fjbuilsRepeatedBehaviorDeliberate2024, teWildt_2010}, 12 cases (17\%) were detained persons \cite{Alao_2006i, Ali_2022g, Apikotoa_2022f, Losanoff_1996, Losanoff_1997e, Qureshi_2016, Tammana_2012j, Trgo_2012f}, 7 cases (10\%) were severely disabled \cite{Atayan_2016, Kerestes_2019, Liu_2005, Ohno_2005, Peixoto_2017f, Yildiz_2016e, teWildt_2010}, 4 cases (6\%) were psychiatric inpatients \cite{DivsalarP._2023a, fjbuilsRepeatedBehaviorDeliberate2024, teWildt_2010}, 3 cases (4\%) were under the influence of alcohol \cite{Benoist_2019e, Csaky_1998e, Thapa_2019f}, 2 cases (3\%) were displaced people \cite{Akay_2015f, Gardner_2017h}. \paragraph*{Motivation} 34 cases (47\%) had a psychiatric motivation \cite{Al-Faham_2020k, Alao_2006i, Ali_2020f, Apikotoa_2022f, Ataya_2013, Atayan_2016, Bhasin_2014, Bhattacharjee_2008, DelgadoSalazar_2020c, DivsalarP._2023a, Emamhadi_2018, Farhadi_2024h, Guinan_2019f, Hardy_2023g, Jehangir_2019h, Jin_2023, Kar_2015, Kariholu_2008, Kerestes_2019, Kobiela_2015, Kumar_2001, Kumar_2019f, Li_2013, Liu_2005, Misra_2013, Ohno_2005, Sakellaridis_2008f, Sultan_2024f, Tammana_2012j, Tanrikulu_2015e, Yasin_2009, teWildt_2010}, 21 cases (29\%) were motivated by self-harm intention \cite{Al-Faham_2020k, AlShaaibi_2021b, Alao_2006i, Ali_2017, CamachoDorado_2018, Chang_2017f, Cox_2007, Csaky_1998e, Fry_2010, Li_2013, Losanoff_1996, Losanoff_1997e, Mesfin_2022a, Sakellaridis_2008f, Tammana_2012j, Tanrikulu_2015e, fjbuilsRepeatedBehaviorDeliberate2024}, 17 cases (24\%) had a psychosocial motivation \cite{Akay_2015f, Benoist_2019e, Bhattacharjee_2008, Cauchi_2002, Goldman_1998f, Hardy_2023g, Kobiela_2015, Li_2013, Naji_2012f, Qureshi_2016, Riva_2018j, Sobnach_2011f, Tay_2004, Thapa_2019f, Tupesis_2004f, Wildhaber_2005, Wnęk_2015f}, 9 cases (12\%) were motivated by protest \cite{Bhumi_2024f, Gardner_2017h, Losanoff_1996, Losanoff_1997e, Tupesis_2004f}, 9 cases (12\%) had another documented motivation \cite{Ali_2020f, Ali_2022g, Emamhadi_2018, Guinan_2019f, Peixoto_2017f, Sakellaridis_2008f, Trgo_2012f, Wadhwa_2015e, Yildiz_2016e}. \paragraph*{Object Characteristics} 51 cases (71\%) ingested a large diameter object (\textgreater{}2.5cm) \cite{Akay_2015f, Al-Faham_2020k, AlShaaibi_2021b, Alao_2006i, Ali_2017, Ali_2022g, Apikotoa_2022f, Atayan_2016, Berry_2021e, Bhasin_2014, CamachoDorado_2018, Cauchi_2002, Chang_2017f, Cox_2007, Csaky_1998e, DivsalarP._2023a, Emamhadi_2018, Gardner_2017h, Guinan_2019f, Jehangir_2019h, Jin_2023, Kariholu_2008, Kerestes_2019, Kobiela_2015, Kumar_2001, Kumar_2019f, Losanoff_1996, Losanoff_1997e, Mesfin_2022a, Misra_2013, Naji_2012f, Ohno_2005, Peixoto_2017f, Qureshi_2016, Riva_2018j, Sakellaridis_2008f, Sultan_2024f, Tanrikulu_2015e, Thapa_2019f, Trgo_2012f, Wnęk_2015f, Yildiz_2016e, fjbuilsRepeatedBehaviorDeliberate2024, teWildt_2010}, 44 cases (61\%) ingested multiple objects \cite{Ali_2020f, Apikotoa_2022f, Ataya_2013, Atayan_2016, Beecroft_1998, Bhattacharjee_2008, Bhumi_2024f, CamachoDorado_2018, Cauchi_2002, Emamhadi_2018, Farhadi_2024h, Fry_2010, Goldman_1998f, Guinan_2019f, Hardy_2023g, Jehangir_2019h, Jin_2023, Kar_2015, Kariholu_2008, Kobiela_2015, Kumar_2001, Kumar_2019f, Li_2013, Liu_2005, Losanoff_1996, Mesfin_2022a, Misra_2013, Naji_2012f, Ohno_2005, Sobnach_2011f, Sultan_2024f, Tammana_2012j, Tanrikulu_2015e, Tay_2004, Thapa_2019f, Wadhwa_2015e, Wildhaber_2005, Yasin_2009, fjbuilsRepeatedBehaviorDeliberate2024, teWildt_2010}, 34 cases (47\%) ingested a sharp object \cite{AlShaaibi_2021b, Alao_2006i, Apikotoa_2022f, Ataya_2013, Benoist_2019e, Bhasin_2014, Bhattacharjee_2008, CamachoDorado_2018, Csaky_1998e, DelgadoSalazar_2020c, DivsalarP._2023a, Emamhadi_2018, Farhadi_2024h, Fry_2010, Guinan_2019f, Hardy_2023g, Jehangir_2019h, Jin_2023, Kariholu_2008, Kobiela_2015, Kumar_2019f, Losanoff_1996, Losanoff_1997e, Mesfin_2022a, Misra_2013, Sobnach_2011f, Yasin_2009, teWildt_2010}, 32 cases (44\%) ingested a long object (\textgreater{}5cm) \cite{Al-Faham_2020k, AlShaaibi_2021b, Ali_2017, Ali_2022g, Atayan_2016, Bhasin_2014, CamachoDorado_2018, Chang_2017f, Cox_2007, Csaky_1998e, DivsalarP._2023a, Emamhadi_2018, Fry_2010, Gardner_2017h, Jin_2023, Kariholu_2008, Kerestes_2019, Kobiela_2015, Kumar_2019f, Mesfin_2022a, Misra_2013, Ohno_2005, Qureshi_2016, Sakellaridis_2008f, Sultan_2024f, Thapa_2019f, Trgo_2012f, Yasin_2009, Yildiz_2016e, teWildt_2010}, 9 cases (12\%) ingested a magnet \cite{Ali_2020f, Bhumi_2024f, Cauchi_2002, Liu_2005, Naji_2012f, Ohno_2005, Tanrikulu_2015e, Tay_2004, Wildhaber_2005}, 2 cases (3\%) ingested a button battery \cite{Berry_2021e, Bhumi_2024f}. \paragraph*{Outcomes} 48 cases (67\%) experienced a complication \cite{Ali_2017, Ali_2020f, Apikotoa_2022f, Atayan_2016, Beecroft_1998, Benoist_2019e, Berry_2021e, Bhasin_2014, Bhumi_2024f, CamachoDorado_2018, Cauchi_2002, Cox_2007, Csaky_1998e, DelgadoSalazar_2020c, DivsalarP._2023a, Emamhadi_2018, Farhadi_2024h, Fry_2010, Gardner_2017h, Goldman_1998f, Jin_2023, Kariholu_2008, Kerestes_2019, Kobiela_2015, Kumar_2001, Kumar_2019f, Liu_2005, Losanoff_1996, Mesfin_2022a, Misra_2013, Naji_2012f, Ohno_2005, Sakellaridis_2008f, Sobnach_2011f, Sultan_2024f, Tanrikulu_2015e, Tay_2004, Thapa_2019f, Trgo_2012f, Tupesis_2004f, Wildhaber_2005, Wnęk_2015f, Yasin_2009, Yildiz_2016e}, 44 cases (61\%) underwent surgery \cite{Al-Faham_2020k, AlShaaibi_2021b, Alao_2006i, Ali_2017, Ali_2020f, Atayan_2016, Beecroft_1998, Bhasin_2014, CamachoDorado_2018, Cauchi_2002, Chang_2017f, Cox_2007, Csaky_1998e, DelgadoSalazar_2020c, DivsalarP._2023a, Farhadi_2024h, Fry_2010, Gardner_2017h, Jin_2023, Kariholu_2008, Kerestes_2019, Kobiela_2015, Kumar_2019f, Liu_2005, Losanoff_1996, Losanoff_1997e, Mesfin_2022a, Misra_2013, Naji_2012f, Sobnach_2011f, Tanrikulu_2015e, Tay_2004, Thapa_2019f, Tupesis_2004f, Wildhaber_2005, Wnęk_2015f, Yasin_2009, Yildiz_2016e, fjbuilsRepeatedBehaviorDeliberate2024}, 31 cases (43\%) underwent endoscopy \cite{Akay_2015f, Ali_2022g, Apikotoa_2022f, Atayan_2016, Benoist_2019e, Berry_2021e, Bhasin_2014, Bhumi_2024f, CamachoDorado_2018, Chang_2017f, DelgadoSalazar_2020c, Gardner_2017h, Guinan_2019f, Hardy_2023g, Jehangir_2019h, Kariholu_2008, Li_2013, Liu_2005, Ohno_2005, Peixoto_2017f, Qureshi_2016, Riva_2018j, Sakellaridis_2008f, Sultan_2024f, Tammana_2012j, Tanrikulu_2015e, Trgo_2012f, Wadhwa_2015e, Wnęk_2015f, teWildt_2010}, 7 cases (10\%) were managed conservatively \cite{Ataya_2013, Bhattacharjee_2008, DivsalarP._2023a, Emamhadi_2018, Goldman_1998f, Kar_2015, Kumar_2001}, 2 cases (3\%) died \cite{Emamhadi_2018, Kumar_2001}. All 90 were male gender. 90 cases (100\%) were detained at the time of ingestion \cite{Elghali_2016, Karp_1991b, Lee_2007}, 88 cases (98\%) were intentional ingestions \cite{Elghali_2016, Karp_1991b, Lee_2007}, 30 cases (33\%) had a psychiatric history documented \cite{Elghali_2016, Karp_1991b, Lee_2007}, 2 cases (2\%) had a history of prior ingestion \cite{Elghali_2016}. No cases were reported for were psychiatric inpatients, were displaced people, were under the influence of alcohol at the time of ingestion, and had a severe disability history.
\paragraph*{Motivation}  70 cases (78\%) reported protest motivation \cite{Elghali_2016, Karp_1991b, Lee_2007}, 12 cases (13\%) reported psychiatric motivation \cite{Karp_1991b}, 6 cases (7\%) reported self-harm motivation \cite{Elghali_2016, Karp_1991b}. No cases were reported for psychosocial motivation and other motivation.
\paragraph*{Object Characteristics}  68 cases (76\%) involved sharp object ingestion \cite{Elghali_2016, Karp_1991b, Lee_2007}, 32 cases (36\%) involved long (\textgreater 5cm) object ingestion \cite{Lee_2007}, 25 cases (28\%) involved ingestion of multiple objects \cite{Elghali_2016, Lee_2007}. No cases were reported for button battery ingestion, magnet ingestion, and involved large diameter (\textgreater 2.5cm) object ingestion.
\paragraph*{Outcomes}  47 cases (52\%) underwent endoscopic intervention \cite{Elghali_2016, Lee_2007}, 29 cases (32\%) were managed conservatively \cite{Elghali_2016, Karp_1991b}, 15 cases (17\%) underwent surgical intervention \cite{Elghali_2016, Karp_1991b, Lee_2007}, 6 cases (7\%) reported complications \cite{Lee_2007}, 1 case (1\%) died \cite{Elghali_2016}.
\paragraph*{Geographical Location}Cases were recorded in 33 countries: 13 cases from USA \cite{Alao_2006i, Ataya_2013, Bhumi_2024f, Fry_2010, Guinan_2019f, Hardy_2023g, Jehangir_2019h, Kerestes_2019, Kumar_2001, Liu_2005, Tammana_2012j, Tay_2004, Tupesis_2004f}; 7 cases from India \cite{Bhasin_2014, Bhattacharjee_2008, Kar_2015, Kariholu_2008, Kumar_2019f, Misra_2013, Wadhwa_2015e} and UK \cite{Beecroft_1998, Berry_2021e, Cauchi_2002, Cox_2007, Gardner_2017h, Qureshi_2016}; 6 cases from Bulgaria \cite{Losanoff_1996, Losanoff_1997e}; 5 cases from Iran \cite{DivsalarP._2023a, Emamhadi_2018, Farhadi_2024h}; 4 cases from Turkey \cite{Akay_2015f, Atayan_2016, Tanrikulu_2015e, Yildiz_2016e}; 2 cases from China \cite{Jin_2023, Li_2013}, Poland \cite{Kobiela_2015, Wnęk_2015f}, and Spain \cite{CamachoDorado_2018, fjbuilsRepeatedBehaviorDeliberate2024}; 1 case from Australia \cite{Apikotoa_2022f}, Bahrain \cite{Ali_2020f}, Croatia \cite{Trgo_2012f}, Ecuador \cite{DelgadoSalazar_2020c}, Egypt \cite{Ali_2022g}, Ethiopia \cite{Mesfin_2022a}, Germany \cite{teWildt_2010}, Greece \cite{Sakellaridis_2008f}, Hungary \cite{Csaky_1998e}, Iraq \cite{Al-Faham_2020k}, Israel \cite{Goldman_1998f}, Italy \cite{Riva_2018j}, Japan \cite{Ohno_2005}, Nepal \cite{Thapa_2019f}, Netherlands \cite{Benoist_2019e}, Oman \cite{AlShaaibi_2021b}, Pakistan \cite{Yasin_2009}, Portugal \cite{Peixoto_2017f}, Qatar \cite{Ali_2017}, Saudi Arabia \cite{Sultan_2024f}, South Africa \cite{Sobnach_2011f}, Sweden \cite{Naji_2012f}, Switzerland \cite{Wildhaber_2005}, and Taiwan \cite{Chang_2017f}. \paragraph*{Gender} 43 cases (60\%) were male \cite{Akay_2015f, Al-Faham_2020k, Alao_2006i, Ali_2017, Ali_2022g, Apikotoa_2022f, Atayan_2016, Benoist_2019e, Berry_2021e, Bhumi_2024f, CamachoDorado_2018, Csaky_1998e, Emamhadi_2018, Farhadi_2024h, Fry_2010, Gardner_2017h, Guinan_2019f, Jehangir_2019h, Jin_2023, Kobiela_2015, Kumar_2001, Kumar_2019f, Liu_2005, Losanoff_1996, Losanoff_1997e, Mesfin_2022a, Misra_2013, Qureshi_2016, Riva_2018j, Sobnach_2011f, Tammana_2012j, Tanrikulu_2015e, Tay_2004, Thapa_2019f, Trgo_2012f, Wadhwa_2015e, Yasin_2009, teWildt_2010}, 28 cases (39\%) were female \cite{AlShaaibi_2021b, Ali_2020f, Ataya_2013, Beecroft_1998, Bhasin_2014, Bhattacharjee_2008, Cauchi_2002, Chang_2017f, Cox_2007, DelgadoSalazar_2020c, DivsalarP._2023a, Goldman_1998f, Hardy_2023g, Kar_2015, Kariholu_2008, Kerestes_2019, Li_2013, Naji_2012f, Ohno_2005, Peixoto_2017f, Sakellaridis_2008f, Sultan_2024f, Tupesis_2004f, Wildhaber_2005, Wnęk_2015f, Yildiz_2016e}, 1 case (1\%) had no gender recorded \cite{fjbuilsRepeatedBehaviorDeliberate2024}. \paragraph*{Age Group} 25 cases (35\%) were between 26 and 40 years of age \cite{Alao_2006i, Ali_2022g, Apikotoa_2022f, Ataya_2013, Benoist_2019e, Bhasin_2014, Chang_2017f, Cox_2007, DelgadoSalazar_2020c, Farhadi_2024h, Fry_2010, Gardner_2017h, Guinan_2019f, Jin_2023, Kumar_2019f, Losanoff_1996, Misra_2013, Qureshi_2016, Riva_2018j, Sakellaridis_2008f, Tammana_2012j, Trgo_2012f, Wnęk_2015f, Yildiz_2016e, fjbuilsRepeatedBehaviorDeliberate2024}, 18 cases (25\%) were between 18 and 25 years of age \cite{Akay_2015f, Ali_2017, Atayan_2016, Bhattacharjee_2008, Csaky_1998e, Kar_2015, Kariholu_2008, Kobiela_2015, Losanoff_1996, Losanoff_1997e, Mesfin_2022a, Peixoto_2017f, Sobnach_2011f, Tupesis_2004f, Yasin_2009}, 13 cases (18\%) were under 18 years of age \cite{AlShaaibi_2021b, Ali_2020f, Cauchi_2002, DivsalarP._2023a, Goldman_1998f, Liu_2005, Naji_2012f, Ohno_2005, Tanrikulu_2015e, Tay_2004, Wildhaber_2005}, 11 cases (15\%) were between 41 and 60 years of age \cite{Al-Faham_2020k, Bhumi_2024f, CamachoDorado_2018, Emamhadi_2018, Hardy_2023g, Jehangir_2019h, Kumar_2001, Sultan_2024f, Thapa_2019f, Wadhwa_2015e, teWildt_2010}, 3 cases (4\%) were over 60 years of age \cite{Beecroft_1998, Kerestes_2019, Li_2013}, 2 cases (3\%) had no age documented \cite{Berry_2021e}. \paragraph*{Population} 36 cases (50\%) had a psychiatric history \cite{AlShaaibi_2021b, Alao_2006i, Ali_2020f, Apikotoa_2022f, Ataya_2013, Atayan_2016, Beecroft_1998, CamachoDorado_2018, Chang_2017f, DelgadoSalazar_2020c, DivsalarP._2023a, Farhadi_2024h, Fry_2010, Guinan_2019f, Hardy_2023g, Jehangir_2019h, Jin_2023, Kar_2015, Kerestes_2019, Kobiela_2015, Kumar_2001, Kumar_2019f, Liu_2005, Mesfin_2022a, Misra_2013, Ohno_2005, Peixoto_2017f, Sakellaridis_2008f, Sultan_2024f, Tammana_2012j, Tanrikulu_2015e, Yildiz_2016e, fjbuilsRepeatedBehaviorDeliberate2024, teWildt_2010}, 19 cases (26\%) had ingested previously \cite{Alao_2006i, Apikotoa_2022f, Berry_2021e, Bhattacharjee_2008, Csaky_1998e, DivsalarP._2023a, Emamhadi_2018, Guinan_2019f, Jehangir_2019h, Jin_2023, Liu_2005, Sakellaridis_2008f, Tanrikulu_2015e, Thapa_2019f, Yildiz_2016e, fjbuilsRepeatedBehaviorDeliberate2024, teWildt_2010}, 12 cases (17\%) were detained persons \cite{Alao_2006i, Ali_2022g, Apikotoa_2022f, Losanoff_1996, Losanoff_1997e, Qureshi_2016, Tammana_2012j, Trgo_2012f}, 7 cases (10\%) were severely disabled \cite{Atayan_2016, Kerestes_2019, Liu_2005, Ohno_2005, Peixoto_2017f, Yildiz_2016e, teWildt_2010}, 4 cases (6\%) were psychiatric inpatients \cite{DivsalarP._2023a, fjbuilsRepeatedBehaviorDeliberate2024, teWildt_2010}, 3 cases (4\%) were under the influence of alcohol \cite{Benoist_2019e, Csaky_1998e, Thapa_2019f}, 2 cases (3\%) were displaced people \cite{Akay_2015f, Gardner_2017h}. \paragraph*{Motivation} 34 cases (47\%) had a psychiatric motivation \cite{Al-Faham_2020k, Alao_2006i, Ali_2020f, Apikotoa_2022f, Ataya_2013, Atayan_2016, Bhasin_2014, Bhattacharjee_2008, DelgadoSalazar_2020c, DivsalarP._2023a, Emamhadi_2018, Farhadi_2024h, Guinan_2019f, Hardy_2023g, Jehangir_2019h, Jin_2023, Kar_2015, Kariholu_2008, Kerestes_2019, Kobiela_2015, Kumar_2001, Kumar_2019f, Li_2013, Liu_2005, Misra_2013, Ohno_2005, Sakellaridis_2008f, Sultan_2024f, Tammana_2012j, Tanrikulu_2015e, Yasin_2009, teWildt_2010}, 21 cases (29\%) were motivated by self-harm intention \cite{Al-Faham_2020k, AlShaaibi_2021b, Alao_2006i, Ali_2017, CamachoDorado_2018, Chang_2017f, Cox_2007, Csaky_1998e, Fry_2010, Li_2013, Losanoff_1996, Losanoff_1997e, Mesfin_2022a, Sakellaridis_2008f, Tammana_2012j, Tanrikulu_2015e, fjbuilsRepeatedBehaviorDeliberate2024}, 17 cases (24\%) had a psychosocial motivation \cite{Akay_2015f, Benoist_2019e, Bhattacharjee_2008, Cauchi_2002, Goldman_1998f, Hardy_2023g, Kobiela_2015, Li_2013, Naji_2012f, Qureshi_2016, Riva_2018j, Sobnach_2011f, Tay_2004, Thapa_2019f, Tupesis_2004f, Wildhaber_2005, Wnęk_2015f}, 9 cases (12\%) were motivated by protest \cite{Bhumi_2024f, Gardner_2017h, Losanoff_1996, Losanoff_1997e, Tupesis_2004f}, 9 cases (12\%) had another documented motivation \cite{Ali_2020f, Ali_2022g, Emamhadi_2018, Guinan_2019f, Peixoto_2017f, Sakellaridis_2008f, Trgo_2012f, Wadhwa_2015e, Yildiz_2016e}. \paragraph*{Object Characteristics} 51 cases (71\%) ingested a large diameter object (\textgreater{}2.5cm) \cite{Akay_2015f, Al-Faham_2020k, AlShaaibi_2021b, Alao_2006i, Ali_2017, Ali_2022g, Apikotoa_2022f, Atayan_2016, Berry_2021e, Bhasin_2014, CamachoDorado_2018, Cauchi_2002, Chang_2017f, Cox_2007, Csaky_1998e, DivsalarP._2023a, Emamhadi_2018, Gardner_2017h, Guinan_2019f, Jehangir_2019h, Jin_2023, Kariholu_2008, Kerestes_2019, Kobiela_2015, Kumar_2001, Kumar_2019f, Losanoff_1996, Losanoff_1997e, Mesfin_2022a, Misra_2013, Naji_2012f, Ohno_2005, Peixoto_2017f, Qureshi_2016, Riva_2018j, Sakellaridis_2008f, Sultan_2024f, Tanrikulu_2015e, Thapa_2019f, Trgo_2012f, Wnęk_2015f, Yildiz_2016e, fjbuilsRepeatedBehaviorDeliberate2024, teWildt_2010}, 44 cases (61\%) ingested multiple objects \cite{Ali_2020f, Apikotoa_2022f, Ataya_2013, Atayan_2016, Beecroft_1998, Bhattacharjee_2008, Bhumi_2024f, CamachoDorado_2018, Cauchi_2002, Emamhadi_2018, Farhadi_2024h, Fry_2010, Goldman_1998f, Guinan_2019f, Hardy_2023g, Jehangir_2019h, Jin_2023, Kar_2015, Kariholu_2008, Kobiela_2015, Kumar_2001, Kumar_2019f, Li_2013, Liu_2005, Losanoff_1996, Mesfin_2022a, Misra_2013, Naji_2012f, Ohno_2005, Sobnach_2011f, Sultan_2024f, Tammana_2012j, Tanrikulu_2015e, Tay_2004, Thapa_2019f, Wadhwa_2015e, Wildhaber_2005, Yasin_2009, fjbuilsRepeatedBehaviorDeliberate2024, teWildt_2010}, 34 cases (47\%) ingested a sharp object \cite{AlShaaibi_2021b, Alao_2006i, Apikotoa_2022f, Ataya_2013, Benoist_2019e, Bhasin_2014, Bhattacharjee_2008, CamachoDorado_2018, Csaky_1998e, DelgadoSalazar_2020c, DivsalarP._2023a, Emamhadi_2018, Farhadi_2024h, Fry_2010, Guinan_2019f, Hardy_2023g, Jehangir_2019h, Jin_2023, Kariholu_2008, Kobiela_2015, Kumar_2019f, Losanoff_1996, Losanoff_1997e, Mesfin_2022a, Misra_2013, Sobnach_2011f, Yasin_2009, teWildt_2010}, 32 cases (44\%) ingested a long object (\textgreater{}5cm) \cite{Al-Faham_2020k, AlShaaibi_2021b, Ali_2017, Ali_2022g, Atayan_2016, Bhasin_2014, CamachoDorado_2018, Chang_2017f, Cox_2007, Csaky_1998e, DivsalarP._2023a, Emamhadi_2018, Fry_2010, Gardner_2017h, Jin_2023, Kariholu_2008, Kerestes_2019, Kobiela_2015, Kumar_2019f, Mesfin_2022a, Misra_2013, Ohno_2005, Qureshi_2016, Sakellaridis_2008f, Sultan_2024f, Thapa_2019f, Trgo_2012f, Yasin_2009, Yildiz_2016e, teWildt_2010}, 9 cases (12\%) ingested a magnet \cite{Ali_2020f, Bhumi_2024f, Cauchi_2002, Liu_2005, Naji_2012f, Ohno_2005, Tanrikulu_2015e, Tay_2004, Wildhaber_2005}, 2 cases (3\%) ingested a button battery \cite{Berry_2021e, Bhumi_2024f}. \paragraph*{Outcomes} 48 cases (67\%) experienced a complication \cite{Ali_2017, Ali_2020f, Apikotoa_2022f, Atayan_2016, Beecroft_1998, Benoist_2019e, Berry_2021e, Bhasin_2014, Bhumi_2024f, CamachoDorado_2018, Cauchi_2002, Cox_2007, Csaky_1998e, DelgadoSalazar_2020c, DivsalarP._2023a, Emamhadi_2018, Farhadi_2024h, Fry_2010, Gardner_2017h, Goldman_1998f, Jin_2023, Kariholu_2008, Kerestes_2019, Kobiela_2015, Kumar_2001, Kumar_2019f, Liu_2005, Losanoff_1996, Mesfin_2022a, Misra_2013, Naji_2012f, Ohno_2005, Sakellaridis_2008f, Sobnach_2011f, Sultan_2024f, Tanrikulu_2015e, Tay_2004, Thapa_2019f, Trgo_2012f, Tupesis_2004f, Wildhaber_2005, Wnęk_2015f, Yasin_2009, Yildiz_2016e}, 44 cases (61\%) underwent surgery \cite{Al-Faham_2020k, AlShaaibi_2021b, Alao_2006i, Ali_2017, Ali_2020f, Atayan_2016, Beecroft_1998, Bhasin_2014, CamachoDorado_2018, Cauchi_2002, Chang_2017f, Cox_2007, Csaky_1998e, DelgadoSalazar_2020c, DivsalarP._2023a, Farhadi_2024h, Fry_2010, Gardner_2017h, Jin_2023, Kariholu_2008, Kerestes_2019, Kobiela_2015, Kumar_2019f, Liu_2005, Losanoff_1996, Losanoff_1997e, Mesfin_2022a, Misra_2013, Naji_2012f, Sobnach_2011f, Tanrikulu_2015e, Tay_2004, Thapa_2019f, Tupesis_2004f, Wildhaber_2005, Wnęk_2015f, Yasin_2009, Yildiz_2016e, fjbuilsRepeatedBehaviorDeliberate2024}, 31 cases (43\%) underwent endoscopy \cite{Akay_2015f, Ali_2022g, Apikotoa_2022f, Atayan_2016, Benoist_2019e, Berry_2021e, Bhasin_2014, Bhumi_2024f, CamachoDorado_2018, Chang_2017f, DelgadoSalazar_2020c, Gardner_2017h, Guinan_2019f, Hardy_2023g, Jehangir_2019h, Kariholu_2008, Li_2013, Liu_2005, Ohno_2005, Peixoto_2017f, Qureshi_2016, Riva_2018j, Sakellaridis_2008f, Sultan_2024f, Tammana_2012j, Tanrikulu_2015e, Trgo_2012f, Wadhwa_2015e, Wnęk_2015f, teWildt_2010}, 7 cases (10\%) were managed conservatively \cite{Ataya_2013, Bhattacharjee_2008, DivsalarP._2023a, Emamhadi_2018, Goldman_1998f, Kar_2015, Kumar_2001}, 2 cases (3\%) died \cite{Emamhadi_2018, Kumar_2001}. All 90 were male gender. 90 cases (100\%) were detained at the time of ingestion \cite{Elghali_2016, Karp_1991b, Lee_2007}, 88 cases (98\%) were intentional ingestions \cite{Elghali_2016, Karp_1991b, Lee_2007}, 30 cases (33\%) had a psychiatric history documented \cite{Elghali_2016, Karp_1991b, Lee_2007}, 2 cases (2\%) had a history of prior ingestion \cite{Elghali_2016}. No cases were reported for were psychiatric inpatients, were displaced people, were under the influence of alcohol at the time of ingestion, and had a severe disability history.
\paragraph*{Motivation}  70 cases (78\%) reported protest motivation \cite{Elghali_2016, Karp_1991b, Lee_2007}, 12 cases (13\%) reported psychiatric motivation \cite{Karp_1991b}, 6 cases (7\%) reported self-harm motivation \cite{Elghali_2016, Karp_1991b}. No cases were reported for psychosocial motivation and other motivation.
\paragraph*{Object Characteristics}  68 cases (76\%) involved sharp object ingestion \cite{Elghali_2016, Karp_1991b, Lee_2007}, 32 cases (36\%) involved long (\textgreater 5cm) object ingestion \cite{Lee_2007}, 25 cases (28\%) involved ingestion of multiple objects \cite{Elghali_2016, Lee_2007}. No cases were reported for button battery ingestion, magnet ingestion, and involved large diameter (\textgreater 2.5cm) object ingestion.
\paragraph*{Outcomes}  47 cases (52\%) underwent endoscopic intervention \cite{Elghali_2016, Lee_2007}, 29 cases (32\%) were managed conservatively \cite{Elghali_2016, Karp_1991b}, 15 cases (17\%) underwent surgical intervention \cite{Elghali_2016, Karp_1991b, Lee_2007}, 6 cases (7\%) reported complications \cite{Lee_2007}, 1 case (1\%) died \cite{Elghali_2016}.
\paragraph*{Geographical Location}Cases were recorded in 33 countries: 13 cases from USA \cite{Alao_2006i, Ataya_2013, Bhumi_2024f, Fry_2010, Guinan_2019f, Hardy_2023g, Jehangir_2019h, Kerestes_2019, Kumar_2001, Liu_2005, Tammana_2012j, Tay_2004, Tupesis_2004f}; 7 cases from India \cite{Bhasin_2014, Bhattacharjee_2008, Kar_2015, Kariholu_2008, Kumar_2019f, Misra_2013, Wadhwa_2015e} and UK \cite{Beecroft_1998, Berry_2021e, Cauchi_2002, Cox_2007, Gardner_2017h, Qureshi_2016}; 6 cases from Bulgaria \cite{Losanoff_1996, Losanoff_1997e}; 5 cases from Iran \cite{DivsalarP._2023a, Emamhadi_2018, Farhadi_2024h}; 4 cases from Turkey \cite{Akay_2015f, Atayan_2016, Tanrikulu_2015e, Yildiz_2016e}; 2 cases from China \cite{Jin_2023, Li_2013}, Poland \cite{Kobiela_2015, Wnęk_2015f}, and Spain \cite{CamachoDorado_2018, fjbuilsRepeatedBehaviorDeliberate2024}; 1 case from Australia \cite{Apikotoa_2022f}, Bahrain \cite{Ali_2020f}, Croatia \cite{Trgo_2012f}, Ecuador \cite{DelgadoSalazar_2020c}, Egypt \cite{Ali_2022g}, Ethiopia \cite{Mesfin_2022a}, Germany \cite{teWildt_2010}, Greece \cite{Sakellaridis_2008f}, Hungary \cite{Csaky_1998e}, Iraq \cite{Al-Faham_2020k}, Israel \cite{Goldman_1998f}, Italy \cite{Riva_2018j}, Japan \cite{Ohno_2005}, Nepal \cite{Thapa_2019f}, Netherlands \cite{Benoist_2019e}, Oman \cite{AlShaaibi_2021b}, Pakistan \cite{Yasin_2009}, Portugal \cite{Peixoto_2017f}, Qatar \cite{Ali_2017}, Saudi Arabia \cite{Sultan_2024f}, South Africa \cite{Sobnach_2011f}, Sweden \cite{Naji_2012f}, Switzerland \cite{Wildhaber_2005}, and Taiwan \cite{Chang_2017f}. \paragraph*{Gender} 43 cases (60\%) were male \cite{Akay_2015f, Al-Faham_2020k, Alao_2006i, Ali_2017, Ali_2022g, Apikotoa_2022f, Atayan_2016, Benoist_2019e, Berry_2021e, Bhumi_2024f, CamachoDorado_2018, Csaky_1998e, Emamhadi_2018, Farhadi_2024h, Fry_2010, Gardner_2017h, Guinan_2019f, Jehangir_2019h, Jin_2023, Kobiela_2015, Kumar_2001, Kumar_2019f, Liu_2005, Losanoff_1996, Losanoff_1997e, Mesfin_2022a, Misra_2013, Qureshi_2016, Riva_2018j, Sobnach_2011f, Tammana_2012j, Tanrikulu_2015e, Tay_2004, Thapa_2019f, Trgo_2012f, Wadhwa_2015e, Yasin_2009, teWildt_2010}, 28 cases (39\%) were female \cite{AlShaaibi_2021b, Ali_2020f, Ataya_2013, Beecroft_1998, Bhasin_2014, Bhattacharjee_2008, Cauchi_2002, Chang_2017f, Cox_2007, DelgadoSalazar_2020c, DivsalarP._2023a, Goldman_1998f, Hardy_2023g, Kar_2015, Kariholu_2008, Kerestes_2019, Li_2013, Naji_2012f, Ohno_2005, Peixoto_2017f, Sakellaridis_2008f, Sultan_2024f, Tupesis_2004f, Wildhaber_2005, Wnęk_2015f, Yildiz_2016e}, 1 case (1\%) had no gender recorded \cite{fjbuilsRepeatedBehaviorDeliberate2024}. \paragraph*{Age Group} 25 cases (35\%) were between 26 and 40 years of age \cite{Alao_2006i, Ali_2022g, Apikotoa_2022f, Ataya_2013, Benoist_2019e, Bhasin_2014, Chang_2017f, Cox_2007, DelgadoSalazar_2020c, Farhadi_2024h, Fry_2010, Gardner_2017h, Guinan_2019f, Jin_2023, Kumar_2019f, Losanoff_1996, Misra_2013, Qureshi_2016, Riva_2018j, Sakellaridis_2008f, Tammana_2012j, Trgo_2012f, Wnęk_2015f, Yildiz_2016e, fjbuilsRepeatedBehaviorDeliberate2024}, 18 cases (25\%) were between 18 and 25 years of age \cite{Akay_2015f, Ali_2017, Atayan_2016, Bhattacharjee_2008, Csaky_1998e, Kar_2015, Kariholu_2008, Kobiela_2015, Losanoff_1996, Losanoff_1997e, Mesfin_2022a, Peixoto_2017f, Sobnach_2011f, Tupesis_2004f, Yasin_2009}, 13 cases (18\%) were under 18 years of age \cite{AlShaaibi_2021b, Ali_2020f, Cauchi_2002, DivsalarP._2023a, Goldman_1998f, Liu_2005, Naji_2012f, Ohno_2005, Tanrikulu_2015e, Tay_2004, Wildhaber_2005}, 11 cases (15\%) were between 41 and 60 years of age \cite{Al-Faham_2020k, Bhumi_2024f, CamachoDorado_2018, Emamhadi_2018, Hardy_2023g, Jehangir_2019h, Kumar_2001, Sultan_2024f, Thapa_2019f, Wadhwa_2015e, teWildt_2010}, 3 cases (4\%) were over 60 years of age \cite{Beecroft_1998, Kerestes_2019, Li_2013}, 2 cases (3\%) had no age documented \cite{Berry_2021e}. \paragraph*{Population} 36 cases (50\%) had a psychiatric history \cite{AlShaaibi_2021b, Alao_2006i, Ali_2020f, Apikotoa_2022f, Ataya_2013, Atayan_2016, Beecroft_1998, CamachoDorado_2018, Chang_2017f, DelgadoSalazar_2020c, DivsalarP._2023a, Farhadi_2024h, Fry_2010, Guinan_2019f, Hardy_2023g, Jehangir_2019h, Jin_2023, Kar_2015, Kerestes_2019, Kobiela_2015, Kumar_2001, Kumar_2019f, Liu_2005, Mesfin_2022a, Misra_2013, Ohno_2005, Peixoto_2017f, Sakellaridis_2008f, Sultan_2024f, Tammana_2012j, Tanrikulu_2015e, Yildiz_2016e, fjbuilsRepeatedBehaviorDeliberate2024, teWildt_2010}, 19 cases (26\%) had ingested previously \cite{Alao_2006i, Apikotoa_2022f, Berry_2021e, Bhattacharjee_2008, Csaky_1998e, DivsalarP._2023a, Emamhadi_2018, Guinan_2019f, Jehangir_2019h, Jin_2023, Liu_2005, Sakellaridis_2008f, Tanrikulu_2015e, Thapa_2019f, Yildiz_2016e, fjbuilsRepeatedBehaviorDeliberate2024, teWildt_2010}, 12 cases (17\%) were detained persons \cite{Alao_2006i, Ali_2022g, Apikotoa_2022f, Losanoff_1996, Losanoff_1997e, Qureshi_2016, Tammana_2012j, Trgo_2012f}, 7 cases (10\%) were severely disabled \cite{Atayan_2016, Kerestes_2019, Liu_2005, Ohno_2005, Peixoto_2017f, Yildiz_2016e, teWildt_2010}, 4 cases (6\%) were psychiatric inpatients \cite{DivsalarP._2023a, fjbuilsRepeatedBehaviorDeliberate2024, teWildt_2010}, 3 cases (4\%) were under the influence of alcohol \cite{Benoist_2019e, Csaky_1998e, Thapa_2019f}, 2 cases (3\%) were displaced people \cite{Akay_2015f, Gardner_2017h}. \paragraph*{Motivation} 34 cases (47\%) had a psychiatric motivation \cite{Al-Faham_2020k, Alao_2006i, Ali_2020f, Apikotoa_2022f, Ataya_2013, Atayan_2016, Bhasin_2014, Bhattacharjee_2008, DelgadoSalazar_2020c, DivsalarP._2023a, Emamhadi_2018, Farhadi_2024h, Guinan_2019f, Hardy_2023g, Jehangir_2019h, Jin_2023, Kar_2015, Kariholu_2008, Kerestes_2019, Kobiela_2015, Kumar_2001, Kumar_2019f, Li_2013, Liu_2005, Misra_2013, Ohno_2005, Sakellaridis_2008f, Sultan_2024f, Tammana_2012j, Tanrikulu_2015e, Yasin_2009, teWildt_2010}, 21 cases (29\%) were motivated by self-harm intention \cite{Al-Faham_2020k, AlShaaibi_2021b, Alao_2006i, Ali_2017, CamachoDorado_2018, Chang_2017f, Cox_2007, Csaky_1998e, Fry_2010, Li_2013, Losanoff_1996, Losanoff_1997e, Mesfin_2022a, Sakellaridis_2008f, Tammana_2012j, Tanrikulu_2015e, fjbuilsRepeatedBehaviorDeliberate2024}, 17 cases (24\%) had a psychosocial motivation \cite{Akay_2015f, Benoist_2019e, Bhattacharjee_2008, Cauchi_2002, Goldman_1998f, Hardy_2023g, Kobiela_2015, Li_2013, Naji_2012f, Qureshi_2016, Riva_2018j, Sobnach_2011f, Tay_2004, Thapa_2019f, Tupesis_2004f, Wildhaber_2005, Wnęk_2015f}, 9 cases (12\%) were motivated by protest \cite{Bhumi_2024f, Gardner_2017h, Losanoff_1996, Losanoff_1997e, Tupesis_2004f}, 9 cases (12\%) had another documented motivation \cite{Ali_2020f, Ali_2022g, Emamhadi_2018, Guinan_2019f, Peixoto_2017f, Sakellaridis_2008f, Trgo_2012f, Wadhwa_2015e, Yildiz_2016e}. \paragraph*{Object Characteristics} 51 cases (71\%) ingested a large diameter object (\textgreater{}2.5cm) \cite{Akay_2015f, Al-Faham_2020k, AlShaaibi_2021b, Alao_2006i, Ali_2017, Ali_2022g, Apikotoa_2022f, Atayan_2016, Berry_2021e, Bhasin_2014, CamachoDorado_2018, Cauchi_2002, Chang_2017f, Cox_2007, Csaky_1998e, DivsalarP._2023a, Emamhadi_2018, Gardner_2017h, Guinan_2019f, Jehangir_2019h, Jin_2023, Kariholu_2008, Kerestes_2019, Kobiela_2015, Kumar_2001, Kumar_2019f, Losanoff_1996, Losanoff_1997e, Mesfin_2022a, Misra_2013, Naji_2012f, Ohno_2005, Peixoto_2017f, Qureshi_2016, Riva_2018j, Sakellaridis_2008f, Sultan_2024f, Tanrikulu_2015e, Thapa_2019f, Trgo_2012f, Wnęk_2015f, Yildiz_2016e, fjbuilsRepeatedBehaviorDeliberate2024, teWildt_2010}, 44 cases (61\%) ingested multiple objects \cite{Ali_2020f, Apikotoa_2022f, Ataya_2013, Atayan_2016, Beecroft_1998, Bhattacharjee_2008, Bhumi_2024f, CamachoDorado_2018, Cauchi_2002, Emamhadi_2018, Farhadi_2024h, Fry_2010, Goldman_1998f, Guinan_2019f, Hardy_2023g, Jehangir_2019h, Jin_2023, Kar_2015, Kariholu_2008, Kobiela_2015, Kumar_2001, Kumar_2019f, Li_2013, Liu_2005, Losanoff_1996, Mesfin_2022a, Misra_2013, Naji_2012f, Ohno_2005, Sobnach_2011f, Sultan_2024f, Tammana_2012j, Tanrikulu_2015e, Tay_2004, Thapa_2019f, Wadhwa_2015e, Wildhaber_2005, Yasin_2009, fjbuilsRepeatedBehaviorDeliberate2024, teWildt_2010}, 34 cases (47\%) ingested a sharp object \cite{AlShaaibi_2021b, Alao_2006i, Apikotoa_2022f, Ataya_2013, Benoist_2019e, Bhasin_2014, Bhattacharjee_2008, CamachoDorado_2018, Csaky_1998e, DelgadoSalazar_2020c, DivsalarP._2023a, Emamhadi_2018, Farhadi_2024h, Fry_2010, Guinan_2019f, Hardy_2023g, Jehangir_2019h, Jin_2023, Kariholu_2008, Kobiela_2015, Kumar_2019f, Losanoff_1996, Losanoff_1997e, Mesfin_2022a, Misra_2013, Sobnach_2011f, Yasin_2009, teWildt_2010}, 32 cases (44\%) ingested a long object (\textgreater{}5cm) \cite{Al-Faham_2020k, AlShaaibi_2021b, Ali_2017, Ali_2022g, Atayan_2016, Bhasin_2014, CamachoDorado_2018, Chang_2017f, Cox_2007, Csaky_1998e, DivsalarP._2023a, Emamhadi_2018, Fry_2010, Gardner_2017h, Jin_2023, Kariholu_2008, Kerestes_2019, Kobiela_2015, Kumar_2019f, Mesfin_2022a, Misra_2013, Ohno_2005, Qureshi_2016, Sakellaridis_2008f, Sultan_2024f, Thapa_2019f, Trgo_2012f, Yasin_2009, Yildiz_2016e, teWildt_2010}, 9 cases (12\%) ingested a magnet \cite{Ali_2020f, Bhumi_2024f, Cauchi_2002, Liu_2005, Naji_2012f, Ohno_2005, Tanrikulu_2015e, Tay_2004, Wildhaber_2005}, 2 cases (3\%) ingested a button battery \cite{Berry_2021e, Bhumi_2024f}. \paragraph*{Outcomes} 48 cases (67\%) experienced a complication \cite{Ali_2017, Ali_2020f, Apikotoa_2022f, Atayan_2016, Beecroft_1998, Benoist_2019e, Berry_2021e, Bhasin_2014, Bhumi_2024f, CamachoDorado_2018, Cauchi_2002, Cox_2007, Csaky_1998e, DelgadoSalazar_2020c, DivsalarP._2023a, Emamhadi_2018, Farhadi_2024h, Fry_2010, Gardner_2017h, Goldman_1998f, Jin_2023, Kariholu_2008, Kerestes_2019, Kobiela_2015, Kumar_2001, Kumar_2019f, Liu_2005, Losanoff_1996, Mesfin_2022a, Misra_2013, Naji_2012f, Ohno_2005, Sakellaridis_2008f, Sobnach_2011f, Sultan_2024f, Tanrikulu_2015e, Tay_2004, Thapa_2019f, Trgo_2012f, Tupesis_2004f, Wildhaber_2005, Wnęk_2015f, Yasin_2009, Yildiz_2016e}, 44 cases (61\%) underwent surgery \cite{Al-Faham_2020k, AlShaaibi_2021b, Alao_2006i, Ali_2017, Ali_2020f, Atayan_2016, Beecroft_1998, Bhasin_2014, CamachoDorado_2018, Cauchi_2002, Chang_2017f, Cox_2007, Csaky_1998e, DelgadoSalazar_2020c, DivsalarP._2023a, Farhadi_2024h, Fry_2010, Gardner_2017h, Jin_2023, Kariholu_2008, Kerestes_2019, Kobiela_2015, Kumar_2019f, Liu_2005, Losanoff_1996, Losanoff_1997e, Mesfin_2022a, Misra_2013, Naji_2012f, Sobnach_2011f, Tanrikulu_2015e, Tay_2004, Thapa_2019f, Tupesis_2004f, Wildhaber_2005, Wnęk_2015f, Yasin_2009, Yildiz_2016e, fjbuilsRepeatedBehaviorDeliberate2024}, 31 cases (43\%) underwent endoscopy \cite{Akay_2015f, Ali_2022g, Apikotoa_2022f, Atayan_2016, Benoist_2019e, Berry_2021e, Bhasin_2014, Bhumi_2024f, CamachoDorado_2018, Chang_2017f, DelgadoSalazar_2020c, Gardner_2017h, Guinan_2019f, Hardy_2023g, Jehangir_2019h, Kariholu_2008, Li_2013, Liu_2005, Ohno_2005, Peixoto_2017f, Qureshi_2016, Riva_2018j, Sakellaridis_2008f, Sultan_2024f, Tammana_2012j, Tanrikulu_2015e, Trgo_2012f, Wadhwa_2015e, Wnęk_2015f, teWildt_2010}, 7 cases (10\%) were managed conservatively \cite{Ataya_2013, Bhattacharjee_2008, DivsalarP._2023a, Emamhadi_2018, Goldman_1998f, Kar_2015, Kumar_2001}, 2 cases (3\%) died \cite{Emamhadi_2018, Kumar_2001}. All 90 were male gender. 90 cases (100\%) were detained at the time of ingestion \cite{Elghali_2016, Karp_1991b, Lee_2007}, 88 cases (98\%) were intentional ingestions \cite{Elghali_2016, Karp_1991b, Lee_2007}, 30 cases (33\%) had a psychiatric history documented \cite{Elghali_2016, Karp_1991b, Lee_2007}, 2 cases (2\%) had a history of prior ingestion \cite{Elghali_2016}. No cases were reported for were psychiatric inpatients, were displaced people, were under the influence of alcohol at the time of ingestion, and had a severe disability history.
\paragraph*{Motivation}  70 cases (78\%) reported protest motivation \cite{Elghali_2016, Karp_1991b, Lee_2007}, 12 cases (13\%) reported psychiatric motivation \cite{Karp_1991b}, 6 cases (7\%) reported self-harm motivation \cite{Elghali_2016, Karp_1991b}. No cases were reported for psychosocial motivation and other motivation.
\paragraph*{Object Characteristics}  68 cases (76\%) involved sharp object ingestion \cite{Elghali_2016, Karp_1991b, Lee_2007}, 32 cases (36\%) involved long (\textgreater 5cm) object ingestion \cite{Lee_2007}, 25 cases (28\%) involved ingestion of multiple objects \cite{Elghali_2016, Lee_2007}. No cases were reported for button battery ingestion, magnet ingestion, and involved large diameter (\textgreater 2.5cm) object ingestion.
\paragraph*{Outcomes}  47 cases (52\%) underwent endoscopic intervention \cite{Elghali_2016, Lee_2007}, 29 cases (32\%) were managed conservatively \cite{Elghali_2016, Karp_1991b}, 15 cases (17\%) underwent surgical intervention \cite{Elghali_2016, Karp_1991b, Lee_2007}, 6 cases (7\%) reported complications \cite{Lee_2007}, 1 case (1\%) died \cite{Elghali_2016}.
\paragraph*{Geographical Location}Cases were recorded in 33 countries: 13 cases from USA \cite{Alao_2006i, Ataya_2013, Bhumi_2024f, Fry_2010, Guinan_2019f, Hardy_2023g, Jehangir_2019h, Kerestes_2019, Kumar_2001, Liu_2005, Tammana_2012j, Tay_2004, Tupesis_2004f}; 7 cases from India \cite{Bhasin_2014, Bhattacharjee_2008, Kar_2015, Kariholu_2008, Kumar_2019f, Misra_2013, Wadhwa_2015e} and UK \cite{Beecroft_1998, Berry_2021e, Cauchi_2002, Cox_2007, Gardner_2017h, Qureshi_2016}; 6 cases from Bulgaria \cite{Losanoff_1996, Losanoff_1997e}; 5 cases from Iran \cite{DivsalarP._2023a, Emamhadi_2018, Farhadi_2024h}; 4 cases from Turkey \cite{Akay_2015f, Atayan_2016, Tanrikulu_2015e, Yildiz_2016e}; 2 cases from China \cite{Jin_2023, Li_2013}, Poland \cite{Kobiela_2015, Wnęk_2015f}, and Spain \cite{CamachoDorado_2018, fjbuilsRepeatedBehaviorDeliberate2024}; 1 case from Australia \cite{Apikotoa_2022f}, Bahrain \cite{Ali_2020f}, Croatia \cite{Trgo_2012f}, Ecuador \cite{DelgadoSalazar_2020c}, Egypt \cite{Ali_2022g}, Ethiopia \cite{Mesfin_2022a}, Germany \cite{teWildt_2010}, Greece \cite{Sakellaridis_2008f}, Hungary \cite{Csaky_1998e}, Iraq \cite{Al-Faham_2020k}, Israel \cite{Goldman_1998f}, Italy \cite{Riva_2018j}, Japan \cite{Ohno_2005}, Nepal \cite{Thapa_2019f}, Netherlands \cite{Benoist_2019e}, Oman \cite{AlShaaibi_2021b}, Pakistan \cite{Yasin_2009}, Portugal \cite{Peixoto_2017f}, Qatar \cite{Ali_2017}, Saudi Arabia \cite{Sultan_2024f}, South Africa \cite{Sobnach_2011f}, Sweden \cite{Naji_2012f}, Switzerland \cite{Wildhaber_2005}, and Taiwan \cite{Chang_2017f}. \paragraph*{Gender} 43 cases (60\%) were male \cite{Akay_2015f, Al-Faham_2020k, Alao_2006i, Ali_2017, Ali_2022g, Apikotoa_2022f, Atayan_2016, Benoist_2019e, Berry_2021e, Bhumi_2024f, CamachoDorado_2018, Csaky_1998e, Emamhadi_2018, Farhadi_2024h, Fry_2010, Gardner_2017h, Guinan_2019f, Jehangir_2019h, Jin_2023, Kobiela_2015, Kumar_2001, Kumar_2019f, Liu_2005, Losanoff_1996, Losanoff_1997e, Mesfin_2022a, Misra_2013, Qureshi_2016, Riva_2018j, Sobnach_2011f, Tammana_2012j, Tanrikulu_2015e, Tay_2004, Thapa_2019f, Trgo_2012f, Wadhwa_2015e, Yasin_2009, teWildt_2010}, 28 cases (39\%) were female \cite{AlShaaibi_2021b, Ali_2020f, Ataya_2013, Beecroft_1998, Bhasin_2014, Bhattacharjee_2008, Cauchi_2002, Chang_2017f, Cox_2007, DelgadoSalazar_2020c, DivsalarP._2023a, Goldman_1998f, Hardy_2023g, Kar_2015, Kariholu_2008, Kerestes_2019, Li_2013, Naji_2012f, Ohno_2005, Peixoto_2017f, Sakellaridis_2008f, Sultan_2024f, Tupesis_2004f, Wildhaber_2005, Wnęk_2015f, Yildiz_2016e}, 1 case (1\%) had no gender recorded \cite{fjbuilsRepeatedBehaviorDeliberate2024}. \paragraph*{Age Group} 25 cases (35\%) were between 26 and 40 years of age \cite{Alao_2006i, Ali_2022g, Apikotoa_2022f, Ataya_2013, Benoist_2019e, Bhasin_2014, Chang_2017f, Cox_2007, DelgadoSalazar_2020c, Farhadi_2024h, Fry_2010, Gardner_2017h, Guinan_2019f, Jin_2023, Kumar_2019f, Losanoff_1996, Misra_2013, Qureshi_2016, Riva_2018j, Sakellaridis_2008f, Tammana_2012j, Trgo_2012f, Wnęk_2015f, Yildiz_2016e, fjbuilsRepeatedBehaviorDeliberate2024}, 18 cases (25\%) were between 18 and 25 years of age \cite{Akay_2015f, Ali_2017, Atayan_2016, Bhattacharjee_2008, Csaky_1998e, Kar_2015, Kariholu_2008, Kobiela_2015, Losanoff_1996, Losanoff_1997e, Mesfin_2022a, Peixoto_2017f, Sobnach_2011f, Tupesis_2004f, Yasin_2009}, 13 cases (18\%) were under 18 years of age \cite{AlShaaibi_2021b, Ali_2020f, Cauchi_2002, DivsalarP._2023a, Goldman_1998f, Liu_2005, Naji_2012f, Ohno_2005, Tanrikulu_2015e, Tay_2004, Wildhaber_2005}, 11 cases (15\%) were between 41 and 60 years of age \cite{Al-Faham_2020k, Bhumi_2024f, CamachoDorado_2018, Emamhadi_2018, Hardy_2023g, Jehangir_2019h, Kumar_2001, Sultan_2024f, Thapa_2019f, Wadhwa_2015e, teWildt_2010}, 3 cases (4\%) were over 60 years of age \cite{Beecroft_1998, Kerestes_2019, Li_2013}, 2 cases (3\%) had no age documented \cite{Berry_2021e}. \paragraph*{Population} 36 cases (50\%) had a psychiatric history \cite{AlShaaibi_2021b, Alao_2006i, Ali_2020f, Apikotoa_2022f, Ataya_2013, Atayan_2016, Beecroft_1998, CamachoDorado_2018, Chang_2017f, DelgadoSalazar_2020c, DivsalarP._2023a, Farhadi_2024h, Fry_2010, Guinan_2019f, Hardy_2023g, Jehangir_2019h, Jin_2023, Kar_2015, Kerestes_2019, Kobiela_2015, Kumar_2001, Kumar_2019f, Liu_2005, Mesfin_2022a, Misra_2013, Ohno_2005, Peixoto_2017f, Sakellaridis_2008f, Sultan_2024f, Tammana_2012j, Tanrikulu_2015e, Yildiz_2016e, fjbuilsRepeatedBehaviorDeliberate2024, teWildt_2010}, 19 cases (26\%) had ingested previously \cite{Alao_2006i, Apikotoa_2022f, Berry_2021e, Bhattacharjee_2008, Csaky_1998e, DivsalarP._2023a, Emamhadi_2018, Guinan_2019f, Jehangir_2019h, Jin_2023, Liu_2005, Sakellaridis_2008f, Tanrikulu_2015e, Thapa_2019f, Yildiz_2016e, fjbuilsRepeatedBehaviorDeliberate2024, teWildt_2010}, 12 cases (17\%) were detained persons \cite{Alao_2006i, Ali_2022g, Apikotoa_2022f, Losanoff_1996, Losanoff_1997e, Qureshi_2016, Tammana_2012j, Trgo_2012f}, 7 cases (10\%) were severely disabled \cite{Atayan_2016, Kerestes_2019, Liu_2005, Ohno_2005, Peixoto_2017f, Yildiz_2016e, teWildt_2010}, 4 cases (6\%) were psychiatric inpatients \cite{DivsalarP._2023a, fjbuilsRepeatedBehaviorDeliberate2024, teWildt_2010}, 3 cases (4\%) were under the influence of alcohol \cite{Benoist_2019e, Csaky_1998e, Thapa_2019f}, 2 cases (3\%) were displaced people \cite{Akay_2015f, Gardner_2017h}. \paragraph*{Motivation} 34 cases (47\%) had a psychiatric motivation \cite{Al-Faham_2020k, Alao_2006i, Ali_2020f, Apikotoa_2022f, Ataya_2013, Atayan_2016, Bhasin_2014, Bhattacharjee_2008, DelgadoSalazar_2020c, DivsalarP._2023a, Emamhadi_2018, Farhadi_2024h, Guinan_2019f, Hardy_2023g, Jehangir_2019h, Jin_2023, Kar_2015, Kariholu_2008, Kerestes_2019, Kobiela_2015, Kumar_2001, Kumar_2019f, Li_2013, Liu_2005, Misra_2013, Ohno_2005, Sakellaridis_2008f, Sultan_2024f, Tammana_2012j, Tanrikulu_2015e, Yasin_2009, teWildt_2010}, 21 cases (29\%) were motivated by self-harm intention \cite{Al-Faham_2020k, AlShaaibi_2021b, Alao_2006i, Ali_2017, CamachoDorado_2018, Chang_2017f, Cox_2007, Csaky_1998e, Fry_2010, Li_2013, Losanoff_1996, Losanoff_1997e, Mesfin_2022a, Sakellaridis_2008f, Tammana_2012j, Tanrikulu_2015e, fjbuilsRepeatedBehaviorDeliberate2024}, 17 cases (24\%) had a psychosocial motivation \cite{Akay_2015f, Benoist_2019e, Bhattacharjee_2008, Cauchi_2002, Goldman_1998f, Hardy_2023g, Kobiela_2015, Li_2013, Naji_2012f, Qureshi_2016, Riva_2018j, Sobnach_2011f, Tay_2004, Thapa_2019f, Tupesis_2004f, Wildhaber_2005, Wnęk_2015f}, 9 cases (12\%) were motivated by protest \cite{Bhumi_2024f, Gardner_2017h, Losanoff_1996, Losanoff_1997e, Tupesis_2004f}, 9 cases (12\%) had another documented motivation \cite{Ali_2020f, Ali_2022g, Emamhadi_2018, Guinan_2019f, Peixoto_2017f, Sakellaridis_2008f, Trgo_2012f, Wadhwa_2015e, Yildiz_2016e}. \paragraph*{Object Characteristics} 51 cases (71\%) ingested a large diameter object (\textgreater{}2.5cm) \cite{Akay_2015f, Al-Faham_2020k, AlShaaibi_2021b, Alao_2006i, Ali_2017, Ali_2022g, Apikotoa_2022f, Atayan_2016, Berry_2021e, Bhasin_2014, CamachoDorado_2018, Cauchi_2002, Chang_2017f, Cox_2007, Csaky_1998e, DivsalarP._2023a, Emamhadi_2018, Gardner_2017h, Guinan_2019f, Jehangir_2019h, Jin_2023, Kariholu_2008, Kerestes_2019, Kobiela_2015, Kumar_2001, Kumar_2019f, Losanoff_1996, Losanoff_1997e, Mesfin_2022a, Misra_2013, Naji_2012f, Ohno_2005, Peixoto_2017f, Qureshi_2016, Riva_2018j, Sakellaridis_2008f, Sultan_2024f, Tanrikulu_2015e, Thapa_2019f, Trgo_2012f, Wnęk_2015f, Yildiz_2016e, fjbuilsRepeatedBehaviorDeliberate2024, teWildt_2010}, 44 cases (61\%) ingested multiple objects \cite{Ali_2020f, Apikotoa_2022f, Ataya_2013, Atayan_2016, Beecroft_1998, Bhattacharjee_2008, Bhumi_2024f, CamachoDorado_2018, Cauchi_2002, Emamhadi_2018, Farhadi_2024h, Fry_2010, Goldman_1998f, Guinan_2019f, Hardy_2023g, Jehangir_2019h, Jin_2023, Kar_2015, Kariholu_2008, Kobiela_2015, Kumar_2001, Kumar_2019f, Li_2013, Liu_2005, Losanoff_1996, Mesfin_2022a, Misra_2013, Naji_2012f, Ohno_2005, Sobnach_2011f, Sultan_2024f, Tammana_2012j, Tanrikulu_2015e, Tay_2004, Thapa_2019f, Wadhwa_2015e, Wildhaber_2005, Yasin_2009, fjbuilsRepeatedBehaviorDeliberate2024, teWildt_2010}, 34 cases (47\%) ingested a sharp object \cite{AlShaaibi_2021b, Alao_2006i, Apikotoa_2022f, Ataya_2013, Benoist_2019e, Bhasin_2014, Bhattacharjee_2008, CamachoDorado_2018, Csaky_1998e, DelgadoSalazar_2020c, DivsalarP._2023a, Emamhadi_2018, Farhadi_2024h, Fry_2010, Guinan_2019f, Hardy_2023g, Jehangir_2019h, Jin_2023, Kariholu_2008, Kobiela_2015, Kumar_2019f, Losanoff_1996, Losanoff_1997e, Mesfin_2022a, Misra_2013, Sobnach_2011f, Yasin_2009, teWildt_2010}, 32 cases (44\%) ingested a long object (\textgreater{}5cm) \cite{Al-Faham_2020k, AlShaaibi_2021b, Ali_2017, Ali_2022g, Atayan_2016, Bhasin_2014, CamachoDorado_2018, Chang_2017f, Cox_2007, Csaky_1998e, DivsalarP._2023a, Emamhadi_2018, Fry_2010, Gardner_2017h, Jin_2023, Kariholu_2008, Kerestes_2019, Kobiela_2015, Kumar_2019f, Mesfin_2022a, Misra_2013, Ohno_2005, Qureshi_2016, Sakellaridis_2008f, Sultan_2024f, Thapa_2019f, Trgo_2012f, Yasin_2009, Yildiz_2016e, teWildt_2010}, 9 cases (12\%) ingested a magnet \cite{Ali_2020f, Bhumi_2024f, Cauchi_2002, Liu_2005, Naji_2012f, Ohno_2005, Tanrikulu_2015e, Tay_2004, Wildhaber_2005}, 2 cases (3\%) ingested a button battery \cite{Berry_2021e, Bhumi_2024f}. \paragraph*{Outcomes} 48 cases (67\%) experienced a complication \cite{Ali_2017, Ali_2020f, Apikotoa_2022f, Atayan_2016, Beecroft_1998, Benoist_2019e, Berry_2021e, Bhasin_2014, Bhumi_2024f, CamachoDorado_2018, Cauchi_2002, Cox_2007, Csaky_1998e, DelgadoSalazar_2020c, DivsalarP._2023a, Emamhadi_2018, Farhadi_2024h, Fry_2010, Gardner_2017h, Goldman_1998f, Jin_2023, Kariholu_2008, Kerestes_2019, Kobiela_2015, Kumar_2001, Kumar_2019f, Liu_2005, Losanoff_1996, Mesfin_2022a, Misra_2013, Naji_2012f, Ohno_2005, Sakellaridis_2008f, Sobnach_2011f, Sultan_2024f, Tanrikulu_2015e, Tay_2004, Thapa_2019f, Trgo_2012f, Tupesis_2004f, Wildhaber_2005, Wnęk_2015f, Yasin_2009, Yildiz_2016e}, 44 cases (61\%) underwent surgery \cite{Al-Faham_2020k, AlShaaibi_2021b, Alao_2006i, Ali_2017, Ali_2020f, Atayan_2016, Beecroft_1998, Bhasin_2014, CamachoDorado_2018, Cauchi_2002, Chang_2017f, Cox_2007, Csaky_1998e, DelgadoSalazar_2020c, DivsalarP._2023a, Farhadi_2024h, Fry_2010, Gardner_2017h, Jin_2023, Kariholu_2008, Kerestes_2019, Kobiela_2015, Kumar_2019f, Liu_2005, Losanoff_1996, Losanoff_1997e, Mesfin_2022a, Misra_2013, Naji_2012f, Sobnach_2011f, Tanrikulu_2015e, Tay_2004, Thapa_2019f, Tupesis_2004f, Wildhaber_2005, Wnęk_2015f, Yasin_2009, Yildiz_2016e, fjbuilsRepeatedBehaviorDeliberate2024}, 31 cases (43\%) underwent endoscopy \cite{Akay_2015f, Ali_2022g, Apikotoa_2022f, Atayan_2016, Benoist_2019e, Berry_2021e, Bhasin_2014, Bhumi_2024f, CamachoDorado_2018, Chang_2017f, DelgadoSalazar_2020c, Gardner_2017h, Guinan_2019f, Hardy_2023g, Jehangir_2019h, Kariholu_2008, Li_2013, Liu_2005, Ohno_2005, Peixoto_2017f, Qureshi_2016, Riva_2018j, Sakellaridis_2008f, Sultan_2024f, Tammana_2012j, Tanrikulu_2015e, Trgo_2012f, Wadhwa_2015e, Wnęk_2015f, teWildt_2010}, 7 cases (10\%) were managed conservatively \cite{Ataya_2013, Bhattacharjee_2008, DivsalarP._2023a, Emamhadi_2018, Goldman_1998f, Kar_2015, Kumar_2001}, 2 cases (3\%) died \cite{Emamhadi_2018, Kumar_2001}. All 90 were male gender. 90 cases (100\%) were detained at the time of ingestion \cite{Elghali_2016, Karp_1991b, Lee_2007}, 88 cases (98\%) were intentional ingestions \cite{Elghali_2016, Karp_1991b, Lee_2007}, 30 cases (33\%) had a psychiatric history documented \cite{Elghali_2016, Karp_1991b, Lee_2007}, 2 cases (2\%) had a history of prior ingestion \cite{Elghali_2016}. No cases were reported for were psychiatric inpatients, were displaced people, were under the influence of alcohol at the time of ingestion, and had a severe disability history.
\paragraph*{Motivation}  70 cases (78\%) reported protest motivation \cite{Elghali_2016, Karp_1991b, Lee_2007}, 12 cases (13\%) reported psychiatric motivation \cite{Karp_1991b}, 6 cases (7\%) reported self-harm motivation \cite{Elghali_2016, Karp_1991b}. No cases were reported for psychosocial motivation and other motivation.
\paragraph*{Object Characteristics}  68 cases (76\%) involved sharp object ingestion \cite{Elghali_2016, Karp_1991b, Lee_2007}, 32 cases (36\%) involved long (\textgreater 5cm) object ingestion \cite{Lee_2007}, 25 cases (28\%) involved ingestion of multiple objects \cite{Elghali_2016, Lee_2007}. No cases were reported for button battery ingestion, magnet ingestion, and involved large diameter (\textgreater 2.5cm) object ingestion.
\paragraph*{Outcomes}  47 cases (52\%) underwent endoscopic intervention \cite{Elghali_2016, Lee_2007}, 29 cases (32\%) were managed conservatively \cite{Elghali_2016, Karp_1991b}, 15 cases (17\%) underwent surgical intervention \cite{Elghali_2016, Karp_1991b, Lee_2007}, 6 cases (7\%) reported complications \cite{Lee_2007}, 1 case (1\%) died \cite{Elghali_2016}.
\paragraph*{Geographical Location}Cases were recorded in 33 countries: 13 cases from USA \cite{Alao_2006i, Ataya_2013, Bhumi_2024f, Fry_2010, Guinan_2019f, Hardy_2023g, Jehangir_2019h, Kerestes_2019, Kumar_2001, Liu_2005, Tammana_2012j, Tay_2004, Tupesis_2004f}; 7 cases from India \cite{Bhasin_2014, Bhattacharjee_2008, Kar_2015, Kariholu_2008, Kumar_2019f, Misra_2013, Wadhwa_2015e} and UK \cite{Beecroft_1998, Berry_2021e, Cauchi_2002, Cox_2007, Gardner_2017h, Qureshi_2016}; 6 cases from Bulgaria \cite{Losanoff_1996, Losanoff_1997e}; 5 cases from Iran \cite{DivsalarP._2023a, Emamhadi_2018, Farhadi_2024h}; 4 cases from Turkey \cite{Akay_2015f, Atayan_2016, Tanrikulu_2015e, Yildiz_2016e}; 2 cases from China \cite{Jin_2023, Li_2013}, Poland \cite{Kobiela_2015, Wnęk_2015f}, and Spain \cite{CamachoDorado_2018, fjbuilsRepeatedBehaviorDeliberate2024}; 1 case from Australia \cite{Apikotoa_2022f}, Bahrain \cite{Ali_2020f}, Croatia \cite{Trgo_2012f}, Ecuador \cite{DelgadoSalazar_2020c}, Egypt \cite{Ali_2022g}, Ethiopia \cite{Mesfin_2022a}, Germany \cite{teWildt_2010}, Greece \cite{Sakellaridis_2008f}, Hungary \cite{Csaky_1998e}, Iraq \cite{Al-Faham_2020k}, Israel \cite{Goldman_1998f}, Italy \cite{Riva_2018j}, Japan \cite{Ohno_2005}, Nepal \cite{Thapa_2019f}, Netherlands \cite{Benoist_2019e}, Oman \cite{AlShaaibi_2021b}, Pakistan \cite{Yasin_2009}, Portugal \cite{Peixoto_2017f}, Qatar \cite{Ali_2017}, Saudi Arabia \cite{Sultan_2024f}, South Africa \cite{Sobnach_2011f}, Sweden \cite{Naji_2012f}, Switzerland \cite{Wildhaber_2005}, and Taiwan \cite{Chang_2017f}. \paragraph*{Gender} 43 cases (60\%) were male \cite{Akay_2015f, Al-Faham_2020k, Alao_2006i, Ali_2017, Ali_2022g, Apikotoa_2022f, Atayan_2016, Benoist_2019e, Berry_2021e, Bhumi_2024f, CamachoDorado_2018, Csaky_1998e, Emamhadi_2018, Farhadi_2024h, Fry_2010, Gardner_2017h, Guinan_2019f, Jehangir_2019h, Jin_2023, Kobiela_2015, Kumar_2001, Kumar_2019f, Liu_2005, Losanoff_1996, Losanoff_1997e, Mesfin_2022a, Misra_2013, Qureshi_2016, Riva_2018j, Sobnach_2011f, Tammana_2012j, Tanrikulu_2015e, Tay_2004, Thapa_2019f, Trgo_2012f, Wadhwa_2015e, Yasin_2009, teWildt_2010}, 28 cases (39\%) were female \cite{AlShaaibi_2021b, Ali_2020f, Ataya_2013, Beecroft_1998, Bhasin_2014, Bhattacharjee_2008, Cauchi_2002, Chang_2017f, Cox_2007, DelgadoSalazar_2020c, DivsalarP._2023a, Goldman_1998f, Hardy_2023g, Kar_2015, Kariholu_2008, Kerestes_2019, Li_2013, Naji_2012f, Ohno_2005, Peixoto_2017f, Sakellaridis_2008f, Sultan_2024f, Tupesis_2004f, Wildhaber_2005, Wnęk_2015f, Yildiz_2016e}, 1 case (1\%) had no gender recorded \cite{fjbuilsRepeatedBehaviorDeliberate2024}. \paragraph*{Age Group} 25 cases (35\%) were between 26 and 40 years of age \cite{Alao_2006i, Ali_2022g, Apikotoa_2022f, Ataya_2013, Benoist_2019e, Bhasin_2014, Chang_2017f, Cox_2007, DelgadoSalazar_2020c, Farhadi_2024h, Fry_2010, Gardner_2017h, Guinan_2019f, Jin_2023, Kumar_2019f, Losanoff_1996, Misra_2013, Qureshi_2016, Riva_2018j, Sakellaridis_2008f, Tammana_2012j, Trgo_2012f, Wnęk_2015f, Yildiz_2016e, fjbuilsRepeatedBehaviorDeliberate2024}, 18 cases (25\%) were between 18 and 25 years of age \cite{Akay_2015f, Ali_2017, Atayan_2016, Bhattacharjee_2008, Csaky_1998e, Kar_2015, Kariholu_2008, Kobiela_2015, Losanoff_1996, Losanoff_1997e, Mesfin_2022a, Peixoto_2017f, Sobnach_2011f, Tupesis_2004f, Yasin_2009}, 13 cases (18\%) were under 18 years of age \cite{AlShaaibi_2021b, Ali_2020f, Cauchi_2002, DivsalarP._2023a, Goldman_1998f, Liu_2005, Naji_2012f, Ohno_2005, Tanrikulu_2015e, Tay_2004, Wildhaber_2005}, 11 cases (15\%) were between 41 and 60 years of age \cite{Al-Faham_2020k, Bhumi_2024f, CamachoDorado_2018, Emamhadi_2018, Hardy_2023g, Jehangir_2019h, Kumar_2001, Sultan_2024f, Thapa_2019f, Wadhwa_2015e, teWildt_2010}, 3 cases (4\%) were over 60 years of age \cite{Beecroft_1998, Kerestes_2019, Li_2013}, 2 cases (3\%) had no age documented \cite{Berry_2021e}. \paragraph*{Population} 36 cases (50\%) had a psychiatric history \cite{AlShaaibi_2021b, Alao_2006i, Ali_2020f, Apikotoa_2022f, Ataya_2013, Atayan_2016, Beecroft_1998, CamachoDorado_2018, Chang_2017f, DelgadoSalazar_2020c, DivsalarP._2023a, Farhadi_2024h, Fry_2010, Guinan_2019f, Hardy_2023g, Jehangir_2019h, Jin_2023, Kar_2015, Kerestes_2019, Kobiela_2015, Kumar_2001, Kumar_2019f, Liu_2005, Mesfin_2022a, Misra_2013, Ohno_2005, Peixoto_2017f, Sakellaridis_2008f, Sultan_2024f, Tammana_2012j, Tanrikulu_2015e, Yildiz_2016e, fjbuilsRepeatedBehaviorDeliberate2024, teWildt_2010}, 19 cases (26\%) had ingested previously \cite{Alao_2006i, Apikotoa_2022f, Berry_2021e, Bhattacharjee_2008, Csaky_1998e, DivsalarP._2023a, Emamhadi_2018, Guinan_2019f, Jehangir_2019h, Jin_2023, Liu_2005, Sakellaridis_2008f, Tanrikulu_2015e, Thapa_2019f, Yildiz_2016e, fjbuilsRepeatedBehaviorDeliberate2024, teWildt_2010}, 12 cases (17\%) were detained persons \cite{Alao_2006i, Ali_2022g, Apikotoa_2022f, Losanoff_1996, Losanoff_1997e, Qureshi_2016, Tammana_2012j, Trgo_2012f}, 7 cases (10\%) were severely disabled \cite{Atayan_2016, Kerestes_2019, Liu_2005, Ohno_2005, Peixoto_2017f, Yildiz_2016e, teWildt_2010}, 4 cases (6\%) were psychiatric inpatients \cite{DivsalarP._2023a, fjbuilsRepeatedBehaviorDeliberate2024, teWildt_2010}, 3 cases (4\%) were under the influence of alcohol \cite{Benoist_2019e, Csaky_1998e, Thapa_2019f}, 2 cases (3\%) were displaced people \cite{Akay_2015f, Gardner_2017h}. \paragraph*{Motivation} 34 cases (47\%) had a psychiatric motivation \cite{Al-Faham_2020k, Alao_2006i, Ali_2020f, Apikotoa_2022f, Ataya_2013, Atayan_2016, Bhasin_2014, Bhattacharjee_2008, DelgadoSalazar_2020c, DivsalarP._2023a, Emamhadi_2018, Farhadi_2024h, Guinan_2019f, Hardy_2023g, Jehangir_2019h, Jin_2023, Kar_2015, Kariholu_2008, Kerestes_2019, Kobiela_2015, Kumar_2001, Kumar_2019f, Li_2013, Liu_2005, Misra_2013, Ohno_2005, Sakellaridis_2008f, Sultan_2024f, Tammana_2012j, Tanrikulu_2015e, Yasin_2009, teWildt_2010}, 21 cases (29\%) were motivated by self-harm intention \cite{Al-Faham_2020k, AlShaaibi_2021b, Alao_2006i, Ali_2017, CamachoDorado_2018, Chang_2017f, Cox_2007, Csaky_1998e, Fry_2010, Li_2013, Losanoff_1996, Losanoff_1997e, Mesfin_2022a, Sakellaridis_2008f, Tammana_2012j, Tanrikulu_2015e, fjbuilsRepeatedBehaviorDeliberate2024}, 17 cases (24\%) had a psychosocial motivation \cite{Akay_2015f, Benoist_2019e, Bhattacharjee_2008, Cauchi_2002, Goldman_1998f, Hardy_2023g, Kobiela_2015, Li_2013, Naji_2012f, Qureshi_2016, Riva_2018j, Sobnach_2011f, Tay_2004, Thapa_2019f, Tupesis_2004f, Wildhaber_2005, Wnęk_2015f}, 9 cases (12\%) were motivated by protest \cite{Bhumi_2024f, Gardner_2017h, Losanoff_1996, Losanoff_1997e, Tupesis_2004f}, 9 cases (12\%) had another documented motivation \cite{Ali_2020f, Ali_2022g, Emamhadi_2018, Guinan_2019f, Peixoto_2017f, Sakellaridis_2008f, Trgo_2012f, Wadhwa_2015e, Yildiz_2016e}. \paragraph*{Object Characteristics} 51 cases (71\%) ingested a large diameter object (\textgreater{}2.5cm) \cite{Akay_2015f, Al-Faham_2020k, AlShaaibi_2021b, Alao_2006i, Ali_2017, Ali_2022g, Apikotoa_2022f, Atayan_2016, Berry_2021e, Bhasin_2014, CamachoDorado_2018, Cauchi_2002, Chang_2017f, Cox_2007, Csaky_1998e, DivsalarP._2023a, Emamhadi_2018, Gardner_2017h, Guinan_2019f, Jehangir_2019h, Jin_2023, Kariholu_2008, Kerestes_2019, Kobiela_2015, Kumar_2001, Kumar_2019f, Losanoff_1996, Losanoff_1997e, Mesfin_2022a, Misra_2013, Naji_2012f, Ohno_2005, Peixoto_2017f, Qureshi_2016, Riva_2018j, Sakellaridis_2008f, Sultan_2024f, Tanrikulu_2015e, Thapa_2019f, Trgo_2012f, Wnęk_2015f, Yildiz_2016e, fjbuilsRepeatedBehaviorDeliberate2024, teWildt_2010}, 44 cases (61\%) ingested multiple objects \cite{Ali_2020f, Apikotoa_2022f, Ataya_2013, Atayan_2016, Beecroft_1998, Bhattacharjee_2008, Bhumi_2024f, CamachoDorado_2018, Cauchi_2002, Emamhadi_2018, Farhadi_2024h, Fry_2010, Goldman_1998f, Guinan_2019f, Hardy_2023g, Jehangir_2019h, Jin_2023, Kar_2015, Kariholu_2008, Kobiela_2015, Kumar_2001, Kumar_2019f, Li_2013, Liu_2005, Losanoff_1996, Mesfin_2022a, Misra_2013, Naji_2012f, Ohno_2005, Sobnach_2011f, Sultan_2024f, Tammana_2012j, Tanrikulu_2015e, Tay_2004, Thapa_2019f, Wadhwa_2015e, Wildhaber_2005, Yasin_2009, fjbuilsRepeatedBehaviorDeliberate2024, teWildt_2010}, 34 cases (47\%) ingested a sharp object \cite{AlShaaibi_2021b, Alao_2006i, Apikotoa_2022f, Ataya_2013, Benoist_2019e, Bhasin_2014, Bhattacharjee_2008, CamachoDorado_2018, Csaky_1998e, DelgadoSalazar_2020c, DivsalarP._2023a, Emamhadi_2018, Farhadi_2024h, Fry_2010, Guinan_2019f, Hardy_2023g, Jehangir_2019h, Jin_2023, Kariholu_2008, Kobiela_2015, Kumar_2019f, Losanoff_1996, Losanoff_1997e, Mesfin_2022a, Misra_2013, Sobnach_2011f, Yasin_2009, teWildt_2010}, 32 cases (44\%) ingested a long object (\textgreater{}5cm) \cite{Al-Faham_2020k, AlShaaibi_2021b, Ali_2017, Ali_2022g, Atayan_2016, Bhasin_2014, CamachoDorado_2018, Chang_2017f, Cox_2007, Csaky_1998e, DivsalarP._2023a, Emamhadi_2018, Fry_2010, Gardner_2017h, Jin_2023, Kariholu_2008, Kerestes_2019, Kobiela_2015, Kumar_2019f, Mesfin_2022a, Misra_2013, Ohno_2005, Qureshi_2016, Sakellaridis_2008f, Sultan_2024f, Thapa_2019f, Trgo_2012f, Yasin_2009, Yildiz_2016e, teWildt_2010}, 9 cases (12\%) ingested a magnet \cite{Ali_2020f, Bhumi_2024f, Cauchi_2002, Liu_2005, Naji_2012f, Ohno_2005, Tanrikulu_2015e, Tay_2004, Wildhaber_2005}, 2 cases (3\%) ingested a button battery \cite{Berry_2021e, Bhumi_2024f}. \paragraph*{Outcomes} 48 cases (67\%) experienced a complication \cite{Ali_2017, Ali_2020f, Apikotoa_2022f, Atayan_2016, Beecroft_1998, Benoist_2019e, Berry_2021e, Bhasin_2014, Bhumi_2024f, CamachoDorado_2018, Cauchi_2002, Cox_2007, Csaky_1998e, DelgadoSalazar_2020c, DivsalarP._2023a, Emamhadi_2018, Farhadi_2024h, Fry_2010, Gardner_2017h, Goldman_1998f, Jin_2023, Kariholu_2008, Kerestes_2019, Kobiela_2015, Kumar_2001, Kumar_2019f, Liu_2005, Losanoff_1996, Mesfin_2022a, Misra_2013, Naji_2012f, Ohno_2005, Sakellaridis_2008f, Sobnach_2011f, Sultan_2024f, Tanrikulu_2015e, Tay_2004, Thapa_2019f, Trgo_2012f, Tupesis_2004f, Wildhaber_2005, Wnęk_2015f, Yasin_2009, Yildiz_2016e}, 44 cases (61\%) underwent surgery \cite{Al-Faham_2020k, AlShaaibi_2021b, Alao_2006i, Ali_2017, Ali_2020f, Atayan_2016, Beecroft_1998, Bhasin_2014, CamachoDorado_2018, Cauchi_2002, Chang_2017f, Cox_2007, Csaky_1998e, DelgadoSalazar_2020c, DivsalarP._2023a, Farhadi_2024h, Fry_2010, Gardner_2017h, Jin_2023, Kariholu_2008, Kerestes_2019, Kobiela_2015, Kumar_2019f, Liu_2005, Losanoff_1996, Losanoff_1997e, Mesfin_2022a, Misra_2013, Naji_2012f, Sobnach_2011f, Tanrikulu_2015e, Tay_2004, Thapa_2019f, Tupesis_2004f, Wildhaber_2005, Wnęk_2015f, Yasin_2009, Yildiz_2016e, fjbuilsRepeatedBehaviorDeliberate2024}, 31 cases (43\%) underwent endoscopy \cite{Akay_2015f, Ali_2022g, Apikotoa_2022f, Atayan_2016, Benoist_2019e, Berry_2021e, Bhasin_2014, Bhumi_2024f, CamachoDorado_2018, Chang_2017f, DelgadoSalazar_2020c, Gardner_2017h, Guinan_2019f, Hardy_2023g, Jehangir_2019h, Kariholu_2008, Li_2013, Liu_2005, Ohno_2005, Peixoto_2017f, Qureshi_2016, Riva_2018j, Sakellaridis_2008f, Sultan_2024f, Tammana_2012j, Tanrikulu_2015e, Trgo_2012f, Wadhwa_2015e, Wnęk_2015f, teWildt_2010}, 7 cases (10\%) were managed conservatively \cite{Ataya_2013, Bhattacharjee_2008, DivsalarP._2023a, Emamhadi_2018, Goldman_1998f, Kar_2015, Kumar_2001}, 2 cases (3\%) died \cite{Emamhadi_2018, Kumar_2001}. All 90 were male gender. 90 cases (100\%) were detained at the time of ingestion \cite{Elghali_2016, Karp_1991b, Lee_2007}, 88 cases (98\%) were intentional ingestions \cite{Elghali_2016, Karp_1991b, Lee_2007}, 30 cases (33\%) had a psychiatric history documented \cite{Elghali_2016, Karp_1991b, Lee_2007}, 2 cases (2\%) had a history of prior ingestion \cite{Elghali_2016}. No cases were reported for were psychiatric inpatients, were displaced people, were under the influence of alcohol at the time of ingestion, and had a severe disability history.
\paragraph*{Motivation}  70 cases (78\%) reported protest motivation \cite{Elghali_2016, Karp_1991b, Lee_2007}, 12 cases (13\%) reported psychiatric motivation \cite{Karp_1991b}, 6 cases (7\%) reported self-harm motivation \cite{Elghali_2016, Karp_1991b}. No cases were reported for psychosocial motivation and other motivation.
\paragraph*{Object Characteristics}  68 cases (76\%) involved sharp object ingestion \cite{Elghali_2016, Karp_1991b, Lee_2007}, 32 cases (36\%) involved long (\textgreater 5cm) object ingestion \cite{Lee_2007}, 25 cases (28\%) involved ingestion of multiple objects \cite{Elghali_2016, Lee_2007}. No cases were reported for button battery ingestion, magnet ingestion, and involved large diameter (\textgreater 2.5cm) object ingestion.
\paragraph*{Outcomes}  47 cases (52\%) underwent endoscopic intervention \cite{Elghali_2016, Lee_2007}, 29 cases (32\%) were managed conservatively \cite{Elghali_2016, Karp_1991b}, 15 cases (17\%) underwent surgical intervention \cite{Elghali_2016, Karp_1991b, Lee_2007}, 6 cases (7\%) reported complications \cite{Lee_2007}, 1 case (1\%) died \cite{Elghali_2016}.
\paragraph*{Geographical Location}Cases were recorded in 33 countries: 13 cases from USA \cite{Alao_2006i, Ataya_2013, Bhumi_2024f, Fry_2010, Guinan_2019f, Hardy_2023g, Jehangir_2019h, Kerestes_2019, Kumar_2001, Liu_2005, Tammana_2012j, Tay_2004, Tupesis_2004f}; 7 cases from India \cite{Bhasin_2014, Bhattacharjee_2008, Kar_2015, Kariholu_2008, Kumar_2019f, Misra_2013, Wadhwa_2015e} and UK \cite{Beecroft_1998, Berry_2021e, Cauchi_2002, Cox_2007, Gardner_2017h, Qureshi_2016}; 6 cases from Bulgaria \cite{Losanoff_1996, Losanoff_1997e}; 5 cases from Iran \cite{DivsalarP._2023a, Emamhadi_2018, Farhadi_2024h}; 4 cases from Turkey \cite{Akay_2015f, Atayan_2016, Tanrikulu_2015e, Yildiz_2016e}; 2 cases from China \cite{Jin_2023, Li_2013}, Poland \cite{Kobiela_2015, Wnęk_2015f}, and Spain \cite{CamachoDorado_2018, fjbuilsRepeatedBehaviorDeliberate2024}; 1 case from Australia \cite{Apikotoa_2022f}, Bahrain \cite{Ali_2020f}, Croatia \cite{Trgo_2012f}, Ecuador \cite{DelgadoSalazar_2020c}, Egypt \cite{Ali_2022g}, Ethiopia \cite{Mesfin_2022a}, Germany \cite{teWildt_2010}, Greece \cite{Sakellaridis_2008f}, Hungary \cite{Csaky_1998e}, Iraq \cite{Al-Faham_2020k}, Israel \cite{Goldman_1998f}, Italy \cite{Riva_2018j}, Japan \cite{Ohno_2005}, Nepal \cite{Thapa_2019f}, Netherlands \cite{Benoist_2019e}, Oman \cite{AlShaaibi_2021b}, Pakistan \cite{Yasin_2009}, Portugal \cite{Peixoto_2017f}, Qatar \cite{Ali_2017}, Saudi Arabia \cite{Sultan_2024f}, South Africa \cite{Sobnach_2011f}, Sweden \cite{Naji_2012f}, Switzerland \cite{Wildhaber_2005}, and Taiwan \cite{Chang_2017f}. \paragraph*{Gender} 43 cases (60\%) were male \cite{Akay_2015f, Al-Faham_2020k, Alao_2006i, Ali_2017, Ali_2022g, Apikotoa_2022f, Atayan_2016, Benoist_2019e, Berry_2021e, Bhumi_2024f, CamachoDorado_2018, Csaky_1998e, Emamhadi_2018, Farhadi_2024h, Fry_2010, Gardner_2017h, Guinan_2019f, Jehangir_2019h, Jin_2023, Kobiela_2015, Kumar_2001, Kumar_2019f, Liu_2005, Losanoff_1996, Losanoff_1997e, Mesfin_2022a, Misra_2013, Qureshi_2016, Riva_2018j, Sobnach_2011f, Tammana_2012j, Tanrikulu_2015e, Tay_2004, Thapa_2019f, Trgo_2012f, Wadhwa_2015e, Yasin_2009, teWildt_2010}, 28 cases (39\%) were female \cite{AlShaaibi_2021b, Ali_2020f, Ataya_2013, Beecroft_1998, Bhasin_2014, Bhattacharjee_2008, Cauchi_2002, Chang_2017f, Cox_2007, DelgadoSalazar_2020c, DivsalarP._2023a, Goldman_1998f, Hardy_2023g, Kar_2015, Kariholu_2008, Kerestes_2019, Li_2013, Naji_2012f, Ohno_2005, Peixoto_2017f, Sakellaridis_2008f, Sultan_2024f, Tupesis_2004f, Wildhaber_2005, Wnęk_2015f, Yildiz_2016e}, 1 case (1\%) had no gender recorded \cite{fjbuilsRepeatedBehaviorDeliberate2024}. \paragraph*{Age Group} 25 cases (35\%) were between 26 and 40 years of age \cite{Alao_2006i, Ali_2022g, Apikotoa_2022f, Ataya_2013, Benoist_2019e, Bhasin_2014, Chang_2017f, Cox_2007, DelgadoSalazar_2020c, Farhadi_2024h, Fry_2010, Gardner_2017h, Guinan_2019f, Jin_2023, Kumar_2019f, Losanoff_1996, Misra_2013, Qureshi_2016, Riva_2018j, Sakellaridis_2008f, Tammana_2012j, Trgo_2012f, Wnęk_2015f, Yildiz_2016e, fjbuilsRepeatedBehaviorDeliberate2024}, 18 cases (25\%) were between 18 and 25 years of age \cite{Akay_2015f, Ali_2017, Atayan_2016, Bhattacharjee_2008, Csaky_1998e, Kar_2015, Kariholu_2008, Kobiela_2015, Losanoff_1996, Losanoff_1997e, Mesfin_2022a, Peixoto_2017f, Sobnach_2011f, Tupesis_2004f, Yasin_2009}, 13 cases (18\%) were under 18 years of age \cite{AlShaaibi_2021b, Ali_2020f, Cauchi_2002, DivsalarP._2023a, Goldman_1998f, Liu_2005, Naji_2012f, Ohno_2005, Tanrikulu_2015e, Tay_2004, Wildhaber_2005}, 11 cases (15\%) were between 41 and 60 years of age \cite{Al-Faham_2020k, Bhumi_2024f, CamachoDorado_2018, Emamhadi_2018, Hardy_2023g, Jehangir_2019h, Kumar_2001, Sultan_2024f, Thapa_2019f, Wadhwa_2015e, teWildt_2010}, 3 cases (4\%) were over 60 years of age \cite{Beecroft_1998, Kerestes_2019, Li_2013}, 2 cases (3\%) had no age documented \cite{Berry_2021e}. \paragraph*{Population} 36 cases (50\%) had a psychiatric history \cite{AlShaaibi_2021b, Alao_2006i, Ali_2020f, Apikotoa_2022f, Ataya_2013, Atayan_2016, Beecroft_1998, CamachoDorado_2018, Chang_2017f, DelgadoSalazar_2020c, DivsalarP._2023a, Farhadi_2024h, Fry_2010, Guinan_2019f, Hardy_2023g, Jehangir_2019h, Jin_2023, Kar_2015, Kerestes_2019, Kobiela_2015, Kumar_2001, Kumar_2019f, Liu_2005, Mesfin_2022a, Misra_2013, Ohno_2005, Peixoto_2017f, Sakellaridis_2008f, Sultan_2024f, Tammana_2012j, Tanrikulu_2015e, Yildiz_2016e, fjbuilsRepeatedBehaviorDeliberate2024, teWildt_2010}, 19 cases (26\%) had ingested previously \cite{Alao_2006i, Apikotoa_2022f, Berry_2021e, Bhattacharjee_2008, Csaky_1998e, DivsalarP._2023a, Emamhadi_2018, Guinan_2019f, Jehangir_2019h, Jin_2023, Liu_2005, Sakellaridis_2008f, Tanrikulu_2015e, Thapa_2019f, Yildiz_2016e, fjbuilsRepeatedBehaviorDeliberate2024, teWildt_2010}, 12 cases (17\%) were detained persons \cite{Alao_2006i, Ali_2022g, Apikotoa_2022f, Losanoff_1996, Losanoff_1997e, Qureshi_2016, Tammana_2012j, Trgo_2012f}, 7 cases (10\%) were severely disabled \cite{Atayan_2016, Kerestes_2019, Liu_2005, Ohno_2005, Peixoto_2017f, Yildiz_2016e, teWildt_2010}, 4 cases (6\%) were psychiatric inpatients \cite{DivsalarP._2023a, fjbuilsRepeatedBehaviorDeliberate2024, teWildt_2010}, 3 cases (4\%) were under the influence of alcohol \cite{Benoist_2019e, Csaky_1998e, Thapa_2019f}, 2 cases (3\%) were displaced people \cite{Akay_2015f, Gardner_2017h}. \paragraph*{Motivation} 34 cases (47\%) had a psychiatric motivation \cite{Al-Faham_2020k, Alao_2006i, Ali_2020f, Apikotoa_2022f, Ataya_2013, Atayan_2016, Bhasin_2014, Bhattacharjee_2008, DelgadoSalazar_2020c, DivsalarP._2023a, Emamhadi_2018, Farhadi_2024h, Guinan_2019f, Hardy_2023g, Jehangir_2019h, Jin_2023, Kar_2015, Kariholu_2008, Kerestes_2019, Kobiela_2015, Kumar_2001, Kumar_2019f, Li_2013, Liu_2005, Misra_2013, Ohno_2005, Sakellaridis_2008f, Sultan_2024f, Tammana_2012j, Tanrikulu_2015e, Yasin_2009, teWildt_2010}, 21 cases (29\%) were motivated by self-harm intention \cite{Al-Faham_2020k, AlShaaibi_2021b, Alao_2006i, Ali_2017, CamachoDorado_2018, Chang_2017f, Cox_2007, Csaky_1998e, Fry_2010, Li_2013, Losanoff_1996, Losanoff_1997e, Mesfin_2022a, Sakellaridis_2008f, Tammana_2012j, Tanrikulu_2015e, fjbuilsRepeatedBehaviorDeliberate2024}, 17 cases (24\%) had a psychosocial motivation \cite{Akay_2015f, Benoist_2019e, Bhattacharjee_2008, Cauchi_2002, Goldman_1998f, Hardy_2023g, Kobiela_2015, Li_2013, Naji_2012f, Qureshi_2016, Riva_2018j, Sobnach_2011f, Tay_2004, Thapa_2019f, Tupesis_2004f, Wildhaber_2005, Wnęk_2015f}, 9 cases (12\%) were motivated by protest \cite{Bhumi_2024f, Gardner_2017h, Losanoff_1996, Losanoff_1997e, Tupesis_2004f}, 9 cases (12\%) had another documented motivation \cite{Ali_2020f, Ali_2022g, Emamhadi_2018, Guinan_2019f, Peixoto_2017f, Sakellaridis_2008f, Trgo_2012f, Wadhwa_2015e, Yildiz_2016e}. \paragraph*{Object Characteristics} 51 cases (71\%) ingested a large diameter object (\textgreater{}2.5cm) \cite{Akay_2015f, Al-Faham_2020k, AlShaaibi_2021b, Alao_2006i, Ali_2017, Ali_2022g, Apikotoa_2022f, Atayan_2016, Berry_2021e, Bhasin_2014, CamachoDorado_2018, Cauchi_2002, Chang_2017f, Cox_2007, Csaky_1998e, DivsalarP._2023a, Emamhadi_2018, Gardner_2017h, Guinan_2019f, Jehangir_2019h, Jin_2023, Kariholu_2008, Kerestes_2019, Kobiela_2015, Kumar_2001, Kumar_2019f, Losanoff_1996, Losanoff_1997e, Mesfin_2022a, Misra_2013, Naji_2012f, Ohno_2005, Peixoto_2017f, Qureshi_2016, Riva_2018j, Sakellaridis_2008f, Sultan_2024f, Tanrikulu_2015e, Thapa_2019f, Trgo_2012f, Wnęk_2015f, Yildiz_2016e, fjbuilsRepeatedBehaviorDeliberate2024, teWildt_2010}, 44 cases (61\%) ingested multiple objects \cite{Ali_2020f, Apikotoa_2022f, Ataya_2013, Atayan_2016, Beecroft_1998, Bhattacharjee_2008, Bhumi_2024f, CamachoDorado_2018, Cauchi_2002, Emamhadi_2018, Farhadi_2024h, Fry_2010, Goldman_1998f, Guinan_2019f, Hardy_2023g, Jehangir_2019h, Jin_2023, Kar_2015, Kariholu_2008, Kobiela_2015, Kumar_2001, Kumar_2019f, Li_2013, Liu_2005, Losanoff_1996, Mesfin_2022a, Misra_2013, Naji_2012f, Ohno_2005, Sobnach_2011f, Sultan_2024f, Tammana_2012j, Tanrikulu_2015e, Tay_2004, Thapa_2019f, Wadhwa_2015e, Wildhaber_2005, Yasin_2009, fjbuilsRepeatedBehaviorDeliberate2024, teWildt_2010}, 34 cases (47\%) ingested a sharp object \cite{AlShaaibi_2021b, Alao_2006i, Apikotoa_2022f, Ataya_2013, Benoist_2019e, Bhasin_2014, Bhattacharjee_2008, CamachoDorado_2018, Csaky_1998e, DelgadoSalazar_2020c, DivsalarP._2023a, Emamhadi_2018, Farhadi_2024h, Fry_2010, Guinan_2019f, Hardy_2023g, Jehangir_2019h, Jin_2023, Kariholu_2008, Kobiela_2015, Kumar_2019f, Losanoff_1996, Losanoff_1997e, Mesfin_2022a, Misra_2013, Sobnach_2011f, Yasin_2009, teWildt_2010}, 32 cases (44\%) ingested a long object (\textgreater{}5cm) \cite{Al-Faham_2020k, AlShaaibi_2021b, Ali_2017, Ali_2022g, Atayan_2016, Bhasin_2014, CamachoDorado_2018, Chang_2017f, Cox_2007, Csaky_1998e, DivsalarP._2023a, Emamhadi_2018, Fry_2010, Gardner_2017h, Jin_2023, Kariholu_2008, Kerestes_2019, Kobiela_2015, Kumar_2019f, Mesfin_2022a, Misra_2013, Ohno_2005, Qureshi_2016, Sakellaridis_2008f, Sultan_2024f, Thapa_2019f, Trgo_2012f, Yasin_2009, Yildiz_2016e, teWildt_2010}, 9 cases (12\%) ingested a magnet \cite{Ali_2020f, Bhumi_2024f, Cauchi_2002, Liu_2005, Naji_2012f, Ohno_2005, Tanrikulu_2015e, Tay_2004, Wildhaber_2005}, 2 cases (3\%) ingested a button battery \cite{Berry_2021e, Bhumi_2024f}. \paragraph*{Outcomes} 48 cases (67\%) experienced a complication \cite{Ali_2017, Ali_2020f, Apikotoa_2022f, Atayan_2016, Beecroft_1998, Benoist_2019e, Berry_2021e, Bhasin_2014, Bhumi_2024f, CamachoDorado_2018, Cauchi_2002, Cox_2007, Csaky_1998e, DelgadoSalazar_2020c, DivsalarP._2023a, Emamhadi_2018, Farhadi_2024h, Fry_2010, Gardner_2017h, Goldman_1998f, Jin_2023, Kariholu_2008, Kerestes_2019, Kobiela_2015, Kumar_2001, Kumar_2019f, Liu_2005, Losanoff_1996, Mesfin_2022a, Misra_2013, Naji_2012f, Ohno_2005, Sakellaridis_2008f, Sobnach_2011f, Sultan_2024f, Tanrikulu_2015e, Tay_2004, Thapa_2019f, Trgo_2012f, Tupesis_2004f, Wildhaber_2005, Wnęk_2015f, Yasin_2009, Yildiz_2016e}, 44 cases (61\%) underwent surgery \cite{Al-Faham_2020k, AlShaaibi_2021b, Alao_2006i, Ali_2017, Ali_2020f, Atayan_2016, Beecroft_1998, Bhasin_2014, CamachoDorado_2018, Cauchi_2002, Chang_2017f, Cox_2007, Csaky_1998e, DelgadoSalazar_2020c, DivsalarP._2023a, Farhadi_2024h, Fry_2010, Gardner_2017h, Jin_2023, Kariholu_2008, Kerestes_2019, Kobiela_2015, Kumar_2019f, Liu_2005, Losanoff_1996, Losanoff_1997e, Mesfin_2022a, Misra_2013, Naji_2012f, Sobnach_2011f, Tanrikulu_2015e, Tay_2004, Thapa_2019f, Tupesis_2004f, Wildhaber_2005, Wnęk_2015f, Yasin_2009, Yildiz_2016e, fjbuilsRepeatedBehaviorDeliberate2024}, 31 cases (43\%) underwent endoscopy \cite{Akay_2015f, Ali_2022g, Apikotoa_2022f, Atayan_2016, Benoist_2019e, Berry_2021e, Bhasin_2014, Bhumi_2024f, CamachoDorado_2018, Chang_2017f, DelgadoSalazar_2020c, Gardner_2017h, Guinan_2019f, Hardy_2023g, Jehangir_2019h, Kariholu_2008, Li_2013, Liu_2005, Ohno_2005, Peixoto_2017f, Qureshi_2016, Riva_2018j, Sakellaridis_2008f, Sultan_2024f, Tammana_2012j, Tanrikulu_2015e, Trgo_2012f, Wadhwa_2015e, Wnęk_2015f, teWildt_2010}, 7 cases (10\%) were managed conservatively \cite{Ataya_2013, Bhattacharjee_2008, DivsalarP._2023a, Emamhadi_2018, Goldman_1998f, Kar_2015, Kumar_2001}, 2 cases (3\%) died \cite{Emamhadi_2018, Kumar_2001}. All 90 were male gender. 90 cases (100\%) were detained at the time of ingestion \cite{Elghali_2016, Karp_1991b, Lee_2007}, 88 cases (98\%) were intentional ingestions \cite{Elghali_2016, Karp_1991b, Lee_2007}, 30 cases (33\%) had a psychiatric history documented \cite{Elghali_2016, Karp_1991b, Lee_2007}, 2 cases (2\%) had a history of prior ingestion \cite{Elghali_2016}. No cases were reported for were psychiatric inpatients, were displaced people, were under the influence of alcohol at the time of ingestion, and had a severe disability history.
\paragraph*{Motivation}  70 cases (78\%) reported protest motivation \cite{Elghali_2016, Karp_1991b, Lee_2007}, 12 cases (13\%) reported psychiatric motivation \cite{Karp_1991b}, 6 cases (7\%) reported self-harm motivation \cite{Elghali_2016, Karp_1991b}. No cases were reported for psychosocial motivation and other motivation.
\paragraph*{Object Characteristics}  68 cases (76\%) involved sharp object ingestion \cite{Elghali_2016, Karp_1991b, Lee_2007}, 32 cases (36\%) involved long (\textgreater 5cm) object ingestion \cite{Lee_2007}, 25 cases (28\%) involved ingestion of multiple objects \cite{Elghali_2016, Lee_2007}. No cases were reported for button battery ingestion, magnet ingestion, and involved large diameter (\textgreater 2.5cm) object ingestion.
\paragraph*{Outcomes}  47 cases (52\%) underwent endoscopic intervention \cite{Elghali_2016, Lee_2007}, 29 cases (32\%) were managed conservatively \cite{Elghali_2016, Karp_1991b}, 15 cases (17\%) underwent surgical intervention \cite{Elghali_2016, Karp_1991b, Lee_2007}, 6 cases (7\%) reported complications \cite{Lee_2007}, 1 case (1\%) died \cite{Elghali_2016}.
\paragraph*{Geographical Location}Cases were recorded in 33 countries: 13 cases from USA \cite{Alao_2006i, Ataya_2013, Bhumi_2024f, Fry_2010, Guinan_2019f, Hardy_2023g, Jehangir_2019h, Kerestes_2019, Kumar_2001, Liu_2005, Tammana_2012j, Tay_2004, Tupesis_2004f}; 7 cases from India \cite{Bhasin_2014, Bhattacharjee_2008, Kar_2015, Kariholu_2008, Kumar_2019f, Misra_2013, Wadhwa_2015e} and UK \cite{Beecroft_1998, Berry_2021e, Cauchi_2002, Cox_2007, Gardner_2017h, Qureshi_2016}; 6 cases from Bulgaria \cite{Losanoff_1996, Losanoff_1997e}; 5 cases from Iran \cite{DivsalarP._2023a, Emamhadi_2018, Farhadi_2024h}; 4 cases from Turkey \cite{Akay_2015f, Atayan_2016, Tanrikulu_2015e, Yildiz_2016e}; 2 cases from China \cite{Jin_2023, Li_2013}, Poland \cite{Kobiela_2015, Wnęk_2015f}, and Spain \cite{CamachoDorado_2018, fjbuilsRepeatedBehaviorDeliberate2024}; 1 case from Australia \cite{Apikotoa_2022f}, Bahrain \cite{Ali_2020f}, Croatia \cite{Trgo_2012f}, Ecuador \cite{DelgadoSalazar_2020c}, Egypt \cite{Ali_2022g}, Ethiopia \cite{Mesfin_2022a}, Germany \cite{teWildt_2010}, Greece \cite{Sakellaridis_2008f}, Hungary \cite{Csaky_1998e}, Iraq \cite{Al-Faham_2020k}, Israel \cite{Goldman_1998f}, Italy \cite{Riva_2018j}, Japan \cite{Ohno_2005}, Nepal \cite{Thapa_2019f}, Netherlands \cite{Benoist_2019e}, Oman \cite{AlShaaibi_2021b}, Pakistan \cite{Yasin_2009}, Portugal \cite{Peixoto_2017f}, Qatar \cite{Ali_2017}, Saudi Arabia \cite{Sultan_2024f}, South Africa \cite{Sobnach_2011f}, Sweden \cite{Naji_2012f}, Switzerland \cite{Wildhaber_2005}, and Taiwan \cite{Chang_2017f}. \paragraph*{Gender} 43 cases (60\%) were male \cite{Akay_2015f, Al-Faham_2020k, Alao_2006i, Ali_2017, Ali_2022g, Apikotoa_2022f, Atayan_2016, Benoist_2019e, Berry_2021e, Bhumi_2024f, CamachoDorado_2018, Csaky_1998e, Emamhadi_2018, Farhadi_2024h, Fry_2010, Gardner_2017h, Guinan_2019f, Jehangir_2019h, Jin_2023, Kobiela_2015, Kumar_2001, Kumar_2019f, Liu_2005, Losanoff_1996, Losanoff_1997e, Mesfin_2022a, Misra_2013, Qureshi_2016, Riva_2018j, Sobnach_2011f, Tammana_2012j, Tanrikulu_2015e, Tay_2004, Thapa_2019f, Trgo_2012f, Wadhwa_2015e, Yasin_2009, teWildt_2010}, 28 cases (39\%) were female \cite{AlShaaibi_2021b, Ali_2020f, Ataya_2013, Beecroft_1998, Bhasin_2014, Bhattacharjee_2008, Cauchi_2002, Chang_2017f, Cox_2007, DelgadoSalazar_2020c, DivsalarP._2023a, Goldman_1998f, Hardy_2023g, Kar_2015, Kariholu_2008, Kerestes_2019, Li_2013, Naji_2012f, Ohno_2005, Peixoto_2017f, Sakellaridis_2008f, Sultan_2024f, Tupesis_2004f, Wildhaber_2005, Wnęk_2015f, Yildiz_2016e}, 1 case (1\%) had no gender recorded \cite{fjbuilsRepeatedBehaviorDeliberate2024}. \paragraph*{Age Group} 25 cases (35\%) were between 26 and 40 years of age \cite{Alao_2006i, Ali_2022g, Apikotoa_2022f, Ataya_2013, Benoist_2019e, Bhasin_2014, Chang_2017f, Cox_2007, DelgadoSalazar_2020c, Farhadi_2024h, Fry_2010, Gardner_2017h, Guinan_2019f, Jin_2023, Kumar_2019f, Losanoff_1996, Misra_2013, Qureshi_2016, Riva_2018j, Sakellaridis_2008f, Tammana_2012j, Trgo_2012f, Wnęk_2015f, Yildiz_2016e, fjbuilsRepeatedBehaviorDeliberate2024}, 18 cases (25\%) were between 18 and 25 years of age \cite{Akay_2015f, Ali_2017, Atayan_2016, Bhattacharjee_2008, Csaky_1998e, Kar_2015, Kariholu_2008, Kobiela_2015, Losanoff_1996, Losanoff_1997e, Mesfin_2022a, Peixoto_2017f, Sobnach_2011f, Tupesis_2004f, Yasin_2009}, 13 cases (18\%) were under 18 years of age \cite{AlShaaibi_2021b, Ali_2020f, Cauchi_2002, DivsalarP._2023a, Goldman_1998f, Liu_2005, Naji_2012f, Ohno_2005, Tanrikulu_2015e, Tay_2004, Wildhaber_2005}, 11 cases (15\%) were between 41 and 60 years of age \cite{Al-Faham_2020k, Bhumi_2024f, CamachoDorado_2018, Emamhadi_2018, Hardy_2023g, Jehangir_2019h, Kumar_2001, Sultan_2024f, Thapa_2019f, Wadhwa_2015e, teWildt_2010}, 3 cases (4\%) were over 60 years of age \cite{Beecroft_1998, Kerestes_2019, Li_2013}, 2 cases (3\%) had no age documented \cite{Berry_2021e}. \paragraph*{Population} 36 cases (50\%) had a psychiatric history \cite{AlShaaibi_2021b, Alao_2006i, Ali_2020f, Apikotoa_2022f, Ataya_2013, Atayan_2016, Beecroft_1998, CamachoDorado_2018, Chang_2017f, DelgadoSalazar_2020c, DivsalarP._2023a, Farhadi_2024h, Fry_2010, Guinan_2019f, Hardy_2023g, Jehangir_2019h, Jin_2023, Kar_2015, Kerestes_2019, Kobiela_2015, Kumar_2001, Kumar_2019f, Liu_2005, Mesfin_2022a, Misra_2013, Ohno_2005, Peixoto_2017f, Sakellaridis_2008f, Sultan_2024f, Tammana_2012j, Tanrikulu_2015e, Yildiz_2016e, fjbuilsRepeatedBehaviorDeliberate2024, teWildt_2010}, 19 cases (26\%) had ingested previously \cite{Alao_2006i, Apikotoa_2022f, Berry_2021e, Bhattacharjee_2008, Csaky_1998e, DivsalarP._2023a, Emamhadi_2018, Guinan_2019f, Jehangir_2019h, Jin_2023, Liu_2005, Sakellaridis_2008f, Tanrikulu_2015e, Thapa_2019f, Yildiz_2016e, fjbuilsRepeatedBehaviorDeliberate2024, teWildt_2010}, 12 cases (17\%) were detained persons \cite{Alao_2006i, Ali_2022g, Apikotoa_2022f, Losanoff_1996, Losanoff_1997e, Qureshi_2016, Tammana_2012j, Trgo_2012f}, 7 cases (10\%) were severely disabled \cite{Atayan_2016, Kerestes_2019, Liu_2005, Ohno_2005, Peixoto_2017f, Yildiz_2016e, teWildt_2010}, 4 cases (6\%) were psychiatric inpatients \cite{DivsalarP._2023a, fjbuilsRepeatedBehaviorDeliberate2024, teWildt_2010}, 3 cases (4\%) were under the influence of alcohol \cite{Benoist_2019e, Csaky_1998e, Thapa_2019f}, 2 cases (3\%) were displaced people \cite{Akay_2015f, Gardner_2017h}. \paragraph*{Motivation} 34 cases (47\%) had a psychiatric motivation \cite{Al-Faham_2020k, Alao_2006i, Ali_2020f, Apikotoa_2022f, Ataya_2013, Atayan_2016, Bhasin_2014, Bhattacharjee_2008, DelgadoSalazar_2020c, DivsalarP._2023a, Emamhadi_2018, Farhadi_2024h, Guinan_2019f, Hardy_2023g, Jehangir_2019h, Jin_2023, Kar_2015, Kariholu_2008, Kerestes_2019, Kobiela_2015, Kumar_2001, Kumar_2019f, Li_2013, Liu_2005, Misra_2013, Ohno_2005, Sakellaridis_2008f, Sultan_2024f, Tammana_2012j, Tanrikulu_2015e, Yasin_2009, teWildt_2010}, 21 cases (29\%) were motivated by self-harm intention \cite{Al-Faham_2020k, AlShaaibi_2021b, Alao_2006i, Ali_2017, CamachoDorado_2018, Chang_2017f, Cox_2007, Csaky_1998e, Fry_2010, Li_2013, Losanoff_1996, Losanoff_1997e, Mesfin_2022a, Sakellaridis_2008f, Tammana_2012j, Tanrikulu_2015e, fjbuilsRepeatedBehaviorDeliberate2024}, 17 cases (24\%) had a psychosocial motivation \cite{Akay_2015f, Benoist_2019e, Bhattacharjee_2008, Cauchi_2002, Goldman_1998f, Hardy_2023g, Kobiela_2015, Li_2013, Naji_2012f, Qureshi_2016, Riva_2018j, Sobnach_2011f, Tay_2004, Thapa_2019f, Tupesis_2004f, Wildhaber_2005, Wnęk_2015f}, 9 cases (12\%) were motivated by protest \cite{Bhumi_2024f, Gardner_2017h, Losanoff_1996, Losanoff_1997e, Tupesis_2004f}, 9 cases (12\%) had another documented motivation \cite{Ali_2020f, Ali_2022g, Emamhadi_2018, Guinan_2019f, Peixoto_2017f, Sakellaridis_2008f, Trgo_2012f, Wadhwa_2015e, Yildiz_2016e}. \paragraph*{Object Characteristics} 51 cases (71\%) ingested a large diameter object (\textgreater{}2.5cm) \cite{Akay_2015f, Al-Faham_2020k, AlShaaibi_2021b, Alao_2006i, Ali_2017, Ali_2022g, Apikotoa_2022f, Atayan_2016, Berry_2021e, Bhasin_2014, CamachoDorado_2018, Cauchi_2002, Chang_2017f, Cox_2007, Csaky_1998e, DivsalarP._2023a, Emamhadi_2018, Gardner_2017h, Guinan_2019f, Jehangir_2019h, Jin_2023, Kariholu_2008, Kerestes_2019, Kobiela_2015, Kumar_2001, Kumar_2019f, Losanoff_1996, Losanoff_1997e, Mesfin_2022a, Misra_2013, Naji_2012f, Ohno_2005, Peixoto_2017f, Qureshi_2016, Riva_2018j, Sakellaridis_2008f, Sultan_2024f, Tanrikulu_2015e, Thapa_2019f, Trgo_2012f, Wnęk_2015f, Yildiz_2016e, fjbuilsRepeatedBehaviorDeliberate2024, teWildt_2010}, 44 cases (61\%) ingested multiple objects \cite{Ali_2020f, Apikotoa_2022f, Ataya_2013, Atayan_2016, Beecroft_1998, Bhattacharjee_2008, Bhumi_2024f, CamachoDorado_2018, Cauchi_2002, Emamhadi_2018, Farhadi_2024h, Fry_2010, Goldman_1998f, Guinan_2019f, Hardy_2023g, Jehangir_2019h, Jin_2023, Kar_2015, Kariholu_2008, Kobiela_2015, Kumar_2001, Kumar_2019f, Li_2013, Liu_2005, Losanoff_1996, Mesfin_2022a, Misra_2013, Naji_2012f, Ohno_2005, Sobnach_2011f, Sultan_2024f, Tammana_2012j, Tanrikulu_2015e, Tay_2004, Thapa_2019f, Wadhwa_2015e, Wildhaber_2005, Yasin_2009, fjbuilsRepeatedBehaviorDeliberate2024, teWildt_2010}, 34 cases (47\%) ingested a sharp object \cite{AlShaaibi_2021b, Alao_2006i, Apikotoa_2022f, Ataya_2013, Benoist_2019e, Bhasin_2014, Bhattacharjee_2008, CamachoDorado_2018, Csaky_1998e, DelgadoSalazar_2020c, DivsalarP._2023a, Emamhadi_2018, Farhadi_2024h, Fry_2010, Guinan_2019f, Hardy_2023g, Jehangir_2019h, Jin_2023, Kariholu_2008, Kobiela_2015, Kumar_2019f, Losanoff_1996, Losanoff_1997e, Mesfin_2022a, Misra_2013, Sobnach_2011f, Yasin_2009, teWildt_2010}, 32 cases (44\%) ingested a long object (\textgreater{}5cm) \cite{Al-Faham_2020k, AlShaaibi_2021b, Ali_2017, Ali_2022g, Atayan_2016, Bhasin_2014, CamachoDorado_2018, Chang_2017f, Cox_2007, Csaky_1998e, DivsalarP._2023a, Emamhadi_2018, Fry_2010, Gardner_2017h, Jin_2023, Kariholu_2008, Kerestes_2019, Kobiela_2015, Kumar_2019f, Mesfin_2022a, Misra_2013, Ohno_2005, Qureshi_2016, Sakellaridis_2008f, Sultan_2024f, Thapa_2019f, Trgo_2012f, Yasin_2009, Yildiz_2016e, teWildt_2010}, 9 cases (12\%) ingested a magnet \cite{Ali_2020f, Bhumi_2024f, Cauchi_2002, Liu_2005, Naji_2012f, Ohno_2005, Tanrikulu_2015e, Tay_2004, Wildhaber_2005}, 2 cases (3\%) ingested a button battery \cite{Berry_2021e, Bhumi_2024f}. \paragraph*{Outcomes} 48 cases (67\%) experienced a complication \cite{Ali_2017, Ali_2020f, Apikotoa_2022f, Atayan_2016, Beecroft_1998, Benoist_2019e, Berry_2021e, Bhasin_2014, Bhumi_2024f, CamachoDorado_2018, Cauchi_2002, Cox_2007, Csaky_1998e, DelgadoSalazar_2020c, DivsalarP._2023a, Emamhadi_2018, Farhadi_2024h, Fry_2010, Gardner_2017h, Goldman_1998f, Jin_2023, Kariholu_2008, Kerestes_2019, Kobiela_2015, Kumar_2001, Kumar_2019f, Liu_2005, Losanoff_1996, Mesfin_2022a, Misra_2013, Naji_2012f, Ohno_2005, Sakellaridis_2008f, Sobnach_2011f, Sultan_2024f, Tanrikulu_2015e, Tay_2004, Thapa_2019f, Trgo_2012f, Tupesis_2004f, Wildhaber_2005, Wnęk_2015f, Yasin_2009, Yildiz_2016e}, 44 cases (61\%) underwent surgery \cite{Al-Faham_2020k, AlShaaibi_2021b, Alao_2006i, Ali_2017, Ali_2020f, Atayan_2016, Beecroft_1998, Bhasin_2014, CamachoDorado_2018, Cauchi_2002, Chang_2017f, Cox_2007, Csaky_1998e, DelgadoSalazar_2020c, DivsalarP._2023a, Farhadi_2024h, Fry_2010, Gardner_2017h, Jin_2023, Kariholu_2008, Kerestes_2019, Kobiela_2015, Kumar_2019f, Liu_2005, Losanoff_1996, Losanoff_1997e, Mesfin_2022a, Misra_2013, Naji_2012f, Sobnach_2011f, Tanrikulu_2015e, Tay_2004, Thapa_2019f, Tupesis_2004f, Wildhaber_2005, Wnęk_2015f, Yasin_2009, Yildiz_2016e, fjbuilsRepeatedBehaviorDeliberate2024}, 31 cases (43\%) underwent endoscopy \cite{Akay_2015f, Ali_2022g, Apikotoa_2022f, Atayan_2016, Benoist_2019e, Berry_2021e, Bhasin_2014, Bhumi_2024f, CamachoDorado_2018, Chang_2017f, DelgadoSalazar_2020c, Gardner_2017h, Guinan_2019f, Hardy_2023g, Jehangir_2019h, Kariholu_2008, Li_2013, Liu_2005, Ohno_2005, Peixoto_2017f, Qureshi_2016, Riva_2018j, Sakellaridis_2008f, Sultan_2024f, Tammana_2012j, Tanrikulu_2015e, Trgo_2012f, Wadhwa_2015e, Wnęk_2015f, teWildt_2010}, 7 cases (10\%) were managed conservatively \cite{Ataya_2013, Bhattacharjee_2008, DivsalarP._2023a, Emamhadi_2018, Goldman_1998f, Kar_2015, Kumar_2001}, 2 cases (3\%) died \cite{Emamhadi_2018, Kumar_2001}. All 90 were male gender. 90 cases (100\%) were detained at the time of ingestion \cite{Elghali_2016, Karp_1991b, Lee_2007}, 88 cases (98\%) were intentional ingestions \cite{Elghali_2016, Karp_1991b, Lee_2007}, 30 cases (33\%) had a psychiatric history documented \cite{Elghali_2016, Karp_1991b, Lee_2007}, 2 cases (2\%) had a history of prior ingestion \cite{Elghali_2016}. No cases were reported for were psychiatric inpatients, were displaced people, were under the influence of alcohol at the time of ingestion, and had a severe disability history.
\paragraph*{Motivation}  70 cases (78\%) reported protest motivation \cite{Elghali_2016, Karp_1991b, Lee_2007}, 12 cases (13\%) reported psychiatric motivation \cite{Karp_1991b}, 6 cases (7\%) reported self-harm motivation \cite{Elghali_2016, Karp_1991b}. No cases were reported for psychosocial motivation and other motivation.
\paragraph*{Object Characteristics}  68 cases (76\%) involved sharp object ingestion \cite{Elghali_2016, Karp_1991b, Lee_2007}, 32 cases (36\%) involved long (\textgreater 5cm) object ingestion \cite{Lee_2007}, 25 cases (28\%) involved ingestion of multiple objects \cite{Elghali_2016, Lee_2007}. No cases were reported for button battery ingestion, magnet ingestion, and involved large diameter (\textgreater 2.5cm) object ingestion.
\paragraph*{Outcomes}  47 cases (52\%) underwent endoscopic intervention \cite{Elghali_2016, Lee_2007}, 29 cases (32\%) were managed conservatively \cite{Elghali_2016, Karp_1991b}, 15 cases (17\%) underwent surgical intervention \cite{Elghali_2016, Karp_1991b, Lee_2007}, 6 cases (7\%) reported complications \cite{Lee_2007}, 1 case (1\%) died \cite{Elghali_2016}.
\paragraph*{Geographical Location}Cases were recorded in 33 countries: 13 cases from USA \cite{Alao_2006i, Ataya_2013, Bhumi_2024f, Fry_2010, Guinan_2019f, Hardy_2023g, Jehangir_2019h, Kerestes_2019, Kumar_2001, Liu_2005, Tammana_2012j, Tay_2004, Tupesis_2004f}; 7 cases from India \cite{Bhasin_2014, Bhattacharjee_2008, Kar_2015, Kariholu_2008, Kumar_2019f, Misra_2013, Wadhwa_2015e} and UK \cite{Beecroft_1998, Berry_2021e, Cauchi_2002, Cox_2007, Gardner_2017h, Qureshi_2016}; 6 cases from Bulgaria \cite{Losanoff_1996, Losanoff_1997e}; 5 cases from Iran \cite{DivsalarP._2023a, Emamhadi_2018, Farhadi_2024h}; 4 cases from Turkey \cite{Akay_2015f, Atayan_2016, Tanrikulu_2015e, Yildiz_2016e}; 2 cases from China \cite{Jin_2023, Li_2013}, Poland \cite{Kobiela_2015, Wnęk_2015f}, and Spain \cite{CamachoDorado_2018, fjbuilsRepeatedBehaviorDeliberate2024}; 1 case from Australia \cite{Apikotoa_2022f}, Bahrain \cite{Ali_2020f}, Croatia \cite{Trgo_2012f}, Ecuador \cite{DelgadoSalazar_2020c}, Egypt \cite{Ali_2022g}, Ethiopia \cite{Mesfin_2022a}, Germany \cite{teWildt_2010}, Greece \cite{Sakellaridis_2008f}, Hungary \cite{Csaky_1998e}, Iraq \cite{Al-Faham_2020k}, Israel \cite{Goldman_1998f}, Italy \cite{Riva_2018j}, Japan \cite{Ohno_2005}, Nepal \cite{Thapa_2019f}, Netherlands \cite{Benoist_2019e}, Oman \cite{AlShaaibi_2021b}, Pakistan \cite{Yasin_2009}, Portugal \cite{Peixoto_2017f}, Qatar \cite{Ali_2017}, Saudi Arabia \cite{Sultan_2024f}, South Africa \cite{Sobnach_2011f}, Sweden \cite{Naji_2012f}, Switzerland \cite{Wildhaber_2005}, and Taiwan \cite{Chang_2017f}. \paragraph*{Gender} 43 cases (60\%) were male \cite{Akay_2015f, Al-Faham_2020k, Alao_2006i, Ali_2017, Ali_2022g, Apikotoa_2022f, Atayan_2016, Benoist_2019e, Berry_2021e, Bhumi_2024f, CamachoDorado_2018, Csaky_1998e, Emamhadi_2018, Farhadi_2024h, Fry_2010, Gardner_2017h, Guinan_2019f, Jehangir_2019h, Jin_2023, Kobiela_2015, Kumar_2001, Kumar_2019f, Liu_2005, Losanoff_1996, Losanoff_1997e, Mesfin_2022a, Misra_2013, Qureshi_2016, Riva_2018j, Sobnach_2011f, Tammana_2012j, Tanrikulu_2015e, Tay_2004, Thapa_2019f, Trgo_2012f, Wadhwa_2015e, Yasin_2009, teWildt_2010}, 28 cases (39\%) were female \cite{AlShaaibi_2021b, Ali_2020f, Ataya_2013, Beecroft_1998, Bhasin_2014, Bhattacharjee_2008, Cauchi_2002, Chang_2017f, Cox_2007, DelgadoSalazar_2020c, DivsalarP._2023a, Goldman_1998f, Hardy_2023g, Kar_2015, Kariholu_2008, Kerestes_2019, Li_2013, Naji_2012f, Ohno_2005, Peixoto_2017f, Sakellaridis_2008f, Sultan_2024f, Tupesis_2004f, Wildhaber_2005, Wnęk_2015f, Yildiz_2016e}, 1 case (1\%) had no gender recorded \cite{fjbuilsRepeatedBehaviorDeliberate2024}. \paragraph*{Age Group} 25 cases (35\%) were between 26 and 40 years of age \cite{Alao_2006i, Ali_2022g, Apikotoa_2022f, Ataya_2013, Benoist_2019e, Bhasin_2014, Chang_2017f, Cox_2007, DelgadoSalazar_2020c, Farhadi_2024h, Fry_2010, Gardner_2017h, Guinan_2019f, Jin_2023, Kumar_2019f, Losanoff_1996, Misra_2013, Qureshi_2016, Riva_2018j, Sakellaridis_2008f, Tammana_2012j, Trgo_2012f, Wnęk_2015f, Yildiz_2016e, fjbuilsRepeatedBehaviorDeliberate2024}, 18 cases (25\%) were between 18 and 25 years of age \cite{Akay_2015f, Ali_2017, Atayan_2016, Bhattacharjee_2008, Csaky_1998e, Kar_2015, Kariholu_2008, Kobiela_2015, Losanoff_1996, Losanoff_1997e, Mesfin_2022a, Peixoto_2017f, Sobnach_2011f, Tupesis_2004f, Yasin_2009}, 13 cases (18\%) were under 18 years of age \cite{AlShaaibi_2021b, Ali_2020f, Cauchi_2002, DivsalarP._2023a, Goldman_1998f, Liu_2005, Naji_2012f, Ohno_2005, Tanrikulu_2015e, Tay_2004, Wildhaber_2005}, 11 cases (15\%) were between 41 and 60 years of age \cite{Al-Faham_2020k, Bhumi_2024f, CamachoDorado_2018, Emamhadi_2018, Hardy_2023g, Jehangir_2019h, Kumar_2001, Sultan_2024f, Thapa_2019f, Wadhwa_2015e, teWildt_2010}, 3 cases (4\%) were over 60 years of age \cite{Beecroft_1998, Kerestes_2019, Li_2013}, 2 cases (3\%) had no age documented \cite{Berry_2021e}. \paragraph*{Population} 36 cases (50\%) had a psychiatric history \cite{AlShaaibi_2021b, Alao_2006i, Ali_2020f, Apikotoa_2022f, Ataya_2013, Atayan_2016, Beecroft_1998, CamachoDorado_2018, Chang_2017f, DelgadoSalazar_2020c, DivsalarP._2023a, Farhadi_2024h, Fry_2010, Guinan_2019f, Hardy_2023g, Jehangir_2019h, Jin_2023, Kar_2015, Kerestes_2019, Kobiela_2015, Kumar_2001, Kumar_2019f, Liu_2005, Mesfin_2022a, Misra_2013, Ohno_2005, Peixoto_2017f, Sakellaridis_2008f, Sultan_2024f, Tammana_2012j, Tanrikulu_2015e, Yildiz_2016e, fjbuilsRepeatedBehaviorDeliberate2024, teWildt_2010}, 19 cases (26\%) had ingested previously \cite{Alao_2006i, Apikotoa_2022f, Berry_2021e, Bhattacharjee_2008, Csaky_1998e, DivsalarP._2023a, Emamhadi_2018, Guinan_2019f, Jehangir_2019h, Jin_2023, Liu_2005, Sakellaridis_2008f, Tanrikulu_2015e, Thapa_2019f, Yildiz_2016e, fjbuilsRepeatedBehaviorDeliberate2024, teWildt_2010}, 12 cases (17\%) were detained persons \cite{Alao_2006i, Ali_2022g, Apikotoa_2022f, Losanoff_1996, Losanoff_1997e, Qureshi_2016, Tammana_2012j, Trgo_2012f}, 7 cases (10\%) were severely disabled \cite{Atayan_2016, Kerestes_2019, Liu_2005, Ohno_2005, Peixoto_2017f, Yildiz_2016e, teWildt_2010}, 4 cases (6\%) were psychiatric inpatients \cite{DivsalarP._2023a, fjbuilsRepeatedBehaviorDeliberate2024, teWildt_2010}, 3 cases (4\%) were under the influence of alcohol \cite{Benoist_2019e, Csaky_1998e, Thapa_2019f}, 2 cases (3\%) were displaced people \cite{Akay_2015f, Gardner_2017h}. \paragraph*{Motivation} 34 cases (47\%) had a psychiatric motivation \cite{Al-Faham_2020k, Alao_2006i, Ali_2020f, Apikotoa_2022f, Ataya_2013, Atayan_2016, Bhasin_2014, Bhattacharjee_2008, DelgadoSalazar_2020c, DivsalarP._2023a, Emamhadi_2018, Farhadi_2024h, Guinan_2019f, Hardy_2023g, Jehangir_2019h, Jin_2023, Kar_2015, Kariholu_2008, Kerestes_2019, Kobiela_2015, Kumar_2001, Kumar_2019f, Li_2013, Liu_2005, Misra_2013, Ohno_2005, Sakellaridis_2008f, Sultan_2024f, Tammana_2012j, Tanrikulu_2015e, Yasin_2009, teWildt_2010}, 21 cases (29\%) were motivated by self-harm intention \cite{Al-Faham_2020k, AlShaaibi_2021b, Alao_2006i, Ali_2017, CamachoDorado_2018, Chang_2017f, Cox_2007, Csaky_1998e, Fry_2010, Li_2013, Losanoff_1996, Losanoff_1997e, Mesfin_2022a, Sakellaridis_2008f, Tammana_2012j, Tanrikulu_2015e, fjbuilsRepeatedBehaviorDeliberate2024}, 17 cases (24\%) had a psychosocial motivation \cite{Akay_2015f, Benoist_2019e, Bhattacharjee_2008, Cauchi_2002, Goldman_1998f, Hardy_2023g, Kobiela_2015, Li_2013, Naji_2012f, Qureshi_2016, Riva_2018j, Sobnach_2011f, Tay_2004, Thapa_2019f, Tupesis_2004f, Wildhaber_2005, Wnęk_2015f}, 9 cases (12\%) were motivated by protest \cite{Bhumi_2024f, Gardner_2017h, Losanoff_1996, Losanoff_1997e, Tupesis_2004f}, 9 cases (12\%) had another documented motivation \cite{Ali_2020f, Ali_2022g, Emamhadi_2018, Guinan_2019f, Peixoto_2017f, Sakellaridis_2008f, Trgo_2012f, Wadhwa_2015e, Yildiz_2016e}. \paragraph*{Object Characteristics} 51 cases (71\%) ingested a large diameter object (\textgreater{}2.5cm) \cite{Akay_2015f, Al-Faham_2020k, AlShaaibi_2021b, Alao_2006i, Ali_2017, Ali_2022g, Apikotoa_2022f, Atayan_2016, Berry_2021e, Bhasin_2014, CamachoDorado_2018, Cauchi_2002, Chang_2017f, Cox_2007, Csaky_1998e, DivsalarP._2023a, Emamhadi_2018, Gardner_2017h, Guinan_2019f, Jehangir_2019h, Jin_2023, Kariholu_2008, Kerestes_2019, Kobiela_2015, Kumar_2001, Kumar_2019f, Losanoff_1996, Losanoff_1997e, Mesfin_2022a, Misra_2013, Naji_2012f, Ohno_2005, Peixoto_2017f, Qureshi_2016, Riva_2018j, Sakellaridis_2008f, Sultan_2024f, Tanrikulu_2015e, Thapa_2019f, Trgo_2012f, Wnęk_2015f, Yildiz_2016e, fjbuilsRepeatedBehaviorDeliberate2024, teWildt_2010}, 44 cases (61\%) ingested multiple objects \cite{Ali_2020f, Apikotoa_2022f, Ataya_2013, Atayan_2016, Beecroft_1998, Bhattacharjee_2008, Bhumi_2024f, CamachoDorado_2018, Cauchi_2002, Emamhadi_2018, Farhadi_2024h, Fry_2010, Goldman_1998f, Guinan_2019f, Hardy_2023g, Jehangir_2019h, Jin_2023, Kar_2015, Kariholu_2008, Kobiela_2015, Kumar_2001, Kumar_2019f, Li_2013, Liu_2005, Losanoff_1996, Mesfin_2022a, Misra_2013, Naji_2012f, Ohno_2005, Sobnach_2011f, Sultan_2024f, Tammana_2012j, Tanrikulu_2015e, Tay_2004, Thapa_2019f, Wadhwa_2015e, Wildhaber_2005, Yasin_2009, fjbuilsRepeatedBehaviorDeliberate2024, teWildt_2010}, 34 cases (47\%) ingested a sharp object \cite{AlShaaibi_2021b, Alao_2006i, Apikotoa_2022f, Ataya_2013, Benoist_2019e, Bhasin_2014, Bhattacharjee_2008, CamachoDorado_2018, Csaky_1998e, DelgadoSalazar_2020c, DivsalarP._2023a, Emamhadi_2018, Farhadi_2024h, Fry_2010, Guinan_2019f, Hardy_2023g, Jehangir_2019h, Jin_2023, Kariholu_2008, Kobiela_2015, Kumar_2019f, Losanoff_1996, Losanoff_1997e, Mesfin_2022a, Misra_2013, Sobnach_2011f, Yasin_2009, teWildt_2010}, 32 cases (44\%) ingested a long object (\textgreater{}5cm) \cite{Al-Faham_2020k, AlShaaibi_2021b, Ali_2017, Ali_2022g, Atayan_2016, Bhasin_2014, CamachoDorado_2018, Chang_2017f, Cox_2007, Csaky_1998e, DivsalarP._2023a, Emamhadi_2018, Fry_2010, Gardner_2017h, Jin_2023, Kariholu_2008, Kerestes_2019, Kobiela_2015, Kumar_2019f, Mesfin_2022a, Misra_2013, Ohno_2005, Qureshi_2016, Sakellaridis_2008f, Sultan_2024f, Thapa_2019f, Trgo_2012f, Yasin_2009, Yildiz_2016e, teWildt_2010}, 9 cases (12\%) ingested a magnet \cite{Ali_2020f, Bhumi_2024f, Cauchi_2002, Liu_2005, Naji_2012f, Ohno_2005, Tanrikulu_2015e, Tay_2004, Wildhaber_2005}, 2 cases (3\%) ingested a button battery \cite{Berry_2021e, Bhumi_2024f}. \paragraph*{Outcomes} 48 cases (67\%) experienced a complication \cite{Ali_2017, Ali_2020f, Apikotoa_2022f, Atayan_2016, Beecroft_1998, Benoist_2019e, Berry_2021e, Bhasin_2014, Bhumi_2024f, CamachoDorado_2018, Cauchi_2002, Cox_2007, Csaky_1998e, DelgadoSalazar_2020c, DivsalarP._2023a, Emamhadi_2018, Farhadi_2024h, Fry_2010, Gardner_2017h, Goldman_1998f, Jin_2023, Kariholu_2008, Kerestes_2019, Kobiela_2015, Kumar_2001, Kumar_2019f, Liu_2005, Losanoff_1996, Mesfin_2022a, Misra_2013, Naji_2012f, Ohno_2005, Sakellaridis_2008f, Sobnach_2011f, Sultan_2024f, Tanrikulu_2015e, Tay_2004, Thapa_2019f, Trgo_2012f, Tupesis_2004f, Wildhaber_2005, Wnęk_2015f, Yasin_2009, Yildiz_2016e}, 44 cases (61\%) underwent surgery \cite{Al-Faham_2020k, AlShaaibi_2021b, Alao_2006i, Ali_2017, Ali_2020f, Atayan_2016, Beecroft_1998, Bhasin_2014, CamachoDorado_2018, Cauchi_2002, Chang_2017f, Cox_2007, Csaky_1998e, DelgadoSalazar_2020c, DivsalarP._2023a, Farhadi_2024h, Fry_2010, Gardner_2017h, Jin_2023, Kariholu_2008, Kerestes_2019, Kobiela_2015, Kumar_2019f, Liu_2005, Losanoff_1996, Losanoff_1997e, Mesfin_2022a, Misra_2013, Naji_2012f, Sobnach_2011f, Tanrikulu_2015e, Tay_2004, Thapa_2019f, Tupesis_2004f, Wildhaber_2005, Wnęk_2015f, Yasin_2009, Yildiz_2016e, fjbuilsRepeatedBehaviorDeliberate2024}, 31 cases (43\%) underwent endoscopy \cite{Akay_2015f, Ali_2022g, Apikotoa_2022f, Atayan_2016, Benoist_2019e, Berry_2021e, Bhasin_2014, Bhumi_2024f, CamachoDorado_2018, Chang_2017f, DelgadoSalazar_2020c, Gardner_2017h, Guinan_2019f, Hardy_2023g, Jehangir_2019h, Kariholu_2008, Li_2013, Liu_2005, Ohno_2005, Peixoto_2017f, Qureshi_2016, Riva_2018j, Sakellaridis_2008f, Sultan_2024f, Tammana_2012j, Tanrikulu_2015e, Trgo_2012f, Wadhwa_2015e, Wnęk_2015f, teWildt_2010}, 7 cases (10\%) were managed conservatively \cite{Ataya_2013, Bhattacharjee_2008, DivsalarP._2023a, Emamhadi_2018, Goldman_1998f, Kar_2015, Kumar_2001}, 2 cases (3\%) died \cite{Emamhadi_2018, Kumar_2001}. All 90 were male gender. 90 cases (100\%) were detained at the time of ingestion \cite{Elghali_2016, Karp_1991b, Lee_2007}, 88 cases (98\%) were intentional ingestions \cite{Elghali_2016, Karp_1991b, Lee_2007}, 30 cases (33\%) had a psychiatric history documented \cite{Elghali_2016, Karp_1991b, Lee_2007}, 2 cases (2\%) had a history of prior ingestion \cite{Elghali_2016}. No cases were reported for were psychiatric inpatients, were displaced people, were under the influence of alcohol at the time of ingestion, and had a severe disability history.
\paragraph*{Motivation}  70 cases (78\%) reported protest motivation \cite{Elghali_2016, Karp_1991b, Lee_2007}, 12 cases (13\%) reported psychiatric motivation \cite{Karp_1991b}, 6 cases (7\%) reported self-harm motivation \cite{Elghali_2016, Karp_1991b}. No cases were reported for psychosocial motivation and other motivation.
\paragraph*{Object Characteristics}  68 cases (76\%) involved sharp object ingestion \cite{Elghali_2016, Karp_1991b, Lee_2007}, 32 cases (36\%) involved long (\textgreater 5cm) object ingestion \cite{Lee_2007}, 25 cases (28\%) involved ingestion of multiple objects \cite{Elghali_2016, Lee_2007}. No cases were reported for button battery ingestion, magnet ingestion, and involved large diameter (\textgreater 2.5cm) object ingestion.
\paragraph*{Outcomes}  47 cases (52\%) underwent endoscopic intervention \cite{Elghali_2016, Lee_2007}, 29 cases (32\%) were managed conservatively \cite{Elghali_2016, Karp_1991b}, 15 cases (17\%) underwent surgical intervention \cite{Elghali_2016, Karp_1991b, Lee_2007}, 6 cases (7\%) reported complications \cite{Lee_2007}, 1 case (1\%) died \cite{Elghali_2016}.
\paragraph*{Geographical Location}Cases were recorded in 33 countries: 13 cases from USA \cite{Alao_2006i, Ataya_2013, Bhumi_2024f, Fry_2010, Guinan_2019f, Hardy_2023g, Jehangir_2019h, Kerestes_2019, Kumar_2001, Liu_2005, Tammana_2012j, Tay_2004, Tupesis_2004f}; 7 cases from India \cite{Bhasin_2014, Bhattacharjee_2008, Kar_2015, Kariholu_2008, Kumar_2019f, Misra_2013, Wadhwa_2015e} and UK \cite{Beecroft_1998, Berry_2021e, Cauchi_2002, Cox_2007, Gardner_2017h, Qureshi_2016}; 6 cases from Bulgaria \cite{Losanoff_1996, Losanoff_1997e}; 5 cases from Iran \cite{DivsalarP._2023a, Emamhadi_2018, Farhadi_2024h}; 4 cases from Turkey \cite{Akay_2015f, Atayan_2016, Tanrikulu_2015e, Yildiz_2016e}; 2 cases from China \cite{Jin_2023, Li_2013}, Poland \cite{Kobiela_2015, Wnęk_2015f}, and Spain \cite{CamachoDorado_2018, fjbuilsRepeatedBehaviorDeliberate2024}; 1 case from Australia \cite{Apikotoa_2022f}, Bahrain \cite{Ali_2020f}, Croatia \cite{Trgo_2012f}, Ecuador \cite{DelgadoSalazar_2020c}, Egypt \cite{Ali_2022g}, Ethiopia \cite{Mesfin_2022a}, Germany \cite{teWildt_2010}, Greece \cite{Sakellaridis_2008f}, Hungary \cite{Csaky_1998e}, Iraq \cite{Al-Faham_2020k}, Israel \cite{Goldman_1998f}, Italy \cite{Riva_2018j}, Japan \cite{Ohno_2005}, Nepal \cite{Thapa_2019f}, Netherlands \cite{Benoist_2019e}, Oman \cite{AlShaaibi_2021b}, Pakistan \cite{Yasin_2009}, Portugal \cite{Peixoto_2017f}, Qatar \cite{Ali_2017}, Saudi Arabia \cite{Sultan_2024f}, South Africa \cite{Sobnach_2011f}, Sweden \cite{Naji_2012f}, Switzerland \cite{Wildhaber_2005}, and Taiwan \cite{Chang_2017f}. \paragraph*{Gender} 43 cases (60\%) were male \cite{Akay_2015f, Al-Faham_2020k, Alao_2006i, Ali_2017, Ali_2022g, Apikotoa_2022f, Atayan_2016, Benoist_2019e, Berry_2021e, Bhumi_2024f, CamachoDorado_2018, Csaky_1998e, Emamhadi_2018, Farhadi_2024h, Fry_2010, Gardner_2017h, Guinan_2019f, Jehangir_2019h, Jin_2023, Kobiela_2015, Kumar_2001, Kumar_2019f, Liu_2005, Losanoff_1996, Losanoff_1997e, Mesfin_2022a, Misra_2013, Qureshi_2016, Riva_2018j, Sobnach_2011f, Tammana_2012j, Tanrikulu_2015e, Tay_2004, Thapa_2019f, Trgo_2012f, Wadhwa_2015e, Yasin_2009, teWildt_2010}, 28 cases (39\%) were female \cite{AlShaaibi_2021b, Ali_2020f, Ataya_2013, Beecroft_1998, Bhasin_2014, Bhattacharjee_2008, Cauchi_2002, Chang_2017f, Cox_2007, DelgadoSalazar_2020c, DivsalarP._2023a, Goldman_1998f, Hardy_2023g, Kar_2015, Kariholu_2008, Kerestes_2019, Li_2013, Naji_2012f, Ohno_2005, Peixoto_2017f, Sakellaridis_2008f, Sultan_2024f, Tupesis_2004f, Wildhaber_2005, Wnęk_2015f, Yildiz_2016e}, 1 case (1\%) had no gender recorded \cite{fjbuilsRepeatedBehaviorDeliberate2024}. \paragraph*{Age Group} 25 cases (35\%) were between 26 and 40 years of age \cite{Alao_2006i, Ali_2022g, Apikotoa_2022f, Ataya_2013, Benoist_2019e, Bhasin_2014, Chang_2017f, Cox_2007, DelgadoSalazar_2020c, Farhadi_2024h, Fry_2010, Gardner_2017h, Guinan_2019f, Jin_2023, Kumar_2019f, Losanoff_1996, Misra_2013, Qureshi_2016, Riva_2018j, Sakellaridis_2008f, Tammana_2012j, Trgo_2012f, Wnęk_2015f, Yildiz_2016e, fjbuilsRepeatedBehaviorDeliberate2024}, 18 cases (25\%) were between 18 and 25 years of age \cite{Akay_2015f, Ali_2017, Atayan_2016, Bhattacharjee_2008, Csaky_1998e, Kar_2015, Kariholu_2008, Kobiela_2015, Losanoff_1996, Losanoff_1997e, Mesfin_2022a, Peixoto_2017f, Sobnach_2011f, Tupesis_2004f, Yasin_2009}, 13 cases (18\%) were under 18 years of age \cite{AlShaaibi_2021b, Ali_2020f, Cauchi_2002, DivsalarP._2023a, Goldman_1998f, Liu_2005, Naji_2012f, Ohno_2005, Tanrikulu_2015e, Tay_2004, Wildhaber_2005}, 11 cases (15\%) were between 41 and 60 years of age \cite{Al-Faham_2020k, Bhumi_2024f, CamachoDorado_2018, Emamhadi_2018, Hardy_2023g, Jehangir_2019h, Kumar_2001, Sultan_2024f, Thapa_2019f, Wadhwa_2015e, teWildt_2010}, 3 cases (4\%) were over 60 years of age \cite{Beecroft_1998, Kerestes_2019, Li_2013}, 2 cases (3\%) had no age documented \cite{Berry_2021e}. \paragraph*{Population} 36 cases (50\%) had a psychiatric history \cite{AlShaaibi_2021b, Alao_2006i, Ali_2020f, Apikotoa_2022f, Ataya_2013, Atayan_2016, Beecroft_1998, CamachoDorado_2018, Chang_2017f, DelgadoSalazar_2020c, DivsalarP._2023a, Farhadi_2024h, Fry_2010, Guinan_2019f, Hardy_2023g, Jehangir_2019h, Jin_2023, Kar_2015, Kerestes_2019, Kobiela_2015, Kumar_2001, Kumar_2019f, Liu_2005, Mesfin_2022a, Misra_2013, Ohno_2005, Peixoto_2017f, Sakellaridis_2008f, Sultan_2024f, Tammana_2012j, Tanrikulu_2015e, Yildiz_2016e, fjbuilsRepeatedBehaviorDeliberate2024, teWildt_2010}, 19 cases (26\%) had ingested previously \cite{Alao_2006i, Apikotoa_2022f, Berry_2021e, Bhattacharjee_2008, Csaky_1998e, DivsalarP._2023a, Emamhadi_2018, Guinan_2019f, Jehangir_2019h, Jin_2023, Liu_2005, Sakellaridis_2008f, Tanrikulu_2015e, Thapa_2019f, Yildiz_2016e, fjbuilsRepeatedBehaviorDeliberate2024, teWildt_2010}, 12 cases (17\%) were detained persons \cite{Alao_2006i, Ali_2022g, Apikotoa_2022f, Losanoff_1996, Losanoff_1997e, Qureshi_2016, Tammana_2012j, Trgo_2012f}, 7 cases (10\%) were severely disabled \cite{Atayan_2016, Kerestes_2019, Liu_2005, Ohno_2005, Peixoto_2017f, Yildiz_2016e, teWildt_2010}, 4 cases (6\%) were psychiatric inpatients \cite{DivsalarP._2023a, fjbuilsRepeatedBehaviorDeliberate2024, teWildt_2010}, 3 cases (4\%) were under the influence of alcohol \cite{Benoist_2019e, Csaky_1998e, Thapa_2019f}, 2 cases (3\%) were displaced people \cite{Akay_2015f, Gardner_2017h}. \paragraph*{Motivation} 34 cases (47\%) had a psychiatric motivation \cite{Al-Faham_2020k, Alao_2006i, Ali_2020f, Apikotoa_2022f, Ataya_2013, Atayan_2016, Bhasin_2014, Bhattacharjee_2008, DelgadoSalazar_2020c, DivsalarP._2023a, Emamhadi_2018, Farhadi_2024h, Guinan_2019f, Hardy_2023g, Jehangir_2019h, Jin_2023, Kar_2015, Kariholu_2008, Kerestes_2019, Kobiela_2015, Kumar_2001, Kumar_2019f, Li_2013, Liu_2005, Misra_2013, Ohno_2005, Sakellaridis_2008f, Sultan_2024f, Tammana_2012j, Tanrikulu_2015e, Yasin_2009, teWildt_2010}, 21 cases (29\%) were motivated by self-harm intention \cite{Al-Faham_2020k, AlShaaibi_2021b, Alao_2006i, Ali_2017, CamachoDorado_2018, Chang_2017f, Cox_2007, Csaky_1998e, Fry_2010, Li_2013, Losanoff_1996, Losanoff_1997e, Mesfin_2022a, Sakellaridis_2008f, Tammana_2012j, Tanrikulu_2015e, fjbuilsRepeatedBehaviorDeliberate2024}, 17 cases (24\%) had a psychosocial motivation \cite{Akay_2015f, Benoist_2019e, Bhattacharjee_2008, Cauchi_2002, Goldman_1998f, Hardy_2023g, Kobiela_2015, Li_2013, Naji_2012f, Qureshi_2016, Riva_2018j, Sobnach_2011f, Tay_2004, Thapa_2019f, Tupesis_2004f, Wildhaber_2005, Wnęk_2015f}, 9 cases (12\%) were motivated by protest \cite{Bhumi_2024f, Gardner_2017h, Losanoff_1996, Losanoff_1997e, Tupesis_2004f}, 9 cases (12\%) had another documented motivation \cite{Ali_2020f, Ali_2022g, Emamhadi_2018, Guinan_2019f, Peixoto_2017f, Sakellaridis_2008f, Trgo_2012f, Wadhwa_2015e, Yildiz_2016e}. \paragraph*{Object Characteristics} 51 cases (71\%) ingested a large diameter object (\textgreater{}2.5cm) \cite{Akay_2015f, Al-Faham_2020k, AlShaaibi_2021b, Alao_2006i, Ali_2017, Ali_2022g, Apikotoa_2022f, Atayan_2016, Berry_2021e, Bhasin_2014, CamachoDorado_2018, Cauchi_2002, Chang_2017f, Cox_2007, Csaky_1998e, DivsalarP._2023a, Emamhadi_2018, Gardner_2017h, Guinan_2019f, Jehangir_2019h, Jin_2023, Kariholu_2008, Kerestes_2019, Kobiela_2015, Kumar_2001, Kumar_2019f, Losanoff_1996, Losanoff_1997e, Mesfin_2022a, Misra_2013, Naji_2012f, Ohno_2005, Peixoto_2017f, Qureshi_2016, Riva_2018j, Sakellaridis_2008f, Sultan_2024f, Tanrikulu_2015e, Thapa_2019f, Trgo_2012f, Wnęk_2015f, Yildiz_2016e, fjbuilsRepeatedBehaviorDeliberate2024, teWildt_2010}, 44 cases (61\%) ingested multiple objects \cite{Ali_2020f, Apikotoa_2022f, Ataya_2013, Atayan_2016, Beecroft_1998, Bhattacharjee_2008, Bhumi_2024f, CamachoDorado_2018, Cauchi_2002, Emamhadi_2018, Farhadi_2024h, Fry_2010, Goldman_1998f, Guinan_2019f, Hardy_2023g, Jehangir_2019h, Jin_2023, Kar_2015, Kariholu_2008, Kobiela_2015, Kumar_2001, Kumar_2019f, Li_2013, Liu_2005, Losanoff_1996, Mesfin_2022a, Misra_2013, Naji_2012f, Ohno_2005, Sobnach_2011f, Sultan_2024f, Tammana_2012j, Tanrikulu_2015e, Tay_2004, Thapa_2019f, Wadhwa_2015e, Wildhaber_2005, Yasin_2009, fjbuilsRepeatedBehaviorDeliberate2024, teWildt_2010}, 34 cases (47\%) ingested a sharp object \cite{AlShaaibi_2021b, Alao_2006i, Apikotoa_2022f, Ataya_2013, Benoist_2019e, Bhasin_2014, Bhattacharjee_2008, CamachoDorado_2018, Csaky_1998e, DelgadoSalazar_2020c, DivsalarP._2023a, Emamhadi_2018, Farhadi_2024h, Fry_2010, Guinan_2019f, Hardy_2023g, Jehangir_2019h, Jin_2023, Kariholu_2008, Kobiela_2015, Kumar_2019f, Losanoff_1996, Losanoff_1997e, Mesfin_2022a, Misra_2013, Sobnach_2011f, Yasin_2009, teWildt_2010}, 32 cases (44\%) ingested a long object (\textgreater{}5cm) \cite{Al-Faham_2020k, AlShaaibi_2021b, Ali_2017, Ali_2022g, Atayan_2016, Bhasin_2014, CamachoDorado_2018, Chang_2017f, Cox_2007, Csaky_1998e, DivsalarP._2023a, Emamhadi_2018, Fry_2010, Gardner_2017h, Jin_2023, Kariholu_2008, Kerestes_2019, Kobiela_2015, Kumar_2019f, Mesfin_2022a, Misra_2013, Ohno_2005, Qureshi_2016, Sakellaridis_2008f, Sultan_2024f, Thapa_2019f, Trgo_2012f, Yasin_2009, Yildiz_2016e, teWildt_2010}, 9 cases (12\%) ingested a magnet \cite{Ali_2020f, Bhumi_2024f, Cauchi_2002, Liu_2005, Naji_2012f, Ohno_2005, Tanrikulu_2015e, Tay_2004, Wildhaber_2005}, 2 cases (3\%) ingested a button battery \cite{Berry_2021e, Bhumi_2024f}. \paragraph*{Outcomes} 48 cases (67\%) experienced a complication \cite{Ali_2017, Ali_2020f, Apikotoa_2022f, Atayan_2016, Beecroft_1998, Benoist_2019e, Berry_2021e, Bhasin_2014, Bhumi_2024f, CamachoDorado_2018, Cauchi_2002, Cox_2007, Csaky_1998e, DelgadoSalazar_2020c, DivsalarP._2023a, Emamhadi_2018, Farhadi_2024h, Fry_2010, Gardner_2017h, Goldman_1998f, Jin_2023, Kariholu_2008, Kerestes_2019, Kobiela_2015, Kumar_2001, Kumar_2019f, Liu_2005, Losanoff_1996, Mesfin_2022a, Misra_2013, Naji_2012f, Ohno_2005, Sakellaridis_2008f, Sobnach_2011f, Sultan_2024f, Tanrikulu_2015e, Tay_2004, Thapa_2019f, Trgo_2012f, Tupesis_2004f, Wildhaber_2005, Wnęk_2015f, Yasin_2009, Yildiz_2016e}, 44 cases (61\%) underwent surgery \cite{Al-Faham_2020k, AlShaaibi_2021b, Alao_2006i, Ali_2017, Ali_2020f, Atayan_2016, Beecroft_1998, Bhasin_2014, CamachoDorado_2018, Cauchi_2002, Chang_2017f, Cox_2007, Csaky_1998e, DelgadoSalazar_2020c, DivsalarP._2023a, Farhadi_2024h, Fry_2010, Gardner_2017h, Jin_2023, Kariholu_2008, Kerestes_2019, Kobiela_2015, Kumar_2019f, Liu_2005, Losanoff_1996, Losanoff_1997e, Mesfin_2022a, Misra_2013, Naji_2012f, Sobnach_2011f, Tanrikulu_2015e, Tay_2004, Thapa_2019f, Tupesis_2004f, Wildhaber_2005, Wnęk_2015f, Yasin_2009, Yildiz_2016e, fjbuilsRepeatedBehaviorDeliberate2024}, 31 cases (43\%) underwent endoscopy \cite{Akay_2015f, Ali_2022g, Apikotoa_2022f, Atayan_2016, Benoist_2019e, Berry_2021e, Bhasin_2014, Bhumi_2024f, CamachoDorado_2018, Chang_2017f, DelgadoSalazar_2020c, Gardner_2017h, Guinan_2019f, Hardy_2023g, Jehangir_2019h, Kariholu_2008, Li_2013, Liu_2005, Ohno_2005, Peixoto_2017f, Qureshi_2016, Riva_2018j, Sakellaridis_2008f, Sultan_2024f, Tammana_2012j, Tanrikulu_2015e, Trgo_2012f, Wadhwa_2015e, Wnęk_2015f, teWildt_2010}, 7 cases (10\%) were managed conservatively \cite{Ataya_2013, Bhattacharjee_2008, DivsalarP._2023a, Emamhadi_2018, Goldman_1998f, Kar_2015, Kumar_2001}, 2 cases (3\%) died \cite{Emamhadi_2018, Kumar_2001}. All 90 were male gender. 90 cases (100\%) were detained at the time of ingestion \cite{Elghali_2016, Karp_1991b, Lee_2007}, 88 cases (98\%) were intentional ingestions \cite{Elghali_2016, Karp_1991b, Lee_2007}, 30 cases (33\%) had a psychiatric history documented \cite{Elghali_2016, Karp_1991b, Lee_2007}, 2 cases (2\%) had a history of prior ingestion \cite{Elghali_2016}. No cases were reported for were psychiatric inpatients, were displaced people, were under the influence of alcohol at the time of ingestion, and had a severe disability history.
\paragraph*{Motivation}  70 cases (78\%) reported protest motivation \cite{Elghali_2016, Karp_1991b, Lee_2007}, 12 cases (13\%) reported psychiatric motivation \cite{Karp_1991b}, 6 cases (7\%) reported self-harm motivation \cite{Elghali_2016, Karp_1991b}. No cases were reported for psychosocial motivation and other motivation.
\paragraph*{Object Characteristics}  68 cases (76\%) involved sharp object ingestion \cite{Elghali_2016, Karp_1991b, Lee_2007}, 32 cases (36\%) involved long (\textgreater 5cm) object ingestion \cite{Lee_2007}, 25 cases (28\%) involved ingestion of multiple objects \cite{Elghali_2016, Lee_2007}. No cases were reported for button battery ingestion, magnet ingestion, and involved large diameter (\textgreater 2.5cm) object ingestion.
\paragraph*{Outcomes}  47 cases (52\%) underwent endoscopic intervention \cite{Elghali_2016, Lee_2007}, 29 cases (32\%) were managed conservatively \cite{Elghali_2016, Karp_1991b}, 15 cases (17\%) underwent surgical intervention \cite{Elghali_2016, Karp_1991b, Lee_2007}, 6 cases (7\%) reported complications \cite{Lee_2007}, 1 case (1\%) died \cite{Elghali_2016}.
\paragraph*{Geographical Location}Cases were recorded in 33 countries: 13 cases from USA \cite{Alao_2006i, Ataya_2013, Bhumi_2024f, Fry_2010, Guinan_2019f, Hardy_2023g, Jehangir_2019h, Kerestes_2019, Kumar_2001, Liu_2005, Tammana_2012j, Tay_2004, Tupesis_2004f}; 7 cases from India \cite{Bhasin_2014, Bhattacharjee_2008, Kar_2015, Kariholu_2008, Kumar_2019f, Misra_2013, Wadhwa_2015e} and UK \cite{Beecroft_1998, Berry_2021e, Cauchi_2002, Cox_2007, Gardner_2017h, Qureshi_2016}; 6 cases from Bulgaria \cite{Losanoff_1996, Losanoff_1997e}; 5 cases from Iran \cite{DivsalarP._2023a, Emamhadi_2018, Farhadi_2024h}; 4 cases from Turkey \cite{Akay_2015f, Atayan_2016, Tanrikulu_2015e, Yildiz_2016e}; 2 cases from China \cite{Jin_2023, Li_2013}, Poland \cite{Kobiela_2015, Wnęk_2015f}, and Spain \cite{CamachoDorado_2018, fjbuilsRepeatedBehaviorDeliberate2024}; 1 case from Australia \cite{Apikotoa_2022f}, Bahrain \cite{Ali_2020f}, Croatia \cite{Trgo_2012f}, Ecuador \cite{DelgadoSalazar_2020c}, Egypt \cite{Ali_2022g}, Ethiopia \cite{Mesfin_2022a}, Germany \cite{teWildt_2010}, Greece \cite{Sakellaridis_2008f}, Hungary \cite{Csaky_1998e}, Iraq \cite{Al-Faham_2020k}, Israel \cite{Goldman_1998f}, Italy \cite{Riva_2018j}, Japan \cite{Ohno_2005}, Nepal \cite{Thapa_2019f}, Netherlands \cite{Benoist_2019e}, Oman \cite{AlShaaibi_2021b}, Pakistan \cite{Yasin_2009}, Portugal \cite{Peixoto_2017f}, Qatar \cite{Ali_2017}, Saudi Arabia \cite{Sultan_2024f}, South Africa \cite{Sobnach_2011f}, Sweden \cite{Naji_2012f}, Switzerland \cite{Wildhaber_2005}, and Taiwan \cite{Chang_2017f}. \paragraph*{Gender} 43 cases (60\%) were male \cite{Akay_2015f, Al-Faham_2020k, Alao_2006i, Ali_2017, Ali_2022g, Apikotoa_2022f, Atayan_2016, Benoist_2019e, Berry_2021e, Bhumi_2024f, CamachoDorado_2018, Csaky_1998e, Emamhadi_2018, Farhadi_2024h, Fry_2010, Gardner_2017h, Guinan_2019f, Jehangir_2019h, Jin_2023, Kobiela_2015, Kumar_2001, Kumar_2019f, Liu_2005, Losanoff_1996, Losanoff_1997e, Mesfin_2022a, Misra_2013, Qureshi_2016, Riva_2018j, Sobnach_2011f, Tammana_2012j, Tanrikulu_2015e, Tay_2004, Thapa_2019f, Trgo_2012f, Wadhwa_2015e, Yasin_2009, teWildt_2010}, 28 cases (39\%) were female \cite{AlShaaibi_2021b, Ali_2020f, Ataya_2013, Beecroft_1998, Bhasin_2014, Bhattacharjee_2008, Cauchi_2002, Chang_2017f, Cox_2007, DelgadoSalazar_2020c, DivsalarP._2023a, Goldman_1998f, Hardy_2023g, Kar_2015, Kariholu_2008, Kerestes_2019, Li_2013, Naji_2012f, Ohno_2005, Peixoto_2017f, Sakellaridis_2008f, Sultan_2024f, Tupesis_2004f, Wildhaber_2005, Wnęk_2015f, Yildiz_2016e}, 1 case (1\%) had no gender recorded \cite{fjbuilsRepeatedBehaviorDeliberate2024}. \paragraph*{Age Group} 25 cases (35\%) were between 26 and 40 years of age \cite{Alao_2006i, Ali_2022g, Apikotoa_2022f, Ataya_2013, Benoist_2019e, Bhasin_2014, Chang_2017f, Cox_2007, DelgadoSalazar_2020c, Farhadi_2024h, Fry_2010, Gardner_2017h, Guinan_2019f, Jin_2023, Kumar_2019f, Losanoff_1996, Misra_2013, Qureshi_2016, Riva_2018j, Sakellaridis_2008f, Tammana_2012j, Trgo_2012f, Wnęk_2015f, Yildiz_2016e, fjbuilsRepeatedBehaviorDeliberate2024}, 18 cases (25\%) were between 18 and 25 years of age \cite{Akay_2015f, Ali_2017, Atayan_2016, Bhattacharjee_2008, Csaky_1998e, Kar_2015, Kariholu_2008, Kobiela_2015, Losanoff_1996, Losanoff_1997e, Mesfin_2022a, Peixoto_2017f, Sobnach_2011f, Tupesis_2004f, Yasin_2009}, 13 cases (18\%) were under 18 years of age \cite{AlShaaibi_2021b, Ali_2020f, Cauchi_2002, DivsalarP._2023a, Goldman_1998f, Liu_2005, Naji_2012f, Ohno_2005, Tanrikulu_2015e, Tay_2004, Wildhaber_2005}, 11 cases (15\%) were between 41 and 60 years of age \cite{Al-Faham_2020k, Bhumi_2024f, CamachoDorado_2018, Emamhadi_2018, Hardy_2023g, Jehangir_2019h, Kumar_2001, Sultan_2024f, Thapa_2019f, Wadhwa_2015e, teWildt_2010}, 3 cases (4\%) were over 60 years of age \cite{Beecroft_1998, Kerestes_2019, Li_2013}, 2 cases (3\%) had no age documented \cite{Berry_2021e}. \paragraph*{Population} 36 cases (50\%) had a psychiatric history \cite{AlShaaibi_2021b, Alao_2006i, Ali_2020f, Apikotoa_2022f, Ataya_2013, Atayan_2016, Beecroft_1998, CamachoDorado_2018, Chang_2017f, DelgadoSalazar_2020c, DivsalarP._2023a, Farhadi_2024h, Fry_2010, Guinan_2019f, Hardy_2023g, Jehangir_2019h, Jin_2023, Kar_2015, Kerestes_2019, Kobiela_2015, Kumar_2001, Kumar_2019f, Liu_2005, Mesfin_2022a, Misra_2013, Ohno_2005, Peixoto_2017f, Sakellaridis_2008f, Sultan_2024f, Tammana_2012j, Tanrikulu_2015e, Yildiz_2016e, fjbuilsRepeatedBehaviorDeliberate2024, teWildt_2010}, 19 cases (26\%) had ingested previously \cite{Alao_2006i, Apikotoa_2022f, Berry_2021e, Bhattacharjee_2008, Csaky_1998e, DivsalarP._2023a, Emamhadi_2018, Guinan_2019f, Jehangir_2019h, Jin_2023, Liu_2005, Sakellaridis_2008f, Tanrikulu_2015e, Thapa_2019f, Yildiz_2016e, fjbuilsRepeatedBehaviorDeliberate2024, teWildt_2010}, 12 cases (17\%) were detained persons \cite{Alao_2006i, Ali_2022g, Apikotoa_2022f, Losanoff_1996, Losanoff_1997e, Qureshi_2016, Tammana_2012j, Trgo_2012f}, 7 cases (10\%) were severely disabled \cite{Atayan_2016, Kerestes_2019, Liu_2005, Ohno_2005, Peixoto_2017f, Yildiz_2016e, teWildt_2010}, 4 cases (6\%) were psychiatric inpatients \cite{DivsalarP._2023a, fjbuilsRepeatedBehaviorDeliberate2024, teWildt_2010}, 3 cases (4\%) were under the influence of alcohol \cite{Benoist_2019e, Csaky_1998e, Thapa_2019f}, 2 cases (3\%) were displaced people \cite{Akay_2015f, Gardner_2017h}. \paragraph*{Motivation} 34 cases (47\%) had a psychiatric motivation \cite{Al-Faham_2020k, Alao_2006i, Ali_2020f, Apikotoa_2022f, Ataya_2013, Atayan_2016, Bhasin_2014, Bhattacharjee_2008, DelgadoSalazar_2020c, DivsalarP._2023a, Emamhadi_2018, Farhadi_2024h, Guinan_2019f, Hardy_2023g, Jehangir_2019h, Jin_2023, Kar_2015, Kariholu_2008, Kerestes_2019, Kobiela_2015, Kumar_2001, Kumar_2019f, Li_2013, Liu_2005, Misra_2013, Ohno_2005, Sakellaridis_2008f, Sultan_2024f, Tammana_2012j, Tanrikulu_2015e, Yasin_2009, teWildt_2010}, 21 cases (29\%) were motivated by self-harm intention \cite{Al-Faham_2020k, AlShaaibi_2021b, Alao_2006i, Ali_2017, CamachoDorado_2018, Chang_2017f, Cox_2007, Csaky_1998e, Fry_2010, Li_2013, Losanoff_1996, Losanoff_1997e, Mesfin_2022a, Sakellaridis_2008f, Tammana_2012j, Tanrikulu_2015e, fjbuilsRepeatedBehaviorDeliberate2024}, 17 cases (24\%) had a psychosocial motivation \cite{Akay_2015f, Benoist_2019e, Bhattacharjee_2008, Cauchi_2002, Goldman_1998f, Hardy_2023g, Kobiela_2015, Li_2013, Naji_2012f, Qureshi_2016, Riva_2018j, Sobnach_2011f, Tay_2004, Thapa_2019f, Tupesis_2004f, Wildhaber_2005, Wnęk_2015f}, 9 cases (12\%) were motivated by protest \cite{Bhumi_2024f, Gardner_2017h, Losanoff_1996, Losanoff_1997e, Tupesis_2004f}, 9 cases (12\%) had another documented motivation \cite{Ali_2020f, Ali_2022g, Emamhadi_2018, Guinan_2019f, Peixoto_2017f, Sakellaridis_2008f, Trgo_2012f, Wadhwa_2015e, Yildiz_2016e}. \paragraph*{Object Characteristics} 51 cases (71\%) ingested a large diameter object (\textgreater{}2.5cm) \cite{Akay_2015f, Al-Faham_2020k, AlShaaibi_2021b, Alao_2006i, Ali_2017, Ali_2022g, Apikotoa_2022f, Atayan_2016, Berry_2021e, Bhasin_2014, CamachoDorado_2018, Cauchi_2002, Chang_2017f, Cox_2007, Csaky_1998e, DivsalarP._2023a, Emamhadi_2018, Gardner_2017h, Guinan_2019f, Jehangir_2019h, Jin_2023, Kariholu_2008, Kerestes_2019, Kobiela_2015, Kumar_2001, Kumar_2019f, Losanoff_1996, Losanoff_1997e, Mesfin_2022a, Misra_2013, Naji_2012f, Ohno_2005, Peixoto_2017f, Qureshi_2016, Riva_2018j, Sakellaridis_2008f, Sultan_2024f, Tanrikulu_2015e, Thapa_2019f, Trgo_2012f, Wnęk_2015f, Yildiz_2016e, fjbuilsRepeatedBehaviorDeliberate2024, teWildt_2010}, 44 cases (61\%) ingested multiple objects \cite{Ali_2020f, Apikotoa_2022f, Ataya_2013, Atayan_2016, Beecroft_1998, Bhattacharjee_2008, Bhumi_2024f, CamachoDorado_2018, Cauchi_2002, Emamhadi_2018, Farhadi_2024h, Fry_2010, Goldman_1998f, Guinan_2019f, Hardy_2023g, Jehangir_2019h, Jin_2023, Kar_2015, Kariholu_2008, Kobiela_2015, Kumar_2001, Kumar_2019f, Li_2013, Liu_2005, Losanoff_1996, Mesfin_2022a, Misra_2013, Naji_2012f, Ohno_2005, Sobnach_2011f, Sultan_2024f, Tammana_2012j, Tanrikulu_2015e, Tay_2004, Thapa_2019f, Wadhwa_2015e, Wildhaber_2005, Yasin_2009, fjbuilsRepeatedBehaviorDeliberate2024, teWildt_2010}, 34 cases (47\%) ingested a sharp object \cite{AlShaaibi_2021b, Alao_2006i, Apikotoa_2022f, Ataya_2013, Benoist_2019e, Bhasin_2014, Bhattacharjee_2008, CamachoDorado_2018, Csaky_1998e, DelgadoSalazar_2020c, DivsalarP._2023a, Emamhadi_2018, Farhadi_2024h, Fry_2010, Guinan_2019f, Hardy_2023g, Jehangir_2019h, Jin_2023, Kariholu_2008, Kobiela_2015, Kumar_2019f, Losanoff_1996, Losanoff_1997e, Mesfin_2022a, Misra_2013, Sobnach_2011f, Yasin_2009, teWildt_2010}, 32 cases (44\%) ingested a long object (\textgreater{}5cm) \cite{Al-Faham_2020k, AlShaaibi_2021b, Ali_2017, Ali_2022g, Atayan_2016, Bhasin_2014, CamachoDorado_2018, Chang_2017f, Cox_2007, Csaky_1998e, DivsalarP._2023a, Emamhadi_2018, Fry_2010, Gardner_2017h, Jin_2023, Kariholu_2008, Kerestes_2019, Kobiela_2015, Kumar_2019f, Mesfin_2022a, Misra_2013, Ohno_2005, Qureshi_2016, Sakellaridis_2008f, Sultan_2024f, Thapa_2019f, Trgo_2012f, Yasin_2009, Yildiz_2016e, teWildt_2010}, 9 cases (12\%) ingested a magnet \cite{Ali_2020f, Bhumi_2024f, Cauchi_2002, Liu_2005, Naji_2012f, Ohno_2005, Tanrikulu_2015e, Tay_2004, Wildhaber_2005}, 2 cases (3\%) ingested a button battery \cite{Berry_2021e, Bhumi_2024f}. \paragraph*{Outcomes} 48 cases (67\%) experienced a complication \cite{Ali_2017, Ali_2020f, Apikotoa_2022f, Atayan_2016, Beecroft_1998, Benoist_2019e, Berry_2021e, Bhasin_2014, Bhumi_2024f, CamachoDorado_2018, Cauchi_2002, Cox_2007, Csaky_1998e, DelgadoSalazar_2020c, DivsalarP._2023a, Emamhadi_2018, Farhadi_2024h, Fry_2010, Gardner_2017h, Goldman_1998f, Jin_2023, Kariholu_2008, Kerestes_2019, Kobiela_2015, Kumar_2001, Kumar_2019f, Liu_2005, Losanoff_1996, Mesfin_2022a, Misra_2013, Naji_2012f, Ohno_2005, Sakellaridis_2008f, Sobnach_2011f, Sultan_2024f, Tanrikulu_2015e, Tay_2004, Thapa_2019f, Trgo_2012f, Tupesis_2004f, Wildhaber_2005, Wnęk_2015f, Yasin_2009, Yildiz_2016e}, 44 cases (61\%) underwent surgery \cite{Al-Faham_2020k, AlShaaibi_2021b, Alao_2006i, Ali_2017, Ali_2020f, Atayan_2016, Beecroft_1998, Bhasin_2014, CamachoDorado_2018, Cauchi_2002, Chang_2017f, Cox_2007, Csaky_1998e, DelgadoSalazar_2020c, DivsalarP._2023a, Farhadi_2024h, Fry_2010, Gardner_2017h, Jin_2023, Kariholu_2008, Kerestes_2019, Kobiela_2015, Kumar_2019f, Liu_2005, Losanoff_1996, Losanoff_1997e, Mesfin_2022a, Misra_2013, Naji_2012f, Sobnach_2011f, Tanrikulu_2015e, Tay_2004, Thapa_2019f, Tupesis_2004f, Wildhaber_2005, Wnęk_2015f, Yasin_2009, Yildiz_2016e, fjbuilsRepeatedBehaviorDeliberate2024}, 31 cases (43\%) underwent endoscopy \cite{Akay_2015f, Ali_2022g, Apikotoa_2022f, Atayan_2016, Benoist_2019e, Berry_2021e, Bhasin_2014, Bhumi_2024f, CamachoDorado_2018, Chang_2017f, DelgadoSalazar_2020c, Gardner_2017h, Guinan_2019f, Hardy_2023g, Jehangir_2019h, Kariholu_2008, Li_2013, Liu_2005, Ohno_2005, Peixoto_2017f, Qureshi_2016, Riva_2018j, Sakellaridis_2008f, Sultan_2024f, Tammana_2012j, Tanrikulu_2015e, Trgo_2012f, Wadhwa_2015e, Wnęk_2015f, teWildt_2010}, 7 cases (10\%) were managed conservatively \cite{Ataya_2013, Bhattacharjee_2008, DivsalarP._2023a, Emamhadi_2018, Goldman_1998f, Kar_2015, Kumar_2001}, 2 cases (3\%) died \cite{Emamhadi_2018, Kumar_2001}. All 90 were male gender. 90 cases (100\%) were detained at the time of ingestion \cite{Elghali_2016, Karp_1991b, Lee_2007}, 88 cases (98\%) were intentional ingestions \cite{Elghali_2016, Karp_1991b, Lee_2007}, 30 cases (33\%) had a psychiatric history documented \cite{Elghali_2016, Karp_1991b, Lee_2007}, 2 cases (2\%) had a history of prior ingestion \cite{Elghali_2016}. No cases were reported for were psychiatric inpatients, were displaced people, were under the influence of alcohol at the time of ingestion, and had a severe disability history.
\paragraph*{Motivation}  70 cases (78\%) reported protest motivation \cite{Elghali_2016, Karp_1991b, Lee_2007}, 12 cases (13\%) reported psychiatric motivation \cite{Karp_1991b}, 6 cases (7\%) reported self-harm motivation \cite{Elghali_2016, Karp_1991b}. No cases were reported for psychosocial motivation and other motivation.
\paragraph*{Object Characteristics}  68 cases (76\%) involved sharp object ingestion \cite{Elghali_2016, Karp_1991b, Lee_2007}, 32 cases (36\%) involved long (\textgreater 5cm) object ingestion \cite{Lee_2007}, 25 cases (28\%) involved ingestion of multiple objects \cite{Elghali_2016, Lee_2007}. No cases were reported for button battery ingestion, magnet ingestion, and involved large diameter (\textgreater 2.5cm) object ingestion.
\paragraph*{Outcomes}  47 cases (52\%) underwent endoscopic intervention \cite{Elghali_2016, Lee_2007}, 29 cases (32\%) were managed conservatively \cite{Elghali_2016, Karp_1991b}, 15 cases (17\%) underwent surgical intervention \cite{Elghali_2016, Karp_1991b, Lee_2007}, 6 cases (7\%) reported complications \cite{Lee_2007}, 1 case (1\%) died \cite{Elghali_2016}.
\paragraph*{Gender} 43 cases (60\%) were male \cite{Akay_2015f, Al-Faham_2020k, Alao_2006i, Ali_2017, Ali_2022g, Apikotoa_2022f, Atayan_2016, Benoist_2019e, Berry_2021e, Bhumi_2024f, CamachoDorado_2018, Csaky_1998e, Emamhadi_2018, Farhadi_2024h, Fry_2010, Gardner_2017h, Guinan_2019f, Jehangir_2019h, Jin_2023, Kobiela_2015, Kumar_2001, Kumar_2019f, Liu_2005, Losanoff_1996, Losanoff_1997e, Mesfin_2022a, Misra_2013, Qureshi_2016, Riva_2018j, Sobnach_2011f, Tammana_2012j, Tanrikulu_2015e, Tay_2004, Thapa_2019f, Trgo_2012f, Wadhwa_2015e, Yasin_2009, teWildt_2010}, 28 cases (39\%) were female \cite{AlShaaibi_2021b, Ali_2020f, Ataya_2013, Beecroft_1998, Bhasin_2014, Bhattacharjee_2008, Cauchi_2002, Chang_2017f, Cox_2007, DelgadoSalazar_2020c, DivsalarP._2023a, Goldman_1998f, Hardy_2023g, Kar_2015, Kariholu_2008, Kerestes_2019, Li_2013, Naji_2012f, Ohno_2005, Peixoto_2017f, Sakellaridis_2008f, Sultan_2024f, Tupesis_2004f, Wildhaber_2005, Wnęk_2015f, Yildiz_2016e}, 1 case (1\%) had no gender recorded \cite{fjbuilsRepeatedBehaviorDeliberate2024}. \paragraph*{Age Group} 25 cases (35\%) were between 26 and 40 years of age \cite{Alao_2006i, Ali_2022g, Apikotoa_2022f, Ataya_2013, Benoist_2019e, Bhasin_2014, Chang_2017f, Cox_2007, DelgadoSalazar_2020c, Farhadi_2024h, Fry_2010, Gardner_2017h, Guinan_2019f, Jin_2023, Kumar_2019f, Losanoff_1996, Misra_2013, Qureshi_2016, Riva_2018j, Sakellaridis_2008f, Tammana_2012j, Trgo_2012f, Wnęk_2015f, Yildiz_2016e, fjbuilsRepeatedBehaviorDeliberate2024}, 18 cases (25\%) were between 18 and 25 years of age \cite{Akay_2015f, Ali_2017, Atayan_2016, Bhattacharjee_2008, Csaky_1998e, Kar_2015, Kariholu_2008, Kobiela_2015, Losanoff_1996, Losanoff_1997e, Mesfin_2022a, Peixoto_2017f, Sobnach_2011f, Tupesis_2004f, Yasin_2009}, 13 cases (18\%) were under 18 years of age \cite{AlShaaibi_2021b, Ali_2020f, Cauchi_2002, DivsalarP._2023a, Goldman_1998f, Liu_2005, Naji_2012f, Ohno_2005, Tanrikulu_2015e, Tay_2004, Wildhaber_2005}, 11 cases (15\%) were between 41 and 60 years of age \cite{Al-Faham_2020k, Bhumi_2024f, CamachoDorado_2018, Emamhadi_2018, Hardy_2023g, Jehangir_2019h, Kumar_2001, Sultan_2024f, Thapa_2019f, Wadhwa_2015e, teWildt_2010}, 3 cases (4\%) were over 60 years of age \cite{Beecroft_1998, Kerestes_2019, Li_2013}, 2 cases (3\%) had no age documented \cite{Berry_2021e}. All 90 were male gender. 90 cases (100\%) were detained at the time of ingestion \cite{Elghali_2016, Karp_1991b, Lee_2007}, 88 cases (98\%) were intentional ingestions \cite{Elghali_2016, Karp_1991b, Lee_2007}, 30 cases (33\%) had a psychiatric history documented \cite{Elghali_2016, Karp_1991b, Lee_2007}, 2 cases (2\%) had a history of prior ingestion \cite{Elghali_2016}. No cases were reported for were psychiatric inpatients, were displaced people, were under the influence of alcohol at the time of ingestion, and had a severe disability history.
\paragraph*{Motivation}  70 cases (78\%) reported protest motivation \cite{Elghali_2016, Karp_1991b, Lee_2007}, 12 cases (13\%) reported psychiatric motivation \cite{Karp_1991b}, 6 cases (7\%) reported self-harm motivation \cite{Elghali_2016, Karp_1991b}. No cases were reported for psychosocial motivation and other motivation.
\paragraph*{Object Characteristics}  68 cases (76\%) involved sharp object ingestion \cite{Elghali_2016, Karp_1991b, Lee_2007}, 32 cases (36\%) involved long (\textgreater 5cm) object ingestion \cite{Lee_2007}, 25 cases (28\%) involved ingestion of multiple objects \cite{Elghali_2016, Lee_2007}. No cases were reported for button battery ingestion, magnet ingestion, and involved large diameter (\textgreater 2.5cm) object ingestion.
\paragraph*{Outcomes}  47 cases (52\%) underwent endoscopic intervention \cite{Elghali_2016, Lee_2007}, 29 cases (32\%) were managed conservatively \cite{Elghali_2016, Karp_1991b}, 15 cases (17\%) underwent surgical intervention \cite{Elghali_2016, Karp_1991b, Lee_2007}, 6 cases (7\%) reported complications \cite{Lee_2007}, 1 case (1\%) died \cite{Elghali_2016}.
\paragraph*{Gender} 43 cases (60\%) were male \cite{Akay_2015f, Al-Faham_2020k, Alao_2006i, Ali_2017, Ali_2022g, Apikotoa_2022f, Atayan_2016, Benoist_2019e, Berry_2021e, Bhumi_2024f, CamachoDorado_2018, Csaky_1998e, Emamhadi_2018, Farhadi_2024h, Fry_2010, Gardner_2017h, Guinan_2019f, Jehangir_2019h, Jin_2023, Kobiela_2015, Kumar_2001, Kumar_2019f, Liu_2005, Losanoff_1996, Losanoff_1997e, Mesfin_2022a, Misra_2013, Qureshi_2016, Riva_2018j, Sobnach_2011f, Tammana_2012j, Tanrikulu_2015e, Tay_2004, Thapa_2019f, Trgo_2012f, Wadhwa_2015e, Yasin_2009, teWildt_2010}, 28 cases (39\%) were female \cite{AlShaaibi_2021b, Ali_2020f, Ataya_2013, Beecroft_1998, Bhasin_2014, Bhattacharjee_2008, Cauchi_2002, Chang_2017f, Cox_2007, DelgadoSalazar_2020c, DivsalarP._2023a, Goldman_1998f, Hardy_2023g, Kar_2015, Kariholu_2008, Kerestes_2019, Li_2013, Naji_2012f, Ohno_2005, Peixoto_2017f, Sakellaridis_2008f, Sultan_2024f, Tupesis_2004f, Wildhaber_2005, Wnęk_2015f, Yildiz_2016e}, 1 case (1\%) had no gender recorded \cite{fjbuilsRepeatedBehaviorDeliberate2024}. \paragraph*{Age Group} 25 cases (35\%) were between 26 and 40 years of age \cite{Alao_2006i, Ali_2022g, Apikotoa_2022f, Ataya_2013, Benoist_2019e, Bhasin_2014, Chang_2017f, Cox_2007, DelgadoSalazar_2020c, Farhadi_2024h, Fry_2010, Gardner_2017h, Guinan_2019f, Jin_2023, Kumar_2019f, Losanoff_1996, Misra_2013, Qureshi_2016, Riva_2018j, Sakellaridis_2008f, Tammana_2012j, Trgo_2012f, Wnęk_2015f, Yildiz_2016e, fjbuilsRepeatedBehaviorDeliberate2024}, 18 cases (25\%) were between 18 and 25 years of age \cite{Akay_2015f, Ali_2017, Atayan_2016, Bhattacharjee_2008, Csaky_1998e, Kar_2015, Kariholu_2008, Kobiela_2015, Losanoff_1996, Losanoff_1997e, Mesfin_2022a, Peixoto_2017f, Sobnach_2011f, Tupesis_2004f, Yasin_2009}, 13 cases (18\%) were under 18 years of age \cite{AlShaaibi_2021b, Ali_2020f, Cauchi_2002, DivsalarP._2023a, Goldman_1998f, Liu_2005, Naji_2012f, Ohno_2005, Tanrikulu_2015e, Tay_2004, Wildhaber_2005}, 11 cases (15\%) were between 41 and 60 years of age \cite{Al-Faham_2020k, Bhumi_2024f, CamachoDorado_2018, Emamhadi_2018, Hardy_2023g, Jehangir_2019h, Kumar_2001, Sultan_2024f, Thapa_2019f, Wadhwa_2015e, teWildt_2010}, 3 cases (4\%) were over 60 years of age \cite{Beecroft_1998, Kerestes_2019, Li_2013}, 2 cases (3\%) had no age documented \cite{Berry_2021e}. All 90 were male gender. 90 cases (100\%) were detained at the time of ingestion \cite{Elghali_2016, Karp_1991b, Lee_2007}, 88 cases (98\%) were intentional ingestions \cite{Elghali_2016, Karp_1991b, Lee_2007}, 30 cases (33\%) had a psychiatric history documented \cite{Elghali_2016, Karp_1991b, Lee_2007}, 2 cases (2\%) had a history of prior ingestion \cite{Elghali_2016}. No cases were reported for were psychiatric inpatients, were displaced people, were under the influence of alcohol at the time of ingestion, and had a severe disability history.
\paragraph*{Motivation}  70 cases (78\%) reported protest motivation \cite{Elghali_2016, Karp_1991b, Lee_2007}, 12 cases (13\%) reported psychiatric motivation \cite{Karp_1991b}, 6 cases (7\%) reported self-harm motivation \cite{Elghali_2016, Karp_1991b}. No cases were reported for psychosocial motivation and other motivation.
\paragraph*{Object Characteristics}  68 cases (76\%) involved sharp object ingestion \cite{Elghali_2016, Karp_1991b, Lee_2007}, 32 cases (36\%) involved long (\textgreater 5cm) object ingestion \cite{Lee_2007}, 25 cases (28\%) involved ingestion of multiple objects \cite{Elghali_2016, Lee_2007}. No cases were reported for button battery ingestion, magnet ingestion, and involved large diameter (\textgreater 2.5cm) object ingestion.
\paragraph*{Outcomes}  47 cases (52\%) underwent endoscopic intervention \cite{Elghali_2016, Lee_2007}, 29 cases (32\%) were managed conservatively \cite{Elghali_2016, Karp_1991b}, 15 cases (17\%) underwent surgical intervention \cite{Elghali_2016, Karp_1991b, Lee_2007}, 6 cases (7\%) reported complications \cite{Lee_2007}, 1 case (1\%) died \cite{Elghali_2016}.
\paragraph*{Gender} 43 cases (60\%) were male \cite{Akay_2015f, Al-Faham_2020k, Alao_2006i, Ali_2017, Ali_2022g, Apikotoa_2022f, Atayan_2016, Benoist_2019e, Berry_2021e, Bhumi_2024f, CamachoDorado_2018, Csaky_1998e, Emamhadi_2018, Farhadi_2024h, Fry_2010, Gardner_2017h, Guinan_2019f, Jehangir_2019h, Jin_2023, Kobiela_2015, Kumar_2001, Kumar_2019f, Liu_2005, Losanoff_1996, Losanoff_1997e, Mesfin_2022a, Misra_2013, Qureshi_2016, Riva_2018j, Sobnach_2011f, Tammana_2012j, Tanrikulu_2015e, Tay_2004, Thapa_2019f, Trgo_2012f, Wadhwa_2015e, Yasin_2009, teWildt_2010}, 28 cases (39\%) were female \cite{AlShaaibi_2021b, Ali_2020f, Ataya_2013, Beecroft_1998, Bhasin_2014, Bhattacharjee_2008, Cauchi_2002, Chang_2017f, Cox_2007, DelgadoSalazar_2020c, DivsalarP._2023a, Goldman_1998f, Hardy_2023g, Kar_2015, Kariholu_2008, Kerestes_2019, Li_2013, Naji_2012f, Ohno_2005, Peixoto_2017f, Sakellaridis_2008f, Sultan_2024f, Tupesis_2004f, Wildhaber_2005, Wnęk_2015f, Yildiz_2016e}, 1 case (1\%) had no gender recorded \cite{fjbuilsRepeatedBehaviorDeliberate2024}. \paragraph*{Age Group} 25 cases (35\%) were between 26 and 40 years of age \cite{Alao_2006i, Ali_2022g, Apikotoa_2022f, Ataya_2013, Benoist_2019e, Bhasin_2014, Chang_2017f, Cox_2007, DelgadoSalazar_2020c, Farhadi_2024h, Fry_2010, Gardner_2017h, Guinan_2019f, Jin_2023, Kumar_2019f, Losanoff_1996, Misra_2013, Qureshi_2016, Riva_2018j, Sakellaridis_2008f, Tammana_2012j, Trgo_2012f, Wnęk_2015f, Yildiz_2016e, fjbuilsRepeatedBehaviorDeliberate2024}, 18 cases (25\%) were between 18 and 25 years of age \cite{Akay_2015f, Ali_2017, Atayan_2016, Bhattacharjee_2008, Csaky_1998e, Kar_2015, Kariholu_2008, Kobiela_2015, Losanoff_1996, Losanoff_1997e, Mesfin_2022a, Peixoto_2017f, Sobnach_2011f, Tupesis_2004f, Yasin_2009}, 13 cases (18\%) were under 18 years of age \cite{AlShaaibi_2021b, Ali_2020f, Cauchi_2002, DivsalarP._2023a, Goldman_1998f, Liu_2005, Naji_2012f, Ohno_2005, Tanrikulu_2015e, Tay_2004, Wildhaber_2005}, 11 cases (15\%) were between 41 and 60 years of age \cite{Al-Faham_2020k, Bhumi_2024f, CamachoDorado_2018, Emamhadi_2018, Hardy_2023g, Jehangir_2019h, Kumar_2001, Sultan_2024f, Thapa_2019f, Wadhwa_2015e, teWildt_2010}, 3 cases (4\%) were over 60 years of age \cite{Beecroft_1998, Kerestes_2019, Li_2013}, 2 cases (3\%) had no age documented \cite{Berry_2021e}. All 90 were male gender. 90 cases (100\%) were detained at the time of ingestion \cite{Elghali_2016, Karp_1991b, Lee_2007}, 88 cases (98\%) were intentional ingestions \cite{Elghali_2016, Karp_1991b, Lee_2007}, 30 cases (33\%) had a psychiatric history documented \cite{Elghali_2016, Karp_1991b, Lee_2007}, 2 cases (2\%) had a history of prior ingestion \cite{Elghali_2016}. No cases were reported for were psychiatric inpatients, were displaced people, were under the influence of alcohol at the time of ingestion, and had a severe disability history.
\paragraph*{Motivation}  70 cases (78\%) reported protest motivation \cite{Elghali_2016, Karp_1991b, Lee_2007}, 12 cases (13\%) reported psychiatric motivation \cite{Karp_1991b}, 6 cases (7\%) reported self-harm motivation \cite{Elghali_2016, Karp_1991b}. No cases were reported for psychosocial motivation and other motivation.
\paragraph*{Object Characteristics}  68 cases (76\%) involved sharp object ingestion \cite{Elghali_2016, Karp_1991b, Lee_2007}, 32 cases (36\%) involved long (\textgreater 5cm) object ingestion \cite{Lee_2007}, 25 cases (28\%) involved ingestion of multiple objects \cite{Elghali_2016, Lee_2007}. No cases were reported for button battery ingestion, magnet ingestion, and involved large diameter (\textgreater 2.5cm) object ingestion.
\paragraph*{Outcomes}  47 cases (52\%) underwent endoscopic intervention \cite{Elghali_2016, Lee_2007}, 29 cases (32\%) were managed conservatively \cite{Elghali_2016, Karp_1991b}, 15 cases (17\%) underwent surgical intervention \cite{Elghali_2016, Karp_1991b, Lee_2007}, 6 cases (7\%) reported complications \cite{Lee_2007}, 1 case (1\%) died \cite{Elghali_2016}.
\paragraph*{Gender} 43 cases (60\%) were male \cite{Akay_2015f, Al-Faham_2020k, Alao_2006i, Ali_2017, Ali_2022g, Apikotoa_2022f, Atayan_2016, Benoist_2019e, Berry_2021e, Bhumi_2024f, CamachoDorado_2018, Csaky_1998e, Emamhadi_2018, Farhadi_2024h, Fry_2010, Gardner_2017h, Guinan_2019f, Jehangir_2019h, Jin_2023, Kobiela_2015, Kumar_2001, Kumar_2019f, Liu_2005, Losanoff_1996, Losanoff_1997e, Mesfin_2022a, Misra_2013, Qureshi_2016, Riva_2018j, Sobnach_2011f, Tammana_2012j, Tanrikulu_2015e, Tay_2004, Thapa_2019f, Trgo_2012f, Wadhwa_2015e, Yasin_2009, teWildt_2010}, 28 cases (39\%) were female \cite{AlShaaibi_2021b, Ali_2020f, Ataya_2013, Beecroft_1998, Bhasin_2014, Bhattacharjee_2008, Cauchi_2002, Chang_2017f, Cox_2007, DelgadoSalazar_2020c, DivsalarP._2023a, Goldman_1998f, Hardy_2023g, Kar_2015, Kariholu_2008, Kerestes_2019, Li_2013, Naji_2012f, Ohno_2005, Peixoto_2017f, Sakellaridis_2008f, Sultan_2024f, Tupesis_2004f, Wildhaber_2005, Wnęk_2015f, Yildiz_2016e}, 1 case (1\%) had no gender recorded \cite{fjbuilsRepeatedBehaviorDeliberate2024}. \paragraph*{Age Group} 25 cases (35\%) were between 26 and 40 years of age \cite{Alao_2006i, Ali_2022g, Apikotoa_2022f, Ataya_2013, Benoist_2019e, Bhasin_2014, Chang_2017f, Cox_2007, DelgadoSalazar_2020c, Farhadi_2024h, Fry_2010, Gardner_2017h, Guinan_2019f, Jin_2023, Kumar_2019f, Losanoff_1996, Misra_2013, Qureshi_2016, Riva_2018j, Sakellaridis_2008f, Tammana_2012j, Trgo_2012f, Wnęk_2015f, Yildiz_2016e, fjbuilsRepeatedBehaviorDeliberate2024}, 18 cases (25\%) were between 18 and 25 years of age \cite{Akay_2015f, Ali_2017, Atayan_2016, Bhattacharjee_2008, Csaky_1998e, Kar_2015, Kariholu_2008, Kobiela_2015, Losanoff_1996, Losanoff_1997e, Mesfin_2022a, Peixoto_2017f, Sobnach_2011f, Tupesis_2004f, Yasin_2009}, 13 cases (18\%) were under 18 years of age \cite{AlShaaibi_2021b, Ali_2020f, Cauchi_2002, DivsalarP._2023a, Goldman_1998f, Liu_2005, Naji_2012f, Ohno_2005, Tanrikulu_2015e, Tay_2004, Wildhaber_2005}, 11 cases (15\%) were between 41 and 60 years of age \cite{Al-Faham_2020k, Bhumi_2024f, CamachoDorado_2018, Emamhadi_2018, Hardy_2023g, Jehangir_2019h, Kumar_2001, Sultan_2024f, Thapa_2019f, Wadhwa_2015e, teWildt_2010}, 3 cases (4\%) were over 60 years of age \cite{Beecroft_1998, Kerestes_2019, Li_2013}, 2 cases (3\%) had no age documented \cite{Berry_2021e}. All 90 were male gender. 90 cases (100\%) were detained at the time of ingestion \cite{Elghali_2016, Karp_1991b, Lee_2007}, 88 cases (98\%) were intentional ingestions \cite{Elghali_2016, Karp_1991b, Lee_2007}, 30 cases (33\%) had a psychiatric history documented \cite{Elghali_2016, Karp_1991b, Lee_2007}, 2 cases (2\%) had a history of prior ingestion \cite{Elghali_2016}. No cases were reported for were psychiatric inpatients, were displaced people, were under the influence of alcohol at the time of ingestion, and had a severe disability history.
\paragraph*{Motivation}  70 cases (78\%) reported protest motivation \cite{Elghali_2016, Karp_1991b, Lee_2007}, 12 cases (13\%) reported psychiatric motivation \cite{Karp_1991b}, 6 cases (7\%) reported self-harm motivation \cite{Elghali_2016, Karp_1991b}. No cases were reported for psychosocial motivation and other motivation.
\paragraph*{Object Characteristics}  68 cases (76\%) involved sharp object ingestion \cite{Elghali_2016, Karp_1991b, Lee_2007}, 32 cases (36\%) involved long (\textgreater 5cm) object ingestion \cite{Lee_2007}, 25 cases (28\%) involved ingestion of multiple objects \cite{Elghali_2016, Lee_2007}. No cases were reported for button battery ingestion, magnet ingestion, and involved large diameter (\textgreater 2.5cm) object ingestion.
\paragraph*{Outcomes}  47 cases (52\%) underwent endoscopic intervention \cite{Elghali_2016, Lee_2007}, 29 cases (32\%) were managed conservatively \cite{Elghali_2016, Karp_1991b}, 15 cases (17\%) underwent surgical intervention \cite{Elghali_2016, Karp_1991b, Lee_2007}, 6 cases (7\%) reported complications \cite{Lee_2007}, 1 case (1\%) died \cite{Elghali_2016}.
\paragraph*{Gender} 43 cases (60\%) were male \cite{Akay_2015f, Al-Faham_2020k, Alao_2006i, Ali_2017, Ali_2022g, Apikotoa_2022f, Atayan_2016, Benoist_2019e, Berry_2021e, Bhumi_2024f, CamachoDorado_2018, Csaky_1998e, Emamhadi_2018, Farhadi_2024h, Fry_2010, Gardner_2017h, Guinan_2019f, Jehangir_2019h, Jin_2023, Kobiela_2015, Kumar_2001, Kumar_2019f, Liu_2005, Losanoff_1996, Losanoff_1997e, Mesfin_2022a, Misra_2013, Qureshi_2016, Riva_2018j, Sobnach_2011f, Tammana_2012j, Tanrikulu_2015e, Tay_2004, Thapa_2019f, Trgo_2012f, Wadhwa_2015e, Yasin_2009, teWildt_2010}, 28 cases (39\%) were female \cite{AlShaaibi_2021b, Ali_2020f, Ataya_2013, Beecroft_1998, Bhasin_2014, Bhattacharjee_2008, Cauchi_2002, Chang_2017f, Cox_2007, DelgadoSalazar_2020c, DivsalarP._2023a, Goldman_1998f, Hardy_2023g, Kar_2015, Kariholu_2008, Kerestes_2019, Li_2013, Naji_2012f, Ohno_2005, Peixoto_2017f, Sakellaridis_2008f, Sultan_2024f, Tupesis_2004f, Wildhaber_2005, Wnęk_2015f, Yildiz_2016e}, 1 case (1\%) had no gender recorded \cite{fjbuilsRepeatedBehaviorDeliberate2024}. \paragraph*{Age Group} 25 cases (35\%) were between 26 and 40 years of age \cite{Alao_2006i, Ali_2022g, Apikotoa_2022f, Ataya_2013, Benoist_2019e, Bhasin_2014, Chang_2017f, Cox_2007, DelgadoSalazar_2020c, Farhadi_2024h, Fry_2010, Gardner_2017h, Guinan_2019f, Jin_2023, Kumar_2019f, Losanoff_1996, Misra_2013, Qureshi_2016, Riva_2018j, Sakellaridis_2008f, Tammana_2012j, Trgo_2012f, Wnęk_2015f, Yildiz_2016e, fjbuilsRepeatedBehaviorDeliberate2024}, 18 cases (25\%) were between 18 and 25 years of age \cite{Akay_2015f, Ali_2017, Atayan_2016, Bhattacharjee_2008, Csaky_1998e, Kar_2015, Kariholu_2008, Kobiela_2015, Losanoff_1996, Losanoff_1997e, Mesfin_2022a, Peixoto_2017f, Sobnach_2011f, Tupesis_2004f, Yasin_2009}, 13 cases (18\%) were under 18 years of age \cite{AlShaaibi_2021b, Ali_2020f, Cauchi_2002, DivsalarP._2023a, Goldman_1998f, Liu_2005, Naji_2012f, Ohno_2005, Tanrikulu_2015e, Tay_2004, Wildhaber_2005}, 11 cases (15\%) were between 41 and 60 years of age \cite{Al-Faham_2020k, Bhumi_2024f, CamachoDorado_2018, Emamhadi_2018, Hardy_2023g, Jehangir_2019h, Kumar_2001, Sultan_2024f, Thapa_2019f, Wadhwa_2015e, teWildt_2010}, 3 cases (4\%) were over 60 years of age \cite{Beecroft_1998, Kerestes_2019, Li_2013}, 2 cases (3\%) had no age documented \cite{Berry_2021e}. All 90 were male gender. 90 cases (100\%) were detained at the time of ingestion \cite{Elghali_2016, Karp_1991b, Lee_2007}, 88 cases (98\%) were intentional ingestions \cite{Elghali_2016, Karp_1991b, Lee_2007}, 30 cases (33\%) had a psychiatric history documented \cite{Elghali_2016, Karp_1991b, Lee_2007}, 2 cases (2\%) had a history of prior ingestion \cite{Elghali_2016}. No cases were reported for were psychiatric inpatients, were displaced people, were under the influence of alcohol at the time of ingestion, and had a severe disability history.
\paragraph*{Motivation}  70 cases (78\%) reported protest motivation \cite{Elghali_2016, Karp_1991b, Lee_2007}, 12 cases (13\%) reported psychiatric motivation \cite{Karp_1991b}, 6 cases (7\%) reported self-harm motivation \cite{Elghali_2016, Karp_1991b}. No cases were reported for psychosocial motivation and other motivation.
\paragraph*{Object Characteristics}  68 cases (76\%) involved sharp object ingestion \cite{Elghali_2016, Karp_1991b, Lee_2007}, 32 cases (36\%) involved long (\textgreater 5cm) object ingestion \cite{Lee_2007}, 25 cases (28\%) involved ingestion of multiple objects \cite{Elghali_2016, Lee_2007}. No cases were reported for button battery ingestion, magnet ingestion, and involved large diameter (\textgreater 2.5cm) object ingestion.
\paragraph*{Outcomes}  47 cases (52\%) underwent endoscopic intervention \cite{Elghali_2016, Lee_2007}, 29 cases (32\%) were managed conservatively \cite{Elghali_2016, Karp_1991b}, 15 cases (17\%) underwent surgical intervention \cite{Elghali_2016, Karp_1991b, Lee_2007}, 6 cases (7\%) reported complications \cite{Lee_2007}, 1 case (1\%) died \cite{Elghali_2016}.
\paragraph*{Gender} 43 cases (60\%) were male \cite{Akay_2015f, Al-Faham_2020k, Alao_2006i, Ali_2017, Ali_2022g, Apikotoa_2022f, Atayan_2016, Benoist_2019e, Berry_2021e, Bhumi_2024f, CamachoDorado_2018, Csaky_1998e, Emamhadi_2018, Farhadi_2024h, Fry_2010, Gardner_2017h, Guinan_2019f, Jehangir_2019h, Jin_2023, Kobiela_2015, Kumar_2001, Kumar_2019f, Liu_2005, Losanoff_1996, Losanoff_1997e, Mesfin_2022a, Misra_2013, Qureshi_2016, Riva_2018j, Sobnach_2011f, Tammana_2012j, Tanrikulu_2015e, Tay_2004, Thapa_2019f, Trgo_2012f, Wadhwa_2015e, Yasin_2009, teWildt_2010}, 28 cases (39\%) were female \cite{AlShaaibi_2021b, Ali_2020f, Ataya_2013, Beecroft_1998, Bhasin_2014, Bhattacharjee_2008, Cauchi_2002, Chang_2017f, Cox_2007, DelgadoSalazar_2020c, DivsalarP._2023a, Goldman_1998f, Hardy_2023g, Kar_2015, Kariholu_2008, Kerestes_2019, Li_2013, Naji_2012f, Ohno_2005, Peixoto_2017f, Sakellaridis_2008f, Sultan_2024f, Tupesis_2004f, Wildhaber_2005, Wnęk_2015f, Yildiz_2016e}, 1 case (1\%) had no gender recorded \cite{fjbuilsRepeatedBehaviorDeliberate2024}. \paragraph*{Age Group} 25 cases (35\%) were between 26 and 40 years of age \cite{Alao_2006i, Ali_2022g, Apikotoa_2022f, Ataya_2013, Benoist_2019e, Bhasin_2014, Chang_2017f, Cox_2007, DelgadoSalazar_2020c, Farhadi_2024h, Fry_2010, Gardner_2017h, Guinan_2019f, Jin_2023, Kumar_2019f, Losanoff_1996, Misra_2013, Qureshi_2016, Riva_2018j, Sakellaridis_2008f, Tammana_2012j, Trgo_2012f, Wnęk_2015f, Yildiz_2016e, fjbuilsRepeatedBehaviorDeliberate2024}, 18 cases (25\%) were between 18 and 25 years of age \cite{Akay_2015f, Ali_2017, Atayan_2016, Bhattacharjee_2008, Csaky_1998e, Kar_2015, Kariholu_2008, Kobiela_2015, Losanoff_1996, Losanoff_1997e, Mesfin_2022a, Peixoto_2017f, Sobnach_2011f, Tupesis_2004f, Yasin_2009}, 13 cases (18\%) were under 18 years of age \cite{AlShaaibi_2021b, Ali_2020f, Cauchi_2002, DivsalarP._2023a, Goldman_1998f, Liu_2005, Naji_2012f, Ohno_2005, Tanrikulu_2015e, Tay_2004, Wildhaber_2005}, 11 cases (15\%) were between 41 and 60 years of age \cite{Al-Faham_2020k, Bhumi_2024f, CamachoDorado_2018, Emamhadi_2018, Hardy_2023g, Jehangir_2019h, Kumar_2001, Sultan_2024f, Thapa_2019f, Wadhwa_2015e, teWildt_2010}, 3 cases (4\%) were over 60 years of age \cite{Beecroft_1998, Kerestes_2019, Li_2013}, 2 cases (3\%) had no age documented \cite{Berry_2021e}. All 90 were male gender. 90 cases (100\%) were detained at the time of ingestion \cite{Elghali_2016, Karp_1991b, Lee_2007}, 88 cases (98\%) were intentional ingestions \cite{Elghali_2016, Karp_1991b, Lee_2007}, 30 cases (33\%) had a psychiatric history documented \cite{Elghali_2016, Karp_1991b, Lee_2007}, 2 cases (2\%) had a history of prior ingestion \cite{Elghali_2016}. No cases were reported for were psychiatric inpatients, were displaced people, were under the influence of alcohol at the time of ingestion, and had a severe disability history.
\paragraph*{Motivation}  70 cases (78\%) reported protest motivation \cite{Elghali_2016, Karp_1991b, Lee_2007}, 12 cases (13\%) reported psychiatric motivation \cite{Karp_1991b}, 6 cases (7\%) reported self-harm motivation \cite{Elghali_2016, Karp_1991b}. No cases were reported for psychosocial motivation and other motivation.
\paragraph*{Object Characteristics}  68 cases (76\%) involved sharp object ingestion \cite{Elghali_2016, Karp_1991b, Lee_2007}, 32 cases (36\%) involved long (\textgreater 5cm) object ingestion \cite{Lee_2007}, 25 cases (28\%) involved ingestion of multiple objects \cite{Elghali_2016, Lee_2007}. No cases were reported for button battery ingestion, magnet ingestion, and involved large diameter (\textgreater 2.5cm) object ingestion.
\paragraph*{Outcomes}  47 cases (52\%) underwent endoscopic intervention \cite{Elghali_2016, Lee_2007}, 29 cases (32\%) were managed conservatively \cite{Elghali_2016, Karp_1991b}, 15 cases (17\%) underwent surgical intervention \cite{Elghali_2016, Karp_1991b, Lee_2007}, 6 cases (7\%) reported complications \cite{Lee_2007}, 1 case (1\%) died \cite{Elghali_2016}.
\paragraph*{Gender} 43 cases (60\%) were male \cite{Akay_2015f, Al-Faham_2020k, Alao_2006i, Ali_2017, Ali_2022g, Apikotoa_2022f, Atayan_2016, Benoist_2019e, Berry_2021e, Bhumi_2024f, CamachoDorado_2018, Csaky_1998e, Emamhadi_2018, Farhadi_2024h, Fry_2010, Gardner_2017h, Guinan_2019f, Jehangir_2019h, Jin_2023, Kobiela_2015, Kumar_2001, Kumar_2019f, Liu_2005, Losanoff_1996, Losanoff_1997e, Mesfin_2022a, Misra_2013, Qureshi_2016, Riva_2018j, Sobnach_2011f, Tammana_2012j, Tanrikulu_2015e, Tay_2004, Thapa_2019f, Trgo_2012f, Wadhwa_2015e, Yasin_2009, teWildt_2010}, 28 cases (39\%) were female \cite{AlShaaibi_2021b, Ali_2020f, Ataya_2013, Beecroft_1998, Bhasin_2014, Bhattacharjee_2008, Cauchi_2002, Chang_2017f, Cox_2007, DelgadoSalazar_2020c, DivsalarP._2023a, Goldman_1998f, Hardy_2023g, Kar_2015, Kariholu_2008, Kerestes_2019, Li_2013, Naji_2012f, Ohno_2005, Peixoto_2017f, Sakellaridis_2008f, Sultan_2024f, Tupesis_2004f, Wildhaber_2005, Wnęk_2015f, Yildiz_2016e}, 1 case (1\%) had no gender recorded \cite{fjbuilsRepeatedBehaviorDeliberate2024}. \paragraph*{Age Group} 25 cases (35\%) were between 26 and 40 years of age \cite{Alao_2006i, Ali_2022g, Apikotoa_2022f, Ataya_2013, Benoist_2019e, Bhasin_2014, Chang_2017f, Cox_2007, DelgadoSalazar_2020c, Farhadi_2024h, Fry_2010, Gardner_2017h, Guinan_2019f, Jin_2023, Kumar_2019f, Losanoff_1996, Misra_2013, Qureshi_2016, Riva_2018j, Sakellaridis_2008f, Tammana_2012j, Trgo_2012f, Wnęk_2015f, Yildiz_2016e, fjbuilsRepeatedBehaviorDeliberate2024}, 18 cases (25\%) were between 18 and 25 years of age \cite{Akay_2015f, Ali_2017, Atayan_2016, Bhattacharjee_2008, Csaky_1998e, Kar_2015, Kariholu_2008, Kobiela_2015, Losanoff_1996, Losanoff_1997e, Mesfin_2022a, Peixoto_2017f, Sobnach_2011f, Tupesis_2004f, Yasin_2009}, 13 cases (18\%) were under 18 years of age \cite{AlShaaibi_2021b, Ali_2020f, Cauchi_2002, DivsalarP._2023a, Goldman_1998f, Liu_2005, Naji_2012f, Ohno_2005, Tanrikulu_2015e, Tay_2004, Wildhaber_2005}, 11 cases (15\%) were between 41 and 60 years of age \cite{Al-Faham_2020k, Bhumi_2024f, CamachoDorado_2018, Emamhadi_2018, Hardy_2023g, Jehangir_2019h, Kumar_2001, Sultan_2024f, Thapa_2019f, Wadhwa_2015e, teWildt_2010}, 3 cases (4\%) were over 60 years of age \cite{Beecroft_1998, Kerestes_2019, Li_2013}, 2 cases (3\%) had no age documented \cite{Berry_2021e}. All 90 were male gender. 90 cases (100\%) were detained at the time of ingestion \cite{Elghali_2016, Karp_1991b, Lee_2007}, 88 cases (98\%) were intentional ingestions \cite{Elghali_2016, Karp_1991b, Lee_2007}, 30 cases (33\%) had a psychiatric history documented \cite{Elghali_2016, Karp_1991b, Lee_2007}, 2 cases (2\%) had a history of prior ingestion \cite{Elghali_2016}. No cases were reported for were psychiatric inpatients, were displaced people, were under the influence of alcohol at the time of ingestion, and had a severe disability history.
\paragraph*{Motivation}  70 cases (78\%) reported protest motivation \cite{Elghali_2016, Karp_1991b, Lee_2007}, 12 cases (13\%) reported psychiatric motivation \cite{Karp_1991b}, 6 cases (7\%) reported self-harm motivation \cite{Elghali_2016, Karp_1991b}. No cases were reported for psychosocial motivation and other motivation.
\paragraph*{Object Characteristics}  68 cases (76\%) involved sharp object ingestion \cite{Elghali_2016, Karp_1991b, Lee_2007}, 32 cases (36\%) involved long (\textgreater 5cm) object ingestion \cite{Lee_2007}, 25 cases (28\%) involved ingestion of multiple objects \cite{Elghali_2016, Lee_2007}. No cases were reported for button battery ingestion, magnet ingestion, and involved large diameter (\textgreater 2.5cm) object ingestion.
\paragraph*{Outcomes}  47 cases (52\%) underwent endoscopic intervention \cite{Elghali_2016, Lee_2007}, 29 cases (32\%) were managed conservatively \cite{Elghali_2016, Karp_1991b}, 15 cases (17\%) underwent surgical intervention \cite{Elghali_2016, Karp_1991b, Lee_2007}, 6 cases (7\%) reported complications \cite{Lee_2007}, 1 case (1\%) died \cite{Elghali_2016}.
\paragraph*{Gender} 43 cases (60\%) were male \cite{Akay_2015f, Al-Faham_2020k, Alao_2006i, Ali_2017, Ali_2022g, Apikotoa_2022f, Atayan_2016, Benoist_2019e, Berry_2021e, Bhumi_2024f, CamachoDorado_2018, Csaky_1998e, Emamhadi_2018, Farhadi_2024h, Fry_2010, Gardner_2017h, Guinan_2019f, Jehangir_2019h, Jin_2023, Kobiela_2015, Kumar_2001, Kumar_2019f, Liu_2005, Losanoff_1996, Losanoff_1997e, Mesfin_2022a, Misra_2013, Qureshi_2016, Riva_2018j, Sobnach_2011f, Tammana_2012j, Tanrikulu_2015e, Tay_2004, Thapa_2019f, Trgo_2012f, Wadhwa_2015e, Yasin_2009, teWildt_2010}, 28 cases (39\%) were female \cite{AlShaaibi_2021b, Ali_2020f, Ataya_2013, Beecroft_1998, Bhasin_2014, Bhattacharjee_2008, Cauchi_2002, Chang_2017f, Cox_2007, DelgadoSalazar_2020c, DivsalarP._2023a, Goldman_1998f, Hardy_2023g, Kar_2015, Kariholu_2008, Kerestes_2019, Li_2013, Naji_2012f, Ohno_2005, Peixoto_2017f, Sakellaridis_2008f, Sultan_2024f, Tupesis_2004f, Wildhaber_2005, Wnęk_2015f, Yildiz_2016e}, 1 case (1\%) had no gender recorded \cite{fjbuilsRepeatedBehaviorDeliberate2024}. \paragraph*{Age Group} 25 cases (35\%) were between 26 and 40 years of age \cite{Alao_2006i, Ali_2022g, Apikotoa_2022f, Ataya_2013, Benoist_2019e, Bhasin_2014, Chang_2017f, Cox_2007, DelgadoSalazar_2020c, Farhadi_2024h, Fry_2010, Gardner_2017h, Guinan_2019f, Jin_2023, Kumar_2019f, Losanoff_1996, Misra_2013, Qureshi_2016, Riva_2018j, Sakellaridis_2008f, Tammana_2012j, Trgo_2012f, Wnęk_2015f, Yildiz_2016e, fjbuilsRepeatedBehaviorDeliberate2024}, 18 cases (25\%) were between 18 and 25 years of age \cite{Akay_2015f, Ali_2017, Atayan_2016, Bhattacharjee_2008, Csaky_1998e, Kar_2015, Kariholu_2008, Kobiela_2015, Losanoff_1996, Losanoff_1997e, Mesfin_2022a, Peixoto_2017f, Sobnach_2011f, Tupesis_2004f, Yasin_2009}, 13 cases (18\%) were under 18 years of age \cite{AlShaaibi_2021b, Ali_2020f, Cauchi_2002, DivsalarP._2023a, Goldman_1998f, Liu_2005, Naji_2012f, Ohno_2005, Tanrikulu_2015e, Tay_2004, Wildhaber_2005}, 11 cases (15\%) were between 41 and 60 years of age \cite{Al-Faham_2020k, Bhumi_2024f, CamachoDorado_2018, Emamhadi_2018, Hardy_2023g, Jehangir_2019h, Kumar_2001, Sultan_2024f, Thapa_2019f, Wadhwa_2015e, teWildt_2010}, 3 cases (4\%) were over 60 years of age \cite{Beecroft_1998, Kerestes_2019, Li_2013}, 2 cases (3\%) had no age documented \cite{Berry_2021e}. All 90 were male gender. 90 cases (100\%) were detained at the time of ingestion \cite{Elghali_2016, Karp_1991b, Lee_2007}, 88 cases (98\%) were intentional ingestions \cite{Elghali_2016, Karp_1991b, Lee_2007}, 30 cases (33\%) had a psychiatric history documented \cite{Elghali_2016, Karp_1991b, Lee_2007}, 2 cases (2\%) had a history of prior ingestion \cite{Elghali_2016}. No cases were reported for were psychiatric inpatients, were displaced people, were under the influence of alcohol at the time of ingestion, and had a severe disability history.
\paragraph*{Motivation}  70 cases (78\%) reported protest motivation \cite{Elghali_2016, Karp_1991b, Lee_2007}, 12 cases (13\%) reported psychiatric motivation \cite{Karp_1991b}, 6 cases (7\%) reported self-harm motivation \cite{Elghali_2016, Karp_1991b}. No cases were reported for psychosocial motivation and other motivation.
\paragraph*{Object Characteristics}  68 cases (76\%) involved sharp object ingestion \cite{Elghali_2016, Karp_1991b, Lee_2007}, 32 cases (36\%) involved long (\textgreater 5cm) object ingestion \cite{Lee_2007}, 25 cases (28\%) involved ingestion of multiple objects \cite{Elghali_2016, Lee_2007}. No cases were reported for button battery ingestion, magnet ingestion, and involved large diameter (\textgreater 2.5cm) object ingestion.
\paragraph*{Outcomes}  47 cases (52\%) underwent endoscopic intervention \cite{Elghali_2016, Lee_2007}, 29 cases (32\%) were managed conservatively \cite{Elghali_2016, Karp_1991b}, 15 cases (17\%) underwent surgical intervention \cite{Elghali_2016, Karp_1991b, Lee_2007}, 6 cases (7\%) reported complications \cite{Lee_2007}, 1 case (1\%) died \cite{Elghali_2016}.
\paragraph*{Gender} 43 cases (60\%) were male \cite{Akay_2015f, Al-Faham_2020k, Alao_2006i, Ali_2017, Ali_2022g, Apikotoa_2022f, Atayan_2016, Benoist_2019e, Berry_2021e, Bhumi_2024f, CamachoDorado_2018, Csaky_1998e, Emamhadi_2018, Farhadi_2024h, Fry_2010, Gardner_2017h, Guinan_2019f, Jehangir_2019h, Jin_2023, Kobiela_2015, Kumar_2001, Kumar_2019f, Liu_2005, Losanoff_1996, Losanoff_1997e, Mesfin_2022a, Misra_2013, Qureshi_2016, Riva_2018j, Sobnach_2011f, Tammana_2012j, Tanrikulu_2015e, Tay_2004, Thapa_2019f, Trgo_2012f, Wadhwa_2015e, Yasin_2009, teWildt_2010}, 28 cases (39\%) were female \cite{AlShaaibi_2021b, Ali_2020f, Ataya_2013, Beecroft_1998, Bhasin_2014, Bhattacharjee_2008, Cauchi_2002, Chang_2017f, Cox_2007, DelgadoSalazar_2020c, DivsalarP._2023a, Goldman_1998f, Hardy_2023g, Kar_2015, Kariholu_2008, Kerestes_2019, Li_2013, Naji_2012f, Ohno_2005, Peixoto_2017f, Sakellaridis_2008f, Sultan_2024f, Tupesis_2004f, Wildhaber_2005, Wnęk_2015f, Yildiz_2016e}, 1 case (1\%) had no gender recorded \cite{fjbuilsRepeatedBehaviorDeliberate2024}. \paragraph*{Age Group} 25 cases (35\%) were between 26 and 40 years of age \cite{Alao_2006i, Ali_2022g, Apikotoa_2022f, Ataya_2013, Benoist_2019e, Bhasin_2014, Chang_2017f, Cox_2007, DelgadoSalazar_2020c, Farhadi_2024h, Fry_2010, Gardner_2017h, Guinan_2019f, Jin_2023, Kumar_2019f, Losanoff_1996, Misra_2013, Qureshi_2016, Riva_2018j, Sakellaridis_2008f, Tammana_2012j, Trgo_2012f, Wnęk_2015f, Yildiz_2016e, fjbuilsRepeatedBehaviorDeliberate2024}, 18 cases (25\%) were between 18 and 25 years of age \cite{Akay_2015f, Ali_2017, Atayan_2016, Bhattacharjee_2008, Csaky_1998e, Kar_2015, Kariholu_2008, Kobiela_2015, Losanoff_1996, Losanoff_1997e, Mesfin_2022a, Peixoto_2017f, Sobnach_2011f, Tupesis_2004f, Yasin_2009}, 13 cases (18\%) were under 18 years of age \cite{AlShaaibi_2021b, Ali_2020f, Cauchi_2002, DivsalarP._2023a, Goldman_1998f, Liu_2005, Naji_2012f, Ohno_2005, Tanrikulu_2015e, Tay_2004, Wildhaber_2005}, 11 cases (15\%) were between 41 and 60 years of age \cite{Al-Faham_2020k, Bhumi_2024f, CamachoDorado_2018, Emamhadi_2018, Hardy_2023g, Jehangir_2019h, Kumar_2001, Sultan_2024f, Thapa_2019f, Wadhwa_2015e, teWildt_2010}, 3 cases (4\%) were over 60 years of age \cite{Beecroft_1998, Kerestes_2019, Li_2013}, 2 cases (3\%) had no age documented \cite{Berry_2021e}. All 90 were male gender. 90 cases (100\%) were detained at the time of ingestion \cite{Elghali_2016, Karp_1991b, Lee_2007}, 88 cases (98\%) were intentional ingestions \cite{Elghali_2016, Karp_1991b, Lee_2007}, 30 cases (33\%) had a psychiatric history documented \cite{Elghali_2016, Karp_1991b, Lee_2007}, 2 cases (2\%) had a history of prior ingestion \cite{Elghali_2016}. No cases were reported for were psychiatric inpatients, were displaced people, were under the influence of alcohol at the time of ingestion, and had a severe disability history.
\paragraph*{Motivation}  70 cases (78\%) reported protest motivation \cite{Elghali_2016, Karp_1991b, Lee_2007}, 12 cases (13\%) reported psychiatric motivation \cite{Karp_1991b}, 6 cases (7\%) reported self-harm motivation \cite{Elghali_2016, Karp_1991b}. No cases were reported for psychosocial motivation and other motivation.
\paragraph*{Object Characteristics}  68 cases (76\%) involved sharp object ingestion \cite{Elghali_2016, Karp_1991b, Lee_2007}, 32 cases (36\%) involved long (\textgreater 5cm) object ingestion \cite{Lee_2007}, 25 cases (28\%) involved ingestion of multiple objects \cite{Elghali_2016, Lee_2007}. No cases were reported for button battery ingestion, magnet ingestion, and involved large diameter (\textgreater 2.5cm) object ingestion.
\paragraph*{Outcomes}  47 cases (52\%) underwent endoscopic intervention \cite{Elghali_2016, Lee_2007}, 29 cases (32\%) were managed conservatively \cite{Elghali_2016, Karp_1991b}, 15 cases (17\%) underwent surgical intervention \cite{Elghali_2016, Karp_1991b, Lee_2007}, 6 cases (7\%) reported complications \cite{Lee_2007}, 1 case (1\%) died \cite{Elghali_2016}.
\paragraph*{Gender} 43 cases (60\%) were male \cite{Akay_2015f, Al-Faham_2020k, Alao_2006i, Ali_2017, Ali_2022g, Apikotoa_2022f, Atayan_2016, Benoist_2019e, Berry_2021e, Bhumi_2024f, CamachoDorado_2018, Csaky_1998e, Emamhadi_2018, Farhadi_2024h, Fry_2010, Gardner_2017h, Guinan_2019f, Jehangir_2019h, Jin_2023, Kobiela_2015, Kumar_2001, Kumar_2019f, Liu_2005, Losanoff_1996, Losanoff_1997e, Mesfin_2022a, Misra_2013, Qureshi_2016, Riva_2018j, Sobnach_2011f, Tammana_2012j, Tanrikulu_2015e, Tay_2004, Thapa_2019f, Trgo_2012f, Wadhwa_2015e, Yasin_2009, teWildt_2010}, 28 cases (39\%) were female \cite{AlShaaibi_2021b, Ali_2020f, Ataya_2013, Beecroft_1998, Bhasin_2014, Bhattacharjee_2008, Cauchi_2002, Chang_2017f, Cox_2007, DelgadoSalazar_2020c, DivsalarP._2023a, Goldman_1998f, Hardy_2023g, Kar_2015, Kariholu_2008, Kerestes_2019, Li_2013, Naji_2012f, Ohno_2005, Peixoto_2017f, Sakellaridis_2008f, Sultan_2024f, Tupesis_2004f, Wildhaber_2005, Wnęk_2015f, Yildiz_2016e}, 1 case (1\%) had no gender recorded \cite{fjbuilsRepeatedBehaviorDeliberate2024}. \paragraph*{Age Group} 25 cases (35\%) were between 26 and 40 years of age \cite{Alao_2006i, Ali_2022g, Apikotoa_2022f, Ataya_2013, Benoist_2019e, Bhasin_2014, Chang_2017f, Cox_2007, DelgadoSalazar_2020c, Farhadi_2024h, Fry_2010, Gardner_2017h, Guinan_2019f, Jin_2023, Kumar_2019f, Losanoff_1996, Misra_2013, Qureshi_2016, Riva_2018j, Sakellaridis_2008f, Tammana_2012j, Trgo_2012f, Wnęk_2015f, Yildiz_2016e, fjbuilsRepeatedBehaviorDeliberate2024}, 18 cases (25\%) were between 18 and 25 years of age \cite{Akay_2015f, Ali_2017, Atayan_2016, Bhattacharjee_2008, Csaky_1998e, Kar_2015, Kariholu_2008, Kobiela_2015, Losanoff_1996, Losanoff_1997e, Mesfin_2022a, Peixoto_2017f, Sobnach_2011f, Tupesis_2004f, Yasin_2009}, 13 cases (18\%) were under 18 years of age \cite{AlShaaibi_2021b, Ali_2020f, Cauchi_2002, DivsalarP._2023a, Goldman_1998f, Liu_2005, Naji_2012f, Ohno_2005, Tanrikulu_2015e, Tay_2004, Wildhaber_2005}, 11 cases (15\%) were between 41 and 60 years of age \cite{Al-Faham_2020k, Bhumi_2024f, CamachoDorado_2018, Emamhadi_2018, Hardy_2023g, Jehangir_2019h, Kumar_2001, Sultan_2024f, Thapa_2019f, Wadhwa_2015e, teWildt_2010}, 3 cases (4\%) were over 60 years of age \cite{Beecroft_1998, Kerestes_2019, Li_2013}, 2 cases (3\%) had no age documented \cite{Berry_2021e}. All 90 were male gender. 90 cases (100\%) were detained at the time of ingestion \cite{Elghali_2016, Karp_1991b, Lee_2007}, 88 cases (98\%) were intentional ingestions \cite{Elghali_2016, Karp_1991b, Lee_2007}, 30 cases (33\%) had a psychiatric history documented \cite{Elghali_2016, Karp_1991b, Lee_2007}, 2 cases (2\%) had a history of prior ingestion \cite{Elghali_2016}. No cases were reported for were psychiatric inpatients, were displaced people, were under the influence of alcohol at the time of ingestion, and had a severe disability history.
\paragraph*{Motivation}  70 cases (78\%) reported protest motivation \cite{Elghali_2016, Karp_1991b, Lee_2007}, 12 cases (13\%) reported psychiatric motivation \cite{Karp_1991b}, 6 cases (7\%) reported self-harm motivation \cite{Elghali_2016, Karp_1991b}. No cases were reported for psychosocial motivation and other motivation.
\paragraph*{Object Characteristics}  68 cases (76\%) involved sharp object ingestion \cite{Elghali_2016, Karp_1991b, Lee_2007}, 32 cases (36\%) involved long (\textgreater 5cm) object ingestion \cite{Lee_2007}, 25 cases (28\%) involved ingestion of multiple objects \cite{Elghali_2016, Lee_2007}. No cases were reported for button battery ingestion, magnet ingestion, and involved large diameter (\textgreater 2.5cm) object ingestion.
\paragraph*{Outcomes}  47 cases (52\%) underwent endoscopic intervention \cite{Elghali_2016, Lee_2007}, 29 cases (32\%) were managed conservatively \cite{Elghali_2016, Karp_1991b}, 15 cases (17\%) underwent surgical intervention \cite{Elghali_2016, Karp_1991b, Lee_2007}, 6 cases (7\%) reported complications \cite{Lee_2007}, 1 case (1\%) died \cite{Elghali_2016}.
\paragraph*{Gender} 43 cases (60\%) were male \cite{Akay_2015f, Al-Faham_2020k, Alao_2006i, Ali_2017, Ali_2022g, Apikotoa_2022f, Atayan_2016, Benoist_2019e, Berry_2021e, Bhumi_2024f, CamachoDorado_2018, Csaky_1998e, Emamhadi_2018, Farhadi_2024h, Fry_2010, Gardner_2017h, Guinan_2019f, Jehangir_2019h, Jin_2023, Kobiela_2015, Kumar_2001, Kumar_2019f, Liu_2005, Losanoff_1996, Losanoff_1997e, Mesfin_2022a, Misra_2013, Qureshi_2016, Riva_2018j, Sobnach_2011f, Tammana_2012j, Tanrikulu_2015e, Tay_2004, Thapa_2019f, Trgo_2012f, Wadhwa_2015e, Yasin_2009, teWildt_2010}, 28 cases (39\%) were female \cite{AlShaaibi_2021b, Ali_2020f, Ataya_2013, Beecroft_1998, Bhasin_2014, Bhattacharjee_2008, Cauchi_2002, Chang_2017f, Cox_2007, DelgadoSalazar_2020c, DivsalarP._2023a, Goldman_1998f, Hardy_2023g, Kar_2015, Kariholu_2008, Kerestes_2019, Li_2013, Naji_2012f, Ohno_2005, Peixoto_2017f, Sakellaridis_2008f, Sultan_2024f, Tupesis_2004f, Wildhaber_2005, Wnęk_2015f, Yildiz_2016e}, 1 case (1\%) had no gender recorded \cite{fjbuilsRepeatedBehaviorDeliberate2024}. \paragraph*{Age Group} 25 cases (35\%) were between 26 and 40 years of age \cite{Alao_2006i, Ali_2022g, Apikotoa_2022f, Ataya_2013, Benoist_2019e, Bhasin_2014, Chang_2017f, Cox_2007, DelgadoSalazar_2020c, Farhadi_2024h, Fry_2010, Gardner_2017h, Guinan_2019f, Jin_2023, Kumar_2019f, Losanoff_1996, Misra_2013, Qureshi_2016, Riva_2018j, Sakellaridis_2008f, Tammana_2012j, Trgo_2012f, Wnęk_2015f, Yildiz_2016e, fjbuilsRepeatedBehaviorDeliberate2024}, 18 cases (25\%) were between 18 and 25 years of age \cite{Akay_2015f, Ali_2017, Atayan_2016, Bhattacharjee_2008, Csaky_1998e, Kar_2015, Kariholu_2008, Kobiela_2015, Losanoff_1996, Losanoff_1997e, Mesfin_2022a, Peixoto_2017f, Sobnach_2011f, Tupesis_2004f, Yasin_2009}, 13 cases (18\%) were under 18 years of age \cite{AlShaaibi_2021b, Ali_2020f, Cauchi_2002, DivsalarP._2023a, Goldman_1998f, Liu_2005, Naji_2012f, Ohno_2005, Tanrikulu_2015e, Tay_2004, Wildhaber_2005}, 11 cases (15\%) were between 41 and 60 years of age \cite{Al-Faham_2020k, Bhumi_2024f, CamachoDorado_2018, Emamhadi_2018, Hardy_2023g, Jehangir_2019h, Kumar_2001, Sultan_2024f, Thapa_2019f, Wadhwa_2015e, teWildt_2010}, 3 cases (4\%) were over 60 years of age \cite{Beecroft_1998, Kerestes_2019, Li_2013}, 2 cases (3\%) had no age documented \cite{Berry_2021e}. All 90 were male gender. 90 cases (100\%) were detained at the time of ingestion \cite{Elghali_2016, Karp_1991b, Lee_2007}, 88 cases (98\%) were intentional ingestions \cite{Elghali_2016, Karp_1991b, Lee_2007}, 30 cases (33\%) had a psychiatric history documented \cite{Elghali_2016, Karp_1991b, Lee_2007}, 2 cases (2\%) had a history of prior ingestion \cite{Elghali_2016}. No cases were reported for were psychiatric inpatients, were displaced people, were under the influence of alcohol at the time of ingestion, and had a severe disability history.
\paragraph*{Motivation}  70 cases (78\%) reported protest motivation \cite{Elghali_2016, Karp_1991b, Lee_2007}, 12 cases (13\%) reported psychiatric motivation \cite{Karp_1991b}, 6 cases (7\%) reported self-harm motivation \cite{Elghali_2016, Karp_1991b}. No cases were reported for psychosocial motivation and other motivation.
\paragraph*{Object Characteristics}  68 cases (76\%) involved sharp object ingestion \cite{Elghali_2016, Karp_1991b, Lee_2007}, 32 cases (36\%) involved long (\textgreater 5cm) object ingestion \cite{Lee_2007}, 25 cases (28\%) involved ingestion of multiple objects \cite{Elghali_2016, Lee_2007}. No cases were reported for button battery ingestion, magnet ingestion, and involved large diameter (\textgreater 2.5cm) object ingestion.
\paragraph*{Outcomes}  47 cases (52\%) underwent endoscopic intervention \cite{Elghali_2016, Lee_2007}, 29 cases (32\%) were managed conservatively \cite{Elghali_2016, Karp_1991b}, 15 cases (17\%) underwent surgical intervention \cite{Elghali_2016, Karp_1991b, Lee_2007}, 6 cases (7\%) reported complications \cite{Lee_2007}, 1 case (1\%) died \cite{Elghali_2016}.
\paragraph*{Gender} 43 cases (60\%) were male \cite{Akay_2015f, Al-Faham_2020k, Alao_2006i, Ali_2017, Ali_2022g, Apikotoa_2022f, Atayan_2016, Benoist_2019e, Berry_2021e, Bhumi_2024f, CamachoDorado_2018, Csaky_1998e, Emamhadi_2018, Farhadi_2024h, Fry_2010, Gardner_2017h, Guinan_2019f, Jehangir_2019h, Jin_2023, Kobiela_2015, Kumar_2001, Kumar_2019f, Liu_2005, Losanoff_1996, Losanoff_1997e, Mesfin_2022a, Misra_2013, Qureshi_2016, Riva_2018j, Sobnach_2011f, Tammana_2012j, Tanrikulu_2015e, Tay_2004, Thapa_2019f, Trgo_2012f, Wadhwa_2015e, Yasin_2009, teWildt_2010}, 28 cases (39\%) were female \cite{AlShaaibi_2021b, Ali_2020f, Ataya_2013, Beecroft_1998, Bhasin_2014, Bhattacharjee_2008, Cauchi_2002, Chang_2017f, Cox_2007, DelgadoSalazar_2020c, DivsalarP._2023a, Goldman_1998f, Hardy_2023g, Kar_2015, Kariholu_2008, Kerestes_2019, Li_2013, Naji_2012f, Ohno_2005, Peixoto_2017f, Sakellaridis_2008f, Sultan_2024f, Tupesis_2004f, Wildhaber_2005, Wnęk_2015f, Yildiz_2016e}, 1 case (1\%) had no gender recorded \cite{fjbuilsRepeatedBehaviorDeliberate2024}. \paragraph*{Age Group} 25 cases (35\%) were between 26 and 40 years of age \cite{Alao_2006i, Ali_2022g, Apikotoa_2022f, Ataya_2013, Benoist_2019e, Bhasin_2014, Chang_2017f, Cox_2007, DelgadoSalazar_2020c, Farhadi_2024h, Fry_2010, Gardner_2017h, Guinan_2019f, Jin_2023, Kumar_2019f, Losanoff_1996, Misra_2013, Qureshi_2016, Riva_2018j, Sakellaridis_2008f, Tammana_2012j, Trgo_2012f, Wnęk_2015f, Yildiz_2016e, fjbuilsRepeatedBehaviorDeliberate2024}, 18 cases (25\%) were between 18 and 25 years of age \cite{Akay_2015f, Ali_2017, Atayan_2016, Bhattacharjee_2008, Csaky_1998e, Kar_2015, Kariholu_2008, Kobiela_2015, Losanoff_1996, Losanoff_1997e, Mesfin_2022a, Peixoto_2017f, Sobnach_2011f, Tupesis_2004f, Yasin_2009}, 13 cases (18\%) were under 18 years of age \cite{AlShaaibi_2021b, Ali_2020f, Cauchi_2002, DivsalarP._2023a, Goldman_1998f, Liu_2005, Naji_2012f, Ohno_2005, Tanrikulu_2015e, Tay_2004, Wildhaber_2005}, 11 cases (15\%) were between 41 and 60 years of age \cite{Al-Faham_2020k, Bhumi_2024f, CamachoDorado_2018, Emamhadi_2018, Hardy_2023g, Jehangir_2019h, Kumar_2001, Sultan_2024f, Thapa_2019f, Wadhwa_2015e, teWildt_2010}, 3 cases (4\%) were over 60 years of age \cite{Beecroft_1998, Kerestes_2019, Li_2013}, 2 cases (3\%) had no age documented \cite{Berry_2021e}. All 90 were male gender. 90 cases (100\%) were detained at the time of ingestion \cite{Elghali_2016, Karp_1991b, Lee_2007}, 88 cases (98\%) were intentional ingestions \cite{Elghali_2016, Karp_1991b, Lee_2007}, 30 cases (33\%) had a psychiatric history documented \cite{Elghali_2016, Karp_1991b, Lee_2007}, 2 cases (2\%) had a history of prior ingestion \cite{Elghali_2016}. No cases were reported for were psychiatric inpatients, were displaced people, were under the influence of alcohol at the time of ingestion, and had a severe disability history.
\paragraph*{Motivation}  70 cases (78\%) reported protest motivation \cite{Elghali_2016, Karp_1991b, Lee_2007}, 12 cases (13\%) reported psychiatric motivation \cite{Karp_1991b}, 6 cases (7\%) reported self-harm motivation \cite{Elghali_2016, Karp_1991b}. No cases were reported for psychosocial motivation and other motivation.
\paragraph*{Object Characteristics}  68 cases (76\%) involved sharp object ingestion \cite{Elghali_2016, Karp_1991b, Lee_2007}, 32 cases (36\%) involved long (\textgreater 5cm) object ingestion \cite{Lee_2007}, 25 cases (28\%) involved ingestion of multiple objects \cite{Elghali_2016, Lee_2007}. No cases were reported for button battery ingestion, magnet ingestion, and involved large diameter (\textgreater 2.5cm) object ingestion.
\paragraph*{Outcomes}  47 cases (52\%) underwent endoscopic intervention \cite{Elghali_2016, Lee_2007}, 29 cases (32\%) were managed conservatively \cite{Elghali_2016, Karp_1991b}, 15 cases (17\%) underwent surgical intervention \cite{Elghali_2016, Karp_1991b, Lee_2007}, 6 cases (7\%) reported complications \cite{Lee_2007}, 1 case (1\%) died \cite{Elghali_2016}.
\paragraph*{Gender} 43 cases (60\%) were male \cite{Akay_2015f, Al-Faham_2020k, Alao_2006i, Ali_2017, Ali_2022g, Apikotoa_2022f, Atayan_2016, Benoist_2019e, Berry_2021e, Bhumi_2024f, CamachoDorado_2018, Csaky_1998e, Emamhadi_2018, Farhadi_2024h, Fry_2010, Gardner_2017h, Guinan_2019f, Jehangir_2019h, Jin_2023, Kobiela_2015, Kumar_2001, Kumar_2019f, Liu_2005, Losanoff_1996, Losanoff_1997e, Mesfin_2022a, Misra_2013, Qureshi_2016, Riva_2018j, Sobnach_2011f, Tammana_2012j, Tanrikulu_2015e, Tay_2004, Thapa_2019f, Trgo_2012f, Wadhwa_2015e, Yasin_2009, teWildt_2010}, 28 cases (39\%) were female \cite{AlShaaibi_2021b, Ali_2020f, Ataya_2013, Beecroft_1998, Bhasin_2014, Bhattacharjee_2008, Cauchi_2002, Chang_2017f, Cox_2007, DelgadoSalazar_2020c, DivsalarP._2023a, Goldman_1998f, Hardy_2023g, Kar_2015, Kariholu_2008, Kerestes_2019, Li_2013, Naji_2012f, Ohno_2005, Peixoto_2017f, Sakellaridis_2008f, Sultan_2024f, Tupesis_2004f, Wildhaber_2005, Wnęk_2015f, Yildiz_2016e}, 1 case (1\%) had no gender recorded \cite{fjbuilsRepeatedBehaviorDeliberate2024}. \paragraph*{Age Group} 25 cases (35\%) were between 26 and 40 years of age \cite{Alao_2006i, Ali_2022g, Apikotoa_2022f, Ataya_2013, Benoist_2019e, Bhasin_2014, Chang_2017f, Cox_2007, DelgadoSalazar_2020c, Farhadi_2024h, Fry_2010, Gardner_2017h, Guinan_2019f, Jin_2023, Kumar_2019f, Losanoff_1996, Misra_2013, Qureshi_2016, Riva_2018j, Sakellaridis_2008f, Tammana_2012j, Trgo_2012f, Wnęk_2015f, Yildiz_2016e, fjbuilsRepeatedBehaviorDeliberate2024}, 18 cases (25\%) were between 18 and 25 years of age \cite{Akay_2015f, Ali_2017, Atayan_2016, Bhattacharjee_2008, Csaky_1998e, Kar_2015, Kariholu_2008, Kobiela_2015, Losanoff_1996, Losanoff_1997e, Mesfin_2022a, Peixoto_2017f, Sobnach_2011f, Tupesis_2004f, Yasin_2009}, 13 cases (18\%) were under 18 years of age \cite{AlShaaibi_2021b, Ali_2020f, Cauchi_2002, DivsalarP._2023a, Goldman_1998f, Liu_2005, Naji_2012f, Ohno_2005, Tanrikulu_2015e, Tay_2004, Wildhaber_2005}, 11 cases (15\%) were between 41 and 60 years of age \cite{Al-Faham_2020k, Bhumi_2024f, CamachoDorado_2018, Emamhadi_2018, Hardy_2023g, Jehangir_2019h, Kumar_2001, Sultan_2024f, Thapa_2019f, Wadhwa_2015e, teWildt_2010}, 3 cases (4\%) were over 60 years of age \cite{Beecroft_1998, Kerestes_2019, Li_2013}, 2 cases (3\%) had no age documented \cite{Berry_2021e}. All 90 were male gender. 90 cases (100\%) were detained at the time of ingestion \cite{Elghali_2016, Karp_1991b, Lee_2007}, 88 cases (98\%) were intentional ingestions \cite{Elghali_2016, Karp_1991b, Lee_2007}, 30 cases (33\%) had a psychiatric history documented \cite{Elghali_2016, Karp_1991b, Lee_2007}, 2 cases (2\%) had a history of prior ingestion \cite{Elghali_2016}. No cases were reported for were psychiatric inpatients, were displaced people, were under the influence of alcohol at the time of ingestion, and had a severe disability history.
\paragraph*{Motivation}  70 cases (78\%) reported protest motivation \cite{Elghali_2016, Karp_1991b, Lee_2007}, 12 cases (13\%) reported psychiatric motivation \cite{Karp_1991b}, 6 cases (7\%) reported self-harm motivation \cite{Elghali_2016, Karp_1991b}. No cases were reported for psychosocial motivation and other motivation.
\paragraph*{Object Characteristics}  68 cases (76\%) involved sharp object ingestion \cite{Elghali_2016, Karp_1991b, Lee_2007}, 32 cases (36\%) involved long (\textgreater 5cm) object ingestion \cite{Lee_2007}, 25 cases (28\%) involved ingestion of multiple objects \cite{Elghali_2016, Lee_2007}. No cases were reported for button battery ingestion, magnet ingestion, and involved large diameter (\textgreater 2.5cm) object ingestion.
\paragraph*{Outcomes}  47 cases (52\%) underwent endoscopic intervention \cite{Elghali_2016, Lee_2007}, 29 cases (32\%) were managed conservatively \cite{Elghali_2016, Karp_1991b}, 15 cases (17\%) underwent surgical intervention \cite{Elghali_2016, Karp_1991b, Lee_2007}, 6 cases (7\%) reported complications \cite{Lee_2007}, 1 case (1\%) died \cite{Elghali_2016}.
\paragraph*{Gender} 43 cases (60\%) were male \cite{Akay_2015f, Al-Faham_2020k, Alao_2006i, Ali_2017, Ali_2022g, Apikotoa_2022f, Atayan_2016, Benoist_2019e, Berry_2021e, Bhumi_2024f, CamachoDorado_2018, Csaky_1998e, Emamhadi_2018, Farhadi_2024h, Fry_2010, Gardner_2017h, Guinan_2019f, Jehangir_2019h, Jin_2023, Kobiela_2015, Kumar_2001, Kumar_2019f, Liu_2005, Losanoff_1996, Losanoff_1997e, Mesfin_2022a, Misra_2013, Qureshi_2016, Riva_2018j, Sobnach_2011f, Tammana_2012j, Tanrikulu_2015e, Tay_2004, Thapa_2019f, Trgo_2012f, Wadhwa_2015e, Yasin_2009, teWildt_2010}, 28 cases (39\%) were female \cite{AlShaaibi_2021b, Ali_2020f, Ataya_2013, Beecroft_1998, Bhasin_2014, Bhattacharjee_2008, Cauchi_2002, Chang_2017f, Cox_2007, DelgadoSalazar_2020c, DivsalarP._2023a, Goldman_1998f, Hardy_2023g, Kar_2015, Kariholu_2008, Kerestes_2019, Li_2013, Naji_2012f, Ohno_2005, Peixoto_2017f, Sakellaridis_2008f, Sultan_2024f, Tupesis_2004f, Wildhaber_2005, Wnęk_2015f, Yildiz_2016e}, 1 case (1\%) had no gender recorded \cite{fjbuilsRepeatedBehaviorDeliberate2024}. \paragraph*{Age Group} 25 cases (35\%) were between 26 and 40 years of age \cite{Alao_2006i, Ali_2022g, Apikotoa_2022f, Ataya_2013, Benoist_2019e, Bhasin_2014, Chang_2017f, Cox_2007, DelgadoSalazar_2020c, Farhadi_2024h, Fry_2010, Gardner_2017h, Guinan_2019f, Jin_2023, Kumar_2019f, Losanoff_1996, Misra_2013, Qureshi_2016, Riva_2018j, Sakellaridis_2008f, Tammana_2012j, Trgo_2012f, Wnęk_2015f, Yildiz_2016e, fjbuilsRepeatedBehaviorDeliberate2024}, 18 cases (25\%) were between 18 and 25 years of age \cite{Akay_2015f, Ali_2017, Atayan_2016, Bhattacharjee_2008, Csaky_1998e, Kar_2015, Kariholu_2008, Kobiela_2015, Losanoff_1996, Losanoff_1997e, Mesfin_2022a, Peixoto_2017f, Sobnach_2011f, Tupesis_2004f, Yasin_2009}, 13 cases (18\%) were under 18 years of age \cite{AlShaaibi_2021b, Ali_2020f, Cauchi_2002, DivsalarP._2023a, Goldman_1998f, Liu_2005, Naji_2012f, Ohno_2005, Tanrikulu_2015e, Tay_2004, Wildhaber_2005}, 11 cases (15\%) were between 41 and 60 years of age \cite{Al-Faham_2020k, Bhumi_2024f, CamachoDorado_2018, Emamhadi_2018, Hardy_2023g, Jehangir_2019h, Kumar_2001, Sultan_2024f, Thapa_2019f, Wadhwa_2015e, teWildt_2010}, 3 cases (4\%) were over 60 years of age \cite{Beecroft_1998, Kerestes_2019, Li_2013}, 2 cases (3\%) had no age documented \cite{Berry_2021e}. All 90 were male gender. 90 cases (100\%) were detained at the time of ingestion \cite{Elghali_2016, Karp_1991b, Lee_2007}, 88 cases (98\%) were intentional ingestions \cite{Elghali_2016, Karp_1991b, Lee_2007}, 30 cases (33\%) had a psychiatric history documented \cite{Elghali_2016, Karp_1991b, Lee_2007}, 2 cases (2\%) had a history of prior ingestion \cite{Elghali_2016}. No cases were reported for were psychiatric inpatients, were displaced people, were under the influence of alcohol at the time of ingestion, and had a severe disability history.
\paragraph*{Motivation}  70 cases (78\%) reported protest motivation \cite{Elghali_2016, Karp_1991b, Lee_2007}, 12 cases (13\%) reported psychiatric motivation \cite{Karp_1991b}, 6 cases (7\%) reported self-harm motivation \cite{Elghali_2016, Karp_1991b}. No cases were reported for psychosocial motivation and other motivation.
\paragraph*{Object Characteristics}  68 cases (76\%) involved sharp object ingestion \cite{Elghali_2016, Karp_1991b, Lee_2007}, 32 cases (36\%) involved long (\textgreater 5cm) object ingestion \cite{Lee_2007}, 25 cases (28\%) involved ingestion of multiple objects \cite{Elghali_2016, Lee_2007}. No cases were reported for button battery ingestion, magnet ingestion, and involved large diameter (\textgreater 2.5cm) object ingestion.
\paragraph*{Outcomes}  47 cases (52\%) underwent endoscopic intervention \cite{Elghali_2016, Lee_2007}, 29 cases (32\%) were managed conservatively \cite{Elghali_2016, Karp_1991b}, 15 cases (17\%) underwent surgical intervention \cite{Elghali_2016, Karp_1991b, Lee_2007}, 6 cases (7\%) reported complications \cite{Lee_2007}, 1 case (1\%) died \cite{Elghali_2016}.
\paragraph*{Gender} 43 cases (60\%) were male \cite{Akay_2015f, Al-Faham_2020k, Alao_2006i, Ali_2017, Ali_2022g, Apikotoa_2022f, Atayan_2016, Benoist_2019e, Berry_2021e, Bhumi_2024f, CamachoDorado_2018, Csaky_1998e, Emamhadi_2018, Farhadi_2024h, Fry_2010, Gardner_2017h, Guinan_2019f, Jehangir_2019h, Jin_2023, Kobiela_2015, Kumar_2001, Kumar_2019f, Liu_2005, Losanoff_1996, Losanoff_1997e, Mesfin_2022a, Misra_2013, Qureshi_2016, Riva_2018j, Sobnach_2011f, Tammana_2012j, Tanrikulu_2015e, Tay_2004, Thapa_2019f, Trgo_2012f, Wadhwa_2015e, Yasin_2009, teWildt_2010}, 28 cases (39\%) were female \cite{AlShaaibi_2021b, Ali_2020f, Ataya_2013, Beecroft_1998, Bhasin_2014, Bhattacharjee_2008, Cauchi_2002, Chang_2017f, Cox_2007, DelgadoSalazar_2020c, DivsalarP._2023a, Goldman_1998f, Hardy_2023g, Kar_2015, Kariholu_2008, Kerestes_2019, Li_2013, Naji_2012f, Ohno_2005, Peixoto_2017f, Sakellaridis_2008f, Sultan_2024f, Tupesis_2004f, Wildhaber_2005, Wnęk_2015f, Yildiz_2016e}, 1 case (1\%) had no gender recorded \cite{fjbuilsRepeatedBehaviorDeliberate2024}. \paragraph*{Age Group} 25 cases (35\%) were between 26 and 40 years of age \cite{Alao_2006i, Ali_2022g, Apikotoa_2022f, Ataya_2013, Benoist_2019e, Bhasin_2014, Chang_2017f, Cox_2007, DelgadoSalazar_2020c, Farhadi_2024h, Fry_2010, Gardner_2017h, Guinan_2019f, Jin_2023, Kumar_2019f, Losanoff_1996, Misra_2013, Qureshi_2016, Riva_2018j, Sakellaridis_2008f, Tammana_2012j, Trgo_2012f, Wnęk_2015f, Yildiz_2016e, fjbuilsRepeatedBehaviorDeliberate2024}, 18 cases (25\%) were between 18 and 25 years of age \cite{Akay_2015f, Ali_2017, Atayan_2016, Bhattacharjee_2008, Csaky_1998e, Kar_2015, Kariholu_2008, Kobiela_2015, Losanoff_1996, Losanoff_1997e, Mesfin_2022a, Peixoto_2017f, Sobnach_2011f, Tupesis_2004f, Yasin_2009}, 13 cases (18\%) were under 18 years of age \cite{AlShaaibi_2021b, Ali_2020f, Cauchi_2002, DivsalarP._2023a, Goldman_1998f, Liu_2005, Naji_2012f, Ohno_2005, Tanrikulu_2015e, Tay_2004, Wildhaber_2005}, 11 cases (15\%) were between 41 and 60 years of age \cite{Al-Faham_2020k, Bhumi_2024f, CamachoDorado_2018, Emamhadi_2018, Hardy_2023g, Jehangir_2019h, Kumar_2001, Sultan_2024f, Thapa_2019f, Wadhwa_2015e, teWildt_2010}, 3 cases (4\%) were over 60 years of age \cite{Beecroft_1998, Kerestes_2019, Li_2013}, 2 cases (3\%) had no age documented \cite{Berry_2021e}. All 90 were male gender. 90 cases (100\%) were detained at the time of ingestion \cite{Elghali_2016, Karp_1991b, Lee_2007}, 88 cases (98\%) were intentional ingestions \cite{Elghali_2016, Karp_1991b, Lee_2007}, 30 cases (33\%) had a psychiatric history documented \cite{Elghali_2016, Karp_1991b, Lee_2007}, 2 cases (2\%) had a history of prior ingestion \cite{Elghali_2016}. No cases were reported for were psychiatric inpatients, were displaced people, were under the influence of alcohol at the time of ingestion, and had a severe disability history.
\paragraph*{Motivation}  70 cases (78\%) reported protest motivation \cite{Elghali_2016, Karp_1991b, Lee_2007}, 12 cases (13\%) reported psychiatric motivation \cite{Karp_1991b}, 6 cases (7\%) reported self-harm motivation \cite{Elghali_2016, Karp_1991b}. No cases were reported for psychosocial motivation and other motivation.
\paragraph*{Object Characteristics}  68 cases (76\%) involved sharp object ingestion \cite{Elghali_2016, Karp_1991b, Lee_2007}, 32 cases (36\%) involved long (\textgreater 5cm) object ingestion \cite{Lee_2007}, 25 cases (28\%) involved ingestion of multiple objects \cite{Elghali_2016, Lee_2007}. No cases were reported for button battery ingestion, magnet ingestion, and involved large diameter (\textgreater 2.5cm) object ingestion.
\paragraph*{Outcomes}  47 cases (52\%) underwent endoscopic intervention \cite{Elghali_2016, Lee_2007}, 29 cases (32\%) were managed conservatively \cite{Elghali_2016, Karp_1991b}, 15 cases (17\%) underwent surgical intervention \cite{Elghali_2016, Karp_1991b, Lee_2007}, 6 cases (7\%) reported complications \cite{Lee_2007}, 1 case (1\%) died \cite{Elghali_2016}.
\paragraph*{Gender} 43 cases (60\%) were male \cite{Akay_2015f, Al-Faham_2020k, Alao_2006i, Ali_2017, Ali_2022g, Apikotoa_2022f, Atayan_2016, Benoist_2019e, Berry_2021e, Bhumi_2024f, CamachoDorado_2018, Csaky_1998e, Emamhadi_2018, Farhadi_2024h, Fry_2010, Gardner_2017h, Guinan_2019f, Jehangir_2019h, Jin_2023, Kobiela_2015, Kumar_2001, Kumar_2019f, Liu_2005, Losanoff_1996, Losanoff_1997e, Mesfin_2022a, Misra_2013, Qureshi_2016, Riva_2018j, Sobnach_2011f, Tammana_2012j, Tanrikulu_2015e, Tay_2004, Thapa_2019f, Trgo_2012f, Wadhwa_2015e, Yasin_2009, teWildt_2010}, 28 cases (39\%) were female \cite{AlShaaibi_2021b, Ali_2020f, Ataya_2013, Beecroft_1998, Bhasin_2014, Bhattacharjee_2008, Cauchi_2002, Chang_2017f, Cox_2007, DelgadoSalazar_2020c, DivsalarP._2023a, Goldman_1998f, Hardy_2023g, Kar_2015, Kariholu_2008, Kerestes_2019, Li_2013, Naji_2012f, Ohno_2005, Peixoto_2017f, Sakellaridis_2008f, Sultan_2024f, Tupesis_2004f, Wildhaber_2005, Wnęk_2015f, Yildiz_2016e}, 1 case (1\%) had no gender recorded \cite{fjbuilsRepeatedBehaviorDeliberate2024}. \paragraph*{Age Group} 25 cases (35\%) were between 26 and 40 years of age \cite{Alao_2006i, Ali_2022g, Apikotoa_2022f, Ataya_2013, Benoist_2019e, Bhasin_2014, Chang_2017f, Cox_2007, DelgadoSalazar_2020c, Farhadi_2024h, Fry_2010, Gardner_2017h, Guinan_2019f, Jin_2023, Kumar_2019f, Losanoff_1996, Misra_2013, Qureshi_2016, Riva_2018j, Sakellaridis_2008f, Tammana_2012j, Trgo_2012f, Wnęk_2015f, Yildiz_2016e, fjbuilsRepeatedBehaviorDeliberate2024}, 18 cases (25\%) were between 18 and 25 years of age \cite{Akay_2015f, Ali_2017, Atayan_2016, Bhattacharjee_2008, Csaky_1998e, Kar_2015, Kariholu_2008, Kobiela_2015, Losanoff_1996, Losanoff_1997e, Mesfin_2022a, Peixoto_2017f, Sobnach_2011f, Tupesis_2004f, Yasin_2009}, 13 cases (18\%) were under 18 years of age \cite{AlShaaibi_2021b, Ali_2020f, Cauchi_2002, DivsalarP._2023a, Goldman_1998f, Liu_2005, Naji_2012f, Ohno_2005, Tanrikulu_2015e, Tay_2004, Wildhaber_2005}, 11 cases (15\%) were between 41 and 60 years of age \cite{Al-Faham_2020k, Bhumi_2024f, CamachoDorado_2018, Emamhadi_2018, Hardy_2023g, Jehangir_2019h, Kumar_2001, Sultan_2024f, Thapa_2019f, Wadhwa_2015e, teWildt_2010}, 3 cases (4\%) were over 60 years of age \cite{Beecroft_1998, Kerestes_2019, Li_2013}, 2 cases (3\%) had no age documented \cite{Berry_2021e}. All 90 were male gender. 90 cases (100\%) were detained at the time of ingestion \cite{Elghali_2016, Karp_1991b, Lee_2007}, 88 cases (98\%) were intentional ingestions \cite{Elghali_2016, Karp_1991b, Lee_2007}, 30 cases (33\%) had a psychiatric history documented \cite{Elghali_2016, Karp_1991b, Lee_2007}, 2 cases (2\%) had a history of prior ingestion \cite{Elghali_2016}. No cases were reported for were psychiatric inpatients, were displaced people, were under the influence of alcohol at the time of ingestion, and had a severe disability history.
\paragraph*{Motivation}  70 cases (78\%) reported protest motivation \cite{Elghali_2016, Karp_1991b, Lee_2007}, 12 cases (13\%) reported psychiatric motivation \cite{Karp_1991b}, 6 cases (7\%) reported self-harm motivation \cite{Elghali_2016, Karp_1991b}. No cases were reported for psychosocial motivation and other motivation.
\paragraph*{Object Characteristics}  68 cases (76\%) involved sharp object ingestion \cite{Elghali_2016, Karp_1991b, Lee_2007}, 32 cases (36\%) involved long (\textgreater 5cm) object ingestion \cite{Lee_2007}, 25 cases (28\%) involved ingestion of multiple objects \cite{Elghali_2016, Lee_2007}. No cases were reported for button battery ingestion, magnet ingestion, and involved large diameter (\textgreater 2.5cm) object ingestion.
\paragraph*{Outcomes}  47 cases (52\%) underwent endoscopic intervention \cite{Elghali_2016, Lee_2007}, 29 cases (32\%) were managed conservatively \cite{Elghali_2016, Karp_1991b}, 15 cases (17\%) underwent surgical intervention \cite{Elghali_2016, Karp_1991b, Lee_2007}, 6 cases (7\%) reported complications \cite{Lee_2007}, 1 case (1\%) died \cite{Elghali_2016}.
\paragraph*{Gender} 43 cases (60\%) were male \cite{Akay_2015f, Al-Faham_2020k, Alao_2006i, Ali_2017, Ali_2022g, Apikotoa_2022f, Atayan_2016, Benoist_2019e, Berry_2021e, Bhumi_2024f, CamachoDorado_2018, Csaky_1998e, Emamhadi_2018, Farhadi_2024h, Fry_2010, Gardner_2017h, Guinan_2019f, Jehangir_2019h, Jin_2023, Kobiela_2015, Kumar_2001, Kumar_2019f, Liu_2005, Losanoff_1996, Losanoff_1997e, Mesfin_2022a, Misra_2013, Qureshi_2016, Riva_2018j, Sobnach_2011f, Tammana_2012j, Tanrikulu_2015e, Tay_2004, Thapa_2019f, Trgo_2012f, Wadhwa_2015e, Yasin_2009, teWildt_2010}, 28 cases (39\%) were female \cite{AlShaaibi_2021b, Ali_2020f, Ataya_2013, Beecroft_1998, Bhasin_2014, Bhattacharjee_2008, Cauchi_2002, Chang_2017f, Cox_2007, DelgadoSalazar_2020c, DivsalarP._2023a, Goldman_1998f, Hardy_2023g, Kar_2015, Kariholu_2008, Kerestes_2019, Li_2013, Naji_2012f, Ohno_2005, Peixoto_2017f, Sakellaridis_2008f, Sultan_2024f, Tupesis_2004f, Wildhaber_2005, Wnęk_2015f, Yildiz_2016e}, 1 case (1\%) had no gender recorded \cite{fjbuilsRepeatedBehaviorDeliberate2024}. \paragraph*{Age Group} 25 cases (35\%) were between 26 and 40 years of age \cite{Alao_2006i, Ali_2022g, Apikotoa_2022f, Ataya_2013, Benoist_2019e, Bhasin_2014, Chang_2017f, Cox_2007, DelgadoSalazar_2020c, Farhadi_2024h, Fry_2010, Gardner_2017h, Guinan_2019f, Jin_2023, Kumar_2019f, Losanoff_1996, Misra_2013, Qureshi_2016, Riva_2018j, Sakellaridis_2008f, Tammana_2012j, Trgo_2012f, Wnęk_2015f, Yildiz_2016e, fjbuilsRepeatedBehaviorDeliberate2024}, 18 cases (25\%) were between 18 and 25 years of age \cite{Akay_2015f, Ali_2017, Atayan_2016, Bhattacharjee_2008, Csaky_1998e, Kar_2015, Kariholu_2008, Kobiela_2015, Losanoff_1996, Losanoff_1997e, Mesfin_2022a, Peixoto_2017f, Sobnach_2011f, Tupesis_2004f, Yasin_2009}, 13 cases (18\%) were under 18 years of age \cite{AlShaaibi_2021b, Ali_2020f, Cauchi_2002, DivsalarP._2023a, Goldman_1998f, Liu_2005, Naji_2012f, Ohno_2005, Tanrikulu_2015e, Tay_2004, Wildhaber_2005}, 11 cases (15\%) were between 41 and 60 years of age \cite{Al-Faham_2020k, Bhumi_2024f, CamachoDorado_2018, Emamhadi_2018, Hardy_2023g, Jehangir_2019h, Kumar_2001, Sultan_2024f, Thapa_2019f, Wadhwa_2015e, teWildt_2010}, 3 cases (4\%) were over 60 years of age \cite{Beecroft_1998, Kerestes_2019, Li_2013}, 2 cases (3\%) had no age documented \cite{Berry_2021e}. All 90 were male gender. 90 cases (100\%) were detained at the time of ingestion \cite{Elghali_2016, Karp_1991b, Lee_2007}, 88 cases (98\%) were intentional ingestions \cite{Elghali_2016, Karp_1991b, Lee_2007}, 30 cases (33\%) had a psychiatric history documented \cite{Elghali_2016, Karp_1991b, Lee_2007}, 2 cases (2\%) had a history of prior ingestion \cite{Elghali_2016}. No cases were reported for were psychiatric inpatients, were displaced people, were under the influence of alcohol at the time of ingestion, and had a severe disability history.
\paragraph*{Motivation}  70 cases (78\%) reported protest motivation \cite{Elghali_2016, Karp_1991b, Lee_2007}, 12 cases (13\%) reported psychiatric motivation \cite{Karp_1991b}, 6 cases (7\%) reported self-harm motivation \cite{Elghali_2016, Karp_1991b}. No cases were reported for psychosocial motivation and other motivation.
\paragraph*{Object Characteristics}  68 cases (76\%) involved sharp object ingestion \cite{Elghali_2016, Karp_1991b, Lee_2007}, 32 cases (36\%) involved long (\textgreater 5cm) object ingestion \cite{Lee_2007}, 25 cases (28\%) involved ingestion of multiple objects \cite{Elghali_2016, Lee_2007}. No cases were reported for button battery ingestion, magnet ingestion, and involved large diameter (\textgreater 2.5cm) object ingestion.
\paragraph*{Outcomes}  47 cases (52\%) underwent endoscopic intervention \cite{Elghali_2016, Lee_2007}, 29 cases (32\%) were managed conservatively \cite{Elghali_2016, Karp_1991b}, 15 cases (17\%) underwent surgical intervention \cite{Elghali_2016, Karp_1991b, Lee_2007}, 6 cases (7\%) reported complications \cite{Lee_2007}, 1 case (1\%) died \cite{Elghali_2016}.
\paragraph*{Gender} 43 cases (60\%) were male \cite{Akay_2015f, Al-Faham_2020k, Alao_2006i, Ali_2017, Ali_2022g, Apikotoa_2022f, Atayan_2016, Benoist_2019e, Berry_2021e, Bhumi_2024f, CamachoDorado_2018, Csaky_1998e, Emamhadi_2018, Farhadi_2024h, Fry_2010, Gardner_2017h, Guinan_2019f, Jehangir_2019h, Jin_2023, Kobiela_2015, Kumar_2001, Kumar_2019f, Liu_2005, Losanoff_1996, Losanoff_1997e, Mesfin_2022a, Misra_2013, Qureshi_2016, Riva_2018j, Sobnach_2011f, Tammana_2012j, Tanrikulu_2015e, Tay_2004, Thapa_2019f, Trgo_2012f, Wadhwa_2015e, Yasin_2009, teWildt_2010}, 28 cases (39\%) were female \cite{AlShaaibi_2021b, Ali_2020f, Ataya_2013, Beecroft_1998, Bhasin_2014, Bhattacharjee_2008, Cauchi_2002, Chang_2017f, Cox_2007, DelgadoSalazar_2020c, DivsalarP._2023a, Goldman_1998f, Hardy_2023g, Kar_2015, Kariholu_2008, Kerestes_2019, Li_2013, Naji_2012f, Ohno_2005, Peixoto_2017f, Sakellaridis_2008f, Sultan_2024f, Tupesis_2004f, Wildhaber_2005, Wnęk_2015f, Yildiz_2016e}, 1 case (1\%) had no gender recorded \cite{fjbuilsRepeatedBehaviorDeliberate2024}. \paragraph*{Age Group} 25 cases (35\%) were between 26 and 40 years of age \cite{Alao_2006i, Ali_2022g, Apikotoa_2022f, Ataya_2013, Benoist_2019e, Bhasin_2014, Chang_2017f, Cox_2007, DelgadoSalazar_2020c, Farhadi_2024h, Fry_2010, Gardner_2017h, Guinan_2019f, Jin_2023, Kumar_2019f, Losanoff_1996, Misra_2013, Qureshi_2016, Riva_2018j, Sakellaridis_2008f, Tammana_2012j, Trgo_2012f, Wnęk_2015f, Yildiz_2016e, fjbuilsRepeatedBehaviorDeliberate2024}, 18 cases (25\%) were between 18 and 25 years of age \cite{Akay_2015f, Ali_2017, Atayan_2016, Bhattacharjee_2008, Csaky_1998e, Kar_2015, Kariholu_2008, Kobiela_2015, Losanoff_1996, Losanoff_1997e, Mesfin_2022a, Peixoto_2017f, Sobnach_2011f, Tupesis_2004f, Yasin_2009}, 13 cases (18\%) were under 18 years of age \cite{AlShaaibi_2021b, Ali_2020f, Cauchi_2002, DivsalarP._2023a, Goldman_1998f, Liu_2005, Naji_2012f, Ohno_2005, Tanrikulu_2015e, Tay_2004, Wildhaber_2005}, 11 cases (15\%) were between 41 and 60 years of age \cite{Al-Faham_2020k, Bhumi_2024f, CamachoDorado_2018, Emamhadi_2018, Hardy_2023g, Jehangir_2019h, Kumar_2001, Sultan_2024f, Thapa_2019f, Wadhwa_2015e, teWildt_2010}, 3 cases (4\%) were over 60 years of age \cite{Beecroft_1998, Kerestes_2019, Li_2013}, 2 cases (3\%) had no age documented \cite{Berry_2021e}. All 90 were male gender. 90 cases (100\%) were detained at the time of ingestion \cite{Elghali_2016, Karp_1991b, Lee_2007}, 88 cases (98\%) were intentional ingestions \cite{Elghali_2016, Karp_1991b, Lee_2007}, 30 cases (33\%) had a psychiatric history documented \cite{Elghali_2016, Karp_1991b, Lee_2007}, 2 cases (2\%) had a history of prior ingestion \cite{Elghali_2016}. No cases were reported for were psychiatric inpatients, were displaced people, were under the influence of alcohol at the time of ingestion, and had a severe disability history.
\paragraph*{Motivation}  70 cases (78\%) reported protest motivation \cite{Elghali_2016, Karp_1991b, Lee_2007}, 12 cases (13\%) reported psychiatric motivation \cite{Karp_1991b}, 6 cases (7\%) reported self-harm motivation \cite{Elghali_2016, Karp_1991b}. No cases were reported for psychosocial motivation and other motivation.
\paragraph*{Object Characteristics}  68 cases (76\%) involved sharp object ingestion \cite{Elghali_2016, Karp_1991b, Lee_2007}, 32 cases (36\%) involved long (\textgreater 5cm) object ingestion \cite{Lee_2007}, 25 cases (28\%) involved ingestion of multiple objects \cite{Elghali_2016, Lee_2007}. No cases were reported for button battery ingestion, magnet ingestion, and involved large diameter (\textgreater 2.5cm) object ingestion.
\paragraph*{Outcomes}  47 cases (52\%) underwent endoscopic intervention \cite{Elghali_2016, Lee_2007}, 29 cases (32\%) were managed conservatively \cite{Elghali_2016, Karp_1991b}, 15 cases (17\%) underwent surgical intervention \cite{Elghali_2016, Karp_1991b, Lee_2007}, 6 cases (7\%) reported complications \cite{Lee_2007}, 1 case (1\%) died \cite{Elghali_2016}.
\paragraph*{Gender} 43 cases (60\%) were male \cite{Akay_2015f, Al-Faham_2020k, Alao_2006i, Ali_2017, Ali_2022g, Apikotoa_2022f, Atayan_2016, Benoist_2019e, Berry_2021e, Bhumi_2024f, CamachoDorado_2018, Csaky_1998e, Emamhadi_2018, Farhadi_2024h, Fry_2010, Gardner_2017h, Guinan_2019f, Jehangir_2019h, Jin_2023, Kobiela_2015, Kumar_2001, Kumar_2019f, Liu_2005, Losanoff_1996, Losanoff_1997e, Mesfin_2022a, Misra_2013, Qureshi_2016, Riva_2018j, Sobnach_2011f, Tammana_2012j, Tanrikulu_2015e, Tay_2004, Thapa_2019f, Trgo_2012f, Wadhwa_2015e, Yasin_2009, teWildt_2010}, 28 cases (39\%) were female \cite{AlShaaibi_2021b, Ali_2020f, Ataya_2013, Beecroft_1998, Bhasin_2014, Bhattacharjee_2008, Cauchi_2002, Chang_2017f, Cox_2007, DelgadoSalazar_2020c, DivsalarP._2023a, Goldman_1998f, Hardy_2023g, Kar_2015, Kariholu_2008, Kerestes_2019, Li_2013, Naji_2012f, Ohno_2005, Peixoto_2017f, Sakellaridis_2008f, Sultan_2024f, Tupesis_2004f, Wildhaber_2005, Wnęk_2015f, Yildiz_2016e}, 1 case (1\%) had no gender recorded \cite{fjbuilsRepeatedBehaviorDeliberate2024}. \paragraph*{Age Group} 25 cases (35\%) were between 26 and 40 years of age \cite{Alao_2006i, Ali_2022g, Apikotoa_2022f, Ataya_2013, Benoist_2019e, Bhasin_2014, Chang_2017f, Cox_2007, DelgadoSalazar_2020c, Farhadi_2024h, Fry_2010, Gardner_2017h, Guinan_2019f, Jin_2023, Kumar_2019f, Losanoff_1996, Misra_2013, Qureshi_2016, Riva_2018j, Sakellaridis_2008f, Tammana_2012j, Trgo_2012f, Wnęk_2015f, Yildiz_2016e, fjbuilsRepeatedBehaviorDeliberate2024}, 18 cases (25\%) were between 18 and 25 years of age \cite{Akay_2015f, Ali_2017, Atayan_2016, Bhattacharjee_2008, Csaky_1998e, Kar_2015, Kariholu_2008, Kobiela_2015, Losanoff_1996, Losanoff_1997e, Mesfin_2022a, Peixoto_2017f, Sobnach_2011f, Tupesis_2004f, Yasin_2009}, 13 cases (18\%) were under 18 years of age \cite{AlShaaibi_2021b, Ali_2020f, Cauchi_2002, DivsalarP._2023a, Goldman_1998f, Liu_2005, Naji_2012f, Ohno_2005, Tanrikulu_2015e, Tay_2004, Wildhaber_2005}, 11 cases (15\%) were between 41 and 60 years of age \cite{Al-Faham_2020k, Bhumi_2024f, CamachoDorado_2018, Emamhadi_2018, Hardy_2023g, Jehangir_2019h, Kumar_2001, Sultan_2024f, Thapa_2019f, Wadhwa_2015e, teWildt_2010}, 3 cases (4\%) were over 60 years of age \cite{Beecroft_1998, Kerestes_2019, Li_2013}, 2 cases (3\%) had no age documented \cite{Berry_2021e}. All 90 were male gender. 90 cases (100\%) were detained at the time of ingestion \cite{Elghali_2016, Karp_1991b, Lee_2007}, 88 cases (98\%) were intentional ingestions \cite{Elghali_2016, Karp_1991b, Lee_2007}, 30 cases (33\%) had a psychiatric history documented \cite{Elghali_2016, Karp_1991b, Lee_2007}, 2 cases (2\%) had a history of prior ingestion \cite{Elghali_2016}. No cases were reported for were psychiatric inpatients, were displaced people, were under the influence of alcohol at the time of ingestion, and had a severe disability history.
\paragraph*{Motivation}  70 cases (78\%) reported protest motivation \cite{Elghali_2016, Karp_1991b, Lee_2007}, 12 cases (13\%) reported psychiatric motivation \cite{Karp_1991b}, 6 cases (7\%) reported self-harm motivation \cite{Elghali_2016, Karp_1991b}. No cases were reported for psychosocial motivation and other motivation.
\paragraph*{Object Characteristics}  68 cases (76\%) involved sharp object ingestion \cite{Elghali_2016, Karp_1991b, Lee_2007}, 32 cases (36\%) involved long (\textgreater 5cm) object ingestion \cite{Lee_2007}, 25 cases (28\%) involved ingestion of multiple objects \cite{Elghali_2016, Lee_2007}. No cases were reported for button battery ingestion, magnet ingestion, and involved large diameter (\textgreater 2.5cm) object ingestion.
\paragraph*{Outcomes}  47 cases (52\%) underwent endoscopic intervention \cite{Elghali_2016, Lee_2007}, 29 cases (32\%) were managed conservatively \cite{Elghali_2016, Karp_1991b}, 15 cases (17\%) underwent surgical intervention \cite{Elghali_2016, Karp_1991b, Lee_2007}, 6 cases (7\%) reported complications \cite{Lee_2007}, 1 case (1\%) died \cite{Elghali_2016}.
\paragraph*{Gender} 43 cases (60\%) were male \cite{Akay_2015f, Al-Faham_2020k, Alao_2006i, Ali_2017, Ali_2022g, Apikotoa_2022f, Atayan_2016, Benoist_2019e, Berry_2021e, Bhumi_2024f, CamachoDorado_2018, Csaky_1998e, Emamhadi_2018, Farhadi_2024h, Fry_2010, Gardner_2017h, Guinan_2019f, Jehangir_2019h, Jin_2023, Kobiela_2015, Kumar_2001, Kumar_2019f, Liu_2005, Losanoff_1996, Losanoff_1997e, Mesfin_2022a, Misra_2013, Qureshi_2016, Riva_2018j, Sobnach_2011f, Tammana_2012j, Tanrikulu_2015e, Tay_2004, Thapa_2019f, Trgo_2012f, Wadhwa_2015e, Yasin_2009, teWildt_2010}, 28 cases (39\%) were female \cite{AlShaaibi_2021b, Ali_2020f, Ataya_2013, Beecroft_1998, Bhasin_2014, Bhattacharjee_2008, Cauchi_2002, Chang_2017f, Cox_2007, DelgadoSalazar_2020c, DivsalarP._2023a, Goldman_1998f, Hardy_2023g, Kar_2015, Kariholu_2008, Kerestes_2019, Li_2013, Naji_2012f, Ohno_2005, Peixoto_2017f, Sakellaridis_2008f, Sultan_2024f, Tupesis_2004f, Wildhaber_2005, Wnęk_2015f, Yildiz_2016e}, 1 case (1\%) had no gender recorded \cite{fjbuilsRepeatedBehaviorDeliberate2024}. \paragraph*{Age Group} 25 cases (35\%) were between 26 and 40 years of age \cite{Alao_2006i, Ali_2022g, Apikotoa_2022f, Ataya_2013, Benoist_2019e, Bhasin_2014, Chang_2017f, Cox_2007, DelgadoSalazar_2020c, Farhadi_2024h, Fry_2010, Gardner_2017h, Guinan_2019f, Jin_2023, Kumar_2019f, Losanoff_1996, Misra_2013, Qureshi_2016, Riva_2018j, Sakellaridis_2008f, Tammana_2012j, Trgo_2012f, Wnęk_2015f, Yildiz_2016e, fjbuilsRepeatedBehaviorDeliberate2024}, 18 cases (25\%) were between 18 and 25 years of age \cite{Akay_2015f, Ali_2017, Atayan_2016, Bhattacharjee_2008, Csaky_1998e, Kar_2015, Kariholu_2008, Kobiela_2015, Losanoff_1996, Losanoff_1997e, Mesfin_2022a, Peixoto_2017f, Sobnach_2011f, Tupesis_2004f, Yasin_2009}, 13 cases (18\%) were under 18 years of age \cite{AlShaaibi_2021b, Ali_2020f, Cauchi_2002, DivsalarP._2023a, Goldman_1998f, Liu_2005, Naji_2012f, Ohno_2005, Tanrikulu_2015e, Tay_2004, Wildhaber_2005}, 11 cases (15\%) were between 41 and 60 years of age \cite{Al-Faham_2020k, Bhumi_2024f, CamachoDorado_2018, Emamhadi_2018, Hardy_2023g, Jehangir_2019h, Kumar_2001, Sultan_2024f, Thapa_2019f, Wadhwa_2015e, teWildt_2010}, 3 cases (4\%) were over 60 years of age \cite{Beecroft_1998, Kerestes_2019, Li_2013}, 2 cases (3\%) had no age documented \cite{Berry_2021e}. All 90 were male gender. 90 cases (100\%) were detained at the time of ingestion \cite{Elghali_2016, Karp_1991b, Lee_2007}, 88 cases (98\%) were intentional ingestions \cite{Elghali_2016, Karp_1991b, Lee_2007}, 30 cases (33\%) had a psychiatric history documented \cite{Elghali_2016, Karp_1991b, Lee_2007}, 2 cases (2\%) had a history of prior ingestion \cite{Elghali_2016}. No cases were reported for were psychiatric inpatients, were displaced people, were under the influence of alcohol at the time of ingestion, and had a severe disability history.
\paragraph*{Motivation}  70 cases (78\%) reported protest motivation \cite{Elghali_2016, Karp_1991b, Lee_2007}, 12 cases (13\%) reported psychiatric motivation \cite{Karp_1991b}, 6 cases (7\%) reported self-harm motivation \cite{Elghali_2016, Karp_1991b}. No cases were reported for psychosocial motivation and other motivation.
\paragraph*{Object Characteristics}  68 cases (76\%) involved sharp object ingestion \cite{Elghali_2016, Karp_1991b, Lee_2007}, 32 cases (36\%) involved long (\textgreater 5cm) object ingestion \cite{Lee_2007}, 25 cases (28\%) involved ingestion of multiple objects \cite{Elghali_2016, Lee_2007}. No cases were reported for button battery ingestion, magnet ingestion, and involved large diameter (\textgreater 2.5cm) object ingestion.
\paragraph*{Outcomes}  47 cases (52\%) underwent endoscopic intervention \cite{Elghali_2016, Lee_2007}, 29 cases (32\%) were managed conservatively \cite{Elghali_2016, Karp_1991b}, 15 cases (17\%) underwent surgical intervention \cite{Elghali_2016, Karp_1991b, Lee_2007}, 6 cases (7\%) reported complications \cite{Lee_2007}, 1 case (1\%) died \cite{Elghali_2016}.
\paragraph*{Gender} 43 cases (60\%) were male \cite{Akay_2015f, Al-Faham_2020k, Alao_2006i, Ali_2017, Ali_2022g, Apikotoa_2022f, Atayan_2016, Benoist_2019e, Berry_2021e, Bhumi_2024f, CamachoDorado_2018, Csaky_1998e, Emamhadi_2018, Farhadi_2024h, Fry_2010, Gardner_2017h, Guinan_2019f, Jehangir_2019h, Jin_2023, Kobiela_2015, Kumar_2001, Kumar_2019f, Liu_2005, Losanoff_1996, Losanoff_1997e, Mesfin_2022a, Misra_2013, Qureshi_2016, Riva_2018j, Sobnach_2011f, Tammana_2012j, Tanrikulu_2015e, Tay_2004, Thapa_2019f, Trgo_2012f, Wadhwa_2015e, Yasin_2009, teWildt_2010}, 28 cases (39\%) were female \cite{AlShaaibi_2021b, Ali_2020f, Ataya_2013, Beecroft_1998, Bhasin_2014, Bhattacharjee_2008, Cauchi_2002, Chang_2017f, Cox_2007, DelgadoSalazar_2020c, DivsalarP._2023a, Goldman_1998f, Hardy_2023g, Kar_2015, Kariholu_2008, Kerestes_2019, Li_2013, Naji_2012f, Ohno_2005, Peixoto_2017f, Sakellaridis_2008f, Sultan_2024f, Tupesis_2004f, Wildhaber_2005, Wnęk_2015f, Yildiz_2016e}, 1 case (1\%) had no gender recorded \cite{fjbuilsRepeatedBehaviorDeliberate2024}. \paragraph*{Age Group} 25 cases (35\%) were between 26 and 40 years of age \cite{Alao_2006i, Ali_2022g, Apikotoa_2022f, Ataya_2013, Benoist_2019e, Bhasin_2014, Chang_2017f, Cox_2007, DelgadoSalazar_2020c, Farhadi_2024h, Fry_2010, Gardner_2017h, Guinan_2019f, Jin_2023, Kumar_2019f, Losanoff_1996, Misra_2013, Qureshi_2016, Riva_2018j, Sakellaridis_2008f, Tammana_2012j, Trgo_2012f, Wnęk_2015f, Yildiz_2016e, fjbuilsRepeatedBehaviorDeliberate2024}, 18 cases (25\%) were between 18 and 25 years of age \cite{Akay_2015f, Ali_2017, Atayan_2016, Bhattacharjee_2008, Csaky_1998e, Kar_2015, Kariholu_2008, Kobiela_2015, Losanoff_1996, Losanoff_1997e, Mesfin_2022a, Peixoto_2017f, Sobnach_2011f, Tupesis_2004f, Yasin_2009}, 13 cases (18\%) were under 18 years of age \cite{AlShaaibi_2021b, Ali_2020f, Cauchi_2002, DivsalarP._2023a, Goldman_1998f, Liu_2005, Naji_2012f, Ohno_2005, Tanrikulu_2015e, Tay_2004, Wildhaber_2005}, 11 cases (15\%) were between 41 and 60 years of age \cite{Al-Faham_2020k, Bhumi_2024f, CamachoDorado_2018, Emamhadi_2018, Hardy_2023g, Jehangir_2019h, Kumar_2001, Sultan_2024f, Thapa_2019f, Wadhwa_2015e, teWildt_2010}, 3 cases (4\%) were over 60 years of age \cite{Beecroft_1998, Kerestes_2019, Li_2013}, 2 cases (3\%) had no age documented \cite{Berry_2021e}. All 90 were male gender. 90 cases (100\%) were detained at the time of ingestion \cite{Elghali_2016, Karp_1991b, Lee_2007}, 88 cases (98\%) were intentional ingestions \cite{Elghali_2016, Karp_1991b, Lee_2007}, 30 cases (33\%) had a psychiatric history documented \cite{Elghali_2016, Karp_1991b, Lee_2007}, 2 cases (2\%) had a history of prior ingestion \cite{Elghali_2016}. No cases were reported for were psychiatric inpatients, were displaced people, were under the influence of alcohol at the time of ingestion, and had a severe disability history.
\paragraph*{Motivation}  70 cases (78\%) reported protest motivation \cite{Elghali_2016, Karp_1991b, Lee_2007}, 12 cases (13\%) reported psychiatric motivation \cite{Karp_1991b}, 6 cases (7\%) reported self-harm motivation \cite{Elghali_2016, Karp_1991b}. No cases were reported for psychosocial motivation and other motivation.
\paragraph*{Object Characteristics}  68 cases (76\%) involved sharp object ingestion \cite{Elghali_2016, Karp_1991b, Lee_2007}, 32 cases (36\%) involved long (\textgreater 5cm) object ingestion \cite{Lee_2007}, 25 cases (28\%) involved ingestion of multiple objects \cite{Elghali_2016, Lee_2007}. No cases were reported for button battery ingestion, magnet ingestion, and involved large diameter (\textgreater 2.5cm) object ingestion.
\paragraph*{Outcomes}  47 cases (52\%) underwent endoscopic intervention \cite{Elghali_2016, Lee_2007}, 29 cases (32\%) were managed conservatively \cite{Elghali_2016, Karp_1991b}, 15 cases (17\%) underwent surgical intervention \cite{Elghali_2016, Karp_1991b, Lee_2007}, 6 cases (7\%) reported complications \cite{Lee_2007}, 1 case (1\%) died \cite{Elghali_2016}.
\paragraph*{Gender} 43 cases (60\%) were male \cite{Akay_2015f, Al-Faham_2020k, Alao_2006i, Ali_2017, Ali_2022g, Apikotoa_2022f, Atayan_2016, Benoist_2019e, Berry_2021e, Bhumi_2024f, CamachoDorado_2018, Csaky_1998e, Emamhadi_2018, Farhadi_2024h, Fry_2010, Gardner_2017h, Guinan_2019f, Jehangir_2019h, Jin_2023, Kobiela_2015, Kumar_2001, Kumar_2019f, Liu_2005, Losanoff_1996, Losanoff_1997e, Mesfin_2022a, Misra_2013, Qureshi_2016, Riva_2018j, Sobnach_2011f, Tammana_2012j, Tanrikulu_2015e, Tay_2004, Thapa_2019f, Trgo_2012f, Wadhwa_2015e, Yasin_2009, teWildt_2010}, 28 cases (39\%) were female \cite{AlShaaibi_2021b, Ali_2020f, Ataya_2013, Beecroft_1998, Bhasin_2014, Bhattacharjee_2008, Cauchi_2002, Chang_2017f, Cox_2007, DelgadoSalazar_2020c, DivsalarP._2023a, Goldman_1998f, Hardy_2023g, Kar_2015, Kariholu_2008, Kerestes_2019, Li_2013, Naji_2012f, Ohno_2005, Peixoto_2017f, Sakellaridis_2008f, Sultan_2024f, Tupesis_2004f, Wildhaber_2005, Wnęk_2015f, Yildiz_2016e}, 1 case (1\%) had no gender recorded \cite{fjbuilsRepeatedBehaviorDeliberate2024}. \paragraph*{Age Group} 25 cases (35\%) were between 26 and 40 years of age \cite{Alao_2006i, Ali_2022g, Apikotoa_2022f, Ataya_2013, Benoist_2019e, Bhasin_2014, Chang_2017f, Cox_2007, DelgadoSalazar_2020c, Farhadi_2024h, Fry_2010, Gardner_2017h, Guinan_2019f, Jin_2023, Kumar_2019f, Losanoff_1996, Misra_2013, Qureshi_2016, Riva_2018j, Sakellaridis_2008f, Tammana_2012j, Trgo_2012f, Wnęk_2015f, Yildiz_2016e, fjbuilsRepeatedBehaviorDeliberate2024}, 18 cases (25\%) were between 18 and 25 years of age \cite{Akay_2015f, Ali_2017, Atayan_2016, Bhattacharjee_2008, Csaky_1998e, Kar_2015, Kariholu_2008, Kobiela_2015, Losanoff_1996, Losanoff_1997e, Mesfin_2022a, Peixoto_2017f, Sobnach_2011f, Tupesis_2004f, Yasin_2009}, 13 cases (18\%) were under 18 years of age \cite{AlShaaibi_2021b, Ali_2020f, Cauchi_2002, DivsalarP._2023a, Goldman_1998f, Liu_2005, Naji_2012f, Ohno_2005, Tanrikulu_2015e, Tay_2004, Wildhaber_2005}, 11 cases (15\%) were between 41 and 60 years of age \cite{Al-Faham_2020k, Bhumi_2024f, CamachoDorado_2018, Emamhadi_2018, Hardy_2023g, Jehangir_2019h, Kumar_2001, Sultan_2024f, Thapa_2019f, Wadhwa_2015e, teWildt_2010}, 3 cases (4\%) were over 60 years of age \cite{Beecroft_1998, Kerestes_2019, Li_2013}, 2 cases (3\%) had no age documented \cite{Berry_2021e}. All 90 were male gender. 90 cases (100\%) were detained at the time of ingestion \cite{Elghali_2016, Karp_1991b, Lee_2007}, 88 cases (98\%) were intentional ingestions \cite{Elghali_2016, Karp_1991b, Lee_2007}, 30 cases (33\%) had a psychiatric history documented \cite{Elghali_2016, Karp_1991b, Lee_2007}, 2 cases (2\%) had a history of prior ingestion \cite{Elghali_2016}. No cases were reported for were psychiatric inpatients, were displaced people, were under the influence of alcohol at the time of ingestion, and had a severe disability history.
\paragraph*{Motivation}  70 cases (78\%) reported protest motivation \cite{Elghali_2016, Karp_1991b, Lee_2007}, 12 cases (13\%) reported psychiatric motivation \cite{Karp_1991b}, 6 cases (7\%) reported self-harm motivation \cite{Elghali_2016, Karp_1991b}. No cases were reported for psychosocial motivation and other motivation.
\paragraph*{Object Characteristics}  68 cases (76\%) involved sharp object ingestion \cite{Elghali_2016, Karp_1991b, Lee_2007}, 32 cases (36\%) involved long (\textgreater 5cm) object ingestion \cite{Lee_2007}, 25 cases (28\%) involved ingestion of multiple objects \cite{Elghali_2016, Lee_2007}. No cases were reported for button battery ingestion, magnet ingestion, and involved large diameter (\textgreater 2.5cm) object ingestion.
\paragraph*{Outcomes}  47 cases (52\%) underwent endoscopic intervention \cite{Elghali_2016, Lee_2007}, 29 cases (32\%) were managed conservatively \cite{Elghali_2016, Karp_1991b}, 15 cases (17\%) underwent surgical intervention \cite{Elghali_2016, Karp_1991b, Lee_2007}, 6 cases (7\%) reported complications \cite{Lee_2007}, 1 case (1\%) died \cite{Elghali_2016}.
\paragraph*{Gender} 43 cases (60\%) were male \cite{Akay_2015f, Al-Faham_2020k, Alao_2006i, Ali_2017, Ali_2022g, Apikotoa_2022f, Atayan_2016, Benoist_2019e, Berry_2021e, Bhumi_2024f, CamachoDorado_2018, Csaky_1998e, Emamhadi_2018, Farhadi_2024h, Fry_2010, Gardner_2017h, Guinan_2019f, Jehangir_2019h, Jin_2023, Kobiela_2015, Kumar_2001, Kumar_2019f, Liu_2005, Losanoff_1996, Losanoff_1997e, Mesfin_2022a, Misra_2013, Qureshi_2016, Riva_2018j, Sobnach_2011f, Tammana_2012j, Tanrikulu_2015e, Tay_2004, Thapa_2019f, Trgo_2012f, Wadhwa_2015e, Yasin_2009, teWildt_2010}, 28 cases (39\%) were female \cite{AlShaaibi_2021b, Ali_2020f, Ataya_2013, Beecroft_1998, Bhasin_2014, Bhattacharjee_2008, Cauchi_2002, Chang_2017f, Cox_2007, DelgadoSalazar_2020c, DivsalarP._2023a, Goldman_1998f, Hardy_2023g, Kar_2015, Kariholu_2008, Kerestes_2019, Li_2013, Naji_2012f, Ohno_2005, Peixoto_2017f, Sakellaridis_2008f, Sultan_2024f, Tupesis_2004f, Wildhaber_2005, Wnęk_2015f, Yildiz_2016e}, 1 case (1\%) had no gender recorded \cite{fjbuilsRepeatedBehaviorDeliberate2024}. \paragraph*{Age Group} 25 cases (35\%) were between 26 and 40 years of age \cite{Alao_2006i, Ali_2022g, Apikotoa_2022f, Ataya_2013, Benoist_2019e, Bhasin_2014, Chang_2017f, Cox_2007, DelgadoSalazar_2020c, Farhadi_2024h, Fry_2010, Gardner_2017h, Guinan_2019f, Jin_2023, Kumar_2019f, Losanoff_1996, Misra_2013, Qureshi_2016, Riva_2018j, Sakellaridis_2008f, Tammana_2012j, Trgo_2012f, Wnęk_2015f, Yildiz_2016e, fjbuilsRepeatedBehaviorDeliberate2024}, 18 cases (25\%) were between 18 and 25 years of age \cite{Akay_2015f, Ali_2017, Atayan_2016, Bhattacharjee_2008, Csaky_1998e, Kar_2015, Kariholu_2008, Kobiela_2015, Losanoff_1996, Losanoff_1997e, Mesfin_2022a, Peixoto_2017f, Sobnach_2011f, Tupesis_2004f, Yasin_2009}, 13 cases (18\%) were under 18 years of age \cite{AlShaaibi_2021b, Ali_2020f, Cauchi_2002, DivsalarP._2023a, Goldman_1998f, Liu_2005, Naji_2012f, Ohno_2005, Tanrikulu_2015e, Tay_2004, Wildhaber_2005}, 11 cases (15\%) were between 41 and 60 years of age \cite{Al-Faham_2020k, Bhumi_2024f, CamachoDorado_2018, Emamhadi_2018, Hardy_2023g, Jehangir_2019h, Kumar_2001, Sultan_2024f, Thapa_2019f, Wadhwa_2015e, teWildt_2010}, 3 cases (4\%) were over 60 years of age \cite{Beecroft_1998, Kerestes_2019, Li_2013}, 2 cases (3\%) had no age documented \cite{Berry_2021e}. All 90 were male gender. 90 cases (100\%) were detained at the time of ingestion \cite{Elghali_2016, Karp_1991b, Lee_2007}, 88 cases (98\%) were intentional ingestions \cite{Elghali_2016, Karp_1991b, Lee_2007}, 30 cases (33\%) had a psychiatric history documented \cite{Elghali_2016, Karp_1991b, Lee_2007}, 2 cases (2\%) had a history of prior ingestion \cite{Elghali_2016}. No cases were reported for were psychiatric inpatients, were displaced people, were under the influence of alcohol at the time of ingestion, and had a severe disability history.
\paragraph*{Motivation}  70 cases (78\%) reported protest motivation \cite{Elghali_2016, Karp_1991b, Lee_2007}, 12 cases (13\%) reported psychiatric motivation \cite{Karp_1991b}, 6 cases (7\%) reported self-harm motivation \cite{Elghali_2016, Karp_1991b}. No cases were reported for psychosocial motivation and other motivation.
\paragraph*{Object Characteristics}  68 cases (76\%) involved sharp object ingestion \cite{Elghali_2016, Karp_1991b, Lee_2007}, 32 cases (36\%) involved long (\textgreater 5cm) object ingestion \cite{Lee_2007}, 25 cases (28\%) involved ingestion of multiple objects \cite{Elghali_2016, Lee_2007}. No cases were reported for button battery ingestion, magnet ingestion, and involved large diameter (\textgreater 2.5cm) object ingestion.
\paragraph*{Outcomes}  47 cases (52\%) underwent endoscopic intervention \cite{Elghali_2016, Lee_2007}, 29 cases (32\%) were managed conservatively \cite{Elghali_2016, Karp_1991b}, 15 cases (17\%) underwent surgical intervention \cite{Elghali_2016, Karp_1991b, Lee_2007}, 6 cases (7\%) reported complications \cite{Lee_2007}, 1 case (1\%) died \cite{Elghali_2016}.
\paragraph*{Gender} 43 cases (60\%) were male \cite{Akay_2015f, Al-Faham_2020k, Alao_2006i, Ali_2017, Ali_2022g, Apikotoa_2022f, Atayan_2016, Benoist_2019e, Berry_2021e, Bhumi_2024f, CamachoDorado_2018, Csaky_1998e, Emamhadi_2018, Farhadi_2024h, Fry_2010, Gardner_2017h, Guinan_2019f, Jehangir_2019h, Jin_2023, Kobiela_2015, Kumar_2001, Kumar_2019f, Liu_2005, Losanoff_1996, Losanoff_1997e, Mesfin_2022a, Misra_2013, Qureshi_2016, Riva_2018j, Sobnach_2011f, Tammana_2012j, Tanrikulu_2015e, Tay_2004, Thapa_2019f, Trgo_2012f, Wadhwa_2015e, Yasin_2009, teWildt_2010}, 28 cases (39\%) were female \cite{AlShaaibi_2021b, Ali_2020f, Ataya_2013, Beecroft_1998, Bhasin_2014, Bhattacharjee_2008, Cauchi_2002, Chang_2017f, Cox_2007, DelgadoSalazar_2020c, DivsalarP._2023a, Goldman_1998f, Hardy_2023g, Kar_2015, Kariholu_2008, Kerestes_2019, Li_2013, Naji_2012f, Ohno_2005, Peixoto_2017f, Sakellaridis_2008f, Sultan_2024f, Tupesis_2004f, Wildhaber_2005, Wnęk_2015f, Yildiz_2016e}, 1 case (1\%) had no gender recorded \cite{fjbuilsRepeatedBehaviorDeliberate2024}. \paragraph*{Age Group} 25 cases (35\%) were between 26 and 40 years of age \cite{Alao_2006i, Ali_2022g, Apikotoa_2022f, Ataya_2013, Benoist_2019e, Bhasin_2014, Chang_2017f, Cox_2007, DelgadoSalazar_2020c, Farhadi_2024h, Fry_2010, Gardner_2017h, Guinan_2019f, Jin_2023, Kumar_2019f, Losanoff_1996, Misra_2013, Qureshi_2016, Riva_2018j, Sakellaridis_2008f, Tammana_2012j, Trgo_2012f, Wnęk_2015f, Yildiz_2016e, fjbuilsRepeatedBehaviorDeliberate2024}, 18 cases (25\%) were between 18 and 25 years of age \cite{Akay_2015f, Ali_2017, Atayan_2016, Bhattacharjee_2008, Csaky_1998e, Kar_2015, Kariholu_2008, Kobiela_2015, Losanoff_1996, Losanoff_1997e, Mesfin_2022a, Peixoto_2017f, Sobnach_2011f, Tupesis_2004f, Yasin_2009}, 13 cases (18\%) were under 18 years of age \cite{AlShaaibi_2021b, Ali_2020f, Cauchi_2002, DivsalarP._2023a, Goldman_1998f, Liu_2005, Naji_2012f, Ohno_2005, Tanrikulu_2015e, Tay_2004, Wildhaber_2005}, 11 cases (15\%) were between 41 and 60 years of age \cite{Al-Faham_2020k, Bhumi_2024f, CamachoDorado_2018, Emamhadi_2018, Hardy_2023g, Jehangir_2019h, Kumar_2001, Sultan_2024f, Thapa_2019f, Wadhwa_2015e, teWildt_2010}, 3 cases (4\%) were over 60 years of age \cite{Beecroft_1998, Kerestes_2019, Li_2013}, 2 cases (3\%) had no age documented \cite{Berry_2021e}. All 90 were male gender. 90 cases (100\%) were detained at the time of ingestion \cite{Elghali_2016, Karp_1991b, Lee_2007}, 88 cases (98\%) were intentional ingestions \cite{Elghali_2016, Karp_1991b, Lee_2007}, 30 cases (33\%) had a psychiatric history documented \cite{Elghali_2016, Karp_1991b, Lee_2007}, 2 cases (2\%) had a history of prior ingestion \cite{Elghali_2016}. No cases were reported for were psychiatric inpatients, were displaced people, were under the influence of alcohol at the time of ingestion, and had a severe disability history.
\paragraph*{Motivation}  70 cases (78\%) reported protest motivation \cite{Elghali_2016, Karp_1991b, Lee_2007}, 12 cases (13\%) reported psychiatric motivation \cite{Karp_1991b}, 6 cases (7\%) reported self-harm motivation \cite{Elghali_2016, Karp_1991b}. No cases were reported for psychosocial motivation and other motivation.
\paragraph*{Object Characteristics}  68 cases (76\%) involved sharp object ingestion \cite{Elghali_2016, Karp_1991b, Lee_2007}, 32 cases (36\%) involved long (\textgreater 5cm) object ingestion \cite{Lee_2007}, 25 cases (28\%) involved ingestion of multiple objects \cite{Elghali_2016, Lee_2007}. No cases were reported for button battery ingestion, magnet ingestion, and involved large diameter (\textgreater 2.5cm) object ingestion.
\paragraph*{Outcomes}  47 cases (52\%) underwent endoscopic intervention \cite{Elghali_2016, Lee_2007}, 29 cases (32\%) were managed conservatively \cite{Elghali_2016, Karp_1991b}, 15 cases (17\%) underwent surgical intervention \cite{Elghali_2016, Karp_1991b, Lee_2007}, 6 cases (7\%) reported complications \cite{Lee_2007}, 1 case (1\%) died \cite{Elghali_2016}.
\paragraph*{Gender} 43 cases (60\%) were male \cite{Akay_2015f, Al-Faham_2020k, Alao_2006i, Ali_2017, Ali_2022g, Apikotoa_2022f, Atayan_2016, Benoist_2019e, Berry_2021e, Bhumi_2024f, CamachoDorado_2018, Csaky_1998e, Emamhadi_2018, Farhadi_2024h, Fry_2010, Gardner_2017h, Guinan_2019f, Jehangir_2019h, Jin_2023, Kobiela_2015, Kumar_2001, Kumar_2019f, Liu_2005, Losanoff_1996, Losanoff_1997e, Mesfin_2022a, Misra_2013, Qureshi_2016, Riva_2018j, Sobnach_2011f, Tammana_2012j, Tanrikulu_2015e, Tay_2004, Thapa_2019f, Trgo_2012f, Wadhwa_2015e, Yasin_2009, teWildt_2010}, 28 cases (39\%) were female \cite{AlShaaibi_2021b, Ali_2020f, Ataya_2013, Beecroft_1998, Bhasin_2014, Bhattacharjee_2008, Cauchi_2002, Chang_2017f, Cox_2007, DelgadoSalazar_2020c, DivsalarP._2023a, Goldman_1998f, Hardy_2023g, Kar_2015, Kariholu_2008, Kerestes_2019, Li_2013, Naji_2012f, Ohno_2005, Peixoto_2017f, Sakellaridis_2008f, Sultan_2024f, Tupesis_2004f, Wildhaber_2005, Wnęk_2015f, Yildiz_2016e}, 1 case (1\%) had no gender recorded \cite{fjbuilsRepeatedBehaviorDeliberate2024}. \paragraph*{Age Group} 25 cases (35\%) were between 26 and 40 years of age \cite{Alao_2006i, Ali_2022g, Apikotoa_2022f, Ataya_2013, Benoist_2019e, Bhasin_2014, Chang_2017f, Cox_2007, DelgadoSalazar_2020c, Farhadi_2024h, Fry_2010, Gardner_2017h, Guinan_2019f, Jin_2023, Kumar_2019f, Losanoff_1996, Misra_2013, Qureshi_2016, Riva_2018j, Sakellaridis_2008f, Tammana_2012j, Trgo_2012f, Wnęk_2015f, Yildiz_2016e, fjbuilsRepeatedBehaviorDeliberate2024}, 18 cases (25\%) were between 18 and 25 years of age \cite{Akay_2015f, Ali_2017, Atayan_2016, Bhattacharjee_2008, Csaky_1998e, Kar_2015, Kariholu_2008, Kobiela_2015, Losanoff_1996, Losanoff_1997e, Mesfin_2022a, Peixoto_2017f, Sobnach_2011f, Tupesis_2004f, Yasin_2009}, 13 cases (18\%) were under 18 years of age \cite{AlShaaibi_2021b, Ali_2020f, Cauchi_2002, DivsalarP._2023a, Goldman_1998f, Liu_2005, Naji_2012f, Ohno_2005, Tanrikulu_2015e, Tay_2004, Wildhaber_2005}, 11 cases (15\%) were between 41 and 60 years of age \cite{Al-Faham_2020k, Bhumi_2024f, CamachoDorado_2018, Emamhadi_2018, Hardy_2023g, Jehangir_2019h, Kumar_2001, Sultan_2024f, Thapa_2019f, Wadhwa_2015e, teWildt_2010}, 3 cases (4\%) were over 60 years of age \cite{Beecroft_1998, Kerestes_2019, Li_2013}, 2 cases (3\%) had no age documented \cite{Berry_2021e}. All 90 were male gender. 90 cases (100\%) were detained at the time of ingestion \cite{Elghali_2016, Karp_1991b, Lee_2007}, 88 cases (98\%) were intentional ingestions \cite{Elghali_2016, Karp_1991b, Lee_2007}, 30 cases (33\%) had a psychiatric history documented \cite{Elghali_2016, Karp_1991b, Lee_2007}, 2 cases (2\%) had a history of prior ingestion \cite{Elghali_2016}. No cases were reported for were psychiatric inpatients, were displaced people, were under the influence of alcohol at the time of ingestion, and had a severe disability history.
\paragraph*{Motivation}  70 cases (78\%) reported protest motivation \cite{Elghali_2016, Karp_1991b, Lee_2007}, 12 cases (13\%) reported psychiatric motivation \cite{Karp_1991b}, 6 cases (7\%) reported self-harm motivation \cite{Elghali_2016, Karp_1991b}. No cases were reported for psychosocial motivation and other motivation.
\paragraph*{Object Characteristics}  68 cases (76\%) involved sharp object ingestion \cite{Elghali_2016, Karp_1991b, Lee_2007}, 32 cases (36\%) involved long (\textgreater 5cm) object ingestion \cite{Lee_2007}, 25 cases (28\%) involved ingestion of multiple objects \cite{Elghali_2016, Lee_2007}. No cases were reported for button battery ingestion, magnet ingestion, and involved large diameter (\textgreater 2.5cm) object ingestion.
\paragraph*{Outcomes}  47 cases (52\%) underwent endoscopic intervention \cite{Elghali_2016, Lee_2007}, 29 cases (32\%) were managed conservatively \cite{Elghali_2016, Karp_1991b}, 15 cases (17\%) underwent surgical intervention \cite{Elghali_2016, Karp_1991b, Lee_2007}, 6 cases (7\%) reported complications \cite{Lee_2007}, 1 case (1\%) died \cite{Elghali_2016}.
\paragraph*{Gender} 43 cases (60\%) were male \cite{Akay_2015f, Al-Faham_2020k, Alao_2006i, Ali_2017, Ali_2022g, Apikotoa_2022f, Atayan_2016, Benoist_2019e, Berry_2021e, Bhumi_2024f, CamachoDorado_2018, Csaky_1998e, Emamhadi_2018, Farhadi_2024h, Fry_2010, Gardner_2017h, Guinan_2019f, Jehangir_2019h, Jin_2023, Kobiela_2015, Kumar_2001, Kumar_2019f, Liu_2005, Losanoff_1996, Losanoff_1997e, Mesfin_2022a, Misra_2013, Qureshi_2016, Riva_2018j, Sobnach_2011f, Tammana_2012j, Tanrikulu_2015e, Tay_2004, Thapa_2019f, Trgo_2012f, Wadhwa_2015e, Yasin_2009, teWildt_2010}, 28 cases (39\%) were female \cite{AlShaaibi_2021b, Ali_2020f, Ataya_2013, Beecroft_1998, Bhasin_2014, Bhattacharjee_2008, Cauchi_2002, Chang_2017f, Cox_2007, DelgadoSalazar_2020c, DivsalarP._2023a, Goldman_1998f, Hardy_2023g, Kar_2015, Kariholu_2008, Kerestes_2019, Li_2013, Naji_2012f, Ohno_2005, Peixoto_2017f, Sakellaridis_2008f, Sultan_2024f, Tupesis_2004f, Wildhaber_2005, Wnęk_2015f, Yildiz_2016e}, 1 case (1\%) had no gender recorded \cite{fjbuilsRepeatedBehaviorDeliberate2024}. \paragraph*{Age Group} 25 cases (35\%) were between 26 and 40 years of age \cite{Alao_2006i, Ali_2022g, Apikotoa_2022f, Ataya_2013, Benoist_2019e, Bhasin_2014, Chang_2017f, Cox_2007, DelgadoSalazar_2020c, Farhadi_2024h, Fry_2010, Gardner_2017h, Guinan_2019f, Jin_2023, Kumar_2019f, Losanoff_1996, Misra_2013, Qureshi_2016, Riva_2018j, Sakellaridis_2008f, Tammana_2012j, Trgo_2012f, Wnęk_2015f, Yildiz_2016e, fjbuilsRepeatedBehaviorDeliberate2024}, 18 cases (25\%) were between 18 and 25 years of age \cite{Akay_2015f, Ali_2017, Atayan_2016, Bhattacharjee_2008, Csaky_1998e, Kar_2015, Kariholu_2008, Kobiela_2015, Losanoff_1996, Losanoff_1997e, Mesfin_2022a, Peixoto_2017f, Sobnach_2011f, Tupesis_2004f, Yasin_2009}, 13 cases (18\%) were under 18 years of age \cite{AlShaaibi_2021b, Ali_2020f, Cauchi_2002, DivsalarP._2023a, Goldman_1998f, Liu_2005, Naji_2012f, Ohno_2005, Tanrikulu_2015e, Tay_2004, Wildhaber_2005}, 11 cases (15\%) were between 41 and 60 years of age \cite{Al-Faham_2020k, Bhumi_2024f, CamachoDorado_2018, Emamhadi_2018, Hardy_2023g, Jehangir_2019h, Kumar_2001, Sultan_2024f, Thapa_2019f, Wadhwa_2015e, teWildt_2010}, 3 cases (4\%) were over 60 years of age \cite{Beecroft_1998, Kerestes_2019, Li_2013}, 2 cases (3\%) had no age documented \cite{Berry_2021e}. All 90 were male gender. 90 cases (100\%) were detained at the time of ingestion \cite{Elghali_2016, Karp_1991b, Lee_2007}, 88 cases (98\%) were intentional ingestions \cite{Elghali_2016, Karp_1991b, Lee_2007}, 30 cases (33\%) had a psychiatric history documented \cite{Elghali_2016, Karp_1991b, Lee_2007}, 2 cases (2\%) had a history of prior ingestion \cite{Elghali_2016}. No cases were reported for were psychiatric inpatients, were displaced people, were under the influence of alcohol at the time of ingestion, and had a severe disability history.
\paragraph*{Motivation}  70 cases (78\%) reported protest motivation \cite{Elghali_2016, Karp_1991b, Lee_2007}, 12 cases (13\%) reported psychiatric motivation \cite{Karp_1991b}, 6 cases (7\%) reported self-harm motivation \cite{Elghali_2016, Karp_1991b}. No cases were reported for psychosocial motivation and other motivation.
\paragraph*{Object Characteristics}  68 cases (76\%) involved sharp object ingestion \cite{Elghali_2016, Karp_1991b, Lee_2007}, 32 cases (36\%) involved long (\textgreater 5cm) object ingestion \cite{Lee_2007}, 25 cases (28\%) involved ingestion of multiple objects \cite{Elghali_2016, Lee_2007}. No cases were reported for button battery ingestion, magnet ingestion, and involved large diameter (\textgreater 2.5cm) object ingestion.
\paragraph*{Outcomes}  47 cases (52\%) underwent endoscopic intervention \cite{Elghali_2016, Lee_2007}, 29 cases (32\%) were managed conservatively \cite{Elghali_2016, Karp_1991b}, 15 cases (17\%) underwent surgical intervention \cite{Elghali_2016, Karp_1991b, Lee_2007}, 6 cases (7\%) reported complications \cite{Lee_2007}, 1 case (1\%) died \cite{Elghali_2016}.
\paragraph*{Gender} 43 cases (60\%) were male \cite{Akay_2015f, Al-Faham_2020k, Alao_2006i, Ali_2017, Ali_2022g, Apikotoa_2022f, Atayan_2016, Benoist_2019e, Berry_2021e, Bhumi_2024f, CamachoDorado_2018, Csaky_1998e, Emamhadi_2018, Farhadi_2024h, Fry_2010, Gardner_2017h, Guinan_2019f, Jehangir_2019h, Jin_2023, Kobiela_2015, Kumar_2001, Kumar_2019f, Liu_2005, Losanoff_1996, Losanoff_1997e, Mesfin_2022a, Misra_2013, Qureshi_2016, Riva_2018j, Sobnach_2011f, Tammana_2012j, Tanrikulu_2015e, Tay_2004, Thapa_2019f, Trgo_2012f, Wadhwa_2015e, Yasin_2009, teWildt_2010}, 28 cases (39\%) were female \cite{AlShaaibi_2021b, Ali_2020f, Ataya_2013, Beecroft_1998, Bhasin_2014, Bhattacharjee_2008, Cauchi_2002, Chang_2017f, Cox_2007, DelgadoSalazar_2020c, DivsalarP._2023a, Goldman_1998f, Hardy_2023g, Kar_2015, Kariholu_2008, Kerestes_2019, Li_2013, Naji_2012f, Ohno_2005, Peixoto_2017f, Sakellaridis_2008f, Sultan_2024f, Tupesis_2004f, Wildhaber_2005, Wnęk_2015f, Yildiz_2016e}, 1 case (1\%) had no gender recorded \cite{fjbuilsRepeatedBehaviorDeliberate2024}. \paragraph*{Age Group} 25 cases (35\%) were between 26 and 40 years of age \cite{Alao_2006i, Ali_2022g, Apikotoa_2022f, Ataya_2013, Benoist_2019e, Bhasin_2014, Chang_2017f, Cox_2007, DelgadoSalazar_2020c, Farhadi_2024h, Fry_2010, Gardner_2017h, Guinan_2019f, Jin_2023, Kumar_2019f, Losanoff_1996, Misra_2013, Qureshi_2016, Riva_2018j, Sakellaridis_2008f, Tammana_2012j, Trgo_2012f, Wnęk_2015f, Yildiz_2016e, fjbuilsRepeatedBehaviorDeliberate2024}, 18 cases (25\%) were between 18 and 25 years of age \cite{Akay_2015f, Ali_2017, Atayan_2016, Bhattacharjee_2008, Csaky_1998e, Kar_2015, Kariholu_2008, Kobiela_2015, Losanoff_1996, Losanoff_1997e, Mesfin_2022a, Peixoto_2017f, Sobnach_2011f, Tupesis_2004f, Yasin_2009}, 13 cases (18\%) were under 18 years of age \cite{AlShaaibi_2021b, Ali_2020f, Cauchi_2002, DivsalarP._2023a, Goldman_1998f, Liu_2005, Naji_2012f, Ohno_2005, Tanrikulu_2015e, Tay_2004, Wildhaber_2005}, 11 cases (15\%) were between 41 and 60 years of age \cite{Al-Faham_2020k, Bhumi_2024f, CamachoDorado_2018, Emamhadi_2018, Hardy_2023g, Jehangir_2019h, Kumar_2001, Sultan_2024f, Thapa_2019f, Wadhwa_2015e, teWildt_2010}, 3 cases (4\%) were over 60 years of age \cite{Beecroft_1998, Kerestes_2019, Li_2013}, 2 cases (3\%) had no age documented \cite{Berry_2021e}. All 90 were male gender. 90 cases (100\%) were detained at the time of ingestion \cite{Elghali_2016, Karp_1991b, Lee_2007}, 88 cases (98\%) were intentional ingestions \cite{Elghali_2016, Karp_1991b, Lee_2007}, 30 cases (33\%) had a psychiatric history documented \cite{Elghali_2016, Karp_1991b, Lee_2007}, 2 cases (2\%) had a history of prior ingestion \cite{Elghali_2016}. No cases were reported for were psychiatric inpatients, were displaced people, were under the influence of alcohol at the time of ingestion, and had a severe disability history.
\paragraph*{Motivation}  70 cases (78\%) reported protest motivation \cite{Elghali_2016, Karp_1991b, Lee_2007}, 12 cases (13\%) reported psychiatric motivation \cite{Karp_1991b}, 6 cases (7\%) reported self-harm motivation \cite{Elghali_2016, Karp_1991b}. No cases were reported for psychosocial motivation and other motivation.
\paragraph*{Object Characteristics}  68 cases (76\%) involved sharp object ingestion \cite{Elghali_2016, Karp_1991b, Lee_2007}, 32 cases (36\%) involved long (\textgreater 5cm) object ingestion \cite{Lee_2007}, 25 cases (28\%) involved ingestion of multiple objects \cite{Elghali_2016, Lee_2007}. No cases were reported for button battery ingestion, magnet ingestion, and involved large diameter (\textgreater 2.5cm) object ingestion.
\paragraph*{Outcomes}  47 cases (52\%) underwent endoscopic intervention \cite{Elghali_2016, Lee_2007}, 29 cases (32\%) were managed conservatively \cite{Elghali_2016, Karp_1991b}, 15 cases (17\%) underwent surgical intervention \cite{Elghali_2016, Karp_1991b, Lee_2007}, 6 cases (7\%) reported complications \cite{Lee_2007}, 1 case (1\%) died \cite{Elghali_2016}.
\paragraph*{Gender} 43 cases (60\%) were male \cite{Akay_2015f, Al-Faham_2020k, Alao_2006i, Ali_2017, Ali_2022g, Apikotoa_2022f, Atayan_2016, Benoist_2019e, Berry_2021e, Bhumi_2024f, CamachoDorado_2018, Csaky_1998e, Emamhadi_2018, Farhadi_2024h, Fry_2010, Gardner_2017h, Guinan_2019f, Jehangir_2019h, Jin_2023, Kobiela_2015, Kumar_2001, Kumar_2019f, Liu_2005, Losanoff_1996, Losanoff_1997e, Mesfin_2022a, Misra_2013, Qureshi_2016, Riva_2018j, Sobnach_2011f, Tammana_2012j, Tanrikulu_2015e, Tay_2004, Thapa_2019f, Trgo_2012f, Wadhwa_2015e, Yasin_2009, teWildt_2010}, 28 cases (39\%) were female \cite{AlShaaibi_2021b, Ali_2020f, Ataya_2013, Beecroft_1998, Bhasin_2014, Bhattacharjee_2008, Cauchi_2002, Chang_2017f, Cox_2007, DelgadoSalazar_2020c, DivsalarP._2023a, Goldman_1998f, Hardy_2023g, Kar_2015, Kariholu_2008, Kerestes_2019, Li_2013, Naji_2012f, Ohno_2005, Peixoto_2017f, Sakellaridis_2008f, Sultan_2024f, Tupesis_2004f, Wildhaber_2005, Wnęk_2015f, Yildiz_2016e}, 1 case (1\%) had no gender recorded \cite{fjbuilsRepeatedBehaviorDeliberate2024}. \paragraph*{Age Group} 25 cases (35\%) were between 26 and 40 years of age \cite{Alao_2006i, Ali_2022g, Apikotoa_2022f, Ataya_2013, Benoist_2019e, Bhasin_2014, Chang_2017f, Cox_2007, DelgadoSalazar_2020c, Farhadi_2024h, Fry_2010, Gardner_2017h, Guinan_2019f, Jin_2023, Kumar_2019f, Losanoff_1996, Misra_2013, Qureshi_2016, Riva_2018j, Sakellaridis_2008f, Tammana_2012j, Trgo_2012f, Wnęk_2015f, Yildiz_2016e, fjbuilsRepeatedBehaviorDeliberate2024}, 18 cases (25\%) were between 18 and 25 years of age \cite{Akay_2015f, Ali_2017, Atayan_2016, Bhattacharjee_2008, Csaky_1998e, Kar_2015, Kariholu_2008, Kobiela_2015, Losanoff_1996, Losanoff_1997e, Mesfin_2022a, Peixoto_2017f, Sobnach_2011f, Tupesis_2004f, Yasin_2009}, 13 cases (18\%) were under 18 years of age \cite{AlShaaibi_2021b, Ali_2020f, Cauchi_2002, DivsalarP._2023a, Goldman_1998f, Liu_2005, Naji_2012f, Ohno_2005, Tanrikulu_2015e, Tay_2004, Wildhaber_2005}, 11 cases (15\%) were between 41 and 60 years of age \cite{Al-Faham_2020k, Bhumi_2024f, CamachoDorado_2018, Emamhadi_2018, Hardy_2023g, Jehangir_2019h, Kumar_2001, Sultan_2024f, Thapa_2019f, Wadhwa_2015e, teWildt_2010}, 3 cases (4\%) were over 60 years of age \cite{Beecroft_1998, Kerestes_2019, Li_2013}, 2 cases (3\%) had no age documented \cite{Berry_2021e}. All 90 were male gender. 90 cases (100\%) were detained at the time of ingestion \cite{Elghali_2016, Karp_1991b, Lee_2007}, 88 cases (98\%) were intentional ingestions \cite{Elghali_2016, Karp_1991b, Lee_2007}, 30 cases (33\%) had a psychiatric history documented \cite{Elghali_2016, Karp_1991b, Lee_2007}, 2 cases (2\%) had a history of prior ingestion \cite{Elghali_2016}. No cases were reported for were psychiatric inpatients, were displaced people, were under the influence of alcohol at the time of ingestion, and had a severe disability history.
\paragraph*{Motivation}  70 cases (78\%) reported protest motivation \cite{Elghali_2016, Karp_1991b, Lee_2007}, 12 cases (13\%) reported psychiatric motivation \cite{Karp_1991b}, 6 cases (7\%) reported self-harm motivation \cite{Elghali_2016, Karp_1991b}. No cases were reported for psychosocial motivation and other motivation.
\paragraph*{Object Characteristics}  68 cases (76\%) involved sharp object ingestion \cite{Elghali_2016, Karp_1991b, Lee_2007}, 32 cases (36\%) involved long (\textgreater 5cm) object ingestion \cite{Lee_2007}, 25 cases (28\%) involved ingestion of multiple objects \cite{Elghali_2016, Lee_2007}. No cases were reported for button battery ingestion, magnet ingestion, and involved large diameter (\textgreater 2.5cm) object ingestion.
\paragraph*{Outcomes}  47 cases (52\%) underwent endoscopic intervention \cite{Elghali_2016, Lee_2007}, 29 cases (32\%) were managed conservatively \cite{Elghali_2016, Karp_1991b}, 15 cases (17\%) underwent surgical intervention \cite{Elghali_2016, Karp_1991b, Lee_2007}, 6 cases (7\%) reported complications \cite{Lee_2007}, 1 case (1\%) died \cite{Elghali_2016}.
\paragraph*{Gender} 43 cases (60\%) were male \cite{Akay_2015f, Al-Faham_2020k, Alao_2006i, Ali_2017, Ali_2022g, Apikotoa_2022f, Atayan_2016, Benoist_2019e, Berry_2021e, Bhumi_2024f, CamachoDorado_2018, Csaky_1998e, Emamhadi_2018, Farhadi_2024h, Fry_2010, Gardner_2017h, Guinan_2019f, Jehangir_2019h, Jin_2023, Kobiela_2015, Kumar_2001, Kumar_2019f, Liu_2005, Losanoff_1996, Losanoff_1997e, Mesfin_2022a, Misra_2013, Qureshi_2016, Riva_2018j, Sobnach_2011f, Tammana_2012j, Tanrikulu_2015e, Tay_2004, Thapa_2019f, Trgo_2012f, Wadhwa_2015e, Yasin_2009, teWildt_2010}, 28 cases (39\%) were female \cite{AlShaaibi_2021b, Ali_2020f, Ataya_2013, Beecroft_1998, Bhasin_2014, Bhattacharjee_2008, Cauchi_2002, Chang_2017f, Cox_2007, DelgadoSalazar_2020c, DivsalarP._2023a, Goldman_1998f, Hardy_2023g, Kar_2015, Kariholu_2008, Kerestes_2019, Li_2013, Naji_2012f, Ohno_2005, Peixoto_2017f, Sakellaridis_2008f, Sultan_2024f, Tupesis_2004f, Wildhaber_2005, Wnęk_2015f, Yildiz_2016e}, 1 case (1\%) had no gender recorded \cite{fjbuilsRepeatedBehaviorDeliberate2024}. \paragraph*{Age Group} 25 cases (35\%) were between 26 and 40 years of age \cite{Alao_2006i, Ali_2022g, Apikotoa_2022f, Ataya_2013, Benoist_2019e, Bhasin_2014, Chang_2017f, Cox_2007, DelgadoSalazar_2020c, Farhadi_2024h, Fry_2010, Gardner_2017h, Guinan_2019f, Jin_2023, Kumar_2019f, Losanoff_1996, Misra_2013, Qureshi_2016, Riva_2018j, Sakellaridis_2008f, Tammana_2012j, Trgo_2012f, Wnęk_2015f, Yildiz_2016e, fjbuilsRepeatedBehaviorDeliberate2024}, 18 cases (25\%) were between 18 and 25 years of age \cite{Akay_2015f, Ali_2017, Atayan_2016, Bhattacharjee_2008, Csaky_1998e, Kar_2015, Kariholu_2008, Kobiela_2015, Losanoff_1996, Losanoff_1997e, Mesfin_2022a, Peixoto_2017f, Sobnach_2011f, Tupesis_2004f, Yasin_2009}, 13 cases (18\%) were under 18 years of age \cite{AlShaaibi_2021b, Ali_2020f, Cauchi_2002, DivsalarP._2023a, Goldman_1998f, Liu_2005, Naji_2012f, Ohno_2005, Tanrikulu_2015e, Tay_2004, Wildhaber_2005}, 11 cases (15\%) were between 41 and 60 years of age \cite{Al-Faham_2020k, Bhumi_2024f, CamachoDorado_2018, Emamhadi_2018, Hardy_2023g, Jehangir_2019h, Kumar_2001, Sultan_2024f, Thapa_2019f, Wadhwa_2015e, teWildt_2010}, 3 cases (4\%) were over 60 years of age \cite{Beecroft_1998, Kerestes_2019, Li_2013}, 2 cases (3\%) had no age documented \cite{Berry_2021e}. All 90 were male gender. 90 cases (100\%) were detained at the time of ingestion \cite{Elghali_2016, Karp_1991b, Lee_2007}, 88 cases (98\%) were intentional ingestions \cite{Elghali_2016, Karp_1991b, Lee_2007}, 30 cases (33\%) had a psychiatric history documented \cite{Elghali_2016, Karp_1991b, Lee_2007}, 2 cases (2\%) had a history of prior ingestion \cite{Elghali_2016}. No cases were reported for were psychiatric inpatients, were displaced people, were under the influence of alcohol at the time of ingestion, and had a severe disability history.
\paragraph*{Motivation}  70 cases (78\%) reported protest motivation \cite{Elghali_2016, Karp_1991b, Lee_2007}, 12 cases (13\%) reported psychiatric motivation \cite{Karp_1991b}, 6 cases (7\%) reported self-harm motivation \cite{Elghali_2016, Karp_1991b}. No cases were reported for psychosocial motivation and other motivation.
\paragraph*{Object Characteristics}  68 cases (76\%) involved sharp object ingestion \cite{Elghali_2016, Karp_1991b, Lee_2007}, 32 cases (36\%) involved long (\textgreater 5cm) object ingestion \cite{Lee_2007}, 25 cases (28\%) involved ingestion of multiple objects \cite{Elghali_2016, Lee_2007}. No cases were reported for button battery ingestion, magnet ingestion, and involved large diameter (\textgreater 2.5cm) object ingestion.
\paragraph*{Outcomes}  47 cases (52\%) underwent endoscopic intervention \cite{Elghali_2016, Lee_2007}, 29 cases (32\%) were managed conservatively \cite{Elghali_2016, Karp_1991b}, 15 cases (17\%) underwent surgical intervention \cite{Elghali_2016, Karp_1991b, Lee_2007}, 6 cases (7\%) reported complications \cite{Lee_2007}, 1 case (1\%) died \cite{Elghali_2016}.
\paragraph*{Gender} 43 cases (60\%) were male \cite{Akay_2015f, Al-Faham_2020k, Alao_2006i, Ali_2017, Ali_2022g, Apikotoa_2022f, Atayan_2016, Benoist_2019e, Berry_2021e, Bhumi_2024f, CamachoDorado_2018, Csaky_1998e, Emamhadi_2018, Farhadi_2024h, Fry_2010, Gardner_2017h, Guinan_2019f, Jehangir_2019h, Jin_2023, Kobiela_2015, Kumar_2001, Kumar_2019f, Liu_2005, Losanoff_1996, Losanoff_1997e, Mesfin_2022a, Misra_2013, Qureshi_2016, Riva_2018j, Sobnach_2011f, Tammana_2012j, Tanrikulu_2015e, Tay_2004, Thapa_2019f, Trgo_2012f, Wadhwa_2015e, Yasin_2009, teWildt_2010}, 28 cases (39\%) were female \cite{AlShaaibi_2021b, Ali_2020f, Ataya_2013, Beecroft_1998, Bhasin_2014, Bhattacharjee_2008, Cauchi_2002, Chang_2017f, Cox_2007, DelgadoSalazar_2020c, DivsalarP._2023a, Goldman_1998f, Hardy_2023g, Kar_2015, Kariholu_2008, Kerestes_2019, Li_2013, Naji_2012f, Ohno_2005, Peixoto_2017f, Sakellaridis_2008f, Sultan_2024f, Tupesis_2004f, Wildhaber_2005, Wnęk_2015f, Yildiz_2016e}, 1 case (1\%) had no gender recorded \cite{fjbuilsRepeatedBehaviorDeliberate2024}. \paragraph*{Age Group} 25 cases (35\%) were between 26 and 40 years of age \cite{Alao_2006i, Ali_2022g, Apikotoa_2022f, Ataya_2013, Benoist_2019e, Bhasin_2014, Chang_2017f, Cox_2007, DelgadoSalazar_2020c, Farhadi_2024h, Fry_2010, Gardner_2017h, Guinan_2019f, Jin_2023, Kumar_2019f, Losanoff_1996, Misra_2013, Qureshi_2016, Riva_2018j, Sakellaridis_2008f, Tammana_2012j, Trgo_2012f, Wnęk_2015f, Yildiz_2016e, fjbuilsRepeatedBehaviorDeliberate2024}, 18 cases (25\%) were between 18 and 25 years of age \cite{Akay_2015f, Ali_2017, Atayan_2016, Bhattacharjee_2008, Csaky_1998e, Kar_2015, Kariholu_2008, Kobiela_2015, Losanoff_1996, Losanoff_1997e, Mesfin_2022a, Peixoto_2017f, Sobnach_2011f, Tupesis_2004f, Yasin_2009}, 13 cases (18\%) were under 18 years of age \cite{AlShaaibi_2021b, Ali_2020f, Cauchi_2002, DivsalarP._2023a, Goldman_1998f, Liu_2005, Naji_2012f, Ohno_2005, Tanrikulu_2015e, Tay_2004, Wildhaber_2005}, 11 cases (15\%) were between 41 and 60 years of age \cite{Al-Faham_2020k, Bhumi_2024f, CamachoDorado_2018, Emamhadi_2018, Hardy_2023g, Jehangir_2019h, Kumar_2001, Sultan_2024f, Thapa_2019f, Wadhwa_2015e, teWildt_2010}, 3 cases (4\%) were over 60 years of age \cite{Beecroft_1998, Kerestes_2019, Li_2013}, 2 cases (3\%) had no age documented \cite{Berry_2021e}. All 90 were male gender. 90 cases (100\%) were detained at the time of ingestion \cite{Elghali_2016, Karp_1991b, Lee_2007}, 88 cases (98\%) were intentional ingestions \cite{Elghali_2016, Karp_1991b, Lee_2007}, 30 cases (33\%) had a psychiatric history documented \cite{Elghali_2016, Karp_1991b, Lee_2007}, 2 cases (2\%) had a history of prior ingestion \cite{Elghali_2016}. No cases were reported for were psychiatric inpatients, were displaced people, were under the influence of alcohol at the time of ingestion, and had a severe disability history.
\paragraph*{Motivation}  70 cases (78\%) reported protest motivation \cite{Elghali_2016, Karp_1991b, Lee_2007}, 12 cases (13\%) reported psychiatric motivation \cite{Karp_1991b}, 6 cases (7\%) reported self-harm motivation \cite{Elghali_2016, Karp_1991b}. No cases were reported for psychosocial motivation and other motivation.
\paragraph*{Object Characteristics}  68 cases (76\%) involved sharp object ingestion \cite{Elghali_2016, Karp_1991b, Lee_2007}, 32 cases (36\%) involved long (\textgreater 5cm) object ingestion \cite{Lee_2007}, 25 cases (28\%) involved ingestion of multiple objects \cite{Elghali_2016, Lee_2007}. No cases were reported for button battery ingestion, magnet ingestion, and involved large diameter (\textgreater 2.5cm) object ingestion.
\paragraph*{Outcomes}  47 cases (52\%) underwent endoscopic intervention \cite{Elghali_2016, Lee_2007}, 29 cases (32\%) were managed conservatively \cite{Elghali_2016, Karp_1991b}, 15 cases (17\%) underwent surgical intervention \cite{Elghali_2016, Karp_1991b, Lee_2007}, 6 cases (7\%) reported complications \cite{Lee_2007}, 1 case (1\%) died \cite{Elghali_2016}.
\paragraph*{Gender} 43 cases (60\%) were male \cite{Akay_2015f, Al-Faham_2020k, Alao_2006i, Ali_2017, Ali_2022g, Apikotoa_2022f, Atayan_2016, Benoist_2019e, Berry_2021e, Bhumi_2024f, CamachoDorado_2018, Csaky_1998e, Emamhadi_2018, Farhadi_2024h, Fry_2010, Gardner_2017h, Guinan_2019f, Jehangir_2019h, Jin_2023, Kobiela_2015, Kumar_2001, Kumar_2019f, Liu_2005, Losanoff_1996, Losanoff_1997e, Mesfin_2022a, Misra_2013, Qureshi_2016, Riva_2018j, Sobnach_2011f, Tammana_2012j, Tanrikulu_2015e, Tay_2004, Thapa_2019f, Trgo_2012f, Wadhwa_2015e, Yasin_2009, teWildt_2010}, 28 cases (39\%) were female \cite{AlShaaibi_2021b, Ali_2020f, Ataya_2013, Beecroft_1998, Bhasin_2014, Bhattacharjee_2008, Cauchi_2002, Chang_2017f, Cox_2007, DelgadoSalazar_2020c, DivsalarP._2023a, Goldman_1998f, Hardy_2023g, Kar_2015, Kariholu_2008, Kerestes_2019, Li_2013, Naji_2012f, Ohno_2005, Peixoto_2017f, Sakellaridis_2008f, Sultan_2024f, Tupesis_2004f, Wildhaber_2005, Wnęk_2015f, Yildiz_2016e}, 1 case (1\%) had no gender recorded \cite{fjbuilsRepeatedBehaviorDeliberate2024}. \paragraph*{Age Group} 25 cases (35\%) were between 26 and 40 years of age \cite{Alao_2006i, Ali_2022g, Apikotoa_2022f, Ataya_2013, Benoist_2019e, Bhasin_2014, Chang_2017f, Cox_2007, DelgadoSalazar_2020c, Farhadi_2024h, Fry_2010, Gardner_2017h, Guinan_2019f, Jin_2023, Kumar_2019f, Losanoff_1996, Misra_2013, Qureshi_2016, Riva_2018j, Sakellaridis_2008f, Tammana_2012j, Trgo_2012f, Wnęk_2015f, Yildiz_2016e, fjbuilsRepeatedBehaviorDeliberate2024}, 18 cases (25\%) were between 18 and 25 years of age \cite{Akay_2015f, Ali_2017, Atayan_2016, Bhattacharjee_2008, Csaky_1998e, Kar_2015, Kariholu_2008, Kobiela_2015, Losanoff_1996, Losanoff_1997e, Mesfin_2022a, Peixoto_2017f, Sobnach_2011f, Tupesis_2004f, Yasin_2009}, 13 cases (18\%) were under 18 years of age \cite{AlShaaibi_2021b, Ali_2020f, Cauchi_2002, DivsalarP._2023a, Goldman_1998f, Liu_2005, Naji_2012f, Ohno_2005, Tanrikulu_2015e, Tay_2004, Wildhaber_2005}, 11 cases (15\%) were between 41 and 60 years of age \cite{Al-Faham_2020k, Bhumi_2024f, CamachoDorado_2018, Emamhadi_2018, Hardy_2023g, Jehangir_2019h, Kumar_2001, Sultan_2024f, Thapa_2019f, Wadhwa_2015e, teWildt_2010}, 3 cases (4\%) were over 60 years of age \cite{Beecroft_1998, Kerestes_2019, Li_2013}, 2 cases (3\%) had no age documented \cite{Berry_2021e}. All 90 were male gender. 90 cases (100\%) were detained at the time of ingestion \cite{Elghali_2016, Karp_1991b, Lee_2007}, 88 cases (98\%) were intentional ingestions \cite{Elghali_2016, Karp_1991b, Lee_2007}, 30 cases (33\%) had a psychiatric history documented \cite{Elghali_2016, Karp_1991b, Lee_2007}, 2 cases (2\%) had a history of prior ingestion \cite{Elghali_2016}. No cases were reported for were psychiatric inpatients, were displaced people, were under the influence of alcohol at the time of ingestion, and had a severe disability history.
\paragraph*{Motivation}  70 cases (78\%) reported protest motivation \cite{Elghali_2016, Karp_1991b, Lee_2007}, 12 cases (13\%) reported psychiatric motivation \cite{Karp_1991b}, 6 cases (7\%) reported self-harm motivation \cite{Elghali_2016, Karp_1991b}. No cases were reported for psychosocial motivation and other motivation.
\paragraph*{Object Characteristics}  68 cases (76\%) involved sharp object ingestion \cite{Elghali_2016, Karp_1991b, Lee_2007}, 32 cases (36\%) involved long (\textgreater 5cm) object ingestion \cite{Lee_2007}, 25 cases (28\%) involved ingestion of multiple objects \cite{Elghali_2016, Lee_2007}. No cases were reported for button battery ingestion, magnet ingestion, and involved large diameter (\textgreater 2.5cm) object ingestion.
\paragraph*{Outcomes}  47 cases (52\%) underwent endoscopic intervention \cite{Elghali_2016, Lee_2007}, 29 cases (32\%) were managed conservatively \cite{Elghali_2016, Karp_1991b}, 15 cases (17\%) underwent surgical intervention \cite{Elghali_2016, Karp_1991b, Lee_2007}, 6 cases (7\%) reported complications \cite{Lee_2007}, 1 case (1\%) died \cite{Elghali_2016}.
\paragraph*{Gender} 43 cases (60\%) were male \cite{Akay_2015f, Al-Faham_2020k, Alao_2006i, Ali_2017, Ali_2022g, Apikotoa_2022f, Atayan_2016, Benoist_2019e, Berry_2021e, Bhumi_2024f, CamachoDorado_2018, Csaky_1998e, Emamhadi_2018, Farhadi_2024h, Fry_2010, Gardner_2017h, Guinan_2019f, Jehangir_2019h, Jin_2023, Kobiela_2015, Kumar_2001, Kumar_2019f, Liu_2005, Losanoff_1996, Losanoff_1997e, Mesfin_2022a, Misra_2013, Qureshi_2016, Riva_2018j, Sobnach_2011f, Tammana_2012j, Tanrikulu_2015e, Tay_2004, Thapa_2019f, Trgo_2012f, Wadhwa_2015e, Yasin_2009, teWildt_2010}, 28 cases (39\%) were female \cite{AlShaaibi_2021b, Ali_2020f, Ataya_2013, Beecroft_1998, Bhasin_2014, Bhattacharjee_2008, Cauchi_2002, Chang_2017f, Cox_2007, DelgadoSalazar_2020c, DivsalarP._2023a, Goldman_1998f, Hardy_2023g, Kar_2015, Kariholu_2008, Kerestes_2019, Li_2013, Naji_2012f, Ohno_2005, Peixoto_2017f, Sakellaridis_2008f, Sultan_2024f, Tupesis_2004f, Wildhaber_2005, Wnęk_2015f, Yildiz_2016e}, 1 case (1\%) had no gender recorded \cite{fjbuilsRepeatedBehaviorDeliberate2024}. \paragraph*{Age Group} 25 cases (35\%) were between 26 and 40 years of age \cite{Alao_2006i, Ali_2022g, Apikotoa_2022f, Ataya_2013, Benoist_2019e, Bhasin_2014, Chang_2017f, Cox_2007, DelgadoSalazar_2020c, Farhadi_2024h, Fry_2010, Gardner_2017h, Guinan_2019f, Jin_2023, Kumar_2019f, Losanoff_1996, Misra_2013, Qureshi_2016, Riva_2018j, Sakellaridis_2008f, Tammana_2012j, Trgo_2012f, Wnęk_2015f, Yildiz_2016e, fjbuilsRepeatedBehaviorDeliberate2024}, 18 cases (25\%) were between 18 and 25 years of age \cite{Akay_2015f, Ali_2017, Atayan_2016, Bhattacharjee_2008, Csaky_1998e, Kar_2015, Kariholu_2008, Kobiela_2015, Losanoff_1996, Losanoff_1997e, Mesfin_2022a, Peixoto_2017f, Sobnach_2011f, Tupesis_2004f, Yasin_2009}, 13 cases (18\%) were under 18 years of age \cite{AlShaaibi_2021b, Ali_2020f, Cauchi_2002, DivsalarP._2023a, Goldman_1998f, Liu_2005, Naji_2012f, Ohno_2005, Tanrikulu_2015e, Tay_2004, Wildhaber_2005}, 11 cases (15\%) were between 41 and 60 years of age \cite{Al-Faham_2020k, Bhumi_2024f, CamachoDorado_2018, Emamhadi_2018, Hardy_2023g, Jehangir_2019h, Kumar_2001, Sultan_2024f, Thapa_2019f, Wadhwa_2015e, teWildt_2010}, 3 cases (4\%) were over 60 years of age \cite{Beecroft_1998, Kerestes_2019, Li_2013}, 2 cases (3\%) had no age documented \cite{Berry_2021e}. All 90 were male gender. 90 cases (100\%) were detained at the time of ingestion \cite{Elghali_2016, Karp_1991b, Lee_2007}, 88 cases (98\%) were intentional ingestions \cite{Elghali_2016, Karp_1991b, Lee_2007}, 30 cases (33\%) had a psychiatric history documented \cite{Elghali_2016, Karp_1991b, Lee_2007}, 2 cases (2\%) had a history of prior ingestion \cite{Elghali_2016}. No cases were reported for were psychiatric inpatients, were displaced people, were under the influence of alcohol at the time of ingestion, and had a severe disability history.
\paragraph*{Motivation}  70 cases (78\%) reported protest motivation \cite{Elghali_2016, Karp_1991b, Lee_2007}, 12 cases (13\%) reported psychiatric motivation \cite{Karp_1991b}, 6 cases (7\%) reported self-harm motivation \cite{Elghali_2016, Karp_1991b}. No cases were reported for psychosocial motivation and other motivation.
\paragraph*{Object Characteristics}  68 cases (76\%) involved sharp object ingestion \cite{Elghali_2016, Karp_1991b, Lee_2007}, 32 cases (36\%) involved long (\textgreater 5cm) object ingestion \cite{Lee_2007}, 25 cases (28\%) involved ingestion of multiple objects \cite{Elghali_2016, Lee_2007}. No cases were reported for button battery ingestion, magnet ingestion, and involved large diameter (\textgreater 2.5cm) object ingestion.
\paragraph*{Outcomes}  47 cases (52\%) underwent endoscopic intervention \cite{Elghali_2016, Lee_2007}, 29 cases (32\%) were managed conservatively \cite{Elghali_2016, Karp_1991b}, 15 cases (17\%) underwent surgical intervention \cite{Elghali_2016, Karp_1991b, Lee_2007}, 6 cases (7\%) reported complications \cite{Lee_2007}, 1 case (1\%) died \cite{Elghali_2016}.
\paragraph*{Gender} 43 cases (60\%) were male \cite{Akay_2015f, Al-Faham_2020k, Alao_2006i, Ali_2017, Ali_2022g, Apikotoa_2022f, Atayan_2016, Benoist_2019e, Berry_2021e, Bhumi_2024f, CamachoDorado_2018, Csaky_1998e, Emamhadi_2018, Farhadi_2024h, Fry_2010, Gardner_2017h, Guinan_2019f, Jehangir_2019h, Jin_2023, Kobiela_2015, Kumar_2001, Kumar_2019f, Liu_2005, Losanoff_1996, Losanoff_1997e, Mesfin_2022a, Misra_2013, Qureshi_2016, Riva_2018j, Sobnach_2011f, Tammana_2012j, Tanrikulu_2015e, Tay_2004, Thapa_2019f, Trgo_2012f, Wadhwa_2015e, Yasin_2009, teWildt_2010}, 28 cases (39\%) were female \cite{AlShaaibi_2021b, Ali_2020f, Ataya_2013, Beecroft_1998, Bhasin_2014, Bhattacharjee_2008, Cauchi_2002, Chang_2017f, Cox_2007, DelgadoSalazar_2020c, DivsalarP._2023a, Goldman_1998f, Hardy_2023g, Kar_2015, Kariholu_2008, Kerestes_2019, Li_2013, Naji_2012f, Ohno_2005, Peixoto_2017f, Sakellaridis_2008f, Sultan_2024f, Tupesis_2004f, Wildhaber_2005, Wnęk_2015f, Yildiz_2016e}, 1 case (1\%) had no gender recorded \cite{fjbuilsRepeatedBehaviorDeliberate2024}. \paragraph*{Age Group} 25 cases (35\%) were between 26 and 40 years of age \cite{Alao_2006i, Ali_2022g, Apikotoa_2022f, Ataya_2013, Benoist_2019e, Bhasin_2014, Chang_2017f, Cox_2007, DelgadoSalazar_2020c, Farhadi_2024h, Fry_2010, Gardner_2017h, Guinan_2019f, Jin_2023, Kumar_2019f, Losanoff_1996, Misra_2013, Qureshi_2016, Riva_2018j, Sakellaridis_2008f, Tammana_2012j, Trgo_2012f, Wnęk_2015f, Yildiz_2016e, fjbuilsRepeatedBehaviorDeliberate2024}, 18 cases (25\%) were between 18 and 25 years of age \cite{Akay_2015f, Ali_2017, Atayan_2016, Bhattacharjee_2008, Csaky_1998e, Kar_2015, Kariholu_2008, Kobiela_2015, Losanoff_1996, Losanoff_1997e, Mesfin_2022a, Peixoto_2017f, Sobnach_2011f, Tupesis_2004f, Yasin_2009}, 13 cases (18\%) were under 18 years of age \cite{AlShaaibi_2021b, Ali_2020f, Cauchi_2002, DivsalarP._2023a, Goldman_1998f, Liu_2005, Naji_2012f, Ohno_2005, Tanrikulu_2015e, Tay_2004, Wildhaber_2005}, 11 cases (15\%) were between 41 and 60 years of age \cite{Al-Faham_2020k, Bhumi_2024f, CamachoDorado_2018, Emamhadi_2018, Hardy_2023g, Jehangir_2019h, Kumar_2001, Sultan_2024f, Thapa_2019f, Wadhwa_2015e, teWildt_2010}, 3 cases (4\%) were over 60 years of age \cite{Beecroft_1998, Kerestes_2019, Li_2013}, 2 cases (3\%) had no age documented \cite{Berry_2021e}. All 90 were male gender. 90 cases (100\%) were detained at the time of ingestion \cite{Elghali_2016, Karp_1991b, Lee_2007}, 88 cases (98\%) were intentional ingestions \cite{Elghali_2016, Karp_1991b, Lee_2007}, 30 cases (33\%) had a psychiatric history documented \cite{Elghali_2016, Karp_1991b, Lee_2007}, 2 cases (2\%) had a history of prior ingestion \cite{Elghali_2016}. No cases were reported for were psychiatric inpatients, were displaced people, were under the influence of alcohol at the time of ingestion, and had a severe disability history.
\paragraph*{Motivation}  70 cases (78\%) reported protest motivation \cite{Elghali_2016, Karp_1991b, Lee_2007}, 12 cases (13\%) reported psychiatric motivation \cite{Karp_1991b}, 6 cases (7\%) reported self-harm motivation \cite{Elghali_2016, Karp_1991b}. No cases were reported for psychosocial motivation and other motivation.
\paragraph*{Object Characteristics}  68 cases (76\%) involved sharp object ingestion \cite{Elghali_2016, Karp_1991b, Lee_2007}, 32 cases (36\%) involved long (\textgreater 5cm) object ingestion \cite{Lee_2007}, 25 cases (28\%) involved ingestion of multiple objects \cite{Elghali_2016, Lee_2007}. No cases were reported for button battery ingestion, magnet ingestion, and involved large diameter (\textgreater 2.5cm) object ingestion.
\paragraph*{Outcomes}  47 cases (52\%) underwent endoscopic intervention \cite{Elghali_2016, Lee_2007}, 29 cases (32\%) were managed conservatively \cite{Elghali_2016, Karp_1991b}, 15 cases (17\%) underwent surgical intervention \cite{Elghali_2016, Karp_1991b, Lee_2007}, 6 cases (7\%) reported complications \cite{Lee_2007}, 1 case (1\%) died \cite{Elghali_2016}.
\paragraph*{Gender} 43 cases (60\%) were male \cite{Akay_2015f, Al-Faham_2020k, Alao_2006i, Ali_2017, Ali_2022g, Apikotoa_2022f, Atayan_2016, Benoist_2019e, Berry_2021e, Bhumi_2024f, CamachoDorado_2018, Csaky_1998e, Emamhadi_2018, Farhadi_2024h, Fry_2010, Gardner_2017h, Guinan_2019f, Jehangir_2019h, Jin_2023, Kobiela_2015, Kumar_2001, Kumar_2019f, Liu_2005, Losanoff_1996, Losanoff_1997e, Mesfin_2022a, Misra_2013, Qureshi_2016, Riva_2018j, Sobnach_2011f, Tammana_2012j, Tanrikulu_2015e, Tay_2004, Thapa_2019f, Trgo_2012f, Wadhwa_2015e, Yasin_2009, teWildt_2010}, 28 cases (39\%) were female \cite{AlShaaibi_2021b, Ali_2020f, Ataya_2013, Beecroft_1998, Bhasin_2014, Bhattacharjee_2008, Cauchi_2002, Chang_2017f, Cox_2007, DelgadoSalazar_2020c, DivsalarP._2023a, Goldman_1998f, Hardy_2023g, Kar_2015, Kariholu_2008, Kerestes_2019, Li_2013, Naji_2012f, Ohno_2005, Peixoto_2017f, Sakellaridis_2008f, Sultan_2024f, Tupesis_2004f, Wildhaber_2005, Wnęk_2015f, Yildiz_2016e}, 1 case (1\%) had no gender recorded \cite{fjbuilsRepeatedBehaviorDeliberate2024}. \paragraph*{Age Group} 25 cases (35\%) were between 26 and 40 years of age \cite{Alao_2006i, Ali_2022g, Apikotoa_2022f, Ataya_2013, Benoist_2019e, Bhasin_2014, Chang_2017f, Cox_2007, DelgadoSalazar_2020c, Farhadi_2024h, Fry_2010, Gardner_2017h, Guinan_2019f, Jin_2023, Kumar_2019f, Losanoff_1996, Misra_2013, Qureshi_2016, Riva_2018j, Sakellaridis_2008f, Tammana_2012j, Trgo_2012f, Wnęk_2015f, Yildiz_2016e, fjbuilsRepeatedBehaviorDeliberate2024}, 18 cases (25\%) were between 18 and 25 years of age \cite{Akay_2015f, Ali_2017, Atayan_2016, Bhattacharjee_2008, Csaky_1998e, Kar_2015, Kariholu_2008, Kobiela_2015, Losanoff_1996, Losanoff_1997e, Mesfin_2022a, Peixoto_2017f, Sobnach_2011f, Tupesis_2004f, Yasin_2009}, 13 cases (18\%) were under 18 years of age \cite{AlShaaibi_2021b, Ali_2020f, Cauchi_2002, DivsalarP._2023a, Goldman_1998f, Liu_2005, Naji_2012f, Ohno_2005, Tanrikulu_2015e, Tay_2004, Wildhaber_2005}, 11 cases (15\%) were between 41 and 60 years of age \cite{Al-Faham_2020k, Bhumi_2024f, CamachoDorado_2018, Emamhadi_2018, Hardy_2023g, Jehangir_2019h, Kumar_2001, Sultan_2024f, Thapa_2019f, Wadhwa_2015e, teWildt_2010}, 3 cases (4\%) were over 60 years of age \cite{Beecroft_1998, Kerestes_2019, Li_2013}, 2 cases (3\%) had no age documented \cite{Berry_2021e}. All 90 were male gender. 90 cases (100\%) were detained at the time of ingestion \cite{Elghali_2016, Karp_1991b, Lee_2007}, 88 cases (98\%) were intentional ingestions \cite{Elghali_2016, Karp_1991b, Lee_2007}, 30 cases (33\%) had a psychiatric history documented \cite{Elghali_2016, Karp_1991b, Lee_2007}, 2 cases (2\%) had a history of prior ingestion \cite{Elghali_2016}. No cases were reported for were psychiatric inpatients, were displaced people, were under the influence of alcohol at the time of ingestion, and had a severe disability history.
\paragraph*{Motivation}  70 cases (78\%) reported protest motivation \cite{Elghali_2016, Karp_1991b, Lee_2007}, 12 cases (13\%) reported psychiatric motivation \cite{Karp_1991b}, 6 cases (7\%) reported self-harm motivation \cite{Elghali_2016, Karp_1991b}. No cases were reported for psychosocial motivation and other motivation.
\paragraph*{Object Characteristics}  68 cases (76\%) involved sharp object ingestion \cite{Elghali_2016, Karp_1991b, Lee_2007}, 32 cases (36\%) involved long (\textgreater 5cm) object ingestion \cite{Lee_2007}, 25 cases (28\%) involved ingestion of multiple objects \cite{Elghali_2016, Lee_2007}. No cases were reported for button battery ingestion, magnet ingestion, and involved large diameter (\textgreater 2.5cm) object ingestion.
\paragraph*{Outcomes}  47 cases (52\%) underwent endoscopic intervention \cite{Elghali_2016, Lee_2007}, 29 cases (32\%) were managed conservatively \cite{Elghali_2016, Karp_1991b}, 15 cases (17\%) underwent surgical intervention \cite{Elghali_2016, Karp_1991b, Lee_2007}, 6 cases (7\%) reported complications \cite{Lee_2007}, 1 case (1\%) died \cite{Elghali_2016}.
\paragraph*{Gender} 43 cases (60\%) were male \cite{Akay_2015f, Al-Faham_2020k, Alao_2006i, Ali_2017, Ali_2022g, Apikotoa_2022f, Atayan_2016, Benoist_2019e, Berry_2021e, Bhumi_2024f, CamachoDorado_2018, Csaky_1998e, Emamhadi_2018, Farhadi_2024h, Fry_2010, Gardner_2017h, Guinan_2019f, Jehangir_2019h, Jin_2023, Kobiela_2015, Kumar_2001, Kumar_2019f, Liu_2005, Losanoff_1996, Losanoff_1997e, Mesfin_2022a, Misra_2013, Qureshi_2016, Riva_2018j, Sobnach_2011f, Tammana_2012j, Tanrikulu_2015e, Tay_2004, Thapa_2019f, Trgo_2012f, Wadhwa_2015e, Yasin_2009, teWildt_2010}, 28 cases (39\%) were female \cite{AlShaaibi_2021b, Ali_2020f, Ataya_2013, Beecroft_1998, Bhasin_2014, Bhattacharjee_2008, Cauchi_2002, Chang_2017f, Cox_2007, DelgadoSalazar_2020c, DivsalarP._2023a, Goldman_1998f, Hardy_2023g, Kar_2015, Kariholu_2008, Kerestes_2019, Li_2013, Naji_2012f, Ohno_2005, Peixoto_2017f, Sakellaridis_2008f, Sultan_2024f, Tupesis_2004f, Wildhaber_2005, Wnęk_2015f, Yildiz_2016e}, 1 case (1\%) had no gender recorded \cite{fjbuilsRepeatedBehaviorDeliberate2024}. \paragraph*{Age Group} 25 cases (35\%) were between 26 and 40 years of age \cite{Alao_2006i, Ali_2022g, Apikotoa_2022f, Ataya_2013, Benoist_2019e, Bhasin_2014, Chang_2017f, Cox_2007, DelgadoSalazar_2020c, Farhadi_2024h, Fry_2010, Gardner_2017h, Guinan_2019f, Jin_2023, Kumar_2019f, Losanoff_1996, Misra_2013, Qureshi_2016, Riva_2018j, Sakellaridis_2008f, Tammana_2012j, Trgo_2012f, Wnęk_2015f, Yildiz_2016e, fjbuilsRepeatedBehaviorDeliberate2024}, 18 cases (25\%) were between 18 and 25 years of age \cite{Akay_2015f, Ali_2017, Atayan_2016, Bhattacharjee_2008, Csaky_1998e, Kar_2015, Kariholu_2008, Kobiela_2015, Losanoff_1996, Losanoff_1997e, Mesfin_2022a, Peixoto_2017f, Sobnach_2011f, Tupesis_2004f, Yasin_2009}, 13 cases (18\%) were under 18 years of age \cite{AlShaaibi_2021b, Ali_2020f, Cauchi_2002, DivsalarP._2023a, Goldman_1998f, Liu_2005, Naji_2012f, Ohno_2005, Tanrikulu_2015e, Tay_2004, Wildhaber_2005}, 11 cases (15\%) were between 41 and 60 years of age \cite{Al-Faham_2020k, Bhumi_2024f, CamachoDorado_2018, Emamhadi_2018, Hardy_2023g, Jehangir_2019h, Kumar_2001, Sultan_2024f, Thapa_2019f, Wadhwa_2015e, teWildt_2010}, 3 cases (4\%) were over 60 years of age \cite{Beecroft_1998, Kerestes_2019, Li_2013}, 2 cases (3\%) had no age documented \cite{Berry_2021e}. All 90 were male gender. 90 cases (100\%) were detained at the time of ingestion \cite{Elghali_2016, Karp_1991b, Lee_2007}, 88 cases (98\%) were intentional ingestions \cite{Elghali_2016, Karp_1991b, Lee_2007}, 30 cases (33\%) had a psychiatric history documented \cite{Elghali_2016, Karp_1991b, Lee_2007}, 2 cases (2\%) had a history of prior ingestion \cite{Elghali_2016}. No cases were reported for were psychiatric inpatients, were displaced people, were under the influence of alcohol at the time of ingestion, and had a severe disability history.
\paragraph*{Motivation}  70 cases (78\%) reported protest motivation \cite{Elghali_2016, Karp_1991b, Lee_2007}, 12 cases (13\%) reported psychiatric motivation \cite{Karp_1991b}, 6 cases (7\%) reported self-harm motivation \cite{Elghali_2016, Karp_1991b}. No cases were reported for psychosocial motivation and other motivation.
\paragraph*{Object Characteristics}  68 cases (76\%) involved sharp object ingestion \cite{Elghali_2016, Karp_1991b, Lee_2007}, 32 cases (36\%) involved long (\textgreater 5cm) object ingestion \cite{Lee_2007}, 25 cases (28\%) involved ingestion of multiple objects \cite{Elghali_2016, Lee_2007}. No cases were reported for button battery ingestion, magnet ingestion, and involved large diameter (\textgreater 2.5cm) object ingestion.
\paragraph*{Outcomes}  47 cases (52\%) underwent endoscopic intervention \cite{Elghali_2016, Lee_2007}, 29 cases (32\%) were managed conservatively \cite{Elghali_2016, Karp_1991b}, 15 cases (17\%) underwent surgical intervention \cite{Elghali_2016, Karp_1991b, Lee_2007}, 6 cases (7\%) reported complications \cite{Lee_2007}, 1 case (1\%) died \cite{Elghali_2016}.
\paragraph*{Gender} 43 cases (60\%) were male \cite{Akay_2015f, Al-Faham_2020k, Alao_2006i, Ali_2017, Ali_2022g, Apikotoa_2022f, Atayan_2016, Benoist_2019e, Berry_2021e, Bhumi_2024f, CamachoDorado_2018, Csaky_1998e, Emamhadi_2018, Farhadi_2024h, Fry_2010, Gardner_2017h, Guinan_2019f, Jehangir_2019h, Jin_2023, Kobiela_2015, Kumar_2001, Kumar_2019f, Liu_2005, Losanoff_1996, Losanoff_1997e, Mesfin_2022a, Misra_2013, Qureshi_2016, Riva_2018j, Sobnach_2011f, Tammana_2012j, Tanrikulu_2015e, Tay_2004, Thapa_2019f, Trgo_2012f, Wadhwa_2015e, Yasin_2009, teWildt_2010}, 28 cases (39\%) were female \cite{AlShaaibi_2021b, Ali_2020f, Ataya_2013, Beecroft_1998, Bhasin_2014, Bhattacharjee_2008, Cauchi_2002, Chang_2017f, Cox_2007, DelgadoSalazar_2020c, DivsalarP._2023a, Goldman_1998f, Hardy_2023g, Kar_2015, Kariholu_2008, Kerestes_2019, Li_2013, Naji_2012f, Ohno_2005, Peixoto_2017f, Sakellaridis_2008f, Sultan_2024f, Tupesis_2004f, Wildhaber_2005, Wnęk_2015f, Yildiz_2016e}, 1 case (1\%) had no gender recorded \cite{fjbuilsRepeatedBehaviorDeliberate2024}. \paragraph*{Age Group} 25 cases (35\%) were between 26 and 40 years of age \cite{Alao_2006i, Ali_2022g, Apikotoa_2022f, Ataya_2013, Benoist_2019e, Bhasin_2014, Chang_2017f, Cox_2007, DelgadoSalazar_2020c, Farhadi_2024h, Fry_2010, Gardner_2017h, Guinan_2019f, Jin_2023, Kumar_2019f, Losanoff_1996, Misra_2013, Qureshi_2016, Riva_2018j, Sakellaridis_2008f, Tammana_2012j, Trgo_2012f, Wnęk_2015f, Yildiz_2016e, fjbuilsRepeatedBehaviorDeliberate2024}, 18 cases (25\%) were between 18 and 25 years of age \cite{Akay_2015f, Ali_2017, Atayan_2016, Bhattacharjee_2008, Csaky_1998e, Kar_2015, Kariholu_2008, Kobiela_2015, Losanoff_1996, Losanoff_1997e, Mesfin_2022a, Peixoto_2017f, Sobnach_2011f, Tupesis_2004f, Yasin_2009}, 13 cases (18\%) were under 18 years of age \cite{AlShaaibi_2021b, Ali_2020f, Cauchi_2002, DivsalarP._2023a, Goldman_1998f, Liu_2005, Naji_2012f, Ohno_2005, Tanrikulu_2015e, Tay_2004, Wildhaber_2005}, 11 cases (15\%) were between 41 and 60 years of age \cite{Al-Faham_2020k, Bhumi_2024f, CamachoDorado_2018, Emamhadi_2018, Hardy_2023g, Jehangir_2019h, Kumar_2001, Sultan_2024f, Thapa_2019f, Wadhwa_2015e, teWildt_2010}, 3 cases (4\%) were over 60 years of age \cite{Beecroft_1998, Kerestes_2019, Li_2013}, 2 cases (3\%) had no age documented \cite{Berry_2021e}. All 90 were male gender. 90 cases (100\%) were detained at the time of ingestion \cite{Elghali_2016, Karp_1991b, Lee_2007}, 88 cases (98\%) were intentional ingestions \cite{Elghali_2016, Karp_1991b, Lee_2007}, 30 cases (33\%) had a psychiatric history documented \cite{Elghali_2016, Karp_1991b, Lee_2007}, 2 cases (2\%) had a history of prior ingestion \cite{Elghali_2016}. No cases were reported for were psychiatric inpatients, were displaced people, were under the influence of alcohol at the time of ingestion, and had a severe disability history.
\paragraph*{Motivation}  70 cases (78\%) reported protest motivation \cite{Elghali_2016, Karp_1991b, Lee_2007}, 12 cases (13\%) reported psychiatric motivation \cite{Karp_1991b}, 6 cases (7\%) reported self-harm motivation \cite{Elghali_2016, Karp_1991b}. No cases were reported for psychosocial motivation and other motivation.
\paragraph*{Object Characteristics}  68 cases (76\%) involved sharp object ingestion \cite{Elghali_2016, Karp_1991b, Lee_2007}, 32 cases (36\%) involved long (\textgreater 5cm) object ingestion \cite{Lee_2007}, 25 cases (28\%) involved ingestion of multiple objects \cite{Elghali_2016, Lee_2007}. No cases were reported for button battery ingestion, magnet ingestion, and involved large diameter (\textgreater 2.5cm) object ingestion.
\paragraph*{Outcomes}  47 cases (52\%) underwent endoscopic intervention \cite{Elghali_2016, Lee_2007}, 29 cases (32\%) were managed conservatively \cite{Elghali_2016, Karp_1991b}, 15 cases (17\%) underwent surgical intervention \cite{Elghali_2016, Karp_1991b, Lee_2007}, 6 cases (7\%) reported complications \cite{Lee_2007}, 1 case (1\%) died \cite{Elghali_2016}.
\paragraph*{Gender} 43 cases (60\%) were male \cite{Akay_2015f, Al-Faham_2020k, Alao_2006i, Ali_2017, Ali_2022g, Apikotoa_2022f, Atayan_2016, Benoist_2019e, Berry_2021e, Bhumi_2024f, CamachoDorado_2018, Csaky_1998e, Emamhadi_2018, Farhadi_2024h, Fry_2010, Gardner_2017h, Guinan_2019f, Jehangir_2019h, Jin_2023, Kobiela_2015, Kumar_2001, Kumar_2019f, Liu_2005, Losanoff_1996, Losanoff_1997e, Mesfin_2022a, Misra_2013, Qureshi_2016, Riva_2018j, Sobnach_2011f, Tammana_2012j, Tanrikulu_2015e, Tay_2004, Thapa_2019f, Trgo_2012f, Wadhwa_2015e, Yasin_2009, teWildt_2010}, 28 cases (39\%) were female \cite{AlShaaibi_2021b, Ali_2020f, Ataya_2013, Beecroft_1998, Bhasin_2014, Bhattacharjee_2008, Cauchi_2002, Chang_2017f, Cox_2007, DelgadoSalazar_2020c, DivsalarP._2023a, Goldman_1998f, Hardy_2023g, Kar_2015, Kariholu_2008, Kerestes_2019, Li_2013, Naji_2012f, Ohno_2005, Peixoto_2017f, Sakellaridis_2008f, Sultan_2024f, Tupesis_2004f, Wildhaber_2005, Wnęk_2015f, Yildiz_2016e}, 1 case (1\%) had no gender recorded \cite{fjbuilsRepeatedBehaviorDeliberate2024}. \paragraph*{Age Group} 25 cases (35\%) were between 26 and 40 years of age \cite{Alao_2006i, Ali_2022g, Apikotoa_2022f, Ataya_2013, Benoist_2019e, Bhasin_2014, Chang_2017f, Cox_2007, DelgadoSalazar_2020c, Farhadi_2024h, Fry_2010, Gardner_2017h, Guinan_2019f, Jin_2023, Kumar_2019f, Losanoff_1996, Misra_2013, Qureshi_2016, Riva_2018j, Sakellaridis_2008f, Tammana_2012j, Trgo_2012f, Wnęk_2015f, Yildiz_2016e, fjbuilsRepeatedBehaviorDeliberate2024}, 18 cases (25\%) were between 18 and 25 years of age \cite{Akay_2015f, Ali_2017, Atayan_2016, Bhattacharjee_2008, Csaky_1998e, Kar_2015, Kariholu_2008, Kobiela_2015, Losanoff_1996, Losanoff_1997e, Mesfin_2022a, Peixoto_2017f, Sobnach_2011f, Tupesis_2004f, Yasin_2009}, 13 cases (18\%) were under 18 years of age \cite{AlShaaibi_2021b, Ali_2020f, Cauchi_2002, DivsalarP._2023a, Goldman_1998f, Liu_2005, Naji_2012f, Ohno_2005, Tanrikulu_2015e, Tay_2004, Wildhaber_2005}, 11 cases (15\%) were between 41 and 60 years of age \cite{Al-Faham_2020k, Bhumi_2024f, CamachoDorado_2018, Emamhadi_2018, Hardy_2023g, Jehangir_2019h, Kumar_2001, Sultan_2024f, Thapa_2019f, Wadhwa_2015e, teWildt_2010}, 3 cases (4\%) were over 60 years of age \cite{Beecroft_1998, Kerestes_2019, Li_2013}, 2 cases (3\%) had no age documented \cite{Berry_2021e}. All 90 were male gender. 90 cases (100\%) were detained at the time of ingestion \cite{Elghali_2016, Karp_1991b, Lee_2007}, 88 cases (98\%) were intentional ingestions \cite{Elghali_2016, Karp_1991b, Lee_2007}, 30 cases (33\%) had a psychiatric history documented \cite{Elghali_2016, Karp_1991b, Lee_2007}, 2 cases (2\%) had a history of prior ingestion \cite{Elghali_2016}. No cases were reported for were psychiatric inpatients, were displaced people, were under the influence of alcohol at the time of ingestion, and had a severe disability history.
\paragraph*{Motivation}  70 cases (78\%) reported protest motivation \cite{Elghali_2016, Karp_1991b, Lee_2007}, 12 cases (13\%) reported psychiatric motivation \cite{Karp_1991b}, 6 cases (7\%) reported self-harm motivation \cite{Elghali_2016, Karp_1991b}. No cases were reported for psychosocial motivation and other motivation.
\paragraph*{Object Characteristics}  68 cases (76\%) involved sharp object ingestion \cite{Elghali_2016, Karp_1991b, Lee_2007}, 32 cases (36\%) involved long (\textgreater 5cm) object ingestion \cite{Lee_2007}, 25 cases (28\%) involved ingestion of multiple objects \cite{Elghali_2016, Lee_2007}. No cases were reported for button battery ingestion, magnet ingestion, and involved large diameter (\textgreater 2.5cm) object ingestion.
\paragraph*{Outcomes}  47 cases (52\%) underwent endoscopic intervention \cite{Elghali_2016, Lee_2007}, 29 cases (32\%) were managed conservatively \cite{Elghali_2016, Karp_1991b}, 15 cases (17\%) underwent surgical intervention \cite{Elghali_2016, Karp_1991b, Lee_2007}, 6 cases (7\%) reported complications \cite{Lee_2007}, 1 case (1\%) died \cite{Elghali_2016}.
\paragraph*{Gender} 43 cases (60\%) were male \cite{Akay_2015f, Al-Faham_2020k, Alao_2006i, Ali_2017, Ali_2022g, Apikotoa_2022f, Atayan_2016, Benoist_2019e, Berry_2021e, Bhumi_2024f, CamachoDorado_2018, Csaky_1998e, Emamhadi_2018, Farhadi_2024h, Fry_2010, Gardner_2017h, Guinan_2019f, Jehangir_2019h, Jin_2023, Kobiela_2015, Kumar_2001, Kumar_2019f, Liu_2005, Losanoff_1996, Losanoff_1997e, Mesfin_2022a, Misra_2013, Qureshi_2016, Riva_2018j, Sobnach_2011f, Tammana_2012j, Tanrikulu_2015e, Tay_2004, Thapa_2019f, Trgo_2012f, Wadhwa_2015e, Yasin_2009, teWildt_2010}, 28 cases (39\%) were female \cite{AlShaaibi_2021b, Ali_2020f, Ataya_2013, Beecroft_1998, Bhasin_2014, Bhattacharjee_2008, Cauchi_2002, Chang_2017f, Cox_2007, DelgadoSalazar_2020c, DivsalarP._2023a, Goldman_1998f, Hardy_2023g, Kar_2015, Kariholu_2008, Kerestes_2019, Li_2013, Naji_2012f, Ohno_2005, Peixoto_2017f, Sakellaridis_2008f, Sultan_2024f, Tupesis_2004f, Wildhaber_2005, Wnęk_2015f, Yildiz_2016e}, 1 case (1\%) had no gender recorded \cite{fjbuilsRepeatedBehaviorDeliberate2024}. \paragraph*{Age Group} 25 cases (35\%) were between 26 and 40 years of age \cite{Alao_2006i, Ali_2022g, Apikotoa_2022f, Ataya_2013, Benoist_2019e, Bhasin_2014, Chang_2017f, Cox_2007, DelgadoSalazar_2020c, Farhadi_2024h, Fry_2010, Gardner_2017h, Guinan_2019f, Jin_2023, Kumar_2019f, Losanoff_1996, Misra_2013, Qureshi_2016, Riva_2018j, Sakellaridis_2008f, Tammana_2012j, Trgo_2012f, Wnęk_2015f, Yildiz_2016e, fjbuilsRepeatedBehaviorDeliberate2024}, 18 cases (25\%) were between 18 and 25 years of age \cite{Akay_2015f, Ali_2017, Atayan_2016, Bhattacharjee_2008, Csaky_1998e, Kar_2015, Kariholu_2008, Kobiela_2015, Losanoff_1996, Losanoff_1997e, Mesfin_2022a, Peixoto_2017f, Sobnach_2011f, Tupesis_2004f, Yasin_2009}, 13 cases (18\%) were under 18 years of age \cite{AlShaaibi_2021b, Ali_2020f, Cauchi_2002, DivsalarP._2023a, Goldman_1998f, Liu_2005, Naji_2012f, Ohno_2005, Tanrikulu_2015e, Tay_2004, Wildhaber_2005}, 11 cases (15\%) were between 41 and 60 years of age \cite{Al-Faham_2020k, Bhumi_2024f, CamachoDorado_2018, Emamhadi_2018, Hardy_2023g, Jehangir_2019h, Kumar_2001, Sultan_2024f, Thapa_2019f, Wadhwa_2015e, teWildt_2010}, 3 cases (4\%) were over 60 years of age \cite{Beecroft_1998, Kerestes_2019, Li_2013}, 2 cases (3\%) had no age documented \cite{Berry_2021e}. All 90 were male gender. 90 cases (100\%) were detained at the time of ingestion \cite{Elghali_2016, Karp_1991b, Lee_2007}, 88 cases (98\%) were intentional ingestions \cite{Elghali_2016, Karp_1991b, Lee_2007}, 30 cases (33\%) had a psychiatric history documented \cite{Elghali_2016, Karp_1991b, Lee_2007}, 2 cases (2\%) had a history of prior ingestion \cite{Elghali_2016}. No cases were reported for were psychiatric inpatients, were displaced people, were under the influence of alcohol at the time of ingestion, and had a severe disability history.
\paragraph*{Motivation}  70 cases (78\%) reported protest motivation \cite{Elghali_2016, Karp_1991b, Lee_2007}, 12 cases (13\%) reported psychiatric motivation \cite{Karp_1991b}, 6 cases (7\%) reported self-harm motivation \cite{Elghali_2016, Karp_1991b}. No cases were reported for psychosocial motivation and other motivation.
\paragraph*{Object Characteristics}  68 cases (76\%) involved sharp object ingestion \cite{Elghali_2016, Karp_1991b, Lee_2007}, 32 cases (36\%) involved long (\textgreater 5cm) object ingestion \cite{Lee_2007}, 25 cases (28\%) involved ingestion of multiple objects \cite{Elghali_2016, Lee_2007}. No cases were reported for button battery ingestion, magnet ingestion, and involved large diameter (\textgreater 2.5cm) object ingestion.
\paragraph*{Outcomes}  47 cases (52\%) underwent endoscopic intervention \cite{Elghali_2016, Lee_2007}, 29 cases (32\%) were managed conservatively \cite{Elghali_2016, Karp_1991b}, 15 cases (17\%) underwent surgical intervention \cite{Elghali_2016, Karp_1991b, Lee_2007}, 6 cases (7\%) reported complications \cite{Lee_2007}, 1 case (1\%) died \cite{Elghali_2016}.
\paragraph*{Gender} 43 cases (60\%) were male \cite{Akay_2015f, Al-Faham_2020k, Alao_2006i, Ali_2017, Ali_2022g, Apikotoa_2022f, Atayan_2016, Benoist_2019e, Berry_2021e, Bhumi_2024f, CamachoDorado_2018, Csaky_1998e, Emamhadi_2018, Farhadi_2024h, Fry_2010, Gardner_2017h, Guinan_2019f, Jehangir_2019h, Jin_2023, Kobiela_2015, Kumar_2001, Kumar_2019f, Liu_2005, Losanoff_1996, Losanoff_1997e, Mesfin_2022a, Misra_2013, Qureshi_2016, Riva_2018j, Sobnach_2011f, Tammana_2012j, Tanrikulu_2015e, Tay_2004, Thapa_2019f, Trgo_2012f, Wadhwa_2015e, Yasin_2009, teWildt_2010}, 28 cases (39\%) were female \cite{AlShaaibi_2021b, Ali_2020f, Ataya_2013, Beecroft_1998, Bhasin_2014, Bhattacharjee_2008, Cauchi_2002, Chang_2017f, Cox_2007, DelgadoSalazar_2020c, DivsalarP._2023a, Goldman_1998f, Hardy_2023g, Kar_2015, Kariholu_2008, Kerestes_2019, Li_2013, Naji_2012f, Ohno_2005, Peixoto_2017f, Sakellaridis_2008f, Sultan_2024f, Tupesis_2004f, Wildhaber_2005, Wnęk_2015f, Yildiz_2016e}, 1 case (1\%) had no gender recorded \cite{fjbuilsRepeatedBehaviorDeliberate2024}. \paragraph*{Age Group} 25 cases (35\%) were between 26 and 40 years of age \cite{Alao_2006i, Ali_2022g, Apikotoa_2022f, Ataya_2013, Benoist_2019e, Bhasin_2014, Chang_2017f, Cox_2007, DelgadoSalazar_2020c, Farhadi_2024h, Fry_2010, Gardner_2017h, Guinan_2019f, Jin_2023, Kumar_2019f, Losanoff_1996, Misra_2013, Qureshi_2016, Riva_2018j, Sakellaridis_2008f, Tammana_2012j, Trgo_2012f, Wnęk_2015f, Yildiz_2016e, fjbuilsRepeatedBehaviorDeliberate2024}, 18 cases (25\%) were between 18 and 25 years of age \cite{Akay_2015f, Ali_2017, Atayan_2016, Bhattacharjee_2008, Csaky_1998e, Kar_2015, Kariholu_2008, Kobiela_2015, Losanoff_1996, Losanoff_1997e, Mesfin_2022a, Peixoto_2017f, Sobnach_2011f, Tupesis_2004f, Yasin_2009}, 13 cases (18\%) were under 18 years of age \cite{AlShaaibi_2021b, Ali_2020f, Cauchi_2002, DivsalarP._2023a, Goldman_1998f, Liu_2005, Naji_2012f, Ohno_2005, Tanrikulu_2015e, Tay_2004, Wildhaber_2005}, 11 cases (15\%) were between 41 and 60 years of age \cite{Al-Faham_2020k, Bhumi_2024f, CamachoDorado_2018, Emamhadi_2018, Hardy_2023g, Jehangir_2019h, Kumar_2001, Sultan_2024f, Thapa_2019f, Wadhwa_2015e, teWildt_2010}, 3 cases (4\%) were over 60 years of age \cite{Beecroft_1998, Kerestes_2019, Li_2013}, 2 cases (3\%) had no age documented \cite{Berry_2021e}. All 90 were male gender. 90 cases (100\%) were detained at the time of ingestion \cite{Elghali_2016, Karp_1991b, Lee_2007}, 88 cases (98\%) were intentional ingestions \cite{Elghali_2016, Karp_1991b, Lee_2007}, 30 cases (33\%) had a psychiatric history documented \cite{Elghali_2016, Karp_1991b, Lee_2007}, 2 cases (2\%) had a history of prior ingestion \cite{Elghali_2016}. No cases were reported for were psychiatric inpatients, were displaced people, were under the influence of alcohol at the time of ingestion, and had a severe disability history.
\paragraph*{Motivation}  70 cases (78\%) reported protest motivation \cite{Elghali_2016, Karp_1991b, Lee_2007}, 12 cases (13\%) reported psychiatric motivation \cite{Karp_1991b}, 6 cases (7\%) reported self-harm motivation \cite{Elghali_2016, Karp_1991b}. No cases were reported for psychosocial motivation and other motivation.
\paragraph*{Object Characteristics}  68 cases (76\%) involved sharp object ingestion \cite{Elghali_2016, Karp_1991b, Lee_2007}, 32 cases (36\%) involved long (\textgreater 5cm) object ingestion \cite{Lee_2007}, 25 cases (28\%) involved ingestion of multiple objects \cite{Elghali_2016, Lee_2007}. No cases were reported for button battery ingestion, magnet ingestion, and involved large diameter (\textgreater 2.5cm) object ingestion.
\paragraph*{Outcomes}  47 cases (52\%) underwent endoscopic intervention \cite{Elghali_2016, Lee_2007}, 29 cases (32\%) were managed conservatively \cite{Elghali_2016, Karp_1991b}, 15 cases (17\%) underwent surgical intervention \cite{Elghali_2016, Karp_1991b, Lee_2007}, 6 cases (7\%) reported complications \cite{Lee_2007}, 1 case (1\%) died \cite{Elghali_2016}.
\paragraph*{Gender} 43 cases (60\%) were male \cite{Akay_2015f, Al-Faham_2020k, Alao_2006i, Ali_2017, Ali_2022g, Apikotoa_2022f, Atayan_2016, Benoist_2019e, Berry_2021e, Bhumi_2024f, CamachoDorado_2018, Csaky_1998e, Emamhadi_2018, Farhadi_2024h, Fry_2010, Gardner_2017h, Guinan_2019f, Jehangir_2019h, Jin_2023, Kobiela_2015, Kumar_2001, Kumar_2019f, Liu_2005, Losanoff_1996, Losanoff_1997e, Mesfin_2022a, Misra_2013, Qureshi_2016, Riva_2018j, Sobnach_2011f, Tammana_2012j, Tanrikulu_2015e, Tay_2004, Thapa_2019f, Trgo_2012f, Wadhwa_2015e, Yasin_2009, teWildt_2010}, 28 cases (39\%) were female \cite{AlShaaibi_2021b, Ali_2020f, Ataya_2013, Beecroft_1998, Bhasin_2014, Bhattacharjee_2008, Cauchi_2002, Chang_2017f, Cox_2007, DelgadoSalazar_2020c, DivsalarP._2023a, Goldman_1998f, Hardy_2023g, Kar_2015, Kariholu_2008, Kerestes_2019, Li_2013, Naji_2012f, Ohno_2005, Peixoto_2017f, Sakellaridis_2008f, Sultan_2024f, Tupesis_2004f, Wildhaber_2005, Wnęk_2015f, Yildiz_2016e}, 1 case (1\%) had no gender recorded \cite{fjbuilsRepeatedBehaviorDeliberate2024}. \paragraph*{Age Group} 25 cases (35\%) were between 26 and 40 years of age \cite{Alao_2006i, Ali_2022g, Apikotoa_2022f, Ataya_2013, Benoist_2019e, Bhasin_2014, Chang_2017f, Cox_2007, DelgadoSalazar_2020c, Farhadi_2024h, Fry_2010, Gardner_2017h, Guinan_2019f, Jin_2023, Kumar_2019f, Losanoff_1996, Misra_2013, Qureshi_2016, Riva_2018j, Sakellaridis_2008f, Tammana_2012j, Trgo_2012f, Wnęk_2015f, Yildiz_2016e, fjbuilsRepeatedBehaviorDeliberate2024}, 18 cases (25\%) were between 18 and 25 years of age \cite{Akay_2015f, Ali_2017, Atayan_2016, Bhattacharjee_2008, Csaky_1998e, Kar_2015, Kariholu_2008, Kobiela_2015, Losanoff_1996, Losanoff_1997e, Mesfin_2022a, Peixoto_2017f, Sobnach_2011f, Tupesis_2004f, Yasin_2009}, 13 cases (18\%) were under 18 years of age \cite{AlShaaibi_2021b, Ali_2020f, Cauchi_2002, DivsalarP._2023a, Goldman_1998f, Liu_2005, Naji_2012f, Ohno_2005, Tanrikulu_2015e, Tay_2004, Wildhaber_2005}, 11 cases (15\%) were between 41 and 60 years of age \cite{Al-Faham_2020k, Bhumi_2024f, CamachoDorado_2018, Emamhadi_2018, Hardy_2023g, Jehangir_2019h, Kumar_2001, Sultan_2024f, Thapa_2019f, Wadhwa_2015e, teWildt_2010}, 3 cases (4\%) were over 60 years of age \cite{Beecroft_1998, Kerestes_2019, Li_2013}, 2 cases (3\%) had no age documented \cite{Berry_2021e}. All 90 were male gender. 90 cases (100\%) were detained at the time of ingestion \cite{Elghali_2016, Karp_1991b, Lee_2007}, 88 cases (98\%) were intentional ingestions \cite{Elghali_2016, Karp_1991b, Lee_2007}, 30 cases (33\%) had a psychiatric history documented \cite{Elghali_2016, Karp_1991b, Lee_2007}, 2 cases (2\%) had a history of prior ingestion \cite{Elghali_2016}. No cases were reported for were psychiatric inpatients, were displaced people, were under the influence of alcohol at the time of ingestion, and had a severe disability history.
\paragraph*{Motivation}  70 cases (78\%) reported protest motivation \cite{Elghali_2016, Karp_1991b, Lee_2007}, 12 cases (13\%) reported psychiatric motivation \cite{Karp_1991b}, 6 cases (7\%) reported self-harm motivation \cite{Elghali_2016, Karp_1991b}. No cases were reported for psychosocial motivation and other motivation.
\paragraph*{Object Characteristics}  68 cases (76\%) involved sharp object ingestion \cite{Elghali_2016, Karp_1991b, Lee_2007}, 32 cases (36\%) involved long (\textgreater 5cm) object ingestion \cite{Lee_2007}, 25 cases (28\%) involved ingestion of multiple objects \cite{Elghali_2016, Lee_2007}. No cases were reported for button battery ingestion, magnet ingestion, and involved large diameter (\textgreater 2.5cm) object ingestion.
\paragraph*{Outcomes}  47 cases (52\%) underwent endoscopic intervention \cite{Elghali_2016, Lee_2007}, 29 cases (32\%) were managed conservatively \cite{Elghali_2016, Karp_1991b}, 15 cases (17\%) underwent surgical intervention \cite{Elghali_2016, Karp_1991b, Lee_2007}, 6 cases (7\%) reported complications \cite{Lee_2007}, 1 case (1\%) died \cite{Elghali_2016}.
\paragraph*{Gender} 43 cases (60\%) were male \cite{Akay_2015f, Al-Faham_2020k, Alao_2006i, Ali_2017, Ali_2022g, Apikotoa_2022f, Atayan_2016, Benoist_2019e, Berry_2021e, Bhumi_2024f, CamachoDorado_2018, Csaky_1998e, Emamhadi_2018, Farhadi_2024h, Fry_2010, Gardner_2017h, Guinan_2019f, Jehangir_2019h, Jin_2023, Kobiela_2015, Kumar_2001, Kumar_2019f, Liu_2005, Losanoff_1996, Losanoff_1997e, Mesfin_2022a, Misra_2013, Qureshi_2016, Riva_2018j, Sobnach_2011f, Tammana_2012j, Tanrikulu_2015e, Tay_2004, Thapa_2019f, Trgo_2012f, Wadhwa_2015e, Yasin_2009, teWildt_2010}, 28 cases (39\%) were female \cite{AlShaaibi_2021b, Ali_2020f, Ataya_2013, Beecroft_1998, Bhasin_2014, Bhattacharjee_2008, Cauchi_2002, Chang_2017f, Cox_2007, DelgadoSalazar_2020c, DivsalarP._2023a, Goldman_1998f, Hardy_2023g, Kar_2015, Kariholu_2008, Kerestes_2019, Li_2013, Naji_2012f, Ohno_2005, Peixoto_2017f, Sakellaridis_2008f, Sultan_2024f, Tupesis_2004f, Wildhaber_2005, Wnęk_2015f, Yildiz_2016e}, 1 case (1\%) had no gender recorded \cite{fjbuilsRepeatedBehaviorDeliberate2024}. \paragraph*{Age Group} 25 cases (35\%) were between 26 and 40 years of age \cite{Alao_2006i, Ali_2022g, Apikotoa_2022f, Ataya_2013, Benoist_2019e, Bhasin_2014, Chang_2017f, Cox_2007, DelgadoSalazar_2020c, Farhadi_2024h, Fry_2010, Gardner_2017h, Guinan_2019f, Jin_2023, Kumar_2019f, Losanoff_1996, Misra_2013, Qureshi_2016, Riva_2018j, Sakellaridis_2008f, Tammana_2012j, Trgo_2012f, Wnęk_2015f, Yildiz_2016e, fjbuilsRepeatedBehaviorDeliberate2024}, 18 cases (25\%) were between 18 and 25 years of age \cite{Akay_2015f, Ali_2017, Atayan_2016, Bhattacharjee_2008, Csaky_1998e, Kar_2015, Kariholu_2008, Kobiela_2015, Losanoff_1996, Losanoff_1997e, Mesfin_2022a, Peixoto_2017f, Sobnach_2011f, Tupesis_2004f, Yasin_2009}, 13 cases (18\%) were under 18 years of age \cite{AlShaaibi_2021b, Ali_2020f, Cauchi_2002, DivsalarP._2023a, Goldman_1998f, Liu_2005, Naji_2012f, Ohno_2005, Tanrikulu_2015e, Tay_2004, Wildhaber_2005}, 11 cases (15\%) were between 41 and 60 years of age \cite{Al-Faham_2020k, Bhumi_2024f, CamachoDorado_2018, Emamhadi_2018, Hardy_2023g, Jehangir_2019h, Kumar_2001, Sultan_2024f, Thapa_2019f, Wadhwa_2015e, teWildt_2010}, 3 cases (4\%) were over 60 years of age \cite{Beecroft_1998, Kerestes_2019, Li_2013}, 2 cases (3\%) had no age documented \cite{Berry_2021e}. All 90 were male gender. 90 cases (100\%) were detained at the time of ingestion \cite{Elghali_2016, Karp_1991b, Lee_2007}, 88 cases (98\%) were intentional ingestions \cite{Elghali_2016, Karp_1991b, Lee_2007}, 30 cases (33\%) had a psychiatric history documented \cite{Elghali_2016, Karp_1991b, Lee_2007}, 2 cases (2\%) had a history of prior ingestion \cite{Elghali_2016}. No cases were reported for were psychiatric inpatients, were displaced people, were under the influence of alcohol at the time of ingestion, and had a severe disability history.
\paragraph*{Motivation}  70 cases (78\%) reported protest motivation \cite{Elghali_2016, Karp_1991b, Lee_2007}, 12 cases (13\%) reported psychiatric motivation \cite{Karp_1991b}, 6 cases (7\%) reported self-harm motivation \cite{Elghali_2016, Karp_1991b}. No cases were reported for psychosocial motivation and other motivation.
\paragraph*{Object Characteristics}  68 cases (76\%) involved sharp object ingestion \cite{Elghali_2016, Karp_1991b, Lee_2007}, 32 cases (36\%) involved long (\textgreater 5cm) object ingestion \cite{Lee_2007}, 25 cases (28\%) involved ingestion of multiple objects \cite{Elghali_2016, Lee_2007}. No cases were reported for button battery ingestion, magnet ingestion, and involved large diameter (\textgreater 2.5cm) object ingestion.
\paragraph*{Outcomes}  47 cases (52\%) underwent endoscopic intervention \cite{Elghali_2016, Lee_2007}, 29 cases (32\%) were managed conservatively \cite{Elghali_2016, Karp_1991b}, 15 cases (17\%) underwent surgical intervention \cite{Elghali_2016, Karp_1991b, Lee_2007}, 6 cases (7\%) reported complications \cite{Lee_2007}, 1 case (1\%) died \cite{Elghali_2016}.
\paragraph*{Gender} 43 cases (60\%) were male \cite{Akay_2015f, Al-Faham_2020k, Alao_2006i, Ali_2017, Ali_2022g, Apikotoa_2022f, Atayan_2016, Benoist_2019e, Berry_2021e, Bhumi_2024f, CamachoDorado_2018, Csaky_1998e, Emamhadi_2018, Farhadi_2024h, Fry_2010, Gardner_2017h, Guinan_2019f, Jehangir_2019h, Jin_2023, Kobiela_2015, Kumar_2001, Kumar_2019f, Liu_2005, Losanoff_1996, Losanoff_1997e, Mesfin_2022a, Misra_2013, Qureshi_2016, Riva_2018j, Sobnach_2011f, Tammana_2012j, Tanrikulu_2015e, Tay_2004, Thapa_2019f, Trgo_2012f, Wadhwa_2015e, Yasin_2009, teWildt_2010}, 28 cases (39\%) were female \cite{AlShaaibi_2021b, Ali_2020f, Ataya_2013, Beecroft_1998, Bhasin_2014, Bhattacharjee_2008, Cauchi_2002, Chang_2017f, Cox_2007, DelgadoSalazar_2020c, DivsalarP._2023a, Goldman_1998f, Hardy_2023g, Kar_2015, Kariholu_2008, Kerestes_2019, Li_2013, Naji_2012f, Ohno_2005, Peixoto_2017f, Sakellaridis_2008f, Sultan_2024f, Tupesis_2004f, Wildhaber_2005, Wnęk_2015f, Yildiz_2016e}, 1 case (1\%) had no gender recorded \cite{fjbuilsRepeatedBehaviorDeliberate2024}. \paragraph*{Age Group} 25 cases (35\%) were between 26 and 40 years of age \cite{Alao_2006i, Ali_2022g, Apikotoa_2022f, Ataya_2013, Benoist_2019e, Bhasin_2014, Chang_2017f, Cox_2007, DelgadoSalazar_2020c, Farhadi_2024h, Fry_2010, Gardner_2017h, Guinan_2019f, Jin_2023, Kumar_2019f, Losanoff_1996, Misra_2013, Qureshi_2016, Riva_2018j, Sakellaridis_2008f, Tammana_2012j, Trgo_2012f, Wnęk_2015f, Yildiz_2016e, fjbuilsRepeatedBehaviorDeliberate2024}, 18 cases (25\%) were between 18 and 25 years of age \cite{Akay_2015f, Ali_2017, Atayan_2016, Bhattacharjee_2008, Csaky_1998e, Kar_2015, Kariholu_2008, Kobiela_2015, Losanoff_1996, Losanoff_1997e, Mesfin_2022a, Peixoto_2017f, Sobnach_2011f, Tupesis_2004f, Yasin_2009}, 13 cases (18\%) were under 18 years of age \cite{AlShaaibi_2021b, Ali_2020f, Cauchi_2002, DivsalarP._2023a, Goldman_1998f, Liu_2005, Naji_2012f, Ohno_2005, Tanrikulu_2015e, Tay_2004, Wildhaber_2005}, 11 cases (15\%) were between 41 and 60 years of age \cite{Al-Faham_2020k, Bhumi_2024f, CamachoDorado_2018, Emamhadi_2018, Hardy_2023g, Jehangir_2019h, Kumar_2001, Sultan_2024f, Thapa_2019f, Wadhwa_2015e, teWildt_2010}, 3 cases (4\%) were over 60 years of age \cite{Beecroft_1998, Kerestes_2019, Li_2013}, 2 cases (3\%) had no age documented \cite{Berry_2021e}. All 90 were male gender. 90 cases (100\%) were detained at the time of ingestion \cite{Elghali_2016, Karp_1991b, Lee_2007}, 88 cases (98\%) were intentional ingestions \cite{Elghali_2016, Karp_1991b, Lee_2007}, 30 cases (33\%) had a psychiatric history documented \cite{Elghali_2016, Karp_1991b, Lee_2007}, 2 cases (2\%) had a history of prior ingestion \cite{Elghali_2016}. No cases were reported for were psychiatric inpatients, were displaced people, were under the influence of alcohol at the time of ingestion, and had a severe disability history.
\paragraph*{Motivation}  70 cases (78\%) reported protest motivation \cite{Elghali_2016, Karp_1991b, Lee_2007}, 12 cases (13\%) reported psychiatric motivation \cite{Karp_1991b}, 6 cases (7\%) reported self-harm motivation \cite{Elghali_2016, Karp_1991b}. No cases were reported for psychosocial motivation and other motivation.
\paragraph*{Object Characteristics}  68 cases (76\%) involved sharp object ingestion \cite{Elghali_2016, Karp_1991b, Lee_2007}, 32 cases (36\%) involved long (\textgreater 5cm) object ingestion \cite{Lee_2007}, 25 cases (28\%) involved ingestion of multiple objects \cite{Elghali_2016, Lee_2007}. No cases were reported for button battery ingestion, magnet ingestion, and involved large diameter (\textgreater 2.5cm) object ingestion.
\paragraph*{Outcomes}  47 cases (52\%) underwent endoscopic intervention \cite{Elghali_2016, Lee_2007}, 29 cases (32\%) were managed conservatively \cite{Elghali_2016, Karp_1991b}, 15 cases (17\%) underwent surgical intervention \cite{Elghali_2016, Karp_1991b, Lee_2007}, 6 cases (7\%) reported complications \cite{Lee_2007}, 1 case (1\%) died \cite{Elghali_2016}.
\paragraph*{Gender} 43 cases (60\%) were male \cite{Akay_2015f, Al-Faham_2020k, Alao_2006i, Ali_2017, Ali_2022g, Apikotoa_2022f, Atayan_2016, Benoist_2019e, Berry_2021e, Bhumi_2024f, CamachoDorado_2018, Csaky_1998e, Emamhadi_2018, Farhadi_2024h, Fry_2010, Gardner_2017h, Guinan_2019f, Jehangir_2019h, Jin_2023, Kobiela_2015, Kumar_2001, Kumar_2019f, Liu_2005, Losanoff_1996, Losanoff_1997e, Mesfin_2022a, Misra_2013, Qureshi_2016, Riva_2018j, Sobnach_2011f, Tammana_2012j, Tanrikulu_2015e, Tay_2004, Thapa_2019f, Trgo_2012f, Wadhwa_2015e, Yasin_2009, teWildt_2010}, 28 cases (39\%) were female \cite{AlShaaibi_2021b, Ali_2020f, Ataya_2013, Beecroft_1998, Bhasin_2014, Bhattacharjee_2008, Cauchi_2002, Chang_2017f, Cox_2007, DelgadoSalazar_2020c, DivsalarP._2023a, Goldman_1998f, Hardy_2023g, Kar_2015, Kariholu_2008, Kerestes_2019, Li_2013, Naji_2012f, Ohno_2005, Peixoto_2017f, Sakellaridis_2008f, Sultan_2024f, Tupesis_2004f, Wildhaber_2005, Wnęk_2015f, Yildiz_2016e}, 1 case (1\%) had no gender recorded \cite{fjbuilsRepeatedBehaviorDeliberate2024}. \paragraph*{Age Group} 25 cases (35\%) were between 26 and 40 years of age \cite{Alao_2006i, Ali_2022g, Apikotoa_2022f, Ataya_2013, Benoist_2019e, Bhasin_2014, Chang_2017f, Cox_2007, DelgadoSalazar_2020c, Farhadi_2024h, Fry_2010, Gardner_2017h, Guinan_2019f, Jin_2023, Kumar_2019f, Losanoff_1996, Misra_2013, Qureshi_2016, Riva_2018j, Sakellaridis_2008f, Tammana_2012j, Trgo_2012f, Wnęk_2015f, Yildiz_2016e, fjbuilsRepeatedBehaviorDeliberate2024}, 18 cases (25\%) were between 18 and 25 years of age \cite{Akay_2015f, Ali_2017, Atayan_2016, Bhattacharjee_2008, Csaky_1998e, Kar_2015, Kariholu_2008, Kobiela_2015, Losanoff_1996, Losanoff_1997e, Mesfin_2022a, Peixoto_2017f, Sobnach_2011f, Tupesis_2004f, Yasin_2009}, 13 cases (18\%) were under 18 years of age \cite{AlShaaibi_2021b, Ali_2020f, Cauchi_2002, DivsalarP._2023a, Goldman_1998f, Liu_2005, Naji_2012f, Ohno_2005, Tanrikulu_2015e, Tay_2004, Wildhaber_2005}, 11 cases (15\%) were between 41 and 60 years of age \cite{Al-Faham_2020k, Bhumi_2024f, CamachoDorado_2018, Emamhadi_2018, Hardy_2023g, Jehangir_2019h, Kumar_2001, Sultan_2024f, Thapa_2019f, Wadhwa_2015e, teWildt_2010}, 3 cases (4\%) were over 60 years of age \cite{Beecroft_1998, Kerestes_2019, Li_2013}, 2 cases (3\%) had no age documented \cite{Berry_2021e}. All 90 were male gender. 90 cases (100\%) were detained at the time of ingestion \cite{Elghali_2016, Karp_1991b, Lee_2007}, 88 cases (98\%) were intentional ingestions \cite{Elghali_2016, Karp_1991b, Lee_2007}, 30 cases (33\%) had a psychiatric history documented \cite{Elghali_2016, Karp_1991b, Lee_2007}, 2 cases (2\%) had a history of prior ingestion \cite{Elghali_2016}. No cases were reported for were psychiatric inpatients, were displaced people, were under the influence of alcohol at the time of ingestion, and had a severe disability history.
\paragraph*{Motivation}  70 cases (78\%) reported protest motivation \cite{Elghali_2016, Karp_1991b, Lee_2007}, 12 cases (13\%) reported psychiatric motivation \cite{Karp_1991b}, 6 cases (7\%) reported self-harm motivation \cite{Elghali_2016, Karp_1991b}. No cases were reported for psychosocial motivation and other motivation.
\paragraph*{Object Characteristics}  68 cases (76\%) involved sharp object ingestion \cite{Elghali_2016, Karp_1991b, Lee_2007}, 32 cases (36\%) involved long (\textgreater 5cm) object ingestion \cite{Lee_2007}, 25 cases (28\%) involved ingestion of multiple objects \cite{Elghali_2016, Lee_2007}. No cases were reported for button battery ingestion, magnet ingestion, and involved large diameter (\textgreater 2.5cm) object ingestion.
\paragraph*{Outcomes}  47 cases (52\%) underwent endoscopic intervention \cite{Elghali_2016, Lee_2007}, 29 cases (32\%) were managed conservatively \cite{Elghali_2016, Karp_1991b}, 15 cases (17\%) underwent surgical intervention \cite{Elghali_2016, Karp_1991b, Lee_2007}, 6 cases (7\%) reported complications \cite{Lee_2007}, 1 case (1\%) died \cite{Elghali_2016}.
\paragraph*{Gender} 43 cases (60\%) were male \cite{Akay_2015f, Al-Faham_2020k, Alao_2006i, Ali_2017, Ali_2022g, Apikotoa_2022f, Atayan_2016, Benoist_2019e, Berry_2021e, Bhumi_2024f, CamachoDorado_2018, Csaky_1998e, Emamhadi_2018, Farhadi_2024h, Fry_2010, Gardner_2017h, Guinan_2019f, Jehangir_2019h, Jin_2023, Kobiela_2015, Kumar_2001, Kumar_2019f, Liu_2005, Losanoff_1996, Losanoff_1997e, Mesfin_2022a, Misra_2013, Qureshi_2016, Riva_2018j, Sobnach_2011f, Tammana_2012j, Tanrikulu_2015e, Tay_2004, Thapa_2019f, Trgo_2012f, Wadhwa_2015e, Yasin_2009, teWildt_2010}, 28 cases (39\%) were female \cite{AlShaaibi_2021b, Ali_2020f, Ataya_2013, Beecroft_1998, Bhasin_2014, Bhattacharjee_2008, Cauchi_2002, Chang_2017f, Cox_2007, DelgadoSalazar_2020c, DivsalarP._2023a, Goldman_1998f, Hardy_2023g, Kar_2015, Kariholu_2008, Kerestes_2019, Li_2013, Naji_2012f, Ohno_2005, Peixoto_2017f, Sakellaridis_2008f, Sultan_2024f, Tupesis_2004f, Wildhaber_2005, Wnęk_2015f, Yildiz_2016e}, 1 case (1\%) had no gender recorded \cite{fjbuilsRepeatedBehaviorDeliberate2024}. \paragraph*{Age Group} 25 cases (35\%) were between 26 and 40 years of age \cite{Alao_2006i, Ali_2022g, Apikotoa_2022f, Ataya_2013, Benoist_2019e, Bhasin_2014, Chang_2017f, Cox_2007, DelgadoSalazar_2020c, Farhadi_2024h, Fry_2010, Gardner_2017h, Guinan_2019f, Jin_2023, Kumar_2019f, Losanoff_1996, Misra_2013, Qureshi_2016, Riva_2018j, Sakellaridis_2008f, Tammana_2012j, Trgo_2012f, Wnęk_2015f, Yildiz_2016e, fjbuilsRepeatedBehaviorDeliberate2024}, 18 cases (25\%) were between 18 and 25 years of age \cite{Akay_2015f, Ali_2017, Atayan_2016, Bhattacharjee_2008, Csaky_1998e, Kar_2015, Kariholu_2008, Kobiela_2015, Losanoff_1996, Losanoff_1997e, Mesfin_2022a, Peixoto_2017f, Sobnach_2011f, Tupesis_2004f, Yasin_2009}, 13 cases (18\%) were under 18 years of age \cite{AlShaaibi_2021b, Ali_2020f, Cauchi_2002, DivsalarP._2023a, Goldman_1998f, Liu_2005, Naji_2012f, Ohno_2005, Tanrikulu_2015e, Tay_2004, Wildhaber_2005}, 11 cases (15\%) were between 41 and 60 years of age \cite{Al-Faham_2020k, Bhumi_2024f, CamachoDorado_2018, Emamhadi_2018, Hardy_2023g, Jehangir_2019h, Kumar_2001, Sultan_2024f, Thapa_2019f, Wadhwa_2015e, teWildt_2010}, 3 cases (4\%) were over 60 years of age \cite{Beecroft_1998, Kerestes_2019, Li_2013}, 2 cases (3\%) had no age documented \cite{Berry_2021e}. All 90 were male gender. 90 cases (100\%) were detained at the time of ingestion \cite{Elghali_2016, Karp_1991b, Lee_2007}, 88 cases (98\%) were intentional ingestions \cite{Elghali_2016, Karp_1991b, Lee_2007}, 30 cases (33\%) had a psychiatric history documented \cite{Elghali_2016, Karp_1991b, Lee_2007}, 2 cases (2\%) had a history of prior ingestion \cite{Elghali_2016}. No cases were reported for were psychiatric inpatients, were displaced people, were under the influence of alcohol at the time of ingestion, and had a severe disability history.
\paragraph*{Motivation}  70 cases (78\%) reported protest motivation \cite{Elghali_2016, Karp_1991b, Lee_2007}, 12 cases (13\%) reported psychiatric motivation \cite{Karp_1991b}, 6 cases (7\%) reported self-harm motivation \cite{Elghali_2016, Karp_1991b}. No cases were reported for psychosocial motivation and other motivation.
\paragraph*{Object Characteristics}  68 cases (76\%) involved sharp object ingestion \cite{Elghali_2016, Karp_1991b, Lee_2007}, 32 cases (36\%) involved long (\textgreater 5cm) object ingestion \cite{Lee_2007}, 25 cases (28\%) involved ingestion of multiple objects \cite{Elghali_2016, Lee_2007}. No cases were reported for button battery ingestion, magnet ingestion, and involved large diameter (\textgreater 2.5cm) object ingestion.
\paragraph*{Outcomes}  47 cases (52\%) underwent endoscopic intervention \cite{Elghali_2016, Lee_2007}, 29 cases (32\%) were managed conservatively \cite{Elghali_2016, Karp_1991b}, 15 cases (17\%) underwent surgical intervention \cite{Elghali_2016, Karp_1991b, Lee_2007}, 6 cases (7\%) reported complications \cite{Lee_2007}, 1 case (1\%) died \cite{Elghali_2016}.
\paragraph*{Gender} 43 cases (60\%) were male \cite{Akay_2015f, Al-Faham_2020k, Alao_2006i, Ali_2017, Ali_2022g, Apikotoa_2022f, Atayan_2016, Benoist_2019e, Berry_2021e, Bhumi_2024f, CamachoDorado_2018, Csaky_1998e, Emamhadi_2018, Farhadi_2024h, Fry_2010, Gardner_2017h, Guinan_2019f, Jehangir_2019h, Jin_2023, Kobiela_2015, Kumar_2001, Kumar_2019f, Liu_2005, Losanoff_1996, Losanoff_1997e, Mesfin_2022a, Misra_2013, Qureshi_2016, Riva_2018j, Sobnach_2011f, Tammana_2012j, Tanrikulu_2015e, Tay_2004, Thapa_2019f, Trgo_2012f, Wadhwa_2015e, Yasin_2009, teWildt_2010}, 28 cases (39\%) were female \cite{AlShaaibi_2021b, Ali_2020f, Ataya_2013, Beecroft_1998, Bhasin_2014, Bhattacharjee_2008, Cauchi_2002, Chang_2017f, Cox_2007, DelgadoSalazar_2020c, DivsalarP._2023a, Goldman_1998f, Hardy_2023g, Kar_2015, Kariholu_2008, Kerestes_2019, Li_2013, Naji_2012f, Ohno_2005, Peixoto_2017f, Sakellaridis_2008f, Sultan_2024f, Tupesis_2004f, Wildhaber_2005, Wnęk_2015f, Yildiz_2016e}, 1 case (1\%) had no gender recorded \cite{fjbuilsRepeatedBehaviorDeliberate2024}. \paragraph*{Age Group} 25 cases (35\%) were between 26 and 40 years of age \cite{Alao_2006i, Ali_2022g, Apikotoa_2022f, Ataya_2013, Benoist_2019e, Bhasin_2014, Chang_2017f, Cox_2007, DelgadoSalazar_2020c, Farhadi_2024h, Fry_2010, Gardner_2017h, Guinan_2019f, Jin_2023, Kumar_2019f, Losanoff_1996, Misra_2013, Qureshi_2016, Riva_2018j, Sakellaridis_2008f, Tammana_2012j, Trgo_2012f, Wnęk_2015f, Yildiz_2016e, fjbuilsRepeatedBehaviorDeliberate2024}, 18 cases (25\%) were between 18 and 25 years of age \cite{Akay_2015f, Ali_2017, Atayan_2016, Bhattacharjee_2008, Csaky_1998e, Kar_2015, Kariholu_2008, Kobiela_2015, Losanoff_1996, Losanoff_1997e, Mesfin_2022a, Peixoto_2017f, Sobnach_2011f, Tupesis_2004f, Yasin_2009}, 13 cases (18\%) were under 18 years of age \cite{AlShaaibi_2021b, Ali_2020f, Cauchi_2002, DivsalarP._2023a, Goldman_1998f, Liu_2005, Naji_2012f, Ohno_2005, Tanrikulu_2015e, Tay_2004, Wildhaber_2005}, 11 cases (15\%) were between 41 and 60 years of age \cite{Al-Faham_2020k, Bhumi_2024f, CamachoDorado_2018, Emamhadi_2018, Hardy_2023g, Jehangir_2019h, Kumar_2001, Sultan_2024f, Thapa_2019f, Wadhwa_2015e, teWildt_2010}, 3 cases (4\%) were over 60 years of age \cite{Beecroft_1998, Kerestes_2019, Li_2013}, 2 cases (3\%) had no age documented \cite{Berry_2021e}. All 90 were male gender. 90 cases (100\%) were detained at the time of ingestion \cite{Elghali_2016, Karp_1991b, Lee_2007}, 88 cases (98\%) were intentional ingestions \cite{Elghali_2016, Karp_1991b, Lee_2007}, 30 cases (33\%) had a psychiatric history documented \cite{Elghali_2016, Karp_1991b, Lee_2007}, 2 cases (2\%) had a history of prior ingestion \cite{Elghali_2016}. No cases were reported for were psychiatric inpatients, were displaced people, were under the influence of alcohol at the time of ingestion, and had a severe disability history.
\paragraph*{Motivation}  70 cases (78\%) reported protest motivation \cite{Elghali_2016, Karp_1991b, Lee_2007}, 12 cases (13\%) reported psychiatric motivation \cite{Karp_1991b}, 6 cases (7\%) reported self-harm motivation \cite{Elghali_2016, Karp_1991b}. No cases were reported for psychosocial motivation and other motivation.
\paragraph*{Object Characteristics}  68 cases (76\%) involved sharp object ingestion \cite{Elghali_2016, Karp_1991b, Lee_2007}, 32 cases (36\%) involved long (\textgreater 5cm) object ingestion \cite{Lee_2007}, 25 cases (28\%) involved ingestion of multiple objects \cite{Elghali_2016, Lee_2007}. No cases were reported for button battery ingestion, magnet ingestion, and involved large diameter (\textgreater 2.5cm) object ingestion.
\paragraph*{Outcomes}  47 cases (52\%) underwent endoscopic intervention \cite{Elghali_2016, Lee_2007}, 29 cases (32\%) were managed conservatively \cite{Elghali_2016, Karp_1991b}, 15 cases (17\%) underwent surgical intervention \cite{Elghali_2016, Karp_1991b, Lee_2007}, 6 cases (7\%) reported complications \cite{Lee_2007}, 1 case (1\%) died \cite{Elghali_2016}.
\paragraph*{Gender} 43 cases (60\%) were male \cite{Akay_2015f, Al-Faham_2020k, Alao_2006i, Ali_2017, Ali_2022g, Apikotoa_2022f, Atayan_2016, Benoist_2019e, Berry_2021e, Bhumi_2024f, CamachoDorado_2018, Csaky_1998e, Emamhadi_2018, Farhadi_2024h, Fry_2010, Gardner_2017h, Guinan_2019f, Jehangir_2019h, Jin_2023, Kobiela_2015, Kumar_2001, Kumar_2019f, Liu_2005, Losanoff_1996, Losanoff_1997e, Mesfin_2022a, Misra_2013, Qureshi_2016, Riva_2018j, Sobnach_2011f, Tammana_2012j, Tanrikulu_2015e, Tay_2004, Thapa_2019f, Trgo_2012f, Wadhwa_2015e, Yasin_2009, teWildt_2010}, 28 cases (39\%) were female \cite{AlShaaibi_2021b, Ali_2020f, Ataya_2013, Beecroft_1998, Bhasin_2014, Bhattacharjee_2008, Cauchi_2002, Chang_2017f, Cox_2007, DelgadoSalazar_2020c, DivsalarP._2023a, Goldman_1998f, Hardy_2023g, Kar_2015, Kariholu_2008, Kerestes_2019, Li_2013, Naji_2012f, Ohno_2005, Peixoto_2017f, Sakellaridis_2008f, Sultan_2024f, Tupesis_2004f, Wildhaber_2005, Wnęk_2015f, Yildiz_2016e}, 1 case (1\%) had no gender recorded \cite{fjbuilsRepeatedBehaviorDeliberate2024}. \paragraph*{Age Group} 25 cases (35\%) were between 26 and 40 years of age \cite{Alao_2006i, Ali_2022g, Apikotoa_2022f, Ataya_2013, Benoist_2019e, Bhasin_2014, Chang_2017f, Cox_2007, DelgadoSalazar_2020c, Farhadi_2024h, Fry_2010, Gardner_2017h, Guinan_2019f, Jin_2023, Kumar_2019f, Losanoff_1996, Misra_2013, Qureshi_2016, Riva_2018j, Sakellaridis_2008f, Tammana_2012j, Trgo_2012f, Wnęk_2015f, Yildiz_2016e, fjbuilsRepeatedBehaviorDeliberate2024}, 18 cases (25\%) were between 18 and 25 years of age \cite{Akay_2015f, Ali_2017, Atayan_2016, Bhattacharjee_2008, Csaky_1998e, Kar_2015, Kariholu_2008, Kobiela_2015, Losanoff_1996, Losanoff_1997e, Mesfin_2022a, Peixoto_2017f, Sobnach_2011f, Tupesis_2004f, Yasin_2009}, 13 cases (18\%) were under 18 years of age \cite{AlShaaibi_2021b, Ali_2020f, Cauchi_2002, DivsalarP._2023a, Goldman_1998f, Liu_2005, Naji_2012f, Ohno_2005, Tanrikulu_2015e, Tay_2004, Wildhaber_2005}, 11 cases (15\%) were between 41 and 60 years of age \cite{Al-Faham_2020k, Bhumi_2024f, CamachoDorado_2018, Emamhadi_2018, Hardy_2023g, Jehangir_2019h, Kumar_2001, Sultan_2024f, Thapa_2019f, Wadhwa_2015e, teWildt_2010}, 3 cases (4\%) were over 60 years of age \cite{Beecroft_1998, Kerestes_2019, Li_2013}, 2 cases (3\%) had no age documented \cite{Berry_2021e}. All 90 were male gender. 90 cases (100\%) were detained at the time of ingestion \cite{Elghali_2016, Karp_1991b, Lee_2007}, 88 cases (98\%) were intentional ingestions \cite{Elghali_2016, Karp_1991b, Lee_2007}, 30 cases (33\%) had a psychiatric history documented \cite{Elghali_2016, Karp_1991b, Lee_2007}, 2 cases (2\%) had a history of prior ingestion \cite{Elghali_2016}. No cases were reported for were psychiatric inpatients, were displaced people, were under the influence of alcohol at the time of ingestion, and had a severe disability history.
\paragraph*{Motivation}  70 cases (78\%) reported protest motivation \cite{Elghali_2016, Karp_1991b, Lee_2007}, 12 cases (13\%) reported psychiatric motivation \cite{Karp_1991b}, 6 cases (7\%) reported self-harm motivation \cite{Elghali_2016, Karp_1991b}. No cases were reported for psychosocial motivation and other motivation.
\paragraph*{Object Characteristics}  68 cases (76\%) involved sharp object ingestion \cite{Elghali_2016, Karp_1991b, Lee_2007}, 32 cases (36\%) involved long (\textgreater 5cm) object ingestion \cite{Lee_2007}, 25 cases (28\%) involved ingestion of multiple objects \cite{Elghali_2016, Lee_2007}. No cases were reported for button battery ingestion, magnet ingestion, and involved large diameter (\textgreater 2.5cm) object ingestion.
\paragraph*{Outcomes}  47 cases (52\%) underwent endoscopic intervention \cite{Elghali_2016, Lee_2007}, 29 cases (32\%) were managed conservatively \cite{Elghali_2016, Karp_1991b}, 15 cases (17\%) underwent surgical intervention \cite{Elghali_2016, Karp_1991b, Lee_2007}, 6 cases (7\%) reported complications \cite{Lee_2007}, 1 case (1\%) died \cite{Elghali_2016}.
\paragraph*{Gender} 43 cases (60\%) were male \cite{Akay_2015f, Al-Faham_2020k, Alao_2006i, Ali_2017, Ali_2022g, Apikotoa_2022f, Atayan_2016, Benoist_2019e, Berry_2021e, Bhumi_2024f, CamachoDorado_2018, Csaky_1998e, Emamhadi_2018, Farhadi_2024h, Fry_2010, Gardner_2017h, Guinan_2019f, Jehangir_2019h, Jin_2023, Kobiela_2015, Kumar_2001, Kumar_2019f, Liu_2005, Losanoff_1996, Losanoff_1997e, Mesfin_2022a, Misra_2013, Qureshi_2016, Riva_2018j, Sobnach_2011f, Tammana_2012j, Tanrikulu_2015e, Tay_2004, Thapa_2019f, Trgo_2012f, Wadhwa_2015e, Yasin_2009, teWildt_2010}, 28 cases (39\%) were female \cite{AlShaaibi_2021b, Ali_2020f, Ataya_2013, Beecroft_1998, Bhasin_2014, Bhattacharjee_2008, Cauchi_2002, Chang_2017f, Cox_2007, DelgadoSalazar_2020c, DivsalarP._2023a, Goldman_1998f, Hardy_2023g, Kar_2015, Kariholu_2008, Kerestes_2019, Li_2013, Naji_2012f, Ohno_2005, Peixoto_2017f, Sakellaridis_2008f, Sultan_2024f, Tupesis_2004f, Wildhaber_2005, Wnęk_2015f, Yildiz_2016e}, 1 case (1\%) had no gender recorded \cite{fjbuilsRepeatedBehaviorDeliberate2024}. \paragraph*{Age Group} 25 cases (35\%) were between 26 and 40 years of age \cite{Alao_2006i, Ali_2022g, Apikotoa_2022f, Ataya_2013, Benoist_2019e, Bhasin_2014, Chang_2017f, Cox_2007, DelgadoSalazar_2020c, Farhadi_2024h, Fry_2010, Gardner_2017h, Guinan_2019f, Jin_2023, Kumar_2019f, Losanoff_1996, Misra_2013, Qureshi_2016, Riva_2018j, Sakellaridis_2008f, Tammana_2012j, Trgo_2012f, Wnęk_2015f, Yildiz_2016e, fjbuilsRepeatedBehaviorDeliberate2024}, 18 cases (25\%) were between 18 and 25 years of age \cite{Akay_2015f, Ali_2017, Atayan_2016, Bhattacharjee_2008, Csaky_1998e, Kar_2015, Kariholu_2008, Kobiela_2015, Losanoff_1996, Losanoff_1997e, Mesfin_2022a, Peixoto_2017f, Sobnach_2011f, Tupesis_2004f, Yasin_2009}, 13 cases (18\%) were under 18 years of age \cite{AlShaaibi_2021b, Ali_2020f, Cauchi_2002, DivsalarP._2023a, Goldman_1998f, Liu_2005, Naji_2012f, Ohno_2005, Tanrikulu_2015e, Tay_2004, Wildhaber_2005}, 11 cases (15\%) were between 41 and 60 years of age \cite{Al-Faham_2020k, Bhumi_2024f, CamachoDorado_2018, Emamhadi_2018, Hardy_2023g, Jehangir_2019h, Kumar_2001, Sultan_2024f, Thapa_2019f, Wadhwa_2015e, teWildt_2010}, 3 cases (4\%) were over 60 years of age \cite{Beecroft_1998, Kerestes_2019, Li_2013}, 2 cases (3\%) had no age documented \cite{Berry_2021e}. All 90 were male gender. 90 cases (100\%) were detained at the time of ingestion \cite{Elghali_2016, Karp_1991b, Lee_2007}, 88 cases (98\%) were intentional ingestions \cite{Elghali_2016, Karp_1991b, Lee_2007}, 30 cases (33\%) had a psychiatric history documented \cite{Elghali_2016, Karp_1991b, Lee_2007}, 2 cases (2\%) had a history of prior ingestion \cite{Elghali_2016}. No cases were reported for were psychiatric inpatients, were displaced people, were under the influence of alcohol at the time of ingestion, and had a severe disability history.
\paragraph*{Motivation}  70 cases (78\%) reported protest motivation \cite{Elghali_2016, Karp_1991b, Lee_2007}, 12 cases (13\%) reported psychiatric motivation \cite{Karp_1991b}, 6 cases (7\%) reported self-harm motivation \cite{Elghali_2016, Karp_1991b}. No cases were reported for psychosocial motivation and other motivation.
\paragraph*{Object Characteristics}  68 cases (76\%) involved sharp object ingestion \cite{Elghali_2016, Karp_1991b, Lee_2007}, 32 cases (36\%) involved long (\textgreater 5cm) object ingestion \cite{Lee_2007}, 25 cases (28\%) involved ingestion of multiple objects \cite{Elghali_2016, Lee_2007}. No cases were reported for button battery ingestion, magnet ingestion, and involved large diameter (\textgreater 2.5cm) object ingestion.
\paragraph*{Outcomes}  47 cases (52\%) underwent endoscopic intervention \cite{Elghali_2016, Lee_2007}, 29 cases (32\%) were managed conservatively \cite{Elghali_2016, Karp_1991b}, 15 cases (17\%) underwent surgical intervention \cite{Elghali_2016, Karp_1991b, Lee_2007}, 6 cases (7\%) reported complications \cite{Lee_2007}, 1 case (1\%) died \cite{Elghali_2016}.
\paragraph*{Gender} 43 cases (60\%) were male \cite{Akay_2015f, Al-Faham_2020k, Alao_2006i, Ali_2017, Ali_2022g, Apikotoa_2022f, Atayan_2016, Benoist_2019e, Berry_2021e, Bhumi_2024f, CamachoDorado_2018, Csaky_1998e, Emamhadi_2018, Farhadi_2024h, Fry_2010, Gardner_2017h, Guinan_2019f, Jehangir_2019h, Jin_2023, Kobiela_2015, Kumar_2001, Kumar_2019f, Liu_2005, Losanoff_1996, Losanoff_1997e, Mesfin_2022a, Misra_2013, Qureshi_2016, Riva_2018j, Sobnach_2011f, Tammana_2012j, Tanrikulu_2015e, Tay_2004, Thapa_2019f, Trgo_2012f, Wadhwa_2015e, Yasin_2009, teWildt_2010}, 28 cases (39\%) were female \cite{AlShaaibi_2021b, Ali_2020f, Ataya_2013, Beecroft_1998, Bhasin_2014, Bhattacharjee_2008, Cauchi_2002, Chang_2017f, Cox_2007, DelgadoSalazar_2020c, DivsalarP._2023a, Goldman_1998f, Hardy_2023g, Kar_2015, Kariholu_2008, Kerestes_2019, Li_2013, Naji_2012f, Ohno_2005, Peixoto_2017f, Sakellaridis_2008f, Sultan_2024f, Tupesis_2004f, Wildhaber_2005, Wnęk_2015f, Yildiz_2016e}, 1 case (1\%) had no gender recorded \cite{fjbuilsRepeatedBehaviorDeliberate2024}. \paragraph*{Age Group} 25 cases (35\%) were between 26 and 40 years of age \cite{Alao_2006i, Ali_2022g, Apikotoa_2022f, Ataya_2013, Benoist_2019e, Bhasin_2014, Chang_2017f, Cox_2007, DelgadoSalazar_2020c, Farhadi_2024h, Fry_2010, Gardner_2017h, Guinan_2019f, Jin_2023, Kumar_2019f, Losanoff_1996, Misra_2013, Qureshi_2016, Riva_2018j, Sakellaridis_2008f, Tammana_2012j, Trgo_2012f, Wnęk_2015f, Yildiz_2016e, fjbuilsRepeatedBehaviorDeliberate2024}, 18 cases (25\%) were between 18 and 25 years of age \cite{Akay_2015f, Ali_2017, Atayan_2016, Bhattacharjee_2008, Csaky_1998e, Kar_2015, Kariholu_2008, Kobiela_2015, Losanoff_1996, Losanoff_1997e, Mesfin_2022a, Peixoto_2017f, Sobnach_2011f, Tupesis_2004f, Yasin_2009}, 13 cases (18\%) were under 18 years of age \cite{AlShaaibi_2021b, Ali_2020f, Cauchi_2002, DivsalarP._2023a, Goldman_1998f, Liu_2005, Naji_2012f, Ohno_2005, Tanrikulu_2015e, Tay_2004, Wildhaber_2005}, 11 cases (15\%) were between 41 and 60 years of age \cite{Al-Faham_2020k, Bhumi_2024f, CamachoDorado_2018, Emamhadi_2018, Hardy_2023g, Jehangir_2019h, Kumar_2001, Sultan_2024f, Thapa_2019f, Wadhwa_2015e, teWildt_2010}, 3 cases (4\%) were over 60 years of age \cite{Beecroft_1998, Kerestes_2019, Li_2013}, 2 cases (3\%) had no age documented \cite{Berry_2021e}. All 90 were male gender. 90 cases (100\%) were detained at the time of ingestion \cite{Elghali_2016, Karp_1991b, Lee_2007}, 88 cases (98\%) were intentional ingestions \cite{Elghali_2016, Karp_1991b, Lee_2007}, 30 cases (33\%) had a psychiatric history documented \cite{Elghali_2016, Karp_1991b, Lee_2007}, 2 cases (2\%) had a history of prior ingestion \cite{Elghali_2016}. No cases were reported for were psychiatric inpatients, were displaced people, were under the influence of alcohol at the time of ingestion, and had a severe disability history.
\paragraph*{Motivation}  70 cases (78\%) reported protest motivation \cite{Elghali_2016, Karp_1991b, Lee_2007}, 12 cases (13\%) reported psychiatric motivation \cite{Karp_1991b}, 6 cases (7\%) reported self-harm motivation \cite{Elghali_2016, Karp_1991b}. No cases were reported for psychosocial motivation and other motivation.
\paragraph*{Object Characteristics}  68 cases (76\%) involved sharp object ingestion \cite{Elghali_2016, Karp_1991b, Lee_2007}, 32 cases (36\%) involved long (\textgreater 5cm) object ingestion \cite{Lee_2007}, 25 cases (28\%) involved ingestion of multiple objects \cite{Elghali_2016, Lee_2007}. No cases were reported for button battery ingestion, magnet ingestion, and involved large diameter (\textgreater 2.5cm) object ingestion.
\paragraph*{Outcomes}  47 cases (52\%) underwent endoscopic intervention \cite{Elghali_2016, Lee_2007}, 29 cases (32\%) were managed conservatively \cite{Elghali_2016, Karp_1991b}, 15 cases (17\%) underwent surgical intervention \cite{Elghali_2016, Karp_1991b, Lee_2007}, 6 cases (7\%) reported complications \cite{Lee_2007}, 1 case (1\%) died \cite{Elghali_2016}.
\paragraph*{Gender} 43 cases (60\%) were male \cite{Akay_2015f, Al-Faham_2020k, Alao_2006i, Ali_2017, Ali_2022g, Apikotoa_2022f, Atayan_2016, Benoist_2019e, Berry_2021e, Bhumi_2024f, CamachoDorado_2018, Csaky_1998e, Emamhadi_2018, Farhadi_2024h, Fry_2010, Gardner_2017h, Guinan_2019f, Jehangir_2019h, Jin_2023, Kobiela_2015, Kumar_2001, Kumar_2019f, Liu_2005, Losanoff_1996, Losanoff_1997e, Mesfin_2022a, Misra_2013, Qureshi_2016, Riva_2018j, Sobnach_2011f, Tammana_2012j, Tanrikulu_2015e, Tay_2004, Thapa_2019f, Trgo_2012f, Wadhwa_2015e, Yasin_2009, teWildt_2010}, 28 cases (39\%) were female \cite{AlShaaibi_2021b, Ali_2020f, Ataya_2013, Beecroft_1998, Bhasin_2014, Bhattacharjee_2008, Cauchi_2002, Chang_2017f, Cox_2007, DelgadoSalazar_2020c, DivsalarP._2023a, Goldman_1998f, Hardy_2023g, Kar_2015, Kariholu_2008, Kerestes_2019, Li_2013, Naji_2012f, Ohno_2005, Peixoto_2017f, Sakellaridis_2008f, Sultan_2024f, Tupesis_2004f, Wildhaber_2005, Wnęk_2015f, Yildiz_2016e}, 1 case (1\%) had no gender recorded \cite{fjbuilsRepeatedBehaviorDeliberate2024}. \paragraph*{Age Group} 25 cases (35\%) were between 26 and 40 years of age \cite{Alao_2006i, Ali_2022g, Apikotoa_2022f, Ataya_2013, Benoist_2019e, Bhasin_2014, Chang_2017f, Cox_2007, DelgadoSalazar_2020c, Farhadi_2024h, Fry_2010, Gardner_2017h, Guinan_2019f, Jin_2023, Kumar_2019f, Losanoff_1996, Misra_2013, Qureshi_2016, Riva_2018j, Sakellaridis_2008f, Tammana_2012j, Trgo_2012f, Wnęk_2015f, Yildiz_2016e, fjbuilsRepeatedBehaviorDeliberate2024}, 18 cases (25\%) were between 18 and 25 years of age \cite{Akay_2015f, Ali_2017, Atayan_2016, Bhattacharjee_2008, Csaky_1998e, Kar_2015, Kariholu_2008, Kobiela_2015, Losanoff_1996, Losanoff_1997e, Mesfin_2022a, Peixoto_2017f, Sobnach_2011f, Tupesis_2004f, Yasin_2009}, 13 cases (18\%) were under 18 years of age \cite{AlShaaibi_2021b, Ali_2020f, Cauchi_2002, DivsalarP._2023a, Goldman_1998f, Liu_2005, Naji_2012f, Ohno_2005, Tanrikulu_2015e, Tay_2004, Wildhaber_2005}, 11 cases (15\%) were between 41 and 60 years of age \cite{Al-Faham_2020k, Bhumi_2024f, CamachoDorado_2018, Emamhadi_2018, Hardy_2023g, Jehangir_2019h, Kumar_2001, Sultan_2024f, Thapa_2019f, Wadhwa_2015e, teWildt_2010}, 3 cases (4\%) were over 60 years of age \cite{Beecroft_1998, Kerestes_2019, Li_2013}, 2 cases (3\%) had no age documented \cite{Berry_2021e}. All 90 were male gender. 90 cases (100\%) were detained at the time of ingestion \cite{Elghali_2016, Karp_1991b, Lee_2007}, 88 cases (98\%) were intentional ingestions \cite{Elghali_2016, Karp_1991b, Lee_2007}, 30 cases (33\%) had a psychiatric history documented \cite{Elghali_2016, Karp_1991b, Lee_2007}, 2 cases (2\%) had a history of prior ingestion \cite{Elghali_2016}. No cases were reported for were psychiatric inpatients, were displaced people, were under the influence of alcohol at the time of ingestion, and had a severe disability history.
\paragraph*{Motivation}  70 cases (78\%) reported protest motivation \cite{Elghali_2016, Karp_1991b, Lee_2007}, 12 cases (13\%) reported psychiatric motivation \cite{Karp_1991b}, 6 cases (7\%) reported self-harm motivation \cite{Elghali_2016, Karp_1991b}. No cases were reported for psychosocial motivation and other motivation.
\paragraph*{Object Characteristics}  68 cases (76\%) involved sharp object ingestion \cite{Elghali_2016, Karp_1991b, Lee_2007}, 32 cases (36\%) involved long (\textgreater 5cm) object ingestion \cite{Lee_2007}, 25 cases (28\%) involved ingestion of multiple objects \cite{Elghali_2016, Lee_2007}. No cases were reported for button battery ingestion, magnet ingestion, and involved large diameter (\textgreater 2.5cm) object ingestion.
\paragraph*{Outcomes}  47 cases (52\%) underwent endoscopic intervention \cite{Elghali_2016, Lee_2007}, 29 cases (32\%) were managed conservatively \cite{Elghali_2016, Karp_1991b}, 15 cases (17\%) underwent surgical intervention \cite{Elghali_2016, Karp_1991b, Lee_2007}, 6 cases (7\%) reported complications \cite{Lee_2007}, 1 case (1\%) died \cite{Elghali_2016}.
\paragraph*{Gender} 43 cases (60\%) were male \cite{Akay_2015f, Al-Faham_2020k, Alao_2006i, Ali_2017, Ali_2022g, Apikotoa_2022f, Atayan_2016, Benoist_2019e, Berry_2021e, Bhumi_2024f, CamachoDorado_2018, Csaky_1998e, Emamhadi_2018, Farhadi_2024h, Fry_2010, Gardner_2017h, Guinan_2019f, Jehangir_2019h, Jin_2023, Kobiela_2015, Kumar_2001, Kumar_2019f, Liu_2005, Losanoff_1996, Losanoff_1997e, Mesfin_2022a, Misra_2013, Qureshi_2016, Riva_2018j, Sobnach_2011f, Tammana_2012j, Tanrikulu_2015e, Tay_2004, Thapa_2019f, Trgo_2012f, Wadhwa_2015e, Yasin_2009, teWildt_2010}, 28 cases (39\%) were female \cite{AlShaaibi_2021b, Ali_2020f, Ataya_2013, Beecroft_1998, Bhasin_2014, Bhattacharjee_2008, Cauchi_2002, Chang_2017f, Cox_2007, DelgadoSalazar_2020c, DivsalarP._2023a, Goldman_1998f, Hardy_2023g, Kar_2015, Kariholu_2008, Kerestes_2019, Li_2013, Naji_2012f, Ohno_2005, Peixoto_2017f, Sakellaridis_2008f, Sultan_2024f, Tupesis_2004f, Wildhaber_2005, Wnęk_2015f, Yildiz_2016e}, 1 case (1\%) had no gender recorded \cite{fjbuilsRepeatedBehaviorDeliberate2024}. \paragraph*{Age Group} 25 cases (35\%) were between 26 and 40 years of age \cite{Alao_2006i, Ali_2022g, Apikotoa_2022f, Ataya_2013, Benoist_2019e, Bhasin_2014, Chang_2017f, Cox_2007, DelgadoSalazar_2020c, Farhadi_2024h, Fry_2010, Gardner_2017h, Guinan_2019f, Jin_2023, Kumar_2019f, Losanoff_1996, Misra_2013, Qureshi_2016, Riva_2018j, Sakellaridis_2008f, Tammana_2012j, Trgo_2012f, Wnęk_2015f, Yildiz_2016e, fjbuilsRepeatedBehaviorDeliberate2024}, 18 cases (25\%) were between 18 and 25 years of age \cite{Akay_2015f, Ali_2017, Atayan_2016, Bhattacharjee_2008, Csaky_1998e, Kar_2015, Kariholu_2008, Kobiela_2015, Losanoff_1996, Losanoff_1997e, Mesfin_2022a, Peixoto_2017f, Sobnach_2011f, Tupesis_2004f, Yasin_2009}, 13 cases (18\%) were under 18 years of age \cite{AlShaaibi_2021b, Ali_2020f, Cauchi_2002, DivsalarP._2023a, Goldman_1998f, Liu_2005, Naji_2012f, Ohno_2005, Tanrikulu_2015e, Tay_2004, Wildhaber_2005}, 11 cases (15\%) were between 41 and 60 years of age \cite{Al-Faham_2020k, Bhumi_2024f, CamachoDorado_2018, Emamhadi_2018, Hardy_2023g, Jehangir_2019h, Kumar_2001, Sultan_2024f, Thapa_2019f, Wadhwa_2015e, teWildt_2010}, 3 cases (4\%) were over 60 years of age \cite{Beecroft_1998, Kerestes_2019, Li_2013}, 2 cases (3\%) had no age documented \cite{Berry_2021e}. All 90 were male gender. 90 cases (100\%) were detained at the time of ingestion \cite{Elghali_2016, Karp_1991b, Lee_2007}, 88 cases (98\%) were intentional ingestions \cite{Elghali_2016, Karp_1991b, Lee_2007}, 30 cases (33\%) had a psychiatric history documented \cite{Elghali_2016, Karp_1991b, Lee_2007}, 2 cases (2\%) had a history of prior ingestion \cite{Elghali_2016}. No cases were reported for were psychiatric inpatients, were displaced people, were under the influence of alcohol at the time of ingestion, and had a severe disability history.
\paragraph*{Motivation}  70 cases (78\%) reported protest motivation \cite{Elghali_2016, Karp_1991b, Lee_2007}, 12 cases (13\%) reported psychiatric motivation \cite{Karp_1991b}, 6 cases (7\%) reported self-harm motivation \cite{Elghali_2016, Karp_1991b}. No cases were reported for psychosocial motivation and other motivation.
\paragraph*{Object Characteristics}  68 cases (76\%) involved sharp object ingestion \cite{Elghali_2016, Karp_1991b, Lee_2007}, 32 cases (36\%) involved long (\textgreater 5cm) object ingestion \cite{Lee_2007}, 25 cases (28\%) involved ingestion of multiple objects \cite{Elghali_2016, Lee_2007}. No cases were reported for button battery ingestion, magnet ingestion, and involved large diameter (\textgreater 2.5cm) object ingestion.
\paragraph*{Outcomes}  47 cases (52\%) underwent endoscopic intervention \cite{Elghali_2016, Lee_2007}, 29 cases (32\%) were managed conservatively \cite{Elghali_2016, Karp_1991b}, 15 cases (17\%) underwent surgical intervention \cite{Elghali_2016, Karp_1991b, Lee_2007}, 6 cases (7\%) reported complications \cite{Lee_2007}, 1 case (1\%) died \cite{Elghali_2016}.
\paragraph*{Gender} 43 cases (60\%) were male \cite{Akay_2015f, Al-Faham_2020k, Alao_2006i, Ali_2017, Ali_2022g, Apikotoa_2022f, Atayan_2016, Benoist_2019e, Berry_2021e, Bhumi_2024f, CamachoDorado_2018, Csaky_1998e, Emamhadi_2018, Farhadi_2024h, Fry_2010, Gardner_2017h, Guinan_2019f, Jehangir_2019h, Jin_2023, Kobiela_2015, Kumar_2001, Kumar_2019f, Liu_2005, Losanoff_1996, Losanoff_1997e, Mesfin_2022a, Misra_2013, Qureshi_2016, Riva_2018j, Sobnach_2011f, Tammana_2012j, Tanrikulu_2015e, Tay_2004, Thapa_2019f, Trgo_2012f, Wadhwa_2015e, Yasin_2009, teWildt_2010}, 28 cases (39\%) were female \cite{AlShaaibi_2021b, Ali_2020f, Ataya_2013, Beecroft_1998, Bhasin_2014, Bhattacharjee_2008, Cauchi_2002, Chang_2017f, Cox_2007, DelgadoSalazar_2020c, DivsalarP._2023a, Goldman_1998f, Hardy_2023g, Kar_2015, Kariholu_2008, Kerestes_2019, Li_2013, Naji_2012f, Ohno_2005, Peixoto_2017f, Sakellaridis_2008f, Sultan_2024f, Tupesis_2004f, Wildhaber_2005, Wnęk_2015f, Yildiz_2016e}, 1 case (1\%) had no gender recorded \cite{fjbuilsRepeatedBehaviorDeliberate2024}. \paragraph*{Age Group} 25 cases (35\%) were between 26 and 40 years of age \cite{Alao_2006i, Ali_2022g, Apikotoa_2022f, Ataya_2013, Benoist_2019e, Bhasin_2014, Chang_2017f, Cox_2007, DelgadoSalazar_2020c, Farhadi_2024h, Fry_2010, Gardner_2017h, Guinan_2019f, Jin_2023, Kumar_2019f, Losanoff_1996, Misra_2013, Qureshi_2016, Riva_2018j, Sakellaridis_2008f, Tammana_2012j, Trgo_2012f, Wnęk_2015f, Yildiz_2016e, fjbuilsRepeatedBehaviorDeliberate2024}, 18 cases (25\%) were between 18 and 25 years of age \cite{Akay_2015f, Ali_2017, Atayan_2016, Bhattacharjee_2008, Csaky_1998e, Kar_2015, Kariholu_2008, Kobiela_2015, Losanoff_1996, Losanoff_1997e, Mesfin_2022a, Peixoto_2017f, Sobnach_2011f, Tupesis_2004f, Yasin_2009}, 13 cases (18\%) were under 18 years of age \cite{AlShaaibi_2021b, Ali_2020f, Cauchi_2002, DivsalarP._2023a, Goldman_1998f, Liu_2005, Naji_2012f, Ohno_2005, Tanrikulu_2015e, Tay_2004, Wildhaber_2005}, 11 cases (15\%) were between 41 and 60 years of age \cite{Al-Faham_2020k, Bhumi_2024f, CamachoDorado_2018, Emamhadi_2018, Hardy_2023g, Jehangir_2019h, Kumar_2001, Sultan_2024f, Thapa_2019f, Wadhwa_2015e, teWildt_2010}, 3 cases (4\%) were over 60 years of age \cite{Beecroft_1998, Kerestes_2019, Li_2013}, 2 cases (3\%) had no age documented \cite{Berry_2021e}. All 90 were male gender. 90 cases (100\%) were detained at the time of ingestion \cite{Elghali_2016, Karp_1991b, Lee_2007}, 88 cases (98\%) were intentional ingestions \cite{Elghali_2016, Karp_1991b, Lee_2007}, 30 cases (33\%) had a psychiatric history documented \cite{Elghali_2016, Karp_1991b, Lee_2007}, 2 cases (2\%) had a history of prior ingestion \cite{Elghali_2016}. No cases were reported for were psychiatric inpatients, were displaced people, were under the influence of alcohol at the time of ingestion, and had a severe disability history.
\paragraph*{Motivation}  70 cases (78\%) reported protest motivation \cite{Elghali_2016, Karp_1991b, Lee_2007}, 12 cases (13\%) reported psychiatric motivation \cite{Karp_1991b}, 6 cases (7\%) reported self-harm motivation \cite{Elghali_2016, Karp_1991b}. No cases were reported for psychosocial motivation and other motivation.
\paragraph*{Object Characteristics}  68 cases (76\%) involved sharp object ingestion \cite{Elghali_2016, Karp_1991b, Lee_2007}, 32 cases (36\%) involved long (\textgreater 5cm) object ingestion \cite{Lee_2007}, 25 cases (28\%) involved ingestion of multiple objects \cite{Elghali_2016, Lee_2007}. No cases were reported for button battery ingestion, magnet ingestion, and involved large diameter (\textgreater 2.5cm) object ingestion.
\paragraph*{Outcomes}  47 cases (52\%) underwent endoscopic intervention \cite{Elghali_2016, Lee_2007}, 29 cases (32\%) were managed conservatively \cite{Elghali_2016, Karp_1991b}, 15 cases (17\%) underwent surgical intervention \cite{Elghali_2016, Karp_1991b, Lee_2007}, 6 cases (7\%) reported complications \cite{Lee_2007}, 1 case (1\%) died \cite{Elghali_2016}.
\paragraph*{Gender} 43 cases (60\%) were male \cite{Akay_2015f, Al-Faham_2020k, Alao_2006i, Ali_2017, Ali_2022g, Apikotoa_2022f, Atayan_2016, Benoist_2019e, Berry_2021e, Bhumi_2024f, CamachoDorado_2018, Csaky_1998e, Emamhadi_2018, Farhadi_2024h, Fry_2010, Gardner_2017h, Guinan_2019f, Jehangir_2019h, Jin_2023, Kobiela_2015, Kumar_2001, Kumar_2019f, Liu_2005, Losanoff_1996, Losanoff_1997e, Mesfin_2022a, Misra_2013, Qureshi_2016, Riva_2018j, Sobnach_2011f, Tammana_2012j, Tanrikulu_2015e, Tay_2004, Thapa_2019f, Trgo_2012f, Wadhwa_2015e, Yasin_2009, teWildt_2010}, 28 cases (39\%) were female \cite{AlShaaibi_2021b, Ali_2020f, Ataya_2013, Beecroft_1998, Bhasin_2014, Bhattacharjee_2008, Cauchi_2002, Chang_2017f, Cox_2007, DelgadoSalazar_2020c, DivsalarP._2023a, Goldman_1998f, Hardy_2023g, Kar_2015, Kariholu_2008, Kerestes_2019, Li_2013, Naji_2012f, Ohno_2005, Peixoto_2017f, Sakellaridis_2008f, Sultan_2024f, Tupesis_2004f, Wildhaber_2005, Wnęk_2015f, Yildiz_2016e}, 1 case (1\%) had no gender recorded \cite{fjbuilsRepeatedBehaviorDeliberate2024}. \paragraph*{Age Group} 25 cases (35\%) were between 26 and 40 years of age \cite{Alao_2006i, Ali_2022g, Apikotoa_2022f, Ataya_2013, Benoist_2019e, Bhasin_2014, Chang_2017f, Cox_2007, DelgadoSalazar_2020c, Farhadi_2024h, Fry_2010, Gardner_2017h, Guinan_2019f, Jin_2023, Kumar_2019f, Losanoff_1996, Misra_2013, Qureshi_2016, Riva_2018j, Sakellaridis_2008f, Tammana_2012j, Trgo_2012f, Wnęk_2015f, Yildiz_2016e, fjbuilsRepeatedBehaviorDeliberate2024}, 18 cases (25\%) were between 18 and 25 years of age \cite{Akay_2015f, Ali_2017, Atayan_2016, Bhattacharjee_2008, Csaky_1998e, Kar_2015, Kariholu_2008, Kobiela_2015, Losanoff_1996, Losanoff_1997e, Mesfin_2022a, Peixoto_2017f, Sobnach_2011f, Tupesis_2004f, Yasin_2009}, 13 cases (18\%) were under 18 years of age \cite{AlShaaibi_2021b, Ali_2020f, Cauchi_2002, DivsalarP._2023a, Goldman_1998f, Liu_2005, Naji_2012f, Ohno_2005, Tanrikulu_2015e, Tay_2004, Wildhaber_2005}, 11 cases (15\%) were between 41 and 60 years of age \cite{Al-Faham_2020k, Bhumi_2024f, CamachoDorado_2018, Emamhadi_2018, Hardy_2023g, Jehangir_2019h, Kumar_2001, Sultan_2024f, Thapa_2019f, Wadhwa_2015e, teWildt_2010}, 3 cases (4\%) were over 60 years of age \cite{Beecroft_1998, Kerestes_2019, Li_2013}, 2 cases (3\%) had no age documented \cite{Berry_2021e}. All 90 were male gender. 90 cases (100\%) were detained at the time of ingestion \cite{Elghali_2016, Karp_1991b, Lee_2007}, 88 cases (98\%) were intentional ingestions \cite{Elghali_2016, Karp_1991b, Lee_2007}, 30 cases (33\%) had a psychiatric history documented \cite{Elghali_2016, Karp_1991b, Lee_2007}, 2 cases (2\%) had a history of prior ingestion \cite{Elghali_2016}. No cases were reported for were psychiatric inpatients, were displaced people, were under the influence of alcohol at the time of ingestion, and had a severe disability history.
\paragraph*{Motivation}  70 cases (78\%) reported protest motivation \cite{Elghali_2016, Karp_1991b, Lee_2007}, 12 cases (13\%) reported psychiatric motivation \cite{Karp_1991b}, 6 cases (7\%) reported self-harm motivation \cite{Elghali_2016, Karp_1991b}. No cases were reported for psychosocial motivation and other motivation.
\paragraph*{Object Characteristics}  68 cases (76\%) involved sharp object ingestion \cite{Elghali_2016, Karp_1991b, Lee_2007}, 32 cases (36\%) involved long (\textgreater 5cm) object ingestion \cite{Lee_2007}, 25 cases (28\%) involved ingestion of multiple objects \cite{Elghali_2016, Lee_2007}. No cases were reported for button battery ingestion, magnet ingestion, and involved large diameter (\textgreater 2.5cm) object ingestion.
\paragraph*{Outcomes}  47 cases (52\%) underwent endoscopic intervention \cite{Elghali_2016, Lee_2007}, 29 cases (32\%) were managed conservatively \cite{Elghali_2016, Karp_1991b}, 15 cases (17\%) underwent surgical intervention \cite{Elghali_2016, Karp_1991b, Lee_2007}, 6 cases (7\%) reported complications \cite{Lee_2007}, 1 case (1\%) died \cite{Elghali_2016}.
\paragraph*{Gender} 43 cases (60\%) were male \cite{Akay_2015f, Al-Faham_2020k, Alao_2006i, Ali_2017, Ali_2022g, Apikotoa_2022f, Atayan_2016, Benoist_2019e, Berry_2021e, Bhumi_2024f, CamachoDorado_2018, Csaky_1998e, Emamhadi_2018, Farhadi_2024h, Fry_2010, Gardner_2017h, Guinan_2019f, Jehangir_2019h, Jin_2023, Kobiela_2015, Kumar_2001, Kumar_2019f, Liu_2005, Losanoff_1996, Losanoff_1997e, Mesfin_2022a, Misra_2013, Qureshi_2016, Riva_2018j, Sobnach_2011f, Tammana_2012j, Tanrikulu_2015e, Tay_2004, Thapa_2019f, Trgo_2012f, Wadhwa_2015e, Yasin_2009, teWildt_2010}, 28 cases (39\%) were female \cite{AlShaaibi_2021b, Ali_2020f, Ataya_2013, Beecroft_1998, Bhasin_2014, Bhattacharjee_2008, Cauchi_2002, Chang_2017f, Cox_2007, DelgadoSalazar_2020c, DivsalarP._2023a, Goldman_1998f, Hardy_2023g, Kar_2015, Kariholu_2008, Kerestes_2019, Li_2013, Naji_2012f, Ohno_2005, Peixoto_2017f, Sakellaridis_2008f, Sultan_2024f, Tupesis_2004f, Wildhaber_2005, Wnęk_2015f, Yildiz_2016e}, 1 case (1\%) had no gender recorded \cite{fjbuilsRepeatedBehaviorDeliberate2024}. \paragraph*{Age Group} 25 cases (35\%) were between 26 and 40 years of age \cite{Alao_2006i, Ali_2022g, Apikotoa_2022f, Ataya_2013, Benoist_2019e, Bhasin_2014, Chang_2017f, Cox_2007, DelgadoSalazar_2020c, Farhadi_2024h, Fry_2010, Gardner_2017h, Guinan_2019f, Jin_2023, Kumar_2019f, Losanoff_1996, Misra_2013, Qureshi_2016, Riva_2018j, Sakellaridis_2008f, Tammana_2012j, Trgo_2012f, Wnęk_2015f, Yildiz_2016e, fjbuilsRepeatedBehaviorDeliberate2024}, 18 cases (25\%) were between 18 and 25 years of age \cite{Akay_2015f, Ali_2017, Atayan_2016, Bhattacharjee_2008, Csaky_1998e, Kar_2015, Kariholu_2008, Kobiela_2015, Losanoff_1996, Losanoff_1997e, Mesfin_2022a, Peixoto_2017f, Sobnach_2011f, Tupesis_2004f, Yasin_2009}, 13 cases (18\%) were under 18 years of age \cite{AlShaaibi_2021b, Ali_2020f, Cauchi_2002, DivsalarP._2023a, Goldman_1998f, Liu_2005, Naji_2012f, Ohno_2005, Tanrikulu_2015e, Tay_2004, Wildhaber_2005}, 11 cases (15\%) were between 41 and 60 years of age \cite{Al-Faham_2020k, Bhumi_2024f, CamachoDorado_2018, Emamhadi_2018, Hardy_2023g, Jehangir_2019h, Kumar_2001, Sultan_2024f, Thapa_2019f, Wadhwa_2015e, teWildt_2010}, 3 cases (4\%) were over 60 years of age \cite{Beecroft_1998, Kerestes_2019, Li_2013}, 2 cases (3\%) had no age documented \cite{Berry_2021e}. All 90 were male gender. 90 cases (100\%) were detained at the time of ingestion \cite{Elghali_2016, Karp_1991b, Lee_2007}, 88 cases (98\%) were intentional ingestions \cite{Elghali_2016, Karp_1991b, Lee_2007}, 30 cases (33\%) had a psychiatric history documented \cite{Elghali_2016, Karp_1991b, Lee_2007}, 2 cases (2\%) had a history of prior ingestion \cite{Elghali_2016}. No cases were reported for were psychiatric inpatients, were displaced people, were under the influence of alcohol at the time of ingestion, and had a severe disability history.
\paragraph*{Motivation}  70 cases (78\%) reported protest motivation \cite{Elghali_2016, Karp_1991b, Lee_2007}, 12 cases (13\%) reported psychiatric motivation \cite{Karp_1991b}, 6 cases (7\%) reported self-harm motivation \cite{Elghali_2016, Karp_1991b}. No cases were reported for psychosocial motivation and other motivation.
\paragraph*{Object Characteristics}  68 cases (76\%) involved sharp object ingestion \cite{Elghali_2016, Karp_1991b, Lee_2007}, 32 cases (36\%) involved long (\textgreater 5cm) object ingestion \cite{Lee_2007}, 25 cases (28\%) involved ingestion of multiple objects \cite{Elghali_2016, Lee_2007}. No cases were reported for button battery ingestion, magnet ingestion, and involved large diameter (\textgreater 2.5cm) object ingestion.
\paragraph*{Outcomes}  47 cases (52\%) underwent endoscopic intervention \cite{Elghali_2016, Lee_2007}, 29 cases (32\%) were managed conservatively \cite{Elghali_2016, Karp_1991b}, 15 cases (17\%) underwent surgical intervention \cite{Elghali_2016, Karp_1991b, Lee_2007}, 6 cases (7\%) reported complications \cite{Lee_2007}, 1 case (1\%) died \cite{Elghali_2016}.
\paragraph*{Gender} 43 cases (60\%) were male \cite{Akay_2015f, Al-Faham_2020k, Alao_2006i, Ali_2017, Ali_2022g, Apikotoa_2022f, Atayan_2016, Benoist_2019e, Berry_2021e, Bhumi_2024f, CamachoDorado_2018, Csaky_1998e, Emamhadi_2018, Farhadi_2024h, Fry_2010, Gardner_2017h, Guinan_2019f, Jehangir_2019h, Jin_2023, Kobiela_2015, Kumar_2001, Kumar_2019f, Liu_2005, Losanoff_1996, Losanoff_1997e, Mesfin_2022a, Misra_2013, Qureshi_2016, Riva_2018j, Sobnach_2011f, Tammana_2012j, Tanrikulu_2015e, Tay_2004, Thapa_2019f, Trgo_2012f, Wadhwa_2015e, Yasin_2009, teWildt_2010}, 28 cases (39\%) were female \cite{AlShaaibi_2021b, Ali_2020f, Ataya_2013, Beecroft_1998, Bhasin_2014, Bhattacharjee_2008, Cauchi_2002, Chang_2017f, Cox_2007, DelgadoSalazar_2020c, DivsalarP._2023a, Goldman_1998f, Hardy_2023g, Kar_2015, Kariholu_2008, Kerestes_2019, Li_2013, Naji_2012f, Ohno_2005, Peixoto_2017f, Sakellaridis_2008f, Sultan_2024f, Tupesis_2004f, Wildhaber_2005, Wnęk_2015f, Yildiz_2016e}, 1 case (1\%) had no gender recorded \cite{fjbuilsRepeatedBehaviorDeliberate2024}. \paragraph*{Age Group} 25 cases (35\%) were between 26 and 40 years of age \cite{Alao_2006i, Ali_2022g, Apikotoa_2022f, Ataya_2013, Benoist_2019e, Bhasin_2014, Chang_2017f, Cox_2007, DelgadoSalazar_2020c, Farhadi_2024h, Fry_2010, Gardner_2017h, Guinan_2019f, Jin_2023, Kumar_2019f, Losanoff_1996, Misra_2013, Qureshi_2016, Riva_2018j, Sakellaridis_2008f, Tammana_2012j, Trgo_2012f, Wnęk_2015f, Yildiz_2016e, fjbuilsRepeatedBehaviorDeliberate2024}, 18 cases (25\%) were between 18 and 25 years of age \cite{Akay_2015f, Ali_2017, Atayan_2016, Bhattacharjee_2008, Csaky_1998e, Kar_2015, Kariholu_2008, Kobiela_2015, Losanoff_1996, Losanoff_1997e, Mesfin_2022a, Peixoto_2017f, Sobnach_2011f, Tupesis_2004f, Yasin_2009}, 13 cases (18\%) were under 18 years of age \cite{AlShaaibi_2021b, Ali_2020f, Cauchi_2002, DivsalarP._2023a, Goldman_1998f, Liu_2005, Naji_2012f, Ohno_2005, Tanrikulu_2015e, Tay_2004, Wildhaber_2005}, 11 cases (15\%) were between 41 and 60 years of age \cite{Al-Faham_2020k, Bhumi_2024f, CamachoDorado_2018, Emamhadi_2018, Hardy_2023g, Jehangir_2019h, Kumar_2001, Sultan_2024f, Thapa_2019f, Wadhwa_2015e, teWildt_2010}, 3 cases (4\%) were over 60 years of age \cite{Beecroft_1998, Kerestes_2019, Li_2013}, 2 cases (3\%) had no age documented \cite{Berry_2021e}. All 90 were male gender. 90 cases (100\%) were detained at the time of ingestion \cite{Elghali_2016, Karp_1991b, Lee_2007}, 88 cases (98\%) were intentional ingestions \cite{Elghali_2016, Karp_1991b, Lee_2007}, 30 cases (33\%) had a psychiatric history documented \cite{Elghali_2016, Karp_1991b, Lee_2007}, 2 cases (2\%) had a history of prior ingestion \cite{Elghali_2016}. No cases were reported for were psychiatric inpatients, were displaced people, were under the influence of alcohol at the time of ingestion, and had a severe disability history.
\paragraph*{Motivation}  70 cases (78\%) reported protest motivation \cite{Elghali_2016, Karp_1991b, Lee_2007}, 12 cases (13\%) reported psychiatric motivation \cite{Karp_1991b}, 6 cases (7\%) reported self-harm motivation \cite{Elghali_2016, Karp_1991b}. No cases were reported for psychosocial motivation and other motivation.
\paragraph*{Object Characteristics}  68 cases (76\%) involved sharp object ingestion \cite{Elghali_2016, Karp_1991b, Lee_2007}, 32 cases (36\%) involved long (\textgreater 5cm) object ingestion \cite{Lee_2007}, 25 cases (28\%) involved ingestion of multiple objects \cite{Elghali_2016, Lee_2007}. No cases were reported for button battery ingestion, magnet ingestion, and involved large diameter (\textgreater 2.5cm) object ingestion.
\paragraph*{Outcomes}  47 cases (52\%) underwent endoscopic intervention \cite{Elghali_2016, Lee_2007}, 29 cases (32\%) were managed conservatively \cite{Elghali_2016, Karp_1991b}, 15 cases (17\%) underwent surgical intervention \cite{Elghali_2016, Karp_1991b, Lee_2007}, 6 cases (7\%) reported complications \cite{Lee_2007}, 1 case (1\%) died \cite{Elghali_2016}.
\paragraph*{Gender} 43 cases (60\%) were male \cite{Akay_2015f, Al-Faham_2020k, Alao_2006i, Ali_2017, Ali_2022g, Apikotoa_2022f, Atayan_2016, Benoist_2019e, Berry_2021e, Bhumi_2024f, CamachoDorado_2018, Csaky_1998e, Emamhadi_2018, Farhadi_2024h, Fry_2010, Gardner_2017h, Guinan_2019f, Jehangir_2019h, Jin_2023, Kobiela_2015, Kumar_2001, Kumar_2019f, Liu_2005, Losanoff_1996, Losanoff_1997e, Mesfin_2022a, Misra_2013, Qureshi_2016, Riva_2018j, Sobnach_2011f, Tammana_2012j, Tanrikulu_2015e, Tay_2004, Thapa_2019f, Trgo_2012f, Wadhwa_2015e, Yasin_2009, teWildt_2010}, 28 cases (39\%) were female \cite{AlShaaibi_2021b, Ali_2020f, Ataya_2013, Beecroft_1998, Bhasin_2014, Bhattacharjee_2008, Cauchi_2002, Chang_2017f, Cox_2007, DelgadoSalazar_2020c, DivsalarP._2023a, Goldman_1998f, Hardy_2023g, Kar_2015, Kariholu_2008, Kerestes_2019, Li_2013, Naji_2012f, Ohno_2005, Peixoto_2017f, Sakellaridis_2008f, Sultan_2024f, Tupesis_2004f, Wildhaber_2005, Wnęk_2015f, Yildiz_2016e}, 1 case (1\%) had no gender recorded \cite{fjbuilsRepeatedBehaviorDeliberate2024}. \paragraph*{Age Group} 25 cases (35\%) were between 26 and 40 years of age \cite{Alao_2006i, Ali_2022g, Apikotoa_2022f, Ataya_2013, Benoist_2019e, Bhasin_2014, Chang_2017f, Cox_2007, DelgadoSalazar_2020c, Farhadi_2024h, Fry_2010, Gardner_2017h, Guinan_2019f, Jin_2023, Kumar_2019f, Losanoff_1996, Misra_2013, Qureshi_2016, Riva_2018j, Sakellaridis_2008f, Tammana_2012j, Trgo_2012f, Wnęk_2015f, Yildiz_2016e, fjbuilsRepeatedBehaviorDeliberate2024}, 18 cases (25\%) were between 18 and 25 years of age \cite{Akay_2015f, Ali_2017, Atayan_2016, Bhattacharjee_2008, Csaky_1998e, Kar_2015, Kariholu_2008, Kobiela_2015, Losanoff_1996, Losanoff_1997e, Mesfin_2022a, Peixoto_2017f, Sobnach_2011f, Tupesis_2004f, Yasin_2009}, 13 cases (18\%) were under 18 years of age \cite{AlShaaibi_2021b, Ali_2020f, Cauchi_2002, DivsalarP._2023a, Goldman_1998f, Liu_2005, Naji_2012f, Ohno_2005, Tanrikulu_2015e, Tay_2004, Wildhaber_2005}, 11 cases (15\%) were between 41 and 60 years of age \cite{Al-Faham_2020k, Bhumi_2024f, CamachoDorado_2018, Emamhadi_2018, Hardy_2023g, Jehangir_2019h, Kumar_2001, Sultan_2024f, Thapa_2019f, Wadhwa_2015e, teWildt_2010}, 3 cases (4\%) were over 60 years of age \cite{Beecroft_1998, Kerestes_2019, Li_2013}, 2 cases (3\%) had no age documented \cite{Berry_2021e}. All 90 were male gender. 90 cases (100\%) were detained at the time of ingestion \cite{Elghali_2016, Karp_1991b, Lee_2007}, 88 cases (98\%) were intentional ingestions \cite{Elghali_2016, Karp_1991b, Lee_2007}, 30 cases (33\%) had a psychiatric history documented \cite{Elghali_2016, Karp_1991b, Lee_2007}, 2 cases (2\%) had a history of prior ingestion \cite{Elghali_2016}. No cases were reported for were psychiatric inpatients, were displaced people, were under the influence of alcohol at the time of ingestion, and had a severe disability history.
\paragraph*{Motivation}  70 cases (78\%) reported protest motivation \cite{Elghali_2016, Karp_1991b, Lee_2007}, 12 cases (13\%) reported psychiatric motivation \cite{Karp_1991b}, 6 cases (7\%) reported self-harm motivation \cite{Elghali_2016, Karp_1991b}. No cases were reported for psychosocial motivation and other motivation.
\paragraph*{Object Characteristics}  68 cases (76\%) involved sharp object ingestion \cite{Elghali_2016, Karp_1991b, Lee_2007}, 32 cases (36\%) involved long (\textgreater 5cm) object ingestion \cite{Lee_2007}, 25 cases (28\%) involved ingestion of multiple objects \cite{Elghali_2016, Lee_2007}. No cases were reported for button battery ingestion, magnet ingestion, and involved large diameter (\textgreater 2.5cm) object ingestion.
\paragraph*{Outcomes}  47 cases (52\%) underwent endoscopic intervention \cite{Elghali_2016, Lee_2007}, 29 cases (32\%) were managed conservatively \cite{Elghali_2016, Karp_1991b}, 15 cases (17\%) underwent surgical intervention \cite{Elghali_2016, Karp_1991b, Lee_2007}, 6 cases (7\%) reported complications \cite{Lee_2007}, 1 case (1\%) died \cite{Elghali_2016}.
\paragraph*{Gender} 43 cases (60\%) were male \cite{Akay_2015f, Al-Faham_2020k, Alao_2006i, Ali_2017, Ali_2022g, Apikotoa_2022f, Atayan_2016, Benoist_2019e, Berry_2021e, Bhumi_2024f, CamachoDorado_2018, Csaky_1998e, Emamhadi_2018, Farhadi_2024h, Fry_2010, Gardner_2017h, Guinan_2019f, Jehangir_2019h, Jin_2023, Kobiela_2015, Kumar_2001, Kumar_2019f, Liu_2005, Losanoff_1996, Losanoff_1997e, Mesfin_2022a, Misra_2013, Qureshi_2016, Riva_2018j, Sobnach_2011f, Tammana_2012j, Tanrikulu_2015e, Tay_2004, Thapa_2019f, Trgo_2012f, Wadhwa_2015e, Yasin_2009, teWildt_2010}, 28 cases (39\%) were female \cite{AlShaaibi_2021b, Ali_2020f, Ataya_2013, Beecroft_1998, Bhasin_2014, Bhattacharjee_2008, Cauchi_2002, Chang_2017f, Cox_2007, DelgadoSalazar_2020c, DivsalarP._2023a, Goldman_1998f, Hardy_2023g, Kar_2015, Kariholu_2008, Kerestes_2019, Li_2013, Naji_2012f, Ohno_2005, Peixoto_2017f, Sakellaridis_2008f, Sultan_2024f, Tupesis_2004f, Wildhaber_2005, Wnęk_2015f, Yildiz_2016e}, 1 case (1\%) had no gender recorded \cite{fjbuilsRepeatedBehaviorDeliberate2024}. \paragraph*{Age Group} 25 cases (35\%) were between 26 and 40 years of age \cite{Alao_2006i, Ali_2022g, Apikotoa_2022f, Ataya_2013, Benoist_2019e, Bhasin_2014, Chang_2017f, Cox_2007, DelgadoSalazar_2020c, Farhadi_2024h, Fry_2010, Gardner_2017h, Guinan_2019f, Jin_2023, Kumar_2019f, Losanoff_1996, Misra_2013, Qureshi_2016, Riva_2018j, Sakellaridis_2008f, Tammana_2012j, Trgo_2012f, Wnęk_2015f, Yildiz_2016e, fjbuilsRepeatedBehaviorDeliberate2024}, 18 cases (25\%) were between 18 and 25 years of age \cite{Akay_2015f, Ali_2017, Atayan_2016, Bhattacharjee_2008, Csaky_1998e, Kar_2015, Kariholu_2008, Kobiela_2015, Losanoff_1996, Losanoff_1997e, Mesfin_2022a, Peixoto_2017f, Sobnach_2011f, Tupesis_2004f, Yasin_2009}, 13 cases (18\%) were under 18 years of age \cite{AlShaaibi_2021b, Ali_2020f, Cauchi_2002, DivsalarP._2023a, Goldman_1998f, Liu_2005, Naji_2012f, Ohno_2005, Tanrikulu_2015e, Tay_2004, Wildhaber_2005}, 11 cases (15\%) were between 41 and 60 years of age \cite{Al-Faham_2020k, Bhumi_2024f, CamachoDorado_2018, Emamhadi_2018, Hardy_2023g, Jehangir_2019h, Kumar_2001, Sultan_2024f, Thapa_2019f, Wadhwa_2015e, teWildt_2010}, 3 cases (4\%) were over 60 years of age \cite{Beecroft_1998, Kerestes_2019, Li_2013}, 2 cases (3\%) had no age documented \cite{Berry_2021e}. All 90 were male gender. 90 cases (100\%) were detained at the time of ingestion \cite{Elghali_2016, Karp_1991b, Lee_2007}, 88 cases (98\%) were intentional ingestions \cite{Elghali_2016, Karp_1991b, Lee_2007}, 30 cases (33\%) had a psychiatric history documented \cite{Elghali_2016, Karp_1991b, Lee_2007}, 2 cases (2\%) had a history of prior ingestion \cite{Elghali_2016}. No cases were reported for were psychiatric inpatients, were displaced people, were under the influence of alcohol at the time of ingestion, and had a severe disability history.
\paragraph*{Motivation}  70 cases (78\%) reported protest motivation \cite{Elghali_2016, Karp_1991b, Lee_2007}, 12 cases (13\%) reported psychiatric motivation \cite{Karp_1991b}, 6 cases (7\%) reported self-harm motivation \cite{Elghali_2016, Karp_1991b}. No cases were reported for psychosocial motivation and other motivation.
\paragraph*{Object Characteristics}  68 cases (76\%) involved sharp object ingestion \cite{Elghali_2016, Karp_1991b, Lee_2007}, 32 cases (36\%) involved long (\textgreater 5cm) object ingestion \cite{Lee_2007}, 25 cases (28\%) involved ingestion of multiple objects \cite{Elghali_2016, Lee_2007}. No cases were reported for button battery ingestion, magnet ingestion, and involved large diameter (\textgreater 2.5cm) object ingestion.
\paragraph*{Outcomes}  47 cases (52\%) underwent endoscopic intervention \cite{Elghali_2016, Lee_2007}, 29 cases (32\%) were managed conservatively \cite{Elghali_2016, Karp_1991b}, 15 cases (17\%) underwent surgical intervention \cite{Elghali_2016, Karp_1991b, Lee_2007}, 6 cases (7\%) reported complications \cite{Lee_2007}, 1 case (1\%) died \cite{Elghali_2016}.
\paragraph*{Gender} 43 cases (60\%) were male \cite{Akay_2015f, Al-Faham_2020k, Alao_2006i, Ali_2017, Ali_2022g, Apikotoa_2022f, Atayan_2016, Benoist_2019e, Berry_2021e, Bhumi_2024f, CamachoDorado_2018, Csaky_1998e, Emamhadi_2018, Farhadi_2024h, Fry_2010, Gardner_2017h, Guinan_2019f, Jehangir_2019h, Jin_2023, Kobiela_2015, Kumar_2001, Kumar_2019f, Liu_2005, Losanoff_1996, Losanoff_1997e, Mesfin_2022a, Misra_2013, Qureshi_2016, Riva_2018j, Sobnach_2011f, Tammana_2012j, Tanrikulu_2015e, Tay_2004, Thapa_2019f, Trgo_2012f, Wadhwa_2015e, Yasin_2009, teWildt_2010}, 28 cases (39\%) were female \cite{AlShaaibi_2021b, Ali_2020f, Ataya_2013, Beecroft_1998, Bhasin_2014, Bhattacharjee_2008, Cauchi_2002, Chang_2017f, Cox_2007, DelgadoSalazar_2020c, DivsalarP._2023a, Goldman_1998f, Hardy_2023g, Kar_2015, Kariholu_2008, Kerestes_2019, Li_2013, Naji_2012f, Ohno_2005, Peixoto_2017f, Sakellaridis_2008f, Sultan_2024f, Tupesis_2004f, Wildhaber_2005, Wnęk_2015f, Yildiz_2016e}, 1 case (1\%) had no gender recorded \cite{fjbuilsRepeatedBehaviorDeliberate2024}. \paragraph*{Age Group} 25 cases (35\%) were between 26 and 40 years of age \cite{Alao_2006i, Ali_2022g, Apikotoa_2022f, Ataya_2013, Benoist_2019e, Bhasin_2014, Chang_2017f, Cox_2007, DelgadoSalazar_2020c, Farhadi_2024h, Fry_2010, Gardner_2017h, Guinan_2019f, Jin_2023, Kumar_2019f, Losanoff_1996, Misra_2013, Qureshi_2016, Riva_2018j, Sakellaridis_2008f, Tammana_2012j, Trgo_2012f, Wnęk_2015f, Yildiz_2016e, fjbuilsRepeatedBehaviorDeliberate2024}, 18 cases (25\%) were between 18 and 25 years of age \cite{Akay_2015f, Ali_2017, Atayan_2016, Bhattacharjee_2008, Csaky_1998e, Kar_2015, Kariholu_2008, Kobiela_2015, Losanoff_1996, Losanoff_1997e, Mesfin_2022a, Peixoto_2017f, Sobnach_2011f, Tupesis_2004f, Yasin_2009}, 13 cases (18\%) were under 18 years of age \cite{AlShaaibi_2021b, Ali_2020f, Cauchi_2002, DivsalarP._2023a, Goldman_1998f, Liu_2005, Naji_2012f, Ohno_2005, Tanrikulu_2015e, Tay_2004, Wildhaber_2005}, 11 cases (15\%) were between 41 and 60 years of age \cite{Al-Faham_2020k, Bhumi_2024f, CamachoDorado_2018, Emamhadi_2018, Hardy_2023g, Jehangir_2019h, Kumar_2001, Sultan_2024f, Thapa_2019f, Wadhwa_2015e, teWildt_2010}, 3 cases (4\%) were over 60 years of age \cite{Beecroft_1998, Kerestes_2019, Li_2013}, 2 cases (3\%) had no age documented \cite{Berry_2021e}. All 90 were male gender. 90 cases (100\%) were detained at the time of ingestion \cite{Elghali_2016, Karp_1991b, Lee_2007}, 88 cases (98\%) were intentional ingestions \cite{Elghali_2016, Karp_1991b, Lee_2007}, 30 cases (33\%) had a psychiatric history documented \cite{Elghali_2016, Karp_1991b, Lee_2007}, 2 cases (2\%) had a history of prior ingestion \cite{Elghali_2016}. No cases were reported for were psychiatric inpatients, were displaced people, were under the influence of alcohol at the time of ingestion, and had a severe disability history.
\paragraph*{Motivation}  70 cases (78\%) reported protest motivation \cite{Elghali_2016, Karp_1991b, Lee_2007}, 12 cases (13\%) reported psychiatric motivation \cite{Karp_1991b}, 6 cases (7\%) reported self-harm motivation \cite{Elghali_2016, Karp_1991b}. No cases were reported for psychosocial motivation and other motivation.
\paragraph*{Object Characteristics}  68 cases (76\%) involved sharp object ingestion \cite{Elghali_2016, Karp_1991b, Lee_2007}, 32 cases (36\%) involved long (\textgreater 5cm) object ingestion \cite{Lee_2007}, 25 cases (28\%) involved ingestion of multiple objects \cite{Elghali_2016, Lee_2007}. No cases were reported for button battery ingestion, magnet ingestion, and involved large diameter (\textgreater 2.5cm) object ingestion.
\paragraph*{Outcomes}  47 cases (52\%) underwent endoscopic intervention \cite{Elghali_2016, Lee_2007}, 29 cases (32\%) were managed conservatively \cite{Elghali_2016, Karp_1991b}, 15 cases (17\%) underwent surgical intervention \cite{Elghali_2016, Karp_1991b, Lee_2007}, 6 cases (7\%) reported complications \cite{Lee_2007}, 1 case (1\%) died \cite{Elghali_2016}.
\paragraph*{Gender} 43 cases (60\%) were male \cite{Akay_2015f, Al-Faham_2020k, Alao_2006i, Ali_2017, Ali_2022g, Apikotoa_2022f, Atayan_2016, Benoist_2019e, Berry_2021e, Bhumi_2024f, CamachoDorado_2018, Csaky_1998e, Emamhadi_2018, Farhadi_2024h, Fry_2010, Gardner_2017h, Guinan_2019f, Jehangir_2019h, Jin_2023, Kobiela_2015, Kumar_2001, Kumar_2019f, Liu_2005, Losanoff_1996, Losanoff_1997e, Mesfin_2022a, Misra_2013, Qureshi_2016, Riva_2018j, Sobnach_2011f, Tammana_2012j, Tanrikulu_2015e, Tay_2004, Thapa_2019f, Trgo_2012f, Wadhwa_2015e, Yasin_2009, teWildt_2010}, 28 cases (39\%) were female \cite{AlShaaibi_2021b, Ali_2020f, Ataya_2013, Beecroft_1998, Bhasin_2014, Bhattacharjee_2008, Cauchi_2002, Chang_2017f, Cox_2007, DelgadoSalazar_2020c, DivsalarP._2023a, Goldman_1998f, Hardy_2023g, Kar_2015, Kariholu_2008, Kerestes_2019, Li_2013, Naji_2012f, Ohno_2005, Peixoto_2017f, Sakellaridis_2008f, Sultan_2024f, Tupesis_2004f, Wildhaber_2005, Wnęk_2015f, Yildiz_2016e}, 1 case (1\%) had no gender recorded \cite{fjbuilsRepeatedBehaviorDeliberate2024}. \paragraph*{Age Group} 25 cases (35\%) were between 26 and 40 years of age \cite{Alao_2006i, Ali_2022g, Apikotoa_2022f, Ataya_2013, Benoist_2019e, Bhasin_2014, Chang_2017f, Cox_2007, DelgadoSalazar_2020c, Farhadi_2024h, Fry_2010, Gardner_2017h, Guinan_2019f, Jin_2023, Kumar_2019f, Losanoff_1996, Misra_2013, Qureshi_2016, Riva_2018j, Sakellaridis_2008f, Tammana_2012j, Trgo_2012f, Wnęk_2015f, Yildiz_2016e, fjbuilsRepeatedBehaviorDeliberate2024}, 18 cases (25\%) were between 18 and 25 years of age \cite{Akay_2015f, Ali_2017, Atayan_2016, Bhattacharjee_2008, Csaky_1998e, Kar_2015, Kariholu_2008, Kobiela_2015, Losanoff_1996, Losanoff_1997e, Mesfin_2022a, Peixoto_2017f, Sobnach_2011f, Tupesis_2004f, Yasin_2009}, 13 cases (18\%) were under 18 years of age \cite{AlShaaibi_2021b, Ali_2020f, Cauchi_2002, DivsalarP._2023a, Goldman_1998f, Liu_2005, Naji_2012f, Ohno_2005, Tanrikulu_2015e, Tay_2004, Wildhaber_2005}, 11 cases (15\%) were between 41 and 60 years of age \cite{Al-Faham_2020k, Bhumi_2024f, CamachoDorado_2018, Emamhadi_2018, Hardy_2023g, Jehangir_2019h, Kumar_2001, Sultan_2024f, Thapa_2019f, Wadhwa_2015e, teWildt_2010}, 3 cases (4\%) were over 60 years of age \cite{Beecroft_1998, Kerestes_2019, Li_2013}, 2 cases (3\%) had no age documented \cite{Berry_2021e}. All 90 were male gender. 90 cases (100\%) were detained at the time of ingestion \cite{Elghali_2016, Karp_1991b, Lee_2007}, 88 cases (98\%) were intentional ingestions \cite{Elghali_2016, Karp_1991b, Lee_2007}, 30 cases (33\%) had a psychiatric history documented \cite{Elghali_2016, Karp_1991b, Lee_2007}, 2 cases (2\%) had a history of prior ingestion \cite{Elghali_2016}. No cases were reported for were psychiatric inpatients, were displaced people, were under the influence of alcohol at the time of ingestion, and had a severe disability history.
\paragraph*{Motivation}  70 cases (78\%) reported protest motivation \cite{Elghali_2016, Karp_1991b, Lee_2007}, 12 cases (13\%) reported psychiatric motivation \cite{Karp_1991b}, 6 cases (7\%) reported self-harm motivation \cite{Elghali_2016, Karp_1991b}. No cases were reported for psychosocial motivation and other motivation.
\paragraph*{Object Characteristics}  68 cases (76\%) involved sharp object ingestion \cite{Elghali_2016, Karp_1991b, Lee_2007}, 32 cases (36\%) involved long (\textgreater 5cm) object ingestion \cite{Lee_2007}, 25 cases (28\%) involved ingestion of multiple objects \cite{Elghali_2016, Lee_2007}. No cases were reported for button battery ingestion, magnet ingestion, and involved large diameter (\textgreater 2.5cm) object ingestion.
\paragraph*{Outcomes}  47 cases (52\%) underwent endoscopic intervention \cite{Elghali_2016, Lee_2007}, 29 cases (32\%) were managed conservatively \cite{Elghali_2016, Karp_1991b}, 15 cases (17\%) underwent surgical intervention \cite{Elghali_2016, Karp_1991b, Lee_2007}, 6 cases (7\%) reported complications \cite{Lee_2007}, 1 case (1\%) died \cite{Elghali_2016}.
\paragraph*{Gender} 43 cases (60\%) were male \cite{Akay_2015f, Al-Faham_2020k, Alao_2006i, Ali_2017, Ali_2022g, Apikotoa_2022f, Atayan_2016, Benoist_2019e, Berry_2021e, Bhumi_2024f, CamachoDorado_2018, Csaky_1998e, Emamhadi_2018, Farhadi_2024h, Fry_2010, Gardner_2017h, Guinan_2019f, Jehangir_2019h, Jin_2023, Kobiela_2015, Kumar_2001, Kumar_2019f, Liu_2005, Losanoff_1996, Losanoff_1997e, Mesfin_2022a, Misra_2013, Qureshi_2016, Riva_2018j, Sobnach_2011f, Tammana_2012j, Tanrikulu_2015e, Tay_2004, Thapa_2019f, Trgo_2012f, Wadhwa_2015e, Yasin_2009, teWildt_2010}, 28 cases (39\%) were female \cite{AlShaaibi_2021b, Ali_2020f, Ataya_2013, Beecroft_1998, Bhasin_2014, Bhattacharjee_2008, Cauchi_2002, Chang_2017f, Cox_2007, DelgadoSalazar_2020c, DivsalarP._2023a, Goldman_1998f, Hardy_2023g, Kar_2015, Kariholu_2008, Kerestes_2019, Li_2013, Naji_2012f, Ohno_2005, Peixoto_2017f, Sakellaridis_2008f, Sultan_2024f, Tupesis_2004f, Wildhaber_2005, Wnęk_2015f, Yildiz_2016e}, 1 case (1\%) had no gender recorded \cite{fjbuilsRepeatedBehaviorDeliberate2024}. \paragraph*{Age Group} 25 cases (35\%) were between 26 and 40 years of age \cite{Alao_2006i, Ali_2022g, Apikotoa_2022f, Ataya_2013, Benoist_2019e, Bhasin_2014, Chang_2017f, Cox_2007, DelgadoSalazar_2020c, Farhadi_2024h, Fry_2010, Gardner_2017h, Guinan_2019f, Jin_2023, Kumar_2019f, Losanoff_1996, Misra_2013, Qureshi_2016, Riva_2018j, Sakellaridis_2008f, Tammana_2012j, Trgo_2012f, Wnęk_2015f, Yildiz_2016e, fjbuilsRepeatedBehaviorDeliberate2024}, 18 cases (25\%) were between 18 and 25 years of age \cite{Akay_2015f, Ali_2017, Atayan_2016, Bhattacharjee_2008, Csaky_1998e, Kar_2015, Kariholu_2008, Kobiela_2015, Losanoff_1996, Losanoff_1997e, Mesfin_2022a, Peixoto_2017f, Sobnach_2011f, Tupesis_2004f, Yasin_2009}, 13 cases (18\%) were under 18 years of age \cite{AlShaaibi_2021b, Ali_2020f, Cauchi_2002, DivsalarP._2023a, Goldman_1998f, Liu_2005, Naji_2012f, Ohno_2005, Tanrikulu_2015e, Tay_2004, Wildhaber_2005}, 11 cases (15\%) were between 41 and 60 years of age \cite{Al-Faham_2020k, Bhumi_2024f, CamachoDorado_2018, Emamhadi_2018, Hardy_2023g, Jehangir_2019h, Kumar_2001, Sultan_2024f, Thapa_2019f, Wadhwa_2015e, teWildt_2010}, 3 cases (4\%) were over 60 years of age \cite{Beecroft_1998, Kerestes_2019, Li_2013}, 2 cases (3\%) had no age documented \cite{Berry_2021e}. All 90 were male gender. 90 cases (100\%) were detained at the time of ingestion \cite{Elghali_2016, Karp_1991b, Lee_2007}, 88 cases (98\%) were intentional ingestions \cite{Elghali_2016, Karp_1991b, Lee_2007}, 30 cases (33\%) had a psychiatric history documented \cite{Elghali_2016, Karp_1991b, Lee_2007}, 2 cases (2\%) had a history of prior ingestion \cite{Elghali_2016}. No cases were reported for were psychiatric inpatients, were displaced people, were under the influence of alcohol at the time of ingestion, and had a severe disability history.
\paragraph*{Motivation}  70 cases (78\%) reported protest motivation \cite{Elghali_2016, Karp_1991b, Lee_2007}, 12 cases (13\%) reported psychiatric motivation \cite{Karp_1991b}, 6 cases (7\%) reported self-harm motivation \cite{Elghali_2016, Karp_1991b}. No cases were reported for psychosocial motivation and other motivation.
\paragraph*{Object Characteristics}  68 cases (76\%) involved sharp object ingestion \cite{Elghali_2016, Karp_1991b, Lee_2007}, 32 cases (36\%) involved long (\textgreater 5cm) object ingestion \cite{Lee_2007}, 25 cases (28\%) involved ingestion of multiple objects \cite{Elghali_2016, Lee_2007}. No cases were reported for button battery ingestion, magnet ingestion, and involved large diameter (\textgreater 2.5cm) object ingestion.
\paragraph*{Outcomes}  47 cases (52\%) underwent endoscopic intervention \cite{Elghali_2016, Lee_2007}, 29 cases (32\%) were managed conservatively \cite{Elghali_2016, Karp_1991b}, 15 cases (17\%) underwent surgical intervention \cite{Elghali_2016, Karp_1991b, Lee_2007}, 6 cases (7\%) reported complications \cite{Lee_2007}, 1 case (1\%) died \cite{Elghali_2016}.
\paragraph*{Gender} 43 cases (60\%) were male \cite{Akay_2015f, Al-Faham_2020k, Alao_2006i, Ali_2017, Ali_2022g, Apikotoa_2022f, Atayan_2016, Benoist_2019e, Berry_2021e, Bhumi_2024f, CamachoDorado_2018, Csaky_1998e, Emamhadi_2018, Farhadi_2024h, Fry_2010, Gardner_2017h, Guinan_2019f, Jehangir_2019h, Jin_2023, Kobiela_2015, Kumar_2001, Kumar_2019f, Liu_2005, Losanoff_1996, Losanoff_1997e, Mesfin_2022a, Misra_2013, Qureshi_2016, Riva_2018j, Sobnach_2011f, Tammana_2012j, Tanrikulu_2015e, Tay_2004, Thapa_2019f, Trgo_2012f, Wadhwa_2015e, Yasin_2009, teWildt_2010}, 28 cases (39\%) were female \cite{AlShaaibi_2021b, Ali_2020f, Ataya_2013, Beecroft_1998, Bhasin_2014, Bhattacharjee_2008, Cauchi_2002, Chang_2017f, Cox_2007, DelgadoSalazar_2020c, DivsalarP._2023a, Goldman_1998f, Hardy_2023g, Kar_2015, Kariholu_2008, Kerestes_2019, Li_2013, Naji_2012f, Ohno_2005, Peixoto_2017f, Sakellaridis_2008f, Sultan_2024f, Tupesis_2004f, Wildhaber_2005, Wnęk_2015f, Yildiz_2016e}, 1 case (1\%) had no gender recorded \cite{fjbuilsRepeatedBehaviorDeliberate2024}. \paragraph*{Age Group} 25 cases (35\%) were between 26 and 40 years of age \cite{Alao_2006i, Ali_2022g, Apikotoa_2022f, Ataya_2013, Benoist_2019e, Bhasin_2014, Chang_2017f, Cox_2007, DelgadoSalazar_2020c, Farhadi_2024h, Fry_2010, Gardner_2017h, Guinan_2019f, Jin_2023, Kumar_2019f, Losanoff_1996, Misra_2013, Qureshi_2016, Riva_2018j, Sakellaridis_2008f, Tammana_2012j, Trgo_2012f, Wnęk_2015f, Yildiz_2016e, fjbuilsRepeatedBehaviorDeliberate2024}, 18 cases (25\%) were between 18 and 25 years of age \cite{Akay_2015f, Ali_2017, Atayan_2016, Bhattacharjee_2008, Csaky_1998e, Kar_2015, Kariholu_2008, Kobiela_2015, Losanoff_1996, Losanoff_1997e, Mesfin_2022a, Peixoto_2017f, Sobnach_2011f, Tupesis_2004f, Yasin_2009}, 13 cases (18\%) were under 18 years of age \cite{AlShaaibi_2021b, Ali_2020f, Cauchi_2002, DivsalarP._2023a, Goldman_1998f, Liu_2005, Naji_2012f, Ohno_2005, Tanrikulu_2015e, Tay_2004, Wildhaber_2005}, 11 cases (15\%) were between 41 and 60 years of age \cite{Al-Faham_2020k, Bhumi_2024f, CamachoDorado_2018, Emamhadi_2018, Hardy_2023g, Jehangir_2019h, Kumar_2001, Sultan_2024f, Thapa_2019f, Wadhwa_2015e, teWildt_2010}, 3 cases (4\%) were over 60 years of age \cite{Beecroft_1998, Kerestes_2019, Li_2013}, 2 cases (3\%) had no age documented \cite{Berry_2021e}. All 90 were male gender. 90 cases (100\%) were detained at the time of ingestion \cite{Elghali_2016, Karp_1991b, Lee_2007}, 88 cases (98\%) were intentional ingestions \cite{Elghali_2016, Karp_1991b, Lee_2007}, 30 cases (33\%) had a psychiatric history documented \cite{Elghali_2016, Karp_1991b, Lee_2007}, 2 cases (2\%) had a history of prior ingestion \cite{Elghali_2016}. No cases were reported for were psychiatric inpatients, were displaced people, were under the influence of alcohol at the time of ingestion, and had a severe disability history.
\paragraph*{Motivation}  70 cases (78\%) reported protest motivation \cite{Elghali_2016, Karp_1991b, Lee_2007}, 12 cases (13\%) reported psychiatric motivation \cite{Karp_1991b}, 6 cases (7\%) reported self-harm motivation \cite{Elghali_2016, Karp_1991b}. No cases were reported for psychosocial motivation and other motivation.
\paragraph*{Object Characteristics}  68 cases (76\%) involved sharp object ingestion \cite{Elghali_2016, Karp_1991b, Lee_2007}, 32 cases (36\%) involved long (\textgreater 5cm) object ingestion \cite{Lee_2007}, 25 cases (28\%) involved ingestion of multiple objects \cite{Elghali_2016, Lee_2007}. No cases were reported for button battery ingestion, magnet ingestion, and involved large diameter (\textgreater 2.5cm) object ingestion.
\paragraph*{Outcomes}  47 cases (52\%) underwent endoscopic intervention \cite{Elghali_2016, Lee_2007}, 29 cases (32\%) were managed conservatively \cite{Elghali_2016, Karp_1991b}, 15 cases (17\%) underwent surgical intervention \cite{Elghali_2016, Karp_1991b, Lee_2007}, 6 cases (7\%) reported complications \cite{Lee_2007}, 1 case (1\%) died \cite{Elghali_2016}.
\paragraph*{Gender} 43 cases (60\%) were male \cite{Akay_2015f, Al-Faham_2020k, Alao_2006i, Ali_2017, Ali_2022g, Apikotoa_2022f, Atayan_2016, Benoist_2019e, Berry_2021e, Bhumi_2024f, CamachoDorado_2018, Csaky_1998e, Emamhadi_2018, Farhadi_2024h, Fry_2010, Gardner_2017h, Guinan_2019f, Jehangir_2019h, Jin_2023, Kobiela_2015, Kumar_2001, Kumar_2019f, Liu_2005, Losanoff_1996, Losanoff_1997e, Mesfin_2022a, Misra_2013, Qureshi_2016, Riva_2018j, Sobnach_2011f, Tammana_2012j, Tanrikulu_2015e, Tay_2004, Thapa_2019f, Trgo_2012f, Wadhwa_2015e, Yasin_2009, teWildt_2010}, 28 cases (39\%) were female \cite{AlShaaibi_2021b, Ali_2020f, Ataya_2013, Beecroft_1998, Bhasin_2014, Bhattacharjee_2008, Cauchi_2002, Chang_2017f, Cox_2007, DelgadoSalazar_2020c, DivsalarP._2023a, Goldman_1998f, Hardy_2023g, Kar_2015, Kariholu_2008, Kerestes_2019, Li_2013, Naji_2012f, Ohno_2005, Peixoto_2017f, Sakellaridis_2008f, Sultan_2024f, Tupesis_2004f, Wildhaber_2005, Wnęk_2015f, Yildiz_2016e}, 1 case (1\%) had no gender recorded \cite{fjbuilsRepeatedBehaviorDeliberate2024}. \paragraph*{Age Group} 25 cases (35\%) were between 26 and 40 years of age \cite{Alao_2006i, Ali_2022g, Apikotoa_2022f, Ataya_2013, Benoist_2019e, Bhasin_2014, Chang_2017f, Cox_2007, DelgadoSalazar_2020c, Farhadi_2024h, Fry_2010, Gardner_2017h, Guinan_2019f, Jin_2023, Kumar_2019f, Losanoff_1996, Misra_2013, Qureshi_2016, Riva_2018j, Sakellaridis_2008f, Tammana_2012j, Trgo_2012f, Wnęk_2015f, Yildiz_2016e, fjbuilsRepeatedBehaviorDeliberate2024}, 18 cases (25\%) were between 18 and 25 years of age \cite{Akay_2015f, Ali_2017, Atayan_2016, Bhattacharjee_2008, Csaky_1998e, Kar_2015, Kariholu_2008, Kobiela_2015, Losanoff_1996, Losanoff_1997e, Mesfin_2022a, Peixoto_2017f, Sobnach_2011f, Tupesis_2004f, Yasin_2009}, 13 cases (18\%) were under 18 years of age \cite{AlShaaibi_2021b, Ali_2020f, Cauchi_2002, DivsalarP._2023a, Goldman_1998f, Liu_2005, Naji_2012f, Ohno_2005, Tanrikulu_2015e, Tay_2004, Wildhaber_2005}, 11 cases (15\%) were between 41 and 60 years of age \cite{Al-Faham_2020k, Bhumi_2024f, CamachoDorado_2018, Emamhadi_2018, Hardy_2023g, Jehangir_2019h, Kumar_2001, Sultan_2024f, Thapa_2019f, Wadhwa_2015e, teWildt_2010}, 3 cases (4\%) were over 60 years of age \cite{Beecroft_1998, Kerestes_2019, Li_2013}, 2 cases (3\%) had no age documented \cite{Berry_2021e}. All 90 were male gender. 90 cases (100\%) were detained at the time of ingestion \cite{Elghali_2016, Karp_1991b, Lee_2007}, 88 cases (98\%) were intentional ingestions \cite{Elghali_2016, Karp_1991b, Lee_2007}, 30 cases (33\%) had a psychiatric history documented \cite{Elghali_2016, Karp_1991b, Lee_2007}, 2 cases (2\%) had a history of prior ingestion \cite{Elghali_2016}. No cases were reported for were psychiatric inpatients, were displaced people, were under the influence of alcohol at the time of ingestion, and had a severe disability history.
\paragraph*{Motivation}  70 cases (78\%) reported protest motivation \cite{Elghali_2016, Karp_1991b, Lee_2007}, 12 cases (13\%) reported psychiatric motivation \cite{Karp_1991b}, 6 cases (7\%) reported self-harm motivation \cite{Elghali_2016, Karp_1991b}. No cases were reported for psychosocial motivation and other motivation.
\paragraph*{Object Characteristics}  68 cases (76\%) involved sharp object ingestion \cite{Elghali_2016, Karp_1991b, Lee_2007}, 32 cases (36\%) involved long (\textgreater 5cm) object ingestion \cite{Lee_2007}, 25 cases (28\%) involved ingestion of multiple objects \cite{Elghali_2016, Lee_2007}. No cases were reported for button battery ingestion, magnet ingestion, and involved large diameter (\textgreater 2.5cm) object ingestion.
\paragraph*{Outcomes}  47 cases (52\%) underwent endoscopic intervention \cite{Elghali_2016, Lee_2007}, 29 cases (32\%) were managed conservatively \cite{Elghali_2016, Karp_1991b}, 15 cases (17\%) underwent surgical intervention \cite{Elghali_2016, Karp_1991b, Lee_2007}, 6 cases (7\%) reported complications \cite{Lee_2007}, 1 case (1\%) died \cite{Elghali_2016}.
\paragraph*{Gender} 43 cases (60\%) were male \cite{Akay_2015f, Al-Faham_2020k, Alao_2006i, Ali_2017, Ali_2022g, Apikotoa_2022f, Atayan_2016, Benoist_2019e, Berry_2021e, Bhumi_2024f, CamachoDorado_2018, Csaky_1998e, Emamhadi_2018, Farhadi_2024h, Fry_2010, Gardner_2017h, Guinan_2019f, Jehangir_2019h, Jin_2023, Kobiela_2015, Kumar_2001, Kumar_2019f, Liu_2005, Losanoff_1996, Losanoff_1997e, Mesfin_2022a, Misra_2013, Qureshi_2016, Riva_2018j, Sobnach_2011f, Tammana_2012j, Tanrikulu_2015e, Tay_2004, Thapa_2019f, Trgo_2012f, Wadhwa_2015e, Yasin_2009, teWildt_2010}, 28 cases (39\%) were female \cite{AlShaaibi_2021b, Ali_2020f, Ataya_2013, Beecroft_1998, Bhasin_2014, Bhattacharjee_2008, Cauchi_2002, Chang_2017f, Cox_2007, DelgadoSalazar_2020c, DivsalarP._2023a, Goldman_1998f, Hardy_2023g, Kar_2015, Kariholu_2008, Kerestes_2019, Li_2013, Naji_2012f, Ohno_2005, Peixoto_2017f, Sakellaridis_2008f, Sultan_2024f, Tupesis_2004f, Wildhaber_2005, Wnęk_2015f, Yildiz_2016e}, 1 case (1\%) had no gender recorded \cite{fjbuilsRepeatedBehaviorDeliberate2024}. \paragraph*{Age Group} 25 cases (35\%) were between 26 and 40 years of age \cite{Alao_2006i, Ali_2022g, Apikotoa_2022f, Ataya_2013, Benoist_2019e, Bhasin_2014, Chang_2017f, Cox_2007, DelgadoSalazar_2020c, Farhadi_2024h, Fry_2010, Gardner_2017h, Guinan_2019f, Jin_2023, Kumar_2019f, Losanoff_1996, Misra_2013, Qureshi_2016, Riva_2018j, Sakellaridis_2008f, Tammana_2012j, Trgo_2012f, Wnęk_2015f, Yildiz_2016e, fjbuilsRepeatedBehaviorDeliberate2024}, 18 cases (25\%) were between 18 and 25 years of age \cite{Akay_2015f, Ali_2017, Atayan_2016, Bhattacharjee_2008, Csaky_1998e, Kar_2015, Kariholu_2008, Kobiela_2015, Losanoff_1996, Losanoff_1997e, Mesfin_2022a, Peixoto_2017f, Sobnach_2011f, Tupesis_2004f, Yasin_2009}, 13 cases (18\%) were under 18 years of age \cite{AlShaaibi_2021b, Ali_2020f, Cauchi_2002, DivsalarP._2023a, Goldman_1998f, Liu_2005, Naji_2012f, Ohno_2005, Tanrikulu_2015e, Tay_2004, Wildhaber_2005}, 11 cases (15\%) were between 41 and 60 years of age \cite{Al-Faham_2020k, Bhumi_2024f, CamachoDorado_2018, Emamhadi_2018, Hardy_2023g, Jehangir_2019h, Kumar_2001, Sultan_2024f, Thapa_2019f, Wadhwa_2015e, teWildt_2010}, 3 cases (4\%) were over 60 years of age \cite{Beecroft_1998, Kerestes_2019, Li_2013}, 2 cases (3\%) had no age documented \cite{Berry_2021e}. All 90 were male gender. 90 cases (100\%) were detained at the time of ingestion \cite{Elghali_2016, Karp_1991b, Lee_2007}, 88 cases (98\%) were intentional ingestions \cite{Elghali_2016, Karp_1991b, Lee_2007}, 30 cases (33\%) had a psychiatric history documented \cite{Elghali_2016, Karp_1991b, Lee_2007}, 2 cases (2\%) had a history of prior ingestion \cite{Elghali_2016}. No cases were reported for were psychiatric inpatients, were displaced people, were under the influence of alcohol at the time of ingestion, and had a severe disability history.
\paragraph*{Motivation}  70 cases (78\%) reported protest motivation \cite{Elghali_2016, Karp_1991b, Lee_2007}, 12 cases (13\%) reported psychiatric motivation \cite{Karp_1991b}, 6 cases (7\%) reported self-harm motivation \cite{Elghali_2016, Karp_1991b}. No cases were reported for psychosocial motivation and other motivation.
\paragraph*{Object Characteristics}  68 cases (76\%) involved sharp object ingestion \cite{Elghali_2016, Karp_1991b, Lee_2007}, 32 cases (36\%) involved long (\textgreater 5cm) object ingestion \cite{Lee_2007}, 25 cases (28\%) involved ingestion of multiple objects \cite{Elghali_2016, Lee_2007}. No cases were reported for button battery ingestion, magnet ingestion, and involved large diameter (\textgreater 2.5cm) object ingestion.
\paragraph*{Outcomes}  47 cases (52\%) underwent endoscopic intervention \cite{Elghali_2016, Lee_2007}, 29 cases (32\%) were managed conservatively \cite{Elghali_2016, Karp_1991b}, 15 cases (17\%) underwent surgical intervention \cite{Elghali_2016, Karp_1991b, Lee_2007}, 6 cases (7\%) reported complications \cite{Lee_2007}, 1 case (1\%) died \cite{Elghali_2016}.
\paragraph*{Gender} 43 cases (60\%) were male \cite{Akay_2015f, Al-Faham_2020k, Alao_2006i, Ali_2017, Ali_2022g, Apikotoa_2022f, Atayan_2016, Benoist_2019e, Berry_2021e, Bhumi_2024f, CamachoDorado_2018, Csaky_1998e, Emamhadi_2018, Farhadi_2024h, Fry_2010, Gardner_2017h, Guinan_2019f, Jehangir_2019h, Jin_2023, Kobiela_2015, Kumar_2001, Kumar_2019f, Liu_2005, Losanoff_1996, Losanoff_1997e, Mesfin_2022a, Misra_2013, Qureshi_2016, Riva_2018j, Sobnach_2011f, Tammana_2012j, Tanrikulu_2015e, Tay_2004, Thapa_2019f, Trgo_2012f, Wadhwa_2015e, Yasin_2009, teWildt_2010}, 28 cases (39\%) were female \cite{AlShaaibi_2021b, Ali_2020f, Ataya_2013, Beecroft_1998, Bhasin_2014, Bhattacharjee_2008, Cauchi_2002, Chang_2017f, Cox_2007, DelgadoSalazar_2020c, DivsalarP._2023a, Goldman_1998f, Hardy_2023g, Kar_2015, Kariholu_2008, Kerestes_2019, Li_2013, Naji_2012f, Ohno_2005, Peixoto_2017f, Sakellaridis_2008f, Sultan_2024f, Tupesis_2004f, Wildhaber_2005, Wnęk_2015f, Yildiz_2016e}, 1 case (1\%) had no gender recorded \cite{fjbuilsRepeatedBehaviorDeliberate2024}. \paragraph*{Age Group} 25 cases (35\%) were between 26 and 40 years of age \cite{Alao_2006i, Ali_2022g, Apikotoa_2022f, Ataya_2013, Benoist_2019e, Bhasin_2014, Chang_2017f, Cox_2007, DelgadoSalazar_2020c, Farhadi_2024h, Fry_2010, Gardner_2017h, Guinan_2019f, Jin_2023, Kumar_2019f, Losanoff_1996, Misra_2013, Qureshi_2016, Riva_2018j, Sakellaridis_2008f, Tammana_2012j, Trgo_2012f, Wnęk_2015f, Yildiz_2016e, fjbuilsRepeatedBehaviorDeliberate2024}, 18 cases (25\%) were between 18 and 25 years of age \cite{Akay_2015f, Ali_2017, Atayan_2016, Bhattacharjee_2008, Csaky_1998e, Kar_2015, Kariholu_2008, Kobiela_2015, Losanoff_1996, Losanoff_1997e, Mesfin_2022a, Peixoto_2017f, Sobnach_2011f, Tupesis_2004f, Yasin_2009}, 13 cases (18\%) were under 18 years of age \cite{AlShaaibi_2021b, Ali_2020f, Cauchi_2002, DivsalarP._2023a, Goldman_1998f, Liu_2005, Naji_2012f, Ohno_2005, Tanrikulu_2015e, Tay_2004, Wildhaber_2005}, 11 cases (15\%) were between 41 and 60 years of age \cite{Al-Faham_2020k, Bhumi_2024f, CamachoDorado_2018, Emamhadi_2018, Hardy_2023g, Jehangir_2019h, Kumar_2001, Sultan_2024f, Thapa_2019f, Wadhwa_2015e, teWildt_2010}, 3 cases (4\%) were over 60 years of age \cite{Beecroft_1998, Kerestes_2019, Li_2013}, 2 cases (3\%) had no age documented \cite{Berry_2021e}. All 90 were male gender. 90 cases (100\%) were detained at the time of ingestion \cite{Elghali_2016, Karp_1991b, Lee_2007}, 88 cases (98\%) were intentional ingestions \cite{Elghali_2016, Karp_1991b, Lee_2007}, 30 cases (33\%) had a psychiatric history documented \cite{Elghali_2016, Karp_1991b, Lee_2007}, 2 cases (2\%) had a history of prior ingestion \cite{Elghali_2016}. No cases were reported for were psychiatric inpatients, were displaced people, were under the influence of alcohol at the time of ingestion, and had a severe disability history.
\paragraph*{Motivation}  70 cases (78\%) reported protest motivation \cite{Elghali_2016, Karp_1991b, Lee_2007}, 12 cases (13\%) reported psychiatric motivation \cite{Karp_1991b}, 6 cases (7\%) reported self-harm motivation \cite{Elghali_2016, Karp_1991b}. No cases were reported for psychosocial motivation and other motivation.
\paragraph*{Object Characteristics}  68 cases (76\%) involved sharp object ingestion \cite{Elghali_2016, Karp_1991b, Lee_2007}, 32 cases (36\%) involved long (\textgreater 5cm) object ingestion \cite{Lee_2007}, 25 cases (28\%) involved ingestion of multiple objects \cite{Elghali_2016, Lee_2007}. No cases were reported for button battery ingestion, magnet ingestion, and involved large diameter (\textgreater 2.5cm) object ingestion.
\paragraph*{Outcomes}  47 cases (52\%) underwent endoscopic intervention \cite{Elghali_2016, Lee_2007}, 29 cases (32\%) were managed conservatively \cite{Elghali_2016, Karp_1991b}, 15 cases (17\%) underwent surgical intervention \cite{Elghali_2016, Karp_1991b, Lee_2007}, 6 cases (7\%) reported complications \cite{Lee_2007}, 1 case (1\%) died \cite{Elghali_2016}.
\paragraph*{Gender} 43 cases (60\%) were male \cite{Akay_2015f, Al-Faham_2020k, Alao_2006i, Ali_2017, Ali_2022g, Apikotoa_2022f, Atayan_2016, Benoist_2019e, Berry_2021e, Bhumi_2024f, CamachoDorado_2018, Csaky_1998e, Emamhadi_2018, Farhadi_2024h, Fry_2010, Gardner_2017h, Guinan_2019f, Jehangir_2019h, Jin_2023, Kobiela_2015, Kumar_2001, Kumar_2019f, Liu_2005, Losanoff_1996, Losanoff_1997e, Mesfin_2022a, Misra_2013, Qureshi_2016, Riva_2018j, Sobnach_2011f, Tammana_2012j, Tanrikulu_2015e, Tay_2004, Thapa_2019f, Trgo_2012f, Wadhwa_2015e, Yasin_2009, teWildt_2010}, 28 cases (39\%) were female \cite{AlShaaibi_2021b, Ali_2020f, Ataya_2013, Beecroft_1998, Bhasin_2014, Bhattacharjee_2008, Cauchi_2002, Chang_2017f, Cox_2007, DelgadoSalazar_2020c, DivsalarP._2023a, Goldman_1998f, Hardy_2023g, Kar_2015, Kariholu_2008, Kerestes_2019, Li_2013, Naji_2012f, Ohno_2005, Peixoto_2017f, Sakellaridis_2008f, Sultan_2024f, Tupesis_2004f, Wildhaber_2005, Wnęk_2015f, Yildiz_2016e}, 1 case (1\%) had no gender recorded \cite{fjbuilsRepeatedBehaviorDeliberate2024}. \paragraph*{Age Group} 25 cases (35\%) were between 26 and 40 years of age \cite{Alao_2006i, Ali_2022g, Apikotoa_2022f, Ataya_2013, Benoist_2019e, Bhasin_2014, Chang_2017f, Cox_2007, DelgadoSalazar_2020c, Farhadi_2024h, Fry_2010, Gardner_2017h, Guinan_2019f, Jin_2023, Kumar_2019f, Losanoff_1996, Misra_2013, Qureshi_2016, Riva_2018j, Sakellaridis_2008f, Tammana_2012j, Trgo_2012f, Wnęk_2015f, Yildiz_2016e, fjbuilsRepeatedBehaviorDeliberate2024}, 18 cases (25\%) were between 18 and 25 years of age \cite{Akay_2015f, Ali_2017, Atayan_2016, Bhattacharjee_2008, Csaky_1998e, Kar_2015, Kariholu_2008, Kobiela_2015, Losanoff_1996, Losanoff_1997e, Mesfin_2022a, Peixoto_2017f, Sobnach_2011f, Tupesis_2004f, Yasin_2009}, 13 cases (18\%) were under 18 years of age \cite{AlShaaibi_2021b, Ali_2020f, Cauchi_2002, DivsalarP._2023a, Goldman_1998f, Liu_2005, Naji_2012f, Ohno_2005, Tanrikulu_2015e, Tay_2004, Wildhaber_2005}, 11 cases (15\%) were between 41 and 60 years of age \cite{Al-Faham_2020k, Bhumi_2024f, CamachoDorado_2018, Emamhadi_2018, Hardy_2023g, Jehangir_2019h, Kumar_2001, Sultan_2024f, Thapa_2019f, Wadhwa_2015e, teWildt_2010}, 3 cases (4\%) were over 60 years of age \cite{Beecroft_1998, Kerestes_2019, Li_2013}, 2 cases (3\%) had no age documented \cite{Berry_2021e}. All 90 were male gender. 90 cases (100\%) were detained at the time of ingestion \cite{Elghali_2016, Karp_1991b, Lee_2007}, 88 cases (98\%) were intentional ingestions \cite{Elghali_2016, Karp_1991b, Lee_2007}, 30 cases (33\%) had a psychiatric history documented \cite{Elghali_2016, Karp_1991b, Lee_2007}, 2 cases (2\%) had a history of prior ingestion \cite{Elghali_2016}. No cases were reported for were psychiatric inpatients, were displaced people, were under the influence of alcohol at the time of ingestion, and had a severe disability history.
\paragraph*{Motivation}  70 cases (78\%) reported protest motivation \cite{Elghali_2016, Karp_1991b, Lee_2007}, 12 cases (13\%) reported psychiatric motivation \cite{Karp_1991b}, 6 cases (7\%) reported self-harm motivation \cite{Elghali_2016, Karp_1991b}. No cases were reported for psychosocial motivation and other motivation.
\paragraph*{Object Characteristics}  68 cases (76\%) involved sharp object ingestion \cite{Elghali_2016, Karp_1991b, Lee_2007}, 32 cases (36\%) involved long (\textgreater 5cm) object ingestion \cite{Lee_2007}, 25 cases (28\%) involved ingestion of multiple objects \cite{Elghali_2016, Lee_2007}. No cases were reported for button battery ingestion, magnet ingestion, and involved large diameter (\textgreater 2.5cm) object ingestion.
\paragraph*{Outcomes}  47 cases (52\%) underwent endoscopic intervention \cite{Elghali_2016, Lee_2007}, 29 cases (32\%) were managed conservatively \cite{Elghali_2016, Karp_1991b}, 15 cases (17\%) underwent surgical intervention \cite{Elghali_2016, Karp_1991b, Lee_2007}, 6 cases (7\%) reported complications \cite{Lee_2007}, 1 case (1\%) died \cite{Elghali_2016}.
\paragraph*{Gender} 43 cases (60\%) were male \cite{Akay_2015f, Al-Faham_2020k, Alao_2006i, Ali_2017, Ali_2022g, Apikotoa_2022f, Atayan_2016, Benoist_2019e, Berry_2021e, Bhumi_2024f, CamachoDorado_2018, Csaky_1998e, Emamhadi_2018, Farhadi_2024h, Fry_2010, Gardner_2017h, Guinan_2019f, Jehangir_2019h, Jin_2023, Kobiela_2015, Kumar_2001, Kumar_2019f, Liu_2005, Losanoff_1996, Losanoff_1997e, Mesfin_2022a, Misra_2013, Qureshi_2016, Riva_2018j, Sobnach_2011f, Tammana_2012j, Tanrikulu_2015e, Tay_2004, Thapa_2019f, Trgo_2012f, Wadhwa_2015e, Yasin_2009, teWildt_2010}, 28 cases (39\%) were female \cite{AlShaaibi_2021b, Ali_2020f, Ataya_2013, Beecroft_1998, Bhasin_2014, Bhattacharjee_2008, Cauchi_2002, Chang_2017f, Cox_2007, DelgadoSalazar_2020c, DivsalarP._2023a, Goldman_1998f, Hardy_2023g, Kar_2015, Kariholu_2008, Kerestes_2019, Li_2013, Naji_2012f, Ohno_2005, Peixoto_2017f, Sakellaridis_2008f, Sultan_2024f, Tupesis_2004f, Wildhaber_2005, Wnęk_2015f, Yildiz_2016e}, 1 case (1\%) had no gender recorded \cite{fjbuilsRepeatedBehaviorDeliberate2024}. \paragraph*{Age Group} 25 cases (35\%) were between 26 and 40 years of age \cite{Alao_2006i, Ali_2022g, Apikotoa_2022f, Ataya_2013, Benoist_2019e, Bhasin_2014, Chang_2017f, Cox_2007, DelgadoSalazar_2020c, Farhadi_2024h, Fry_2010, Gardner_2017h, Guinan_2019f, Jin_2023, Kumar_2019f, Losanoff_1996, Misra_2013, Qureshi_2016, Riva_2018j, Sakellaridis_2008f, Tammana_2012j, Trgo_2012f, Wnęk_2015f, Yildiz_2016e, fjbuilsRepeatedBehaviorDeliberate2024}, 18 cases (25\%) were between 18 and 25 years of age \cite{Akay_2015f, Ali_2017, Atayan_2016, Bhattacharjee_2008, Csaky_1998e, Kar_2015, Kariholu_2008, Kobiela_2015, Losanoff_1996, Losanoff_1997e, Mesfin_2022a, Peixoto_2017f, Sobnach_2011f, Tupesis_2004f, Yasin_2009}, 13 cases (18\%) were under 18 years of age \cite{AlShaaibi_2021b, Ali_2020f, Cauchi_2002, DivsalarP._2023a, Goldman_1998f, Liu_2005, Naji_2012f, Ohno_2005, Tanrikulu_2015e, Tay_2004, Wildhaber_2005}, 11 cases (15\%) were between 41 and 60 years of age \cite{Al-Faham_2020k, Bhumi_2024f, CamachoDorado_2018, Emamhadi_2018, Hardy_2023g, Jehangir_2019h, Kumar_2001, Sultan_2024f, Thapa_2019f, Wadhwa_2015e, teWildt_2010}, 3 cases (4\%) were over 60 years of age \cite{Beecroft_1998, Kerestes_2019, Li_2013}, 2 cases (3\%) had no age documented \cite{Berry_2021e}. All 90 were male gender. 90 cases (100\%) were detained at the time of ingestion \cite{Elghali_2016, Karp_1991b, Lee_2007}, 88 cases (98\%) were intentional ingestions \cite{Elghali_2016, Karp_1991b, Lee_2007}, 30 cases (33\%) had a psychiatric history documented \cite{Elghali_2016, Karp_1991b, Lee_2007}, 2 cases (2\%) had a history of prior ingestion \cite{Elghali_2016}. No cases were reported for were psychiatric inpatients, were displaced people, were under the influence of alcohol at the time of ingestion, and had a severe disability history.
\paragraph*{Motivation}  70 cases (78\%) reported protest motivation \cite{Elghali_2016, Karp_1991b, Lee_2007}, 12 cases (13\%) reported psychiatric motivation \cite{Karp_1991b}, 6 cases (7\%) reported self-harm motivation \cite{Elghali_2016, Karp_1991b}. No cases were reported for psychosocial motivation and other motivation.
\paragraph*{Object Characteristics}  68 cases (76\%) involved sharp object ingestion \cite{Elghali_2016, Karp_1991b, Lee_2007}, 32 cases (36\%) involved long (\textgreater 5cm) object ingestion \cite{Lee_2007}, 25 cases (28\%) involved ingestion of multiple objects \cite{Elghali_2016, Lee_2007}. No cases were reported for button battery ingestion, magnet ingestion, and involved large diameter (\textgreater 2.5cm) object ingestion.
\paragraph*{Outcomes}  47 cases (52\%) underwent endoscopic intervention \cite{Elghali_2016, Lee_2007}, 29 cases (32\%) were managed conservatively \cite{Elghali_2016, Karp_1991b}, 15 cases (17\%) underwent surgical intervention \cite{Elghali_2016, Karp_1991b, Lee_2007}, 6 cases (7\%) reported complications \cite{Lee_2007}, 1 case (1\%) died \cite{Elghali_2016}.
\paragraph*{Gender} 43 cases (60\%) were male \cite{Akay_2015f, Al-Faham_2020k, Alao_2006i, Ali_2017, Ali_2022g, Apikotoa_2022f, Atayan_2016, Benoist_2019e, Berry_2021e, Bhumi_2024f, CamachoDorado_2018, Csaky_1998e, Emamhadi_2018, Farhadi_2024h, Fry_2010, Gardner_2017h, Guinan_2019f, Jehangir_2019h, Jin_2023, Kobiela_2015, Kumar_2001, Kumar_2019f, Liu_2005, Losanoff_1996, Losanoff_1997e, Mesfin_2022a, Misra_2013, Qureshi_2016, Riva_2018j, Sobnach_2011f, Tammana_2012j, Tanrikulu_2015e, Tay_2004, Thapa_2019f, Trgo_2012f, Wadhwa_2015e, Yasin_2009, teWildt_2010}, 28 cases (39\%) were female \cite{AlShaaibi_2021b, Ali_2020f, Ataya_2013, Beecroft_1998, Bhasin_2014, Bhattacharjee_2008, Cauchi_2002, Chang_2017f, Cox_2007, DelgadoSalazar_2020c, DivsalarP._2023a, Goldman_1998f, Hardy_2023g, Kar_2015, Kariholu_2008, Kerestes_2019, Li_2013, Naji_2012f, Ohno_2005, Peixoto_2017f, Sakellaridis_2008f, Sultan_2024f, Tupesis_2004f, Wildhaber_2005, Wnęk_2015f, Yildiz_2016e}, 1 case (1\%) had no gender recorded \cite{fjbuilsRepeatedBehaviorDeliberate2024}. \paragraph*{Age Group} 25 cases (35\%) were between 26 and 40 years of age \cite{Alao_2006i, Ali_2022g, Apikotoa_2022f, Ataya_2013, Benoist_2019e, Bhasin_2014, Chang_2017f, Cox_2007, DelgadoSalazar_2020c, Farhadi_2024h, Fry_2010, Gardner_2017h, Guinan_2019f, Jin_2023, Kumar_2019f, Losanoff_1996, Misra_2013, Qureshi_2016, Riva_2018j, Sakellaridis_2008f, Tammana_2012j, Trgo_2012f, Wnęk_2015f, Yildiz_2016e, fjbuilsRepeatedBehaviorDeliberate2024}, 18 cases (25\%) were between 18 and 25 years of age \cite{Akay_2015f, Ali_2017, Atayan_2016, Bhattacharjee_2008, Csaky_1998e, Kar_2015, Kariholu_2008, Kobiela_2015, Losanoff_1996, Losanoff_1997e, Mesfin_2022a, Peixoto_2017f, Sobnach_2011f, Tupesis_2004f, Yasin_2009}, 13 cases (18\%) were under 18 years of age \cite{AlShaaibi_2021b, Ali_2020f, Cauchi_2002, DivsalarP._2023a, Goldman_1998f, Liu_2005, Naji_2012f, Ohno_2005, Tanrikulu_2015e, Tay_2004, Wildhaber_2005}, 11 cases (15\%) were between 41 and 60 years of age \cite{Al-Faham_2020k, Bhumi_2024f, CamachoDorado_2018, Emamhadi_2018, Hardy_2023g, Jehangir_2019h, Kumar_2001, Sultan_2024f, Thapa_2019f, Wadhwa_2015e, teWildt_2010}, 3 cases (4\%) were over 60 years of age \cite{Beecroft_1998, Kerestes_2019, Li_2013}, 2 cases (3\%) had no age documented \cite{Berry_2021e}. All 90 were male gender. 90 cases (100\%) were detained at the time of ingestion \cite{Elghali_2016, Karp_1991b, Lee_2007}, 88 cases (98\%) were intentional ingestions \cite{Elghali_2016, Karp_1991b, Lee_2007}, 30 cases (33\%) had a psychiatric history documented \cite{Elghali_2016, Karp_1991b, Lee_2007}, 2 cases (2\%) had a history of prior ingestion \cite{Elghali_2016}. No cases were reported for were psychiatric inpatients, were displaced people, were under the influence of alcohol at the time of ingestion, and had a severe disability history.
\paragraph*{Motivation}  70 cases (78\%) reported protest motivation \cite{Elghali_2016, Karp_1991b, Lee_2007}, 12 cases (13\%) reported psychiatric motivation \cite{Karp_1991b}, 6 cases (7\%) reported self-harm motivation \cite{Elghali_2016, Karp_1991b}. No cases were reported for psychosocial motivation and other motivation.
\paragraph*{Object Characteristics}  68 cases (76\%) involved sharp object ingestion \cite{Elghali_2016, Karp_1991b, Lee_2007}, 32 cases (36\%) involved long (\textgreater 5cm) object ingestion \cite{Lee_2007}, 25 cases (28\%) involved ingestion of multiple objects \cite{Elghali_2016, Lee_2007}. No cases were reported for button battery ingestion, magnet ingestion, and involved large diameter (\textgreater 2.5cm) object ingestion.
\paragraph*{Outcomes}  47 cases (52\%) underwent endoscopic intervention \cite{Elghali_2016, Lee_2007}, 29 cases (32\%) were managed conservatively \cite{Elghali_2016, Karp_1991b}, 15 cases (17\%) underwent surgical intervention \cite{Elghali_2016, Karp_1991b, Lee_2007}, 6 cases (7\%) reported complications \cite{Lee_2007}, 1 case (1\%) died \cite{Elghali_2016}.
\paragraph*{Gender} 43 cases (60\%) were male \cite{Akay_2015f, Al-Faham_2020k, Alao_2006i, Ali_2017, Ali_2022g, Apikotoa_2022f, Atayan_2016, Benoist_2019e, Berry_2021e, Bhumi_2024f, CamachoDorado_2018, Csaky_1998e, Emamhadi_2018, Farhadi_2024h, Fry_2010, Gardner_2017h, Guinan_2019f, Jehangir_2019h, Jin_2023, Kobiela_2015, Kumar_2001, Kumar_2019f, Liu_2005, Losanoff_1996, Losanoff_1997e, Mesfin_2022a, Misra_2013, Qureshi_2016, Riva_2018j, Sobnach_2011f, Tammana_2012j, Tanrikulu_2015e, Tay_2004, Thapa_2019f, Trgo_2012f, Wadhwa_2015e, Yasin_2009, teWildt_2010}, 28 cases (39\%) were female \cite{AlShaaibi_2021b, Ali_2020f, Ataya_2013, Beecroft_1998, Bhasin_2014, Bhattacharjee_2008, Cauchi_2002, Chang_2017f, Cox_2007, DelgadoSalazar_2020c, DivsalarP._2023a, Goldman_1998f, Hardy_2023g, Kar_2015, Kariholu_2008, Kerestes_2019, Li_2013, Naji_2012f, Ohno_2005, Peixoto_2017f, Sakellaridis_2008f, Sultan_2024f, Tupesis_2004f, Wildhaber_2005, Wnęk_2015f, Yildiz_2016e}, 1 case (1\%) had no gender recorded \cite{fjbuilsRepeatedBehaviorDeliberate2024}. \paragraph*{Age Group} 25 cases (35\%) were between 26 and 40 years of age \cite{Alao_2006i, Ali_2022g, Apikotoa_2022f, Ataya_2013, Benoist_2019e, Bhasin_2014, Chang_2017f, Cox_2007, DelgadoSalazar_2020c, Farhadi_2024h, Fry_2010, Gardner_2017h, Guinan_2019f, Jin_2023, Kumar_2019f, Losanoff_1996, Misra_2013, Qureshi_2016, Riva_2018j, Sakellaridis_2008f, Tammana_2012j, Trgo_2012f, Wnęk_2015f, Yildiz_2016e, fjbuilsRepeatedBehaviorDeliberate2024}, 18 cases (25\%) were between 18 and 25 years of age \cite{Akay_2015f, Ali_2017, Atayan_2016, Bhattacharjee_2008, Csaky_1998e, Kar_2015, Kariholu_2008, Kobiela_2015, Losanoff_1996, Losanoff_1997e, Mesfin_2022a, Peixoto_2017f, Sobnach_2011f, Tupesis_2004f, Yasin_2009}, 13 cases (18\%) were under 18 years of age \cite{AlShaaibi_2021b, Ali_2020f, Cauchi_2002, DivsalarP._2023a, Goldman_1998f, Liu_2005, Naji_2012f, Ohno_2005, Tanrikulu_2015e, Tay_2004, Wildhaber_2005}, 11 cases (15\%) were between 41 and 60 years of age \cite{Al-Faham_2020k, Bhumi_2024f, CamachoDorado_2018, Emamhadi_2018, Hardy_2023g, Jehangir_2019h, Kumar_2001, Sultan_2024f, Thapa_2019f, Wadhwa_2015e, teWildt_2010}, 3 cases (4\%) were over 60 years of age \cite{Beecroft_1998, Kerestes_2019, Li_2013}, 2 cases (3\%) had no age documented \cite{Berry_2021e}. All 90 were male gender. 90 cases (100\%) were detained at the time of ingestion \cite{Elghali_2016, Karp_1991b, Lee_2007}, 88 cases (98\%) were intentional ingestions \cite{Elghali_2016, Karp_1991b, Lee_2007}, 30 cases (33\%) had a psychiatric history documented \cite{Elghali_2016, Karp_1991b, Lee_2007}, 2 cases (2\%) had a history of prior ingestion \cite{Elghali_2016}. No cases were reported for were psychiatric inpatients, were displaced people, were under the influence of alcohol at the time of ingestion, and had a severe disability history.
\paragraph*{Motivation}  70 cases (78\%) reported protest motivation \cite{Elghali_2016, Karp_1991b, Lee_2007}, 12 cases (13\%) reported psychiatric motivation \cite{Karp_1991b}, 6 cases (7\%) reported self-harm motivation \cite{Elghali_2016, Karp_1991b}. No cases were reported for psychosocial motivation and other motivation.
\paragraph*{Object Characteristics}  68 cases (76\%) involved sharp object ingestion \cite{Elghali_2016, Karp_1991b, Lee_2007}, 32 cases (36\%) involved long (\textgreater 5cm) object ingestion \cite{Lee_2007}, 25 cases (28\%) involved ingestion of multiple objects \cite{Elghali_2016, Lee_2007}. No cases were reported for button battery ingestion, magnet ingestion, and involved large diameter (\textgreater 2.5cm) object ingestion.
\paragraph*{Outcomes}  47 cases (52\%) underwent endoscopic intervention \cite{Elghali_2016, Lee_2007}, 29 cases (32\%) were managed conservatively \cite{Elghali_2016, Karp_1991b}, 15 cases (17\%) underwent surgical intervention \cite{Elghali_2016, Karp_1991b, Lee_2007}, 6 cases (7\%) reported complications \cite{Lee_2007}, 1 case (1\%) died \cite{Elghali_2016}.
\paragraph*{Gender} 43 cases (60\%) were male \cite{Akay_2015f, Al-Faham_2020k, Alao_2006i, Ali_2017, Ali_2022g, Apikotoa_2022f, Atayan_2016, Benoist_2019e, Berry_2021e, Bhumi_2024f, CamachoDorado_2018, Csaky_1998e, Emamhadi_2018, Farhadi_2024h, Fry_2010, Gardner_2017h, Guinan_2019f, Jehangir_2019h, Jin_2023, Kobiela_2015, Kumar_2001, Kumar_2019f, Liu_2005, Losanoff_1996, Losanoff_1997e, Mesfin_2022a, Misra_2013, Qureshi_2016, Riva_2018j, Sobnach_2011f, Tammana_2012j, Tanrikulu_2015e, Tay_2004, Thapa_2019f, Trgo_2012f, Wadhwa_2015e, Yasin_2009, teWildt_2010}, 28 cases (39\%) were female \cite{AlShaaibi_2021b, Ali_2020f, Ataya_2013, Beecroft_1998, Bhasin_2014, Bhattacharjee_2008, Cauchi_2002, Chang_2017f, Cox_2007, DelgadoSalazar_2020c, DivsalarP._2023a, Goldman_1998f, Hardy_2023g, Kar_2015, Kariholu_2008, Kerestes_2019, Li_2013, Naji_2012f, Ohno_2005, Peixoto_2017f, Sakellaridis_2008f, Sultan_2024f, Tupesis_2004f, Wildhaber_2005, Wnęk_2015f, Yildiz_2016e}, 1 case (1\%) had no gender recorded \cite{fjbuilsRepeatedBehaviorDeliberate2024}. \paragraph*{Age Group} 25 cases (35\%) were between 26 and 40 years of age \cite{Alao_2006i, Ali_2022g, Apikotoa_2022f, Ataya_2013, Benoist_2019e, Bhasin_2014, Chang_2017f, Cox_2007, DelgadoSalazar_2020c, Farhadi_2024h, Fry_2010, Gardner_2017h, Guinan_2019f, Jin_2023, Kumar_2019f, Losanoff_1996, Misra_2013, Qureshi_2016, Riva_2018j, Sakellaridis_2008f, Tammana_2012j, Trgo_2012f, Wnęk_2015f, Yildiz_2016e, fjbuilsRepeatedBehaviorDeliberate2024}, 18 cases (25\%) were between 18 and 25 years of age \cite{Akay_2015f, Ali_2017, Atayan_2016, Bhattacharjee_2008, Csaky_1998e, Kar_2015, Kariholu_2008, Kobiela_2015, Losanoff_1996, Losanoff_1997e, Mesfin_2022a, Peixoto_2017f, Sobnach_2011f, Tupesis_2004f, Yasin_2009}, 13 cases (18\%) were under 18 years of age \cite{AlShaaibi_2021b, Ali_2020f, Cauchi_2002, DivsalarP._2023a, Goldman_1998f, Liu_2005, Naji_2012f, Ohno_2005, Tanrikulu_2015e, Tay_2004, Wildhaber_2005}, 11 cases (15\%) were between 41 and 60 years of age \cite{Al-Faham_2020k, Bhumi_2024f, CamachoDorado_2018, Emamhadi_2018, Hardy_2023g, Jehangir_2019h, Kumar_2001, Sultan_2024f, Thapa_2019f, Wadhwa_2015e, teWildt_2010}, 3 cases (4\%) were over 60 years of age \cite{Beecroft_1998, Kerestes_2019, Li_2013}, 2 cases (3\%) had no age documented \cite{Berry_2021e}. All 90 were male gender. 90 cases (100\%) were detained at the time of ingestion \cite{Elghali_2016, Karp_1991b, Lee_2007}, 88 cases (98\%) were intentional ingestions \cite{Elghali_2016, Karp_1991b, Lee_2007}, 30 cases (33\%) had a psychiatric history documented \cite{Elghali_2016, Karp_1991b, Lee_2007}, 2 cases (2\%) had a history of prior ingestion \cite{Elghali_2016}. No cases were reported for were psychiatric inpatients, were displaced people, were under the influence of alcohol at the time of ingestion, and had a severe disability history.
\paragraph*{Motivation}  70 cases (78\%) reported protest motivation \cite{Elghali_2016, Karp_1991b, Lee_2007}, 12 cases (13\%) reported psychiatric motivation \cite{Karp_1991b}, 6 cases (7\%) reported self-harm motivation \cite{Elghali_2016, Karp_1991b}. No cases were reported for psychosocial motivation and other motivation.
\paragraph*{Object Characteristics}  68 cases (76\%) involved sharp object ingestion \cite{Elghali_2016, Karp_1991b, Lee_2007}, 32 cases (36\%) involved long (\textgreater 5cm) object ingestion \cite{Lee_2007}, 25 cases (28\%) involved ingestion of multiple objects \cite{Elghali_2016, Lee_2007}. No cases were reported for button battery ingestion, magnet ingestion, and involved large diameter (\textgreater 2.5cm) object ingestion.
\paragraph*{Outcomes}  47 cases (52\%) underwent endoscopic intervention \cite{Elghali_2016, Lee_2007}, 29 cases (32\%) were managed conservatively \cite{Elghali_2016, Karp_1991b}, 15 cases (17\%) underwent surgical intervention \cite{Elghali_2016, Karp_1991b, Lee_2007}, 6 cases (7\%) reported complications \cite{Lee_2007}, 1 case (1\%) died \cite{Elghali_2016}.
\paragraph*{Gender} 43 cases (60\%) were male \cite{Akay_2015f, Al-Faham_2020k, Alao_2006i, Ali_2017, Ali_2022g, Apikotoa_2022f, Atayan_2016, Benoist_2019e, Berry_2021e, Bhumi_2024f, CamachoDorado_2018, Csaky_1998e, Emamhadi_2018, Farhadi_2024h, Fry_2010, Gardner_2017h, Guinan_2019f, Jehangir_2019h, Jin_2023, Kobiela_2015, Kumar_2001, Kumar_2019f, Liu_2005, Losanoff_1996, Losanoff_1997e, Mesfin_2022a, Misra_2013, Qureshi_2016, Riva_2018j, Sobnach_2011f, Tammana_2012j, Tanrikulu_2015e, Tay_2004, Thapa_2019f, Trgo_2012f, Wadhwa_2015e, Yasin_2009, teWildt_2010}, 28 cases (39\%) were female \cite{AlShaaibi_2021b, Ali_2020f, Ataya_2013, Beecroft_1998, Bhasin_2014, Bhattacharjee_2008, Cauchi_2002, Chang_2017f, Cox_2007, DelgadoSalazar_2020c, DivsalarP._2023a, Goldman_1998f, Hardy_2023g, Kar_2015, Kariholu_2008, Kerestes_2019, Li_2013, Naji_2012f, Ohno_2005, Peixoto_2017f, Sakellaridis_2008f, Sultan_2024f, Tupesis_2004f, Wildhaber_2005, Wnęk_2015f, Yildiz_2016e}, 1 case (1\%) had no gender recorded \cite{fjbuilsRepeatedBehaviorDeliberate2024}. \paragraph*{Age Group} 25 cases (35\%) were between 26 and 40 years of age \cite{Alao_2006i, Ali_2022g, Apikotoa_2022f, Ataya_2013, Benoist_2019e, Bhasin_2014, Chang_2017f, Cox_2007, DelgadoSalazar_2020c, Farhadi_2024h, Fry_2010, Gardner_2017h, Guinan_2019f, Jin_2023, Kumar_2019f, Losanoff_1996, Misra_2013, Qureshi_2016, Riva_2018j, Sakellaridis_2008f, Tammana_2012j, Trgo_2012f, Wnęk_2015f, Yildiz_2016e, fjbuilsRepeatedBehaviorDeliberate2024}, 18 cases (25\%) were between 18 and 25 years of age \cite{Akay_2015f, Ali_2017, Atayan_2016, Bhattacharjee_2008, Csaky_1998e, Kar_2015, Kariholu_2008, Kobiela_2015, Losanoff_1996, Losanoff_1997e, Mesfin_2022a, Peixoto_2017f, Sobnach_2011f, Tupesis_2004f, Yasin_2009}, 13 cases (18\%) were under 18 years of age \cite{AlShaaibi_2021b, Ali_2020f, Cauchi_2002, DivsalarP._2023a, Goldman_1998f, Liu_2005, Naji_2012f, Ohno_2005, Tanrikulu_2015e, Tay_2004, Wildhaber_2005}, 11 cases (15\%) were between 41 and 60 years of age \cite{Al-Faham_2020k, Bhumi_2024f, CamachoDorado_2018, Emamhadi_2018, Hardy_2023g, Jehangir_2019h, Kumar_2001, Sultan_2024f, Thapa_2019f, Wadhwa_2015e, teWildt_2010}, 3 cases (4\%) were over 60 years of age \cite{Beecroft_1998, Kerestes_2019, Li_2013}, 2 cases (3\%) had no age documented \cite{Berry_2021e}. All 90 were male gender. 90 cases (100\%) were detained at the time of ingestion \cite{Elghali_2016, Karp_1991b, Lee_2007}, 88 cases (98\%) were intentional ingestions \cite{Elghali_2016, Karp_1991b, Lee_2007}, 30 cases (33\%) had a psychiatric history documented \cite{Elghali_2016, Karp_1991b, Lee_2007}, 2 cases (2\%) had a history of prior ingestion \cite{Elghali_2016}. No cases were reported for were psychiatric inpatients, were displaced people, were under the influence of alcohol at the time of ingestion, and had a severe disability history.
\paragraph*{Motivation}  70 cases (78\%) reported protest motivation \cite{Elghali_2016, Karp_1991b, Lee_2007}, 12 cases (13\%) reported psychiatric motivation \cite{Karp_1991b}, 6 cases (7\%) reported self-harm motivation \cite{Elghali_2016, Karp_1991b}. No cases were reported for psychosocial motivation and other motivation.
\paragraph*{Object Characteristics}  68 cases (76\%) involved sharp object ingestion \cite{Elghali_2016, Karp_1991b, Lee_2007}, 32 cases (36\%) involved long (\textgreater 5cm) object ingestion \cite{Lee_2007}, 25 cases (28\%) involved ingestion of multiple objects \cite{Elghali_2016, Lee_2007}. No cases were reported for button battery ingestion, magnet ingestion, and involved large diameter (\textgreater 2.5cm) object ingestion.
\paragraph*{Outcomes}  47 cases (52\%) underwent endoscopic intervention \cite{Elghali_2016, Lee_2007}, 29 cases (32\%) were managed conservatively \cite{Elghali_2016, Karp_1991b}, 15 cases (17\%) underwent surgical intervention \cite{Elghali_2016, Karp_1991b, Lee_2007}, 6 cases (7\%) reported complications \cite{Lee_2007}, 1 case (1\%) died \cite{Elghali_2016}.
\paragraph*{Gender} 43 cases (60\%) were male \cite{Akay_2015f, Al-Faham_2020k, Alao_2006i, Ali_2017, Ali_2022g, Apikotoa_2022f, Atayan_2016, Benoist_2019e, Berry_2021e, Bhumi_2024f, CamachoDorado_2018, Csaky_1998e, Emamhadi_2018, Farhadi_2024h, Fry_2010, Gardner_2017h, Guinan_2019f, Jehangir_2019h, Jin_2023, Kobiela_2015, Kumar_2001, Kumar_2019f, Liu_2005, Losanoff_1996, Losanoff_1997e, Mesfin_2022a, Misra_2013, Qureshi_2016, Riva_2018j, Sobnach_2011f, Tammana_2012j, Tanrikulu_2015e, Tay_2004, Thapa_2019f, Trgo_2012f, Wadhwa_2015e, Yasin_2009, teWildt_2010}, 28 cases (39\%) were female \cite{AlShaaibi_2021b, Ali_2020f, Ataya_2013, Beecroft_1998, Bhasin_2014, Bhattacharjee_2008, Cauchi_2002, Chang_2017f, Cox_2007, DelgadoSalazar_2020c, DivsalarP._2023a, Goldman_1998f, Hardy_2023g, Kar_2015, Kariholu_2008, Kerestes_2019, Li_2013, Naji_2012f, Ohno_2005, Peixoto_2017f, Sakellaridis_2008f, Sultan_2024f, Tupesis_2004f, Wildhaber_2005, Wnęk_2015f, Yildiz_2016e}, 1 case (1\%) had no gender recorded \cite{fjbuilsRepeatedBehaviorDeliberate2024}. \paragraph*{Age Group} 25 cases (35\%) were between 26 and 40 years of age \cite{Alao_2006i, Ali_2022g, Apikotoa_2022f, Ataya_2013, Benoist_2019e, Bhasin_2014, Chang_2017f, Cox_2007, DelgadoSalazar_2020c, Farhadi_2024h, Fry_2010, Gardner_2017h, Guinan_2019f, Jin_2023, Kumar_2019f, Losanoff_1996, Misra_2013, Qureshi_2016, Riva_2018j, Sakellaridis_2008f, Tammana_2012j, Trgo_2012f, Wnęk_2015f, Yildiz_2016e, fjbuilsRepeatedBehaviorDeliberate2024}, 18 cases (25\%) were between 18 and 25 years of age \cite{Akay_2015f, Ali_2017, Atayan_2016, Bhattacharjee_2008, Csaky_1998e, Kar_2015, Kariholu_2008, Kobiela_2015, Losanoff_1996, Losanoff_1997e, Mesfin_2022a, Peixoto_2017f, Sobnach_2011f, Tupesis_2004f, Yasin_2009}, 13 cases (18\%) were under 18 years of age \cite{AlShaaibi_2021b, Ali_2020f, Cauchi_2002, DivsalarP._2023a, Goldman_1998f, Liu_2005, Naji_2012f, Ohno_2005, Tanrikulu_2015e, Tay_2004, Wildhaber_2005}, 11 cases (15\%) were between 41 and 60 years of age \cite{Al-Faham_2020k, Bhumi_2024f, CamachoDorado_2018, Emamhadi_2018, Hardy_2023g, Jehangir_2019h, Kumar_2001, Sultan_2024f, Thapa_2019f, Wadhwa_2015e, teWildt_2010}, 3 cases (4\%) were over 60 years of age \cite{Beecroft_1998, Kerestes_2019, Li_2013}, 2 cases (3\%) had no age documented \cite{Berry_2021e}. All 90 were male gender. 90 cases (100\%) were detained at the time of ingestion \cite{Elghali_2016, Karp_1991b, Lee_2007}, 88 cases (98\%) were intentional ingestions \cite{Elghali_2016, Karp_1991b, Lee_2007}, 30 cases (33\%) had a psychiatric history documented \cite{Elghali_2016, Karp_1991b, Lee_2007}, 2 cases (2\%) had a history of prior ingestion \cite{Elghali_2016}. No cases were reported for were psychiatric inpatients, were displaced people, were under the influence of alcohol at the time of ingestion, and had a severe disability history.
\paragraph*{Motivation}  70 cases (78\%) reported protest motivation \cite{Elghali_2016, Karp_1991b, Lee_2007}, 12 cases (13\%) reported psychiatric motivation \cite{Karp_1991b}, 6 cases (7\%) reported self-harm motivation \cite{Elghali_2016, Karp_1991b}. No cases were reported for psychosocial motivation and other motivation.
\paragraph*{Object Characteristics}  68 cases (76\%) involved sharp object ingestion \cite{Elghali_2016, Karp_1991b, Lee_2007}, 32 cases (36\%) involved long (\textgreater 5cm) object ingestion \cite{Lee_2007}, 25 cases (28\%) involved ingestion of multiple objects \cite{Elghali_2016, Lee_2007}. No cases were reported for button battery ingestion, magnet ingestion, and involved large diameter (\textgreater 2.5cm) object ingestion.
\paragraph*{Outcomes}  47 cases (52\%) underwent endoscopic intervention \cite{Elghali_2016, Lee_2007}, 29 cases (32\%) were managed conservatively \cite{Elghali_2016, Karp_1991b}, 15 cases (17\%) underwent surgical intervention \cite{Elghali_2016, Karp_1991b, Lee_2007}, 6 cases (7\%) reported complications \cite{Lee_2007}, 1 case (1\%) died \cite{Elghali_2016}.
\paragraph*{Gender} 43 cases (60\%) were male \cite{Akay_2015f, Al-Faham_2020k, Alao_2006i, Ali_2017, Ali_2022g, Apikotoa_2022f, Atayan_2016, Benoist_2019e, Berry_2021e, Bhumi_2024f, CamachoDorado_2018, Csaky_1998e, Emamhadi_2018, Farhadi_2024h, Fry_2010, Gardner_2017h, Guinan_2019f, Jehangir_2019h, Jin_2023, Kobiela_2015, Kumar_2001, Kumar_2019f, Liu_2005, Losanoff_1996, Losanoff_1997e, Mesfin_2022a, Misra_2013, Qureshi_2016, Riva_2018j, Sobnach_2011f, Tammana_2012j, Tanrikulu_2015e, Tay_2004, Thapa_2019f, Trgo_2012f, Wadhwa_2015e, Yasin_2009, teWildt_2010}, 28 cases (39\%) were female \cite{AlShaaibi_2021b, Ali_2020f, Ataya_2013, Beecroft_1998, Bhasin_2014, Bhattacharjee_2008, Cauchi_2002, Chang_2017f, Cox_2007, DelgadoSalazar_2020c, DivsalarP._2023a, Goldman_1998f, Hardy_2023g, Kar_2015, Kariholu_2008, Kerestes_2019, Li_2013, Naji_2012f, Ohno_2005, Peixoto_2017f, Sakellaridis_2008f, Sultan_2024f, Tupesis_2004f, Wildhaber_2005, Wnęk_2015f, Yildiz_2016e}, 1 case (1\%) had no gender recorded \cite{fjbuilsRepeatedBehaviorDeliberate2024}. \paragraph*{Age Group} 25 cases (35\%) were between 26 and 40 years of age \cite{Alao_2006i, Ali_2022g, Apikotoa_2022f, Ataya_2013, Benoist_2019e, Bhasin_2014, Chang_2017f, Cox_2007, DelgadoSalazar_2020c, Farhadi_2024h, Fry_2010, Gardner_2017h, Guinan_2019f, Jin_2023, Kumar_2019f, Losanoff_1996, Misra_2013, Qureshi_2016, Riva_2018j, Sakellaridis_2008f, Tammana_2012j, Trgo_2012f, Wnęk_2015f, Yildiz_2016e, fjbuilsRepeatedBehaviorDeliberate2024}, 18 cases (25\%) were between 18 and 25 years of age \cite{Akay_2015f, Ali_2017, Atayan_2016, Bhattacharjee_2008, Csaky_1998e, Kar_2015, Kariholu_2008, Kobiela_2015, Losanoff_1996, Losanoff_1997e, Mesfin_2022a, Peixoto_2017f, Sobnach_2011f, Tupesis_2004f, Yasin_2009}, 13 cases (18\%) were under 18 years of age \cite{AlShaaibi_2021b, Ali_2020f, Cauchi_2002, DivsalarP._2023a, Goldman_1998f, Liu_2005, Naji_2012f, Ohno_2005, Tanrikulu_2015e, Tay_2004, Wildhaber_2005}, 11 cases (15\%) were between 41 and 60 years of age \cite{Al-Faham_2020k, Bhumi_2024f, CamachoDorado_2018, Emamhadi_2018, Hardy_2023g, Jehangir_2019h, Kumar_2001, Sultan_2024f, Thapa_2019f, Wadhwa_2015e, teWildt_2010}, 3 cases (4\%) were over 60 years of age \cite{Beecroft_1998, Kerestes_2019, Li_2013}, 2 cases (3\%) had no age documented \cite{Berry_2021e}. All 90 were male gender. 90 cases (100\%) were detained at the time of ingestion \cite{Elghali_2016, Karp_1991b, Lee_2007}, 88 cases (98\%) were intentional ingestions \cite{Elghali_2016, Karp_1991b, Lee_2007}, 30 cases (33\%) had a psychiatric history documented \cite{Elghali_2016, Karp_1991b, Lee_2007}, 2 cases (2\%) had a history of prior ingestion \cite{Elghali_2016}. No cases were reported for were psychiatric inpatients, were displaced people, were under the influence of alcohol at the time of ingestion, and had a severe disability history.
\paragraph*{Motivation}  70 cases (78\%) reported protest motivation \cite{Elghali_2016, Karp_1991b, Lee_2007}, 12 cases (13\%) reported psychiatric motivation \cite{Karp_1991b}, 6 cases (7\%) reported self-harm motivation \cite{Elghali_2016, Karp_1991b}. No cases were reported for psychosocial motivation and other motivation.
\paragraph*{Object Characteristics}  68 cases (76\%) involved sharp object ingestion \cite{Elghali_2016, Karp_1991b, Lee_2007}, 32 cases (36\%) involved long (\textgreater 5cm) object ingestion \cite{Lee_2007}, 25 cases (28\%) involved ingestion of multiple objects \cite{Elghali_2016, Lee_2007}. No cases were reported for button battery ingestion, magnet ingestion, and involved large diameter (\textgreater 2.5cm) object ingestion.
\paragraph*{Outcomes}  47 cases (52\%) underwent endoscopic intervention \cite{Elghali_2016, Lee_2007}, 29 cases (32\%) were managed conservatively \cite{Elghali_2016, Karp_1991b}, 15 cases (17\%) underwent surgical intervention \cite{Elghali_2016, Karp_1991b, Lee_2007}, 6 cases (7\%) reported complications \cite{Lee_2007}, 1 case (1\%) died \cite{Elghali_2016}.
\paragraph*{Gender} 43 cases (60\%) were male \cite{Akay_2015f, Al-Faham_2020k, Alao_2006i, Ali_2017, Ali_2022g, Apikotoa_2022f, Atayan_2016, Benoist_2019e, Berry_2021e, Bhumi_2024f, CamachoDorado_2018, Csaky_1998e, Emamhadi_2018, Farhadi_2024h, Fry_2010, Gardner_2017h, Guinan_2019f, Jehangir_2019h, Jin_2023, Kobiela_2015, Kumar_2001, Kumar_2019f, Liu_2005, Losanoff_1996, Losanoff_1997e, Mesfin_2022a, Misra_2013, Qureshi_2016, Riva_2018j, Sobnach_2011f, Tammana_2012j, Tanrikulu_2015e, Tay_2004, Thapa_2019f, Trgo_2012f, Wadhwa_2015e, Yasin_2009, teWildt_2010}, 28 cases (39\%) were female \cite{AlShaaibi_2021b, Ali_2020f, Ataya_2013, Beecroft_1998, Bhasin_2014, Bhattacharjee_2008, Cauchi_2002, Chang_2017f, Cox_2007, DelgadoSalazar_2020c, DivsalarP._2023a, Goldman_1998f, Hardy_2023g, Kar_2015, Kariholu_2008, Kerestes_2019, Li_2013, Naji_2012f, Ohno_2005, Peixoto_2017f, Sakellaridis_2008f, Sultan_2024f, Tupesis_2004f, Wildhaber_2005, Wnęk_2015f, Yildiz_2016e}, 1 case (1\%) had no gender recorded \cite{fjbuilsRepeatedBehaviorDeliberate2024}. \paragraph*{Age Group} 25 cases (35\%) were between 26 and 40 years of age \cite{Alao_2006i, Ali_2022g, Apikotoa_2022f, Ataya_2013, Benoist_2019e, Bhasin_2014, Chang_2017f, Cox_2007, DelgadoSalazar_2020c, Farhadi_2024h, Fry_2010, Gardner_2017h, Guinan_2019f, Jin_2023, Kumar_2019f, Losanoff_1996, Misra_2013, Qureshi_2016, Riva_2018j, Sakellaridis_2008f, Tammana_2012j, Trgo_2012f, Wnęk_2015f, Yildiz_2016e, fjbuilsRepeatedBehaviorDeliberate2024}, 18 cases (25\%) were between 18 and 25 years of age \cite{Akay_2015f, Ali_2017, Atayan_2016, Bhattacharjee_2008, Csaky_1998e, Kar_2015, Kariholu_2008, Kobiela_2015, Losanoff_1996, Losanoff_1997e, Mesfin_2022a, Peixoto_2017f, Sobnach_2011f, Tupesis_2004f, Yasin_2009}, 13 cases (18\%) were under 18 years of age \cite{AlShaaibi_2021b, Ali_2020f, Cauchi_2002, DivsalarP._2023a, Goldman_1998f, Liu_2005, Naji_2012f, Ohno_2005, Tanrikulu_2015e, Tay_2004, Wildhaber_2005}, 11 cases (15\%) were between 41 and 60 years of age \cite{Al-Faham_2020k, Bhumi_2024f, CamachoDorado_2018, Emamhadi_2018, Hardy_2023g, Jehangir_2019h, Kumar_2001, Sultan_2024f, Thapa_2019f, Wadhwa_2015e, teWildt_2010}, 3 cases (4\%) were over 60 years of age \cite{Beecroft_1998, Kerestes_2019, Li_2013}, 2 cases (3\%) had no age documented \cite{Berry_2021e}. All 90 were male gender. 90 cases (100\%) were detained at the time of ingestion \cite{Elghali_2016, Karp_1991b, Lee_2007}, 88 cases (98\%) were intentional ingestions \cite{Elghali_2016, Karp_1991b, Lee_2007}, 30 cases (33\%) had a psychiatric history documented \cite{Elghali_2016, Karp_1991b, Lee_2007}, 2 cases (2\%) had a history of prior ingestion \cite{Elghali_2016}. No cases were reported for were psychiatric inpatients, were displaced people, were under the influence of alcohol at the time of ingestion, and had a severe disability history.
\paragraph*{Motivation}  70 cases (78\%) reported protest motivation \cite{Elghali_2016, Karp_1991b, Lee_2007}, 12 cases (13\%) reported psychiatric motivation \cite{Karp_1991b}, 6 cases (7\%) reported self-harm motivation \cite{Elghali_2016, Karp_1991b}. No cases were reported for psychosocial motivation and other motivation.
\paragraph*{Object Characteristics}  68 cases (76\%) involved sharp object ingestion \cite{Elghali_2016, Karp_1991b, Lee_2007}, 32 cases (36\%) involved long (\textgreater 5cm) object ingestion \cite{Lee_2007}, 25 cases (28\%) involved ingestion of multiple objects \cite{Elghali_2016, Lee_2007}. No cases were reported for button battery ingestion, magnet ingestion, and involved large diameter (\textgreater 2.5cm) object ingestion.
\paragraph*{Outcomes}  47 cases (52\%) underwent endoscopic intervention \cite{Elghali_2016, Lee_2007}, 29 cases (32\%) were managed conservatively \cite{Elghali_2016, Karp_1991b}, 15 cases (17\%) underwent surgical intervention \cite{Elghali_2016, Karp_1991b, Lee_2007}, 6 cases (7\%) reported complications \cite{Lee_2007}, 1 case (1\%) died \cite{Elghali_2016}.
\paragraph*{Gender} 43 cases (60\%) were male \cite{Akay_2015f, Al-Faham_2020k, Alao_2006i, Ali_2017, Ali_2022g, Apikotoa_2022f, Atayan_2016, Benoist_2019e, Berry_2021e, Bhumi_2024f, CamachoDorado_2018, Csaky_1998e, Emamhadi_2018, Farhadi_2024h, Fry_2010, Gardner_2017h, Guinan_2019f, Jehangir_2019h, Jin_2023, Kobiela_2015, Kumar_2001, Kumar_2019f, Liu_2005, Losanoff_1996, Losanoff_1997e, Mesfin_2022a, Misra_2013, Qureshi_2016, Riva_2018j, Sobnach_2011f, Tammana_2012j, Tanrikulu_2015e, Tay_2004, Thapa_2019f, Trgo_2012f, Wadhwa_2015e, Yasin_2009, teWildt_2010}, 28 cases (39\%) were female \cite{AlShaaibi_2021b, Ali_2020f, Ataya_2013, Beecroft_1998, Bhasin_2014, Bhattacharjee_2008, Cauchi_2002, Chang_2017f, Cox_2007, DelgadoSalazar_2020c, DivsalarP._2023a, Goldman_1998f, Hardy_2023g, Kar_2015, Kariholu_2008, Kerestes_2019, Li_2013, Naji_2012f, Ohno_2005, Peixoto_2017f, Sakellaridis_2008f, Sultan_2024f, Tupesis_2004f, Wildhaber_2005, Wnęk_2015f, Yildiz_2016e}, 1 case (1\%) had no gender recorded \cite{fjbuilsRepeatedBehaviorDeliberate2024}. \paragraph*{Age Group} 25 cases (35\%) were between 26 and 40 years of age \cite{Alao_2006i, Ali_2022g, Apikotoa_2022f, Ataya_2013, Benoist_2019e, Bhasin_2014, Chang_2017f, Cox_2007, DelgadoSalazar_2020c, Farhadi_2024h, Fry_2010, Gardner_2017h, Guinan_2019f, Jin_2023, Kumar_2019f, Losanoff_1996, Misra_2013, Qureshi_2016, Riva_2018j, Sakellaridis_2008f, Tammana_2012j, Trgo_2012f, Wnęk_2015f, Yildiz_2016e, fjbuilsRepeatedBehaviorDeliberate2024}, 18 cases (25\%) were between 18 and 25 years of age \cite{Akay_2015f, Ali_2017, Atayan_2016, Bhattacharjee_2008, Csaky_1998e, Kar_2015, Kariholu_2008, Kobiela_2015, Losanoff_1996, Losanoff_1997e, Mesfin_2022a, Peixoto_2017f, Sobnach_2011f, Tupesis_2004f, Yasin_2009}, 13 cases (18\%) were under 18 years of age \cite{AlShaaibi_2021b, Ali_2020f, Cauchi_2002, DivsalarP._2023a, Goldman_1998f, Liu_2005, Naji_2012f, Ohno_2005, Tanrikulu_2015e, Tay_2004, Wildhaber_2005}, 11 cases (15\%) were between 41 and 60 years of age \cite{Al-Faham_2020k, Bhumi_2024f, CamachoDorado_2018, Emamhadi_2018, Hardy_2023g, Jehangir_2019h, Kumar_2001, Sultan_2024f, Thapa_2019f, Wadhwa_2015e, teWildt_2010}, 3 cases (4\%) were over 60 years of age \cite{Beecroft_1998, Kerestes_2019, Li_2013}, 2 cases (3\%) had no age documented \cite{Berry_2021e}. All 90 were male gender. 90 cases (100\%) were detained at the time of ingestion \cite{Elghali_2016, Karp_1991b, Lee_2007}, 88 cases (98\%) were intentional ingestions \cite{Elghali_2016, Karp_1991b, Lee_2007}, 30 cases (33\%) had a psychiatric history documented \cite{Elghali_2016, Karp_1991b, Lee_2007}, 2 cases (2\%) had a history of prior ingestion \cite{Elghali_2016}. No cases were reported for were psychiatric inpatients, were displaced people, were under the influence of alcohol at the time of ingestion, and had a severe disability history.
\paragraph*{Motivation}  70 cases (78\%) reported protest motivation \cite{Elghali_2016, Karp_1991b, Lee_2007}, 12 cases (13\%) reported psychiatric motivation \cite{Karp_1991b}, 6 cases (7\%) reported self-harm motivation \cite{Elghali_2016, Karp_1991b}. No cases were reported for psychosocial motivation and other motivation.
\paragraph*{Object Characteristics}  68 cases (76\%) involved sharp object ingestion \cite{Elghali_2016, Karp_1991b, Lee_2007}, 32 cases (36\%) involved long (\textgreater 5cm) object ingestion \cite{Lee_2007}, 25 cases (28\%) involved ingestion of multiple objects \cite{Elghali_2016, Lee_2007}. No cases were reported for button battery ingestion, magnet ingestion, and involved large diameter (\textgreater 2.5cm) object ingestion.
\paragraph*{Outcomes}  47 cases (52\%) underwent endoscopic intervention \cite{Elghali_2016, Lee_2007}, 29 cases (32\%) were managed conservatively \cite{Elghali_2016, Karp_1991b}, 15 cases (17\%) underwent surgical intervention \cite{Elghali_2016, Karp_1991b, Lee_2007}, 6 cases (7\%) reported complications \cite{Lee_2007}, 1 case (1\%) died \cite{Elghali_2016}.
\paragraph*{Gender} 43 cases (60\%) were male \cite{Akay_2015f, Al-Faham_2020k, Alao_2006i, Ali_2017, Ali_2022g, Apikotoa_2022f, Atayan_2016, Benoist_2019e, Berry_2021e, Bhumi_2024f, CamachoDorado_2018, Csaky_1998e, Emamhadi_2018, Farhadi_2024h, Fry_2010, Gardner_2017h, Guinan_2019f, Jehangir_2019h, Jin_2023, Kobiela_2015, Kumar_2001, Kumar_2019f, Liu_2005, Losanoff_1996, Losanoff_1997e, Mesfin_2022a, Misra_2013, Qureshi_2016, Riva_2018j, Sobnach_2011f, Tammana_2012j, Tanrikulu_2015e, Tay_2004, Thapa_2019f, Trgo_2012f, Wadhwa_2015e, Yasin_2009, teWildt_2010}, 28 cases (39\%) were female \cite{AlShaaibi_2021b, Ali_2020f, Ataya_2013, Beecroft_1998, Bhasin_2014, Bhattacharjee_2008, Cauchi_2002, Chang_2017f, Cox_2007, DelgadoSalazar_2020c, DivsalarP._2023a, Goldman_1998f, Hardy_2023g, Kar_2015, Kariholu_2008, Kerestes_2019, Li_2013, Naji_2012f, Ohno_2005, Peixoto_2017f, Sakellaridis_2008f, Sultan_2024f, Tupesis_2004f, Wildhaber_2005, Wnęk_2015f, Yildiz_2016e}, 1 case (1\%) had no gender recorded \cite{fjbuilsRepeatedBehaviorDeliberate2024}. \paragraph*{Age Group} 25 cases (35\%) were between 26 and 40 years of age \cite{Alao_2006i, Ali_2022g, Apikotoa_2022f, Ataya_2013, Benoist_2019e, Bhasin_2014, Chang_2017f, Cox_2007, DelgadoSalazar_2020c, Farhadi_2024h, Fry_2010, Gardner_2017h, Guinan_2019f, Jin_2023, Kumar_2019f, Losanoff_1996, Misra_2013, Qureshi_2016, Riva_2018j, Sakellaridis_2008f, Tammana_2012j, Trgo_2012f, Wnęk_2015f, Yildiz_2016e, fjbuilsRepeatedBehaviorDeliberate2024}, 18 cases (25\%) were between 18 and 25 years of age \cite{Akay_2015f, Ali_2017, Atayan_2016, Bhattacharjee_2008, Csaky_1998e, Kar_2015, Kariholu_2008, Kobiela_2015, Losanoff_1996, Losanoff_1997e, Mesfin_2022a, Peixoto_2017f, Sobnach_2011f, Tupesis_2004f, Yasin_2009}, 13 cases (18\%) were under 18 years of age \cite{AlShaaibi_2021b, Ali_2020f, Cauchi_2002, DivsalarP._2023a, Goldman_1998f, Liu_2005, Naji_2012f, Ohno_2005, Tanrikulu_2015e, Tay_2004, Wildhaber_2005}, 11 cases (15\%) were between 41 and 60 years of age \cite{Al-Faham_2020k, Bhumi_2024f, CamachoDorado_2018, Emamhadi_2018, Hardy_2023g, Jehangir_2019h, Kumar_2001, Sultan_2024f, Thapa_2019f, Wadhwa_2015e, teWildt_2010}, 3 cases (4\%) were over 60 years of age \cite{Beecroft_1998, Kerestes_2019, Li_2013}, 2 cases (3\%) had no age documented \cite{Berry_2021e}. All 90 were male gender. 90 cases (100\%) were detained at the time of ingestion \cite{Elghali_2016, Karp_1991b, Lee_2007}, 88 cases (98\%) were intentional ingestions \cite{Elghali_2016, Karp_1991b, Lee_2007}, 30 cases (33\%) had a psychiatric history documented \cite{Elghali_2016, Karp_1991b, Lee_2007}, 2 cases (2\%) had a history of prior ingestion \cite{Elghali_2016}. No cases were reported for were psychiatric inpatients, were displaced people, were under the influence of alcohol at the time of ingestion, and had a severe disability history.
\paragraph*{Motivation}  70 cases (78\%) reported protest motivation \cite{Elghali_2016, Karp_1991b, Lee_2007}, 12 cases (13\%) reported psychiatric motivation \cite{Karp_1991b}, 6 cases (7\%) reported self-harm motivation \cite{Elghali_2016, Karp_1991b}. No cases were reported for psychosocial motivation and other motivation.
\paragraph*{Object Characteristics}  68 cases (76\%) involved sharp object ingestion \cite{Elghali_2016, Karp_1991b, Lee_2007}, 32 cases (36\%) involved long (\textgreater 5cm) object ingestion \cite{Lee_2007}, 25 cases (28\%) involved ingestion of multiple objects \cite{Elghali_2016, Lee_2007}. No cases were reported for button battery ingestion, magnet ingestion, and involved large diameter (\textgreater 2.5cm) object ingestion.
\paragraph*{Outcomes}  47 cases (52\%) underwent endoscopic intervention \cite{Elghali_2016, Lee_2007}, 29 cases (32\%) were managed conservatively \cite{Elghali_2016, Karp_1991b}, 15 cases (17\%) underwent surgical intervention \cite{Elghali_2016, Karp_1991b, Lee_2007}, 6 cases (7\%) reported complications \cite{Lee_2007}, 1 case (1\%) died \cite{Elghali_2016}.
\paragraph*{Gender} 43 cases (60\%) were male \cite{Akay_2015f, Al-Faham_2020k, Alao_2006i, Ali_2017, Ali_2022g, Apikotoa_2022f, Atayan_2016, Benoist_2019e, Berry_2021e, Bhumi_2024f, CamachoDorado_2018, Csaky_1998e, Emamhadi_2018, Farhadi_2024h, Fry_2010, Gardner_2017h, Guinan_2019f, Jehangir_2019h, Jin_2023, Kobiela_2015, Kumar_2001, Kumar_2019f, Liu_2005, Losanoff_1996, Losanoff_1997e, Mesfin_2022a, Misra_2013, Qureshi_2016, Riva_2018j, Sobnach_2011f, Tammana_2012j, Tanrikulu_2015e, Tay_2004, Thapa_2019f, Trgo_2012f, Wadhwa_2015e, Yasin_2009, teWildt_2010}, 28 cases (39\%) were female \cite{AlShaaibi_2021b, Ali_2020f, Ataya_2013, Beecroft_1998, Bhasin_2014, Bhattacharjee_2008, Cauchi_2002, Chang_2017f, Cox_2007, DelgadoSalazar_2020c, DivsalarP._2023a, Goldman_1998f, Hardy_2023g, Kar_2015, Kariholu_2008, Kerestes_2019, Li_2013, Naji_2012f, Ohno_2005, Peixoto_2017f, Sakellaridis_2008f, Sultan_2024f, Tupesis_2004f, Wildhaber_2005, Wnęk_2015f, Yildiz_2016e}, 1 case (1\%) had no gender recorded \cite{fjbuilsRepeatedBehaviorDeliberate2024}. \paragraph*{Age Group} 25 cases (35\%) were between 26 and 40 years of age \cite{Alao_2006i, Ali_2022g, Apikotoa_2022f, Ataya_2013, Benoist_2019e, Bhasin_2014, Chang_2017f, Cox_2007, DelgadoSalazar_2020c, Farhadi_2024h, Fry_2010, Gardner_2017h, Guinan_2019f, Jin_2023, Kumar_2019f, Losanoff_1996, Misra_2013, Qureshi_2016, Riva_2018j, Sakellaridis_2008f, Tammana_2012j, Trgo_2012f, Wnęk_2015f, Yildiz_2016e, fjbuilsRepeatedBehaviorDeliberate2024}, 18 cases (25\%) were between 18 and 25 years of age \cite{Akay_2015f, Ali_2017, Atayan_2016, Bhattacharjee_2008, Csaky_1998e, Kar_2015, Kariholu_2008, Kobiela_2015, Losanoff_1996, Losanoff_1997e, Mesfin_2022a, Peixoto_2017f, Sobnach_2011f, Tupesis_2004f, Yasin_2009}, 13 cases (18\%) were under 18 years of age \cite{AlShaaibi_2021b, Ali_2020f, Cauchi_2002, DivsalarP._2023a, Goldman_1998f, Liu_2005, Naji_2012f, Ohno_2005, Tanrikulu_2015e, Tay_2004, Wildhaber_2005}, 11 cases (15\%) were between 41 and 60 years of age \cite{Al-Faham_2020k, Bhumi_2024f, CamachoDorado_2018, Emamhadi_2018, Hardy_2023g, Jehangir_2019h, Kumar_2001, Sultan_2024f, Thapa_2019f, Wadhwa_2015e, teWildt_2010}, 3 cases (4\%) were over 60 years of age \cite{Beecroft_1998, Kerestes_2019, Li_2013}, 2 cases (3\%) had no age documented \cite{Berry_2021e}. All 90 were male gender. 90 cases (100\%) were detained at the time of ingestion \cite{Elghali_2016, Karp_1991b, Lee_2007}, 88 cases (98\%) were intentional ingestions \cite{Elghali_2016, Karp_1991b, Lee_2007}, 30 cases (33\%) had a psychiatric history documented \cite{Elghali_2016, Karp_1991b, Lee_2007}, 2 cases (2\%) had a history of prior ingestion \cite{Elghali_2016}. No cases were reported for were psychiatric inpatients, were displaced people, were under the influence of alcohol at the time of ingestion, and had a severe disability history.
\paragraph*{Motivation}  70 cases (78\%) reported protest motivation \cite{Elghali_2016, Karp_1991b, Lee_2007}, 12 cases (13\%) reported psychiatric motivation \cite{Karp_1991b}, 6 cases (7\%) reported self-harm motivation \cite{Elghali_2016, Karp_1991b}. No cases were reported for psychosocial motivation and other motivation.
\paragraph*{Object Characteristics}  68 cases (76\%) involved sharp object ingestion \cite{Elghali_2016, Karp_1991b, Lee_2007}, 32 cases (36\%) involved long (\textgreater 5cm) object ingestion \cite{Lee_2007}, 25 cases (28\%) involved ingestion of multiple objects \cite{Elghali_2016, Lee_2007}. No cases were reported for button battery ingestion, magnet ingestion, and involved large diameter (\textgreater 2.5cm) object ingestion.
\paragraph*{Outcomes}  47 cases (52\%) underwent endoscopic intervention \cite{Elghali_2016, Lee_2007}, 29 cases (32\%) were managed conservatively \cite{Elghali_2016, Karp_1991b}, 15 cases (17\%) underwent surgical intervention \cite{Elghali_2016, Karp_1991b, Lee_2007}, 6 cases (7\%) reported complications \cite{Lee_2007}, 1 case (1\%) died \cite{Elghali_2016}.
\paragraph*{Gender} 43 cases (60\%) were male \cite{Akay_2015f, Al-Faham_2020k, Alao_2006i, Ali_2017, Ali_2022g, Apikotoa_2022f, Atayan_2016, Benoist_2019e, Berry_2021e, Bhumi_2024f, CamachoDorado_2018, Csaky_1998e, Emamhadi_2018, Farhadi_2024h, Fry_2010, Gardner_2017h, Guinan_2019f, Jehangir_2019h, Jin_2023, Kobiela_2015, Kumar_2001, Kumar_2019f, Liu_2005, Losanoff_1996, Losanoff_1997e, Mesfin_2022a, Misra_2013, Qureshi_2016, Riva_2018j, Sobnach_2011f, Tammana_2012j, Tanrikulu_2015e, Tay_2004, Thapa_2019f, Trgo_2012f, Wadhwa_2015e, Yasin_2009, teWildt_2010}, 28 cases (39\%) were female \cite{AlShaaibi_2021b, Ali_2020f, Ataya_2013, Beecroft_1998, Bhasin_2014, Bhattacharjee_2008, Cauchi_2002, Chang_2017f, Cox_2007, DelgadoSalazar_2020c, DivsalarP._2023a, Goldman_1998f, Hardy_2023g, Kar_2015, Kariholu_2008, Kerestes_2019, Li_2013, Naji_2012f, Ohno_2005, Peixoto_2017f, Sakellaridis_2008f, Sultan_2024f, Tupesis_2004f, Wildhaber_2005, Wnęk_2015f, Yildiz_2016e}, 1 case (1\%) had no gender recorded \cite{fjbuilsRepeatedBehaviorDeliberate2024}. \paragraph*{Age Group} 25 cases (35\%) were between 26 and 40 years of age \cite{Alao_2006i, Ali_2022g, Apikotoa_2022f, Ataya_2013, Benoist_2019e, Bhasin_2014, Chang_2017f, Cox_2007, DelgadoSalazar_2020c, Farhadi_2024h, Fry_2010, Gardner_2017h, Guinan_2019f, Jin_2023, Kumar_2019f, Losanoff_1996, Misra_2013, Qureshi_2016, Riva_2018j, Sakellaridis_2008f, Tammana_2012j, Trgo_2012f, Wnęk_2015f, Yildiz_2016e, fjbuilsRepeatedBehaviorDeliberate2024}, 18 cases (25\%) were between 18 and 25 years of age \cite{Akay_2015f, Ali_2017, Atayan_2016, Bhattacharjee_2008, Csaky_1998e, Kar_2015, Kariholu_2008, Kobiela_2015, Losanoff_1996, Losanoff_1997e, Mesfin_2022a, Peixoto_2017f, Sobnach_2011f, Tupesis_2004f, Yasin_2009}, 13 cases (18\%) were under 18 years of age \cite{AlShaaibi_2021b, Ali_2020f, Cauchi_2002, DivsalarP._2023a, Goldman_1998f, Liu_2005, Naji_2012f, Ohno_2005, Tanrikulu_2015e, Tay_2004, Wildhaber_2005}, 11 cases (15\%) were between 41 and 60 years of age \cite{Al-Faham_2020k, Bhumi_2024f, CamachoDorado_2018, Emamhadi_2018, Hardy_2023g, Jehangir_2019h, Kumar_2001, Sultan_2024f, Thapa_2019f, Wadhwa_2015e, teWildt_2010}, 3 cases (4\%) were over 60 years of age \cite{Beecroft_1998, Kerestes_2019, Li_2013}, 2 cases (3\%) had no age documented \cite{Berry_2021e}. All 90 were male gender. 90 cases (100\%) were detained at the time of ingestion \cite{Elghali_2016, Karp_1991b, Lee_2007}, 88 cases (98\%) were intentional ingestions \cite{Elghali_2016, Karp_1991b, Lee_2007}, 30 cases (33\%) had a psychiatric history documented \cite{Elghali_2016, Karp_1991b, Lee_2007}, 2 cases (2\%) had a history of prior ingestion \cite{Elghali_2016}. No cases were reported for were psychiatric inpatients, were displaced people, were under the influence of alcohol at the time of ingestion, and had a severe disability history.
\paragraph*{Motivation}  70 cases (78\%) reported protest motivation \cite{Elghali_2016, Karp_1991b, Lee_2007}, 12 cases (13\%) reported psychiatric motivation \cite{Karp_1991b}, 6 cases (7\%) reported self-harm motivation \cite{Elghali_2016, Karp_1991b}. No cases were reported for psychosocial motivation and other motivation.
\paragraph*{Object Characteristics}  68 cases (76\%) involved sharp object ingestion \cite{Elghali_2016, Karp_1991b, Lee_2007}, 32 cases (36\%) involved long (\textgreater 5cm) object ingestion \cite{Lee_2007}, 25 cases (28\%) involved ingestion of multiple objects \cite{Elghali_2016, Lee_2007}. No cases were reported for button battery ingestion, magnet ingestion, and involved large diameter (\textgreater 2.5cm) object ingestion.
\paragraph*{Outcomes}  47 cases (52\%) underwent endoscopic intervention \cite{Elghali_2016, Lee_2007}, 29 cases (32\%) were managed conservatively \cite{Elghali_2016, Karp_1991b}, 15 cases (17\%) underwent surgical intervention \cite{Elghali_2016, Karp_1991b, Lee_2007}, 6 cases (7\%) reported complications \cite{Lee_2007}, 1 case (1\%) died \cite{Elghali_2016}.
\paragraph*{Gender} 43 cases (60\%) were male \cite{Akay_2015f, Al-Faham_2020k, Alao_2006i, Ali_2017, Ali_2022g, Apikotoa_2022f, Atayan_2016, Benoist_2019e, Berry_2021e, Bhumi_2024f, CamachoDorado_2018, Csaky_1998e, Emamhadi_2018, Farhadi_2024h, Fry_2010, Gardner_2017h, Guinan_2019f, Jehangir_2019h, Jin_2023, Kobiela_2015, Kumar_2001, Kumar_2019f, Liu_2005, Losanoff_1996, Losanoff_1997e, Mesfin_2022a, Misra_2013, Qureshi_2016, Riva_2018j, Sobnach_2011f, Tammana_2012j, Tanrikulu_2015e, Tay_2004, Thapa_2019f, Trgo_2012f, Wadhwa_2015e, Yasin_2009, teWildt_2010}, 28 cases (39\%) were female \cite{AlShaaibi_2021b, Ali_2020f, Ataya_2013, Beecroft_1998, Bhasin_2014, Bhattacharjee_2008, Cauchi_2002, Chang_2017f, Cox_2007, DelgadoSalazar_2020c, DivsalarP._2023a, Goldman_1998f, Hardy_2023g, Kar_2015, Kariholu_2008, Kerestes_2019, Li_2013, Naji_2012f, Ohno_2005, Peixoto_2017f, Sakellaridis_2008f, Sultan_2024f, Tupesis_2004f, Wildhaber_2005, Wnęk_2015f, Yildiz_2016e}, 1 case (1\%) had no gender recorded \cite{fjbuilsRepeatedBehaviorDeliberate2024}. \paragraph*{Age Group} 25 cases (35\%) were between 26 and 40 years of age \cite{Alao_2006i, Ali_2022g, Apikotoa_2022f, Ataya_2013, Benoist_2019e, Bhasin_2014, Chang_2017f, Cox_2007, DelgadoSalazar_2020c, Farhadi_2024h, Fry_2010, Gardner_2017h, Guinan_2019f, Jin_2023, Kumar_2019f, Losanoff_1996, Misra_2013, Qureshi_2016, Riva_2018j, Sakellaridis_2008f, Tammana_2012j, Trgo_2012f, Wnęk_2015f, Yildiz_2016e, fjbuilsRepeatedBehaviorDeliberate2024}, 18 cases (25\%) were between 18 and 25 years of age \cite{Akay_2015f, Ali_2017, Atayan_2016, Bhattacharjee_2008, Csaky_1998e, Kar_2015, Kariholu_2008, Kobiela_2015, Losanoff_1996, Losanoff_1997e, Mesfin_2022a, Peixoto_2017f, Sobnach_2011f, Tupesis_2004f, Yasin_2009}, 13 cases (18\%) were under 18 years of age \cite{AlShaaibi_2021b, Ali_2020f, Cauchi_2002, DivsalarP._2023a, Goldman_1998f, Liu_2005, Naji_2012f, Ohno_2005, Tanrikulu_2015e, Tay_2004, Wildhaber_2005}, 11 cases (15\%) were between 41 and 60 years of age \cite{Al-Faham_2020k, Bhumi_2024f, CamachoDorado_2018, Emamhadi_2018, Hardy_2023g, Jehangir_2019h, Kumar_2001, Sultan_2024f, Thapa_2019f, Wadhwa_2015e, teWildt_2010}, 3 cases (4\%) were over 60 years of age \cite{Beecroft_1998, Kerestes_2019, Li_2013}, 2 cases (3\%) had no age documented \cite{Berry_2021e}. All 90 were male gender. 90 cases (100\%) were detained at the time of ingestion \cite{Elghali_2016, Karp_1991b, Lee_2007}, 88 cases (98\%) were intentional ingestions \cite{Elghali_2016, Karp_1991b, Lee_2007}, 30 cases (33\%) had a psychiatric history documented \cite{Elghali_2016, Karp_1991b, Lee_2007}, 2 cases (2\%) had a history of prior ingestion \cite{Elghali_2016}. No cases were reported for were psychiatric inpatients, were displaced people, were under the influence of alcohol at the time of ingestion, and had a severe disability history.
\paragraph*{Motivation}  70 cases (78\%) reported protest motivation \cite{Elghali_2016, Karp_1991b, Lee_2007}, 12 cases (13\%) reported psychiatric motivation \cite{Karp_1991b}, 6 cases (7\%) reported self-harm motivation \cite{Elghali_2016, Karp_1991b}. No cases were reported for psychosocial motivation and other motivation.
\paragraph*{Object Characteristics}  68 cases (76\%) involved sharp object ingestion \cite{Elghali_2016, Karp_1991b, Lee_2007}, 32 cases (36\%) involved long (\textgreater 5cm) object ingestion \cite{Lee_2007}, 25 cases (28\%) involved ingestion of multiple objects \cite{Elghali_2016, Lee_2007}. No cases were reported for button battery ingestion, magnet ingestion, and involved large diameter (\textgreater 2.5cm) object ingestion.
\paragraph*{Outcomes}  47 cases (52\%) underwent endoscopic intervention \cite{Elghali_2016, Lee_2007}, 29 cases (32\%) were managed conservatively \cite{Elghali_2016, Karp_1991b}, 15 cases (17\%) underwent surgical intervention \cite{Elghali_2016, Karp_1991b, Lee_2007}, 6 cases (7\%) reported complications \cite{Lee_2007}, 1 case (1\%) died \cite{Elghali_2016}.
\paragraph*{Gender} 43 cases (60\%) were male \cite{Akay_2015f, Al-Faham_2020k, Alao_2006i, Ali_2017, Ali_2022g, Apikotoa_2022f, Atayan_2016, Benoist_2019e, Berry_2021e, Bhumi_2024f, CamachoDorado_2018, Csaky_1998e, Emamhadi_2018, Farhadi_2024h, Fry_2010, Gardner_2017h, Guinan_2019f, Jehangir_2019h, Jin_2023, Kobiela_2015, Kumar_2001, Kumar_2019f, Liu_2005, Losanoff_1996, Losanoff_1997e, Mesfin_2022a, Misra_2013, Qureshi_2016, Riva_2018j, Sobnach_2011f, Tammana_2012j, Tanrikulu_2015e, Tay_2004, Thapa_2019f, Trgo_2012f, Wadhwa_2015e, Yasin_2009, teWildt_2010}, 28 cases (39\%) were female \cite{AlShaaibi_2021b, Ali_2020f, Ataya_2013, Beecroft_1998, Bhasin_2014, Bhattacharjee_2008, Cauchi_2002, Chang_2017f, Cox_2007, DelgadoSalazar_2020c, DivsalarP._2023a, Goldman_1998f, Hardy_2023g, Kar_2015, Kariholu_2008, Kerestes_2019, Li_2013, Naji_2012f, Ohno_2005, Peixoto_2017f, Sakellaridis_2008f, Sultan_2024f, Tupesis_2004f, Wildhaber_2005, Wnęk_2015f, Yildiz_2016e}, 1 case (1\%) had no gender recorded \cite{fjbuilsRepeatedBehaviorDeliberate2024}. \paragraph*{Age Group} 25 cases (35\%) were between 26 and 40 years of age \cite{Alao_2006i, Ali_2022g, Apikotoa_2022f, Ataya_2013, Benoist_2019e, Bhasin_2014, Chang_2017f, Cox_2007, DelgadoSalazar_2020c, Farhadi_2024h, Fry_2010, Gardner_2017h, Guinan_2019f, Jin_2023, Kumar_2019f, Losanoff_1996, Misra_2013, Qureshi_2016, Riva_2018j, Sakellaridis_2008f, Tammana_2012j, Trgo_2012f, Wnęk_2015f, Yildiz_2016e, fjbuilsRepeatedBehaviorDeliberate2024}, 18 cases (25\%) were between 18 and 25 years of age \cite{Akay_2015f, Ali_2017, Atayan_2016, Bhattacharjee_2008, Csaky_1998e, Kar_2015, Kariholu_2008, Kobiela_2015, Losanoff_1996, Losanoff_1997e, Mesfin_2022a, Peixoto_2017f, Sobnach_2011f, Tupesis_2004f, Yasin_2009}, 13 cases (18\%) were under 18 years of age \cite{AlShaaibi_2021b, Ali_2020f, Cauchi_2002, DivsalarP._2023a, Goldman_1998f, Liu_2005, Naji_2012f, Ohno_2005, Tanrikulu_2015e, Tay_2004, Wildhaber_2005}, 11 cases (15\%) were between 41 and 60 years of age \cite{Al-Faham_2020k, Bhumi_2024f, CamachoDorado_2018, Emamhadi_2018, Hardy_2023g, Jehangir_2019h, Kumar_2001, Sultan_2024f, Thapa_2019f, Wadhwa_2015e, teWildt_2010}, 3 cases (4\%) were over 60 years of age \cite{Beecroft_1998, Kerestes_2019, Li_2013}, 2 cases (3\%) had no age documented \cite{Berry_2021e}. All 90 were male gender. 90 cases (100\%) were detained at the time of ingestion \cite{Elghali_2016, Karp_1991b, Lee_2007}, 88 cases (98\%) were intentional ingestions \cite{Elghali_2016, Karp_1991b, Lee_2007}, 30 cases (33\%) had a psychiatric history documented \cite{Elghali_2016, Karp_1991b, Lee_2007}, 2 cases (2\%) had a history of prior ingestion \cite{Elghali_2016}. No cases were reported for were psychiatric inpatients, were displaced people, were under the influence of alcohol at the time of ingestion, and had a severe disability history.
\paragraph*{Motivation}  70 cases (78\%) reported protest motivation \cite{Elghali_2016, Karp_1991b, Lee_2007}, 12 cases (13\%) reported psychiatric motivation \cite{Karp_1991b}, 6 cases (7\%) reported self-harm motivation \cite{Elghali_2016, Karp_1991b}. No cases were reported for psychosocial motivation and other motivation.
\paragraph*{Object Characteristics}  68 cases (76\%) involved sharp object ingestion \cite{Elghali_2016, Karp_1991b, Lee_2007}, 32 cases (36\%) involved long (\textgreater 5cm) object ingestion \cite{Lee_2007}, 25 cases (28\%) involved ingestion of multiple objects \cite{Elghali_2016, Lee_2007}. No cases were reported for button battery ingestion, magnet ingestion, and involved large diameter (\textgreater 2.5cm) object ingestion.
\paragraph*{Outcomes}  47 cases (52\%) underwent endoscopic intervention \cite{Elghali_2016, Lee_2007}, 29 cases (32\%) were managed conservatively \cite{Elghali_2016, Karp_1991b}, 15 cases (17\%) underwent surgical intervention \cite{Elghali_2016, Karp_1991b, Lee_2007}, 6 cases (7\%) reported complications \cite{Lee_2007}, 1 case (1\%) died \cite{Elghali_2016}.
\paragraph*{Gender} 43 cases (60\%) were male \cite{Akay_2015f, Al-Faham_2020k, Alao_2006i, Ali_2017, Ali_2022g, Apikotoa_2022f, Atayan_2016, Benoist_2019e, Berry_2021e, Bhumi_2024f, CamachoDorado_2018, Csaky_1998e, Emamhadi_2018, Farhadi_2024h, Fry_2010, Gardner_2017h, Guinan_2019f, Jehangir_2019h, Jin_2023, Kobiela_2015, Kumar_2001, Kumar_2019f, Liu_2005, Losanoff_1996, Losanoff_1997e, Mesfin_2022a, Misra_2013, Qureshi_2016, Riva_2018j, Sobnach_2011f, Tammana_2012j, Tanrikulu_2015e, Tay_2004, Thapa_2019f, Trgo_2012f, Wadhwa_2015e, Yasin_2009, teWildt_2010}, 28 cases (39\%) were female \cite{AlShaaibi_2021b, Ali_2020f, Ataya_2013, Beecroft_1998, Bhasin_2014, Bhattacharjee_2008, Cauchi_2002, Chang_2017f, Cox_2007, DelgadoSalazar_2020c, DivsalarP._2023a, Goldman_1998f, Hardy_2023g, Kar_2015, Kariholu_2008, Kerestes_2019, Li_2013, Naji_2012f, Ohno_2005, Peixoto_2017f, Sakellaridis_2008f, Sultan_2024f, Tupesis_2004f, Wildhaber_2005, Wnęk_2015f, Yildiz_2016e}, 1 case (1\%) had no gender recorded \cite{fjbuilsRepeatedBehaviorDeliberate2024}. \paragraph*{Age Group} 25 cases (35\%) were between 26 and 40 years of age \cite{Alao_2006i, Ali_2022g, Apikotoa_2022f, Ataya_2013, Benoist_2019e, Bhasin_2014, Chang_2017f, Cox_2007, DelgadoSalazar_2020c, Farhadi_2024h, Fry_2010, Gardner_2017h, Guinan_2019f, Jin_2023, Kumar_2019f, Losanoff_1996, Misra_2013, Qureshi_2016, Riva_2018j, Sakellaridis_2008f, Tammana_2012j, Trgo_2012f, Wnęk_2015f, Yildiz_2016e, fjbuilsRepeatedBehaviorDeliberate2024}, 18 cases (25\%) were between 18 and 25 years of age \cite{Akay_2015f, Ali_2017, Atayan_2016, Bhattacharjee_2008, Csaky_1998e, Kar_2015, Kariholu_2008, Kobiela_2015, Losanoff_1996, Losanoff_1997e, Mesfin_2022a, Peixoto_2017f, Sobnach_2011f, Tupesis_2004f, Yasin_2009}, 13 cases (18\%) were under 18 years of age \cite{AlShaaibi_2021b, Ali_2020f, Cauchi_2002, DivsalarP._2023a, Goldman_1998f, Liu_2005, Naji_2012f, Ohno_2005, Tanrikulu_2015e, Tay_2004, Wildhaber_2005}, 11 cases (15\%) were between 41 and 60 years of age \cite{Al-Faham_2020k, Bhumi_2024f, CamachoDorado_2018, Emamhadi_2018, Hardy_2023g, Jehangir_2019h, Kumar_2001, Sultan_2024f, Thapa_2019f, Wadhwa_2015e, teWildt_2010}, 3 cases (4\%) were over 60 years of age \cite{Beecroft_1998, Kerestes_2019, Li_2013}, 2 cases (3\%) had no age documented \cite{Berry_2021e}. All 90 were male gender. 90 cases (100\%) were detained at the time of ingestion \cite{Elghali_2016, Karp_1991b, Lee_2007}, 88 cases (98\%) were intentional ingestions \cite{Elghali_2016, Karp_1991b, Lee_2007}, 30 cases (33\%) had a psychiatric history documented \cite{Elghali_2016, Karp_1991b, Lee_2007}, 2 cases (2\%) had a history of prior ingestion \cite{Elghali_2016}. No cases were reported for were psychiatric inpatients, were displaced people, were under the influence of alcohol at the time of ingestion, and had a severe disability history.
\paragraph*{Motivation}  70 cases (78\%) reported protest motivation \cite{Elghali_2016, Karp_1991b, Lee_2007}, 12 cases (13\%) reported psychiatric motivation \cite{Karp_1991b}, 6 cases (7\%) reported self-harm motivation \cite{Elghali_2016, Karp_1991b}. No cases were reported for psychosocial motivation and other motivation.
\paragraph*{Object Characteristics}  68 cases (76\%) involved sharp object ingestion \cite{Elghali_2016, Karp_1991b, Lee_2007}, 32 cases (36\%) involved long (\textgreater 5cm) object ingestion \cite{Lee_2007}, 25 cases (28\%) involved ingestion of multiple objects \cite{Elghali_2016, Lee_2007}. No cases were reported for button battery ingestion, magnet ingestion, and involved large diameter (\textgreater 2.5cm) object ingestion.
\paragraph*{Outcomes}  47 cases (52\%) underwent endoscopic intervention \cite{Elghali_2016, Lee_2007}, 29 cases (32\%) were managed conservatively \cite{Elghali_2016, Karp_1991b}, 15 cases (17\%) underwent surgical intervention \cite{Elghali_2016, Karp_1991b, Lee_2007}, 6 cases (7\%) reported complications \cite{Lee_2007}, 1 case (1\%) died \cite{Elghali_2016}.
\paragraph*{Gender} 43 cases (60\%) were male \cite{Akay_2015f, Al-Faham_2020k, Alao_2006i, Ali_2017, Ali_2022g, Apikotoa_2022f, Atayan_2016, Benoist_2019e, Berry_2021e, Bhumi_2024f, CamachoDorado_2018, Csaky_1998e, Emamhadi_2018, Farhadi_2024h, Fry_2010, Gardner_2017h, Guinan_2019f, Jehangir_2019h, Jin_2023, Kobiela_2015, Kumar_2001, Kumar_2019f, Liu_2005, Losanoff_1996, Losanoff_1997e, Mesfin_2022a, Misra_2013, Qureshi_2016, Riva_2018j, Sobnach_2011f, Tammana_2012j, Tanrikulu_2015e, Tay_2004, Thapa_2019f, Trgo_2012f, Wadhwa_2015e, Yasin_2009, teWildt_2010}, 28 cases (39\%) were female \cite{AlShaaibi_2021b, Ali_2020f, Ataya_2013, Beecroft_1998, Bhasin_2014, Bhattacharjee_2008, Cauchi_2002, Chang_2017f, Cox_2007, DelgadoSalazar_2020c, DivsalarP._2023a, Goldman_1998f, Hardy_2023g, Kar_2015, Kariholu_2008, Kerestes_2019, Li_2013, Naji_2012f, Ohno_2005, Peixoto_2017f, Sakellaridis_2008f, Sultan_2024f, Tupesis_2004f, Wildhaber_2005, Wnęk_2015f, Yildiz_2016e}, 1 case (1\%) had no gender recorded \cite{fjbuilsRepeatedBehaviorDeliberate2024}. \paragraph*{Age Group} 25 cases (35\%) were between 26 and 40 years of age \cite{Alao_2006i, Ali_2022g, Apikotoa_2022f, Ataya_2013, Benoist_2019e, Bhasin_2014, Chang_2017f, Cox_2007, DelgadoSalazar_2020c, Farhadi_2024h, Fry_2010, Gardner_2017h, Guinan_2019f, Jin_2023, Kumar_2019f, Losanoff_1996, Misra_2013, Qureshi_2016, Riva_2018j, Sakellaridis_2008f, Tammana_2012j, Trgo_2012f, Wnęk_2015f, Yildiz_2016e, fjbuilsRepeatedBehaviorDeliberate2024}, 18 cases (25\%) were between 18 and 25 years of age \cite{Akay_2015f, Ali_2017, Atayan_2016, Bhattacharjee_2008, Csaky_1998e, Kar_2015, Kariholu_2008, Kobiela_2015, Losanoff_1996, Losanoff_1997e, Mesfin_2022a, Peixoto_2017f, Sobnach_2011f, Tupesis_2004f, Yasin_2009}, 13 cases (18\%) were under 18 years of age \cite{AlShaaibi_2021b, Ali_2020f, Cauchi_2002, DivsalarP._2023a, Goldman_1998f, Liu_2005, Naji_2012f, Ohno_2005, Tanrikulu_2015e, Tay_2004, Wildhaber_2005}, 11 cases (15\%) were between 41 and 60 years of age \cite{Al-Faham_2020k, Bhumi_2024f, CamachoDorado_2018, Emamhadi_2018, Hardy_2023g, Jehangir_2019h, Kumar_2001, Sultan_2024f, Thapa_2019f, Wadhwa_2015e, teWildt_2010}, 3 cases (4\%) were over 60 years of age \cite{Beecroft_1998, Kerestes_2019, Li_2013}, 2 cases (3\%) had no age documented \cite{Berry_2021e}. All 90 were male gender. 90 cases (100\%) were detained at the time of ingestion \cite{Elghali_2016, Karp_1991b, Lee_2007}, 88 cases (98\%) were intentional ingestions \cite{Elghali_2016, Karp_1991b, Lee_2007}, 30 cases (33\%) had a psychiatric history documented \cite{Elghali_2016, Karp_1991b, Lee_2007}, 2 cases (2\%) had a history of prior ingestion \cite{Elghali_2016}. No cases were reported for were psychiatric inpatients, were displaced people, were under the influence of alcohol at the time of ingestion, and had a severe disability history.
\paragraph*{Motivation}  70 cases (78\%) reported protest motivation \cite{Elghali_2016, Karp_1991b, Lee_2007}, 12 cases (13\%) reported psychiatric motivation \cite{Karp_1991b}, 6 cases (7\%) reported self-harm motivation \cite{Elghali_2016, Karp_1991b}. No cases were reported for psychosocial motivation and other motivation.
\paragraph*{Object Characteristics}  68 cases (76\%) involved sharp object ingestion \cite{Elghali_2016, Karp_1991b, Lee_2007}, 32 cases (36\%) involved long (\textgreater 5cm) object ingestion \cite{Lee_2007}, 25 cases (28\%) involved ingestion of multiple objects \cite{Elghali_2016, Lee_2007}. No cases were reported for button battery ingestion, magnet ingestion, and involved large diameter (\textgreater 2.5cm) object ingestion.
\paragraph*{Outcomes}  47 cases (52\%) underwent endoscopic intervention \cite{Elghali_2016, Lee_2007}, 29 cases (32\%) were managed conservatively \cite{Elghali_2016, Karp_1991b}, 15 cases (17\%) underwent surgical intervention \cite{Elghali_2016, Karp_1991b, Lee_2007}, 6 cases (7\%) reported complications \cite{Lee_2007}, 1 case (1\%) died \cite{Elghali_2016}.
\paragraph*{Gender} 43 cases (60\%) were male \cite{Akay_2015f, Al-Faham_2020k, Alao_2006i, Ali_2017, Ali_2022g, Apikotoa_2022f, Atayan_2016, Benoist_2019e, Berry_2021e, Bhumi_2024f, CamachoDorado_2018, Csaky_1998e, Emamhadi_2018, Farhadi_2024h, Fry_2010, Gardner_2017h, Guinan_2019f, Jehangir_2019h, Jin_2023, Kobiela_2015, Kumar_2001, Kumar_2019f, Liu_2005, Losanoff_1996, Losanoff_1997e, Mesfin_2022a, Misra_2013, Qureshi_2016, Riva_2018j, Sobnach_2011f, Tammana_2012j, Tanrikulu_2015e, Tay_2004, Thapa_2019f, Trgo_2012f, Wadhwa_2015e, Yasin_2009, teWildt_2010}, 28 cases (39\%) were female \cite{AlShaaibi_2021b, Ali_2020f, Ataya_2013, Beecroft_1998, Bhasin_2014, Bhattacharjee_2008, Cauchi_2002, Chang_2017f, Cox_2007, DelgadoSalazar_2020c, DivsalarP._2023a, Goldman_1998f, Hardy_2023g, Kar_2015, Kariholu_2008, Kerestes_2019, Li_2013, Naji_2012f, Ohno_2005, Peixoto_2017f, Sakellaridis_2008f, Sultan_2024f, Tupesis_2004f, Wildhaber_2005, Wnęk_2015f, Yildiz_2016e}, 1 case (1\%) had no gender recorded \cite{fjbuilsRepeatedBehaviorDeliberate2024}. \paragraph*{Age Group} 25 cases (35\%) were between 26 and 40 years of age \cite{Alao_2006i, Ali_2022g, Apikotoa_2022f, Ataya_2013, Benoist_2019e, Bhasin_2014, Chang_2017f, Cox_2007, DelgadoSalazar_2020c, Farhadi_2024h, Fry_2010, Gardner_2017h, Guinan_2019f, Jin_2023, Kumar_2019f, Losanoff_1996, Misra_2013, Qureshi_2016, Riva_2018j, Sakellaridis_2008f, Tammana_2012j, Trgo_2012f, Wnęk_2015f, Yildiz_2016e, fjbuilsRepeatedBehaviorDeliberate2024}, 18 cases (25\%) were between 18 and 25 years of age \cite{Akay_2015f, Ali_2017, Atayan_2016, Bhattacharjee_2008, Csaky_1998e, Kar_2015, Kariholu_2008, Kobiela_2015, Losanoff_1996, Losanoff_1997e, Mesfin_2022a, Peixoto_2017f, Sobnach_2011f, Tupesis_2004f, Yasin_2009}, 13 cases (18\%) were under 18 years of age \cite{AlShaaibi_2021b, Ali_2020f, Cauchi_2002, DivsalarP._2023a, Goldman_1998f, Liu_2005, Naji_2012f, Ohno_2005, Tanrikulu_2015e, Tay_2004, Wildhaber_2005}, 11 cases (15\%) were between 41 and 60 years of age \cite{Al-Faham_2020k, Bhumi_2024f, CamachoDorado_2018, Emamhadi_2018, Hardy_2023g, Jehangir_2019h, Kumar_2001, Sultan_2024f, Thapa_2019f, Wadhwa_2015e, teWildt_2010}, 3 cases (4\%) were over 60 years of age \cite{Beecroft_1998, Kerestes_2019, Li_2013}, 2 cases (3\%) had no age documented \cite{Berry_2021e}. All 90 were male gender. 90 cases (100\%) were detained at the time of ingestion \cite{Elghali_2016, Karp_1991b, Lee_2007}, 88 cases (98\%) were intentional ingestions \cite{Elghali_2016, Karp_1991b, Lee_2007}, 30 cases (33\%) had a psychiatric history documented \cite{Elghali_2016, Karp_1991b, Lee_2007}, 2 cases (2\%) had a history of prior ingestion \cite{Elghali_2016}. No cases were reported for were psychiatric inpatients, were displaced people, were under the influence of alcohol at the time of ingestion, and had a severe disability history.
\paragraph*{Motivation}  70 cases (78\%) reported protest motivation \cite{Elghali_2016, Karp_1991b, Lee_2007}, 12 cases (13\%) reported psychiatric motivation \cite{Karp_1991b}, 6 cases (7\%) reported self-harm motivation \cite{Elghali_2016, Karp_1991b}. No cases were reported for psychosocial motivation and other motivation.
\paragraph*{Object Characteristics}  68 cases (76\%) involved sharp object ingestion \cite{Elghali_2016, Karp_1991b, Lee_2007}, 32 cases (36\%) involved long (\textgreater 5cm) object ingestion \cite{Lee_2007}, 25 cases (28\%) involved ingestion of multiple objects \cite{Elghali_2016, Lee_2007}. No cases were reported for button battery ingestion, magnet ingestion, and involved large diameter (\textgreater 2.5cm) object ingestion.
\paragraph*{Outcomes}  47 cases (52\%) underwent endoscopic intervention \cite{Elghali_2016, Lee_2007}, 29 cases (32\%) were managed conservatively \cite{Elghali_2016, Karp_1991b}, 15 cases (17\%) underwent surgical intervention \cite{Elghali_2016, Karp_1991b, Lee_2007}, 6 cases (7\%) reported complications \cite{Lee_2007}, 1 case (1\%) died \cite{Elghali_2016}.
\paragraph*{Gender} 43 cases (60\%) were male \cite{Akay_2015f, Al-Faham_2020k, Alao_2006i, Ali_2017, Ali_2022g, Apikotoa_2022f, Atayan_2016, Benoist_2019e, Berry_2021e, Bhumi_2024f, CamachoDorado_2018, Csaky_1998e, Emamhadi_2018, Farhadi_2024h, Fry_2010, Gardner_2017h, Guinan_2019f, Jehangir_2019h, Jin_2023, Kobiela_2015, Kumar_2001, Kumar_2019f, Liu_2005, Losanoff_1996, Losanoff_1997e, Mesfin_2022a, Misra_2013, Qureshi_2016, Riva_2018j, Sobnach_2011f, Tammana_2012j, Tanrikulu_2015e, Tay_2004, Thapa_2019f, Trgo_2012f, Wadhwa_2015e, Yasin_2009, teWildt_2010}, 28 cases (39\%) were female \cite{AlShaaibi_2021b, Ali_2020f, Ataya_2013, Beecroft_1998, Bhasin_2014, Bhattacharjee_2008, Cauchi_2002, Chang_2017f, Cox_2007, DelgadoSalazar_2020c, DivsalarP._2023a, Goldman_1998f, Hardy_2023g, Kar_2015, Kariholu_2008, Kerestes_2019, Li_2013, Naji_2012f, Ohno_2005, Peixoto_2017f, Sakellaridis_2008f, Sultan_2024f, Tupesis_2004f, Wildhaber_2005, Wnęk_2015f, Yildiz_2016e}, 1 case (1\%) had no gender recorded \cite{fjbuilsRepeatedBehaviorDeliberate2024}. \paragraph*{Age Group} 25 cases (35\%) were between 26 and 40 years of age \cite{Alao_2006i, Ali_2022g, Apikotoa_2022f, Ataya_2013, Benoist_2019e, Bhasin_2014, Chang_2017f, Cox_2007, DelgadoSalazar_2020c, Farhadi_2024h, Fry_2010, Gardner_2017h, Guinan_2019f, Jin_2023, Kumar_2019f, Losanoff_1996, Misra_2013, Qureshi_2016, Riva_2018j, Sakellaridis_2008f, Tammana_2012j, Trgo_2012f, Wnęk_2015f, Yildiz_2016e, fjbuilsRepeatedBehaviorDeliberate2024}, 18 cases (25\%) were between 18 and 25 years of age \cite{Akay_2015f, Ali_2017, Atayan_2016, Bhattacharjee_2008, Csaky_1998e, Kar_2015, Kariholu_2008, Kobiela_2015, Losanoff_1996, Losanoff_1997e, Mesfin_2022a, Peixoto_2017f, Sobnach_2011f, Tupesis_2004f, Yasin_2009}, 13 cases (18\%) were under 18 years of age \cite{AlShaaibi_2021b, Ali_2020f, Cauchi_2002, DivsalarP._2023a, Goldman_1998f, Liu_2005, Naji_2012f, Ohno_2005, Tanrikulu_2015e, Tay_2004, Wildhaber_2005}, 11 cases (15\%) were between 41 and 60 years of age \cite{Al-Faham_2020k, Bhumi_2024f, CamachoDorado_2018, Emamhadi_2018, Hardy_2023g, Jehangir_2019h, Kumar_2001, Sultan_2024f, Thapa_2019f, Wadhwa_2015e, teWildt_2010}, 3 cases (4\%) were over 60 years of age \cite{Beecroft_1998, Kerestes_2019, Li_2013}, 2 cases (3\%) had no age documented \cite{Berry_2021e}. All 90 were male gender. 90 cases (100\%) were detained at the time of ingestion \cite{Elghali_2016, Karp_1991b, Lee_2007}, 88 cases (98\%) were intentional ingestions \cite{Elghali_2016, Karp_1991b, Lee_2007}, 30 cases (33\%) had a psychiatric history documented \cite{Elghali_2016, Karp_1991b, Lee_2007}, 2 cases (2\%) had a history of prior ingestion \cite{Elghali_2016}. No cases were reported for were psychiatric inpatients, were displaced people, were under the influence of alcohol at the time of ingestion, and had a severe disability history.
\paragraph*{Motivation}  70 cases (78\%) reported protest motivation \cite{Elghali_2016, Karp_1991b, Lee_2007}, 12 cases (13\%) reported psychiatric motivation \cite{Karp_1991b}, 6 cases (7\%) reported self-harm motivation \cite{Elghali_2016, Karp_1991b}. No cases were reported for psychosocial motivation and other motivation.
\paragraph*{Object Characteristics}  68 cases (76\%) involved sharp object ingestion \cite{Elghali_2016, Karp_1991b, Lee_2007}, 32 cases (36\%) involved long (\textgreater 5cm) object ingestion \cite{Lee_2007}, 25 cases (28\%) involved ingestion of multiple objects \cite{Elghali_2016, Lee_2007}. No cases were reported for button battery ingestion, magnet ingestion, and involved large diameter (\textgreater 2.5cm) object ingestion.
\paragraph*{Outcomes}  47 cases (52\%) underwent endoscopic intervention \cite{Elghali_2016, Lee_2007}, 29 cases (32\%) were managed conservatively \cite{Elghali_2016, Karp_1991b}, 15 cases (17\%) underwent surgical intervention \cite{Elghali_2016, Karp_1991b, Lee_2007}, 6 cases (7\%) reported complications \cite{Lee_2007}, 1 case (1\%) died \cite{Elghali_2016}.
\paragraph*{Gender} 43 cases (60\%) were male \cite{Akay_2015f, Al-Faham_2020k, Alao_2006i, Ali_2017, Ali_2022g, Apikotoa_2022f, Atayan_2016, Benoist_2019e, Berry_2021e, Bhumi_2024f, CamachoDorado_2018, Csaky_1998e, Emamhadi_2018, Farhadi_2024h, Fry_2010, Gardner_2017h, Guinan_2019f, Jehangir_2019h, Jin_2023, Kobiela_2015, Kumar_2001, Kumar_2019f, Liu_2005, Losanoff_1996, Losanoff_1997e, Mesfin_2022a, Misra_2013, Qureshi_2016, Riva_2018j, Sobnach_2011f, Tammana_2012j, Tanrikulu_2015e, Tay_2004, Thapa_2019f, Trgo_2012f, Wadhwa_2015e, Yasin_2009, teWildt_2010}, 28 cases (39\%) were female \cite{AlShaaibi_2021b, Ali_2020f, Ataya_2013, Beecroft_1998, Bhasin_2014, Bhattacharjee_2008, Cauchi_2002, Chang_2017f, Cox_2007, DelgadoSalazar_2020c, DivsalarP._2023a, Goldman_1998f, Hardy_2023g, Kar_2015, Kariholu_2008, Kerestes_2019, Li_2013, Naji_2012f, Ohno_2005, Peixoto_2017f, Sakellaridis_2008f, Sultan_2024f, Tupesis_2004f, Wildhaber_2005, Wnęk_2015f, Yildiz_2016e}, 1 case (1\%) had no gender recorded \cite{fjbuilsRepeatedBehaviorDeliberate2024}. \paragraph*{Age Group} 25 cases (35\%) were between 26 and 40 years of age \cite{Alao_2006i, Ali_2022g, Apikotoa_2022f, Ataya_2013, Benoist_2019e, Bhasin_2014, Chang_2017f, Cox_2007, DelgadoSalazar_2020c, Farhadi_2024h, Fry_2010, Gardner_2017h, Guinan_2019f, Jin_2023, Kumar_2019f, Losanoff_1996, Misra_2013, Qureshi_2016, Riva_2018j, Sakellaridis_2008f, Tammana_2012j, Trgo_2012f, Wnęk_2015f, Yildiz_2016e, fjbuilsRepeatedBehaviorDeliberate2024}, 18 cases (25\%) were between 18 and 25 years of age \cite{Akay_2015f, Ali_2017, Atayan_2016, Bhattacharjee_2008, Csaky_1998e, Kar_2015, Kariholu_2008, Kobiela_2015, Losanoff_1996, Losanoff_1997e, Mesfin_2022a, Peixoto_2017f, Sobnach_2011f, Tupesis_2004f, Yasin_2009}, 13 cases (18\%) were under 18 years of age \cite{AlShaaibi_2021b, Ali_2020f, Cauchi_2002, DivsalarP._2023a, Goldman_1998f, Liu_2005, Naji_2012f, Ohno_2005, Tanrikulu_2015e, Tay_2004, Wildhaber_2005}, 11 cases (15\%) were between 41 and 60 years of age \cite{Al-Faham_2020k, Bhumi_2024f, CamachoDorado_2018, Emamhadi_2018, Hardy_2023g, Jehangir_2019h, Kumar_2001, Sultan_2024f, Thapa_2019f, Wadhwa_2015e, teWildt_2010}, 3 cases (4\%) were over 60 years of age \cite{Beecroft_1998, Kerestes_2019, Li_2013}, 2 cases (3\%) had no age documented \cite{Berry_2021e}. All 90 were male gender. 90 cases (100\%) were detained at the time of ingestion \cite{Elghali_2016, Karp_1991b, Lee_2007}, 88 cases (98\%) were intentional ingestions \cite{Elghali_2016, Karp_1991b, Lee_2007}, 30 cases (33\%) had a psychiatric history documented \cite{Elghali_2016, Karp_1991b, Lee_2007}, 2 cases (2\%) had a history of prior ingestion \cite{Elghali_2016}. No cases were reported for were psychiatric inpatients, were displaced people, were under the influence of alcohol at the time of ingestion, and had a severe disability history.
\paragraph*{Motivation}  70 cases (78\%) reported protest motivation \cite{Elghali_2016, Karp_1991b, Lee_2007}, 12 cases (13\%) reported psychiatric motivation \cite{Karp_1991b}, 6 cases (7\%) reported self-harm motivation \cite{Elghali_2016, Karp_1991b}. No cases were reported for psychosocial motivation and other motivation.
\paragraph*{Object Characteristics}  68 cases (76\%) involved sharp object ingestion \cite{Elghali_2016, Karp_1991b, Lee_2007}, 32 cases (36\%) involved long (\textgreater 5cm) object ingestion \cite{Lee_2007}, 25 cases (28\%) involved ingestion of multiple objects \cite{Elghali_2016, Lee_2007}. No cases were reported for button battery ingestion, magnet ingestion, and involved large diameter (\textgreater 2.5cm) object ingestion.
\paragraph*{Outcomes}  47 cases (52\%) underwent endoscopic intervention \cite{Elghali_2016, Lee_2007}, 29 cases (32\%) were managed conservatively \cite{Elghali_2016, Karp_1991b}, 15 cases (17\%) underwent surgical intervention \cite{Elghali_2016, Karp_1991b, Lee_2007}, 6 cases (7\%) reported complications \cite{Lee_2007}, 1 case (1\%) died \cite{Elghali_2016}.
\paragraph*{Gender} 43 cases (60\%) were male \cite{Akay_2015f, Al-Faham_2020k, Alao_2006i, Ali_2017, Ali_2022g, Apikotoa_2022f, Atayan_2016, Benoist_2019e, Berry_2021e, Bhumi_2024f, CamachoDorado_2018, Csaky_1998e, Emamhadi_2018, Farhadi_2024h, Fry_2010, Gardner_2017h, Guinan_2019f, Jehangir_2019h, Jin_2023, Kobiela_2015, Kumar_2001, Kumar_2019f, Liu_2005, Losanoff_1996, Losanoff_1997e, Mesfin_2022a, Misra_2013, Qureshi_2016, Riva_2018j, Sobnach_2011f, Tammana_2012j, Tanrikulu_2015e, Tay_2004, Thapa_2019f, Trgo_2012f, Wadhwa_2015e, Yasin_2009, teWildt_2010}, 28 cases (39\%) were female \cite{AlShaaibi_2021b, Ali_2020f, Ataya_2013, Beecroft_1998, Bhasin_2014, Bhattacharjee_2008, Cauchi_2002, Chang_2017f, Cox_2007, DelgadoSalazar_2020c, DivsalarP._2023a, Goldman_1998f, Hardy_2023g, Kar_2015, Kariholu_2008, Kerestes_2019, Li_2013, Naji_2012f, Ohno_2005, Peixoto_2017f, Sakellaridis_2008f, Sultan_2024f, Tupesis_2004f, Wildhaber_2005, Wnęk_2015f, Yildiz_2016e}, 1 case (1\%) had no gender recorded \cite{fjbuilsRepeatedBehaviorDeliberate2024}. \paragraph*{Age Group} 25 cases (35\%) were between 26 and 40 years of age \cite{Alao_2006i, Ali_2022g, Apikotoa_2022f, Ataya_2013, Benoist_2019e, Bhasin_2014, Chang_2017f, Cox_2007, DelgadoSalazar_2020c, Farhadi_2024h, Fry_2010, Gardner_2017h, Guinan_2019f, Jin_2023, Kumar_2019f, Losanoff_1996, Misra_2013, Qureshi_2016, Riva_2018j, Sakellaridis_2008f, Tammana_2012j, Trgo_2012f, Wnęk_2015f, Yildiz_2016e, fjbuilsRepeatedBehaviorDeliberate2024}, 18 cases (25\%) were between 18 and 25 years of age \cite{Akay_2015f, Ali_2017, Atayan_2016, Bhattacharjee_2008, Csaky_1998e, Kar_2015, Kariholu_2008, Kobiela_2015, Losanoff_1996, Losanoff_1997e, Mesfin_2022a, Peixoto_2017f, Sobnach_2011f, Tupesis_2004f, Yasin_2009}, 13 cases (18\%) were under 18 years of age \cite{AlShaaibi_2021b, Ali_2020f, Cauchi_2002, DivsalarP._2023a, Goldman_1998f, Liu_2005, Naji_2012f, Ohno_2005, Tanrikulu_2015e, Tay_2004, Wildhaber_2005}, 11 cases (15\%) were between 41 and 60 years of age \cite{Al-Faham_2020k, Bhumi_2024f, CamachoDorado_2018, Emamhadi_2018, Hardy_2023g, Jehangir_2019h, Kumar_2001, Sultan_2024f, Thapa_2019f, Wadhwa_2015e, teWildt_2010}, 3 cases (4\%) were over 60 years of age \cite{Beecroft_1998, Kerestes_2019, Li_2013}, 2 cases (3\%) had no age documented \cite{Berry_2021e}. All 90 were male gender. 90 cases (100\%) were detained at the time of ingestion \cite{Elghali_2016, Karp_1991b, Lee_2007}, 88 cases (98\%) were intentional ingestions \cite{Elghali_2016, Karp_1991b, Lee_2007}, 30 cases (33\%) had a psychiatric history documented \cite{Elghali_2016, Karp_1991b, Lee_2007}, 2 cases (2\%) had a history of prior ingestion \cite{Elghali_2016}. No cases were reported for were psychiatric inpatients, were displaced people, were under the influence of alcohol at the time of ingestion, and had a severe disability history.
\paragraph*{Motivation}  70 cases (78\%) reported protest motivation \cite{Elghali_2016, Karp_1991b, Lee_2007}, 12 cases (13\%) reported psychiatric motivation \cite{Karp_1991b}, 6 cases (7\%) reported self-harm motivation \cite{Elghali_2016, Karp_1991b}. No cases were reported for psychosocial motivation and other motivation.
\paragraph*{Object Characteristics}  68 cases (76\%) involved sharp object ingestion \cite{Elghali_2016, Karp_1991b, Lee_2007}, 32 cases (36\%) involved long (\textgreater 5cm) object ingestion \cite{Lee_2007}, 25 cases (28\%) involved ingestion of multiple objects \cite{Elghali_2016, Lee_2007}. No cases were reported for button battery ingestion, magnet ingestion, and involved large diameter (\textgreater 2.5cm) object ingestion.
\paragraph*{Outcomes}  47 cases (52\%) underwent endoscopic intervention \cite{Elghali_2016, Lee_2007}, 29 cases (32\%) were managed conservatively \cite{Elghali_2016, Karp_1991b}, 15 cases (17\%) underwent surgical intervention \cite{Elghali_2016, Karp_1991b, Lee_2007}, 6 cases (7\%) reported complications \cite{Lee_2007}, 1 case (1\%) died \cite{Elghali_2016}.
\paragraph*{Gender} 43 cases (60\%) were male \cite{Akay_2015f, Al-Faham_2020k, Alao_2006i, Ali_2017, Ali_2022g, Apikotoa_2022f, Atayan_2016, Benoist_2019e, Berry_2021e, Bhumi_2024f, CamachoDorado_2018, Csaky_1998e, Emamhadi_2018, Farhadi_2024h, Fry_2010, Gardner_2017h, Guinan_2019f, Jehangir_2019h, Jin_2023, Kobiela_2015, Kumar_2001, Kumar_2019f, Liu_2005, Losanoff_1996, Losanoff_1997e, Mesfin_2022a, Misra_2013, Qureshi_2016, Riva_2018j, Sobnach_2011f, Tammana_2012j, Tanrikulu_2015e, Tay_2004, Thapa_2019f, Trgo_2012f, Wadhwa_2015e, Yasin_2009, teWildt_2010}, 28 cases (39\%) were female \cite{AlShaaibi_2021b, Ali_2020f, Ataya_2013, Beecroft_1998, Bhasin_2014, Bhattacharjee_2008, Cauchi_2002, Chang_2017f, Cox_2007, DelgadoSalazar_2020c, DivsalarP._2023a, Goldman_1998f, Hardy_2023g, Kar_2015, Kariholu_2008, Kerestes_2019, Li_2013, Naji_2012f, Ohno_2005, Peixoto_2017f, Sakellaridis_2008f, Sultan_2024f, Tupesis_2004f, Wildhaber_2005, Wnęk_2015f, Yildiz_2016e}, 1 case (1\%) had no gender recorded \cite{fjbuilsRepeatedBehaviorDeliberate2024}. \paragraph*{Age Group} 25 cases (35\%) were between 26 and 40 years of age \cite{Alao_2006i, Ali_2022g, Apikotoa_2022f, Ataya_2013, Benoist_2019e, Bhasin_2014, Chang_2017f, Cox_2007, DelgadoSalazar_2020c, Farhadi_2024h, Fry_2010, Gardner_2017h, Guinan_2019f, Jin_2023, Kumar_2019f, Losanoff_1996, Misra_2013, Qureshi_2016, Riva_2018j, Sakellaridis_2008f, Tammana_2012j, Trgo_2012f, Wnęk_2015f, Yildiz_2016e, fjbuilsRepeatedBehaviorDeliberate2024}, 18 cases (25\%) were between 18 and 25 years of age \cite{Akay_2015f, Ali_2017, Atayan_2016, Bhattacharjee_2008, Csaky_1998e, Kar_2015, Kariholu_2008, Kobiela_2015, Losanoff_1996, Losanoff_1997e, Mesfin_2022a, Peixoto_2017f, Sobnach_2011f, Tupesis_2004f, Yasin_2009}, 13 cases (18\%) were under 18 years of age \cite{AlShaaibi_2021b, Ali_2020f, Cauchi_2002, DivsalarP._2023a, Goldman_1998f, Liu_2005, Naji_2012f, Ohno_2005, Tanrikulu_2015e, Tay_2004, Wildhaber_2005}, 11 cases (15\%) were between 41 and 60 years of age \cite{Al-Faham_2020k, Bhumi_2024f, CamachoDorado_2018, Emamhadi_2018, Hardy_2023g, Jehangir_2019h, Kumar_2001, Sultan_2024f, Thapa_2019f, Wadhwa_2015e, teWildt_2010}, 3 cases (4\%) were over 60 years of age \cite{Beecroft_1998, Kerestes_2019, Li_2013}, 2 cases (3\%) had no age documented \cite{Berry_2021e}. All 90 were male gender. 90 cases (100\%) were detained at the time of ingestion \cite{Elghali_2016, Karp_1991b, Lee_2007}, 88 cases (98\%) were intentional ingestions \cite{Elghali_2016, Karp_1991b, Lee_2007}, 30 cases (33\%) had a psychiatric history documented \cite{Elghali_2016, Karp_1991b, Lee_2007}, 2 cases (2\%) had a history of prior ingestion \cite{Elghali_2016}. No cases were reported for were psychiatric inpatients, were displaced people, were under the influence of alcohol at the time of ingestion, and had a severe disability history.
\paragraph*{Motivation}  70 cases (78\%) reported protest motivation \cite{Elghali_2016, Karp_1991b, Lee_2007}, 12 cases (13\%) reported psychiatric motivation \cite{Karp_1991b}, 6 cases (7\%) reported self-harm motivation \cite{Elghali_2016, Karp_1991b}. No cases were reported for psychosocial motivation and other motivation.
\paragraph*{Object Characteristics}  68 cases (76\%) involved sharp object ingestion \cite{Elghali_2016, Karp_1991b, Lee_2007}, 32 cases (36\%) involved long (\textgreater 5cm) object ingestion \cite{Lee_2007}, 25 cases (28\%) involved ingestion of multiple objects \cite{Elghali_2016, Lee_2007}. No cases were reported for button battery ingestion, magnet ingestion, and involved large diameter (\textgreater 2.5cm) object ingestion.
\paragraph*{Outcomes}  47 cases (52\%) underwent endoscopic intervention \cite{Elghali_2016, Lee_2007}, 29 cases (32\%) were managed conservatively \cite{Elghali_2016, Karp_1991b}, 15 cases (17\%) underwent surgical intervention \cite{Elghali_2016, Karp_1991b, Lee_2007}, 6 cases (7\%) reported complications \cite{Lee_2007}, 1 case (1\%) died \cite{Elghali_2016}.
\paragraph*{Gender} 43 cases (60\%) were male \cite{Akay_2015f, Al-Faham_2020k, Alao_2006i, Ali_2017, Ali_2022g, Apikotoa_2022f, Atayan_2016, Benoist_2019e, Berry_2021e, Bhumi_2024f, CamachoDorado_2018, Csaky_1998e, Emamhadi_2018, Farhadi_2024h, Fry_2010, Gardner_2017h, Guinan_2019f, Jehangir_2019h, Jin_2023, Kobiela_2015, Kumar_2001, Kumar_2019f, Liu_2005, Losanoff_1996, Losanoff_1997e, Mesfin_2022a, Misra_2013, Qureshi_2016, Riva_2018j, Sobnach_2011f, Tammana_2012j, Tanrikulu_2015e, Tay_2004, Thapa_2019f, Trgo_2012f, Wadhwa_2015e, Yasin_2009, teWildt_2010}, 28 cases (39\%) were female \cite{AlShaaibi_2021b, Ali_2020f, Ataya_2013, Beecroft_1998, Bhasin_2014, Bhattacharjee_2008, Cauchi_2002, Chang_2017f, Cox_2007, DelgadoSalazar_2020c, DivsalarP._2023a, Goldman_1998f, Hardy_2023g, Kar_2015, Kariholu_2008, Kerestes_2019, Li_2013, Naji_2012f, Ohno_2005, Peixoto_2017f, Sakellaridis_2008f, Sultan_2024f, Tupesis_2004f, Wildhaber_2005, Wnęk_2015f, Yildiz_2016e}, 1 case (1\%) had no gender recorded \cite{fjbuilsRepeatedBehaviorDeliberate2024}. \paragraph*{Age Group} 25 cases (35\%) were between 26 and 40 years of age \cite{Alao_2006i, Ali_2022g, Apikotoa_2022f, Ataya_2013, Benoist_2019e, Bhasin_2014, Chang_2017f, Cox_2007, DelgadoSalazar_2020c, Farhadi_2024h, Fry_2010, Gardner_2017h, Guinan_2019f, Jin_2023, Kumar_2019f, Losanoff_1996, Misra_2013, Qureshi_2016, Riva_2018j, Sakellaridis_2008f, Tammana_2012j, Trgo_2012f, Wnęk_2015f, Yildiz_2016e, fjbuilsRepeatedBehaviorDeliberate2024}, 18 cases (25\%) were between 18 and 25 years of age \cite{Akay_2015f, Ali_2017, Atayan_2016, Bhattacharjee_2008, Csaky_1998e, Kar_2015, Kariholu_2008, Kobiela_2015, Losanoff_1996, Losanoff_1997e, Mesfin_2022a, Peixoto_2017f, Sobnach_2011f, Tupesis_2004f, Yasin_2009}, 13 cases (18\%) were under 18 years of age \cite{AlShaaibi_2021b, Ali_2020f, Cauchi_2002, DivsalarP._2023a, Goldman_1998f, Liu_2005, Naji_2012f, Ohno_2005, Tanrikulu_2015e, Tay_2004, Wildhaber_2005}, 11 cases (15\%) were between 41 and 60 years of age \cite{Al-Faham_2020k, Bhumi_2024f, CamachoDorado_2018, Emamhadi_2018, Hardy_2023g, Jehangir_2019h, Kumar_2001, Sultan_2024f, Thapa_2019f, Wadhwa_2015e, teWildt_2010}, 3 cases (4\%) were over 60 years of age \cite{Beecroft_1998, Kerestes_2019, Li_2013}, 2 cases (3\%) had no age documented \cite{Berry_2021e}. All 90 were male gender. 90 cases (100\%) were detained at the time of ingestion \cite{Elghali_2016, Karp_1991b, Lee_2007}, 88 cases (98\%) were intentional ingestions \cite{Elghali_2016, Karp_1991b, Lee_2007}, 30 cases (33\%) had a psychiatric history documented \cite{Elghali_2016, Karp_1991b, Lee_2007}, 2 cases (2\%) had a history of prior ingestion \cite{Elghali_2016}. No cases were reported for were psychiatric inpatients, were displaced people, were under the influence of alcohol at the time of ingestion, and had a severe disability history.
\paragraph*{Motivation}  70 cases (78\%) reported protest motivation \cite{Elghali_2016, Karp_1991b, Lee_2007}, 12 cases (13\%) reported psychiatric motivation \cite{Karp_1991b}, 6 cases (7\%) reported self-harm motivation \cite{Elghali_2016, Karp_1991b}. No cases were reported for psychosocial motivation and other motivation.
\paragraph*{Object Characteristics}  68 cases (76\%) involved sharp object ingestion \cite{Elghali_2016, Karp_1991b, Lee_2007}, 32 cases (36\%) involved long (\textgreater 5cm) object ingestion \cite{Lee_2007}, 25 cases (28\%) involved ingestion of multiple objects \cite{Elghali_2016, Lee_2007}. No cases were reported for button battery ingestion, magnet ingestion, and involved large diameter (\textgreater 2.5cm) object ingestion.
\paragraph*{Outcomes}  47 cases (52\%) underwent endoscopic intervention \cite{Elghali_2016, Lee_2007}, 29 cases (32\%) were managed conservatively \cite{Elghali_2016, Karp_1991b}, 15 cases (17\%) underwent surgical intervention \cite{Elghali_2016, Karp_1991b, Lee_2007}, 6 cases (7\%) reported complications \cite{Lee_2007}, 1 case (1\%) died \cite{Elghali_2016}.
\paragraph*{Gender} 43 cases (60\%) were male \cite{Akay_2015f, Al-Faham_2020k, Alao_2006i, Ali_2017, Ali_2022g, Apikotoa_2022f, Atayan_2016, Benoist_2019e, Berry_2021e, Bhumi_2024f, CamachoDorado_2018, Csaky_1998e, Emamhadi_2018, Farhadi_2024h, Fry_2010, Gardner_2017h, Guinan_2019f, Jehangir_2019h, Jin_2023, Kobiela_2015, Kumar_2001, Kumar_2019f, Liu_2005, Losanoff_1996, Losanoff_1997e, Mesfin_2022a, Misra_2013, Qureshi_2016, Riva_2018j, Sobnach_2011f, Tammana_2012j, Tanrikulu_2015e, Tay_2004, Thapa_2019f, Trgo_2012f, Wadhwa_2015e, Yasin_2009, teWildt_2010}, 28 cases (39\%) were female \cite{AlShaaibi_2021b, Ali_2020f, Ataya_2013, Beecroft_1998, Bhasin_2014, Bhattacharjee_2008, Cauchi_2002, Chang_2017f, Cox_2007, DelgadoSalazar_2020c, DivsalarP._2023a, Goldman_1998f, Hardy_2023g, Kar_2015, Kariholu_2008, Kerestes_2019, Li_2013, Naji_2012f, Ohno_2005, Peixoto_2017f, Sakellaridis_2008f, Sultan_2024f, Tupesis_2004f, Wildhaber_2005, Wnęk_2015f, Yildiz_2016e}, 1 case (1\%) had no gender recorded \cite{fjbuilsRepeatedBehaviorDeliberate2024}. \paragraph*{Age Group} 25 cases (35\%) were between 26 and 40 years of age \cite{Alao_2006i, Ali_2022g, Apikotoa_2022f, Ataya_2013, Benoist_2019e, Bhasin_2014, Chang_2017f, Cox_2007, DelgadoSalazar_2020c, Farhadi_2024h, Fry_2010, Gardner_2017h, Guinan_2019f, Jin_2023, Kumar_2019f, Losanoff_1996, Misra_2013, Qureshi_2016, Riva_2018j, Sakellaridis_2008f, Tammana_2012j, Trgo_2012f, Wnęk_2015f, Yildiz_2016e, fjbuilsRepeatedBehaviorDeliberate2024}, 18 cases (25\%) were between 18 and 25 years of age \cite{Akay_2015f, Ali_2017, Atayan_2016, Bhattacharjee_2008, Csaky_1998e, Kar_2015, Kariholu_2008, Kobiela_2015, Losanoff_1996, Losanoff_1997e, Mesfin_2022a, Peixoto_2017f, Sobnach_2011f, Tupesis_2004f, Yasin_2009}, 13 cases (18\%) were under 18 years of age \cite{AlShaaibi_2021b, Ali_2020f, Cauchi_2002, DivsalarP._2023a, Goldman_1998f, Liu_2005, Naji_2012f, Ohno_2005, Tanrikulu_2015e, Tay_2004, Wildhaber_2005}, 11 cases (15\%) were between 41 and 60 years of age \cite{Al-Faham_2020k, Bhumi_2024f, CamachoDorado_2018, Emamhadi_2018, Hardy_2023g, Jehangir_2019h, Kumar_2001, Sultan_2024f, Thapa_2019f, Wadhwa_2015e, teWildt_2010}, 3 cases (4\%) were over 60 years of age \cite{Beecroft_1998, Kerestes_2019, Li_2013}, 2 cases (3\%) had no age documented \cite{Berry_2021e}. All 90 were male gender. 90 cases (100\%) were detained at the time of ingestion \cite{Elghali_2016, Karp_1991b, Lee_2007}, 88 cases (98\%) were intentional ingestions \cite{Elghali_2016, Karp_1991b, Lee_2007}, 30 cases (33\%) had a psychiatric history documented \cite{Elghali_2016, Karp_1991b, Lee_2007}, 2 cases (2\%) had a history of prior ingestion \cite{Elghali_2016}. No cases were reported for were psychiatric inpatients, were displaced people, were under the influence of alcohol at the time of ingestion, and had a severe disability history.
\paragraph*{Motivation}  70 cases (78\%) reported protest motivation \cite{Elghali_2016, Karp_1991b, Lee_2007}, 12 cases (13\%) reported psychiatric motivation \cite{Karp_1991b}, 6 cases (7\%) reported self-harm motivation \cite{Elghali_2016, Karp_1991b}. No cases were reported for psychosocial motivation and other motivation.
\paragraph*{Object Characteristics}  68 cases (76\%) involved sharp object ingestion \cite{Elghali_2016, Karp_1991b, Lee_2007}, 32 cases (36\%) involved long (\textgreater 5cm) object ingestion \cite{Lee_2007}, 25 cases (28\%) involved ingestion of multiple objects \cite{Elghali_2016, Lee_2007}. No cases were reported for button battery ingestion, magnet ingestion, and involved large diameter (\textgreater 2.5cm) object ingestion.
\paragraph*{Outcomes}  47 cases (52\%) underwent endoscopic intervention \cite{Elghali_2016, Lee_2007}, 29 cases (32\%) were managed conservatively \cite{Elghali_2016, Karp_1991b}, 15 cases (17\%) underwent surgical intervention \cite{Elghali_2016, Karp_1991b, Lee_2007}, 6 cases (7\%) reported complications \cite{Lee_2007}, 1 case (1\%) died \cite{Elghali_2016}.
\paragraph*{Gender} 43 cases (60\%) were male \cite{Akay_2015f, Al-Faham_2020k, Alao_2006i, Ali_2017, Ali_2022g, Apikotoa_2022f, Atayan_2016, Benoist_2019e, Berry_2021e, Bhumi_2024f, CamachoDorado_2018, Csaky_1998e, Emamhadi_2018, Farhadi_2024h, Fry_2010, Gardner_2017h, Guinan_2019f, Jehangir_2019h, Jin_2023, Kobiela_2015, Kumar_2001, Kumar_2019f, Liu_2005, Losanoff_1996, Losanoff_1997e, Mesfin_2022a, Misra_2013, Qureshi_2016, Riva_2018j, Sobnach_2011f, Tammana_2012j, Tanrikulu_2015e, Tay_2004, Thapa_2019f, Trgo_2012f, Wadhwa_2015e, Yasin_2009, teWildt_2010}, 28 cases (39\%) were female \cite{AlShaaibi_2021b, Ali_2020f, Ataya_2013, Beecroft_1998, Bhasin_2014, Bhattacharjee_2008, Cauchi_2002, Chang_2017f, Cox_2007, DelgadoSalazar_2020c, DivsalarP._2023a, Goldman_1998f, Hardy_2023g, Kar_2015, Kariholu_2008, Kerestes_2019, Li_2013, Naji_2012f, Ohno_2005, Peixoto_2017f, Sakellaridis_2008f, Sultan_2024f, Tupesis_2004f, Wildhaber_2005, Wnęk_2015f, Yildiz_2016e}, 1 case (1\%) had no gender recorded \cite{fjbuilsRepeatedBehaviorDeliberate2024}. \paragraph*{Age Group} 25 cases (35\%) were between 26 and 40 years of age \cite{Alao_2006i, Ali_2022g, Apikotoa_2022f, Ataya_2013, Benoist_2019e, Bhasin_2014, Chang_2017f, Cox_2007, DelgadoSalazar_2020c, Farhadi_2024h, Fry_2010, Gardner_2017h, Guinan_2019f, Jin_2023, Kumar_2019f, Losanoff_1996, Misra_2013, Qureshi_2016, Riva_2018j, Sakellaridis_2008f, Tammana_2012j, Trgo_2012f, Wnęk_2015f, Yildiz_2016e, fjbuilsRepeatedBehaviorDeliberate2024}, 18 cases (25\%) were between 18 and 25 years of age \cite{Akay_2015f, Ali_2017, Atayan_2016, Bhattacharjee_2008, Csaky_1998e, Kar_2015, Kariholu_2008, Kobiela_2015, Losanoff_1996, Losanoff_1997e, Mesfin_2022a, Peixoto_2017f, Sobnach_2011f, Tupesis_2004f, Yasin_2009}, 13 cases (18\%) were under 18 years of age \cite{AlShaaibi_2021b, Ali_2020f, Cauchi_2002, DivsalarP._2023a, Goldman_1998f, Liu_2005, Naji_2012f, Ohno_2005, Tanrikulu_2015e, Tay_2004, Wildhaber_2005}, 11 cases (15\%) were between 41 and 60 years of age \cite{Al-Faham_2020k, Bhumi_2024f, CamachoDorado_2018, Emamhadi_2018, Hardy_2023g, Jehangir_2019h, Kumar_2001, Sultan_2024f, Thapa_2019f, Wadhwa_2015e, teWildt_2010}, 3 cases (4\%) were over 60 years of age \cite{Beecroft_1998, Kerestes_2019, Li_2013}, 2 cases (3\%) had no age documented \cite{Berry_2021e}. All 90 were male gender. 90 cases (100\%) were detained at the time of ingestion \cite{Elghali_2016, Karp_1991b, Lee_2007}, 88 cases (98\%) were intentional ingestions \cite{Elghali_2016, Karp_1991b, Lee_2007}, 30 cases (33\%) had a psychiatric history documented \cite{Elghali_2016, Karp_1991b, Lee_2007}, 2 cases (2\%) had a history of prior ingestion \cite{Elghali_2016}. No cases were reported for were psychiatric inpatients, were displaced people, were under the influence of alcohol at the time of ingestion, and had a severe disability history.
\paragraph*{Motivation}  70 cases (78\%) reported protest motivation \cite{Elghali_2016, Karp_1991b, Lee_2007}, 12 cases (13\%) reported psychiatric motivation \cite{Karp_1991b}, 6 cases (7\%) reported self-harm motivation \cite{Elghali_2016, Karp_1991b}. No cases were reported for psychosocial motivation and other motivation.
\paragraph*{Object Characteristics}  68 cases (76\%) involved sharp object ingestion \cite{Elghali_2016, Karp_1991b, Lee_2007}, 32 cases (36\%) involved long (\textgreater 5cm) object ingestion \cite{Lee_2007}, 25 cases (28\%) involved ingestion of multiple objects \cite{Elghali_2016, Lee_2007}. No cases were reported for button battery ingestion, magnet ingestion, and involved large diameter (\textgreater 2.5cm) object ingestion.
\paragraph*{Outcomes}  47 cases (52\%) underwent endoscopic intervention \cite{Elghali_2016, Lee_2007}, 29 cases (32\%) were managed conservatively \cite{Elghali_2016, Karp_1991b}, 15 cases (17\%) underwent surgical intervention \cite{Elghali_2016, Karp_1991b, Lee_2007}, 6 cases (7\%) reported complications \cite{Lee_2007}, 1 case (1\%) died \cite{Elghali_2016}.
\paragraph*{Gender} 43 cases (60\%) were male \cite{Akay_2015f, Al-Faham_2020k, Alao_2006i, Ali_2017, Ali_2022g, Apikotoa_2022f, Atayan_2016, Benoist_2019e, Berry_2021e, Bhumi_2024f, CamachoDorado_2018, Csaky_1998e, Emamhadi_2018, Farhadi_2024h, Fry_2010, Gardner_2017h, Guinan_2019f, Jehangir_2019h, Jin_2023, Kobiela_2015, Kumar_2001, Kumar_2019f, Liu_2005, Losanoff_1996, Losanoff_1997e, Mesfin_2022a, Misra_2013, Qureshi_2016, Riva_2018j, Sobnach_2011f, Tammana_2012j, Tanrikulu_2015e, Tay_2004, Thapa_2019f, Trgo_2012f, Wadhwa_2015e, Yasin_2009, teWildt_2010}, 28 cases (39\%) were female \cite{AlShaaibi_2021b, Ali_2020f, Ataya_2013, Beecroft_1998, Bhasin_2014, Bhattacharjee_2008, Cauchi_2002, Chang_2017f, Cox_2007, DelgadoSalazar_2020c, DivsalarP._2023a, Goldman_1998f, Hardy_2023g, Kar_2015, Kariholu_2008, Kerestes_2019, Li_2013, Naji_2012f, Ohno_2005, Peixoto_2017f, Sakellaridis_2008f, Sultan_2024f, Tupesis_2004f, Wildhaber_2005, Wnęk_2015f, Yildiz_2016e}, 1 case (1\%) had no gender recorded \cite{fjbuilsRepeatedBehaviorDeliberate2024}. \paragraph*{Age Group} 25 cases (35\%) were between 26 and 40 years of age \cite{Alao_2006i, Ali_2022g, Apikotoa_2022f, Ataya_2013, Benoist_2019e, Bhasin_2014, Chang_2017f, Cox_2007, DelgadoSalazar_2020c, Farhadi_2024h, Fry_2010, Gardner_2017h, Guinan_2019f, Jin_2023, Kumar_2019f, Losanoff_1996, Misra_2013, Qureshi_2016, Riva_2018j, Sakellaridis_2008f, Tammana_2012j, Trgo_2012f, Wnęk_2015f, Yildiz_2016e, fjbuilsRepeatedBehaviorDeliberate2024}, 18 cases (25\%) were between 18 and 25 years of age \cite{Akay_2015f, Ali_2017, Atayan_2016, Bhattacharjee_2008, Csaky_1998e, Kar_2015, Kariholu_2008, Kobiela_2015, Losanoff_1996, Losanoff_1997e, Mesfin_2022a, Peixoto_2017f, Sobnach_2011f, Tupesis_2004f, Yasin_2009}, 13 cases (18\%) were under 18 years of age \cite{AlShaaibi_2021b, Ali_2020f, Cauchi_2002, DivsalarP._2023a, Goldman_1998f, Liu_2005, Naji_2012f, Ohno_2005, Tanrikulu_2015e, Tay_2004, Wildhaber_2005}, 11 cases (15\%) were between 41 and 60 years of age \cite{Al-Faham_2020k, Bhumi_2024f, CamachoDorado_2018, Emamhadi_2018, Hardy_2023g, Jehangir_2019h, Kumar_2001, Sultan_2024f, Thapa_2019f, Wadhwa_2015e, teWildt_2010}, 3 cases (4\%) were over 60 years of age \cite{Beecroft_1998, Kerestes_2019, Li_2013}, 2 cases (3\%) had no age documented \cite{Berry_2021e}. All 90 were male gender. 90 cases (100\%) were detained at the time of ingestion \cite{Elghali_2016, Karp_1991b, Lee_2007}, 88 cases (98\%) were intentional ingestions \cite{Elghali_2016, Karp_1991b, Lee_2007}, 30 cases (33\%) had a psychiatric history documented \cite{Elghali_2016, Karp_1991b, Lee_2007}, 2 cases (2\%) had a history of prior ingestion \cite{Elghali_2016}. No cases were reported for were psychiatric inpatients, were displaced people, were under the influence of alcohol at the time of ingestion, and had a severe disability history.
\paragraph*{Motivation}  70 cases (78\%) reported protest motivation \cite{Elghali_2016, Karp_1991b, Lee_2007}, 12 cases (13\%) reported psychiatric motivation \cite{Karp_1991b}, 6 cases (7\%) reported self-harm motivation \cite{Elghali_2016, Karp_1991b}. No cases were reported for psychosocial motivation and other motivation.
\paragraph*{Object Characteristics}  68 cases (76\%) involved sharp object ingestion \cite{Elghali_2016, Karp_1991b, Lee_2007}, 32 cases (36\%) involved long (\textgreater 5cm) object ingestion \cite{Lee_2007}, 25 cases (28\%) involved ingestion of multiple objects \cite{Elghali_2016, Lee_2007}. No cases were reported for button battery ingestion, magnet ingestion, and involved large diameter (\textgreater 2.5cm) object ingestion.
\paragraph*{Outcomes}  47 cases (52\%) underwent endoscopic intervention \cite{Elghali_2016, Lee_2007}, 29 cases (32\%) were managed conservatively \cite{Elghali_2016, Karp_1991b}, 15 cases (17\%) underwent surgical intervention \cite{Elghali_2016, Karp_1991b, Lee_2007}, 6 cases (7\%) reported complications \cite{Lee_2007}, 1 case (1\%) died \cite{Elghali_2016}.
\paragraph*{Gender} 43 cases (60\%) were male \cite{Akay_2015f, Al-Faham_2020k, Alao_2006i, Ali_2017, Ali_2022g, Apikotoa_2022f, Atayan_2016, Benoist_2019e, Berry_2021e, Bhumi_2024f, CamachoDorado_2018, Csaky_1998e, Emamhadi_2018, Farhadi_2024h, Fry_2010, Gardner_2017h, Guinan_2019f, Jehangir_2019h, Jin_2023, Kobiela_2015, Kumar_2001, Kumar_2019f, Liu_2005, Losanoff_1996, Losanoff_1997e, Mesfin_2022a, Misra_2013, Qureshi_2016, Riva_2018j, Sobnach_2011f, Tammana_2012j, Tanrikulu_2015e, Tay_2004, Thapa_2019f, Trgo_2012f, Wadhwa_2015e, Yasin_2009, teWildt_2010}, 28 cases (39\%) were female \cite{AlShaaibi_2021b, Ali_2020f, Ataya_2013, Beecroft_1998, Bhasin_2014, Bhattacharjee_2008, Cauchi_2002, Chang_2017f, Cox_2007, DelgadoSalazar_2020c, DivsalarP._2023a, Goldman_1998f, Hardy_2023g, Kar_2015, Kariholu_2008, Kerestes_2019, Li_2013, Naji_2012f, Ohno_2005, Peixoto_2017f, Sakellaridis_2008f, Sultan_2024f, Tupesis_2004f, Wildhaber_2005, Wnęk_2015f, Yildiz_2016e}, 1 case (1\%) had no gender recorded \cite{fjbuilsRepeatedBehaviorDeliberate2024}. \paragraph*{Age Group} 25 cases (35\%) were between 26 and 40 years of age \cite{Alao_2006i, Ali_2022g, Apikotoa_2022f, Ataya_2013, Benoist_2019e, Bhasin_2014, Chang_2017f, Cox_2007, DelgadoSalazar_2020c, Farhadi_2024h, Fry_2010, Gardner_2017h, Guinan_2019f, Jin_2023, Kumar_2019f, Losanoff_1996, Misra_2013, Qureshi_2016, Riva_2018j, Sakellaridis_2008f, Tammana_2012j, Trgo_2012f, Wnęk_2015f, Yildiz_2016e, fjbuilsRepeatedBehaviorDeliberate2024}, 18 cases (25\%) were between 18 and 25 years of age \cite{Akay_2015f, Ali_2017, Atayan_2016, Bhattacharjee_2008, Csaky_1998e, Kar_2015, Kariholu_2008, Kobiela_2015, Losanoff_1996, Losanoff_1997e, Mesfin_2022a, Peixoto_2017f, Sobnach_2011f, Tupesis_2004f, Yasin_2009}, 13 cases (18\%) were under 18 years of age \cite{AlShaaibi_2021b, Ali_2020f, Cauchi_2002, DivsalarP._2023a, Goldman_1998f, Liu_2005, Naji_2012f, Ohno_2005, Tanrikulu_2015e, Tay_2004, Wildhaber_2005}, 11 cases (15\%) were between 41 and 60 years of age \cite{Al-Faham_2020k, Bhumi_2024f, CamachoDorado_2018, Emamhadi_2018, Hardy_2023g, Jehangir_2019h, Kumar_2001, Sultan_2024f, Thapa_2019f, Wadhwa_2015e, teWildt_2010}, 3 cases (4\%) were over 60 years of age \cite{Beecroft_1998, Kerestes_2019, Li_2013}, 2 cases (3\%) had no age documented \cite{Berry_2021e}. All 90 were male gender. 90 cases (100\%) were detained at the time of ingestion \cite{Elghali_2016, Karp_1991b, Lee_2007}, 88 cases (98\%) were intentional ingestions \cite{Elghali_2016, Karp_1991b, Lee_2007}, 30 cases (33\%) had a psychiatric history documented \cite{Elghali_2016, Karp_1991b, Lee_2007}, 2 cases (2\%) had a history of prior ingestion \cite{Elghali_2016}. No cases were reported for were psychiatric inpatients, were displaced people, were under the influence of alcohol at the time of ingestion, and had a severe disability history.
\paragraph*{Motivation}  70 cases (78\%) reported protest motivation \cite{Elghali_2016, Karp_1991b, Lee_2007}, 12 cases (13\%) reported psychiatric motivation \cite{Karp_1991b}, 6 cases (7\%) reported self-harm motivation \cite{Elghali_2016, Karp_1991b}. No cases were reported for psychosocial motivation and other motivation.
\paragraph*{Object Characteristics}  68 cases (76\%) involved sharp object ingestion \cite{Elghali_2016, Karp_1991b, Lee_2007}, 32 cases (36\%) involved long (\textgreater 5cm) object ingestion \cite{Lee_2007}, 25 cases (28\%) involved ingestion of multiple objects \cite{Elghali_2016, Lee_2007}. No cases were reported for button battery ingestion, magnet ingestion, and involved large diameter (\textgreater 2.5cm) object ingestion.
\paragraph*{Outcomes}  47 cases (52\%) underwent endoscopic intervention \cite{Elghali_2016, Lee_2007}, 29 cases (32\%) were managed conservatively \cite{Elghali_2016, Karp_1991b}, 15 cases (17\%) underwent surgical intervention \cite{Elghali_2016, Karp_1991b, Lee_2007}, 6 cases (7\%) reported complications \cite{Lee_2007}, 1 case (1\%) died \cite{Elghali_2016}.
\paragraph*{Gender} 43 cases (60\%) were male \cite{Akay_2015f, Al-Faham_2020k, Alao_2006i, Ali_2017, Ali_2022g, Apikotoa_2022f, Atayan_2016, Benoist_2019e, Berry_2021e, Bhumi_2024f, CamachoDorado_2018, Csaky_1998e, Emamhadi_2018, Farhadi_2024h, Fry_2010, Gardner_2017h, Guinan_2019f, Jehangir_2019h, Jin_2023, Kobiela_2015, Kumar_2001, Kumar_2019f, Liu_2005, Losanoff_1996, Losanoff_1997e, Mesfin_2022a, Misra_2013, Qureshi_2016, Riva_2018j, Sobnach_2011f, Tammana_2012j, Tanrikulu_2015e, Tay_2004, Thapa_2019f, Trgo_2012f, Wadhwa_2015e, Yasin_2009, teWildt_2010}, 28 cases (39\%) were female \cite{AlShaaibi_2021b, Ali_2020f, Ataya_2013, Beecroft_1998, Bhasin_2014, Bhattacharjee_2008, Cauchi_2002, Chang_2017f, Cox_2007, DelgadoSalazar_2020c, DivsalarP._2023a, Goldman_1998f, Hardy_2023g, Kar_2015, Kariholu_2008, Kerestes_2019, Li_2013, Naji_2012f, Ohno_2005, Peixoto_2017f, Sakellaridis_2008f, Sultan_2024f, Tupesis_2004f, Wildhaber_2005, Wnęk_2015f, Yildiz_2016e}, 1 case (1\%) had no gender recorded \cite{fjbuilsRepeatedBehaviorDeliberate2024}. \paragraph*{Age Group} 25 cases (35\%) were between 26 and 40 years of age \cite{Alao_2006i, Ali_2022g, Apikotoa_2022f, Ataya_2013, Benoist_2019e, Bhasin_2014, Chang_2017f, Cox_2007, DelgadoSalazar_2020c, Farhadi_2024h, Fry_2010, Gardner_2017h, Guinan_2019f, Jin_2023, Kumar_2019f, Losanoff_1996, Misra_2013, Qureshi_2016, Riva_2018j, Sakellaridis_2008f, Tammana_2012j, Trgo_2012f, Wnęk_2015f, Yildiz_2016e, fjbuilsRepeatedBehaviorDeliberate2024}, 18 cases (25\%) were between 18 and 25 years of age \cite{Akay_2015f, Ali_2017, Atayan_2016, Bhattacharjee_2008, Csaky_1998e, Kar_2015, Kariholu_2008, Kobiela_2015, Losanoff_1996, Losanoff_1997e, Mesfin_2022a, Peixoto_2017f, Sobnach_2011f, Tupesis_2004f, Yasin_2009}, 13 cases (18\%) were under 18 years of age \cite{AlShaaibi_2021b, Ali_2020f, Cauchi_2002, DivsalarP._2023a, Goldman_1998f, Liu_2005, Naji_2012f, Ohno_2005, Tanrikulu_2015e, Tay_2004, Wildhaber_2005}, 11 cases (15\%) were between 41 and 60 years of age \cite{Al-Faham_2020k, Bhumi_2024f, CamachoDorado_2018, Emamhadi_2018, Hardy_2023g, Jehangir_2019h, Kumar_2001, Sultan_2024f, Thapa_2019f, Wadhwa_2015e, teWildt_2010}, 3 cases (4\%) were over 60 years of age \cite{Beecroft_1998, Kerestes_2019, Li_2013}, 2 cases (3\%) had no age documented \cite{Berry_2021e}. All 90 were male gender. 90 cases (100\%) were detained at the time of ingestion \cite{Elghali_2016, Karp_1991b, Lee_2007}, 88 cases (98\%) were intentional ingestions \cite{Elghali_2016, Karp_1991b, Lee_2007}, 30 cases (33\%) had a psychiatric history documented \cite{Elghali_2016, Karp_1991b, Lee_2007}, 2 cases (2\%) had a history of prior ingestion \cite{Elghali_2016}. No cases were reported for were psychiatric inpatients, were displaced people, were under the influence of alcohol at the time of ingestion, and had a severe disability history.
\paragraph*{Motivation}  70 cases (78\%) reported protest motivation \cite{Elghali_2016, Karp_1991b, Lee_2007}, 12 cases (13\%) reported psychiatric motivation \cite{Karp_1991b}, 6 cases (7\%) reported self-harm motivation \cite{Elghali_2016, Karp_1991b}. No cases were reported for psychosocial motivation and other motivation.
\paragraph*{Object Characteristics}  68 cases (76\%) involved sharp object ingestion \cite{Elghali_2016, Karp_1991b, Lee_2007}, 32 cases (36\%) involved long (\textgreater 5cm) object ingestion \cite{Lee_2007}, 25 cases (28\%) involved ingestion of multiple objects \cite{Elghali_2016, Lee_2007}. No cases were reported for button battery ingestion, magnet ingestion, and involved large diameter (\textgreater 2.5cm) object ingestion.
\paragraph*{Outcomes}  47 cases (52\%) underwent endoscopic intervention \cite{Elghali_2016, Lee_2007}, 29 cases (32\%) were managed conservatively \cite{Elghali_2016, Karp_1991b}, 15 cases (17\%) underwent surgical intervention \cite{Elghali_2016, Karp_1991b, Lee_2007}, 6 cases (7\%) reported complications \cite{Lee_2007}, 1 case (1\%) died \cite{Elghali_2016}.
\paragraph*{Gender} 43 cases (60\%) were male \cite{Akay_2015f, Al-Faham_2020k, Alao_2006i, Ali_2017, Ali_2022g, Apikotoa_2022f, Atayan_2016, Benoist_2019e, Berry_2021e, Bhumi_2024f, CamachoDorado_2018, Csaky_1998e, Emamhadi_2018, Farhadi_2024h, Fry_2010, Gardner_2017h, Guinan_2019f, Jehangir_2019h, Jin_2023, Kobiela_2015, Kumar_2001, Kumar_2019f, Liu_2005, Losanoff_1996, Losanoff_1997e, Mesfin_2022a, Misra_2013, Qureshi_2016, Riva_2018j, Sobnach_2011f, Tammana_2012j, Tanrikulu_2015e, Tay_2004, Thapa_2019f, Trgo_2012f, Wadhwa_2015e, Yasin_2009, teWildt_2010}, 28 cases (39\%) were female \cite{AlShaaibi_2021b, Ali_2020f, Ataya_2013, Beecroft_1998, Bhasin_2014, Bhattacharjee_2008, Cauchi_2002, Chang_2017f, Cox_2007, DelgadoSalazar_2020c, DivsalarP._2023a, Goldman_1998f, Hardy_2023g, Kar_2015, Kariholu_2008, Kerestes_2019, Li_2013, Naji_2012f, Ohno_2005, Peixoto_2017f, Sakellaridis_2008f, Sultan_2024f, Tupesis_2004f, Wildhaber_2005, Wnęk_2015f, Yildiz_2016e}, 1 case (1\%) had no gender recorded \cite{fjbuilsRepeatedBehaviorDeliberate2024}. \paragraph*{Age Group} 25 cases (35\%) were between 26 and 40 years of age \cite{Alao_2006i, Ali_2022g, Apikotoa_2022f, Ataya_2013, Benoist_2019e, Bhasin_2014, Chang_2017f, Cox_2007, DelgadoSalazar_2020c, Farhadi_2024h, Fry_2010, Gardner_2017h, Guinan_2019f, Jin_2023, Kumar_2019f, Losanoff_1996, Misra_2013, Qureshi_2016, Riva_2018j, Sakellaridis_2008f, Tammana_2012j, Trgo_2012f, Wnęk_2015f, Yildiz_2016e, fjbuilsRepeatedBehaviorDeliberate2024}, 18 cases (25\%) were between 18 and 25 years of age \cite{Akay_2015f, Ali_2017, Atayan_2016, Bhattacharjee_2008, Csaky_1998e, Kar_2015, Kariholu_2008, Kobiela_2015, Losanoff_1996, Losanoff_1997e, Mesfin_2022a, Peixoto_2017f, Sobnach_2011f, Tupesis_2004f, Yasin_2009}, 13 cases (18\%) were under 18 years of age \cite{AlShaaibi_2021b, Ali_2020f, Cauchi_2002, DivsalarP._2023a, Goldman_1998f, Liu_2005, Naji_2012f, Ohno_2005, Tanrikulu_2015e, Tay_2004, Wildhaber_2005}, 11 cases (15\%) were between 41 and 60 years of age \cite{Al-Faham_2020k, Bhumi_2024f, CamachoDorado_2018, Emamhadi_2018, Hardy_2023g, Jehangir_2019h, Kumar_2001, Sultan_2024f, Thapa_2019f, Wadhwa_2015e, teWildt_2010}, 3 cases (4\%) were over 60 years of age \cite{Beecroft_1998, Kerestes_2019, Li_2013}, 2 cases (3\%) had no age documented \cite{Berry_2021e}. All 90 were male gender. 90 cases (100\%) were detained at the time of ingestion \cite{Elghali_2016, Karp_1991b, Lee_2007}, 88 cases (98\%) were intentional ingestions \cite{Elghali_2016, Karp_1991b, Lee_2007}, 30 cases (33\%) had a psychiatric history documented \cite{Elghali_2016, Karp_1991b, Lee_2007}, 2 cases (2\%) had a history of prior ingestion \cite{Elghali_2016}. No cases were reported for were psychiatric inpatients, were displaced people, were under the influence of alcohol at the time of ingestion, and had a severe disability history.
\paragraph*{Motivation}  70 cases (78\%) reported protest motivation \cite{Elghali_2016, Karp_1991b, Lee_2007}, 12 cases (13\%) reported psychiatric motivation \cite{Karp_1991b}, 6 cases (7\%) reported self-harm motivation \cite{Elghali_2016, Karp_1991b}. No cases were reported for psychosocial motivation and other motivation.
\paragraph*{Object Characteristics}  68 cases (76\%) involved sharp object ingestion \cite{Elghali_2016, Karp_1991b, Lee_2007}, 32 cases (36\%) involved long (\textgreater 5cm) object ingestion \cite{Lee_2007}, 25 cases (28\%) involved ingestion of multiple objects \cite{Elghali_2016, Lee_2007}. No cases were reported for button battery ingestion, magnet ingestion, and involved large diameter (\textgreater 2.5cm) object ingestion.
\paragraph*{Outcomes}  47 cases (52\%) underwent endoscopic intervention \cite{Elghali_2016, Lee_2007}, 29 cases (32\%) were managed conservatively \cite{Elghali_2016, Karp_1991b}, 15 cases (17\%) underwent surgical intervention \cite{Elghali_2016, Karp_1991b, Lee_2007}, 6 cases (7\%) reported complications \cite{Lee_2007}, 1 case (1\%) died \cite{Elghali_2016}.
\paragraph*{Gender} 43 cases (60\%) were male \cite{Akay_2015f, Al-Faham_2020k, Alao_2006i, Ali_2017, Ali_2022g, Apikotoa_2022f, Atayan_2016, Benoist_2019e, Berry_2021e, Bhumi_2024f, CamachoDorado_2018, Csaky_1998e, Emamhadi_2018, Farhadi_2024h, Fry_2010, Gardner_2017h, Guinan_2019f, Jehangir_2019h, Jin_2023, Kobiela_2015, Kumar_2001, Kumar_2019f, Liu_2005, Losanoff_1996, Losanoff_1997e, Mesfin_2022a, Misra_2013, Qureshi_2016, Riva_2018j, Sobnach_2011f, Tammana_2012j, Tanrikulu_2015e, Tay_2004, Thapa_2019f, Trgo_2012f, Wadhwa_2015e, Yasin_2009, teWildt_2010}, 28 cases (39\%) were female \cite{AlShaaibi_2021b, Ali_2020f, Ataya_2013, Beecroft_1998, Bhasin_2014, Bhattacharjee_2008, Cauchi_2002, Chang_2017f, Cox_2007, DelgadoSalazar_2020c, DivsalarP._2023a, Goldman_1998f, Hardy_2023g, Kar_2015, Kariholu_2008, Kerestes_2019, Li_2013, Naji_2012f, Ohno_2005, Peixoto_2017f, Sakellaridis_2008f, Sultan_2024f, Tupesis_2004f, Wildhaber_2005, Wnęk_2015f, Yildiz_2016e}, 1 case (1\%) had no gender recorded \cite{fjbuilsRepeatedBehaviorDeliberate2024}. \paragraph*{Age Group} 25 cases (35\%) were between 26 and 40 years of age \cite{Alao_2006i, Ali_2022g, Apikotoa_2022f, Ataya_2013, Benoist_2019e, Bhasin_2014, Chang_2017f, Cox_2007, DelgadoSalazar_2020c, Farhadi_2024h, Fry_2010, Gardner_2017h, Guinan_2019f, Jin_2023, Kumar_2019f, Losanoff_1996, Misra_2013, Qureshi_2016, Riva_2018j, Sakellaridis_2008f, Tammana_2012j, Trgo_2012f, Wnęk_2015f, Yildiz_2016e, fjbuilsRepeatedBehaviorDeliberate2024}, 18 cases (25\%) were between 18 and 25 years of age \cite{Akay_2015f, Ali_2017, Atayan_2016, Bhattacharjee_2008, Csaky_1998e, Kar_2015, Kariholu_2008, Kobiela_2015, Losanoff_1996, Losanoff_1997e, Mesfin_2022a, Peixoto_2017f, Sobnach_2011f, Tupesis_2004f, Yasin_2009}, 13 cases (18\%) were under 18 years of age \cite{AlShaaibi_2021b, Ali_2020f, Cauchi_2002, DivsalarP._2023a, Goldman_1998f, Liu_2005, Naji_2012f, Ohno_2005, Tanrikulu_2015e, Tay_2004, Wildhaber_2005}, 11 cases (15\%) were between 41 and 60 years of age \cite{Al-Faham_2020k, Bhumi_2024f, CamachoDorado_2018, Emamhadi_2018, Hardy_2023g, Jehangir_2019h, Kumar_2001, Sultan_2024f, Thapa_2019f, Wadhwa_2015e, teWildt_2010}, 3 cases (4\%) were over 60 years of age \cite{Beecroft_1998, Kerestes_2019, Li_2013}, 2 cases (3\%) had no age documented \cite{Berry_2021e}. All 90 were male gender. 90 cases (100\%) were detained at the time of ingestion \cite{Elghali_2016, Karp_1991b, Lee_2007}, 88 cases (98\%) were intentional ingestions \cite{Elghali_2016, Karp_1991b, Lee_2007}, 30 cases (33\%) had a psychiatric history documented \cite{Elghali_2016, Karp_1991b, Lee_2007}, 2 cases (2\%) had a history of prior ingestion \cite{Elghali_2016}. No cases were reported for were psychiatric inpatients, were displaced people, were under the influence of alcohol at the time of ingestion, and had a severe disability history.
\paragraph*{Motivation}  70 cases (78\%) reported protest motivation \cite{Elghali_2016, Karp_1991b, Lee_2007}, 12 cases (13\%) reported psychiatric motivation \cite{Karp_1991b}, 6 cases (7\%) reported self-harm motivation \cite{Elghali_2016, Karp_1991b}. No cases were reported for psychosocial motivation and other motivation.
\paragraph*{Object Characteristics}  68 cases (76\%) involved sharp object ingestion \cite{Elghali_2016, Karp_1991b, Lee_2007}, 32 cases (36\%) involved long (\textgreater 5cm) object ingestion \cite{Lee_2007}, 25 cases (28\%) involved ingestion of multiple objects \cite{Elghali_2016, Lee_2007}. No cases were reported for button battery ingestion, magnet ingestion, and involved large diameter (\textgreater 2.5cm) object ingestion.
\paragraph*{Outcomes}  47 cases (52\%) underwent endoscopic intervention \cite{Elghali_2016, Lee_2007}, 29 cases (32\%) were managed conservatively \cite{Elghali_2016, Karp_1991b}, 15 cases (17\%) underwent surgical intervention \cite{Elghali_2016, Karp_1991b, Lee_2007}, 6 cases (7\%) reported complications \cite{Lee_2007}, 1 case (1\%) died \cite{Elghali_2016}.
\paragraph*{Gender} 43 cases (60\%) were male \cite{Akay_2015f, Al-Faham_2020k, Alao_2006i, Ali_2017, Ali_2022g, Apikotoa_2022f, Atayan_2016, Benoist_2019e, Berry_2021e, Bhumi_2024f, CamachoDorado_2018, Csaky_1998e, Emamhadi_2018, Farhadi_2024h, Fry_2010, Gardner_2017h, Guinan_2019f, Jehangir_2019h, Jin_2023, Kobiela_2015, Kumar_2001, Kumar_2019f, Liu_2005, Losanoff_1996, Losanoff_1997e, Mesfin_2022a, Misra_2013, Qureshi_2016, Riva_2018j, Sobnach_2011f, Tammana_2012j, Tanrikulu_2015e, Tay_2004, Thapa_2019f, Trgo_2012f, Wadhwa_2015e, Yasin_2009, teWildt_2010}, 28 cases (39\%) were female \cite{AlShaaibi_2021b, Ali_2020f, Ataya_2013, Beecroft_1998, Bhasin_2014, Bhattacharjee_2008, Cauchi_2002, Chang_2017f, Cox_2007, DelgadoSalazar_2020c, DivsalarP._2023a, Goldman_1998f, Hardy_2023g, Kar_2015, Kariholu_2008, Kerestes_2019, Li_2013, Naji_2012f, Ohno_2005, Peixoto_2017f, Sakellaridis_2008f, Sultan_2024f, Tupesis_2004f, Wildhaber_2005, Wnęk_2015f, Yildiz_2016e}, 1 case (1\%) had no gender recorded \cite{fjbuilsRepeatedBehaviorDeliberate2024}. \paragraph*{Age Group} 25 cases (35\%) were between 26 and 40 years of age \cite{Alao_2006i, Ali_2022g, Apikotoa_2022f, Ataya_2013, Benoist_2019e, Bhasin_2014, Chang_2017f, Cox_2007, DelgadoSalazar_2020c, Farhadi_2024h, Fry_2010, Gardner_2017h, Guinan_2019f, Jin_2023, Kumar_2019f, Losanoff_1996, Misra_2013, Qureshi_2016, Riva_2018j, Sakellaridis_2008f, Tammana_2012j, Trgo_2012f, Wnęk_2015f, Yildiz_2016e, fjbuilsRepeatedBehaviorDeliberate2024}, 18 cases (25\%) were between 18 and 25 years of age \cite{Akay_2015f, Ali_2017, Atayan_2016, Bhattacharjee_2008, Csaky_1998e, Kar_2015, Kariholu_2008, Kobiela_2015, Losanoff_1996, Losanoff_1997e, Mesfin_2022a, Peixoto_2017f, Sobnach_2011f, Tupesis_2004f, Yasin_2009}, 13 cases (18\%) were under 18 years of age \cite{AlShaaibi_2021b, Ali_2020f, Cauchi_2002, DivsalarP._2023a, Goldman_1998f, Liu_2005, Naji_2012f, Ohno_2005, Tanrikulu_2015e, Tay_2004, Wildhaber_2005}, 11 cases (15\%) were between 41 and 60 years of age \cite{Al-Faham_2020k, Bhumi_2024f, CamachoDorado_2018, Emamhadi_2018, Hardy_2023g, Jehangir_2019h, Kumar_2001, Sultan_2024f, Thapa_2019f, Wadhwa_2015e, teWildt_2010}, 3 cases (4\%) were over 60 years of age \cite{Beecroft_1998, Kerestes_2019, Li_2013}, 2 cases (3\%) had no age documented \cite{Berry_2021e}. All 90 were male gender. 90 cases (100\%) were detained at the time of ingestion \cite{Elghali_2016, Karp_1991b, Lee_2007}, 88 cases (98\%) were intentional ingestions \cite{Elghali_2016, Karp_1991b, Lee_2007}, 30 cases (33\%) had a psychiatric history documented \cite{Elghali_2016, Karp_1991b, Lee_2007}, 2 cases (2\%) had a history of prior ingestion \cite{Elghali_2016}. No cases were reported for were psychiatric inpatients, were displaced people, were under the influence of alcohol at the time of ingestion, and had a severe disability history.
\paragraph*{Motivation}  70 cases (78\%) reported protest motivation \cite{Elghali_2016, Karp_1991b, Lee_2007}, 12 cases (13\%) reported psychiatric motivation \cite{Karp_1991b}, 6 cases (7\%) reported self-harm motivation \cite{Elghali_2016, Karp_1991b}. No cases were reported for psychosocial motivation and other motivation.
\paragraph*{Object Characteristics}  68 cases (76\%) involved sharp object ingestion \cite{Elghali_2016, Karp_1991b, Lee_2007}, 32 cases (36\%) involved long (\textgreater 5cm) object ingestion \cite{Lee_2007}, 25 cases (28\%) involved ingestion of multiple objects \cite{Elghali_2016, Lee_2007}. No cases were reported for button battery ingestion, magnet ingestion, and involved large diameter (\textgreater 2.5cm) object ingestion.
\paragraph*{Outcomes}  47 cases (52\%) underwent endoscopic intervention \cite{Elghali_2016, Lee_2007}, 29 cases (32\%) were managed conservatively \cite{Elghali_2016, Karp_1991b}, 15 cases (17\%) underwent surgical intervention \cite{Elghali_2016, Karp_1991b, Lee_2007}, 6 cases (7\%) reported complications \cite{Lee_2007}, 1 case (1\%) died \cite{Elghali_2016}.
\paragraph*{Gender} 43 cases (60\%) were male \cite{Akay_2015f, Al-Faham_2020k, Alao_2006i, Ali_2017, Ali_2022g, Apikotoa_2022f, Atayan_2016, Benoist_2019e, Berry_2021e, Bhumi_2024f, CamachoDorado_2018, Csaky_1998e, Emamhadi_2018, Farhadi_2024h, Fry_2010, Gardner_2017h, Guinan_2019f, Jehangir_2019h, Jin_2023, Kobiela_2015, Kumar_2001, Kumar_2019f, Liu_2005, Losanoff_1996, Losanoff_1997e, Mesfin_2022a, Misra_2013, Qureshi_2016, Riva_2018j, Sobnach_2011f, Tammana_2012j, Tanrikulu_2015e, Tay_2004, Thapa_2019f, Trgo_2012f, Wadhwa_2015e, Yasin_2009, teWildt_2010}, 28 cases (39\%) were female \cite{AlShaaibi_2021b, Ali_2020f, Ataya_2013, Beecroft_1998, Bhasin_2014, Bhattacharjee_2008, Cauchi_2002, Chang_2017f, Cox_2007, DelgadoSalazar_2020c, DivsalarP._2023a, Goldman_1998f, Hardy_2023g, Kar_2015, Kariholu_2008, Kerestes_2019, Li_2013, Naji_2012f, Ohno_2005, Peixoto_2017f, Sakellaridis_2008f, Sultan_2024f, Tupesis_2004f, Wildhaber_2005, Wnęk_2015f, Yildiz_2016e}, 1 case (1\%) had no gender recorded \cite{fjbuilsRepeatedBehaviorDeliberate2024}. \paragraph*{Age Group} 25 cases (35\%) were between 26 and 40 years of age \cite{Alao_2006i, Ali_2022g, Apikotoa_2022f, Ataya_2013, Benoist_2019e, Bhasin_2014, Chang_2017f, Cox_2007, DelgadoSalazar_2020c, Farhadi_2024h, Fry_2010, Gardner_2017h, Guinan_2019f, Jin_2023, Kumar_2019f, Losanoff_1996, Misra_2013, Qureshi_2016, Riva_2018j, Sakellaridis_2008f, Tammana_2012j, Trgo_2012f, Wnęk_2015f, Yildiz_2016e, fjbuilsRepeatedBehaviorDeliberate2024}, 18 cases (25\%) were between 18 and 25 years of age \cite{Akay_2015f, Ali_2017, Atayan_2016, Bhattacharjee_2008, Csaky_1998e, Kar_2015, Kariholu_2008, Kobiela_2015, Losanoff_1996, Losanoff_1997e, Mesfin_2022a, Peixoto_2017f, Sobnach_2011f, Tupesis_2004f, Yasin_2009}, 13 cases (18\%) were under 18 years of age \cite{AlShaaibi_2021b, Ali_2020f, Cauchi_2002, DivsalarP._2023a, Goldman_1998f, Liu_2005, Naji_2012f, Ohno_2005, Tanrikulu_2015e, Tay_2004, Wildhaber_2005}, 11 cases (15\%) were between 41 and 60 years of age \cite{Al-Faham_2020k, Bhumi_2024f, CamachoDorado_2018, Emamhadi_2018, Hardy_2023g, Jehangir_2019h, Kumar_2001, Sultan_2024f, Thapa_2019f, Wadhwa_2015e, teWildt_2010}, 3 cases (4\%) were over 60 years of age \cite{Beecroft_1998, Kerestes_2019, Li_2013}, 2 cases (3\%) had no age documented \cite{Berry_2021e}. All 90 were male gender. 90 cases (100\%) were detained at the time of ingestion \cite{Elghali_2016, Karp_1991b, Lee_2007}, 88 cases (98\%) were intentional ingestions \cite{Elghali_2016, Karp_1991b, Lee_2007}, 30 cases (33\%) had a psychiatric history documented \cite{Elghali_2016, Karp_1991b, Lee_2007}, 2 cases (2\%) had a history of prior ingestion \cite{Elghali_2016}. No cases were reported for were psychiatric inpatients, were displaced people, were under the influence of alcohol at the time of ingestion, and had a severe disability history.
\paragraph*{Motivation}  70 cases (78\%) reported protest motivation \cite{Elghali_2016, Karp_1991b, Lee_2007}, 12 cases (13\%) reported psychiatric motivation \cite{Karp_1991b}, 6 cases (7\%) reported self-harm motivation \cite{Elghali_2016, Karp_1991b}. No cases were reported for psychosocial motivation and other motivation.
\paragraph*{Object Characteristics}  68 cases (76\%) involved sharp object ingestion \cite{Elghali_2016, Karp_1991b, Lee_2007}, 32 cases (36\%) involved long (\textgreater 5cm) object ingestion \cite{Lee_2007}, 25 cases (28\%) involved ingestion of multiple objects \cite{Elghali_2016, Lee_2007}. No cases were reported for button battery ingestion, magnet ingestion, and involved large diameter (\textgreater 2.5cm) object ingestion.
\paragraph*{Outcomes}  47 cases (52\%) underwent endoscopic intervention \cite{Elghali_2016, Lee_2007}, 29 cases (32\%) were managed conservatively \cite{Elghali_2016, Karp_1991b}, 15 cases (17\%) underwent surgical intervention \cite{Elghali_2016, Karp_1991b, Lee_2007}, 6 cases (7\%) reported complications \cite{Lee_2007}, 1 case (1\%) died \cite{Elghali_2016}.
\paragraph*{Gender} 43 cases (60\%) were male \cite{Akay_2015f, Al-Faham_2020k, Alao_2006i, Ali_2017, Ali_2022g, Apikotoa_2022f, Atayan_2016, Benoist_2019e, Berry_2021e, Bhumi_2024f, CamachoDorado_2018, Csaky_1998e, Emamhadi_2018, Farhadi_2024h, Fry_2010, Gardner_2017h, Guinan_2019f, Jehangir_2019h, Jin_2023, Kobiela_2015, Kumar_2001, Kumar_2019f, Liu_2005, Losanoff_1996, Losanoff_1997e, Mesfin_2022a, Misra_2013, Qureshi_2016, Riva_2018j, Sobnach_2011f, Tammana_2012j, Tanrikulu_2015e, Tay_2004, Thapa_2019f, Trgo_2012f, Wadhwa_2015e, Yasin_2009, teWildt_2010}, 28 cases (39\%) were female \cite{AlShaaibi_2021b, Ali_2020f, Ataya_2013, Beecroft_1998, Bhasin_2014, Bhattacharjee_2008, Cauchi_2002, Chang_2017f, Cox_2007, DelgadoSalazar_2020c, DivsalarP._2023a, Goldman_1998f, Hardy_2023g, Kar_2015, Kariholu_2008, Kerestes_2019, Li_2013, Naji_2012f, Ohno_2005, Peixoto_2017f, Sakellaridis_2008f, Sultan_2024f, Tupesis_2004f, Wildhaber_2005, Wnęk_2015f, Yildiz_2016e}, 1 case (1\%) had no gender recorded \cite{fjbuilsRepeatedBehaviorDeliberate2024}. \paragraph*{Age Group} 25 cases (35\%) were between 26 and 40 years of age \cite{Alao_2006i, Ali_2022g, Apikotoa_2022f, Ataya_2013, Benoist_2019e, Bhasin_2014, Chang_2017f, Cox_2007, DelgadoSalazar_2020c, Farhadi_2024h, Fry_2010, Gardner_2017h, Guinan_2019f, Jin_2023, Kumar_2019f, Losanoff_1996, Misra_2013, Qureshi_2016, Riva_2018j, Sakellaridis_2008f, Tammana_2012j, Trgo_2012f, Wnęk_2015f, Yildiz_2016e, fjbuilsRepeatedBehaviorDeliberate2024}, 18 cases (25\%) were between 18 and 25 years of age \cite{Akay_2015f, Ali_2017, Atayan_2016, Bhattacharjee_2008, Csaky_1998e, Kar_2015, Kariholu_2008, Kobiela_2015, Losanoff_1996, Losanoff_1997e, Mesfin_2022a, Peixoto_2017f, Sobnach_2011f, Tupesis_2004f, Yasin_2009}, 13 cases (18\%) were under 18 years of age \cite{AlShaaibi_2021b, Ali_2020f, Cauchi_2002, DivsalarP._2023a, Goldman_1998f, Liu_2005, Naji_2012f, Ohno_2005, Tanrikulu_2015e, Tay_2004, Wildhaber_2005}, 11 cases (15\%) were between 41 and 60 years of age \cite{Al-Faham_2020k, Bhumi_2024f, CamachoDorado_2018, Emamhadi_2018, Hardy_2023g, Jehangir_2019h, Kumar_2001, Sultan_2024f, Thapa_2019f, Wadhwa_2015e, teWildt_2010}, 3 cases (4\%) were over 60 years of age \cite{Beecroft_1998, Kerestes_2019, Li_2013}, 2 cases (3\%) had no age documented \cite{Berry_2021e}. All 90 were male gender. 90 cases (100\%) were detained at the time of ingestion \cite{Elghali_2016, Karp_1991b, Lee_2007}, 88 cases (98\%) were intentional ingestions \cite{Elghali_2016, Karp_1991b, Lee_2007}, 30 cases (33\%) had a psychiatric history documented \cite{Elghali_2016, Karp_1991b, Lee_2007}, 2 cases (2\%) had a history of prior ingestion \cite{Elghali_2016}. No cases were reported for were psychiatric inpatients, were displaced people, were under the influence of alcohol at the time of ingestion, and had a severe disability history.
\paragraph*{Motivation}  70 cases (78\%) reported protest motivation \cite{Elghali_2016, Karp_1991b, Lee_2007}, 12 cases (13\%) reported psychiatric motivation \cite{Karp_1991b}, 6 cases (7\%) reported self-harm motivation \cite{Elghali_2016, Karp_1991b}. No cases were reported for psychosocial motivation and other motivation.
\paragraph*{Object Characteristics}  68 cases (76\%) involved sharp object ingestion \cite{Elghali_2016, Karp_1991b, Lee_2007}, 32 cases (36\%) involved long (\textgreater 5cm) object ingestion \cite{Lee_2007}, 25 cases (28\%) involved ingestion of multiple objects \cite{Elghali_2016, Lee_2007}. No cases were reported for button battery ingestion, magnet ingestion, and involved large diameter (\textgreater 2.5cm) object ingestion.
\paragraph*{Outcomes}  47 cases (52\%) underwent endoscopic intervention \cite{Elghali_2016, Lee_2007}, 29 cases (32\%) were managed conservatively \cite{Elghali_2016, Karp_1991b}, 15 cases (17\%) underwent surgical intervention \cite{Elghali_2016, Karp_1991b, Lee_2007}, 6 cases (7\%) reported complications \cite{Lee_2007}, 1 case (1\%) died \cite{Elghali_2016}.
\paragraph*{Gender} 43 cases (60\%) were male \cite{Akay_2015f, Al-Faham_2020k, Alao_2006i, Ali_2017, Ali_2022g, Apikotoa_2022f, Atayan_2016, Benoist_2019e, Berry_2021e, Bhumi_2024f, CamachoDorado_2018, Csaky_1998e, Emamhadi_2018, Farhadi_2024h, Fry_2010, Gardner_2017h, Guinan_2019f, Jehangir_2019h, Jin_2023, Kobiela_2015, Kumar_2001, Kumar_2019f, Liu_2005, Losanoff_1996, Losanoff_1997e, Mesfin_2022a, Misra_2013, Qureshi_2016, Riva_2018j, Sobnach_2011f, Tammana_2012j, Tanrikulu_2015e, Tay_2004, Thapa_2019f, Trgo_2012f, Wadhwa_2015e, Yasin_2009, teWildt_2010}, 28 cases (39\%) were female \cite{AlShaaibi_2021b, Ali_2020f, Ataya_2013, Beecroft_1998, Bhasin_2014, Bhattacharjee_2008, Cauchi_2002, Chang_2017f, Cox_2007, DelgadoSalazar_2020c, DivsalarP._2023a, Goldman_1998f, Hardy_2023g, Kar_2015, Kariholu_2008, Kerestes_2019, Li_2013, Naji_2012f, Ohno_2005, Peixoto_2017f, Sakellaridis_2008f, Sultan_2024f, Tupesis_2004f, Wildhaber_2005, Wnęk_2015f, Yildiz_2016e}, 1 case (1\%) had no gender recorded \cite{fjbuilsRepeatedBehaviorDeliberate2024}. \paragraph*{Age Group} 25 cases (35\%) were between 26 and 40 years of age \cite{Alao_2006i, Ali_2022g, Apikotoa_2022f, Ataya_2013, Benoist_2019e, Bhasin_2014, Chang_2017f, Cox_2007, DelgadoSalazar_2020c, Farhadi_2024h, Fry_2010, Gardner_2017h, Guinan_2019f, Jin_2023, Kumar_2019f, Losanoff_1996, Misra_2013, Qureshi_2016, Riva_2018j, Sakellaridis_2008f, Tammana_2012j, Trgo_2012f, Wnęk_2015f, Yildiz_2016e, fjbuilsRepeatedBehaviorDeliberate2024}, 18 cases (25\%) were between 18 and 25 years of age \cite{Akay_2015f, Ali_2017, Atayan_2016, Bhattacharjee_2008, Csaky_1998e, Kar_2015, Kariholu_2008, Kobiela_2015, Losanoff_1996, Losanoff_1997e, Mesfin_2022a, Peixoto_2017f, Sobnach_2011f, Tupesis_2004f, Yasin_2009}, 13 cases (18\%) were under 18 years of age \cite{AlShaaibi_2021b, Ali_2020f, Cauchi_2002, DivsalarP._2023a, Goldman_1998f, Liu_2005, Naji_2012f, Ohno_2005, Tanrikulu_2015e, Tay_2004, Wildhaber_2005}, 11 cases (15\%) were between 41 and 60 years of age \cite{Al-Faham_2020k, Bhumi_2024f, CamachoDorado_2018, Emamhadi_2018, Hardy_2023g, Jehangir_2019h, Kumar_2001, Sultan_2024f, Thapa_2019f, Wadhwa_2015e, teWildt_2010}, 3 cases (4\%) were over 60 years of age \cite{Beecroft_1998, Kerestes_2019, Li_2013}, 2 cases (3\%) had no age documented \cite{Berry_2021e}. All 90 were male gender. 90 cases (100\%) were detained at the time of ingestion \cite{Elghali_2016, Karp_1991b, Lee_2007}, 88 cases (98\%) were intentional ingestions \cite{Elghali_2016, Karp_1991b, Lee_2007}, 30 cases (33\%) had a psychiatric history documented \cite{Elghali_2016, Karp_1991b, Lee_2007}, 2 cases (2\%) had a history of prior ingestion \cite{Elghali_2016}. No cases were reported for were psychiatric inpatients, were displaced people, were under the influence of alcohol at the time of ingestion, and had a severe disability history.
\paragraph*{Motivation}  70 cases (78\%) reported protest motivation \cite{Elghali_2016, Karp_1991b, Lee_2007}, 12 cases (13\%) reported psychiatric motivation \cite{Karp_1991b}, 6 cases (7\%) reported self-harm motivation \cite{Elghali_2016, Karp_1991b}. No cases were reported for psychosocial motivation and other motivation.
\paragraph*{Object Characteristics}  68 cases (76\%) involved sharp object ingestion \cite{Elghali_2016, Karp_1991b, Lee_2007}, 32 cases (36\%) involved long (\textgreater 5cm) object ingestion \cite{Lee_2007}, 25 cases (28\%) involved ingestion of multiple objects \cite{Elghali_2016, Lee_2007}. No cases were reported for button battery ingestion, magnet ingestion, and involved large diameter (\textgreater 2.5cm) object ingestion.
\paragraph*{Outcomes}  47 cases (52\%) underwent endoscopic intervention \cite{Elghali_2016, Lee_2007}, 29 cases (32\%) were managed conservatively \cite{Elghali_2016, Karp_1991b}, 15 cases (17\%) underwent surgical intervention \cite{Elghali_2016, Karp_1991b, Lee_2007}, 6 cases (7\%) reported complications \cite{Lee_2007}, 1 case (1\%) died \cite{Elghali_2016}.
\paragraph*{Gender} 43 cases (60\%) were male \cite{Akay_2015f, Al-Faham_2020k, Alao_2006i, Ali_2017, Ali_2022g, Apikotoa_2022f, Atayan_2016, Benoist_2019e, Berry_2021e, Bhumi_2024f, CamachoDorado_2018, Csaky_1998e, Emamhadi_2018, Farhadi_2024h, Fry_2010, Gardner_2017h, Guinan_2019f, Jehangir_2019h, Jin_2023, Kobiela_2015, Kumar_2001, Kumar_2019f, Liu_2005, Losanoff_1996, Losanoff_1997e, Mesfin_2022a, Misra_2013, Qureshi_2016, Riva_2018j, Sobnach_2011f, Tammana_2012j, Tanrikulu_2015e, Tay_2004, Thapa_2019f, Trgo_2012f, Wadhwa_2015e, Yasin_2009, teWildt_2010}, 28 cases (39\%) were female \cite{AlShaaibi_2021b, Ali_2020f, Ataya_2013, Beecroft_1998, Bhasin_2014, Bhattacharjee_2008, Cauchi_2002, Chang_2017f, Cox_2007, DelgadoSalazar_2020c, DivsalarP._2023a, Goldman_1998f, Hardy_2023g, Kar_2015, Kariholu_2008, Kerestes_2019, Li_2013, Naji_2012f, Ohno_2005, Peixoto_2017f, Sakellaridis_2008f, Sultan_2024f, Tupesis_2004f, Wildhaber_2005, Wnęk_2015f, Yildiz_2016e}, 1 case (1\%) had no gender recorded \cite{fjbuilsRepeatedBehaviorDeliberate2024}. \paragraph*{Age Group} 25 cases (35\%) were between 26 and 40 years of age \cite{Alao_2006i, Ali_2022g, Apikotoa_2022f, Ataya_2013, Benoist_2019e, Bhasin_2014, Chang_2017f, Cox_2007, DelgadoSalazar_2020c, Farhadi_2024h, Fry_2010, Gardner_2017h, Guinan_2019f, Jin_2023, Kumar_2019f, Losanoff_1996, Misra_2013, Qureshi_2016, Riva_2018j, Sakellaridis_2008f, Tammana_2012j, Trgo_2012f, Wnęk_2015f, Yildiz_2016e, fjbuilsRepeatedBehaviorDeliberate2024}, 18 cases (25\%) were between 18 and 25 years of age \cite{Akay_2015f, Ali_2017, Atayan_2016, Bhattacharjee_2008, Csaky_1998e, Kar_2015, Kariholu_2008, Kobiela_2015, Losanoff_1996, Losanoff_1997e, Mesfin_2022a, Peixoto_2017f, Sobnach_2011f, Tupesis_2004f, Yasin_2009}, 13 cases (18\%) were under 18 years of age \cite{AlShaaibi_2021b, Ali_2020f, Cauchi_2002, DivsalarP._2023a, Goldman_1998f, Liu_2005, Naji_2012f, Ohno_2005, Tanrikulu_2015e, Tay_2004, Wildhaber_2005}, 11 cases (15\%) were between 41 and 60 years of age \cite{Al-Faham_2020k, Bhumi_2024f, CamachoDorado_2018, Emamhadi_2018, Hardy_2023g, Jehangir_2019h, Kumar_2001, Sultan_2024f, Thapa_2019f, Wadhwa_2015e, teWildt_2010}, 3 cases (4\%) were over 60 years of age \cite{Beecroft_1998, Kerestes_2019, Li_2013}, 2 cases (3\%) had no age documented \cite{Berry_2021e}. All 90 were male gender. 90 cases (100\%) were detained at the time of ingestion \cite{Elghali_2016, Karp_1991b, Lee_2007}, 88 cases (98\%) were intentional ingestions \cite{Elghali_2016, Karp_1991b, Lee_2007}, 30 cases (33\%) had a psychiatric history documented \cite{Elghali_2016, Karp_1991b, Lee_2007}, 2 cases (2\%) had a history of prior ingestion \cite{Elghali_2016}. No cases were reported for were psychiatric inpatients, were displaced people, were under the influence of alcohol at the time of ingestion, and had a severe disability history.
\paragraph*{Motivation}  70 cases (78\%) reported protest motivation \cite{Elghali_2016, Karp_1991b, Lee_2007}, 12 cases (13\%) reported psychiatric motivation \cite{Karp_1991b}, 6 cases (7\%) reported self-harm motivation \cite{Elghali_2016, Karp_1991b}. No cases were reported for psychosocial motivation and other motivation.
\paragraph*{Object Characteristics}  68 cases (76\%) involved sharp object ingestion \cite{Elghali_2016, Karp_1991b, Lee_2007}, 32 cases (36\%) involved long (\textgreater 5cm) object ingestion \cite{Lee_2007}, 25 cases (28\%) involved ingestion of multiple objects \cite{Elghali_2016, Lee_2007}. No cases were reported for button battery ingestion, magnet ingestion, and involved large diameter (\textgreater 2.5cm) object ingestion.
\paragraph*{Outcomes}  47 cases (52\%) underwent endoscopic intervention \cite{Elghali_2016, Lee_2007}, 29 cases (32\%) were managed conservatively \cite{Elghali_2016, Karp_1991b}, 15 cases (17\%) underwent surgical intervention \cite{Elghali_2016, Karp_1991b, Lee_2007}, 6 cases (7\%) reported complications \cite{Lee_2007}, 1 case (1\%) died \cite{Elghali_2016}.
\paragraph*{Gender} 43 cases (60\%) were male \cite{Akay_2015f, Al-Faham_2020k, Alao_2006i, Ali_2017, Ali_2022g, Apikotoa_2022f, Atayan_2016, Benoist_2019e, Berry_2021e, Bhumi_2024f, CamachoDorado_2018, Csaky_1998e, Emamhadi_2018, Farhadi_2024h, Fry_2010, Gardner_2017h, Guinan_2019f, Jehangir_2019h, Jin_2023, Kobiela_2015, Kumar_2001, Kumar_2019f, Liu_2005, Losanoff_1996, Losanoff_1997e, Mesfin_2022a, Misra_2013, Qureshi_2016, Riva_2018j, Sobnach_2011f, Tammana_2012j, Tanrikulu_2015e, Tay_2004, Thapa_2019f, Trgo_2012f, Wadhwa_2015e, Yasin_2009, teWildt_2010}, 28 cases (39\%) were female \cite{AlShaaibi_2021b, Ali_2020f, Ataya_2013, Beecroft_1998, Bhasin_2014, Bhattacharjee_2008, Cauchi_2002, Chang_2017f, Cox_2007, DelgadoSalazar_2020c, DivsalarP._2023a, Goldman_1998f, Hardy_2023g, Kar_2015, Kariholu_2008, Kerestes_2019, Li_2013, Naji_2012f, Ohno_2005, Peixoto_2017f, Sakellaridis_2008f, Sultan_2024f, Tupesis_2004f, Wildhaber_2005, Wnęk_2015f, Yildiz_2016e}, 1 case (1\%) had no gender recorded \cite{fjbuilsRepeatedBehaviorDeliberate2024}. \paragraph*{Age Group} 25 cases (35\%) were between 26 and 40 years of age \cite{Alao_2006i, Ali_2022g, Apikotoa_2022f, Ataya_2013, Benoist_2019e, Bhasin_2014, Chang_2017f, Cox_2007, DelgadoSalazar_2020c, Farhadi_2024h, Fry_2010, Gardner_2017h, Guinan_2019f, Jin_2023, Kumar_2019f, Losanoff_1996, Misra_2013, Qureshi_2016, Riva_2018j, Sakellaridis_2008f, Tammana_2012j, Trgo_2012f, Wnęk_2015f, Yildiz_2016e, fjbuilsRepeatedBehaviorDeliberate2024}, 18 cases (25\%) were between 18 and 25 years of age \cite{Akay_2015f, Ali_2017, Atayan_2016, Bhattacharjee_2008, Csaky_1998e, Kar_2015, Kariholu_2008, Kobiela_2015, Losanoff_1996, Losanoff_1997e, Mesfin_2022a, Peixoto_2017f, Sobnach_2011f, Tupesis_2004f, Yasin_2009}, 13 cases (18\%) were under 18 years of age \cite{AlShaaibi_2021b, Ali_2020f, Cauchi_2002, DivsalarP._2023a, Goldman_1998f, Liu_2005, Naji_2012f, Ohno_2005, Tanrikulu_2015e, Tay_2004, Wildhaber_2005}, 11 cases (15\%) were between 41 and 60 years of age \cite{Al-Faham_2020k, Bhumi_2024f, CamachoDorado_2018, Emamhadi_2018, Hardy_2023g, Jehangir_2019h, Kumar_2001, Sultan_2024f, Thapa_2019f, Wadhwa_2015e, teWildt_2010}, 3 cases (4\%) were over 60 years of age \cite{Beecroft_1998, Kerestes_2019, Li_2013}, 2 cases (3\%) had no age documented \cite{Berry_2021e}. All 90 were male gender. 90 cases (100\%) were detained at the time of ingestion \cite{Elghali_2016, Karp_1991b, Lee_2007}, 88 cases (98\%) were intentional ingestions \cite{Elghali_2016, Karp_1991b, Lee_2007}, 30 cases (33\%) had a psychiatric history documented \cite{Elghali_2016, Karp_1991b, Lee_2007}, 2 cases (2\%) had a history of prior ingestion \cite{Elghali_2016}. No cases were reported for were psychiatric inpatients, were displaced people, were under the influence of alcohol at the time of ingestion, and had a severe disability history.
\paragraph*{Motivation}  70 cases (78\%) reported protest motivation \cite{Elghali_2016, Karp_1991b, Lee_2007}, 12 cases (13\%) reported psychiatric motivation \cite{Karp_1991b}, 6 cases (7\%) reported self-harm motivation \cite{Elghali_2016, Karp_1991b}. No cases were reported for psychosocial motivation and other motivation.
\paragraph*{Object Characteristics}  68 cases (76\%) involved sharp object ingestion \cite{Elghali_2016, Karp_1991b, Lee_2007}, 32 cases (36\%) involved long (\textgreater 5cm) object ingestion \cite{Lee_2007}, 25 cases (28\%) involved ingestion of multiple objects \cite{Elghali_2016, Lee_2007}. No cases were reported for button battery ingestion, magnet ingestion, and involved large diameter (\textgreater 2.5cm) object ingestion.
\paragraph*{Outcomes}  47 cases (52\%) underwent endoscopic intervention \cite{Elghali_2016, Lee_2007}, 29 cases (32\%) were managed conservatively \cite{Elghali_2016, Karp_1991b}, 15 cases (17\%) underwent surgical intervention \cite{Elghali_2016, Karp_1991b, Lee_2007}, 6 cases (7\%) reported complications \cite{Lee_2007}, 1 case (1\%) died \cite{Elghali_2016}.
\paragraph*{Gender} 43 cases (60\%) were male \cite{Akay_2015f, Al-Faham_2020k, Alao_2006i, Ali_2017, Ali_2022g, Apikotoa_2022f, Atayan_2016, Benoist_2019e, Berry_2021e, Bhumi_2024f, CamachoDorado_2018, Csaky_1998e, Emamhadi_2018, Farhadi_2024h, Fry_2010, Gardner_2017h, Guinan_2019f, Jehangir_2019h, Jin_2023, Kobiela_2015, Kumar_2001, Kumar_2019f, Liu_2005, Losanoff_1996, Losanoff_1997e, Mesfin_2022a, Misra_2013, Qureshi_2016, Riva_2018j, Sobnach_2011f, Tammana_2012j, Tanrikulu_2015e, Tay_2004, Thapa_2019f, Trgo_2012f, Wadhwa_2015e, Yasin_2009, teWildt_2010}, 28 cases (39\%) were female \cite{AlShaaibi_2021b, Ali_2020f, Ataya_2013, Beecroft_1998, Bhasin_2014, Bhattacharjee_2008, Cauchi_2002, Chang_2017f, Cox_2007, DelgadoSalazar_2020c, DivsalarP._2023a, Goldman_1998f, Hardy_2023g, Kar_2015, Kariholu_2008, Kerestes_2019, Li_2013, Naji_2012f, Ohno_2005, Peixoto_2017f, Sakellaridis_2008f, Sultan_2024f, Tupesis_2004f, Wildhaber_2005, Wnęk_2015f, Yildiz_2016e}, 1 case (1\%) had no gender recorded \cite{fjbuilsRepeatedBehaviorDeliberate2024}. \paragraph*{Age Group} 25 cases (35\%) were between 26 and 40 years of age \cite{Alao_2006i, Ali_2022g, Apikotoa_2022f, Ataya_2013, Benoist_2019e, Bhasin_2014, Chang_2017f, Cox_2007, DelgadoSalazar_2020c, Farhadi_2024h, Fry_2010, Gardner_2017h, Guinan_2019f, Jin_2023, Kumar_2019f, Losanoff_1996, Misra_2013, Qureshi_2016, Riva_2018j, Sakellaridis_2008f, Tammana_2012j, Trgo_2012f, Wnęk_2015f, Yildiz_2016e, fjbuilsRepeatedBehaviorDeliberate2024}, 18 cases (25\%) were between 18 and 25 years of age \cite{Akay_2015f, Ali_2017, Atayan_2016, Bhattacharjee_2008, Csaky_1998e, Kar_2015, Kariholu_2008, Kobiela_2015, Losanoff_1996, Losanoff_1997e, Mesfin_2022a, Peixoto_2017f, Sobnach_2011f, Tupesis_2004f, Yasin_2009}, 13 cases (18\%) were under 18 years of age \cite{AlShaaibi_2021b, Ali_2020f, Cauchi_2002, DivsalarP._2023a, Goldman_1998f, Liu_2005, Naji_2012f, Ohno_2005, Tanrikulu_2015e, Tay_2004, Wildhaber_2005}, 11 cases (15\%) were between 41 and 60 years of age \cite{Al-Faham_2020k, Bhumi_2024f, CamachoDorado_2018, Emamhadi_2018, Hardy_2023g, Jehangir_2019h, Kumar_2001, Sultan_2024f, Thapa_2019f, Wadhwa_2015e, teWildt_2010}, 3 cases (4\%) were over 60 years of age \cite{Beecroft_1998, Kerestes_2019, Li_2013}, 2 cases (3\%) had no age documented \cite{Berry_2021e}. All 90 were male gender. 90 cases (100\%) were detained at the time of ingestion \cite{Elghali_2016, Karp_1991b, Lee_2007}, 88 cases (98\%) were intentional ingestions \cite{Elghali_2016, Karp_1991b, Lee_2007}, 30 cases (33\%) had a psychiatric history documented \cite{Elghali_2016, Karp_1991b, Lee_2007}, 2 cases (2\%) had a history of prior ingestion \cite{Elghali_2016}. No cases were reported for were psychiatric inpatients, were displaced people, were under the influence of alcohol at the time of ingestion, and had a severe disability history.
\paragraph*{Motivation}  70 cases (78\%) reported protest motivation \cite{Elghali_2016, Karp_1991b, Lee_2007}, 12 cases (13\%) reported psychiatric motivation \cite{Karp_1991b}, 6 cases (7\%) reported self-harm motivation \cite{Elghali_2016, Karp_1991b}. No cases were reported for psychosocial motivation and other motivation.
\paragraph*{Object Characteristics}  68 cases (76\%) involved sharp object ingestion \cite{Elghali_2016, Karp_1991b, Lee_2007}, 32 cases (36\%) involved long (\textgreater 5cm) object ingestion \cite{Lee_2007}, 25 cases (28\%) involved ingestion of multiple objects \cite{Elghali_2016, Lee_2007}. No cases were reported for button battery ingestion, magnet ingestion, and involved large diameter (\textgreater 2.5cm) object ingestion.
\paragraph*{Outcomes}  47 cases (52\%) underwent endoscopic intervention \cite{Elghali_2016, Lee_2007}, 29 cases (32\%) were managed conservatively \cite{Elghali_2016, Karp_1991b}, 15 cases (17\%) underwent surgical intervention \cite{Elghali_2016, Karp_1991b, Lee_2007}, 6 cases (7\%) reported complications \cite{Lee_2007}, 1 case (1\%) died \cite{Elghali_2016}.
\paragraph*{Gender} 43 cases (60\%) were male \cite{Akay_2015f, Al-Faham_2020k, Alao_2006i, Ali_2017, Ali_2022g, Apikotoa_2022f, Atayan_2016, Benoist_2019e, Berry_2021e, Bhumi_2024f, CamachoDorado_2018, Csaky_1998e, Emamhadi_2018, Farhadi_2024h, Fry_2010, Gardner_2017h, Guinan_2019f, Jehangir_2019h, Jin_2023, Kobiela_2015, Kumar_2001, Kumar_2019f, Liu_2005, Losanoff_1996, Losanoff_1997e, Mesfin_2022a, Misra_2013, Qureshi_2016, Riva_2018j, Sobnach_2011f, Tammana_2012j, Tanrikulu_2015e, Tay_2004, Thapa_2019f, Trgo_2012f, Wadhwa_2015e, Yasin_2009, teWildt_2010}, 28 cases (39\%) were female \cite{AlShaaibi_2021b, Ali_2020f, Ataya_2013, Beecroft_1998, Bhasin_2014, Bhattacharjee_2008, Cauchi_2002, Chang_2017f, Cox_2007, DelgadoSalazar_2020c, DivsalarP._2023a, Goldman_1998f, Hardy_2023g, Kar_2015, Kariholu_2008, Kerestes_2019, Li_2013, Naji_2012f, Ohno_2005, Peixoto_2017f, Sakellaridis_2008f, Sultan_2024f, Tupesis_2004f, Wildhaber_2005, Wnęk_2015f, Yildiz_2016e}, 1 case (1\%) had no gender recorded \cite{fjbuilsRepeatedBehaviorDeliberate2024}. \paragraph*{Age Group} 25 cases (35\%) were between 26 and 40 years of age \cite{Alao_2006i, Ali_2022g, Apikotoa_2022f, Ataya_2013, Benoist_2019e, Bhasin_2014, Chang_2017f, Cox_2007, DelgadoSalazar_2020c, Farhadi_2024h, Fry_2010, Gardner_2017h, Guinan_2019f, Jin_2023, Kumar_2019f, Losanoff_1996, Misra_2013, Qureshi_2016, Riva_2018j, Sakellaridis_2008f, Tammana_2012j, Trgo_2012f, Wnęk_2015f, Yildiz_2016e, fjbuilsRepeatedBehaviorDeliberate2024}, 18 cases (25\%) were between 18 and 25 years of age \cite{Akay_2015f, Ali_2017, Atayan_2016, Bhattacharjee_2008, Csaky_1998e, Kar_2015, Kariholu_2008, Kobiela_2015, Losanoff_1996, Losanoff_1997e, Mesfin_2022a, Peixoto_2017f, Sobnach_2011f, Tupesis_2004f, Yasin_2009}, 13 cases (18\%) were under 18 years of age \cite{AlShaaibi_2021b, Ali_2020f, Cauchi_2002, DivsalarP._2023a, Goldman_1998f, Liu_2005, Naji_2012f, Ohno_2005, Tanrikulu_2015e, Tay_2004, Wildhaber_2005}, 11 cases (15\%) were between 41 and 60 years of age \cite{Al-Faham_2020k, Bhumi_2024f, CamachoDorado_2018, Emamhadi_2018, Hardy_2023g, Jehangir_2019h, Kumar_2001, Sultan_2024f, Thapa_2019f, Wadhwa_2015e, teWildt_2010}, 3 cases (4\%) were over 60 years of age \cite{Beecroft_1998, Kerestes_2019, Li_2013}, 2 cases (3\%) had no age documented \cite{Berry_2021e}. All 90 were male gender. 90 cases (100\%) were detained at the time of ingestion \cite{Elghali_2016, Karp_1991b, Lee_2007}, 88 cases (98\%) were intentional ingestions \cite{Elghali_2016, Karp_1991b, Lee_2007}, 30 cases (33\%) had a psychiatric history documented \cite{Elghali_2016, Karp_1991b, Lee_2007}, 2 cases (2\%) had a history of prior ingestion \cite{Elghali_2016}. No cases were reported for were psychiatric inpatients, were displaced people, were under the influence of alcohol at the time of ingestion, and had a severe disability history.
\paragraph*{Motivation}  70 cases (78\%) reported protest motivation \cite{Elghali_2016, Karp_1991b, Lee_2007}, 12 cases (13\%) reported psychiatric motivation \cite{Karp_1991b}, 6 cases (7\%) reported self-harm motivation \cite{Elghali_2016, Karp_1991b}. No cases were reported for psychosocial motivation and other motivation.
\paragraph*{Object Characteristics}  68 cases (76\%) involved sharp object ingestion \cite{Elghali_2016, Karp_1991b, Lee_2007}, 32 cases (36\%) involved long (\textgreater 5cm) object ingestion \cite{Lee_2007}, 25 cases (28\%) involved ingestion of multiple objects \cite{Elghali_2016, Lee_2007}. No cases were reported for button battery ingestion, magnet ingestion, and involved large diameter (\textgreater 2.5cm) object ingestion.
\paragraph*{Outcomes}  47 cases (52\%) underwent endoscopic intervention \cite{Elghali_2016, Lee_2007}, 29 cases (32\%) were managed conservatively \cite{Elghali_2016, Karp_1991b}, 15 cases (17\%) underwent surgical intervention \cite{Elghali_2016, Karp_1991b, Lee_2007}, 6 cases (7\%) reported complications \cite{Lee_2007}, 1 case (1\%) died \cite{Elghali_2016}.
\paragraph*{Gender} 43 cases (60\%) were male \cite{Akay_2015f, Al-Faham_2020k, Alao_2006i, Ali_2017, Ali_2022g, Apikotoa_2022f, Atayan_2016, Benoist_2019e, Berry_2021e, Bhumi_2024f, CamachoDorado_2018, Csaky_1998e, Emamhadi_2018, Farhadi_2024h, Fry_2010, Gardner_2017h, Guinan_2019f, Jehangir_2019h, Jin_2023, Kobiela_2015, Kumar_2001, Kumar_2019f, Liu_2005, Losanoff_1996, Losanoff_1997e, Mesfin_2022a, Misra_2013, Qureshi_2016, Riva_2018j, Sobnach_2011f, Tammana_2012j, Tanrikulu_2015e, Tay_2004, Thapa_2019f, Trgo_2012f, Wadhwa_2015e, Yasin_2009, teWildt_2010}, 28 cases (39\%) were female \cite{AlShaaibi_2021b, Ali_2020f, Ataya_2013, Beecroft_1998, Bhasin_2014, Bhattacharjee_2008, Cauchi_2002, Chang_2017f, Cox_2007, DelgadoSalazar_2020c, DivsalarP._2023a, Goldman_1998f, Hardy_2023g, Kar_2015, Kariholu_2008, Kerestes_2019, Li_2013, Naji_2012f, Ohno_2005, Peixoto_2017f, Sakellaridis_2008f, Sultan_2024f, Tupesis_2004f, Wildhaber_2005, Wnęk_2015f, Yildiz_2016e}, 1 case (1\%) had no gender recorded \cite{fjbuilsRepeatedBehaviorDeliberate2024}. \paragraph*{Age Group} 25 cases (35\%) were between 26 and 40 years of age \cite{Alao_2006i, Ali_2022g, Apikotoa_2022f, Ataya_2013, Benoist_2019e, Bhasin_2014, Chang_2017f, Cox_2007, DelgadoSalazar_2020c, Farhadi_2024h, Fry_2010, Gardner_2017h, Guinan_2019f, Jin_2023, Kumar_2019f, Losanoff_1996, Misra_2013, Qureshi_2016, Riva_2018j, Sakellaridis_2008f, Tammana_2012j, Trgo_2012f, Wnęk_2015f, Yildiz_2016e, fjbuilsRepeatedBehaviorDeliberate2024}, 18 cases (25\%) were between 18 and 25 years of age \cite{Akay_2015f, Ali_2017, Atayan_2016, Bhattacharjee_2008, Csaky_1998e, Kar_2015, Kariholu_2008, Kobiela_2015, Losanoff_1996, Losanoff_1997e, Mesfin_2022a, Peixoto_2017f, Sobnach_2011f, Tupesis_2004f, Yasin_2009}, 13 cases (18\%) were under 18 years of age \cite{AlShaaibi_2021b, Ali_2020f, Cauchi_2002, DivsalarP._2023a, Goldman_1998f, Liu_2005, Naji_2012f, Ohno_2005, Tanrikulu_2015e, Tay_2004, Wildhaber_2005}, 11 cases (15\%) were between 41 and 60 years of age \cite{Al-Faham_2020k, Bhumi_2024f, CamachoDorado_2018, Emamhadi_2018, Hardy_2023g, Jehangir_2019h, Kumar_2001, Sultan_2024f, Thapa_2019f, Wadhwa_2015e, teWildt_2010}, 3 cases (4\%) were over 60 years of age \cite{Beecroft_1998, Kerestes_2019, Li_2013}, 2 cases (3\%) had no age documented \cite{Berry_2021e}. All 90 were male gender. 90 cases (100\%) were detained at the time of ingestion \cite{Elghali_2016, Karp_1991b, Lee_2007}, 88 cases (98\%) were intentional ingestions \cite{Elghali_2016, Karp_1991b, Lee_2007}, 30 cases (33\%) had a psychiatric history documented \cite{Elghali_2016, Karp_1991b, Lee_2007}, 2 cases (2\%) had a history of prior ingestion \cite{Elghali_2016}. No cases were reported for were psychiatric inpatients, were displaced people, were under the influence of alcohol at the time of ingestion, and had a severe disability history.
\paragraph*{Motivation}  70 cases (78\%) reported protest motivation \cite{Elghali_2016, Karp_1991b, Lee_2007}, 12 cases (13\%) reported psychiatric motivation \cite{Karp_1991b}, 6 cases (7\%) reported self-harm motivation \cite{Elghali_2016, Karp_1991b}. No cases were reported for psychosocial motivation and other motivation.
\paragraph*{Object Characteristics}  68 cases (76\%) involved sharp object ingestion \cite{Elghali_2016, Karp_1991b, Lee_2007}, 32 cases (36\%) involved long (\textgreater 5cm) object ingestion \cite{Lee_2007}, 25 cases (28\%) involved ingestion of multiple objects \cite{Elghali_2016, Lee_2007}. No cases were reported for button battery ingestion, magnet ingestion, and involved large diameter (\textgreater 2.5cm) object ingestion.
\paragraph*{Outcomes}  47 cases (52\%) underwent endoscopic intervention \cite{Elghali_2016, Lee_2007}, 29 cases (32\%) were managed conservatively \cite{Elghali_2016, Karp_1991b}, 15 cases (17\%) underwent surgical intervention \cite{Elghali_2016, Karp_1991b, Lee_2007}, 6 cases (7\%) reported complications \cite{Lee_2007}, 1 case (1\%) died \cite{Elghali_2016}.
\paragraph*{Gender} 43 cases (60\%) were male \cite{Akay_2015f, Al-Faham_2020k, Alao_2006i, Ali_2017, Ali_2022g, Apikotoa_2022f, Atayan_2016, Benoist_2019e, Berry_2021e, Bhumi_2024f, CamachoDorado_2018, Csaky_1998e, Emamhadi_2018, Farhadi_2024h, Fry_2010, Gardner_2017h, Guinan_2019f, Jehangir_2019h, Jin_2023, Kobiela_2015, Kumar_2001, Kumar_2019f, Liu_2005, Losanoff_1996, Losanoff_1997e, Mesfin_2022a, Misra_2013, Qureshi_2016, Riva_2018j, Sobnach_2011f, Tammana_2012j, Tanrikulu_2015e, Tay_2004, Thapa_2019f, Trgo_2012f, Wadhwa_2015e, Yasin_2009, teWildt_2010}, 28 cases (39\%) were female \cite{AlShaaibi_2021b, Ali_2020f, Ataya_2013, Beecroft_1998, Bhasin_2014, Bhattacharjee_2008, Cauchi_2002, Chang_2017f, Cox_2007, DelgadoSalazar_2020c, DivsalarP._2023a, Goldman_1998f, Hardy_2023g, Kar_2015, Kariholu_2008, Kerestes_2019, Li_2013, Naji_2012f, Ohno_2005, Peixoto_2017f, Sakellaridis_2008f, Sultan_2024f, Tupesis_2004f, Wildhaber_2005, Wnęk_2015f, Yildiz_2016e}, 1 case (1\%) had no gender recorded \cite{fjbuilsRepeatedBehaviorDeliberate2024}. \paragraph*{Age Group} 25 cases (35\%) were between 26 and 40 years of age \cite{Alao_2006i, Ali_2022g, Apikotoa_2022f, Ataya_2013, Benoist_2019e, Bhasin_2014, Chang_2017f, Cox_2007, DelgadoSalazar_2020c, Farhadi_2024h, Fry_2010, Gardner_2017h, Guinan_2019f, Jin_2023, Kumar_2019f, Losanoff_1996, Misra_2013, Qureshi_2016, Riva_2018j, Sakellaridis_2008f, Tammana_2012j, Trgo_2012f, Wnęk_2015f, Yildiz_2016e, fjbuilsRepeatedBehaviorDeliberate2024}, 18 cases (25\%) were between 18 and 25 years of age \cite{Akay_2015f, Ali_2017, Atayan_2016, Bhattacharjee_2008, Csaky_1998e, Kar_2015, Kariholu_2008, Kobiela_2015, Losanoff_1996, Losanoff_1997e, Mesfin_2022a, Peixoto_2017f, Sobnach_2011f, Tupesis_2004f, Yasin_2009}, 13 cases (18\%) were under 18 years of age \cite{AlShaaibi_2021b, Ali_2020f, Cauchi_2002, DivsalarP._2023a, Goldman_1998f, Liu_2005, Naji_2012f, Ohno_2005, Tanrikulu_2015e, Tay_2004, Wildhaber_2005}, 11 cases (15\%) were between 41 and 60 years of age \cite{Al-Faham_2020k, Bhumi_2024f, CamachoDorado_2018, Emamhadi_2018, Hardy_2023g, Jehangir_2019h, Kumar_2001, Sultan_2024f, Thapa_2019f, Wadhwa_2015e, teWildt_2010}, 3 cases (4\%) were over 60 years of age \cite{Beecroft_1998, Kerestes_2019, Li_2013}, 2 cases (3\%) had no age documented \cite{Berry_2021e}. All 90 were male gender. 90 cases (100\%) were detained at the time of ingestion \cite{Elghali_2016, Karp_1991b, Lee_2007}, 88 cases (98\%) were intentional ingestions \cite{Elghali_2016, Karp_1991b, Lee_2007}, 30 cases (33\%) had a psychiatric history documented \cite{Elghali_2016, Karp_1991b, Lee_2007}, 2 cases (2\%) had a history of prior ingestion \cite{Elghali_2016}. No cases were reported for were psychiatric inpatients, were displaced people, were under the influence of alcohol at the time of ingestion, and had a severe disability history.
\paragraph*{Motivation}  70 cases (78\%) reported protest motivation \cite{Elghali_2016, Karp_1991b, Lee_2007}, 12 cases (13\%) reported psychiatric motivation \cite{Karp_1991b}, 6 cases (7\%) reported self-harm motivation \cite{Elghali_2016, Karp_1991b}. No cases were reported for psychosocial motivation and other motivation.
\paragraph*{Object Characteristics}  68 cases (76\%) involved sharp object ingestion \cite{Elghali_2016, Karp_1991b, Lee_2007}, 32 cases (36\%) involved long (\textgreater 5cm) object ingestion \cite{Lee_2007}, 25 cases (28\%) involved ingestion of multiple objects \cite{Elghali_2016, Lee_2007}. No cases were reported for button battery ingestion, magnet ingestion, and involved large diameter (\textgreater 2.5cm) object ingestion.
\paragraph*{Outcomes}  47 cases (52\%) underwent endoscopic intervention \cite{Elghali_2016, Lee_2007}, 29 cases (32\%) were managed conservatively \cite{Elghali_2016, Karp_1991b}, 15 cases (17\%) underwent surgical intervention \cite{Elghali_2016, Karp_1991b, Lee_2007}, 6 cases (7\%) reported complications \cite{Lee_2007}, 1 case (1\%) died \cite{Elghali_2016}.
\paragraph*{Gender} 43 cases (60\%) were male \cite{Akay_2015f, Al-Faham_2020k, Alao_2006i, Ali_2017, Ali_2022g, Apikotoa_2022f, Atayan_2016, Benoist_2019e, Berry_2021e, Bhumi_2024f, CamachoDorado_2018, Csaky_1998e, Emamhadi_2018, Farhadi_2024h, Fry_2010, Gardner_2017h, Guinan_2019f, Jehangir_2019h, Jin_2023, Kobiela_2015, Kumar_2001, Kumar_2019f, Liu_2005, Losanoff_1996, Losanoff_1997e, Mesfin_2022a, Misra_2013, Qureshi_2016, Riva_2018j, Sobnach_2011f, Tammana_2012j, Tanrikulu_2015e, Tay_2004, Thapa_2019f, Trgo_2012f, Wadhwa_2015e, Yasin_2009, teWildt_2010}, 28 cases (39\%) were female \cite{AlShaaibi_2021b, Ali_2020f, Ataya_2013, Beecroft_1998, Bhasin_2014, Bhattacharjee_2008, Cauchi_2002, Chang_2017f, Cox_2007, DelgadoSalazar_2020c, DivsalarP._2023a, Goldman_1998f, Hardy_2023g, Kar_2015, Kariholu_2008, Kerestes_2019, Li_2013, Naji_2012f, Ohno_2005, Peixoto_2017f, Sakellaridis_2008f, Sultan_2024f, Tupesis_2004f, Wildhaber_2005, Wnęk_2015f, Yildiz_2016e}, 1 case (1\%) had no gender recorded \cite{fjbuilsRepeatedBehaviorDeliberate2024}. \paragraph*{Age Group} 25 cases (35\%) were between 26 and 40 years of age \cite{Alao_2006i, Ali_2022g, Apikotoa_2022f, Ataya_2013, Benoist_2019e, Bhasin_2014, Chang_2017f, Cox_2007, DelgadoSalazar_2020c, Farhadi_2024h, Fry_2010, Gardner_2017h, Guinan_2019f, Jin_2023, Kumar_2019f, Losanoff_1996, Misra_2013, Qureshi_2016, Riva_2018j, Sakellaridis_2008f, Tammana_2012j, Trgo_2012f, Wnęk_2015f, Yildiz_2016e, fjbuilsRepeatedBehaviorDeliberate2024}, 18 cases (25\%) were between 18 and 25 years of age \cite{Akay_2015f, Ali_2017, Atayan_2016, Bhattacharjee_2008, Csaky_1998e, Kar_2015, Kariholu_2008, Kobiela_2015, Losanoff_1996, Losanoff_1997e, Mesfin_2022a, Peixoto_2017f, Sobnach_2011f, Tupesis_2004f, Yasin_2009}, 13 cases (18\%) were under 18 years of age \cite{AlShaaibi_2021b, Ali_2020f, Cauchi_2002, DivsalarP._2023a, Goldman_1998f, Liu_2005, Naji_2012f, Ohno_2005, Tanrikulu_2015e, Tay_2004, Wildhaber_2005}, 11 cases (15\%) were between 41 and 60 years of age \cite{Al-Faham_2020k, Bhumi_2024f, CamachoDorado_2018, Emamhadi_2018, Hardy_2023g, Jehangir_2019h, Kumar_2001, Sultan_2024f, Thapa_2019f, Wadhwa_2015e, teWildt_2010}, 3 cases (4\%) were over 60 years of age \cite{Beecroft_1998, Kerestes_2019, Li_2013}, 2 cases (3\%) had no age documented \cite{Berry_2021e}. All 90 were male gender. 90 cases (100\%) were detained at the time of ingestion \cite{Elghali_2016, Karp_1991b, Lee_2007}, 88 cases (98\%) were intentional ingestions \cite{Elghali_2016, Karp_1991b, Lee_2007}, 30 cases (33\%) had a psychiatric history documented \cite{Elghali_2016, Karp_1991b, Lee_2007}, 2 cases (2\%) had a history of prior ingestion \cite{Elghali_2016}. No cases were reported for were psychiatric inpatients, were displaced people, were under the influence of alcohol at the time of ingestion, and had a severe disability history.
\paragraph*{Motivation}  70 cases (78\%) reported protest motivation \cite{Elghali_2016, Karp_1991b, Lee_2007}, 12 cases (13\%) reported psychiatric motivation \cite{Karp_1991b}, 6 cases (7\%) reported self-harm motivation \cite{Elghali_2016, Karp_1991b}. No cases were reported for psychosocial motivation and other motivation.
\paragraph*{Object Characteristics}  68 cases (76\%) involved sharp object ingestion \cite{Elghali_2016, Karp_1991b, Lee_2007}, 32 cases (36\%) involved long (\textgreater 5cm) object ingestion \cite{Lee_2007}, 25 cases (28\%) involved ingestion of multiple objects \cite{Elghali_2016, Lee_2007}. No cases were reported for button battery ingestion, magnet ingestion, and involved large diameter (\textgreater 2.5cm) object ingestion.
\paragraph*{Outcomes}  47 cases (52\%) underwent endoscopic intervention \cite{Elghali_2016, Lee_2007}, 29 cases (32\%) were managed conservatively \cite{Elghali_2016, Karp_1991b}, 15 cases (17\%) underwent surgical intervention \cite{Elghali_2016, Karp_1991b, Lee_2007}, 6 cases (7\%) reported complications \cite{Lee_2007}, 1 case (1\%) died \cite{Elghali_2016}.
\paragraph*{Gender} 43 cases (60\%) were male \cite{Akay_2015f, Al-Faham_2020k, Alao_2006i, Ali_2017, Ali_2022g, Apikotoa_2022f, Atayan_2016, Benoist_2019e, Berry_2021e, Bhumi_2024f, CamachoDorado_2018, Csaky_1998e, Emamhadi_2018, Farhadi_2024h, Fry_2010, Gardner_2017h, Guinan_2019f, Jehangir_2019h, Jin_2023, Kobiela_2015, Kumar_2001, Kumar_2019f, Liu_2005, Losanoff_1996, Losanoff_1997e, Mesfin_2022a, Misra_2013, Qureshi_2016, Riva_2018j, Sobnach_2011f, Tammana_2012j, Tanrikulu_2015e, Tay_2004, Thapa_2019f, Trgo_2012f, Wadhwa_2015e, Yasin_2009, teWildt_2010}, 28 cases (39\%) were female \cite{AlShaaibi_2021b, Ali_2020f, Ataya_2013, Beecroft_1998, Bhasin_2014, Bhattacharjee_2008, Cauchi_2002, Chang_2017f, Cox_2007, DelgadoSalazar_2020c, DivsalarP._2023a, Goldman_1998f, Hardy_2023g, Kar_2015, Kariholu_2008, Kerestes_2019, Li_2013, Naji_2012f, Ohno_2005, Peixoto_2017f, Sakellaridis_2008f, Sultan_2024f, Tupesis_2004f, Wildhaber_2005, Wnęk_2015f, Yildiz_2016e}, 1 case (1\%) had no gender recorded \cite{fjbuilsRepeatedBehaviorDeliberate2024}. \paragraph*{Age Group} 25 cases (35\%) were between 26 and 40 years of age \cite{Alao_2006i, Ali_2022g, Apikotoa_2022f, Ataya_2013, Benoist_2019e, Bhasin_2014, Chang_2017f, Cox_2007, DelgadoSalazar_2020c, Farhadi_2024h, Fry_2010, Gardner_2017h, Guinan_2019f, Jin_2023, Kumar_2019f, Losanoff_1996, Misra_2013, Qureshi_2016, Riva_2018j, Sakellaridis_2008f, Tammana_2012j, Trgo_2012f, Wnęk_2015f, Yildiz_2016e, fjbuilsRepeatedBehaviorDeliberate2024}, 18 cases (25\%) were between 18 and 25 years of age \cite{Akay_2015f, Ali_2017, Atayan_2016, Bhattacharjee_2008, Csaky_1998e, Kar_2015, Kariholu_2008, Kobiela_2015, Losanoff_1996, Losanoff_1997e, Mesfin_2022a, Peixoto_2017f, Sobnach_2011f, Tupesis_2004f, Yasin_2009}, 13 cases (18\%) were under 18 years of age \cite{AlShaaibi_2021b, Ali_2020f, Cauchi_2002, DivsalarP._2023a, Goldman_1998f, Liu_2005, Naji_2012f, Ohno_2005, Tanrikulu_2015e, Tay_2004, Wildhaber_2005}, 11 cases (15\%) were between 41 and 60 years of age \cite{Al-Faham_2020k, Bhumi_2024f, CamachoDorado_2018, Emamhadi_2018, Hardy_2023g, Jehangir_2019h, Kumar_2001, Sultan_2024f, Thapa_2019f, Wadhwa_2015e, teWildt_2010}, 3 cases (4\%) were over 60 years of age \cite{Beecroft_1998, Kerestes_2019, Li_2013}, 2 cases (3\%) had no age documented \cite{Berry_2021e}. All 90 were male gender. 90 cases (100\%) were detained at the time of ingestion \cite{Elghali_2016, Karp_1991b, Lee_2007}, 88 cases (98\%) were intentional ingestions \cite{Elghali_2016, Karp_1991b, Lee_2007}, 30 cases (33\%) had a psychiatric history documented \cite{Elghali_2016, Karp_1991b, Lee_2007}, 2 cases (2\%) had a history of prior ingestion \cite{Elghali_2016}. No cases were reported for were psychiatric inpatients, were displaced people, were under the influence of alcohol at the time of ingestion, and had a severe disability history.
\paragraph*{Motivation}  70 cases (78\%) reported protest motivation \cite{Elghali_2016, Karp_1991b, Lee_2007}, 12 cases (13\%) reported psychiatric motivation \cite{Karp_1991b}, 6 cases (7\%) reported self-harm motivation \cite{Elghali_2016, Karp_1991b}. No cases were reported for psychosocial motivation and other motivation.
\paragraph*{Object Characteristics}  68 cases (76\%) involved sharp object ingestion \cite{Elghali_2016, Karp_1991b, Lee_2007}, 32 cases (36\%) involved long (\textgreater 5cm) object ingestion \cite{Lee_2007}, 25 cases (28\%) involved ingestion of multiple objects \cite{Elghali_2016, Lee_2007}. No cases were reported for button battery ingestion, magnet ingestion, and involved large diameter (\textgreater 2.5cm) object ingestion.
\paragraph*{Outcomes}  47 cases (52\%) underwent endoscopic intervention \cite{Elghali_2016, Lee_2007}, 29 cases (32\%) were managed conservatively \cite{Elghali_2016, Karp_1991b}, 15 cases (17\%) underwent surgical intervention \cite{Elghali_2016, Karp_1991b, Lee_2007}, 6 cases (7\%) reported complications \cite{Lee_2007}, 1 case (1\%) died \cite{Elghali_2016}.
\paragraph*{Gender} 43 cases (60\%) were male \cite{Akay_2015f, Al-Faham_2020k, Alao_2006i, Ali_2017, Ali_2022g, Apikotoa_2022f, Atayan_2016, Benoist_2019e, Berry_2021e, Bhumi_2024f, CamachoDorado_2018, Csaky_1998e, Emamhadi_2018, Farhadi_2024h, Fry_2010, Gardner_2017h, Guinan_2019f, Jehangir_2019h, Jin_2023, Kobiela_2015, Kumar_2001, Kumar_2019f, Liu_2005, Losanoff_1996, Losanoff_1997e, Mesfin_2022a, Misra_2013, Qureshi_2016, Riva_2018j, Sobnach_2011f, Tammana_2012j, Tanrikulu_2015e, Tay_2004, Thapa_2019f, Trgo_2012f, Wadhwa_2015e, Yasin_2009, teWildt_2010}, 28 cases (39\%) were female \cite{AlShaaibi_2021b, Ali_2020f, Ataya_2013, Beecroft_1998, Bhasin_2014, Bhattacharjee_2008, Cauchi_2002, Chang_2017f, Cox_2007, DelgadoSalazar_2020c, DivsalarP._2023a, Goldman_1998f, Hardy_2023g, Kar_2015, Kariholu_2008, Kerestes_2019, Li_2013, Naji_2012f, Ohno_2005, Peixoto_2017f, Sakellaridis_2008f, Sultan_2024f, Tupesis_2004f, Wildhaber_2005, Wnęk_2015f, Yildiz_2016e}, 1 case (1\%) had no gender recorded \cite{fjbuilsRepeatedBehaviorDeliberate2024}. \paragraph*{Age Group} 25 cases (35\%) were between 26 and 40 years of age \cite{Alao_2006i, Ali_2022g, Apikotoa_2022f, Ataya_2013, Benoist_2019e, Bhasin_2014, Chang_2017f, Cox_2007, DelgadoSalazar_2020c, Farhadi_2024h, Fry_2010, Gardner_2017h, Guinan_2019f, Jin_2023, Kumar_2019f, Losanoff_1996, Misra_2013, Qureshi_2016, Riva_2018j, Sakellaridis_2008f, Tammana_2012j, Trgo_2012f, Wnęk_2015f, Yildiz_2016e, fjbuilsRepeatedBehaviorDeliberate2024}, 18 cases (25\%) were between 18 and 25 years of age \cite{Akay_2015f, Ali_2017, Atayan_2016, Bhattacharjee_2008, Csaky_1998e, Kar_2015, Kariholu_2008, Kobiela_2015, Losanoff_1996, Losanoff_1997e, Mesfin_2022a, Peixoto_2017f, Sobnach_2011f, Tupesis_2004f, Yasin_2009}, 13 cases (18\%) were under 18 years of age \cite{AlShaaibi_2021b, Ali_2020f, Cauchi_2002, DivsalarP._2023a, Goldman_1998f, Liu_2005, Naji_2012f, Ohno_2005, Tanrikulu_2015e, Tay_2004, Wildhaber_2005}, 11 cases (15\%) were between 41 and 60 years of age \cite{Al-Faham_2020k, Bhumi_2024f, CamachoDorado_2018, Emamhadi_2018, Hardy_2023g, Jehangir_2019h, Kumar_2001, Sultan_2024f, Thapa_2019f, Wadhwa_2015e, teWildt_2010}, 3 cases (4\%) were over 60 years of age \cite{Beecroft_1998, Kerestes_2019, Li_2013}, 2 cases (3\%) had no age documented \cite{Berry_2021e}. All 90 were male gender. 90 cases (100\%) were detained at the time of ingestion \cite{Elghali_2016, Karp_1991b, Lee_2007}, 88 cases (98\%) were intentional ingestions \cite{Elghali_2016, Karp_1991b, Lee_2007}, 30 cases (33\%) had a psychiatric history documented \cite{Elghali_2016, Karp_1991b, Lee_2007}, 2 cases (2\%) had a history of prior ingestion \cite{Elghali_2016}. No cases were reported for were psychiatric inpatients, were displaced people, were under the influence of alcohol at the time of ingestion, and had a severe disability history.
\paragraph*{Motivation}  70 cases (78\%) reported protest motivation \cite{Elghali_2016, Karp_1991b, Lee_2007}, 12 cases (13\%) reported psychiatric motivation \cite{Karp_1991b}, 6 cases (7\%) reported self-harm motivation \cite{Elghali_2016, Karp_1991b}. No cases were reported for psychosocial motivation and other motivation.
\paragraph*{Object Characteristics}  68 cases (76\%) involved sharp object ingestion \cite{Elghali_2016, Karp_1991b, Lee_2007}, 32 cases (36\%) involved long (\textgreater 5cm) object ingestion \cite{Lee_2007}, 25 cases (28\%) involved ingestion of multiple objects \cite{Elghali_2016, Lee_2007}. No cases were reported for button battery ingestion, magnet ingestion, and involved large diameter (\textgreater 2.5cm) object ingestion.
\paragraph*{Outcomes}  47 cases (52\%) underwent endoscopic intervention \cite{Elghali_2016, Lee_2007}, 29 cases (32\%) were managed conservatively \cite{Elghali_2016, Karp_1991b}, 15 cases (17\%) underwent surgical intervention \cite{Elghali_2016, Karp_1991b, Lee_2007}, 6 cases (7\%) reported complications \cite{Lee_2007}, 1 case (1\%) died \cite{Elghali_2016}.
\paragraph*{Gender} 43 cases (60\%) were male \cite{Akay_2015f, Al-Faham_2020k, Alao_2006i, Ali_2017, Ali_2022g, Apikotoa_2022f, Atayan_2016, Benoist_2019e, Berry_2021e, Bhumi_2024f, CamachoDorado_2018, Csaky_1998e, Emamhadi_2018, Farhadi_2024h, Fry_2010, Gardner_2017h, Guinan_2019f, Jehangir_2019h, Jin_2023, Kobiela_2015, Kumar_2001, Kumar_2019f, Liu_2005, Losanoff_1996, Losanoff_1997e, Mesfin_2022a, Misra_2013, Qureshi_2016, Riva_2018j, Sobnach_2011f, Tammana_2012j, Tanrikulu_2015e, Tay_2004, Thapa_2019f, Trgo_2012f, Wadhwa_2015e, Yasin_2009, teWildt_2010}, 28 cases (39\%) were female \cite{AlShaaibi_2021b, Ali_2020f, Ataya_2013, Beecroft_1998, Bhasin_2014, Bhattacharjee_2008, Cauchi_2002, Chang_2017f, Cox_2007, DelgadoSalazar_2020c, DivsalarP._2023a, Goldman_1998f, Hardy_2023g, Kar_2015, Kariholu_2008, Kerestes_2019, Li_2013, Naji_2012f, Ohno_2005, Peixoto_2017f, Sakellaridis_2008f, Sultan_2024f, Tupesis_2004f, Wildhaber_2005, Wnęk_2015f, Yildiz_2016e}, 1 case (1\%) had no gender recorded \cite{fjbuilsRepeatedBehaviorDeliberate2024}. \paragraph*{Age Group} 25 cases (35\%) were between 26 and 40 years of age \cite{Alao_2006i, Ali_2022g, Apikotoa_2022f, Ataya_2013, Benoist_2019e, Bhasin_2014, Chang_2017f, Cox_2007, DelgadoSalazar_2020c, Farhadi_2024h, Fry_2010, Gardner_2017h, Guinan_2019f, Jin_2023, Kumar_2019f, Losanoff_1996, Misra_2013, Qureshi_2016, Riva_2018j, Sakellaridis_2008f, Tammana_2012j, Trgo_2012f, Wnęk_2015f, Yildiz_2016e, fjbuilsRepeatedBehaviorDeliberate2024}, 18 cases (25\%) were between 18 and 25 years of age \cite{Akay_2015f, Ali_2017, Atayan_2016, Bhattacharjee_2008, Csaky_1998e, Kar_2015, Kariholu_2008, Kobiela_2015, Losanoff_1996, Losanoff_1997e, Mesfin_2022a, Peixoto_2017f, Sobnach_2011f, Tupesis_2004f, Yasin_2009}, 13 cases (18\%) were under 18 years of age \cite{AlShaaibi_2021b, Ali_2020f, Cauchi_2002, DivsalarP._2023a, Goldman_1998f, Liu_2005, Naji_2012f, Ohno_2005, Tanrikulu_2015e, Tay_2004, Wildhaber_2005}, 11 cases (15\%) were between 41 and 60 years of age \cite{Al-Faham_2020k, Bhumi_2024f, CamachoDorado_2018, Emamhadi_2018, Hardy_2023g, Jehangir_2019h, Kumar_2001, Sultan_2024f, Thapa_2019f, Wadhwa_2015e, teWildt_2010}, 3 cases (4\%) were over 60 years of age \cite{Beecroft_1998, Kerestes_2019, Li_2013}, 2 cases (3\%) had no age documented \cite{Berry_2021e}. All 90 were male gender. 90 cases (100\%) were detained at the time of ingestion \cite{Elghali_2016, Karp_1991b, Lee_2007}, 88 cases (98\%) were intentional ingestions \cite{Elghali_2016, Karp_1991b, Lee_2007}, 30 cases (33\%) had a psychiatric history documented \cite{Elghali_2016, Karp_1991b, Lee_2007}, 2 cases (2\%) had a history of prior ingestion \cite{Elghali_2016}. No cases were reported for were psychiatric inpatients, were displaced people, were under the influence of alcohol at the time of ingestion, and had a severe disability history.
\paragraph*{Motivation}  70 cases (78\%) reported protest motivation \cite{Elghali_2016, Karp_1991b, Lee_2007}, 12 cases (13\%) reported psychiatric motivation \cite{Karp_1991b}, 6 cases (7\%) reported self-harm motivation \cite{Elghali_2016, Karp_1991b}. No cases were reported for psychosocial motivation and other motivation.
\paragraph*{Object Characteristics}  68 cases (76\%) involved sharp object ingestion \cite{Elghali_2016, Karp_1991b, Lee_2007}, 32 cases (36\%) involved long (\textgreater 5cm) object ingestion \cite{Lee_2007}, 25 cases (28\%) involved ingestion of multiple objects \cite{Elghali_2016, Lee_2007}. No cases were reported for button battery ingestion, magnet ingestion, and involved large diameter (\textgreater 2.5cm) object ingestion.
\paragraph*{Outcomes}  47 cases (52\%) underwent endoscopic intervention \cite{Elghali_2016, Lee_2007}, 29 cases (32\%) were managed conservatively \cite{Elghali_2016, Karp_1991b}, 15 cases (17\%) underwent surgical intervention \cite{Elghali_2016, Karp_1991b, Lee_2007}, 6 cases (7\%) reported complications \cite{Lee_2007}, 1 case (1\%) died \cite{Elghali_2016}.
\paragraph*{Gender} 43 cases (60\%) were male \cite{Akay_2015f, Al-Faham_2020k, Alao_2006i, Ali_2017, Ali_2022g, Apikotoa_2022f, Atayan_2016, Benoist_2019e, Berry_2021e, Bhumi_2024f, CamachoDorado_2018, Csaky_1998e, Emamhadi_2018, Farhadi_2024h, Fry_2010, Gardner_2017h, Guinan_2019f, Jehangir_2019h, Jin_2023, Kobiela_2015, Kumar_2001, Kumar_2019f, Liu_2005, Losanoff_1996, Losanoff_1997e, Mesfin_2022a, Misra_2013, Qureshi_2016, Riva_2018j, Sobnach_2011f, Tammana_2012j, Tanrikulu_2015e, Tay_2004, Thapa_2019f, Trgo_2012f, Wadhwa_2015e, Yasin_2009, teWildt_2010}, 28 cases (39\%) were female \cite{AlShaaibi_2021b, Ali_2020f, Ataya_2013, Beecroft_1998, Bhasin_2014, Bhattacharjee_2008, Cauchi_2002, Chang_2017f, Cox_2007, DelgadoSalazar_2020c, DivsalarP._2023a, Goldman_1998f, Hardy_2023g, Kar_2015, Kariholu_2008, Kerestes_2019, Li_2013, Naji_2012f, Ohno_2005, Peixoto_2017f, Sakellaridis_2008f, Sultan_2024f, Tupesis_2004f, Wildhaber_2005, Wnęk_2015f, Yildiz_2016e}, 1 case (1\%) had no gender recorded \cite{fjbuilsRepeatedBehaviorDeliberate2024}. \paragraph*{Age Group} 25 cases (35\%) were between 26 and 40 years of age \cite{Alao_2006i, Ali_2022g, Apikotoa_2022f, Ataya_2013, Benoist_2019e, Bhasin_2014, Chang_2017f, Cox_2007, DelgadoSalazar_2020c, Farhadi_2024h, Fry_2010, Gardner_2017h, Guinan_2019f, Jin_2023, Kumar_2019f, Losanoff_1996, Misra_2013, Qureshi_2016, Riva_2018j, Sakellaridis_2008f, Tammana_2012j, Trgo_2012f, Wnęk_2015f, Yildiz_2016e, fjbuilsRepeatedBehaviorDeliberate2024}, 18 cases (25\%) were between 18 and 25 years of age \cite{Akay_2015f, Ali_2017, Atayan_2016, Bhattacharjee_2008, Csaky_1998e, Kar_2015, Kariholu_2008, Kobiela_2015, Losanoff_1996, Losanoff_1997e, Mesfin_2022a, Peixoto_2017f, Sobnach_2011f, Tupesis_2004f, Yasin_2009}, 13 cases (18\%) were under 18 years of age \cite{AlShaaibi_2021b, Ali_2020f, Cauchi_2002, DivsalarP._2023a, Goldman_1998f, Liu_2005, Naji_2012f, Ohno_2005, Tanrikulu_2015e, Tay_2004, Wildhaber_2005}, 11 cases (15\%) were between 41 and 60 years of age \cite{Al-Faham_2020k, Bhumi_2024f, CamachoDorado_2018, Emamhadi_2018, Hardy_2023g, Jehangir_2019h, Kumar_2001, Sultan_2024f, Thapa_2019f, Wadhwa_2015e, teWildt_2010}, 3 cases (4\%) were over 60 years of age \cite{Beecroft_1998, Kerestes_2019, Li_2013}, 2 cases (3\%) had no age documented \cite{Berry_2021e}. All 90 were male gender. 90 cases (100\%) were detained at the time of ingestion \cite{Elghali_2016, Karp_1991b, Lee_2007}, 88 cases (98\%) were intentional ingestions \cite{Elghali_2016, Karp_1991b, Lee_2007}, 30 cases (33\%) had a psychiatric history documented \cite{Elghali_2016, Karp_1991b, Lee_2007}, 2 cases (2\%) had a history of prior ingestion \cite{Elghali_2016}. No cases were reported for were psychiatric inpatients, were displaced people, were under the influence of alcohol at the time of ingestion, and had a severe disability history.
\paragraph*{Motivation}  70 cases (78\%) reported protest motivation \cite{Elghali_2016, Karp_1991b, Lee_2007}, 12 cases (13\%) reported psychiatric motivation \cite{Karp_1991b}, 6 cases (7\%) reported self-harm motivation \cite{Elghali_2016, Karp_1991b}. No cases were reported for psychosocial motivation and other motivation.
\paragraph*{Object Characteristics}  68 cases (76\%) involved sharp object ingestion \cite{Elghali_2016, Karp_1991b, Lee_2007}, 32 cases (36\%) involved long (\textgreater 5cm) object ingestion \cite{Lee_2007}, 25 cases (28\%) involved ingestion of multiple objects \cite{Elghali_2016, Lee_2007}. No cases were reported for button battery ingestion, magnet ingestion, and involved large diameter (\textgreater 2.5cm) object ingestion.
\paragraph*{Outcomes}  47 cases (52\%) underwent endoscopic intervention \cite{Elghali_2016, Lee_2007}, 29 cases (32\%) were managed conservatively \cite{Elghali_2016, Karp_1991b}, 15 cases (17\%) underwent surgical intervention \cite{Elghali_2016, Karp_1991b, Lee_2007}, 6 cases (7\%) reported complications \cite{Lee_2007}, 1 case (1\%) died \cite{Elghali_2016}.
\paragraph*{Gender} 43 cases (60\%) were male \cite{Akay_2015f, Al-Faham_2020k, Alao_2006i, Ali_2017, Ali_2022g, Apikotoa_2022f, Atayan_2016, Benoist_2019e, Berry_2021e, Bhumi_2024f, CamachoDorado_2018, Csaky_1998e, Emamhadi_2018, Farhadi_2024h, Fry_2010, Gardner_2017h, Guinan_2019f, Jehangir_2019h, Jin_2023, Kobiela_2015, Kumar_2001, Kumar_2019f, Liu_2005, Losanoff_1996, Losanoff_1997e, Mesfin_2022a, Misra_2013, Qureshi_2016, Riva_2018j, Sobnach_2011f, Tammana_2012j, Tanrikulu_2015e, Tay_2004, Thapa_2019f, Trgo_2012f, Wadhwa_2015e, Yasin_2009, teWildt_2010}, 28 cases (39\%) were female \cite{AlShaaibi_2021b, Ali_2020f, Ataya_2013, Beecroft_1998, Bhasin_2014, Bhattacharjee_2008, Cauchi_2002, Chang_2017f, Cox_2007, DelgadoSalazar_2020c, DivsalarP._2023a, Goldman_1998f, Hardy_2023g, Kar_2015, Kariholu_2008, Kerestes_2019, Li_2013, Naji_2012f, Ohno_2005, Peixoto_2017f, Sakellaridis_2008f, Sultan_2024f, Tupesis_2004f, Wildhaber_2005, Wnęk_2015f, Yildiz_2016e}, 1 case (1\%) had no gender recorded \cite{fjbuilsRepeatedBehaviorDeliberate2024}. \paragraph*{Age Group} 25 cases (35\%) were between 26 and 40 years of age \cite{Alao_2006i, Ali_2022g, Apikotoa_2022f, Ataya_2013, Benoist_2019e, Bhasin_2014, Chang_2017f, Cox_2007, DelgadoSalazar_2020c, Farhadi_2024h, Fry_2010, Gardner_2017h, Guinan_2019f, Jin_2023, Kumar_2019f, Losanoff_1996, Misra_2013, Qureshi_2016, Riva_2018j, Sakellaridis_2008f, Tammana_2012j, Trgo_2012f, Wnęk_2015f, Yildiz_2016e, fjbuilsRepeatedBehaviorDeliberate2024}, 18 cases (25\%) were between 18 and 25 years of age \cite{Akay_2015f, Ali_2017, Atayan_2016, Bhattacharjee_2008, Csaky_1998e, Kar_2015, Kariholu_2008, Kobiela_2015, Losanoff_1996, Losanoff_1997e, Mesfin_2022a, Peixoto_2017f, Sobnach_2011f, Tupesis_2004f, Yasin_2009}, 13 cases (18\%) were under 18 years of age \cite{AlShaaibi_2021b, Ali_2020f, Cauchi_2002, DivsalarP._2023a, Goldman_1998f, Liu_2005, Naji_2012f, Ohno_2005, Tanrikulu_2015e, Tay_2004, Wildhaber_2005}, 11 cases (15\%) were between 41 and 60 years of age \cite{Al-Faham_2020k, Bhumi_2024f, CamachoDorado_2018, Emamhadi_2018, Hardy_2023g, Jehangir_2019h, Kumar_2001, Sultan_2024f, Thapa_2019f, Wadhwa_2015e, teWildt_2010}, 3 cases (4\%) were over 60 years of age \cite{Beecroft_1998, Kerestes_2019, Li_2013}, 2 cases (3\%) had no age documented \cite{Berry_2021e}. All 90 were male gender. 90 cases (100\%) were detained at the time of ingestion \cite{Elghali_2016, Karp_1991b, Lee_2007}, 88 cases (98\%) were intentional ingestions \cite{Elghali_2016, Karp_1991b, Lee_2007}, 30 cases (33\%) had a psychiatric history documented \cite{Elghali_2016, Karp_1991b, Lee_2007}, 2 cases (2\%) had a history of prior ingestion \cite{Elghali_2016}. No cases were reported for were psychiatric inpatients, were displaced people, were under the influence of alcohol at the time of ingestion, and had a severe disability history.
\paragraph*{Motivation}  70 cases (78\%) reported protest motivation \cite{Elghali_2016, Karp_1991b, Lee_2007}, 12 cases (13\%) reported psychiatric motivation \cite{Karp_1991b}, 6 cases (7\%) reported self-harm motivation \cite{Elghali_2016, Karp_1991b}. No cases were reported for psychosocial motivation and other motivation.
\paragraph*{Object Characteristics}  68 cases (76\%) involved sharp object ingestion \cite{Elghali_2016, Karp_1991b, Lee_2007}, 32 cases (36\%) involved long (\textgreater 5cm) object ingestion \cite{Lee_2007}, 25 cases (28\%) involved ingestion of multiple objects \cite{Elghali_2016, Lee_2007}. No cases were reported for button battery ingestion, magnet ingestion, and involved large diameter (\textgreater 2.5cm) object ingestion.
\paragraph*{Outcomes}  47 cases (52\%) underwent endoscopic intervention \cite{Elghali_2016, Lee_2007}, 29 cases (32\%) were managed conservatively \cite{Elghali_2016, Karp_1991b}, 15 cases (17\%) underwent surgical intervention \cite{Elghali_2016, Karp_1991b, Lee_2007}, 6 cases (7\%) reported complications \cite{Lee_2007}, 1 case (1\%) died \cite{Elghali_2016}.
\paragraph*{Gender} 43 cases (60\%) were male \cite{Akay_2015f, Al-Faham_2020k, Alao_2006i, Ali_2017, Ali_2022g, Apikotoa_2022f, Atayan_2016, Benoist_2019e, Berry_2021e, Bhumi_2024f, CamachoDorado_2018, Csaky_1998e, Emamhadi_2018, Farhadi_2024h, Fry_2010, Gardner_2017h, Guinan_2019f, Jehangir_2019h, Jin_2023, Kobiela_2015, Kumar_2001, Kumar_2019f, Liu_2005, Losanoff_1996, Losanoff_1997e, Mesfin_2022a, Misra_2013, Qureshi_2016, Riva_2018j, Sobnach_2011f, Tammana_2012j, Tanrikulu_2015e, Tay_2004, Thapa_2019f, Trgo_2012f, Wadhwa_2015e, Yasin_2009, teWildt_2010}, 28 cases (39\%) were female \cite{AlShaaibi_2021b, Ali_2020f, Ataya_2013, Beecroft_1998, Bhasin_2014, Bhattacharjee_2008, Cauchi_2002, Chang_2017f, Cox_2007, DelgadoSalazar_2020c, DivsalarP._2023a, Goldman_1998f, Hardy_2023g, Kar_2015, Kariholu_2008, Kerestes_2019, Li_2013, Naji_2012f, Ohno_2005, Peixoto_2017f, Sakellaridis_2008f, Sultan_2024f, Tupesis_2004f, Wildhaber_2005, Wnęk_2015f, Yildiz_2016e}, 1 case (1\%) had no gender recorded \cite{fjbuilsRepeatedBehaviorDeliberate2024}. \paragraph*{Age Group} 25 cases (35\%) were between 26 and 40 years of age \cite{Alao_2006i, Ali_2022g, Apikotoa_2022f, Ataya_2013, Benoist_2019e, Bhasin_2014, Chang_2017f, Cox_2007, DelgadoSalazar_2020c, Farhadi_2024h, Fry_2010, Gardner_2017h, Guinan_2019f, Jin_2023, Kumar_2019f, Losanoff_1996, Misra_2013, Qureshi_2016, Riva_2018j, Sakellaridis_2008f, Tammana_2012j, Trgo_2012f, Wnęk_2015f, Yildiz_2016e, fjbuilsRepeatedBehaviorDeliberate2024}, 18 cases (25\%) were between 18 and 25 years of age \cite{Akay_2015f, Ali_2017, Atayan_2016, Bhattacharjee_2008, Csaky_1998e, Kar_2015, Kariholu_2008, Kobiela_2015, Losanoff_1996, Losanoff_1997e, Mesfin_2022a, Peixoto_2017f, Sobnach_2011f, Tupesis_2004f, Yasin_2009}, 13 cases (18\%) were under 18 years of age \cite{AlShaaibi_2021b, Ali_2020f, Cauchi_2002, DivsalarP._2023a, Goldman_1998f, Liu_2005, Naji_2012f, Ohno_2005, Tanrikulu_2015e, Tay_2004, Wildhaber_2005}, 11 cases (15\%) were between 41 and 60 years of age \cite{Al-Faham_2020k, Bhumi_2024f, CamachoDorado_2018, Emamhadi_2018, Hardy_2023g, Jehangir_2019h, Kumar_2001, Sultan_2024f, Thapa_2019f, Wadhwa_2015e, teWildt_2010}, 3 cases (4\%) were over 60 years of age \cite{Beecroft_1998, Kerestes_2019, Li_2013}, 2 cases (3\%) had no age documented \cite{Berry_2021e}. All 90 were male gender. 90 cases (100\%) were detained at the time of ingestion \cite{Elghali_2016, Karp_1991b, Lee_2007}, 88 cases (98\%) were intentional ingestions \cite{Elghali_2016, Karp_1991b, Lee_2007}, 30 cases (33\%) had a psychiatric history documented \cite{Elghali_2016, Karp_1991b, Lee_2007}, 2 cases (2\%) had a history of prior ingestion \cite{Elghali_2016}. No cases were reported for were psychiatric inpatients, were displaced people, were under the influence of alcohol at the time of ingestion, and had a severe disability history.
\paragraph*{Motivation}  70 cases (78\%) reported protest motivation \cite{Elghali_2016, Karp_1991b, Lee_2007}, 12 cases (13\%) reported psychiatric motivation \cite{Karp_1991b}, 6 cases (7\%) reported self-harm motivation \cite{Elghali_2016, Karp_1991b}. No cases were reported for psychosocial motivation and other motivation.
\paragraph*{Object Characteristics}  68 cases (76\%) involved sharp object ingestion \cite{Elghali_2016, Karp_1991b, Lee_2007}, 32 cases (36\%) involved long (\textgreater 5cm) object ingestion \cite{Lee_2007}, 25 cases (28\%) involved ingestion of multiple objects \cite{Elghali_2016, Lee_2007}. No cases were reported for button battery ingestion, magnet ingestion, and involved large diameter (\textgreater 2.5cm) object ingestion.
\paragraph*{Outcomes}  47 cases (52\%) underwent endoscopic intervention \cite{Elghali_2016, Lee_2007}, 29 cases (32\%) were managed conservatively \cite{Elghali_2016, Karp_1991b}, 15 cases (17\%) underwent surgical intervention \cite{Elghali_2016, Karp_1991b, Lee_2007}, 6 cases (7\%) reported complications \cite{Lee_2007}, 1 case (1\%) died \cite{Elghali_2016}.
\paragraph*{Gender} 43 cases (60\%) were male \cite{Akay_2015f, Al-Faham_2020k, Alao_2006i, Ali_2017, Ali_2022g, Apikotoa_2022f, Atayan_2016, Benoist_2019e, Berry_2021e, Bhumi_2024f, CamachoDorado_2018, Csaky_1998e, Emamhadi_2018, Farhadi_2024h, Fry_2010, Gardner_2017h, Guinan_2019f, Jehangir_2019h, Jin_2023, Kobiela_2015, Kumar_2001, Kumar_2019f, Liu_2005, Losanoff_1996, Losanoff_1997e, Mesfin_2022a, Misra_2013, Qureshi_2016, Riva_2018j, Sobnach_2011f, Tammana_2012j, Tanrikulu_2015e, Tay_2004, Thapa_2019f, Trgo_2012f, Wadhwa_2015e, Yasin_2009, teWildt_2010}, 28 cases (39\%) were female \cite{AlShaaibi_2021b, Ali_2020f, Ataya_2013, Beecroft_1998, Bhasin_2014, Bhattacharjee_2008, Cauchi_2002, Chang_2017f, Cox_2007, DelgadoSalazar_2020c, DivsalarP._2023a, Goldman_1998f, Hardy_2023g, Kar_2015, Kariholu_2008, Kerestes_2019, Li_2013, Naji_2012f, Ohno_2005, Peixoto_2017f, Sakellaridis_2008f, Sultan_2024f, Tupesis_2004f, Wildhaber_2005, Wnęk_2015f, Yildiz_2016e}, 1 case (1\%) had no gender recorded \cite{fjbuilsRepeatedBehaviorDeliberate2024}. \paragraph*{Age Group} 25 cases (35\%) were between 26 and 40 years of age \cite{Alao_2006i, Ali_2022g, Apikotoa_2022f, Ataya_2013, Benoist_2019e, Bhasin_2014, Chang_2017f, Cox_2007, DelgadoSalazar_2020c, Farhadi_2024h, Fry_2010, Gardner_2017h, Guinan_2019f, Jin_2023, Kumar_2019f, Losanoff_1996, Misra_2013, Qureshi_2016, Riva_2018j, Sakellaridis_2008f, Tammana_2012j, Trgo_2012f, Wnęk_2015f, Yildiz_2016e, fjbuilsRepeatedBehaviorDeliberate2024}, 18 cases (25\%) were between 18 and 25 years of age \cite{Akay_2015f, Ali_2017, Atayan_2016, Bhattacharjee_2008, Csaky_1998e, Kar_2015, Kariholu_2008, Kobiela_2015, Losanoff_1996, Losanoff_1997e, Mesfin_2022a, Peixoto_2017f, Sobnach_2011f, Tupesis_2004f, Yasin_2009}, 13 cases (18\%) were under 18 years of age \cite{AlShaaibi_2021b, Ali_2020f, Cauchi_2002, DivsalarP._2023a, Goldman_1998f, Liu_2005, Naji_2012f, Ohno_2005, Tanrikulu_2015e, Tay_2004, Wildhaber_2005}, 11 cases (15\%) were between 41 and 60 years of age \cite{Al-Faham_2020k, Bhumi_2024f, CamachoDorado_2018, Emamhadi_2018, Hardy_2023g, Jehangir_2019h, Kumar_2001, Sultan_2024f, Thapa_2019f, Wadhwa_2015e, teWildt_2010}, 3 cases (4\%) were over 60 years of age \cite{Beecroft_1998, Kerestes_2019, Li_2013}, 2 cases (3\%) had no age documented \cite{Berry_2021e}. All 90 were male gender. 90 cases (100\%) were detained at the time of ingestion \cite{Elghali_2016, Karp_1991b, Lee_2007}, 88 cases (98\%) were intentional ingestions \cite{Elghali_2016, Karp_1991b, Lee_2007}, 30 cases (33\%) had a psychiatric history documented \cite{Elghali_2016, Karp_1991b, Lee_2007}, 2 cases (2\%) had a history of prior ingestion \cite{Elghali_2016}. No cases were reported for were psychiatric inpatients, were displaced people, were under the influence of alcohol at the time of ingestion, and had a severe disability history.
\paragraph*{Motivation}  70 cases (78\%) reported protest motivation \cite{Elghali_2016, Karp_1991b, Lee_2007}, 12 cases (13\%) reported psychiatric motivation \cite{Karp_1991b}, 6 cases (7\%) reported self-harm motivation \cite{Elghali_2016, Karp_1991b}. No cases were reported for psychosocial motivation and other motivation.
\paragraph*{Object Characteristics}  68 cases (76\%) involved sharp object ingestion \cite{Elghali_2016, Karp_1991b, Lee_2007}, 32 cases (36\%) involved long (\textgreater 5cm) object ingestion \cite{Lee_2007}, 25 cases (28\%) involved ingestion of multiple objects \cite{Elghali_2016, Lee_2007}. No cases were reported for button battery ingestion, magnet ingestion, and involved large diameter (\textgreater 2.5cm) object ingestion.
\paragraph*{Outcomes}  47 cases (52\%) underwent endoscopic intervention \cite{Elghali_2016, Lee_2007}, 29 cases (32\%) were managed conservatively \cite{Elghali_2016, Karp_1991b}, 15 cases (17\%) underwent surgical intervention \cite{Elghali_2016, Karp_1991b, Lee_2007}, 6 cases (7\%) reported complications \cite{Lee_2007}, 1 case (1\%) died \cite{Elghali_2016}.
\paragraph*{Gender} 43 cases (60\%) were male \cite{Akay_2015f, Al-Faham_2020k, Alao_2006i, Ali_2017, Ali_2022g, Apikotoa_2022f, Atayan_2016, Benoist_2019e, Berry_2021e, Bhumi_2024f, CamachoDorado_2018, Csaky_1998e, Emamhadi_2018, Farhadi_2024h, Fry_2010, Gardner_2017h, Guinan_2019f, Jehangir_2019h, Jin_2023, Kobiela_2015, Kumar_2001, Kumar_2019f, Liu_2005, Losanoff_1996, Losanoff_1997e, Mesfin_2022a, Misra_2013, Qureshi_2016, Riva_2018j, Sobnach_2011f, Tammana_2012j, Tanrikulu_2015e, Tay_2004, Thapa_2019f, Trgo_2012f, Wadhwa_2015e, Yasin_2009, teWildt_2010}, 28 cases (39\%) were female \cite{AlShaaibi_2021b, Ali_2020f, Ataya_2013, Beecroft_1998, Bhasin_2014, Bhattacharjee_2008, Cauchi_2002, Chang_2017f, Cox_2007, DelgadoSalazar_2020c, DivsalarP._2023a, Goldman_1998f, Hardy_2023g, Kar_2015, Kariholu_2008, Kerestes_2019, Li_2013, Naji_2012f, Ohno_2005, Peixoto_2017f, Sakellaridis_2008f, Sultan_2024f, Tupesis_2004f, Wildhaber_2005, Wnęk_2015f, Yildiz_2016e}, 1 case (1\%) had no gender recorded \cite{fjbuilsRepeatedBehaviorDeliberate2024}. \paragraph*{Age Group} 25 cases (35\%) were between 26 and 40 years of age \cite{Alao_2006i, Ali_2022g, Apikotoa_2022f, Ataya_2013, Benoist_2019e, Bhasin_2014, Chang_2017f, Cox_2007, DelgadoSalazar_2020c, Farhadi_2024h, Fry_2010, Gardner_2017h, Guinan_2019f, Jin_2023, Kumar_2019f, Losanoff_1996, Misra_2013, Qureshi_2016, Riva_2018j, Sakellaridis_2008f, Tammana_2012j, Trgo_2012f, Wnęk_2015f, Yildiz_2016e, fjbuilsRepeatedBehaviorDeliberate2024}, 18 cases (25\%) were between 18 and 25 years of age \cite{Akay_2015f, Ali_2017, Atayan_2016, Bhattacharjee_2008, Csaky_1998e, Kar_2015, Kariholu_2008, Kobiela_2015, Losanoff_1996, Losanoff_1997e, Mesfin_2022a, Peixoto_2017f, Sobnach_2011f, Tupesis_2004f, Yasin_2009}, 13 cases (18\%) were under 18 years of age \cite{AlShaaibi_2021b, Ali_2020f, Cauchi_2002, DivsalarP._2023a, Goldman_1998f, Liu_2005, Naji_2012f, Ohno_2005, Tanrikulu_2015e, Tay_2004, Wildhaber_2005}, 11 cases (15\%) were between 41 and 60 years of age \cite{Al-Faham_2020k, Bhumi_2024f, CamachoDorado_2018, Emamhadi_2018, Hardy_2023g, Jehangir_2019h, Kumar_2001, Sultan_2024f, Thapa_2019f, Wadhwa_2015e, teWildt_2010}, 3 cases (4\%) were over 60 years of age \cite{Beecroft_1998, Kerestes_2019, Li_2013}, 2 cases (3\%) had no age documented \cite{Berry_2021e}. All 90 were male gender. 90 cases (100\%) were detained at the time of ingestion \cite{Elghali_2016, Karp_1991b, Lee_2007}, 88 cases (98\%) were intentional ingestions \cite{Elghali_2016, Karp_1991b, Lee_2007}, 30 cases (33\%) had a psychiatric history documented \cite{Elghali_2016, Karp_1991b, Lee_2007}, 2 cases (2\%) had a history of prior ingestion \cite{Elghali_2016}. No cases were reported for were psychiatric inpatients, were displaced people, were under the influence of alcohol at the time of ingestion, and had a severe disability history.
\paragraph*{Motivation}  70 cases (78\%) reported protest motivation \cite{Elghali_2016, Karp_1991b, Lee_2007}, 12 cases (13\%) reported psychiatric motivation \cite{Karp_1991b}, 6 cases (7\%) reported self-harm motivation \cite{Elghali_2016, Karp_1991b}. No cases were reported for psychosocial motivation and other motivation.
\paragraph*{Object Characteristics}  68 cases (76\%) involved sharp object ingestion \cite{Elghali_2016, Karp_1991b, Lee_2007}, 32 cases (36\%) involved long (\textgreater 5cm) object ingestion \cite{Lee_2007}, 25 cases (28\%) involved ingestion of multiple objects \cite{Elghali_2016, Lee_2007}. No cases were reported for button battery ingestion, magnet ingestion, and involved large diameter (\textgreater 2.5cm) object ingestion.
\paragraph*{Outcomes}  47 cases (52\%) underwent endoscopic intervention \cite{Elghali_2016, Lee_2007}, 29 cases (32\%) were managed conservatively \cite{Elghali_2016, Karp_1991b}, 15 cases (17\%) underwent surgical intervention \cite{Elghali_2016, Karp_1991b, Lee_2007}, 6 cases (7\%) reported complications \cite{Lee_2007}, 1 case (1\%) died \cite{Elghali_2016}.
\paragraph*{Gender} 43 cases (60\%) were male \cite{Akay_2015f, Al-Faham_2020k, Alao_2006i, Ali_2017, Ali_2022g, Apikotoa_2022f, Atayan_2016, Benoist_2019e, Berry_2021e, Bhumi_2024f, CamachoDorado_2018, Csaky_1998e, Emamhadi_2018, Farhadi_2024h, Fry_2010, Gardner_2017h, Guinan_2019f, Jehangir_2019h, Jin_2023, Kobiela_2015, Kumar_2001, Kumar_2019f, Liu_2005, Losanoff_1996, Losanoff_1997e, Mesfin_2022a, Misra_2013, Qureshi_2016, Riva_2018j, Sobnach_2011f, Tammana_2012j, Tanrikulu_2015e, Tay_2004, Thapa_2019f, Trgo_2012f, Wadhwa_2015e, Yasin_2009, teWildt_2010}, 28 cases (39\%) were female \cite{AlShaaibi_2021b, Ali_2020f, Ataya_2013, Beecroft_1998, Bhasin_2014, Bhattacharjee_2008, Cauchi_2002, Chang_2017f, Cox_2007, DelgadoSalazar_2020c, DivsalarP._2023a, Goldman_1998f, Hardy_2023g, Kar_2015, Kariholu_2008, Kerestes_2019, Li_2013, Naji_2012f, Ohno_2005, Peixoto_2017f, Sakellaridis_2008f, Sultan_2024f, Tupesis_2004f, Wildhaber_2005, Wnęk_2015f, Yildiz_2016e}, 1 case (1\%) had no gender recorded \cite{fjbuilsRepeatedBehaviorDeliberate2024}. \paragraph*{Age Group} 25 cases (35\%) were between 26 and 40 years of age \cite{Alao_2006i, Ali_2022g, Apikotoa_2022f, Ataya_2013, Benoist_2019e, Bhasin_2014, Chang_2017f, Cox_2007, DelgadoSalazar_2020c, Farhadi_2024h, Fry_2010, Gardner_2017h, Guinan_2019f, Jin_2023, Kumar_2019f, Losanoff_1996, Misra_2013, Qureshi_2016, Riva_2018j, Sakellaridis_2008f, Tammana_2012j, Trgo_2012f, Wnęk_2015f, Yildiz_2016e, fjbuilsRepeatedBehaviorDeliberate2024}, 18 cases (25\%) were between 18 and 25 years of age \cite{Akay_2015f, Ali_2017, Atayan_2016, Bhattacharjee_2008, Csaky_1998e, Kar_2015, Kariholu_2008, Kobiela_2015, Losanoff_1996, Losanoff_1997e, Mesfin_2022a, Peixoto_2017f, Sobnach_2011f, Tupesis_2004f, Yasin_2009}, 13 cases (18\%) were under 18 years of age \cite{AlShaaibi_2021b, Ali_2020f, Cauchi_2002, DivsalarP._2023a, Goldman_1998f, Liu_2005, Naji_2012f, Ohno_2005, Tanrikulu_2015e, Tay_2004, Wildhaber_2005}, 11 cases (15\%) were between 41 and 60 years of age \cite{Al-Faham_2020k, Bhumi_2024f, CamachoDorado_2018, Emamhadi_2018, Hardy_2023g, Jehangir_2019h, Kumar_2001, Sultan_2024f, Thapa_2019f, Wadhwa_2015e, teWildt_2010}, 3 cases (4\%) were over 60 years of age \cite{Beecroft_1998, Kerestes_2019, Li_2013}, 2 cases (3\%) had no age documented \cite{Berry_2021e}. All 90 were male gender. 90 cases (100\%) were detained at the time of ingestion \cite{Elghali_2016, Karp_1991b, Lee_2007}, 88 cases (98\%) were intentional ingestions \cite{Elghali_2016, Karp_1991b, Lee_2007}, 30 cases (33\%) had a psychiatric history documented \cite{Elghali_2016, Karp_1991b, Lee_2007}, 2 cases (2\%) had a history of prior ingestion \cite{Elghali_2016}. No cases were reported for were psychiatric inpatients, were displaced people, were under the influence of alcohol at the time of ingestion, and had a severe disability history.
\paragraph*{Motivation}  70 cases (78\%) reported protest motivation \cite{Elghali_2016, Karp_1991b, Lee_2007}, 12 cases (13\%) reported psychiatric motivation \cite{Karp_1991b}, 6 cases (7\%) reported self-harm motivation \cite{Elghali_2016, Karp_1991b}. No cases were reported for psychosocial motivation and other motivation.
\paragraph*{Object Characteristics}  68 cases (76\%) involved sharp object ingestion \cite{Elghali_2016, Karp_1991b, Lee_2007}, 32 cases (36\%) involved long (\textgreater 5cm) object ingestion \cite{Lee_2007}, 25 cases (28\%) involved ingestion of multiple objects \cite{Elghali_2016, Lee_2007}. No cases were reported for button battery ingestion, magnet ingestion, and involved large diameter (\textgreater 2.5cm) object ingestion.
\paragraph*{Outcomes}  47 cases (52\%) underwent endoscopic intervention \cite{Elghali_2016, Lee_2007}, 29 cases (32\%) were managed conservatively \cite{Elghali_2016, Karp_1991b}, 15 cases (17\%) underwent surgical intervention \cite{Elghali_2016, Karp_1991b, Lee_2007}, 6 cases (7\%) reported complications \cite{Lee_2007}, 1 case (1\%) died \cite{Elghali_2016}.
\paragraph*{Gender} 43 cases (60\%) were male \cite{Akay_2015f, Al-Faham_2020k, Alao_2006i, Ali_2017, Ali_2022g, Apikotoa_2022f, Atayan_2016, Benoist_2019e, Berry_2021e, Bhumi_2024f, CamachoDorado_2018, Csaky_1998e, Emamhadi_2018, Farhadi_2024h, Fry_2010, Gardner_2017h, Guinan_2019f, Jehangir_2019h, Jin_2023, Kobiela_2015, Kumar_2001, Kumar_2019f, Liu_2005, Losanoff_1996, Losanoff_1997e, Mesfin_2022a, Misra_2013, Qureshi_2016, Riva_2018j, Sobnach_2011f, Tammana_2012j, Tanrikulu_2015e, Tay_2004, Thapa_2019f, Trgo_2012f, Wadhwa_2015e, Yasin_2009, teWildt_2010}, 28 cases (39\%) were female \cite{AlShaaibi_2021b, Ali_2020f, Ataya_2013, Beecroft_1998, Bhasin_2014, Bhattacharjee_2008, Cauchi_2002, Chang_2017f, Cox_2007, DelgadoSalazar_2020c, DivsalarP._2023a, Goldman_1998f, Hardy_2023g, Kar_2015, Kariholu_2008, Kerestes_2019, Li_2013, Naji_2012f, Ohno_2005, Peixoto_2017f, Sakellaridis_2008f, Sultan_2024f, Tupesis_2004f, Wildhaber_2005, Wnęk_2015f, Yildiz_2016e}, 1 case (1\%) had no gender recorded \cite{fjbuilsRepeatedBehaviorDeliberate2024}. \paragraph*{Age Group} 25 cases (35\%) were between 26 and 40 years of age \cite{Alao_2006i, Ali_2022g, Apikotoa_2022f, Ataya_2013, Benoist_2019e, Bhasin_2014, Chang_2017f, Cox_2007, DelgadoSalazar_2020c, Farhadi_2024h, Fry_2010, Gardner_2017h, Guinan_2019f, Jin_2023, Kumar_2019f, Losanoff_1996, Misra_2013, Qureshi_2016, Riva_2018j, Sakellaridis_2008f, Tammana_2012j, Trgo_2012f, Wnęk_2015f, Yildiz_2016e, fjbuilsRepeatedBehaviorDeliberate2024}, 18 cases (25\%) were between 18 and 25 years of age \cite{Akay_2015f, Ali_2017, Atayan_2016, Bhattacharjee_2008, Csaky_1998e, Kar_2015, Kariholu_2008, Kobiela_2015, Losanoff_1996, Losanoff_1997e, Mesfin_2022a, Peixoto_2017f, Sobnach_2011f, Tupesis_2004f, Yasin_2009}, 13 cases (18\%) were under 18 years of age \cite{AlShaaibi_2021b, Ali_2020f, Cauchi_2002, DivsalarP._2023a, Goldman_1998f, Liu_2005, Naji_2012f, Ohno_2005, Tanrikulu_2015e, Tay_2004, Wildhaber_2005}, 11 cases (15\%) were between 41 and 60 years of age \cite{Al-Faham_2020k, Bhumi_2024f, CamachoDorado_2018, Emamhadi_2018, Hardy_2023g, Jehangir_2019h, Kumar_2001, Sultan_2024f, Thapa_2019f, Wadhwa_2015e, teWildt_2010}, 3 cases (4\%) were over 60 years of age \cite{Beecroft_1998, Kerestes_2019, Li_2013}, 2 cases (3\%) had no age documented \cite{Berry_2021e}. All 90 were male gender. 90 cases (100\%) were detained at the time of ingestion \cite{Elghali_2016, Karp_1991b, Lee_2007}, 88 cases (98\%) were intentional ingestions \cite{Elghali_2016, Karp_1991b, Lee_2007}, 30 cases (33\%) had a psychiatric history documented \cite{Elghali_2016, Karp_1991b, Lee_2007}, 2 cases (2\%) had a history of prior ingestion \cite{Elghali_2016}. No cases were reported for were psychiatric inpatients, were displaced people, were under the influence of alcohol at the time of ingestion, and had a severe disability history.
\paragraph*{Motivation}  70 cases (78\%) reported protest motivation \cite{Elghali_2016, Karp_1991b, Lee_2007}, 12 cases (13\%) reported psychiatric motivation \cite{Karp_1991b}, 6 cases (7\%) reported self-harm motivation \cite{Elghali_2016, Karp_1991b}. No cases were reported for psychosocial motivation and other motivation.
\paragraph*{Object Characteristics}  68 cases (76\%) involved sharp object ingestion \cite{Elghali_2016, Karp_1991b, Lee_2007}, 32 cases (36\%) involved long (\textgreater 5cm) object ingestion \cite{Lee_2007}, 25 cases (28\%) involved ingestion of multiple objects \cite{Elghali_2016, Lee_2007}. No cases were reported for button battery ingestion, magnet ingestion, and involved large diameter (\textgreater 2.5cm) object ingestion.
\paragraph*{Outcomes}  47 cases (52\%) underwent endoscopic intervention \cite{Elghali_2016, Lee_2007}, 29 cases (32\%) were managed conservatively \cite{Elghali_2016, Karp_1991b}, 15 cases (17\%) underwent surgical intervention \cite{Elghali_2016, Karp_1991b, Lee_2007}, 6 cases (7\%) reported complications \cite{Lee_2007}, 1 case (1\%) died \cite{Elghali_2016}.
\paragraph*{Gender} 43 cases (60\%) were male \cite{Akay_2015f, Al-Faham_2020k, Alao_2006i, Ali_2017, Ali_2022g, Apikotoa_2022f, Atayan_2016, Benoist_2019e, Berry_2021e, Bhumi_2024f, CamachoDorado_2018, Csaky_1998e, Emamhadi_2018, Farhadi_2024h, Fry_2010, Gardner_2017h, Guinan_2019f, Jehangir_2019h, Jin_2023, Kobiela_2015, Kumar_2001, Kumar_2019f, Liu_2005, Losanoff_1996, Losanoff_1997e, Mesfin_2022a, Misra_2013, Qureshi_2016, Riva_2018j, Sobnach_2011f, Tammana_2012j, Tanrikulu_2015e, Tay_2004, Thapa_2019f, Trgo_2012f, Wadhwa_2015e, Yasin_2009, teWildt_2010}, 28 cases (39\%) were female \cite{AlShaaibi_2021b, Ali_2020f, Ataya_2013, Beecroft_1998, Bhasin_2014, Bhattacharjee_2008, Cauchi_2002, Chang_2017f, Cox_2007, DelgadoSalazar_2020c, DivsalarP._2023a, Goldman_1998f, Hardy_2023g, Kar_2015, Kariholu_2008, Kerestes_2019, Li_2013, Naji_2012f, Ohno_2005, Peixoto_2017f, Sakellaridis_2008f, Sultan_2024f, Tupesis_2004f, Wildhaber_2005, Wnęk_2015f, Yildiz_2016e}, 1 case (1\%) had no gender recorded \cite{fjbuilsRepeatedBehaviorDeliberate2024}. \paragraph*{Age Group} 25 cases (35\%) were between 26 and 40 years of age \cite{Alao_2006i, Ali_2022g, Apikotoa_2022f, Ataya_2013, Benoist_2019e, Bhasin_2014, Chang_2017f, Cox_2007, DelgadoSalazar_2020c, Farhadi_2024h, Fry_2010, Gardner_2017h, Guinan_2019f, Jin_2023, Kumar_2019f, Losanoff_1996, Misra_2013, Qureshi_2016, Riva_2018j, Sakellaridis_2008f, Tammana_2012j, Trgo_2012f, Wnęk_2015f, Yildiz_2016e, fjbuilsRepeatedBehaviorDeliberate2024}, 18 cases (25\%) were between 18 and 25 years of age \cite{Akay_2015f, Ali_2017, Atayan_2016, Bhattacharjee_2008, Csaky_1998e, Kar_2015, Kariholu_2008, Kobiela_2015, Losanoff_1996, Losanoff_1997e, Mesfin_2022a, Peixoto_2017f, Sobnach_2011f, Tupesis_2004f, Yasin_2009}, 13 cases (18\%) were under 18 years of age \cite{AlShaaibi_2021b, Ali_2020f, Cauchi_2002, DivsalarP._2023a, Goldman_1998f, Liu_2005, Naji_2012f, Ohno_2005, Tanrikulu_2015e, Tay_2004, Wildhaber_2005}, 11 cases (15\%) were between 41 and 60 years of age \cite{Al-Faham_2020k, Bhumi_2024f, CamachoDorado_2018, Emamhadi_2018, Hardy_2023g, Jehangir_2019h, Kumar_2001, Sultan_2024f, Thapa_2019f, Wadhwa_2015e, teWildt_2010}, 3 cases (4\%) were over 60 years of age \cite{Beecroft_1998, Kerestes_2019, Li_2013}, 2 cases (3\%) had no age documented \cite{Berry_2021e}. All 90 were male gender. 90 cases (100\%) were detained at the time of ingestion \cite{Elghali_2016, Karp_1991b, Lee_2007}, 88 cases (98\%) were intentional ingestions \cite{Elghali_2016, Karp_1991b, Lee_2007}, 30 cases (33\%) had a psychiatric history documented \cite{Elghali_2016, Karp_1991b, Lee_2007}, 2 cases (2\%) had a history of prior ingestion \cite{Elghali_2016}. No cases were reported for were psychiatric inpatients, were displaced people, were under the influence of alcohol at the time of ingestion, and had a severe disability history.
\paragraph*{Motivation}  70 cases (78\%) reported protest motivation \cite{Elghali_2016, Karp_1991b, Lee_2007}, 12 cases (13\%) reported psychiatric motivation \cite{Karp_1991b}, 6 cases (7\%) reported self-harm motivation \cite{Elghali_2016, Karp_1991b}. No cases were reported for psychosocial motivation and other motivation.
\paragraph*{Object Characteristics}  68 cases (76\%) involved sharp object ingestion \cite{Elghali_2016, Karp_1991b, Lee_2007}, 32 cases (36\%) involved long (\textgreater 5cm) object ingestion \cite{Lee_2007}, 25 cases (28\%) involved ingestion of multiple objects \cite{Elghali_2016, Lee_2007}. No cases were reported for button battery ingestion, magnet ingestion, and involved large diameter (\textgreater 2.5cm) object ingestion.
\paragraph*{Outcomes}  47 cases (52\%) underwent endoscopic intervention \cite{Elghali_2016, Lee_2007}, 29 cases (32\%) were managed conservatively \cite{Elghali_2016, Karp_1991b}, 15 cases (17\%) underwent surgical intervention \cite{Elghali_2016, Karp_1991b, Lee_2007}, 6 cases (7\%) reported complications \cite{Lee_2007}, 1 case (1\%) died \cite{Elghali_2016}.
\paragraph*{Gender} 43 cases (60\%) were male \cite{Akay_2015f, Al-Faham_2020k, Alao_2006i, Ali_2017, Ali_2022g, Apikotoa_2022f, Atayan_2016, Benoist_2019e, Berry_2021e, Bhumi_2024f, CamachoDorado_2018, Csaky_1998e, Emamhadi_2018, Farhadi_2024h, Fry_2010, Gardner_2017h, Guinan_2019f, Jehangir_2019h, Jin_2023, Kobiela_2015, Kumar_2001, Kumar_2019f, Liu_2005, Losanoff_1996, Losanoff_1997e, Mesfin_2022a, Misra_2013, Qureshi_2016, Riva_2018j, Sobnach_2011f, Tammana_2012j, Tanrikulu_2015e, Tay_2004, Thapa_2019f, Trgo_2012f, Wadhwa_2015e, Yasin_2009, teWildt_2010}, 28 cases (39\%) were female \cite{AlShaaibi_2021b, Ali_2020f, Ataya_2013, Beecroft_1998, Bhasin_2014, Bhattacharjee_2008, Cauchi_2002, Chang_2017f, Cox_2007, DelgadoSalazar_2020c, DivsalarP._2023a, Goldman_1998f, Hardy_2023g, Kar_2015, Kariholu_2008, Kerestes_2019, Li_2013, Naji_2012f, Ohno_2005, Peixoto_2017f, Sakellaridis_2008f, Sultan_2024f, Tupesis_2004f, Wildhaber_2005, Wnęk_2015f, Yildiz_2016e}, 1 case (1\%) had no gender recorded \cite{fjbuilsRepeatedBehaviorDeliberate2024}. \paragraph*{Age Group} 25 cases (35\%) were between 26 and 40 years of age \cite{Alao_2006i, Ali_2022g, Apikotoa_2022f, Ataya_2013, Benoist_2019e, Bhasin_2014, Chang_2017f, Cox_2007, DelgadoSalazar_2020c, Farhadi_2024h, Fry_2010, Gardner_2017h, Guinan_2019f, Jin_2023, Kumar_2019f, Losanoff_1996, Misra_2013, Qureshi_2016, Riva_2018j, Sakellaridis_2008f, Tammana_2012j, Trgo_2012f, Wnęk_2015f, Yildiz_2016e, fjbuilsRepeatedBehaviorDeliberate2024}, 18 cases (25\%) were between 18 and 25 years of age \cite{Akay_2015f, Ali_2017, Atayan_2016, Bhattacharjee_2008, Csaky_1998e, Kar_2015, Kariholu_2008, Kobiela_2015, Losanoff_1996, Losanoff_1997e, Mesfin_2022a, Peixoto_2017f, Sobnach_2011f, Tupesis_2004f, Yasin_2009}, 13 cases (18\%) were under 18 years of age \cite{AlShaaibi_2021b, Ali_2020f, Cauchi_2002, DivsalarP._2023a, Goldman_1998f, Liu_2005, Naji_2012f, Ohno_2005, Tanrikulu_2015e, Tay_2004, Wildhaber_2005}, 11 cases (15\%) were between 41 and 60 years of age \cite{Al-Faham_2020k, Bhumi_2024f, CamachoDorado_2018, Emamhadi_2018, Hardy_2023g, Jehangir_2019h, Kumar_2001, Sultan_2024f, Thapa_2019f, Wadhwa_2015e, teWildt_2010}, 3 cases (4\%) were over 60 years of age \cite{Beecroft_1998, Kerestes_2019, Li_2013}, 2 cases (3\%) had no age documented \cite{Berry_2021e}. All 90 were male gender. 90 cases (100\%) were detained at the time of ingestion \cite{Elghali_2016, Karp_1991b, Lee_2007}, 88 cases (98\%) were intentional ingestions \cite{Elghali_2016, Karp_1991b, Lee_2007}, 30 cases (33\%) had a psychiatric history documented \cite{Elghali_2016, Karp_1991b, Lee_2007}, 2 cases (2\%) had a history of prior ingestion \cite{Elghali_2016}. No cases were reported for were psychiatric inpatients, were displaced people, were under the influence of alcohol at the time of ingestion, and had a severe disability history.
\paragraph*{Motivation}  70 cases (78\%) reported protest motivation \cite{Elghali_2016, Karp_1991b, Lee_2007}, 12 cases (13\%) reported psychiatric motivation \cite{Karp_1991b}, 6 cases (7\%) reported self-harm motivation \cite{Elghali_2016, Karp_1991b}. No cases were reported for psychosocial motivation and other motivation.
\paragraph*{Object Characteristics}  68 cases (76\%) involved sharp object ingestion \cite{Elghali_2016, Karp_1991b, Lee_2007}, 32 cases (36\%) involved long (\textgreater 5cm) object ingestion \cite{Lee_2007}, 25 cases (28\%) involved ingestion of multiple objects \cite{Elghali_2016, Lee_2007}. No cases were reported for button battery ingestion, magnet ingestion, and involved large diameter (\textgreater 2.5cm) object ingestion.
\paragraph*{Outcomes}  47 cases (52\%) underwent endoscopic intervention \cite{Elghali_2016, Lee_2007}, 29 cases (32\%) were managed conservatively \cite{Elghali_2016, Karp_1991b}, 15 cases (17\%) underwent surgical intervention \cite{Elghali_2016, Karp_1991b, Lee_2007}, 6 cases (7\%) reported complications \cite{Lee_2007}, 1 case (1\%) died \cite{Elghali_2016}.
\paragraph*{Gender} 43 cases (60\%) were male \cite{Akay_2015f, Al-Faham_2020k, Alao_2006i, Ali_2017, Ali_2022g, Apikotoa_2022f, Atayan_2016, Benoist_2019e, Berry_2021e, Bhumi_2024f, CamachoDorado_2018, Csaky_1998e, Emamhadi_2018, Farhadi_2024h, Fry_2010, Gardner_2017h, Guinan_2019f, Jehangir_2019h, Jin_2023, Kobiela_2015, Kumar_2001, Kumar_2019f, Liu_2005, Losanoff_1996, Losanoff_1997e, Mesfin_2022a, Misra_2013, Qureshi_2016, Riva_2018j, Sobnach_2011f, Tammana_2012j, Tanrikulu_2015e, Tay_2004, Thapa_2019f, Trgo_2012f, Wadhwa_2015e, Yasin_2009, teWildt_2010}, 28 cases (39\%) were female \cite{AlShaaibi_2021b, Ali_2020f, Ataya_2013, Beecroft_1998, Bhasin_2014, Bhattacharjee_2008, Cauchi_2002, Chang_2017f, Cox_2007, DelgadoSalazar_2020c, DivsalarP._2023a, Goldman_1998f, Hardy_2023g, Kar_2015, Kariholu_2008, Kerestes_2019, Li_2013, Naji_2012f, Ohno_2005, Peixoto_2017f, Sakellaridis_2008f, Sultan_2024f, Tupesis_2004f, Wildhaber_2005, Wnęk_2015f, Yildiz_2016e}, 1 case (1\%) had no gender recorded \cite{fjbuilsRepeatedBehaviorDeliberate2024}. \paragraph*{Age Group} 25 cases (35\%) were between 26 and 40 years of age \cite{Alao_2006i, Ali_2022g, Apikotoa_2022f, Ataya_2013, Benoist_2019e, Bhasin_2014, Chang_2017f, Cox_2007, DelgadoSalazar_2020c, Farhadi_2024h, Fry_2010, Gardner_2017h, Guinan_2019f, Jin_2023, Kumar_2019f, Losanoff_1996, Misra_2013, Qureshi_2016, Riva_2018j, Sakellaridis_2008f, Tammana_2012j, Trgo_2012f, Wnęk_2015f, Yildiz_2016e, fjbuilsRepeatedBehaviorDeliberate2024}, 18 cases (25\%) were between 18 and 25 years of age \cite{Akay_2015f, Ali_2017, Atayan_2016, Bhattacharjee_2008, Csaky_1998e, Kar_2015, Kariholu_2008, Kobiela_2015, Losanoff_1996, Losanoff_1997e, Mesfin_2022a, Peixoto_2017f, Sobnach_2011f, Tupesis_2004f, Yasin_2009}, 13 cases (18\%) were under 18 years of age \cite{AlShaaibi_2021b, Ali_2020f, Cauchi_2002, DivsalarP._2023a, Goldman_1998f, Liu_2005, Naji_2012f, Ohno_2005, Tanrikulu_2015e, Tay_2004, Wildhaber_2005}, 11 cases (15\%) were between 41 and 60 years of age \cite{Al-Faham_2020k, Bhumi_2024f, CamachoDorado_2018, Emamhadi_2018, Hardy_2023g, Jehangir_2019h, Kumar_2001, Sultan_2024f, Thapa_2019f, Wadhwa_2015e, teWildt_2010}, 3 cases (4\%) were over 60 years of age \cite{Beecroft_1998, Kerestes_2019, Li_2013}, 2 cases (3\%) had no age documented \cite{Berry_2021e}. All 90 were male gender. 90 cases (100\%) were detained at the time of ingestion \cite{Elghali_2016, Karp_1991b, Lee_2007}, 88 cases (98\%) were intentional ingestions \cite{Elghali_2016, Karp_1991b, Lee_2007}, 30 cases (33\%) had a psychiatric history documented \cite{Elghali_2016, Karp_1991b, Lee_2007}, 2 cases (2\%) had a history of prior ingestion \cite{Elghali_2016}. No cases were reported for were psychiatric inpatients, were displaced people, were under the influence of alcohol at the time of ingestion, and had a severe disability history.
\paragraph*{Motivation}  70 cases (78\%) reported protest motivation \cite{Elghali_2016, Karp_1991b, Lee_2007}, 12 cases (13\%) reported psychiatric motivation \cite{Karp_1991b}, 6 cases (7\%) reported self-harm motivation \cite{Elghali_2016, Karp_1991b}. No cases were reported for psychosocial motivation and other motivation.
\paragraph*{Object Characteristics}  68 cases (76\%) involved sharp object ingestion \cite{Elghali_2016, Karp_1991b, Lee_2007}, 32 cases (36\%) involved long (\textgreater 5cm) object ingestion \cite{Lee_2007}, 25 cases (28\%) involved ingestion of multiple objects \cite{Elghali_2016, Lee_2007}. No cases were reported for button battery ingestion, magnet ingestion, and involved large diameter (\textgreater 2.5cm) object ingestion.
\paragraph*{Outcomes}  47 cases (52\%) underwent endoscopic intervention \cite{Elghali_2016, Lee_2007}, 29 cases (32\%) were managed conservatively \cite{Elghali_2016, Karp_1991b}, 15 cases (17\%) underwent surgical intervention \cite{Elghali_2016, Karp_1991b, Lee_2007}, 6 cases (7\%) reported complications \cite{Lee_2007}, 1 case (1\%) died \cite{Elghali_2016}.
\paragraph*{Gender} 43 cases (60\%) were male \cite{Akay_2015f, Al-Faham_2020k, Alao_2006i, Ali_2017, Ali_2022g, Apikotoa_2022f, Atayan_2016, Benoist_2019e, Berry_2021e, Bhumi_2024f, CamachoDorado_2018, Csaky_1998e, Emamhadi_2018, Farhadi_2024h, Fry_2010, Gardner_2017h, Guinan_2019f, Jehangir_2019h, Jin_2023, Kobiela_2015, Kumar_2001, Kumar_2019f, Liu_2005, Losanoff_1996, Losanoff_1997e, Mesfin_2022a, Misra_2013, Qureshi_2016, Riva_2018j, Sobnach_2011f, Tammana_2012j, Tanrikulu_2015e, Tay_2004, Thapa_2019f, Trgo_2012f, Wadhwa_2015e, Yasin_2009, teWildt_2010}, 28 cases (39\%) were female \cite{AlShaaibi_2021b, Ali_2020f, Ataya_2013, Beecroft_1998, Bhasin_2014, Bhattacharjee_2008, Cauchi_2002, Chang_2017f, Cox_2007, DelgadoSalazar_2020c, DivsalarP._2023a, Goldman_1998f, Hardy_2023g, Kar_2015, Kariholu_2008, Kerestes_2019, Li_2013, Naji_2012f, Ohno_2005, Peixoto_2017f, Sakellaridis_2008f, Sultan_2024f, Tupesis_2004f, Wildhaber_2005, Wnęk_2015f, Yildiz_2016e}, 1 case (1\%) had no gender recorded \cite{fjbuilsRepeatedBehaviorDeliberate2024}. \paragraph*{Age Group} 25 cases (35\%) were between 26 and 40 years of age \cite{Alao_2006i, Ali_2022g, Apikotoa_2022f, Ataya_2013, Benoist_2019e, Bhasin_2014, Chang_2017f, Cox_2007, DelgadoSalazar_2020c, Farhadi_2024h, Fry_2010, Gardner_2017h, Guinan_2019f, Jin_2023, Kumar_2019f, Losanoff_1996, Misra_2013, Qureshi_2016, Riva_2018j, Sakellaridis_2008f, Tammana_2012j, Trgo_2012f, Wnęk_2015f, Yildiz_2016e, fjbuilsRepeatedBehaviorDeliberate2024}, 18 cases (25\%) were between 18 and 25 years of age \cite{Akay_2015f, Ali_2017, Atayan_2016, Bhattacharjee_2008, Csaky_1998e, Kar_2015, Kariholu_2008, Kobiela_2015, Losanoff_1996, Losanoff_1997e, Mesfin_2022a, Peixoto_2017f, Sobnach_2011f, Tupesis_2004f, Yasin_2009}, 13 cases (18\%) were under 18 years of age \cite{AlShaaibi_2021b, Ali_2020f, Cauchi_2002, DivsalarP._2023a, Goldman_1998f, Liu_2005, Naji_2012f, Ohno_2005, Tanrikulu_2015e, Tay_2004, Wildhaber_2005}, 11 cases (15\%) were between 41 and 60 years of age \cite{Al-Faham_2020k, Bhumi_2024f, CamachoDorado_2018, Emamhadi_2018, Hardy_2023g, Jehangir_2019h, Kumar_2001, Sultan_2024f, Thapa_2019f, Wadhwa_2015e, teWildt_2010}, 3 cases (4\%) were over 60 years of age \cite{Beecroft_1998, Kerestes_2019, Li_2013}, 2 cases (3\%) had no age documented \cite{Berry_2021e}. All 90 were male gender. 90 cases (100\%) were detained at the time of ingestion \cite{Elghali_2016, Karp_1991b, Lee_2007}, 88 cases (98\%) were intentional ingestions \cite{Elghali_2016, Karp_1991b, Lee_2007}, 30 cases (33\%) had a psychiatric history documented \cite{Elghali_2016, Karp_1991b, Lee_2007}, 2 cases (2\%) had a history of prior ingestion \cite{Elghali_2016}. No cases were reported for were psychiatric inpatients, were displaced people, were under the influence of alcohol at the time of ingestion, and had a severe disability history.
\paragraph*{Motivation}  70 cases (78\%) reported protest motivation \cite{Elghali_2016, Karp_1991b, Lee_2007}, 12 cases (13\%) reported psychiatric motivation \cite{Karp_1991b}, 6 cases (7\%) reported self-harm motivation \cite{Elghali_2016, Karp_1991b}. No cases were reported for psychosocial motivation and other motivation.
\paragraph*{Object Characteristics}  68 cases (76\%) involved sharp object ingestion \cite{Elghali_2016, Karp_1991b, Lee_2007}, 32 cases (36\%) involved long (\textgreater 5cm) object ingestion \cite{Lee_2007}, 25 cases (28\%) involved ingestion of multiple objects \cite{Elghali_2016, Lee_2007}. No cases were reported for button battery ingestion, magnet ingestion, and involved large diameter (\textgreater 2.5cm) object ingestion.
\paragraph*{Outcomes}  47 cases (52\%) underwent endoscopic intervention \cite{Elghali_2016, Lee_2007}, 29 cases (32\%) were managed conservatively \cite{Elghali_2016, Karp_1991b}, 15 cases (17\%) underwent surgical intervention \cite{Elghali_2016, Karp_1991b, Lee_2007}, 6 cases (7\%) reported complications \cite{Lee_2007}, 1 case (1\%) died \cite{Elghali_2016}.
\paragraph*{Gender} 43 cases (60\%) were male \cite{Akay_2015f, Al-Faham_2020k, Alao_2006i, Ali_2017, Ali_2022g, Apikotoa_2022f, Atayan_2016, Benoist_2019e, Berry_2021e, Bhumi_2024f, CamachoDorado_2018, Csaky_1998e, Emamhadi_2018, Farhadi_2024h, Fry_2010, Gardner_2017h, Guinan_2019f, Jehangir_2019h, Jin_2023, Kobiela_2015, Kumar_2001, Kumar_2019f, Liu_2005, Losanoff_1996, Losanoff_1997e, Mesfin_2022a, Misra_2013, Qureshi_2016, Riva_2018j, Sobnach_2011f, Tammana_2012j, Tanrikulu_2015e, Tay_2004, Thapa_2019f, Trgo_2012f, Wadhwa_2015e, Yasin_2009, teWildt_2010}, 28 cases (39\%) were female \cite{AlShaaibi_2021b, Ali_2020f, Ataya_2013, Beecroft_1998, Bhasin_2014, Bhattacharjee_2008, Cauchi_2002, Chang_2017f, Cox_2007, DelgadoSalazar_2020c, DivsalarP._2023a, Goldman_1998f, Hardy_2023g, Kar_2015, Kariholu_2008, Kerestes_2019, Li_2013, Naji_2012f, Ohno_2005, Peixoto_2017f, Sakellaridis_2008f, Sultan_2024f, Tupesis_2004f, Wildhaber_2005, Wnęk_2015f, Yildiz_2016e}, 1 case (1\%) had no gender recorded \cite{fjbuilsRepeatedBehaviorDeliberate2024}. \paragraph*{Age Group} 25 cases (35\%) were between 26 and 40 years of age \cite{Alao_2006i, Ali_2022g, Apikotoa_2022f, Ataya_2013, Benoist_2019e, Bhasin_2014, Chang_2017f, Cox_2007, DelgadoSalazar_2020c, Farhadi_2024h, Fry_2010, Gardner_2017h, Guinan_2019f, Jin_2023, Kumar_2019f, Losanoff_1996, Misra_2013, Qureshi_2016, Riva_2018j, Sakellaridis_2008f, Tammana_2012j, Trgo_2012f, Wnęk_2015f, Yildiz_2016e, fjbuilsRepeatedBehaviorDeliberate2024}, 18 cases (25\%) were between 18 and 25 years of age \cite{Akay_2015f, Ali_2017, Atayan_2016, Bhattacharjee_2008, Csaky_1998e, Kar_2015, Kariholu_2008, Kobiela_2015, Losanoff_1996, Losanoff_1997e, Mesfin_2022a, Peixoto_2017f, Sobnach_2011f, Tupesis_2004f, Yasin_2009}, 13 cases (18\%) were under 18 years of age \cite{AlShaaibi_2021b, Ali_2020f, Cauchi_2002, DivsalarP._2023a, Goldman_1998f, Liu_2005, Naji_2012f, Ohno_2005, Tanrikulu_2015e, Tay_2004, Wildhaber_2005}, 11 cases (15\%) were between 41 and 60 years of age \cite{Al-Faham_2020k, Bhumi_2024f, CamachoDorado_2018, Emamhadi_2018, Hardy_2023g, Jehangir_2019h, Kumar_2001, Sultan_2024f, Thapa_2019f, Wadhwa_2015e, teWildt_2010}, 3 cases (4\%) were over 60 years of age \cite{Beecroft_1998, Kerestes_2019, Li_2013}, 2 cases (3\%) had no age documented \cite{Berry_2021e}. All 90 were male gender. 90 cases (100\%) were detained at the time of ingestion \cite{Elghali_2016, Karp_1991b, Lee_2007}, 88 cases (98\%) were intentional ingestions \cite{Elghali_2016, Karp_1991b, Lee_2007}, 30 cases (33\%) had a psychiatric history documented \cite{Elghali_2016, Karp_1991b, Lee_2007}, 2 cases (2\%) had a history of prior ingestion \cite{Elghali_2016}. No cases were reported for were psychiatric inpatients, were displaced people, were under the influence of alcohol at the time of ingestion, and had a severe disability history.
\paragraph*{Motivation}  70 cases (78\%) reported protest motivation \cite{Elghali_2016, Karp_1991b, Lee_2007}, 12 cases (13\%) reported psychiatric motivation \cite{Karp_1991b}, 6 cases (7\%) reported self-harm motivation \cite{Elghali_2016, Karp_1991b}. No cases were reported for psychosocial motivation and other motivation.
\paragraph*{Object Characteristics}  68 cases (76\%) involved sharp object ingestion \cite{Elghali_2016, Karp_1991b, Lee_2007}, 32 cases (36\%) involved long (\textgreater 5cm) object ingestion \cite{Lee_2007}, 25 cases (28\%) involved ingestion of multiple objects \cite{Elghali_2016, Lee_2007}. No cases were reported for button battery ingestion, magnet ingestion, and involved large diameter (\textgreater 2.5cm) object ingestion.
\paragraph*{Outcomes}  47 cases (52\%) underwent endoscopic intervention \cite{Elghali_2016, Lee_2007}, 29 cases (32\%) were managed conservatively \cite{Elghali_2016, Karp_1991b}, 15 cases (17\%) underwent surgical intervention \cite{Elghali_2016, Karp_1991b, Lee_2007}, 6 cases (7\%) reported complications \cite{Lee_2007}, 1 case (1\%) died \cite{Elghali_2016}.
\paragraph*{Gender} 43 cases (60\%) were male \cite{Akay_2015f, Al-Faham_2020k, Alao_2006i, Ali_2017, Ali_2022g, Apikotoa_2022f, Atayan_2016, Benoist_2019e, Berry_2021e, Bhumi_2024f, CamachoDorado_2018, Csaky_1998e, Emamhadi_2018, Farhadi_2024h, Fry_2010, Gardner_2017h, Guinan_2019f, Jehangir_2019h, Jin_2023, Kobiela_2015, Kumar_2001, Kumar_2019f, Liu_2005, Losanoff_1996, Losanoff_1997e, Mesfin_2022a, Misra_2013, Qureshi_2016, Riva_2018j, Sobnach_2011f, Tammana_2012j, Tanrikulu_2015e, Tay_2004, Thapa_2019f, Trgo_2012f, Wadhwa_2015e, Yasin_2009, teWildt_2010}, 28 cases (39\%) were female \cite{AlShaaibi_2021b, Ali_2020f, Ataya_2013, Beecroft_1998, Bhasin_2014, Bhattacharjee_2008, Cauchi_2002, Chang_2017f, Cox_2007, DelgadoSalazar_2020c, DivsalarP._2023a, Goldman_1998f, Hardy_2023g, Kar_2015, Kariholu_2008, Kerestes_2019, Li_2013, Naji_2012f, Ohno_2005, Peixoto_2017f, Sakellaridis_2008f, Sultan_2024f, Tupesis_2004f, Wildhaber_2005, Wnęk_2015f, Yildiz_2016e}, 1 case (1\%) had no gender recorded \cite{fjbuilsRepeatedBehaviorDeliberate2024}. \paragraph*{Age Group} 25 cases (35\%) were between 26 and 40 years of age \cite{Alao_2006i, Ali_2022g, Apikotoa_2022f, Ataya_2013, Benoist_2019e, Bhasin_2014, Chang_2017f, Cox_2007, DelgadoSalazar_2020c, Farhadi_2024h, Fry_2010, Gardner_2017h, Guinan_2019f, Jin_2023, Kumar_2019f, Losanoff_1996, Misra_2013, Qureshi_2016, Riva_2018j, Sakellaridis_2008f, Tammana_2012j, Trgo_2012f, Wnęk_2015f, Yildiz_2016e, fjbuilsRepeatedBehaviorDeliberate2024}, 18 cases (25\%) were between 18 and 25 years of age \cite{Akay_2015f, Ali_2017, Atayan_2016, Bhattacharjee_2008, Csaky_1998e, Kar_2015, Kariholu_2008, Kobiela_2015, Losanoff_1996, Losanoff_1997e, Mesfin_2022a, Peixoto_2017f, Sobnach_2011f, Tupesis_2004f, Yasin_2009}, 13 cases (18\%) were under 18 years of age \cite{AlShaaibi_2021b, Ali_2020f, Cauchi_2002, DivsalarP._2023a, Goldman_1998f, Liu_2005, Naji_2012f, Ohno_2005, Tanrikulu_2015e, Tay_2004, Wildhaber_2005}, 11 cases (15\%) were between 41 and 60 years of age \cite{Al-Faham_2020k, Bhumi_2024f, CamachoDorado_2018, Emamhadi_2018, Hardy_2023g, Jehangir_2019h, Kumar_2001, Sultan_2024f, Thapa_2019f, Wadhwa_2015e, teWildt_2010}, 3 cases (4\%) were over 60 years of age \cite{Beecroft_1998, Kerestes_2019, Li_2013}, 2 cases (3\%) had no age documented \cite{Berry_2021e}. All 90 were male gender. 90 cases (100\%) were detained at the time of ingestion \cite{Elghali_2016, Karp_1991b, Lee_2007}, 88 cases (98\%) were intentional ingestions \cite{Elghali_2016, Karp_1991b, Lee_2007}, 30 cases (33\%) had a psychiatric history documented \cite{Elghali_2016, Karp_1991b, Lee_2007}, 2 cases (2\%) had a history of prior ingestion \cite{Elghali_2016}. No cases were reported for were psychiatric inpatients, were displaced people, were under the influence of alcohol at the time of ingestion, and had a severe disability history.
\paragraph*{Motivation}  70 cases (78\%) reported protest motivation \cite{Elghali_2016, Karp_1991b, Lee_2007}, 12 cases (13\%) reported psychiatric motivation \cite{Karp_1991b}, 6 cases (7\%) reported self-harm motivation \cite{Elghali_2016, Karp_1991b}. No cases were reported for psychosocial motivation and other motivation.
\paragraph*{Object Characteristics}  68 cases (76\%) involved sharp object ingestion \cite{Elghali_2016, Karp_1991b, Lee_2007}, 32 cases (36\%) involved long (\textgreater 5cm) object ingestion \cite{Lee_2007}, 25 cases (28\%) involved ingestion of multiple objects \cite{Elghali_2016, Lee_2007}. No cases were reported for button battery ingestion, magnet ingestion, and involved large diameter (\textgreater 2.5cm) object ingestion.
\paragraph*{Outcomes}  47 cases (52\%) underwent endoscopic intervention \cite{Elghali_2016, Lee_2007}, 29 cases (32\%) were managed conservatively \cite{Elghali_2016, Karp_1991b}, 15 cases (17\%) underwent surgical intervention \cite{Elghali_2016, Karp_1991b, Lee_2007}, 6 cases (7\%) reported complications \cite{Lee_2007}, 1 case (1\%) died \cite{Elghali_2016}.
\paragraph*{Gender} 43 cases (60\%) were male \cite{Akay_2015f, Al-Faham_2020k, Alao_2006i, Ali_2017, Ali_2022g, Apikotoa_2022f, Atayan_2016, Benoist_2019e, Berry_2021e, Bhumi_2024f, CamachoDorado_2018, Csaky_1998e, Emamhadi_2018, Farhadi_2024h, Fry_2010, Gardner_2017h, Guinan_2019f, Jehangir_2019h, Jin_2023, Kobiela_2015, Kumar_2001, Kumar_2019f, Liu_2005, Losanoff_1996, Losanoff_1997e, Mesfin_2022a, Misra_2013, Qureshi_2016, Riva_2018j, Sobnach_2011f, Tammana_2012j, Tanrikulu_2015e, Tay_2004, Thapa_2019f, Trgo_2012f, Wadhwa_2015e, Yasin_2009, teWildt_2010}, 28 cases (39\%) were female \cite{AlShaaibi_2021b, Ali_2020f, Ataya_2013, Beecroft_1998, Bhasin_2014, Bhattacharjee_2008, Cauchi_2002, Chang_2017f, Cox_2007, DelgadoSalazar_2020c, DivsalarP._2023a, Goldman_1998f, Hardy_2023g, Kar_2015, Kariholu_2008, Kerestes_2019, Li_2013, Naji_2012f, Ohno_2005, Peixoto_2017f, Sakellaridis_2008f, Sultan_2024f, Tupesis_2004f, Wildhaber_2005, Wnęk_2015f, Yildiz_2016e}, 1 case (1\%) had no gender recorded \cite{fjbuilsRepeatedBehaviorDeliberate2024}. \paragraph*{Age Group} 25 cases (35\%) were between 26 and 40 years of age \cite{Alao_2006i, Ali_2022g, Apikotoa_2022f, Ataya_2013, Benoist_2019e, Bhasin_2014, Chang_2017f, Cox_2007, DelgadoSalazar_2020c, Farhadi_2024h, Fry_2010, Gardner_2017h, Guinan_2019f, Jin_2023, Kumar_2019f, Losanoff_1996, Misra_2013, Qureshi_2016, Riva_2018j, Sakellaridis_2008f, Tammana_2012j, Trgo_2012f, Wnęk_2015f, Yildiz_2016e, fjbuilsRepeatedBehaviorDeliberate2024}, 18 cases (25\%) were between 18 and 25 years of age \cite{Akay_2015f, Ali_2017, Atayan_2016, Bhattacharjee_2008, Csaky_1998e, Kar_2015, Kariholu_2008, Kobiela_2015, Losanoff_1996, Losanoff_1997e, Mesfin_2022a, Peixoto_2017f, Sobnach_2011f, Tupesis_2004f, Yasin_2009}, 13 cases (18\%) were under 18 years of age \cite{AlShaaibi_2021b, Ali_2020f, Cauchi_2002, DivsalarP._2023a, Goldman_1998f, Liu_2005, Naji_2012f, Ohno_2005, Tanrikulu_2015e, Tay_2004, Wildhaber_2005}, 11 cases (15\%) were between 41 and 60 years of age \cite{Al-Faham_2020k, Bhumi_2024f, CamachoDorado_2018, Emamhadi_2018, Hardy_2023g, Jehangir_2019h, Kumar_2001, Sultan_2024f, Thapa_2019f, Wadhwa_2015e, teWildt_2010}, 3 cases (4\%) were over 60 years of age \cite{Beecroft_1998, Kerestes_2019, Li_2013}, 2 cases (3\%) had no age documented \cite{Berry_2021e}. All 90 were male gender. 90 cases (100\%) were detained at the time of ingestion \cite{Elghali_2016, Karp_1991b, Lee_2007}, 88 cases (98\%) were intentional ingestions \cite{Elghali_2016, Karp_1991b, Lee_2007}, 30 cases (33\%) had a psychiatric history documented \cite{Elghali_2016, Karp_1991b, Lee_2007}, 2 cases (2\%) had a history of prior ingestion \cite{Elghali_2016}. No cases were reported for were psychiatric inpatients, were displaced people, were under the influence of alcohol at the time of ingestion, and had a severe disability history.
\paragraph*{Motivation}  70 cases (78\%) reported protest motivation \cite{Elghali_2016, Karp_1991b, Lee_2007}, 12 cases (13\%) reported psychiatric motivation \cite{Karp_1991b}, 6 cases (7\%) reported self-harm motivation \cite{Elghali_2016, Karp_1991b}. No cases were reported for psychosocial motivation and other motivation.
\paragraph*{Object Characteristics}  68 cases (76\%) involved sharp object ingestion \cite{Elghali_2016, Karp_1991b, Lee_2007}, 32 cases (36\%) involved long (\textgreater 5cm) object ingestion \cite{Lee_2007}, 25 cases (28\%) involved ingestion of multiple objects \cite{Elghali_2016, Lee_2007}. No cases were reported for button battery ingestion, magnet ingestion, and involved large diameter (\textgreater 2.5cm) object ingestion.
\paragraph*{Outcomes}  47 cases (52\%) underwent endoscopic intervention \cite{Elghali_2016, Lee_2007}, 29 cases (32\%) were managed conservatively \cite{Elghali_2016, Karp_1991b}, 15 cases (17\%) underwent surgical intervention \cite{Elghali_2016, Karp_1991b, Lee_2007}, 6 cases (7\%) reported complications \cite{Lee_2007}, 1 case (1\%) died \cite{Elghali_2016}.
\paragraph*{Gender} 43 cases (60\%) were male \cite{Akay_2015f, Al-Faham_2020k, Alao_2006i, Ali_2017, Ali_2022g, Apikotoa_2022f, Atayan_2016, Benoist_2019e, Berry_2021e, Bhumi_2024f, CamachoDorado_2018, Csaky_1998e, Emamhadi_2018, Farhadi_2024h, Fry_2010, Gardner_2017h, Guinan_2019f, Jehangir_2019h, Jin_2023, Kobiela_2015, Kumar_2001, Kumar_2019f, Liu_2005, Losanoff_1996, Losanoff_1997e, Mesfin_2022a, Misra_2013, Qureshi_2016, Riva_2018j, Sobnach_2011f, Tammana_2012j, Tanrikulu_2015e, Tay_2004, Thapa_2019f, Trgo_2012f, Wadhwa_2015e, Yasin_2009, teWildt_2010}, 28 cases (39\%) were female \cite{AlShaaibi_2021b, Ali_2020f, Ataya_2013, Beecroft_1998, Bhasin_2014, Bhattacharjee_2008, Cauchi_2002, Chang_2017f, Cox_2007, DelgadoSalazar_2020c, DivsalarP._2023a, Goldman_1998f, Hardy_2023g, Kar_2015, Kariholu_2008, Kerestes_2019, Li_2013, Naji_2012f, Ohno_2005, Peixoto_2017f, Sakellaridis_2008f, Sultan_2024f, Tupesis_2004f, Wildhaber_2005, Wnęk_2015f, Yildiz_2016e}, 1 case (1\%) had no gender recorded \cite{fjbuilsRepeatedBehaviorDeliberate2024}. \paragraph*{Age Group} 25 cases (35\%) were between 26 and 40 years of age \cite{Alao_2006i, Ali_2022g, Apikotoa_2022f, Ataya_2013, Benoist_2019e, Bhasin_2014, Chang_2017f, Cox_2007, DelgadoSalazar_2020c, Farhadi_2024h, Fry_2010, Gardner_2017h, Guinan_2019f, Jin_2023, Kumar_2019f, Losanoff_1996, Misra_2013, Qureshi_2016, Riva_2018j, Sakellaridis_2008f, Tammana_2012j, Trgo_2012f, Wnęk_2015f, Yildiz_2016e, fjbuilsRepeatedBehaviorDeliberate2024}, 18 cases (25\%) were between 18 and 25 years of age \cite{Akay_2015f, Ali_2017, Atayan_2016, Bhattacharjee_2008, Csaky_1998e, Kar_2015, Kariholu_2008, Kobiela_2015, Losanoff_1996, Losanoff_1997e, Mesfin_2022a, Peixoto_2017f, Sobnach_2011f, Tupesis_2004f, Yasin_2009}, 13 cases (18\%) were under 18 years of age \cite{AlShaaibi_2021b, Ali_2020f, Cauchi_2002, DivsalarP._2023a, Goldman_1998f, Liu_2005, Naji_2012f, Ohno_2005, Tanrikulu_2015e, Tay_2004, Wildhaber_2005}, 11 cases (15\%) were between 41 and 60 years of age \cite{Al-Faham_2020k, Bhumi_2024f, CamachoDorado_2018, Emamhadi_2018, Hardy_2023g, Jehangir_2019h, Kumar_2001, Sultan_2024f, Thapa_2019f, Wadhwa_2015e, teWildt_2010}, 3 cases (4\%) were over 60 years of age \cite{Beecroft_1998, Kerestes_2019, Li_2013}, 2 cases (3\%) had no age documented \cite{Berry_2021e}. All 90 were male gender. 90 cases (100\%) were detained at the time of ingestion \cite{Elghali_2016, Karp_1991b, Lee_2007}, 88 cases (98\%) were intentional ingestions \cite{Elghali_2016, Karp_1991b, Lee_2007}, 30 cases (33\%) had a psychiatric history documented \cite{Elghali_2016, Karp_1991b, Lee_2007}, 2 cases (2\%) had a history of prior ingestion \cite{Elghali_2016}. No cases were reported for were psychiatric inpatients, were displaced people, were under the influence of alcohol at the time of ingestion, and had a severe disability history.
\paragraph*{Motivation}  70 cases (78\%) reported protest motivation \cite{Elghali_2016, Karp_1991b, Lee_2007}, 12 cases (13\%) reported psychiatric motivation \cite{Karp_1991b}, 6 cases (7\%) reported self-harm motivation \cite{Elghali_2016, Karp_1991b}. No cases were reported for psychosocial motivation and other motivation.
\paragraph*{Object Characteristics}  68 cases (76\%) involved sharp object ingestion \cite{Elghali_2016, Karp_1991b, Lee_2007}, 32 cases (36\%) involved long (\textgreater 5cm) object ingestion \cite{Lee_2007}, 25 cases (28\%) involved ingestion of multiple objects \cite{Elghali_2016, Lee_2007}. No cases were reported for button battery ingestion, magnet ingestion, and involved large diameter (\textgreater 2.5cm) object ingestion.
\paragraph*{Outcomes}  47 cases (52\%) underwent endoscopic intervention \cite{Elghali_2016, Lee_2007}, 29 cases (32\%) were managed conservatively \cite{Elghali_2016, Karp_1991b}, 15 cases (17\%) underwent surgical intervention \cite{Elghali_2016, Karp_1991b, Lee_2007}, 6 cases (7\%) reported complications \cite{Lee_2007}, 1 case (1\%) died \cite{Elghali_2016}.
\paragraph*{Gender} 43 cases (60\%) were male \cite{Akay_2015f, Al-Faham_2020k, Alao_2006i, Ali_2017, Ali_2022g, Apikotoa_2022f, Atayan_2016, Benoist_2019e, Berry_2021e, Bhumi_2024f, CamachoDorado_2018, Csaky_1998e, Emamhadi_2018, Farhadi_2024h, Fry_2010, Gardner_2017h, Guinan_2019f, Jehangir_2019h, Jin_2023, Kobiela_2015, Kumar_2001, Kumar_2019f, Liu_2005, Losanoff_1996, Losanoff_1997e, Mesfin_2022a, Misra_2013, Qureshi_2016, Riva_2018j, Sobnach_2011f, Tammana_2012j, Tanrikulu_2015e, Tay_2004, Thapa_2019f, Trgo_2012f, Wadhwa_2015e, Yasin_2009, teWildt_2010}, 28 cases (39\%) were female \cite{AlShaaibi_2021b, Ali_2020f, Ataya_2013, Beecroft_1998, Bhasin_2014, Bhattacharjee_2008, Cauchi_2002, Chang_2017f, Cox_2007, DelgadoSalazar_2020c, DivsalarP._2023a, Goldman_1998f, Hardy_2023g, Kar_2015, Kariholu_2008, Kerestes_2019, Li_2013, Naji_2012f, Ohno_2005, Peixoto_2017f, Sakellaridis_2008f, Sultan_2024f, Tupesis_2004f, Wildhaber_2005, Wnęk_2015f, Yildiz_2016e}, 1 case (1\%) had no gender recorded \cite{fjbuilsRepeatedBehaviorDeliberate2024}. \paragraph*{Age Group} 25 cases (35\%) were between 26 and 40 years of age \cite{Alao_2006i, Ali_2022g, Apikotoa_2022f, Ataya_2013, Benoist_2019e, Bhasin_2014, Chang_2017f, Cox_2007, DelgadoSalazar_2020c, Farhadi_2024h, Fry_2010, Gardner_2017h, Guinan_2019f, Jin_2023, Kumar_2019f, Losanoff_1996, Misra_2013, Qureshi_2016, Riva_2018j, Sakellaridis_2008f, Tammana_2012j, Trgo_2012f, Wnęk_2015f, Yildiz_2016e, fjbuilsRepeatedBehaviorDeliberate2024}, 18 cases (25\%) were between 18 and 25 years of age \cite{Akay_2015f, Ali_2017, Atayan_2016, Bhattacharjee_2008, Csaky_1998e, Kar_2015, Kariholu_2008, Kobiela_2015, Losanoff_1996, Losanoff_1997e, Mesfin_2022a, Peixoto_2017f, Sobnach_2011f, Tupesis_2004f, Yasin_2009}, 13 cases (18\%) were under 18 years of age \cite{AlShaaibi_2021b, Ali_2020f, Cauchi_2002, DivsalarP._2023a, Goldman_1998f, Liu_2005, Naji_2012f, Ohno_2005, Tanrikulu_2015e, Tay_2004, Wildhaber_2005}, 11 cases (15\%) were between 41 and 60 years of age \cite{Al-Faham_2020k, Bhumi_2024f, CamachoDorado_2018, Emamhadi_2018, Hardy_2023g, Jehangir_2019h, Kumar_2001, Sultan_2024f, Thapa_2019f, Wadhwa_2015e, teWildt_2010}, 3 cases (4\%) were over 60 years of age \cite{Beecroft_1998, Kerestes_2019, Li_2013}, 2 cases (3\%) had no age documented \cite{Berry_2021e}. All 90 were male gender. 90 cases (100\%) were detained at the time of ingestion \cite{Elghali_2016, Karp_1991b, Lee_2007}, 88 cases (98\%) were intentional ingestions \cite{Elghali_2016, Karp_1991b, Lee_2007}, 30 cases (33\%) had a psychiatric history documented \cite{Elghali_2016, Karp_1991b, Lee_2007}, 2 cases (2\%) had a history of prior ingestion \cite{Elghali_2016}. No cases were reported for were psychiatric inpatients, were displaced people, were under the influence of alcohol at the time of ingestion, and had a severe disability history.
\paragraph*{Motivation}  70 cases (78\%) reported protest motivation \cite{Elghali_2016, Karp_1991b, Lee_2007}, 12 cases (13\%) reported psychiatric motivation \cite{Karp_1991b}, 6 cases (7\%) reported self-harm motivation \cite{Elghali_2016, Karp_1991b}. No cases were reported for psychosocial motivation and other motivation.
\paragraph*{Object Characteristics}  68 cases (76\%) involved sharp object ingestion \cite{Elghali_2016, Karp_1991b, Lee_2007}, 32 cases (36\%) involved long (\textgreater 5cm) object ingestion \cite{Lee_2007}, 25 cases (28\%) involved ingestion of multiple objects \cite{Elghali_2016, Lee_2007}. No cases were reported for button battery ingestion, magnet ingestion, and involved large diameter (\textgreater 2.5cm) object ingestion.
\paragraph*{Outcomes}  47 cases (52\%) underwent endoscopic intervention \cite{Elghali_2016, Lee_2007}, 29 cases (32\%) were managed conservatively \cite{Elghali_2016, Karp_1991b}, 15 cases (17\%) underwent surgical intervention \cite{Elghali_2016, Karp_1991b, Lee_2007}, 6 cases (7\%) reported complications \cite{Lee_2007}, 1 case (1\%) died \cite{Elghali_2016}.

\section{Results of Synthesis Expanded}
\label{appendix:synthesis_results}
Using the DerSimonian & Laird (DL) method, meta-analysis of proportions was undertaken on endoscopy, surgery, death, complication, and conservative. The pooled proportion of patients that endoscopy was 50.4\% (95\% CI 26.9\%--73.6\%), with substantial heterogeneity ($I^2$ = 90.5\%). The pooled proportion of patients that surgery was 30.5\% (95\% CI 16.5\%--49.4\%), with substantial heterogeneity ($I^2$ = 87.2\%). The pooled proportion of patients that death was 2.5\% (95\% CI 1.1\%--5.5\%), with low heterogeneity ($I^2$ <0.5\%). The pooled proportion of patients that complication was 34.6\% (95\% CI 13.0\%--65.2\%), with substantial heterogeneity ($I^2$ = 94.5\%). The pooled proportion of patients that conservative management was 41.5\% (95\% CI 14.4\%--75.0\%), with substantial heterogeneity ($I^2$ = 93.1\%).
\begin{figure}[t]
\centering
\includegraphics[width=\columnwidth]{figures/reml_meta_summary_plot.png}
\caption{Meta-analysis of proportions using random-effects models with restricted maximum likelihood (REML) estimation and Hartung-Knapp (HK) adjustments for confidence intervals.}
\label{fig:reml_meta_plot_appendix}
\end{figure}

\begin{figure}[t]
\centering
\includegraphics[width=\columnwidth]{figures/dl_meta_summary_plot.png}
\caption{Meta-analysis of proportions using the DerSimion \& Laird Method.}
\label{fig:dl_meta_plot_appendix}
\end{figure}


\section{Computational Risk of Bias Assessment}
\subsubsection*{Case Reports}
For case reports, the JBI Checklist for Case Reports was used. This tool assesses eight domains of reporting quality, including whether patient demographics were clearly described, a timeline of clinical history was provided, the presenting condition and diagnostic assessment were outlined, and whether the intervention, post-intervention condition, and any adverse events were reported. The final domain evaluates whether the case provides meaningful takeaway lessons.

In addition to manual JBI appraisal, a logic-based validation filter was applied to all case reports using \textit{Python Pandas}~\cite{ThePandasDevelopmentTeam_2020}. This secondary filter assessed whether key variables --- specifically, outcomes, object characteristics, and motivation --- were completely unreported. For each domain, a binary flag was generated:

\begin{itemize}
\item \textit{Outcome\_Unknown} was marked \texttt{1} if all outcome-related fields were either missing or marked as unknown.
\item \textit{Object\_Unknown} was marked \texttt{1} if all object-related fields (excluding \textit{Object\_Other\_Long}) were missing or unknown.
\item \textit{Motivation\_Unknown} was predefined in the dataset and indicated absence of motivational information.
\end{itemize}

If any of these flags were triggered, the corresponding JBI item most affected by the missing domain was marked as not reported (e.g., \textit{Post\_Intervention\_Condition\_Described} or \textit{History\_Timeline} set to \texttt{N}). Finally, an \textit{Overall\_Appraisal} score of \textit{Exclude} was assigned, indicating high risk of bias and exclusion from analysis. This ensured that only case reports with sufficient information to meaningfully contribute to the review question were retained.\\

\subsubsection*{Case Series}
For case series, the JBI Checklist for Case Series was applied. The JBI Checklist for Case Series assesses 10 domains of methodological and reporting quality. These include whether the case series defined clear inclusion criteria, applied valid and consistent methods to identify the condition, and included participants consecutively and completely. The checklist also evaluates whether participant demographics and clinical information were clearly reported, whether outcomes or follow-up results were adequately described, and whether the study setting was detailed. Finally, it considers whether the statistical analysis used was appropriate for the data presented.

In addition to manual JBI appraisal, a logic-based exclusion filter was applied using \textit{Python Pandas}~\cite{ThePandasDevelopmentTeam_2020}. This filter assessed whether key variables --- specifically, motivation, object characteristics, and outcomes --- were unreported for the entire study population. For each of these domains, a derived rate variable was calculated:

\begin{itemize}
\item \textit{Outcome\_Unknown\_Rate} was marked as \texttt{1} if all outcome-related fields were missing or marked as unknown (i.e. the entire population had an had an unknown outcome).
\item \textit{Motivation\_Unknown\_Rate} indicated whether motivation was absent or only partially reported across cases within the study.
\item \textit{Object\_Unknown\_Rate} was derived if all object-related fields were missing or unknown.
\end{itemize}

If any of these indicators were flagged, the corresponding JBI checklist item (e.g., \textit{Clear\_Outcome\_Followup\_Reported}, \textit{Clear\_Demographic\_Reporting}, or \textit{Clear\_Clinical\_Info\_Reporting}) was marked as \texttt{N}, and the study received an \textit{Overall\_Appraisal} of \texttt{Exclude}. This logic-based validation ensured that case series lacking essential variables could be systematically excluded from the final analysis, maintaining consistency with the review question and minimising risk of bias in the dataset.\\


\end{document}