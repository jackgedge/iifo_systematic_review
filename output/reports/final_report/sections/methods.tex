This study was conducted according to the Preferred Reporting Items for Systematic Reviews and Meta-Analyses (PRISMA) guidelines \cite{Page_2021}. Ethical approval was not required as all analysis was based on published data. Eligibility criteria were structured using the PICOS (Population, Intervention, Comparator, Outcome, Studies) framework.

\subsection{Eligibility Criteria}

\begin{table}[H]
\renewcommand{\arraystretch}{1.05}
\centering
\begin{tabular}{p{3.0cm} p{5.0cm}}
\toprule
\textbf{Category} & \textbf{Details} \\
\midrule

Population & Any human; any age group. \\
\midrule

Interventions or exposures & Non-accidental ingestion of a true foreign body (non-nutritive items). \\
\midrule

Comparators / Control group & 
Demographics: \newline
Gender, age, detained person, psychiatric inpatient, displaced person, under influence of alcohol, psychiatric history, severely disabled, previous ingestion. \newline
Motivation: \newline
Intent to harm, psychiatric, psychosocial, protest, other. \newline
Object characteristics: \newline
Button battery, magnet, long ($>$5 cm), large diameter ($>$2.5 cm), multiple, blunt objects, sharp-pointed objects. \\
\midrule

Outcomes of interest & Endoscopic intervention, surgical intervention, conservative management, complication rates, mortality. \\
\midrule

Setting & Any setting. \\
\midrule

Study designs & Any design. \\
\bottomrule
\end{tabular}
\caption{Inclusion criteria structured using the PICOS framework.}
\label{tab:inclusion-criteria-intro}
\end{table}

A full list of eligibility criteria is shown in Table~\ref{tab:inclusion-criteria-intro}. This is reproduced in a larger format for clarity in Appendix~\ref{tab:inclusion-criteria}. A full list of exclusion criteria is available in Appendix~\ref{tab:exclusion-criteria} and in the PRISMA diagram shown in Figure~\ref{fig:prisma}.

\subsection{Information Sources}

Relevant articles were identified through a systematic search of PubMed, Web of Science, Embase, Scopus, PsycINFO, CENTRAL and Google Scholar during January 2025, with the assistance of a librarian. Included articles then had their bibliography's searched by the primary author (JGE) on 14th May 2025 to identify any potentional additional literature not uncovered in the primary search. The search was conducted using keywords and MeSH terms based on the concepts underpinning this review. The search queries, keywords and MeSH terms used can be found in Appendix~\ref{appendix:search-strategy}. 

All identified articles were collated and duplicate articles were identified and removed. Remaining articles underwent independent title and abstract screening conducted by the first author (JGE). A randomly selected 10\% sample of these articles underwent independent screening by a second author (MS). Any discrepancies identified between these two reviewers were resolved by a third reviewer (GC). Inter-reviewer agreement was calculated using Cohen's Kappa \cite{Machin_2008}. Remaining articles proceeded to full text review, where the same independent screening process was repeated on full text articles.

\subsection{Data Extraction}

Data were initially extracted by a single reviewer (JGE) into \textit{Microsoft Excel} \cite{MicrosoftCorporation_2025} and processed in \textit{Python}~\cite{PythonSoftwareFoundation_2025} using \textit{Pandas}~\cite{ThePandasDevelopmentTeam_2020}. This process is outlined in Appendix~\ref{appendix:data_extraction}. 

Data was first extracted from case reports. Predictors were grouped into four subgroups: gender; age group; demographic characteristics, motivation; object characteristics. These are shortened hereafter to: gender, age group, demographic and motivation. Outcome data were extracted for rates of endoscopy, surgery, conservative management, mortality, and complications. All outcomes were binary and coded per event, rather than per individual. Predictor variables and outcome variables were not mutually exclusive, nor were outcomes. For example, patients, or ingestors -- hereafter referred to as the latter -- could have multiple outcomes (e.g. endoscopy and surgery) and multiple predictors from each group (e.g. intent-to-harm and other, and detained and displaced person).

After case report data extraction, data was collapsed and aggregated to form a \enquote{series}. This data was used as a template for case series data extraction to homogenise data and reduce heterogeneity. 

Full definitions of all variables (predictors and outcomes) are provided in Appendix~\ref{appendix:data_extraction}. The full dataset of extracted \href{https://github.com/jackgedge/iifo_systematic_review/blob/main/input/processed_data/final_report/case_data_export.csv}{case-level} and \href{https://github.com/jackgedge/iifo_systematic_review/blob/main/input/processed_data/final_report/aggregate_series_data_export.csv}{series-level} data (including bias assesments), is available on \href{https://github.com/jackgedge/iifo_systematic_review/blob/main/input/processed_data/final_report/}{Github}.

\subsection{Risk of Bias Assessment}

Risk of bias was assessed manually for all included studies by a single reviewer (JGE), using the \textit{Joanna Briggs Institute (JBI) Critical Appraisal Checklists for Case Reports and Case Series}~\cite{Moola_2020}. Studies were first classified as either case reports or case series based on the level of granularity in the data. Each study was then evaluated using the corresponding JBI tool. A novel computational risk of bias filter was then applied in \textit{Pandas} \cite{ThePandasDevelopmentTeam_2020}. That process is outlined in Appendix~\ref{appendix:bias}

\subsection{Synthesis Methods}

Associations between binary predictors and outcomes were assessed using univariate logistic regression, reporting odds ratios (ORs), 95\% confidence intervals, and p-values. Where appropriate, chi-squared ($\chi^2$) tests were also used to evaluate differences in outcome proportions between groups \cite{Machin_2008}. 

Multivariable logistic regression was not performed due to the limited sample size, high collinearity between predictors (e.g., overlapping motivations), and the exploratory nature of the analysis. 

For series-level data, univariate meta-regression will be conducted to assess associations between binary series-level predictors and pooled outcome proportions, where sufficient data are available. Each predictor will be entered separately to account for incomplete reporting across studies and to avoid overfitting. Effect estimates will be reported as odds ratios (ORs) with 95\% confidence intervals, using a random-effects model with restricted maximum likelihood (REML) estimation \cite{Tanriver-Ayder_2021}.

Due to the inclusion of primarily case reports and small case series, formal assessment of reporting bias (e.g., via funnel plots or statistical tests for asymmetry) was not feasible. 

Confidence in the body of evidence was not formally graded but was considered low to very low due to reliance on uncontrolled observational designs, small sample sizes, and incomplete reporting.

