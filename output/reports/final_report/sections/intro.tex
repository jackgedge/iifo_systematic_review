\section{Introduction}

\subsection{Rationale}
As of May 2024, over 100 million individuals were forcibly displaced worldwide \cite{UNHCR_2022}. Refugees and asylum seekers often endure extreme hardship—including violence, trauma, and detention—leading to elevated rates of mental health disorders \cite{UNHCR_2010, AmnestyInternational_2024, Athwal_2015, Sundvall_2015, Nickerson_2019, Bevione_2024}. Among the most alarming manifestations is self-harm, which is up to 216 times more common in offshore detention settings than in the general population \cite{vonWerthern_2018, Hedrick_2019, GlobalDetentionProject_2024}. Methods vary and include cutting, poisoning, hanging, self-immolation, and intentional ingestion of foreign objects (IIFO) \cite{Davidson_2016, Hedrick_2019, Ajdacic-Gross_2008}.

IIFO—the non-accidental ingestion of non-nutritive items—is a serious clinical issue, with 10–20\% of cases requiring endoscopy and up to 1\% needing surgery \cite{Becq_2021, Ikenberry_2011, Birk_2016}. In displaced populations, delayed access to care increases risks \cite{Vujkovic_2019}. Rates of IIFO are rising globally; in the U.S., cases doubled in 2017, and intentional ingestion is common in lower socioeconomic groups \cite{Hsieh_2020, Palta_2009}. While techniques for foreign body removal have advanced—from early gastrotomy to modern endoscopy—clinical outcomes depend on multiple factors including object characteristics, patient co-morbidities, and timeliness of intervention \cite{Moehlau_1895, Saint_1929, Barros_1991, Lerche_1911, Jackson_1957, Ricote_1985, Chalk_1928, Ikenberry_2011}.

IIFO motivations vary widely. In detention, it may be a form of protest or communication \cite{Puggioni_2014}; in psychiatric contexts, it may stem from conditions like psychosis, personality disorders, pica, or malingering \cite{Gitlin_2007, Poynter_2011, Tromans_2019, Al-Faham_2020b, Pantazopoulos_2022, Aitchison_2024, Losanoff_1997e}. In borderline personality disorder, it may function as emotional regulation rather than a suicide attempt \cite{Gitlin_2007}. Rare cases of repeated IIFO have prompted palliative approaches to care \cite{Jaini_2023}.

Despite its growing prevalence, little research explores how these differing motivations affect clinical management and outcomes \cite{Bhugra_2010, Haase_2022, Hedrick_2023a}. Understanding motivation is crucial, as it may influence decisions around intervention. For example, ingestion as protest may be less likely to involve high-risk behaviours, supporting conservative treatment \cite{Ataya_2013, Albeldawi_2014}.

This review aims to examine how motivation shapes IIFO outcomes—specifically rates of endoscopic and surgical intervention, conservative management, complications, and mortality—to better guide care in vulnerable populations.


\subsection{Objectives}
The primary object of this systematic review was to quantify the rates of endoscopy, surgery, death, complication and conservative management following intentional ingestion of foreign objects in human populations. The review sought to examine how individual factors such as demographic/population characteristics, object characteristics and motivations for ingestion influence the likelihood of these outcomes.
