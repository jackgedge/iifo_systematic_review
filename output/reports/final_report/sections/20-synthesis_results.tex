\subsection*{Synthesis}

Across 72 single‐patient case reports and three case series (total $n=162$) we extracted uniform data on patient motive, object features, treatment approach and outcome.

Random–effects pooling of the three case series yielded a conservative–management rate of 40.6 \% (95 \% CI 29.2–52.8; $I^{2}\!\approx\!75$), an endoscopy rate of 41.6 \% (29.5–54.7; $I^{2}\!\approx\!95$) and a surgical rate of 17.8 \% (12.0–25.8; $I^{2}\!\approx\!17$). Overall complications occurred in 7.3 \% of patients (2.9–17.3) and there were three reported deaths.

\paragraph*{Motivation}
Series-level meta-regression confirmed motivation as the dominant predictor of management.  
\begin{itemize}
    \item \textbf{Intentional self-harm} lowered the probability of conservative care by 17 \% (aOR 0.83, 95 \% CI 0.79–0.86, $p=0.012$) and raised surgical use by 11 \% (aOR 1.11, 1.05–1.18, $p=0.015$); endoscopy was unaffected (aOR 0.92, 0.27–3.13, $p=0.559$). Case-level logistic modelling amplified this signal, showing a 15-fold increase in surgery (aOR 15.97, 1.32–192.40, $p=0.029$).  
    \item \textbf{Psychiatric (non-self-harm)} motivation produced a smaller but significant rise in surgical intervention (aOR 1.07, 1.03–1.11, $p=0.017$) with no change in other pathways; the effect disappeared in multivariable analyses.  
    \item \textbf{Psychosocial} motives behaved similarly, decreasing conservative management (aOR 0.82, 0.70–0.96) and increasing surgery (aOR 1.12, 1.01–1.25), both $p=0.040$.  
    \item \textbf{Protest} ingestions—almost exclusively in custodial settings—did not alter the odds of any single treatment modality in the series-level model, though case-level analysis linked protest with an extreme rise in complications (aOR 290.44, 1.40–60 043.40).
\end{itemize}

\paragraph*{Object}
\begin{itemize}
    \item \textbf{Sharp objects} significantly reduced conservative management by 19 \% (aOR 0.81, 0.68–0.98, $p=0.044$) but had no independent effect on surgery or endoscopy; numbers were too small for stable multivariable estimates.  
    \item \textbf{Multiple objects} and \textbf{length $>$ 5 cm} were neutral in univariate meta-regression, yet individual-patient models linked long objects to a seven-fold rise in surgery and a twenty-fold rise in complications, while multiple-object ingestion drove an eighteen-fold increase in complications.
\end{itemize}

\paragraph*{Demographics and setting}
Male sex curtailed conservative treatment by 13 \% (OR 0.87, 0.78–0.97, $p=0.039$) without influencing other modalities. Detention status increased conservative management by 62 \% (OR 1.62, 1.11–2.38, $p=0.039$) and, in case-level analysis, was associated with a 97 \% reduction in recorded complications (aOR 0.03, 0.00–0.59). A binary history of psychiatric illness showed no significant association with any outcome in series-level modelling.

\paragraph*{Certainty and heterogeneity}
All included studies were uncontrolled case reports or case series. High between-study heterogeneity for several endpoints ($I^{2}$ up to 95 \%), shifting definitions of motive and object descriptors, and limited follow-up constrain the certainty of pooled estimates. No GRADE assessment was performed.

\paragraph*{Data availability}
All extraction scripts and raw data are openly hosted at \url{https://github.com/jackgedge/iifo_systematic_review}.\paragraph*{Investigation of Heterogeneity}

Heterogeneity was explored across study results using two complementary approaches.

Firstly, meta-analyses of proportions were conducted for each outcome of interest across included case series. Between-study heterogeneity was quantified using the $I^2$ statistic and $\tau^2$ variance. The degree of heterogeneity varied substantially across outcomes:
\begin{itemize}
\item Substantial heterogeneity ($I^2 = 95\%$) was observed for endoscopy.
\item Heterogeneity was moderate for surgery ($I^2 = 17\%$).
\item Heterogeneity was low ($I^2 = 0\%$) for conservative management.
\item Meta-analyses could not be meaningfully conducted for death and complications due to insufficient data.
\end{itemize}

Secondly, to investigate potential sources of heterogeneity, univariate meta-regression was performed for each outcome. Given the small number of case series available, series-level data was combined with collapsed aggregate case reports to increase the number of contributing studies. Meta-regressions examined associations between outcome proportions and patient-level predictors including gender, population characteristics, ingestion motivations, and object type.

Significant associations between certain predictors and outcomes were identified, particularly in relation to motivation and surgery or conservative management outcomes. This suggests that clinical heterogeneity in patient characteristics likely contributed to observed between-study differences.

No additional subgroup or sensitivity analyses were performed, given the limited number of studies and inconsistent reporting of potential study-level moderators. \paragraph*{Sensitivity Analyses}

Formal sensitivity analyses were not performed due to the limited number of included case series and the small number of studies available for several outcomes. Given the small number of eligible series and heterogeneity in reporting, the meta-analyses of proportions and meta-regressions were considered primarily descriptive in nature.

To partially address robustness, aggregate case report data was incorporated into the meta-regression models to increase the effective number of series contributing to each outcome. This allowed exploratory assessment of predictor-outcome relationships across a larger pooled dataset. However, no additional sensitivity analyses (e.g. leave-one-out analyses, exclusion of small studies, or alternative meta-analytic models) were conducted, as such analyses would not have been statistically meaningful in the context of the available data. 