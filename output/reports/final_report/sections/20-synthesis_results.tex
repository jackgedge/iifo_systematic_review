\subsection*{Synthesis}\subsubsection*{Univariate Association Testing} 
Overall, sample size restricted case report analysis. A full table of case-level univariate association testing results for is available in Table~\ref{tab:univariate_results}.
In the age-based subgroup analysis, being between 41 and 60 years of age was significantly associated with reduced odds of undergoing surgery (OR = 0.18, 95\% CI [0.04, 0.76], \emph{p} = 0.019). Further comparison of this subgroup (\emph{n} = 11) against the remaining case reports (\emph{n} = 61) revealed several distinguishing characteristics. 

Notably, there was a marked increase in the proportion of male patients (over 20\%). Motivations were more frequently psychiatric, and there was a higher incidence of previous ingestions. Patients in this age group were more likely to be psychiatric inpatients and to have a documented psychiatric history. They more commonly ingested multiple objects, while the objects were less frequently sharp or of large diameter. Additionally, this subgroup included fewer detained individuals. 
In the motivation subgroup, intent to harm was significantly associated with increased odds of undergoing surgery (OR = 5.40, 95\% CI [1.36, 21.43], p = 0.024); another documented motivation was significantly associated with reduced odds of undergoing surgery (OR = 0.14, 95\% CI [0.03, 0.75], p = 0.023).

Further descriptive analysis of this subgroup (\emph{n}=21 vs \emph{n}=51) revealed significantly higher rates of ingestions involving large-diameter and sharp objects. These object characteristics are widely recognised as high-risk and may partially explain the increased likelihood of surgical intervention observed in this group.
Deeper analysis of the “other” motivation subgroup (\emph{n} = 9 vs \emph{n} = 63) revealed several notable differences. This group had significantly fewer cases involving intent to harm, protest, or psychosocial motivations—nearly 20\% lower in each category. In contrast, there was a higher proportion of females, individuals with a history of severe disability, and those with prior ingestions.

These findings suggest that the absence of an explicit intention to self-harm may contribute to the reduced odds of surgery observed in this subgroup. Conversely, the presence of repeat ingestion—a known clinical red flag—may signal elevated risk, potentially offsetting this protective effect in some contexts.
\subsubsection*{Meta-analysis of Proportions} 
Given that small sample size severely effected the case-level analysis, for series-level analyses, to increase numbers, case reports were flattened into a series and presented alongside other case series. This is demonstrated in Table~\ref{tab:series_results}. 

Meta-analyses of proportions were performed using random-effects models with restricted maximum likelihood (REML) estimation and Hartung-Knapp (HK) adjustments for confidence intervals, appropriate for small numbers of studies \cite{Tanriver-Ayder_2021}. In general, REML yielded wider confidence intervals and lower heterogeneity estimates than the DerSimonian & Laird (DL) method, reflecting greater uncertainty in pooled estimates from small samples. 

A plot of aggregate series-level meta-analysis of pooled outcome proportions is shown in Figure~\ref{fig:reml_meta}. For comparison, DL estimates were also calculated, yielding narrower confidence intervals and higher I\textsuperscript{2} values. These are included in Table~\ref{fig:reml_meta_plot_appendix} and \ref{fig:dl_meta_plot_appendix} of Appendix~\ref{appendix:synthesis_results}.

Using REML with HK adjusted confidence intervals, meta-analysis of proportions was undertaken on endoscopy, surgery, death, complication, and conservative. The pooled proportion of patients that endoscopy was 47.3\% (95\% CI 4.3\%--94.7\%), with moderate heterogeneity ($I^2$ = 34.1\%). The pooled proportion of patients that surgery was 30.6\% (95\% CI 12.0\%--58.9\%), with low heterogeneity ($I^2$ = 15.9\%). The pooled proportion of patients that death was 2.5\% (95\% CI 0.8\%--7.7\%), with low heterogeneity ($I^2$ <0.5\%). The pooled proportion of patients that complication was 34.7\% (95\% CI 1.8\%--93.8\%), with moderate heterogeneity ($I^2$ = 35.2\%). The pooled proportion of patients that conservative management was 41.6\% (95\% CI 4.4\%--91.7\%), with moderate heterogeneity ($I^2$ = 25.7\%).\subsubsection*{Meta Regression} A full table of grouped aggregate series-level results for univariate meta-regression is available in Table~\ref{tab:univariate_meta_regression}. In the gender subgroup, there were no significant associations with outcomes. In the demographic subgroup, severe disability was associated with a reduced likelihood of death (OR = 0.59, 95\% CI [0.41, 0.84], p = 0.034); displacement status was associated with a reduced likelihood of death (OR = 0.22, 95\% CI [0.08, 0.61], p = 0.034); alcohol influence was associated with a reduced likelihood of death (OR = 0.32, 95\% CI [0.15, 0.69], p = 0.034). All other comparisons in this subgroup were non-significant. In the motivation subgroup, there were no significant associations with outcomes. In the object subgroup, multiple was associated with a reduced likelihood of death (OR = 0.98, 95\% CI [0.98, 0.99], p = 0.014); large diameter was associated with a reduced likelihood of death (OR = 0.93, 95\% CI [0.88, 0.98], p = 0.034). All other comparisons in this subgroup were non-significant. 