
\begin{table}[htbp]
\small
\centering
\caption{Summary of Case Series Data}
\label{tab:series_summary}
\renewcommand{\arraystretch}{1.2}
\begin{tabularx}{\textwidth}{|X|c|c|c|}
\hline
\textbf{Variable} & \textbf{Study 1} & \textbf{Study 2} & \textbf{Study 3} \\ \hline
\textbf{Paper Info} &  &  &  \\
Publication Year & 1991 & 2007 & 2016 \\
Authors & Karp, J. G.; Whitman, L.; Convit, A. & Lee, Tae Hee; Kang, Young Woo; Kim, Hyun Jin; Kim, Sun Moon; Im, Euyi Heog; Huh, Kyu Chan; Choi, Young Woo; Kim, Tae Hyo; Lee, Ok Jae; Jung, Un Tae & Elghali, Mohamed amine; Ghrissi, Rafik; Fadhl, Houssem; Mahjoub, Mohamed; Jarrar, Mohamed Salah; Jedidi, Maher; Letaief, Rached; Hamila, Fehmi \\
Title & Intentional ingestion of foreign objects by male prison inmates & Foreign Objects in Korean Prisoners & The Management of Voluntary Ingestion of Razor Blades by Inmates \\
Publication Title & Hospital & Community Psychiatry & The Korean Journal of Internal Medicine & International Surgery \\
Doi & 10.1176/ps.42.5.533 & 10.3904/kjim.2007.22.4.275 & 10.9738/INTSURG-D-16-00204.1 \\
Study Country & USA & Republic of Korea & Tunisia \\
Study Design & Case Series & Case Series & Case Series \\
Citekey & Karp\_1991b & Lee\_2007 & Elghali\_2016 \\
Setting Population & Medical prison ward for men in a New York City public general hospital, New York, USA & Prisoners attended two hospitals in South Korea. & Patients referred from prison to Department of General and Digestive Surgery, Farhat Hached University Hospital of Sousse, Sousse, Tunisia \\
Study Objective & Records were examined for demographic and psychiatric characteristics, as well as for the cir cumstances surrounding the ingestion of a foreign object. & To investigate the time interval between foreign object ingestion and the visit to the ER, the locations, kinds and sizes of foreign objects, the methods of foreign object removal, complications of endoscopic management and the combined disorders of the prisoners. & A descriptive study including all detainees ingesting a razor blade, transferred from the prison to Farhat Hached University Hospital of Sousse, from January 1, 2014 to December 31, 2015. \\
Study Method & Two psychiatrically trained raters examined the records of all patients (N= 19) admitted to a medical prison ward for men in a New York City public general hospital for deliberately swallowing objects between September 1, 1985, and October 15, 1988. & Reviewed the medical records and endoscopic findings of 33 prisoners (52 episodes) who were admitted due to ingestion of foreign objects  between January 1998 and June 2004 to either Konyang University Hospital or Gyeongsang National University Hospital. & - \\
Study Results & See statistical breakdown. & All the patients were male with a mean age of 35 years. The most common duration from ingestion to the visit to the ER was within 24 hours (25/52 episodes). Most of the foreign objects were located in the esophagus (42.3\%) and stomach (42.3\%). The number of foreign objects was one in 28 episodes, two in 12 episodes and three or more in twelve episodes. The most common foreign objects were metal wires (26/52 episodes). The mean size of the foreign objects was 11.9 centimeters long. Successful endoscopic treatment was performed in most patients (46/52 episodes, 88.5\%). The remaining six cases were treated surgically. & There were 16 men with a mean age of 24 years, ranging from 19 to 27 years. Three patients had a history of self-harm; one of them was having a psychiatric follow-up for depressive disorders. An inmate had ingested a half blade 3 times and another had ingested a half blade 2 times, so we had 19 swallowed razor blade episodes. This act was a form of protest in 17 cases and a suicide attempt in 2 cases. \\
Study Conclusion & In our series of 19 prisoners who ingested objects, 12 prisoners (64 percent) were judged to be suicidal by the psychiatric consultant. In addition, the high reported incidence of past suicide attempts by other methods is startling. In our sample, a remarkable 84 percent descnibed a history ofsuicide attempts, and most of these attempts occurred before imprisonment. Also common in this sample was psychosis. About three-quarters of the patients were judged by the psy chiatnic consultant to have bad corn mand hallucinations, suicidal idea tion, or both. In addition, the swat lowing ofantennae and toothbrushes can be construed as psychotic behav ion. Psychosis may have occurred in the setting of schizophrenia; almost a third of the sample had that diagnosis. No patient swallowed a foreign body before his first impnisonment. Desire to leave prison may explain this behavior, as may suicidal idea tion in response to incarceration. Sm cide pacts or copycat phenomena may contribute to foreign body in gestion in prison; two ofthe patients had shared a razor blade, each swal lowing half. All 19 patients ingested sharp or pointed objects, possibly be cause swallowing such objects is like ly to result in transfer from the prison to the hospital. & The foreign objects in prisoners were a variety of unusual things because of the prison environment, and endoscopy is a mainstay of treatment for foreign object removal in Korean prisoners. & Our experience enabled us to confirm the few data in the literature that surgical removal of intragastrointestinal sharp foreign bodies should not be systematic. \\
\textbf{Object Summary} &  &  &  \\
Object Count & - & - & - \\
Object Button Battery Cases & - & 0.0 & 0.0 \\
Object Button Battery Rate & - & 0.0 & 0.0 \\
Object Magnet Cases & - & 0.0 & 0.0 \\
Object Magnet Rate & - & 0.0 & 0.0 \\
Object Long Cases & - & 32.0 & 0.0 \\
Object Long Rate & - & 1.0 & 0.0 \\
Object Diameter Large Cases & - & - & - \\
Object Diameter Large Rate & - & - & - \\
Object Sharp Cases & 19.0 & 33.0 & 16.0 \\
Object Sharp Rate & 1.0 & 1.0 & 1.0 \\
Object Multiple Cases & - & 24.0 & 1.0 \\
Object Multiple Rate & - & 0.7 & 0.1 \\
Object Long Sharp Cases & - & - & 0.0 \\
Object Long Sharp Rate & - & - & 0.0 \\
Object Short Cases & - & 1.0 & 16.0 \\
Object Short Rate & - & 0.0 & 1.0 \\
Object Short Sharp Cases & - & - & 16.0 \\
Object Short Sharp Rate & - & - & 1.0 \\
Object Unknown Rate & 0 & 0 & 0 \\
\textbf{Outcome Summary} &  &  &  \\
Outcome Endoscopy Cases & - & 46.0 & 1.0 \\
Outcome Endoscopy Rate & - & 0.9 & 0.1 \\
Outcome Surgery Cases & 5.0 & 6.0 & 4.0 \\
Outcome Surgery Rate & 0.3 & 0.1 & 0.2 \\
Outcome Death Cases & 0.0 & 0.0 & 1.0 \\
Outcome Death Rate & 0.0 & 0.0 & 0.1 \\
Outcome Injury Needing Intervention Cases & - & 6.0 & - \\
Outcome Injury Needing Intervention Rate & - & 0.1 & - \\
Outcome Perforation Cases & - & 1.0 & 1.0 \\
Outcome Perforation Rate & - & 0.0 & 0.1 \\
Outcome Obstruction Cases & - & - & 1.0 \\
Outcome Obstruction Rate & - & - & 0.1 \\
Outcome Other Cases & - & 5.0 & 2.0 \\
Outcome Other Rate & - & 0.1 & 0.1 \\
Outcome Conservative Cases & 14.0 & 0.0 & 15.0 \\
Outcome Conservative Rate & 0.7 & 0.0 & 0.8 \\
Outcome Endoscopy Surgery Cases & - & 1.0 & 1.0 \\
Outcome Endoscopy Surgery Rate & - & 0.0 & 0.1 \\
Outcome Unknown Rate & 0 & 0 & 0 \\
\textbf{Motivation Summary} &  &  &  \\
Motivation Intent To Harm Cases & 4.0 & 0.0 & 2.0 \\
Motivation Intent To Harm Rate & 0.2 & 0.0 & 0.1 \\
Motivation Protest Cases & 3.0 & 50.4 & 17.0 \\
Motivation Protest Rate & 0.2 & 1.0 & 0.9 \\
Motivation Psychiatric Cases & 12.0 & 0.0 & 0.0 \\
Motivation Psychiatric Rate & 0.6 & 0.0 & 0.0 \\
Motivation Psychosocial Cases & 0.0 & 0.0 & 0.0 \\
Motivation Psychosocial Rate & 0.0 & 0.0 & 0.0 \\
Motivation Unknown Cases & 0.0 & 0.0 & 0.0 \\
Motivation Unknown Rate & 0.0 & 0.0 & 0.0 \\
Motivation Other Cases & 0.0 & 0.0 & 0.0 \\
Motivation Other Rate & 0.0 & 0.0 & 0.0 \\
Motivation Other Psych Hx Cases & 0.0 & 0.0 & 0.0 \\
Motivation Other Psych Hx Rate & 0.0 & 0.0 & 0.0 \\
Motivation Other Severe Disability Hx Cases & 0.0 & 0.0 & 0.0 \\
Motivation Other Severe Disability Hx Rate & 0.0 & 0.0 & 0.0 \\
\textbf{Demographic/Population Summary} &  &  &  \\
Age Low & 17.0 & 25.0 & 19.0 \\
Age High & 40.0 & 50.0 & 27.0 \\
Age Mean & 24.0 & - & 24.0 \\
Age Median & - & 35.0 & - \\
Gender Male Cases & 19.0 & - & - \\
Gender Male Rate & 1.0 & 1.0 & 1.0 \\
Gender Female Cases & 0.0 & - & 0.0 \\
Gender Female Rate & 0.0 & 0.0 & 0.0 \\
Gender Unknown Cases & 0.0 & 0.0 & 0.0 \\
Gender Unknown Rate & 0.0 & 0.0 & 0.0 \\
Is Intentional Cases & - & - & - \\
Is Intentional Rate & 1.0 & 1.0 & 1.0 \\
Is Prisoner Cases & - & - & - \\
Is Prisoner Rate & 1.0 & 1.0 & 1.0 \\
Is Psych Inpat Cases & 0.0 & 0.0 & 0.0 \\
Is Psych Inpat Rate & 0.0 & 0.0 & 0.0 \\
Is Displaced Person Cases & - & - & - \\
Is Displaced Person Rate & - & - & - \\
Under Influence Alcohol Cases & - & - & - \\
Under Influence Alcohol Rate & - & - & - \\
Psych Hx Cases & 18.0 & - & - \\
Psych Hx Rate & 0.9 & 0.2 & 0.1 \\
Severe Disability Hx Cases & - & 0.0 & - \\
Severe Disability Hx Rate & - & 0.0 & - \\
Previous Ingestions Cases & - & - & 2.0 \\
Previous Ingestions Rate & - & - & - \\
\hline
\end{tabularx}
\end{table}
