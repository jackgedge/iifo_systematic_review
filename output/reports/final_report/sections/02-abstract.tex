\section{Abstract}

\subsection*{Background}
Intentional ingestion of foreign objects (IIFO) is clinically distinct from accidental ingestion, yet its outcome profile and drivers of morbidity remain poorly defined. Whether motivation itself alters risk has never been examined systematically.

\subsection*{Objectives}
To synthesise all available evidence on outcomes after intentional foreign-object ingestion, and to determine whether patient motivation, object characteristics, or demographic factors modify those outcomes.

\subsection*{Methods}
A comprehensive search of PubMed, Embase, CENTRAL, Web of Science, Scopus, PsycINFO, and Google Scholar (1906--31 March 2025) combined free-text and controlled-vocabulary terms for foreign bodies, intentional ingestion, and clinical interventions.

\paragraph*{Inclusion} human studies of any design, any age, reporting non-accidental ingestion of true foreign bodies.  
\paragraph*{Exclusion} accidental or substance ingestion, animal studies, non-English full texts, pre-1906 reports, and studies lacking motivation, object, or outcome data.  
\paragraph*{Outcomes} endoscopy, surgery, conservative management, complications, mortality.  

Case reports (\emph{n} = 72; 72 patients) and case series (\emph{n} = 3; 90 patients) were extracted with a prespecified codebook; risk of bias was appraised with JBI checklists augmented by automated logic filters.  
Effect measures for case reports were odds ratios (ORs) with 95\,\% confidence intervals (CIs) derived from 2~$\times$~2 tables ($\chi^2$ or Fisher’s exact test). Multivariate case-level associations were explored with logistic regression.  
To enable valid series-level meta-analysis despite small sample sizes, individual case reports were flattened and grouped alongside other series.  
Series-level proportions were pooled using random-effects models with restricted maximum likelihood (REML) estimation and Hartung--Knapp adjustments to confidence intervals, appropriate for small-study meta-analysis.  
DerSimonian--Laird estimates were also computed for comparison.  
Univariate meta-regression examined study-level modifiers.

\subsection*{Results}
Across 162 unique individuals, pooled series-level event rates using REML were:  
endoscopy 47.3\,\% (95\,\% CI: 4.3–94.7),  
surgery 30.6\,\% (95\,\% CI: 12.0–58.9),  
conservative management 41.6\,\% (95\,\% CI: 4.4–91.7),  
complications 34.7\,\% (95\,\% CI: 1.8–93.8),  
mortality 2.5\,\% (95\,\% CI: 0.8–7.7).  

Heterogeneity was moderate to low for all outcomes except death ($I^2 < 1\,\%$).  
DerSimonian--Laird estimates yielded narrower confidence intervals and higher $I^2$ values, consistent with known limitations in small samples.  

Case-level testing showed that \enquote{intent to harm} quintupled the odds of surgery (OR~$\approx$~5, $p < 0.05$), whereas \enquote{other} motives reduced surgical odds by 86\,\%. Sharp objects increased the likelihood of endoscopy, and long objects trended toward higher surgical rates.  
Age 41--60 years predicted fewer surgeries (OR~$\approx$~0.18, $p < 0.05$). No demographic factor independently altered mortality, although detained status and severe disability exhibited non-significant trends.  
Meta-regression of series data did not replicate these signals, likely due to small study numbers.
\subsection*{Limitations}
Single-reviewer screening and extraction, high clinical heterogeneity, small cell sizes, and publication bias (particularly for dramatic or anatomically challenging cases). Twenty-five predictors in four case series risked model overfitting and inflated Type I error.

\subsection*{Conclusions}
Intention itself appears to be a key driver of adverse clinical course after foreign-object ingestion, with \enquote{intent to harm}, sharp items, and object length portending more invasive management. Age between 41 and 60 years may be protective against surgery. The findings should inform risk stratification but must be interpreted cautiously given methodological limitations and the descriptive nature of much of the evidence.

\subsection*{Registration and Data}
This protocol was not registerd with PROSPERO. Full dataset and analysis scripts are openly available at \url{https://github.com/jackgedge/iifo_systematic_review}.

