\section*{Abstract}

\paragraph*{Background} Intentional ingestion of foreign objects (IIFO) is a recurrent presentation in emergency, gastro-surgical and custodial settings, yet the prognostic value of patient \emph{motivation} has never been quantified.  

\paragraph*{Objective} To establish whether declared motive, object characteristics and demographic factors predict management strategy and clinical outcome after IIFO.  

\paragraph*{Methods} Five databases were searched from inception to October~2024 (PRISMA-2020).  Seventy-two single-patient case reports and three homogeneous case series (\emph{n}=162) met predefined criteria.  Pooled management proportions were derived with random-effects models; grouped univariate meta-regression (series level) and mixed-effects logistic regression (case level) examined associations with treatment modality and complications.  

\paragraph*{Results}  Pooled management rates were 40.6\,\% for conservative care (95\,\%~CI 29.2–52.8; $I^{2}\!\approx\!75$), 41.6\,\% for endoscopy (29.5–54.7; $I^{2}\!\approx\!95$) and 17.8\,\% for surgery (12.0–25.8; $I^{2}\!\approx\!17$).\\
\emph{Motivation.}  A declared intent to self-harm increased surgical use by 11\,\% (series-level aOR 1.11, 95\,\%~CI 1.05–1.18) and reduced conservative management by 17\,\% (aOR 0.83, 0.79–0.86); patient-level modelling showed a 16-fold rise in operative intervention (aOR 15.97, 1.32–192.40).  Protest-motivated ingestions—almost exclusive to detainees—conferred the highest complication risk (aOR 290.44, 1.40–60\,043.40) without a parallel rise in surgery.  Psychiatric and broader psychosocial motives produced smaller, directionally similar shifts.\\
\emph{Object factors.}  Objects $>$5\,cm increased surgical odds seven-fold and complication odds twenty-fold; ingesting multiple objects raised complications eighteen-fold.  Sharp items reduced endoscopic retrieval (OR 0.34, 0.13–0.90).\\
\emph{Demographics.}  Male sex lowered conservative treatment by 13\,\%; detention status paradoxically correlated with fewer complications (aOR 0.03, 0.00–0.59).\\
Overall, complications occurred in 7.3\,\% (2.9–17.3) of patients and three deaths were reported.

\paragraph*{Conclusions}  Motivation rivals object size as a determinant of outcome in IIFO.  Explicit self-harm signals early surgical need, whereas protest ingestion heralds delayed but severe morbidity.  Routine documentation of motive, object metrics—particularly in custodial environments—should be integrated into triage algorithms. Evidence certainty is low, highlighting the need for prospective, multi-centre registries with standardised data dictionaries and longitudinal follow-up.
