\subsection*{Risk of Bias Assessment}

Risk of bias was assessed manually for all included studies by a single reviewer (JGE), using the Joanna Briggs Institute (JBI) Critical Appraisal Checklists for Case Reports and Case Series \cite{Moola_2020}. Studies were first classified as either case reports or case series based on the level of granularity in the data. Case reports were assessed using the JBI checklist for case reports, and case series were assessed using the corresponding JBI tool.

Following manual appraisal, a secondary risk-of-bias filter was applied using Python Pandas \cite{ThePandasDevelopmentTeam_2020}. This logic-based filter identified studies where key variables — specifically \textit{Outcome}, \textit{Motivation}, or \textit{Object} — were missing or marked as unknown (\enquote{UK}). For case series, if any of the derived aggregate fields (e.g. \textit{Outcome\_Unknown\_Rate}, \textit{Motivation\_Unknown\_Rate}, \textit{Object\_Unknown\_Rate}) equalled 1, the study was flagged as high risk. Similarly, case reports where any of these variables were unknown were also considered high risk.

Studies classified as high risk through this process were excluded from analysis. This two-stage approach — involving initial manual assessment and subsequent automated validation — ensured both qualitative and quantitative scrutiny of bias across the dataset.
