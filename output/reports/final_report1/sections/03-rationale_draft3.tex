\subsection*{Rationale}

The global displacement crisis has reached unprecedented levels, with over 100 million forcibly displaced individuals reported by the United Nations High Commissioner for Refugees (UNHCR) as of May 2024 \cite{UNHCR_2022}. Refugees and asylum seekers often endure severe hardship—including violence, persecution, and perilous journeys—compounding their vulnerability upon arrival in host countries \cite{UNHCR_2010, AmnestyInternational_2024}. This population is at heightened risk of mental health challenges due to trauma, detention, and insecure legal status \cite{Athwal_2015, Sundvall_2015, Nickerson_2019, Bevione_2024}.

Self-harm is a particularly concerning manifestation of psychological distress among asylum seekers, with rates significantly exceeding those in the general population—up to 216 times higher in offshore detention settings \cite{vonWerthern_2018, Hedrick_2019, GlobalDetentionProject_2024}. Common methods include cutting, blunt trauma, hanging, self immolation, poisoning, and intentional ingestion of foreign objects (IIFO) \cite{Hedrick_2019, Davidson_2016}, influenced by access to means, cultural norms, and underlying motivations \cite{Ajdacic-Gross_2008}.

IIFO is defined as the non-accidental ingestion of non-nutritive foreign bodies \cite{Becq_2021}. While 80–90\% of cases resolve spontaneously, 10–20\% require endoscopic intervention and up to 1\% require surgery \cite{Ikenberry_2011, Birk_2016}. Timely access to care is essential, but often constrained in refugee contexts by geographic isolation and limited medical infrastructure, increasing the risk of complications \cite{Vujkovic_2019}.

Global rates of IIFO appear to be rising. In the United States, incidence doubled in 2017, with 14\% of cases classified as intentional \cite{Hsieh_2020}. One review found up to 92\% of adult ingestions were intentional among individuals from lower socioeconomic backgrounds, suggesting even higher prevalence in displaced populations \cite{Palta_2009}.

The management of IIFO has evolved significantly since the 17\textsuperscript{th} century, from open gastrotomy to rigid esophagoscopy and, more recently, endoscopic techniques \cite{Moehlau_1895, Saint_1929, Barros_1991, Lerche_1911, Jackson_1957}. Historical case reports illustrate the extremes of this behaviour—including one psychiatric patient who ingested over 2,500 objects weighing 21 kilograms \cite{Chalk_1928}, and another who swallowed an item 28 cm in length \cite{Ricote_1985}.

Clinical outcomes depend on multiple factors: age, comorbidities, object size and characteristics, location, and time to presentation. Current guidelines recommend intervention strategies based on these variables \cite{Ikenberry_2011}. However, most literature to date focuses on IIFO in prisons or psychiatric populations, with limited data from refugee or asylum-seeking groups.

In detention settings, ingestion may be used as a protest or distress signal when other forms of communication are blocked \cite{Puggioni_2014}. In psychiatric contexts, motivations may include psychosis, personality disorders, pica, malingering, or affective dysregulation \cite{Gitlin_2007, Tromans_2019, Al-Faham_2020b, Pantazopoulos_2022, Aitchison_2024, Poynter_2011}. Malingering is especially reported in prisons, often as a means to access hospital care \cite{Losanoff_1997e, Gitlin_2007}. Conversely, in borderline personality disorder, ingestion may serve as an emotional regulation strategy rather than an indication of suicidal intent \cite{Gitlin_2007, Poynter_2011}. In rare cases, repeated IIFO has led to discussions around palliative care approaches, recognising the futility of repeated surgical intervention in treatment-resistant psychiatric illness \cite{Jaini_2023}.

Despite increasing prevalence, the heterogeneous motivations behind IIFO—and their potential impact on clinical decision-making—remain poorly understood \cite{Bhugra_2010, Haase_2022, Hedrick_2023a}. Motivation likely influences management strategies and outcomes. For instance, protest-driven ingestion may avoid high-risk behaviours, lowering the threshold for conservative management \cite{Ataya_2013, Albeldawi_2014}.

This systematic review aims to address these gaps by exploring how motivation for IIFO affects clinical outcomes. Specifically, it examines associations between motivation and rates of endoscopic and surgical intervention, conservative management, complications, and mortality—toward informing more responsive and appropriate healthcare strategies for vulnerable populations.