/Users/jackgedge/Projects/msc_dissertation/iifo_motivation/output/reports/final_report/sections/appendix_f.texIn the age group subgroup, 41--60 age group was significantly associated with reduced odds of surgery (OR = 0.18, 95\% CI [0.04, 0.76], p = 0.019). In the motivation subgroup, intent to harm motivation was significantly associated with increased odds of surgery (OR = 5.40, 95\% CI [1.36, 21.43], p = 0.024); other motivation was significantly associated with reduced odds of surgery (OR = 0.14, 95\% CI [0.03, 0.75], p = 0.023). In the object subgroup, sharp object ingestion was significantly associated with reduced odds of endoscopy (OR = 0.34, 95\% CI [0.13, 0.90], p = 0.048). There were no significant associations with outcomes in the gender, and population subgroups.
\paragraph*{Multivariate Logistic Regression} A full table of grouped series-level logistic regression results is available in Table~\ref{tab:grouped_logistic_wide}. In the motivation subgroup, intent to harm motivation was significantly associated with increased odds of surgery (aOR = 15.97, 95\% CI [1.32, 192.40], p = 0.029); protest motivation was significantly associated with increased odds of complication (aOR = 290.44, 95\% CI [1.40, 60043.40], p = 0.037). In the object subgroup, long (\textgreater{}5cm) object ingestion was significantly associated with increased odds of surgery (aOR = 7.66, 95\% CI [1.04, 56.33], p = 0.045); multiple object ingestion was significantly associated with increased odds of complication (aOR = 18.24, 95\% CI [2.53, 131.53], p = 0.004); long (\textgreater{}5cm) object ingestion was significantly associated with increased odds of complication (aOR = 20.39, 95\% CI [1.89, 220.01], p = 0.013). In the population subgroup, detained person was significantly associated with reduced odds of complication (aOR = 0.03, 95\% CI [0.00, 0.59], p = 0.021). There were no significant associations with outcomes in the age group, and gender subgroups.
\subsubsection*{Case Series} 
\paragraph*{Meta-analysis of Proportions} A plot of series-level meta-analysis of pooled outcome proportions is shown in Figure~\ref{fig:meta}. Meta-analyses of proportions was performed for endoscopy, conservative, and surgery. The pooled proportion of patients undergoing endoscopy was 41.6\% (95\% CI 0.6\%--98.9\%), with substantial heterogeneity ($I^2$ = 94.9\%). The pooled proportion of patients conservative management was 76.2\% (95\% CI 60.2\%--87.1\%), with low heterogeneity ($I^2$ <0.5\%). The pooled proportion of patients undergoing surgery was 17.8\% (95\% CI 10.4\%--28.8\%), with low heterogeneity ($I^2$ = 17.4\%).\paragraph*{Meta Regression} A full table of grouped aggregate series-level results for univariate meta-regression is available in Table~\ref{tab:series_meta_regression}. In the gender subgroup, male gender was associated with a reduced likelihood of conservative management (OR = 0.87, 95\% CI [0.78, 0.97], p = 0.039). All other comparisons in this subgroup were non-significant. In the population subgroup, being a detained person was associated with an increased likelihood of conservative management (OR = 1.62, 95\% CI [1.11, 2.38], p = 0.039). All other comparisons in this subgroup were non-significant. In the motivation subgroup, intent to harm was associated with a reduced likelihood of conservative management (OR = 0.83, 95\% CI [0.79, 0.86], p = 0.012); intent to harm was associated with an increased likelihood of undergoing surgery (OR = 1.11, 95\% CI [1.05, 1.18], p = 0.015); psychiatric motivation was associated with an increased likelihood of undergoing surgery (OR = 1.07, 95\% CI [1.03, 1.11], p = 0.017); another documented motivation was associated with a reduced likelihood of conservative management (OR = 0.69, 95\% CI [0.51, 0.92], p = 0.039); psychosocial motivation was associated with a reduced likelihood of conservative management (OR = 0.82, 95\% CI [0.70, 0.96], p = 0.039); another documented motivation was associated with an increased likelihood of undergoing surgery (OR = 1.25, 95\% CI [1.02, 1.52], p = 0.040); psychosocial motivation was associated with an increased likelihood of undergoing surgery (OR = 1.12, 95\% CI [1.01, 1.25], p = 0.040). All other comparisons in this subgroup were non-significant. In the object subgroup, Sharp object ingestion was associated with a reduced likelihood of conservative management (OR = 0.81, 95\% CI [0.68, 0.98], p = 0.044). All other comparisons in this subgroup were non-significant. 