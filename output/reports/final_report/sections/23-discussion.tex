\subsection*{Interpretation of Results}
The present review represents, to the author's knowledge, the largest synthesis of intentional ingestion of foreign objects (IIFO) cases published to date, aggregating evidence from 72 case reports (\emph{n}=72 individuals) and three case series (\emph{n}=90 individuals). Several clinically salient patterns emerged.

\paragraph*{Patient motivation is a key driver of management and outcomes} Across aggregated case‑level data, an explicit intent to self‑harm was strongly and independently associated with a greater likelihood of surgical intervention (adjusted odds ratio [aOR] 15.97, 95\% CI 1.32–192.40).  In the series‑level meta‑regression, the same motivation increased the pooled probability of surgery by 11\% and simultaneously reduced the probability of conservative management by 17\%. These findings suggest that self‑harm intention may correlate with the ingestion of more dangerous objects (e.g., longer, multiple, or sharp items) or with more limited cooperation during endoscopic retrieval, thereby lowering the threshold for operative management.

By contrast, protest‑related ingestions-almost exclusively reported in detained populations—were associated with a markedly elevated risk of complications (aOR 290.44, 95\% CI 1.40–60,043.40) despite being less frequently managed surgically. This paradox may reflect the ingestion of less inherently hazardous objects (e.g., multiple coins or cutlery, perhaps based on availability of objects) but delayed disclosure, refusal of treatment, and prolonged mucosal contact before medical assessment. Psychiatric and psychosocial motivations were intermediate, modestly increasing surgical rates while decreasing conservative approaches in the series‑level analyses, perhaps reflecting the diversity of motivations that are presented in these motivation categories.

\paragraph*{Object characteristics modulate both procedural choice and adverse events} Long objects (>5 cm) increased the odds of surgery seven‑fold and complications twenty‑fold, underscoring existing guideline thresholds that recommend operative removal when length exceeds 5–6 cm. Multiple objects behaved similarly, multiplying the adjusted odds of complications by 18, whereas sharp objects paradoxically reduced the likelihood of endoscopic retrieval (OR 0.34, 95\% CI 0.13–0.90).  The latter may reflect pre‑emptive surgical management of sharp items or technical constraints that prompt a conservative waiting period until spontaneous passage or localisation stabilises for operative extraction.

\paragraph*{Demographic and setting factors exert more modest effects} Age between 41 and 60 years was independently linked to a lower probability of surgery in univariate testing, but this effect did not persist after multivariable adjustment, suggesting confounding by motivation or object type. Male sex—over‑represented in both the case series (100\%) and the aggregated cohort (60\%)—reduced the pooled probability of conservative management by 13\%.  Detention status, while associated with protest motivation, surprisingly correlated with fewer complications in multivariable models (aOR 0.03, 95\% CI 0.00–0.59). One plausible explanation is the heightened clinical vigilance and rapid escalation pathways available in custodial healthcare settings. Another explanation, as seen in some cases \cite{Albeldawi_2014, Ataya_2013}, could be that when ingesting as a form of protest, ingestors may modify seemingly radiographically high risk looking objects (e.g. metal razorblades) with radiolucent protection (e.g. tape, paper, or chewing gum). Objects would appear high risk on imaging, but the physical mucosal risks from sharp edges (and perhaps thus complications) would be abutted.

\paragraph*{Treatment patterns are heterogeneous.}  Meta‑analysis of case series demonstrated wide variation in endoscopic use (pooled proportion 41.6\%, $I^{2}=94.9\%$), whereas surgical (17.8\%, $I^{2}=17.4%$) and conservative (76.2\%, $I^{2}<0.5%$) rates were more consistent.  The extreme heterogeneity for endoscopy likely reflects differences in local endoscopic expertise, resource availability, and object profiles across centres.  Collectively, the data support a risk‑stratified approach: cooperative patients who ingest single, blunt, short objects appear suitable for observation or endoscopy, while individuals with self‑harm intent, long or multiple objects, or protest motivations warrant early surgical consideration.

\paragraph*{Implications for practice and research.} These findings highlight the importance of systematically documenting the psychosocial context of ingestion alongside detailed object metrics.  Future prospective registries should integrate standardised motivation coding and time‑to‑presentation metrics to refine predictive models. Clinicians should maintain a high index of suspicion for complications in detained or protest‑motivated ingestions even when initial imaging appears favourable.

\subsection*{Limitations of the Evidence Base}
The certainty of the above interpretations is tempered by several inherent weaknesses in the available literature.

\paragraph*{Predominance of uncontrolled designs} All included studies were descriptive case reports or case series lacking comparator groups. This precludes causal inference and amplifies susceptibility to selection bias, as more severe or dramatic presentations are preferentially published.
\paragraph*{Small sample sizes and sparse data} Even after pooling, outcome events such as death (\emph{n}=3) were rare, limiting statistical power and inflating confidence intervals—most notably in the multivariable models where some estimates span several orders of magnitude.
\paragraph*{Heterogeneous—and sometimes opaque—variable definitions} Motivation categories were author‑defined and variably applied, raising the risk of misclassification. Object descriptors ("long", "sharp", "large diameter") were inconsistently measured, and key parameters such as exact dimensions or material composition were often omitted.
\paragraph*{Incomplete outcome ascertainment and follow‑up} Many reports truncated follow‑up at hospital discharge, obscuring delayed complications such as stricture formation or psychiatric relapse. Without uniform longitudinal data, outcome proportions may underestimate true morbidity.
\paragraph*{Publication and reporting biases} Funnel‑plot assessment was infeasible, but the over‑representation of male detainees and dramatic surgical cases suggests selective reporting.  Negative or conservatively managed cases, particularly in community psychiatry settings, are likely under‑published.
\paragraph*{Limited generalisability} The aggregated series drew heavily from custodial environments in North America and Europe. Cultural, legal, and healthcare‑system differences mean that motivation profiles, object availability, and management pathways may diverge in other regions.
\paragraph*{Temporal changes in clinical practice} The study period spans nearly three decades, during which endoscopic technology, retrieval devices, and guideline recommendations have evolved.  Earlier cases may not reflect contemporary management capabilities.
\paragraph*{Missing motivation data and unclear intentionality} Fifty otherwise eligible papers did not specify patient motivation and were therefore excluded from quantitative synthesis; as motivation strongly influences management decisions, this omission likely biases effect estimates toward cohorts with better-documented psychosocial backgrounds. An additional 94 papers failed to confirm whether the ingestion was intentional, leading to their exclusion. These gaps constrain the representativeness of our sample and may under- or over-estimate the frequency of self-harm or protest-driven ingestions.
\paragraph*{Absence of anatomical location data} Although many primary reports specified the site of lodgement (e.g., oesophagus, stomach, small bowel), systematic extract or coding this variable was not undertaken. Failure to adjust for location—particularly oesophageal impaction, which is linked to higher perforation risk and lower endoscopic success\cite{Ikenberry_2011,Birk_2016}—may confound the observed associations between motivation, object characteristics, and outcomes.
\paragraph*{Undifferentiated psychiatric history} Psychiatric history was extracted only as a binary yes/no field, with no breakdown by diagnostic category, severity, or treatment status. Collapsing diverse conditions—such as borderline personality disorder, intellectual disability, or psychosis—into one umbrella variable likely diluted diagnosis‑specific risk signals and limited our ability to model nuanced links between mental‑health phenotypes, motivation, and outcomes. Prospective studies should capture granular psychiatric diagnoses and therapeutic engagement to refine risk stratification.
\paragraph*{Case‐series dominance by male detainees} All three pooled series described exclusively male prison inmates—most with protest or self‑harm motives and sharp objects such as razor blades \cite{Karp_1991b, Lee_2007, Elghali_2016}. Their homogeneous demographic and custodial context disproportionately increased the weight of detained‑person, male‑gender, and protest‑motivation categories in the series‑level models, likely amplifying the observed association between protest motivation and complications and contributing to the extreme heterogeneity seen for endoscopic rates. Consequently, pooled proportions may not extrapolate to civilian, paediatric settings.

\paragraph*{Recommendation for future reporting} Given the prognostic importance of motivation, this review strongly encourage authors and clinicians to document explicitly whether an ingestion is intentional or accidental and to record the underlying motive when known. Such standardised reporting would enhance the comparability of studies and improve risk-stratified care.

Taken together, these limitations necessitate cautious interpretation of effect sizes and underscore the need for prospective, multi‑centre registries with standardised data dictionaries and longer follow‑up horizons.
\subsection*{Limitations of the Review Process} Several limitations of the review process should be acknowledged. Firstly, although a comprehensive search strategy was implemented, it is possible that relevant studies were missed, particularly unpublished case series or reports not indexed in major databases. Secondly, screening and data extraction were conducted by a single reviewer, which may introduce the risk of human error or subjective bias in study selection and coding. To mitigate this, detailed protocols and reproducible scripts were used, and key results were cross-checked. Thirdly, reporting of outcomes and predictor variables varied substantially across included studies, particularly in case reports, which limited the completeness and consistency of data synthesis. Finally, no formal assessment of study-level risk of bias was performed, as the majority of included studies were descriptive case series and reports without comparative designs. These limitations should be considered when interpreting the findings of this review.