\subsection*{Interpretation of Results}


\subsubsection*{Strengths}
The present review represents, to the author's knowledge, this is the first review of IIFO that focuses on motivations for ingestion, aggregating evidence from 162 ingestion cases with documented motivations.

The main strength of this review lies in its methodology. Where motivation was unclear, cases were excluded—ensuring that the effects measured were genuinely reflective of intentionality. This strict inclusion criterion helps isolate intentional ingestion as an independent clinical phenomenon. Additionally, by avoiding age restrictions, this review includes a wide age range (from 7 to 100 years), improving the generalisability of its findings across the lifespan.

A broad search strategy was implemented, covering multiple databases, bibliographies, and grey literature sources. This increases the likelihood that the findings reflect a wide cross-section of the literature and reduces the impact of publication bias due to indexing alone.

Despite substantial heterogeneity in the review population, outcome proportions were pooled using random-effects meta-analysis with restricted maximum likelihood estimation and Hartung–Knapp adjustments—techniques appropriate for small numbers of studies and high between-study variance. Following IIFO, patients underwent endoscopy in 47.3\% of cases (95\% CI 4.3\%--94.7\%), surgery in 30.6\% (95\% CI 12.0\%--58.9\%), and conservative management in 41.6\% (95\% CI 4.4\%--91.7\%). Complications occurred in 34.7\% of cases (95\% CI 1.8\%--93.8\%), and 2.5\% of patients died (95\% CI 0.8\%--7.7\%). These figures are substantially higher than previously reported in the literature~\cite{Ikenberry_2011, Birk_2016}. 

Given the methodological rigour of this review---including strict exclusion of non-intentional cases---and despite the wide uncertainty intervals, the findings strongly suggest that \emph{intentionality alone} is a key driver of morbidity and mortality following foreign body ingestion.

\subsubsection*{Associations Between Motivation and Outcome}
While the univariate meta-regression (predominantly conducted on detained males) didn’t show statistically significant correlations between motivations and outcomes, signals did emerge in the univariate association testing conducted on the more varied population represented in the case reports. This population was more diverse—39\% female, 17\% detained individuals, and 50\% with a psychiatric history—and included a wider range of object types (e.g., magnets, multiple items, long objects). 

In this context, \enquote{intent to harm} was associated with a fivefold increase in the odds of surgery, whereas “other” motivations were associated with an 86\% decrease in the odds of surgery. These findings suggest that motivations likely influence clinical outcomes, particularly in more heterogeneous and less institutionalised populations.

\subsubsection*{Subgroup Findings}
Interestingly, among severely disabled people (defined here as individuals with reduced levels of consciousness or severe learning disabilities), univariate meta-regression showed a 41\% decrease in the odds of death. While this cohort was small (\emph{n}=7), and all cases were published as case reports, this may reflect publication bias—authors may be more likely to report complex or anatomically challenging cases \cite{Ken_1993}. However, it might also reflect the diligence of carers, earlier access to intervention, and perhaps the success of public health campaigns aimed at recognising and addressing disproportionate harms in this population.

Also noteworthy, displaced individuals had reduced odds of death (OR 0.22, 95\% CI 0.08–0.61, \emph{p} = 0.034) in the meta-regression. This is unlikely to be solely due to their displacement status. The two cases underpinning this result \cite{Akay_2015f, Gardner_2017} are strikingly different. One involved a person who swallowed a pen in protest of an asylum decision and developed a double duodenal perforation, requiring surgery; the other involved a man who swallowed \$1000 wrapped in plastic for safe-keeping, who then had it endoscopically removed and returned to him. These examples illustrate well how both motivation and object characteristics can drive clinical outcomes—and why both should be considered in future analysis.

\subsection*{Limitations of the Review Process} 
Several limitations of the review process should be acknowledged.

\subsubsection*{Selection and Screening}
Many studies were excluded because intentionality was not clearly stated. Given the nuanced and inconsistent language often used in clinical reporting, some relevant cases may have been omitted. This likely introduced selection bias, especially in a single-author review, where subjective judgment plays a larger role.

Although the search strategy was broad, relevant studies may still have been missed—particularly unpublished material or reports not indexed in major databases. This is a recognised limitation in rare-event research.

Screening and data extraction were carried out by a single reviewer. While detailed protocols and reproducible scripts were used to minimise error, the absence of a second full-text reviewer and data extracter increases the risk of human error and subjective bias, particularly when interpreting ambiguous descriptions of intent and motivation. Manually reviewing over 480 full texts and extracting detailed case-level data from almost half is a substantial undertaking and a recognised limitation.

\subsubsection*{Motivation Classification}
The motivation categories were author-defined and applied post hoc following preliminary analysis. While this was necessary due to the lack of standardised classifications, it introduces classification bias, especially in a field as complex and subjective as IIFO. Motivational constructs such as \enquote{intent to harm}, \enquote{protest}, or \enquote{other} may overlap or be poorly described in reports, affecting how cases are grouped and interpreted.

\subsubsection*{Statistical Limitations}
The analytical approach may have been overly complex given the volume and quality of the available literature. Conducting regression on 25 variables across only four case series risks overfitting, reducing the reliability of these models. A more parsimonious strategy might have involved collapsing object variables into a binary \enquote{high-risk vs low-risk} classification, informed by existing literature on object characteristics and classification~\cite{Ikenberry_2011, Birk_2016}.

Similarly, psychiatric history was treated as a binary variable, but as noted in the introduction, psychiatric presentations associated with IIFO are varied and complex. The binary coding limits the ability to identify nuanced associations. A more granular analysis of psychiatric subtypes could provide deeper insight but would further complicate models already stretched by sample size constraints. 

\subsubsection*{Other Gaps}
This review did not examine the types or effectiveness of conservative management strategies, despite their prevalence and clinical relevance. Nor did it consider object location, which is known to be a critical determinant of clinical risk. Both factors are likely to affect outcomes and should be included in future studies.

It is possible that collapsing the case report data into a series introduced heterogeneity into the series level data. In the series', all cases were males and most were detained individuals.

The psychiatric motivation component may have been underestimated. In psychiatric case reports, authors frequently report diagnoses and lengthy case timelines, but not explicit motivations for individual motivations.

\subsection*{Takeaway points}
Ultimately, the literature on motivations in IIFO remains sparse, and clinical outcomes remain highly heterogeneous. The evidence base is not yet strong enough to draw definitive conclusions. As with many rare and complex presentations, prospective, randomised studies are needed—but these must be preceded by expert consensus on motivation frameworks.

IIFO is a nuanced and multifaceted form of self-harm. Motivation is rarely reported, and intentionality itself is often assumed rather than stated. Following this review, authors should heed the advice of experts in the field: a multidisciplinary approach is essential. Psychiatric and psychological input is needed not just for treatment, but also for accurate classification and understanding of patient motivations.

With more precise reporting of motivational and clinical variables, future reviews will be better placed to inform evidence-based guidance, ultimately improving care for this complex patient group.
