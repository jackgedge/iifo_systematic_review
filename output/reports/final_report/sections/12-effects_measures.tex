\subsection*{Effects Measures}

\subsubsection*{Case Reports}

\paragraph*{Univariate Association Testing}

For binary outcomes (endoscopy, surgery, complications, and death), the effect measure used was the odds ratio (OR), calculated from 2x2 contingency tables. Each odds ratio was accompanied by a 95\% confidence interval (CI) and a p-value from a chi-square test of independence.

This approach was used consistently across all pairwise comparisons between binary exposure variables (e.g., motivations, object types, population characteristics) and binary outcome variables. Significant associations were identified at a threshold of $p < 0.05$ and reported alongside their respective ORs and CIs. Due to the small number of deaths observed, effect estimates for death should be interpreted with caution.

\paragraph*{Logistic Regression Modelling}

To explore which factors were independently associated with clinical outcomes, we performed multivariate logistic regression analyses for four outcomes of interest: endoscopy, surgery, complications, and death. For each outcome, we constructed a logistic regression model including the afforementioned groups of predictor variables: age group, gender, demographic, ingestion motivation, and object characteristics.

Age group, gender, and motivation variables were entered as one-hot encoded categorical variables with a reference category omitted. Population characteristics and object type variables were included as binary indicators. A constant term was included in each model. All variables were selected a priori based on clinical relevance and prior univariable (chi-square) analyses.

Missing values in predictor variables were imputed as zero. Models were fitted using maximum likelihood estimation via the 	extit{statsmodels} Python library. In the event that a model failed to converge or could not be fitted (as occurred for the death outcome due to small sample size), an empty result table was substituted to maintain consistency of reporting across outcomes.

For each predictor, we reported the odds ratio (OR) with corresponding 95\% confidence interval (CI) and p-value. Results from all models were summarised in a single grouped wide table, with predictors grouped into their logical domains (age, gender, demographic, motivation, object). The intercept term (	exttt{const}) was excluded from the summary table. Significant predictors ($p < 0.05$) were flagged with an asterisk.

\subsubsection*{Case Series}

\paragraph*{Meta-analysis of Proportions}

To provide descriptive summary estimates of clinical outcomes across included case series, we conducted meta-analyses of proportions for five outcomes of interest: endoscopy, surgery, complications, death, and conservative management. For each outcome, we calculated the observed proportion within each series and performed a random-effects meta-analysis using the DerSimonian-Laird method to pool proportions on the logit scale. Between-study heterogeneity was quantified using the $I^2$ statistic and between-study variance ($	au^2$). These analyses provided overall estimates of outcome frequencies across studies and informed interpretation of subsequent meta-regression analyses. All meta-analyses were conducted using custom Python code implementing standard meta-analytic formulas.

\paragraph*{Meta-Regression}

We anticipated that the number of independent case series would be small, limiting the feasibility of multivariable modelling. To increase the effective number of contributing series, we combined the series-level data with the aggregate case report data, collapsed to series level.

We then performed univariate meta-regression to explore associations between predictor variables (gender, demographic, ingestion motivations, object characteristics) and clinical outcomes (endoscopy, surgery, conservative management, complications, death). For each outcome, we modelled the logit-transformed proportion of cases using weighted least squares, with inverse variance weighting to account for differing series sizes.

Significant associations ($p < 0.05$) were reported for each predictor, grouped by conceptual domain. All other comparisons were noted as non-significant.
