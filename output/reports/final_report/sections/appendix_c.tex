\clearpage
\section{Computational Variable Definitions}
\label{appendix:variables_table}

\begin{table}[H]
\caption{Variables used for case report data extraction. Aggregates of which where used to create Variable\_Rate and Variable\_Cases.}
\centering
\begin{tabular}{p{4.5cm} p{10.5cm}}
\toprule
\textbf{Variable} & \textbf{Definition} \\
\midrule
Is\_Prisoner & Documented in prison, police custody, or detained (including immigration detention) at the time of the encounter; 'N' if not detained; 'UK' if unknown. \\
Psych\_Hx & Documented DSM-V mental disorder (including substance-related disorders) \cite{AmericanPsychiatricAssociation_2013}; 'N' if no diagnosis; 'UK' if data unavailable. \\
Is\_Displaced\_Person & 'Y' if: meets the UN General Assembly \cite{UNGeneralAssembly_1967} definition of 'Refugee'; or meets UNHCR~\cite{Deng_1998} definition of an 'internally displaced person'; or meets the UNHCR~\cite{UnitedNationsHighCommissionerforRefugees_2025} definition for 'asylum seeker'; 'N' if not displaced; 'UK' if unknown. \\
Under\_Influence\_Alcohol & Evidence, suspicion, or self-report of alcohol influence at presentation; 'N' if no indication; 'UK' if unknown. \\
Is\_Psych\_Inpat & Admitted (voluntarily or involuntarily) to a psychiatric facility/ward at encounter; 'N' if not admitted; 'UK' if unknown. \\
Severe\_Disability\_Hx & History of severe learning disability or impaired consciousness; 'N' if absent; 'UK' if unknown. \\
Previous\_Ingestions & Prior episode of foreign-body ingestion documented; 'N' if first ingestion; 'UK' if history unknown. \\
Motivation\_Intent\_To\_Harm & Ingestion intended for self-harm, self-injury, or suicide; 'N' if other motive; 'UK' if unclear. \\
Motivation\_Protest & Ingestion as protest, demonstration, or manipulation (e.g., objection to detention conditions); 'N' if not protest-related; 'UK' if unclear. \\
Motivation\_Psychiatric & Ingestion driven primarily by an underlying psychiatric condition (psychosis, impulsivity, etc.); 'N' if not psychiatric; 'UK' if unclear. \\
Motivation\_Psychosocial & Ingestion motivated by social or interpersonal factors (imitative acts, shock value, body-image, safekeeping, etc.); 'N' if not psychosocial; 'UK' if unclear. \\
Motivation\_Unknown & No clear motivation identified in documentation; 'N' if specific motive recorded; 'UK' if ambiguous. \\
Object\_Button\_Battery & Button battery ingested; 'N' if not; 'UK' if object type not recorded. \\
Object\_Magnet & Magnet ingested; 'N' if none; 'UK' if unknown. \\
Object\_Long & Ingested object length $>$ 5 cm; 'N' if $\leq$ 5 cm; 'UK' if dimensions unknown. \\
Object\_Sharp & Object described as sharp or pointed (e.g., blades, nails, needles); 'N' if not sharp; 'UK' if unclear. \\
Object\_Multiple & More than one object ingested in same episode; 'N' for single object; 'UK' if number unspecified. \\
Object\_Unknown & Where object characteristics are unknown. 'N' if known; 'UK' if Unknown. \\
Outcome\_Endoscopy & Endoscopic intervention performed during episode; 'N' if not; 'UK' if unavailable. \\
Outcome\_Surgery & Surgical intervention performed (operative procedure under anaesthesia); 'N' if not; 'UK' if not documented. \\
Outcome\_Conservative & 'Y' if managed without endoscopy or surgery; 'N' if either procedure performed. \\
Outcome\_Death & Death causally related to ingestion complications; 'N' if survived; 'UK' if outcome unknown. \\
Outcome\_Complication & 'Y' if any complication directly related to ingestion or resulting from management strategy; 'N' if no complication; 'UK' if unknown. \\
Outcome\_Unknown & Where no outcome identified; 'N' if outcome identified; 'UK' if Unknown. \\
\bottomrule
\end{tabular}
\end{table}

\twocolumn
\section*{Variable Definitions}
\label{appendix:variables_text}

\subsection*{Demographic}
Individuals were considered detained if they were documented to be in prison, police custody, or immigration detention at the time of the encounter. If they were not detained, this was recorded as ‘N’; if the status was unknown, it was marked as ‘UK’. Psychiatric history was defined as a documented diagnosis of a mental disorder according to DSM-V criteria, including substance-related disorders \cite{AmericanPsychiatricAssociation_2013}. Absence of a diagnosis was recorded as ‘N’, and unavailable data as ‘UK’. Displacement status was defined as meeting one of the following: the UN General Assembly’s definition of a ‘refugee’ \cite{UNGeneralAssembly_1967}, the UNHCR’s definition of an ‘internally displaced person’ \cite{Deng_1998}, or the UNHCR’s definition of an ‘asylum seeker’ \cite{UnitedNationsHighCommissionerforRefugees_2025}. Those who did not meet these definitions were marked as ‘N’, and those with unknown status as ‘UK’. Alcohol use was defined as any evidence, suspicion, or self-report of being under the influence of alcohol at the time of presentation. A lack of such indication was recorded as ‘N’, and unknown status as ‘UK’. Psychiatric inpatient status was defined as being admitted (voluntarily or involuntarily) to a psychiatric facility or ward at the time of the encounter. If the person was not admitted, this was recorded as ‘N’; unknown status was marked as ‘UK’. A history of severe disability included documented cases of severe learning disability or impaired consciousness. Absence of such a history was recorded as ‘N’, and ‘UK’ was used when data were unavailable. Previous ingestions referred to any documented prior episode of foreign-body ingestion. If it was the first recorded incident, this was marked as ‘N’; if unknown, ‘UK’ was used.
\label{appendix:variables_text}

\subsection*{Motivation}
Intent to harm was defined as ingestion carried out with the purpose of self-harm, self-injury, or suicide. If the ingestion was for another reason, it was recorded as ‘N’. If the intent was unclear, it was marked as ‘UK’. Protest-related ingestion was defined as an act of ingestion carried out as a form of protest, demonstration, or manipulation—for example, in response to detention conditions. If the ingestion was not motivated by protest, it was marked as ‘N’; if unclear, as ‘UK’. Psychosocial motivation included ingestion motivated by social or interpersonal dynamics—such as imitative behaviour, a desire to shock, body-image concerns, or the use of ingestion for safekeeping objects. If these factors were not involved, the case was marked as ‘N’; if unclear, as ‘UK’. Unknown motivation was recorded when no clear motivation was identifiable in the documentation. If a specific motive was documented, this was marked as ‘N’. If the available information was ambiguous, it was recorded as ‘UK’. 

\subsection*{Object}
Button battery ingestion was recorded when documentation confirmed that a button battery had been ingested. If no button battery was ingested, this was marked as ‘N’; if the type of object was not recorded, it was marked as ‘UK’. Magnet ingestion was identified when a magnet was reported as the ingested object. If no magnet was ingested, this was marked as ‘N’; if the object type was unknown, it was marked as ‘UK’. Long objects were defined as any ingested object longer than 5 cm. If the object was 5 cm or shorter, this was recorded as ‘N’. If the object’s dimensions were not known, the entry was marked as ‘UK’. Sharp objects included those described as pointed or capable of causing injury—such as blades, nails, or needles. If the object was not sharp, it was marked as ‘N’; if the description was unclear, it was recorded as ‘UK’. Multiple object ingestion referred to the ingestion of more than one object during a single episode. A single ingested object was recorded as ‘N’, and cases with unspecified object counts were marked as ‘UK’. Unknown object characteristics were recorded when no reliable information about the ingested object was available. If the object was identified, this was marked as ‘N’; if unknown, it was marked as ‘UK’.

\subsection*{Outcome}
An endoscopic intervention was recorded when a procedure using flexible or rigid endoscopy via natural orifices was performed to retrieve an ingested object or assess complications. If no endoscopy was performed, this was recorded as ‘N’; if the information was unavailable, it was marked as ‘UK’. A surgical intervention was defined as any operative procedure carried out in a sterile operating theatre under general or regional anaesthesia. This included procedures such as laparotomy, laparoscopy, thoracotomy, or cervical exploration—either to remove an ingested object or to manage complications. If surgery was not performed, the case was marked as ‘N’; if it was not documented, it was recorded as ‘UK’. Conservative management referred to cases managed without either endoscopy or surgery. These were marked as ‘Y’. If either procedure was performed, the case was recorded as ‘N’. Mortality was defined as death causally related to ingestion or its complications. If the patient survived, this was recorded as ‘N’; if the outcome was unknown, it was marked as ‘UK’. Complications were recorded when there was a direct clinical consequence related to the ingestion or the treatment strategy—for example, perforation, bleeding, or infection. Absence of complications was recorded as ‘N’; if the status was unclear or undocumented, it was marked as ‘UK’. Finally, unknown outcome was recorded when no clinical outcome (such as discharge, intervention, or death) could be identified from documentation. If an outcome was clearly documented, this was marked as ‘N’; if unknown, it was marked as ‘UK’.
"""

\FloatBarrier