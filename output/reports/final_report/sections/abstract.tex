\section{Abstract}

\subsection*{Background}
Intentional ingestion of foreign objects (IIFO) is a clinically distinct form of self-harm, yet its outcome profile and drivers of morbidity remain poorly defined. Whether patient motivation modifies clinical outcomes has never been systematically evaluated.

\subsection*{Objectives}
To synthesise all available evidence on outcomes following intentional foreign-object ingestion, and to determine whether patient motivation, object characteristics, or demographic factors influence the need for intervention, risk of complications, or mortality.

\subsection*{Methods}
A comprehensive search of PubMed, Embase, CENTRAL, Web of Science, Scopus, PsycINFO, and Google Scholar (1\textsuperscript{st} January 1906--31\textsuperscript{st} March 2025) identified studies reporting non-accidental ingestion of true foreign bodies in humans. 

Human studies of any design, any age, reporting intentional ingestion of non-digestible foreign objects that reported outcomes of endoscopy, surgery, conservative management, complications and mortality were included. Accidental or substance ingestion, animal studies, non-English full texts, pre-1906 reports, and studies lacking motivation, object, or outcome data were excluded. 

Titles, abstracts, and full texts were screened by two reviewers. Data were extracted for study-level and case-level characteristics including motivation, object type, demographics, and clinical outcomes. Meta-analysis of proportions (REML with Hartung–Knapp adjustments) was conducted on pooled series data. Univariate meta-regression was used to explore predictors of surgical intervention. Case reports were analysed descriptively and stratified by motivation.

\subsection*{Results}
Seventy-one case reports ($n = 71$) and three case series ($n = 90$). Case series populations were entirely male and detained, with pooled protest motivation rates of $\approx$80\% (moderate heterogeneity) and pooled sharp object ingestion rates of $\approx$90\%. Meta-regression suggested protest motivation was associated with significantly reduced odds of surgery ($\beta = -0.021$, $p = 0.003$). Intent-to-harm showed a non-significant trend toward increased surgery ($\beta = 0.257$, $p = 0.133$).

Among a more demographically diverse cohort of case reports, intent-to-harm was associated with nearly sixfold increased odds of surgery (OR = 5.68, 95\% CI [1.43--22.64], $p = 0.020$), while protest and other motivations were associated with reduced odds. Adults aged 40--64 had lower surgical rates (OR = 0.19, 95\% CI [0.04--0.78], $p = 0.020$), but all deaths in this review occurred in this age group, in the context of psychiatric co-morbidities, suggesting clinically high risk subgroup. Sharp object ingestion was associated with decreased likelihood of endoscopy, likely attributed to publication bias.

\subsection*{Conclusions}
Motivation appears to meaningfully influence management and outcomes following IIFO. Protest-driven ingestion, particularly in detained populations, may be more amenable to conservative management. Conversely, intent-to-harm significantly associated with surgical intervention, as does psychiatric comorbidity and middle age with mortality. Age, object type, and psychiatric history also shape clinical decisions. However, interpretation is limited by publication bias, selection, small sample sizes, and incomplete reporting.

These findings underscore the need for standardised, prospective reporting of intentional ingestion cases, with explicit attention to motivation, context, and object characteristics.

\begin{figure}[tb]
\centering
\includegraphics[width=\columnwidth]{figures/case_country_plot.png}
\caption{Heat map of all cases per country.}
\label{fig:country_heat_map}
\end{figure}