/Users/jackgedge/Projects/msc_dissertation/iifo_motivation/output/reports/final_report/sections/appendix_f.texIn the age group subgroup, 41--60 age group was significantly associated with reduced odds of surgery (OR = 0.18, 95\% CI [0.04, 0.76], p = 0.019). In the motivation subgroup, intent to harm motivation was significantly associated with increased odds of surgery (OR = 5.40, 95\% CI [1.36, 21.43], p = 0.024); other motivation was significantly associated with reduced odds of surgery (OR = 0.14, 95\% CI [0.03, 0.75], p = 0.023). In the object subgroup, sharp object ingestion was significantly associated with reduced odds of endoscopy (OR = 0.34, 95\% CI [0.13, 0.90], p = 0.048). There were no significant associations with outcomes in the gender, and population subgroups.
\paragraph*{Multivariate Logistic Regression} A full table of grouped series-level logistic regression results is available in Table~\ref{tab:grouped_logistic_wide}. In the motivation subgroup, protest motivation was significantly associated with increased odds of complication (aOR = 231.67, 95\% CI [1.16, 46389.11], p = 0.044). In the object subgroup, multiple object ingestion was significantly associated with increased odds of complication (aOR = 27.80, 95\% CI [3.18, 242.85], p = 0.003); long (\textgreater{}5cm) object ingestion was significantly associated with increased odds of complication (aOR = 13.74, 95\% CI [1.27, 148.57], p = 0.031). In the population subgroup, detained person was significantly associated with reduced odds of surgery (aOR = 0.04, 95\% CI [0.00, 0.77], p = 0.033); detained person was significantly associated with reduced odds of complication (aOR = 0.02, 95\% CI [0.00, 0.47], p = 0.015). There were no significant associations with outcomes in the age group, and gender subgroups.
\subsubsection*{Case Series} 
\paragraph*{Meta-analysis of Proportions} A plot of aggregate series-level meta-analysis of pooled outcome proportions is shown in Figure~\ref{fig:meta}. Meta-analyses of proportions was performed for endoscopy, surgery, death, complication, and conservative. The pooled proportion of patients undergoing endoscopy was 45.4\% (95\% CI 9.4\%--86.9\%), with substantial heterogeneity ($I^2$ = 93.5\%). The pooled proportion of patients undergoing surgery was 27.9\% (95\% CI 9.4\%--59.0\%), with substantial heterogeneity ($I^2$ = 90.3\%). The pooled proportion of patients dying was 3.4\% (95\% CI 1.1\%--10.1\%), with low heterogeneity ($I^2$ <0.5\%). The pooled proportion of patients experiencing a complication was 34.3\% (95\% CI 3.5\%--88.4\%), with substantial heterogeneity ($I^2$ = 96.6\%). The pooled proportion of patients conservative management was 50.4\% (95\% CI 8.3\%--91.9\%), with substantial heterogeneity ($I^2$ = 94.7\%).\paragraph*{Meta Regression} A full table of grouped aggregate series-level results for univariate meta-regression is available in Table~\ref{tab:series_meta_regression}. 