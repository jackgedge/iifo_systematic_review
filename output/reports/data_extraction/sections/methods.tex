\section{Methods}

\subsection{Search Strategy}
Relevant articles were identified through a systematic search of PubMed, Web of Science, Embase, Scopus, PsycINFO, CENTRAL and Google Scholar on 15th January 2025, with the assistance of a librarian. 

The search was conducted using keywords and MeSH terms based on the concepts underpinning this review. The bibliography of each included article was searched for any further relevant articles. The keywords and MeSH terms used can be found in Appendix \ref{appendix:search-strategy}.

\subsection{Eligibility Criteria}
We included original studies involving humans of any age group who had intentionally ingested a foreign object through the oral cavity (mouth). Studies were excluded if the ingestion was not explicitly documented as intentional or if empirical data on individual intentional ingestions were unavailable.

Motivations or reasons for ingestion considered included protest, suicidal intent, self-harm, psychiatric conditions, and other documented motivations. Intervention details assessed included the number of ingestions and the management strategies employed (conservative, endoscopic, surgical). Object characteristics evaluated encompassed the ingestion of multiple, blunt, sharp-pointed, long (\(>5\text{ cm}\)), and short (\(<5\text{ cm}\))

\subsection{Outcomes of Interest}
The primary outcomes of interest were rates of intervention: endoscopic intervention (defined as undergoing a minimally invasive procedure involving insertion of an endoscope to visually examine internal organs or tissues), surgical intervention (defined as any operative procedure involving an incision to retrieve ingested foreign objects or manage resulting complications), and conservative management (defined as cases not undergoing endoscopic or surgical intervention). Secondary outcomes included complication and mortality rates.

\subsection{Study Selection}

The study selection process is illustrated in Figure~\ref{fig:prisma-diagram}.

\begin{figure}[H]
    \centering
    \includegraphics[width=\textwidth]{figures/prisma_diagram.png}
    \caption{PRISMA 2020 flow diagram outlining the identification, screening, and inclusion of studies \cite{Page_2021}.}
    \label{fig:prisma-diagram}
\end{figure}

All identified articles were collated using Python (Pandas) \cite{ThePandasDevelopmentTeam_2020}. Duplicate articles were identified and removed based on non-unique combinations of author, title, and DOI.

Following duplicate removal, all remaining articles underwent title and abstract screening conducted by the first author (JGE). To ensure consistency, a randomly selected 10\% sample of these articles underwent independent screening by a second author (MS). Any discrepancies identified between these two reviewers were resolved by a third reviewer (GC).

Articles included after title and abstract screening proceeded to full-text review, which was initially performed by JGE. Again, a random 10\% sample of these full-text articles underwent independent assessment by MS. Discrepancies between JGE and MS at the full-text screening stage were similarly resolved by a third review from GC.

Inter-reviewer agreement at each screening stage was calculated using Python (Pandas for data management and Sci-kit Learn for statistical analysis) \cite{scikit-learn}. MS reviewed 50 papers during title and abstract screening. Fair agreement was reached with a Cohen's Kappa value of 0.381. 16 disagreements were reviewed by GC. In full text review, MS reviewed 31 papers. Almost perfect agreement was reached, with a Cohen's Kappa value of 0.832.

\subsection{Data Collection Process}

\paragraph{Demographic Variables}

\subsubsection{Prisoner Status \textit{(Is\_Prisoner)}}
Individuals were classified as \textit{Is\_Prisoner} = \enquote{Y} if they were documented as being in prison, held in police custody, or otherwise detained at the time of the encounter. This included immigration detention and other forms of custodial supervision. Where there was no indication of detention status, \textit{Is\_Prisoner} was marked as negative (N), or \enquote{UK} if unknown.

\subsubsection{Psychiatric History \textit{(Psych\_Hx)}}
Psychiatric history was classified as positive (\textit{Psych\_Hx = \enquote{Y}}) if the individual had a documented diagnosis of a mental disorder as defined by the \textit{Diagnostic and Statistical Manual of Mental Disorders}, Fifth Edition (DSM-5) \cite{AmericanPsychiatricAssociation_2013}. This included any clinical diagnosis such as depression, anxiety disorders, psychotic disorders, personality disorders, or neurodevelopmental disorders. Substance-related and addictive disorders, including alcohol misuse disorders, were also encompassed within this category. Where no such diagnosis was recorded, \textit{Psych\_Hx} was marked as negative (N), or \enquote{UK} if data were unavailable.

\subsubsection{Displacement Status \textit{(Is\_Displaced\_Person)}}
Individuals were classified as \textit{Is\_Displaced\_Person = \enquote{Y}} if they met the definition of displaced persons as outlined by the International Organisation for Migration (IOM). This includes individuals who have been forced or obliged to flee or leave their homes or places of habitual residence, particularly as a result of or in order to avoid the effects of armed conflict, situations of generalized violence, violations of human rights, or natural or human-made disasters \cite{Deng_1998}.
Where no such displacement status was recorded, \textit{Is\_Displaced\_Person} was marked as negative (N), or \enquote{UK} if unknown.

\subsubsection{Alcohol Influence \textit{(Under\_Influence\_Alcohol)}}
The variable \textit{Under\_Influence\_Alcohol} was marked as \enquote{Y} if there was documented evidence, clinical suspicion, or patient self-report indicating that the individual was under the influence of alcohol at the time of presentation. This included signs such as slurred speech, impaired coordination, smell of alcohol, or confirmed positive alcohol tests where available. The presence of alcohol use was considered relevant due to its potential influence on clinical presentation, risk behaviours, decision-making capacity, and healthcare outcomes. Where no such indication was present, the variable was marked as ‘N’ or \enquote{UK} if unknown.

\subsubsection{Psychiatric Inpatients \textit{(Is\_Psych\_Inpat)}}
Individuals were classified as psychiatric inpatients if they were admitted to a psychiatric facility, psychiatric ward, or designated mental health unit at the time of data collection or during the relevant clinical encounter. This included both voluntary and involuntary admissions. Classification was based on documentation in medical records or transfer/referral notes. Identifying psychiatric inpatients allowed for analysis of patterns and outcomes specific to individuals receiving inpatient mental health care.

\subsubsection{Severe Disability History \textit{(Severe\_Disability\_Hx)}}
The variable \textit{Severe\_Disability\_Hx} was marked as \enquote{Y} if the individual had a documented history of significant cognitive or functional impairment consistent with severe disability. This was limited to individuals with:

\begin{itemize}
    \item Severe learning disabilities (e.g. profound intellectual disability, global developmental delay), and/or
    \item Impairments of consciousness (e.g. persistent vegetative state, minimally conscious state, or severe acquired brain injury with loss of awareness).
\end{itemize}

This classification excluded milder forms of disability or functional limitation. The variable was marked as \enquote{N}, or \enquote{UK} where no such history was documented.

\subsubsection{History of Previous Ingestions \textit{(Previous\_Ingestions)}}
The variable \textit{Previous\_Ingestions} was marked as \enquote{Y} if there was documented evidence that the individual had a prior episode of foreign body ingestion before the current presentation. This included both intentional and unintentional ingestions, regardless of the time elapsed since the previous event. Documentation could include clinical notes, referral information, or electronic health records. The variable was marked as \enquote{N} where it was explicitly stated that this was the first ingestion, or marked \enquote{UK} if prior history was unknown.

\paragraph{Motivation Variables}

\subsubsection{Motivation - Intent to Harm \textit{(Motivation\_Intent\_To\_Harm)}}
The variable \textit{Motivation\_Intent\_To\_Harm} was marked as \enquote{Y} if there was documented evidence that the ingestion was carried out with the intent to cause self-harm, self-injury, or suicide. This included explicit statements by the individual, clinical impressions recorded by healthcare professionals, or circumstances strongly suggesting deliberate self-injurious behaviour. Ingestions motivated by other factors (e.g. attention-seeking, protest, escape, or psychosis without suicidal intent) were not included in this category. The variable was marked as ‘N’ where motivation was determined to be non-harm-related or marked \enquote{UK} if intent could not be clearly established.

\subsubsection{Motivation - Protest \textit{(Motivation\_Protest)}}
The variable \textit{Motivation\_Protest} was marked as \enquote{Y} if there was documented evidence that the ingestion was carried out as a form of protest, demonstration, or to express objection or dissatisfaction, including cases involving manipulation or attempts to secure betterment of conditions. This included ingestions in response to perceived injustice, detention conditions, delays in asylum processes, or efforts to influence external decision-making. Classification was based on explicit statements by the individual or clinical documentation suggesting protest-related intent. The variable was marked as \enquote{N} where protest was not identified as a motivation, or marked \enquote{UK} if intent was unclear.

\subsubsection{Motivation - Psychiatric \textit{(Motivation\_Psychiatric)}}
The variable \textit{Motivation\_Psychiatric} was marked as \enquote{Y} if the ingestion was considered to be primarily driven by an underlying psychiatric condition. This included cases where ingestion occurred in the context of psychosis, impulsivity related to personality disorder, intellectual disability, severe emotional dysregulation, or other recognised mental health diagnoses. Classification was based on clinical documentation indicating a psychiatric motive or context, even if the individual did not explicitly state intent. The variable was marked as \enquote{N} where no psychiatric motivation was identified, or marked \enquote{UK} if unclear.

\subsubsection{Motivation – Psychosocial \textit{(Motivation\_Psychosocial)}}

The variable \textit{Motivation\_Psychosocial} was developed to classify cases in which ingestion appeared to be motivated by social, emotional, or interpersonal factors—distinct from psychiatric illness, medical indications, or criminal intent. Prior to the introduction of this category, 'Motivation Other' accounted for 34.5\% of cases, making it the joint most frequently recorded motivation.

Psychosocial motivations are understood as behaviours shaped by the interaction between psychological states and social environments. These include influences such as interpersonal relationships, cultural norms, identity, and social roles. In the literature, psychosocial factors are defined as “the interrelation of social factors and individual thought and behaviour” \cite{Martikainen_2002}.

This classification emerged following a manual review of the \textit{Motivation\_Other\_Long} field in the dataset. Relevant entries were first identified using basic keyword filtering in Python, and then refined through manual reading and interpretation of the free-text responses in the Excel spreadsheet.

Examples of motivations reclassified under this category include:
\begin{itemize}
    \item Imitative or performance behaviours, such as jugglers, sword swallowers, or individuals imitating these acts.
    \item Statements such as \enquote{cleaning oesophagus with a wire,} interpreted as symbolic or performative acts.
    \item Ingestions intended to provoke a reaction, such as \enquote{in front of husband for shock value} or \enquote{to impress friends.}
    \item Ingestions related to body image or weight concerns, e.g., \enquote{for dietary purposes} or \enquote{to induce weight loss.}
    \item Ingestions carried out \enquote{by a refugee for safe keeping,} indicating a socially-motivated protective act.
\end{itemize}

This reclassification enabled a more nuanced understanding of ingestion behaviours that fall outside conventional psychiatric, medical, or criminal frameworks, yet reflect significant psychosocial drivers.

\subsubsection{Motivation - Unknown \textit{(Motivation\_Unknown)}}
The variable \textit{Motivation\_Unknown} was marked as \enquote{Y} when no clear motivation for the ingestion could be identified from available documentation. This included cases where the individual did not disclose a reason, was unable to communicate, or where clinical notes did not specify a suspected or confirmed motive. The variable was marked as \enquote{N} when a specific motivation was documented, or marked \enquote{UK} if documentation was incomplete or ambiguous.

\subsubsection{Motivation Count \textit{(Motivation\_Count)}}
This variable was computed using Python. Motivation-related variables marked as \enquote{Y} were converted to binary values (1), while \enquote{N} and \enquote{UK} were treated as 0. The sum of these values across all motivation columns was used to calculate a total Motivation Count for each individual. This measure was used to explore potential associations between the number of identified motivations and clinical outcomes.

\paragraph{Object Variables}

\subsubsection{Object - Button Batteries \textit{(Object\_Button\_Battery)}}
This variable was marked as \enquote{Y} if the ingested object was identified as a button battery. Classification was based on clinical documentation, radiological findings, or patient report. The variable was marked as \enquote{N} when a button battery was not ingested, or marked \enquote{UK} if object type was not recorded.

\subsubsection{Object - Magnets \textit{(Object\_Magnet)}}
This variable was marked as \enquote{Y} if the ingested object was a magnet or included magnets. Special consideration was given to cases involving multiple magnets due to elevated clinical risk. Classification was based on clinical records, imaging, or patient report. The variable was marked as \enquote{N} if no magnets were ingested, or marked \enquote{UK} if unknown.

\subsubsection{Object - Long (\(>5\text{ cm}\)) \textit{(Object\_Long)}}
This variable was marked as \enquote{Y} if the ingested object exceeded 5 cm in length, consistent with standard clinical thresholds for increased risk of obstruction or complications. Length was determined based on documentation, radiology, or object description. The variable was marked as \enquote{N} for shorter objects, or marked \enquote{UK} if object dimensions were not available.

\subsubsection{Object – Long and Sharp \textit{(Object\_Long\_Sharp)}}
The variable Object\_Long\_Sharp was marked as \enquote{Y} if both Object\_Long and Object\_Sharp were marked as \enquote{Y}, indicating that the ingested object was both longer than 5 cm and sharp or pointed. This classification was generated programmatically in Python by identifying cases where both conditions were true. The variable was marked as \enquote{N} if either Object\_Long or Object\_Sharp was \enquote{N}, and marked as \enquote{UK} if either contributing variable was \enquote{UK}.

\subsubsection{Object - Short (\(<5\text{ cm}\)) \textit{(Object\_Short)}}

The variable \textit{Object\_Short} was created to classify ingested objects that were less than 5 cm in length. This threshold aligns with clinical guidelines indicating that objects shorter than 5 cm are generally considered lower risk for gastrointestinal obstruction or complication, particularly in adults.

The classification was derived programmatically using Python and the Pandas library. A value of \enquote{Y} was assigned if the corresponding \textit{Object\_Long} variable was marked as \enquote{N}, indicating the object was not considered long. 

Objects with unknown dimensions were retained as \enquote{UK} in the \textit{Object\_Long} variable and not reassigned in this field, thus preserving uncertainty in classification.

\subsubsection{Object – Short Sharp (\(<5\text{ cm}\) and Sharp) \textit{(Object\_Short\_Sharp)}}

The variable \textit{Object\_Short\_Sharp} was created to identify cases in which the ingested object was both less than 5 cm in length and sharp. This classification was developed based on the recognition that small sharp objects—although not exceeding the standard 5 cm threshold for increased obstruction risk—may still present significant clinical hazards, such as mucosal injury or perforation.

To derive this variable, a two-step classification was applied using Python and the Pandas library \cite{ThePandasDevelopmentTeam_2020}. 

Then, \textit{Object\_Short\_Sharp} was assigned a value of \enquote{Y} if both \textit{Object\_Short} and \textit{Object\_Sharp} were marked \enquote{Y}. The variable was marked \enquote{N} otherwise, and \enquote{UK} if either component variable was unknown.

This classification allows for targeted analysis of cases involving small but potentially dangerous objects, which may warrant different management strategies than either blunt or long foreign bodies.

\subsubsection{Object - Sharp \textit{(Object\_Sharp)}}
This variable was marked as \enquote{Y} if the ingested object was described as sharp, pointed, or capable of causing mucosal injury or perforation. Examples included razor blades, nails, glass, and needles. Classification was based on object description or radiological appearance. The variable was marked as \enquote{N} if no sharp object was ingested, or marked \enquote{UK} if object type was unclear.

\subsubsection{Object - Multiple \textit{(Object\_Multiple)}}
This variable was marked as \enquote{Y} if the individual ingested more than one object during the same episode. This included ingestion of identical or different objects. Classification was based on clinical notes, imaging, or patient report. The variable was marked as \enquote{N} for single-object ingestion, or marked \enquote{UK} if number of objects was not specified.

\subsubsection{Object - Diameter (\(>2,5\text{ cm}\)) \textit{(Object\_Diameter\_Large)}}
This variable was marked as \enquote{Y} if the individual ingested an object with a diameter \(>2,5\text{ cm}\). The variable was marked as \enquote{N} for objects with a diameter \(<2,5\text{ cm}\), or marked \enquote{UK} if the diameter was not specified and could not be calculated.

\subsubsection{Object - Free Text \textit{(Object\_Other\_Long)}}
Additional information on objects ingested was recorded as free text in the \textit{Object\_Other\_Long} column. This can be interogated at a later date. 

\paragraph{Outcome Variables}

\subsubsection{Outcome - Endoscopy \textit{(Outcome\_Endoscopy)}}
The variable \textit{Outcome\_Endoscopy} was marked as \enquote{Y} if the individual underwent endoscopic intervention during the clinical episode. Endoscopy was defined as a "minimally invasive medical procedure involving the insertion of a flexible tube equipped with a light and camera (an endoscope) into the body to visually examine internal organs or tissues". This included both diagnostic and therapeutic endoscopic procedures related to the ingestion. The variable was marked as \enquote{N} if no endoscopy was performed, or \enquote{UK} if this information was unavailable.

\subsubsection{Outcome – Surgery \textit{(Outcome\_Surgery)}}
The variable \textit{Outcome\_Surgery} was marked as \enquote{Y} if the individual required surgical intervention as a result of the ingestion. Surgery was defined as any operative procedure performed under general or regional anaesthesia in a theatre setting, intended to retrieve the ingested object or to treat complications arising from the ingestion (e.g., perforation, obstruction, haemorrhage). The variable was marked as \enquote{N} if no surgery was performed, or marked \enquote{UK} if not documented.

\subsubsection{Outcome – Endoscopy and Surgery \textit{(Outcome\_Endoscopy\_Surgery)}}
The variable \textit{Outcome\_Endoscopy\_Surgery} was marked as \enquote{Y} if both \textit{Outcome\_Endoscopy} and \textit{Outcome\_Surgery} were recorded as \enquote{Y} for a given individual, indicating that the patient underwent both endoscopic and surgical interventions during the clinical episode. This composite variable was computed using Python and reflects the occurrence of both procedures, regardless of their order or whether one was diagnostic and the other therapeutic. The variable was marked as \enquote{N} if one or neither procedure was performed, and \enquote{UK} if data were insufficient to determine this.

\subsubsection*{Outcome – Conservative \textit{(Outcome\_Conservative)}}
The variable \textit{Outcome\_Conservative} was created to identify cases managed without procedural intervention. A value of \enquote{Y} was assigned if the individual did not undergo either surgery or endoscopy during the clinical episode. Conversely, a value of \enquote{N} was assigned if either surgical or endoscopic intervention was documented.

This variable was derived programmatically using Python and the Pandas library \cite{ThePandasDevelopmentTeam_2020}. The following logic was applied: if both Outcome\_Surgery and Outcome\_Endoscopy were not equal to \enquote{Y}, the case was considered to have been managed conservatively.

This allowed for consistent identification of conservative management across the dataset, based on the absence of documented procedural intervention.

\subsubsection{Outcome – Death \textit{(Outcome\_Death)}}
The variable \textit{Outcome\_Death} was marked as \enquote{Y} if the ingestion was temporally or causally associated with death due to direct medical complications (e.g., perforation, sepsis, aspiration). Deaths attributable solely to comorbid psychiatric conditions or suicide, where the ingestion was not directly responsible, were excluded. The variable was marked as \enquote{N} if the individual survived, or marked \enquote{UK} if outcome was unknown.

\subsubsection{Outcome – Perforation \textit{(Outcome\_Perforation)}}
The variable \textit{Outcome\_Perforation} was marked as \enquote{Y} if there was clinical or radiological evidence of gastrointestinal or airway perforation resulting from the ingestion. This included any confirmed full-thickness breach of the gastrointestinal tract, oesophagus, or other affected structures. The variable was marked as \enquote{N} if perforation was ruled out or absent, or marked \enquote{UK} if unknown.

\subsubsection{Outcome – Obstruction \textit{(Outcome\_Obstruction)}}
The variable \textit{Outcome\_Obstruction} was marked as \enquote{Y} if the ingestion led to a confirmed or clinically suspected obstruction of the gastrointestinal tract. Diagnosis was based on clinical assessment, imaging, or procedural findings. The variable was marked as \enquote{N} if no obstruction occurred, or marked \enquote{UK} if not documented.

\subsubsection{Outcome – Injury Requiring Intervention \textit{(Outcome\_Injury\_Needing\_Intervention)}}

The variable \textit{Outcome\_Injury\_Needing\_Intervention} was marked as \enquote{Y} if there was clinical evidence that the ingestion caused an internal injury significant enough to require medical or procedural intervention, and this injury contributed to the clinical decision to proceed with endoscopy or surgery. This classification was used to support assessment of whether invasive intervention was necessary, rather than to catalogue all injuries. The variable was marked as \enquote{N} if no such injury was identified, or marked \enquote{UK} if data were unavailable.

\subsubsection{Outcome – Other \textit{(Outcome\_Other)}}
The variable \textit{Outcome\_Other} was marked as \enquote{Y} if the ingestion led to a clinically significant outcome not covered by the other defined outcome variables. Examples included aspiration pneumonitis, sepsis without perforation, prolonged hospitalisation due to psychiatric sequelae, or other medical complications directly linked to the ingestion. The variable was marked as \enquote{N} if no such outcome occurred, or marked \enquote{UK} if data were insufficient. Free text outcomes were then recorded in \textit{Outcome\_Other\_Long} to elaborate on other outcomes. 