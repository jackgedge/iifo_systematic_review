\section{Background}


Psychosis, Malingering, Pica, Personality Disorder are considered the main psychiatric conditions that manifest intentional ingestion of foreign objects \cite{Poynter_2011,Gitlin_2007}. 

Malingering can take many forms. Much IIFO data come from prison populations where transfer from prisoner is a primary goal \cite{Poynter_2011,Gitlin_2007,Losanoff_1997e}. Generally, a brief medical intervention to manage the ingestion, with minimal reinforcement of the behavior, followed by an immediate discharge back to prison is the best management \cite{Blaho_1998}.

Patients can report anxiety, restlessness prior to ingestion, then a sense of relief afterwards with OCD \cite{Poynter_2011}

The scant literature on IIFO in people with personality disorder is summarised by Gitlin et. al \cite{Gitlin_2007}, however none of these studies focussed on IIFO. Typical of borderline pathology, it often involves perceived abandonment or rejection. One should not presume that the act was a suicide attempt. Rather, it may have functioned as an affect regulator and a (albeit self-damaging) means of seeking help \cite{Poynter_2011}

Swallowing behavior in and of itself is not necessarily of suicidal intent \cite{Poynter_2011}.

