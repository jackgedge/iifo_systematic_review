\section*{Abstract}
\subsection*{Background}
Intentional ingestion of foreign objects (IIFO) is a clinically distinct form of self-harm, yet its outcome profile and drivers of morbidity remain poorly defined. Whether patient motivation modifies clinical outcomes has never been systematically evaluated.

\subsection*{Objectives}
To synthesise all available evidence on outcomes following intentional foreign-object ingestion, and to determine whether patient motivation, object characteristics, or demographic factors influence the need for intervention, risk of complications, or mortality.

\subsection*{Methods}
A comprehensive search of PubMed, Embase, CENTRAL, Web of Science, Scopus, PsycINFO, and Google Scholar (1\textsuperscript{st} January 1906--31\textsuperscript{st} March 2025) identified studies reporting non-accidental ingestion of true foreign bodies in humans. 

\paragraph*{Inclusion} Human studies of any design, any age, reporting intentional ingestion of non-digestible foreign objects.  
\paragraph*{Exclusion} Accidental or substance ingestion, animal studies, non-English full texts, pre-1906 reports, and studies lacking motivation, object, or outcome data.  
\paragraph*{Outcomes} Endoscopy, surgery, conservative management, complications, mortality.


Case reports (\emph{n} = 72; 72 patients) and case series (\emph{n} = 3; 90 patients) were extracted using a pre-specified codebook. Bias was appraised using JBI checklists with logical filters. Odds ratios and 95\% confidence intervals were calculated from 2 x 2 tables and logistic regression models. Series-level data, including flattened case reports, were analysed using random-effects meta-analysis (REML with Hartung--Knapp adjustment); DerSimonian--Laird estimates were computed for comparison. Meta-regression assessed potential effect modifiers.

\subsection*{Results}
Across 162 individuals, pooled outcome rates were:
endoscopy 45.1\% (95\% CI: 0.2\%--99.7\%), moderate heterogeneity (I\textsuperscript{2} = 37.2);
surgery 28.5\% (95\% CI: 6.5\%--69.6\%), low heterogeneity (I\textsuperscript{2} = 17.6);
mortality 3.4\% (95\% CI: 1.1\%--10.1\%), low heterogeneity (I\textsuperscript{2} = 0.0);
complications 34.8\% (95\% CI: 7.5\%--78.0\%), substantial heterogeneity (I\textsuperscript{2} = 50.1);
conservative 50.0\% (95\% CI: 0.7\%--99.3\%), moderate heterogeneity (I\textsuperscript{2} = 32.3).
\\
Univariate logistic regression of published case reports identified significant associations:
age 18--25 was associated with reduced odds of endoscopy (OR = 0.29, 95\% CI [0.08, 0.98], \emph{p} = 0.046);
age 41--60 predicted reduced surgery likelihood (OR = 0.18, 95\% CI [0.04, 0.76], \emph{p} = 0.020);
intent to harm quintupled the odds of surgery (OR = 5.77, 95\% CI [1.51, 22.03], \emph{p} = 0.010); 
other motivation reduced surgical odds by 86\% (OR = 0.14, 95\% CI [0,03, 0.75], \emph{p} = 0.021);
sharp objects were less likely to undergo endoscopy (OR = 0.34, 95\% CI [0.13, 0.90], \emph{p} = 0.048). 
The sample size was too small for multivariate logistic regression and predictions were too unstable to report.

Series-level meta-regression did not replicate these associations, likely due to small sample sizes and homogeneity in those cohorts (predominantly detained males), but identified significant associations:
female gender was associated with increased odds of surgery (OR = 1.05, 95\% CI [1.00, 1.10], \emph{p} = 0.044);
positive psychiatric inpatient status was associated with increased odds of surgery (OR = 1.30, 95\% CI [1.01, 1.92], \emph{p} = 0.044);
psychosocial motivation was associated with increased odds of surgery (OR = 1.08, 95\% CI [1.00, 1.17], \emph{p} = 0.044);
other motivation increased surgical odds (OR = 1.16, 95\% CI [1.01, 1.34], \emph{p} = 0.044)


\subsection*{Limitations}
Single-reviewer screening and extraction; risk of publication bias favouring complex or dramatic cases; small sample sizes; possible overfitting in regression models.

\subsection*{Conclusions}
Patient motivation is an important and previously under-recognised modifier of clinical outcomes following IIFO. Specific motives (e.g., intent to harm) and object types (e.g., sharp, long objects) are associated with more invasive management. Demographic factors, particularly gender and psychiatric status, may confound these associations. While intention appears to influence outcome, findings require cautious interpretation and underscore the need for improved motivational reporting and larger, prospective studies.

\subsection*{Registration and Data}
Protocol not registered. Full dataset and analysis scripts are openly available at \url{https://github.com/jackgedge/iifo_systematic_review}.
