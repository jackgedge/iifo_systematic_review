\section{Results}

\subsection{Age and Gender}

\begin{table}[H]
\centering
\caption{Summary statistics for age in years.}
\label{tab:age-summary}
\begin{tabular}{lr}
\toprule
Statistic & Age (Years) \\
\midrule
Count & 119.0 \\
Mean & 30.5 \\
Std & 11.1 \\
Min & 4.0 \\
25\% & 23.0 \\
50\% & 29.0 \\
75\% & 36.0 \\
Max & 62.0 \\
\bottomrule
\end{tabular}

\end{table}
\FloatBarrier


\begin{figure}[H]
    \centering
    \includegraphics[width=0.8\textwidth]{figures/kde_age_plot.png}
    \caption{Kernel density estimate (KDE) plot showing the distribution of age among included cases.}
    \label{fig:age-kde}
\end{figure}


\begin{table}[H]
\centering
\caption{Counts and percentages of 1s in Gender Column (sorted descending)}
\label{tab:gender-summary}
\begin{tabular}{lrr}
\toprule
Gender & Count (N) & Percentage (\%) \\
\midrule
Male & 119 & 68.0 \\
Female & 48 & 27.4 \\
Unknown & 8 & 4.6 \\
\bottomrule
\end{tabular}

\end{table}
\FloatBarrier




\FloatBarrier

\subsection{Subgroup Data}
\begin{table}[H]
        \centering
        \caption{Counts and percentages of 1s for population summary (sorted descending by percentage).}
        \begin{tabular}{lrrr}
\toprule
 & Count (1s) & Valid N & Percentage (\%) \\
\midrule
Psych Hx & 57 & 90 & 63.3 \\
Previous Ingestions & 36 & 90 & 40.0 \\
Is Prisoner & 25 & 90 & 27.8 \\
Is Psych Inpat & 8 & 90 & 8.9 \\
Severe Disability Hx & 5 & 90 & 5.6 \\
Under Influence Alcohol & 3 & 90 & 3.3 \\
Is Displaced Person & 2 & 90 & 2.2 \\
\bottomrule
\end{tabular}
        \label{tab:population_summary}
        \end{table}
        \FloatBarrier

\begin{table}[H]
\centering
\begin{table}
\caption{Summary of motivations for ingestion.}
\label{tab:motivation-summary}
\begin{tabular}{llrr}
\toprule
Motivation & Motivation\_Long & Count & Percentage \\
\midrule
Motivation\_Psychiatric & Psychiatric & 34.0 & 35.4 \\
Motivation\_Intent\_To\_Harm & Self-Harm & 20.0 & 20.8 \\
Motivation\_Psychosocial & Psychosocial & 17.0 & 17.7 \\
Motivation\_Protest & Protest & 9.0 & 9.4 \\
Motivation\_Other & Other & 9.0 & 9.4 \\
Motivation\_Other\_Psych\_Hx & Other (with Psychiatric History) & 5.0 & 5.2 \\
Motivation\_Other\_Severe\_Disability\_Hx & Other (with Severe Disability History) & 2.0 & 2.1 \\
\bottomrule
\end{tabular}
\end{table}

\end{table}
\FloatBarrier

\begin{table}[H]
        \centering
        \caption{Counts and percentages of 1s for object summary (sorted descending by percentage).}
        \begin{tabular}{lrrr}
\toprule
 & Count (1s) & Valid N & Percentage (\%) \\
\midrule
Object Diameter Large & 48 & 60 & 80.0 \\
Object Long & 29 & 60 & 48.3 \\
Object Short & 28 & 60 & 46.7 \\
Object Sharp & 25 & 60 & 41.7 \\
Object Multiple & 24 & 60 & 40.0 \\
Object Short Sharp & 14 & 60 & 23.3 \\
Object Long Sharp & 10 & 60 & 16.7 \\
Object Magnet & 3 & 60 & 5.0 \\
Object Button Battery & 0 & 60 & 0.0 \\
\bottomrule
\end{tabular}
        \label{tab:object_summary}
        \end{table}
        \FloatBarrier
\begin{table}[H]
        \centering
        \caption{Counts and percentages of 1s for outcome summary (sorted descending by percentage).}
        \begin{tabular}{lrrr}
\toprule
 & Count (1s) & Valid N & Percentage (\%) \\
\midrule
Outcome Surgery & 72 & 117 & 61.5 \\
Outcome Injury Needing Intervention & 64 & 117 & 54.7 \\
Outcome Endoscopy & 45 & 116 & 38.8 \\
Outcome Other & 45 & 117 & 38.5 \\
Outcome Perforation & 34 & 117 & 29.1 \\
Outcome Obstruction & 17 & 117 & 14.5 \\
Outcome Conservative & 11 & 117 & 9.4 \\
Outcome Endoscopy Surgery & 11 & 117 & 9.4 \\
Outcome Death & 6 & 117 & 5.1 \\
\bottomrule
\end{tabular}
        \label{tab:outcome_summary}
        \end{table}
        \FloatBarrier

From Table~\ref{tab:population_summary}, we observe that 39.7\% of the population had a psychiatric history, 20.7\% were prisoners, and 19.0\% had a history of previous ingestions. Smaller proportions were psychiatric inpatients (6.9\%), had a history of severe disability (5.2\%), were displaced persons (3.4\%), or were under the influence of alcohol (3.4\%) at the time of presentation.

From Table~\ref{tab:motivation_summary}, the most frequently recorded motivations were unknown (34.5\%), intent to harm (32.8\%), and psychiatric reasons (31.0\%). Psychosocial motivations accounted for 20.7\% of cases, followed by protest-related ingestion (13.8\%), and motivations classified as 'other' (12.1\%). Additional subcategories included 'Other Psych Hx' (6.9\%).

From Table~\ref{tab:object_summary}, the most commonly ingested object types were long objects (50\%), sharp objects (41.4\%), and multiple objects (39.7\%). Less frequently, ingestions involved long sharp objects (17.2\%), magnets (5.2\%), while no cases involved button batteries (0.0\%).

From Table~\ref{tab:outcome_summary}, surgical intervention was the most common outcome (72.4\%), followed by injuries requiring intervention (53.4\%) and perforation (36.2\%). Endoscopy was performed in 26.3\% of cases. Other outcomes included miscellaneous findings (22.4\%), conservative management (10.3\%), death (8.6\%), and obstruction (8.6\%).


\begin{table}[H]
\centering
\renewcommand{\arraystretch}{1.2}
\caption{List of comments made during data extraction in Motivation\_Other\_Long where Motivation\_Other marked 'Y'.}
\label{tab:motivation-other-summary}
\begin{tabular}{p{3cm}p{12cm}}
\toprule
Patient\ ID & Free Text Motivation\\
\midrule
3-012 & Delirium Tremens \\
3-014 & Cleaning Oesophagus with sponge on wire \\
113-001 & "Reported that thermometer ingestion and suicide attempts were her best chance to obtain narcotic medications. History of narcotic use since teens, multiple admissions and surgeries related to similar behavior." \\
168-001 & "Foreign body ingested in police custody to conceal object, likely related to narcotics smuggling. Lighter was double-wrapped in cellophane and retained for 17 months without prior symptoms." \\
217-001 & Smuggling \\
238-001 & "A 29-year-old mentally retarded female patient was admitted to the emergency service with a two-day history of abdominal pain, nausea, vomiting, and failure to eliminate feces or pass gas. Patient history revealed that the patient had undergone two surgeries due to repeated foreign body ingestion within the last 6 months." \\
260-001 & "history of cerebral palsy and self-destructive behaviour" \\
328-001 & "PICA", "anxiety and an empty prescription for alprazolam as the primary trigger leading to the ingestions.", "past psychiatric history included major depressive disorder, generalized anxiety disorder, posttraumatic stress disorder, borderline personality disorder, and pica with a history of more than twenty admissions for ingestion behaviors often requiring endoscopic retrieval" \\
349-001 & "adjustment disorder who developed a gastrocolic fistula following the deliberate ingestion of multiple magnets.", "Although she acknowledged this was a wrong and harmful act, she refused to explain her action." \\
\bottomrule
\end{tabular}
\end{table}
\FloatBarrier


\input{t}


\begin{table}[H]
\centering
\renewcommand{\arraystretch}{1.2}
\caption{List of comments made during data extraction where Outcome\_Other marked 'Y'.}
\label{tab:outcome-other-summary}
\begin{tabular}{p{3cm}p{12cm}}
\toprule
Patient\ ID & Free Text \\
\midrule
3-008 & "stomach and abdominal surfaces everywhere adherent". \\
3-009 & ""Fifteen months later an abscess appeared in the right hypochondrium. Two months later a large amount of pus was discharged through the mouth. Une month later opening of the abscess outwardly. After repeated closing and breaking out of the wound, the sharp end of the fork appeared. For eleven months the condition continued the same, without the fork protruding any farther. Then for the first time the opening of the abscess was enlarged, and the fork removed. Convulsions followed, with trifling haemorrhage. For thirty hours' food was discharged through the wound. Fifteen days later it was healed." \\
3-012 & "small fluctuating spot with a fistula, secreting fetid pus, in which the metallic body could be felt.", "caries of the sternum and ribs. Autopsy showed complete adhesion of the stomach wound with the abdominal parietes.2" \\
39-001 & "Fluroscopy" \\
61-001 & Interintestinal adhesions \\
85-001 & "lodged near the ileocecal valve and an inflammatory mass had formed around the intraluminal coin, causing a 10 \_x0001\_ 7 cm fibrous tumor to completely obstruct the small bowel" \\
99-001 & "Subsequent post-mortem confirmed a mediastinal abscess and aortoesophageal fistula but no significant clot in the gastrointestinal tract" \\
113-001 & "Thermometer in Mediastinum" \\
148-001 & "a median sternotomy was performed and pericardiotomy evacuated 150 ml of blood. A needle found embedded within the left ventricle was surgically removed", "n the small bowel, with no peritonism clinically. An exploratory laparotomy was performed six days post-sternotomy and the needle, which was embedded within the pylorus removed through an enterotomy. No diaphragmatic or hollow viscous injuries were noted intraoperatively. The patient recovered uneventfully." \\
168-001 & "presence of an ulcer in the angular part of the stomach (Forrest III) as well as a large foreign body wrapped in dark-colored cellophane in the middle of the stomach." \\
214-001 & Severe antral erosion is observed, likely related to pressure effect of the magnet clump, but there was no sign of perforation \\
261-001 & "complicated by respiratory failure requiring continued ventilator support for 6 days. A postoperative contrast esophagram showed contrast extravasation at the esophagotomy site. It was managed with antibiotics and continued chest tube drainage. Seven days after the operation, CT of the chest showed fluid and gas collections in the paraesophageal and right lower hemithorax that were not being drained by the chest tubes. Thoracentesis confirmed empyema with pleural fl uid growingActinomyces meyeri and Streptococcus mitis. Antibiotics were switched to piperacillin-tazobactam and fluconazole and given for 14 days.", "nd delirium, which were treated with antipsychotics. A follow-up chest x-ray 25 days after the operation showed resolution of empyema. A repeat esophagram at 29 days showed no leak. Th e patient was discharged to a nursing home 53 days after admission" \\
273-001 & "pneumoperitoneum and free intra-abdominal fluid" \\
300-001 & "early mucosal appendicitis" \\
322-001 & "gastro-oesophagoscopy showed oedema and a haematoma of the left piriform sinus. The oesophagus showed no signs of perforation. Due to the presence of laryngeal oedema and haematoma the patient was left intubated and admitted to the intensive care unit.", "Two days later, a follow-up CT scan still showed a small foreign body at the posterior wall of hypopharynx (Figure 3). A second endoscopic procedure was required to remove it", "After removal, the oedema reduced and the patient could be extubated safely and was transferred" \\
327-001 & "perforated sigmoid colon secondary to a linear metallic foreign body or 2 adjacent foreign bodies with surrounding inflammatory changes and a 3.5-cm abscess", "transrectal 10F pigtail drain placed by interventional radiology to drain the pelvic abscess. Under general anesthesia and fluoroscopy support, flexible sigmoidoscopy was performed." \\
349-001 & "gastrocolic fistula" \\
360-001 & "During the upper endoscopy, the razor blade was detected in the antrum and was embedded in the mucosa" \\
369-001 & Retroperitoneal Abscess "onsidering the severe adherence of viscera on the right side of the abdomen, this area was inspected, and a retroperitoneal abscess was observed posterior to the ascending colon on the psoas muscle", "The posterior wall of the ascending colon was macerated to the presence of an adjacent abscess, and a small perforation was present" \\
370-001 & "hypodermic needle FB that transmigrated to the left atrium and presented as a left atrial mass \\
373-001 & "thermometer was outside the esophageal wall and was located in the left tracheoesophageal groove and wrapped in the fibrous membrane. The foreign body was removed successfully (Figure 3). The incision primarily healed and he was discharged after 5 d with no complications." \\
380-001 & "fluoroscopy", "transferred to the psychiatric hospital on day three postoperative because of the high risk of suicidal ideation and poor insight" \\
382-001 & "severe ulceration." \\
386-001 & "foreign body ingestion mimicking IBD" \\
399-001 & Septic shock, "On opening the thoracic cavity, approximately 300 ml of dark red effusion was observed in both thoracic cavities. Additionally, 60 ml of dark red effusion was present in the pericardium. On opening the abdominal cavity, approximately 400 ml of dark red effusion was present in the abdominal cavity, and a long hard FB was located in the duodenal area. The visceral surface of the liver adhered tightly to the duodenum. In the duodenum, an 11.9 cm long hard toothbrush handle was detected, with one end of the toothbrush perforating the duodenum and penetrating the right lobe of the liver (Figure 1 ). The liver was incised along the toothbrush, and the depth of the liver penetration was approximately 6.5 cm, with a large abscess present on the right lobe of the liver. The total handle was 14.5 cm long, only 11.9 cm of it protruded into the duodenum (Figure 2 ). The diameter of the duodenum rupture was approximately 1.0 cm. No other FBs or perforations were documented in the digestive tract" \\
405-001 & "emergent tracheostomy under local anesthesia" \\
406-001 & "medical admission was complicated by the repeated swallowing of mask wires and popsicle sticks, minimal oral intake and repeat vaginal packing all with a constant companion present" \\
409-001 & "n episode of temporary laryngospasm that required intubation and monitoring in the intensive care unit post operatively \\
416-001 & "foreign metallic body (sewing needle, Fig. 5) was seen on the pancreas surface and was removed gently and there was no organ damage of perforation during the operation" \\
421-002 & "lodged in the distal duodenum" \\
421-004 & "CT examination disclosed an entire toothbrush that had been entrapped in the ascending colon (Fig. 5b). It was removed by performing laparoscopic surgery." \\
431-001 & "chronically embedded with ulceration and unable to be removed endoscopically." \\
443-002 & "sigmoid rupture and damage to the left ovary", "patient underwent a laparotomy not only to remove the knife but also to repair the left ovary and sigmoid" \\
451-001 & "fistula" \\
460-001 & Conservative management with H. pylori induced gastroduodenitis diagnosed by endoscopy. \\
465-001 & partial oesophageal obstruction, "impaction of two screws", "Bedside laryngoscopy demonstrated two screws in the vallecula" \\
465-003 & Partial oesophageal obstruction ""Neck radiographs showed 4.5 cm linear overlapping radiodensities (razorblades) at the level of the hyoid bone (A and B" \\
465-004 & "extensive subcutaneous soft tissue emphysema in the neck. The patient underwent laryngoscopy under general anesthesia, which was negative for a foreign body in the neck. The patient was subsequently managed conservatively with imaging follow-up" \\
465-005 & "a soda can pull-tab in the vallecula" \\
465-007 & "successful C4 through T2 laminectomy for epidural abscess evacuation" \\
471-001 & "Postoperatively, the patient remained in the intensivecare unit (ICU) due to difficult weaning from the ventilator and the presence of acute kidney injury requiring multiple renal replacement therapies. She was extubated successfully and transferred to the ward." \\
475-001 & "Pharyngeal perforation" and "Retro-esophageal foreign body (Fig. 2A) that had entirely crossed the posterior pharyngeal wall, accounting for it not being seen on initial endoscopy" \\
476-001 & "escaped from the surgical ward without completing the follow-up and treatment course" \\
482-001 & "Following surgery, the patient was transferred to the intensive care unit (ICU)" \\
484-001 & "The patient presented a good clinical evolution, maintaining suicidal ideas and with a high risk of escape, so contacted her psychiatric center where she was referred on the third postoperative day" \\
499-001 & "passed the remaining 37 coins in the stool." \\
504-002 & "complicated by a faecal fistula." \\
504-004 & "Following an unsuccessful attempt at endoscopic removal, the pins were removed by gastrotomy which was complicated by a subphrenic abscess." \\
504-009 & "mediastinal abscesses which were the result of an oesophageal tear at the level of the azygos arch and which required a thoracotomy for drainage." \\
\bottomrule
\end{tabular}
\end{table}
\FloatBarrier


We can expand on the 'Outcome\_Other' by examining the comments made during Data Extraction. This is shown in Table~\ref{tab:outcome-other-summary}.

\FloatBarrier

\subsection{Correlations}

\begin{figure}[H]
    \centering
    \includegraphics[width=0.8\textwidth]{figures/correlation_matrix.png}
    \caption{Correlation matrix showing correlation between variables.}
    \label{fig:correlation-matrix}
\end{figure}

\FloatBarrier


\begin{table}[H]
\centering
\caption{Top 30 strongest pairwise Pearson correlations between variables}
\label{tab:top-correlations}
\begin{tabular}{llr}
\toprule
Variable A & Variable B & Correlation \\
\midrule
Object Long & Object Short & -0.916000 \\
Motivation Other & Motivation Other Psych Hx & 0.760000 \\
Severe Disability Hx & Motivation Other Severe Disability Hx & 0.624000 \\
Object Short & Object Short Sharp & 0.617000 \\
Object Sharp & Object Short Sharp & 0.616000 \\
Object Long & Object Diameter Large & 0.601000 \\
Psych Hx & Motivation Psychiatric & 0.597000 \\
Outcome Endoscopy & Outcome Surgery & -0.595000 \\
Outcome Injury Needing Intervention & Outcome Perforation & 0.591000 \\
Motivation Other Psych Hx & Motivation Other Severe Disability Hx & 0.567000 \\
Object Long & Object Short Sharp & -0.564000 \\
Object Long & Object Long Sharp & 0.550000 \\
Object Diameter Large & Object Short & -0.505000 \\
Object Long Sharp & Object Short & -0.497000 \\
Object Sharp & Object Long Sharp & 0.496000 \\
Outcome Surgery & Outcome Injury Needing Intervention & 0.496000 \\
Object Multiple & Object Short & 0.486000 \\
Object Diameter Large & Object Short Sharp & -0.475000 \\
Object Multiple & Object Short Sharp & 0.458000 \\
Object Long & Object Multiple & -0.450000 \\
Motivation Other & Motivation Other Severe Disability Hx & 0.431000 \\
Outcome Surgery & Outcome Perforation & 0.425000 \\
Outcome Surgery & Outcome Conservative & -0.414000 \\
Outcome Endoscopy & Outcome Endoscopy Surgery & 0.413000 \\
Outcome Endoscopy & Outcome Perforation & -0.392000 \\
Psych Hx & Previous Ingestions & 0.365000 \\
Psych Hx & Motivation Psychosocial & -0.364000 \\
Motivation Psychosocial & Motivation Unknown & -0.354000 \\
Motivation Intent To Harm & Motivation Unknown & -0.353000 \\
Previous Ingestions & Outcome Surgery & -0.342000 \\
\bottomrule
\end{tabular}

\end{table}
\FloatBarrier


\subsubsection*{Interpretation}

From Table~\ref{tab:top-correlations} several strong correlations were observed between key variables:

\begin{itemize}
    \item \textbf{Psychiatric history} showed a strong positive correlation with \textbf{psychiatric motivation} ($r = 0.83$), as expected.
    \item \textbf{Injury requiring intervention} was strongly associated with \textbf{perforation} ($r = 0.70$), aligning with clinical severity.
    \item \textbf{Being a prisoner} was notably linked to \textbf{protest-related motivation} ($r = 0.54$).
    \item \textbf{Surgical intervention} correlated positively with both \textbf{injury needing intervention} ($r = 0.51$) and \textbf{perforation} ($r = 0.47$), while showing negative associations with \textbf{conservative management} ($r = -0.55$) and \textbf{psychiatric motivation} ($r = -0.42$).
    \item \textbf{Endoscopy} was negatively associated with both \textbf{sharp object ingestion} ($r = -0.43$) and \textbf{perforation} ($r = -0.37$).
    \item \textbf{Conservative management} negatively correlated with \textbf{injury needing intervention} ($r = -0.36$) but showed a positive association with \textbf{previous ingestions} ($r = 0.41$).
    \item \textbf{Psychiatric motivation} also correlated positively with \textbf{multiple object ingestion} ($r = 0.37$), possibly reflecting impulsivity or complexity in intent.
\end{itemize}

\subsection{Aggregate Data}

These columns represent aggregate data computed using Python and the \texttt{pandas} library. They summarise key demographic, clinical, motivational, object-related, and outcome variables—such as age ranges, gender distribution, psychiatric history, ingestion motivations, and clinical outcomes like endoscopy or surgery rates. Each variable is captured in terms of both absolute case counts (e.g., \textit{Psych\_Hx\_Cases}) and proportions relative to the total number of cases (e.g., \textit{Psych\_Hx\_Rate}). 

This structure enables case data from individual reports and small case series to be standardised into a common format. As described earlier in the report, this method of generating aggregate-level data is designed to support the integration of studies with differing data structures, particularly larger studies where individual-level data may not be available. By reducing heterogeneity in data format and content, these aggregated variables facilitate comparative analysis across diverse sources, strengthening the overall synthesis in case series and systematic reviews.

Columns are listed below and can be generated computationally where individual case data is available. The Python-based workflow allows for flexible and scalable creation of additional \texttt{\_Cases} and \texttt{\_Rate} columns for any new variables of interest, enabling consistent integration of diverse data sources into a unified structure.

\subsubsection*{List of Aggregate Data Columns (Excluding \texttt{\_Long} Fields)}

\begin{itemize}
    \item Total\_Cases
    \item Age\_Low
    \item Age\_High
    \item Age\_Mean
    \item Age\_Median
    \item Gender\_Male\_Cases
    \item Gender\_Male\_Rate
    \item Gender\_Female\_Cases
    \item Gender\_Female\_Rate
    \item Gender\_Unknown\_Cases
    \item Gender\_Unknown\_Rate
    \item Is\_Prisoner\_Cases
    \item Is\_Prisoner\_Rate
    \item Is\_Psych\_Inpat\_Cases
    \item Is\_Psych\_Inpat\_Rate
    \item Is\_Displaced\_Person\_Cases
    \item Is\_Displaced\_Person\_Rate
    \item Under\_Influence\_Alcohol\_Cases
    \item Under\_Influence\_Alcohol\_Rate
    \item Psych\_Hx\_Cases
    \item Psych\_Hx\_Rate
    \item Severe\_Disability\_Hx\_Cases
    \item Severe\_Disability\_Hx\_Rate
    \item Previous\_Ingestions\_Cases
    \item Previous\_Ingestions\_Rate
    \item Motivation\_Intent\_To\_Harm\_Cases
    \item Motivation\_Intent\_To\_Harm\_Rate
    \item Motivation\_Protest\_Cases
    \item Motivation\_Protest\_Rate
    \item Motivation\_Psychiatric\_Cases
    \item Motivation\_Psychiatric\_Rate
    \item Motivation\_Psychosocial\_Cases
    \item Motivation\_Psychosocial\_Rate
    \item Motivation\_Unknown\_Cases
    \item Motivation\_Unknown\_Rate
    \item Motivation\_Other\_Cases
    \item Motivation\_Other\_Rate
    \item Motivation\_Other\_Psych\_Hx\_Cases
    \item Motivation\_Other\_Psych\_Hx\_Rate
    \item Motivation\_Other\_Severe\_Disability\_Hx\_Cases
    \item Motivation\_Other\_Severe\_Disability\_Hx\_Rate
    \item Object\_Button\_Battery\_Cases
    \item Object\_Button\_Battery\_Rate
    \item Object\_Magnet\_Cases
    \item Object\_Magnet\_Rate
    \item Object\_Long\_Cases
    \item Object\_Long\_Rate
    \item Object\_Sharp\_Cases
    \item Object\_Sharp\_Rate
    \item Object\_Multiple\_Cases
    \item Object\_Multiple\_Rate
    \item Object\_Long\_Sharp\_Cases
    \item Object\_Long\_Sharp\_Rate
    \item Object\_Short\_Cases
    \item Object\_Short\_Rate
    \item Object\_Short\_Sharp\_Cases
    \item Object\_Short\_Sharp\_Rate
    \item Outcome\_Endoscopy\_Cases
    \item Outcome\_Endoscopy\_Rate
    \item Outcome\_Surgery\_Cases
    \item Outcome\_Surgery\_Rate
    \item Outcome\_Death\_Cases
    \item Outcome\_Death\_Rate
    \item Outcome\_Injury\_Needing\_Intervention\_Cases
    \item Outcome\_Injury\_Needing\_Intervention\_Rate
    \item Outcome\_Perforation\_Cases
    \item Outcome\_Perforation\_Rate
    \item Outcome\_Obstruction\_Cases
    \item Outcome\_Obstruction\_Rate
    \item Outcome\_Other\_Cases
    \item Outcome\_Other\_Rate
    \item Outcome\_Conservative\_Cases
    \item Outcome\_Conservative\_Rate
    \item Outcome\_Endoscopy\_Surgery\_Cases
    \item Outcome\_Endoscopy\_Surgery\_Rate
\end{itemize}
