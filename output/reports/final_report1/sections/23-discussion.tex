\subsection*{Interpretation of Results}

\subsubsection*{Strengths}
To the author's knowledge, this is the first systematic review of intentional ingestion of foreign objects (IIFO) that focuses explicitly on patient motivation as a central analytic lens. It aggregates evidence from 162 ingestion cases with documented motivations, drawing on both case series and detailed case reports.

A major methodological strength of this review lies in its strict inclusion criteria. Only cases with clearly stated motivations were included, thereby isolating intentional ingestion as an independent clinical phenomenon. By excluding unclear or assumed motivations, this review avoids conflating intentional and accidental ingestion, which has historically limited clarity in the literature.

The review includes a broad age range (7–100 years) and applies no geographic or demographic exclusions, enhancing generalisability. A comprehensive search strategy—covering multiple databases, reference lists, and grey literature—reduced the risk of publication bias from indexing alone.

Meta-analyses used random-effects models with restricted maximum likelihood (REML) estimation and Hartung–Knapp (HK) adjustments, robust methods suited to small sample sizes and substantial between-study heterogeneity. Outcome rates are substantially higher than those previously reported in general ingestion cohorts~\cite{Ikenberry_2011, Birk_2016}, underscoring the distinct clinical profile of IIFO.

Taken together, the findings suggest that \emph{intentionality alone} is a key driver of morbidity and intervention in ingestion cases.

\subsubsection*{Associations Between Motivation and Outcome}

While motivation–outcome associations were not statistically significant in series-level meta-regression (which skewed heavily male and detained), they became clearer in case-level association testing, which included more women, psychiatric cases, and varied object types.

In this broader context: intent to harm was associated with a 5.8-fold increase in the odds of surgery; other motivations were associated with an 86\% reduction in surgical likelihood; psychosocial motivation was associated with increased odds of surgery (OR = 1.08, \emph{p} = 0.044), though the clinical relevance of this small effect is unclear.

The type of motivation also influenced demographic and behavioural profiles. For example, psychosocial cases were less likely to involve protest or intent to harm, and included unique ingestion scenarios (e.g., for weight loss or games) with serious clinical consequences. In contrast, those with intent-to-harm motivation were more often male, ingested sharp or large-diameter objects, and had higher surgery rates.

These findings support the hypothesis that motivation shapes not only intent but also behaviour (e.g., object choice) and outcome.

\subsubsection*{Subgroup Findings}

Some demographic groups showed notable outcome trends:

Females were more likely to have psychiatric motivation, a psychiatric history, and to avoid protest- or harm-based motivations. They were also less likely to be detained. They had higher odds of undergoing surgery (OR = 1.05, \emph{p} = 0.044), potentially mediated by psychiatric comorbidities or object characteristics. 

Patients aged 41–60 had significantly reduced odds of surgery (OR = 0.18, \emph{p} = 0.020). This group tended to ingest fewer high-risk objects and included more patients with prior psychiatric care and repeat ingestions.

Sharp object ingestion was associated with reduced odds of endoscopy (OR = 0.34, \emph{p} = 0.029), possibly because of contraindications for retrieval or complications requiring surgery.

Meta-regression also suggested reduced odds of death among displaced persons (OR = 0.22, \emph{p} = 0.034) and those with severe disabilities, although small numbers and case report reliance suggest caution in interpreting these effects. For example, the two displaced person cases involved very different scenarios—a protest ingestion vs money concealment—demonstrating how motivation and object jointly shape outcomes.

\subsubsection*{Methodological Caveats and Classification Bias}

Motivations were author-defined and assigned post-hoc. While necessary due to inconsistent reporting, this may introduce classification bias. Motivational states such as protest intent to harm often overlap and are rarely well-documented. Notably, psychiatric inpatients were underrepresented in the protest group, and psychosocial motivations were often ambiguous.

\subsubsection*{Statistical and Structural Limitations}

Regression modelling on small series numbers risks overfitting. While REML was chosen over DerSimonian–Laird to mitigate this, confidence intervals remained wide, and many estimates were unstable. Some analyses would have benefited from collapsing object types into fewer, risk-stratified categories based on known clinical severity.

In addition, psychiatric history was coded as a binary variable, obscuring diagnostic nuance. Future studies should distinguish between diagnoses (e.g., psychosis vs personality disorders), which may have differing associations with IIFO and clinical outcomes.

Finally, this review did not examine object location or differing conservative management strategies—two critical clinical determinants. Nor did it differentiate between ingestion as self-harm vs suicidality, a gap that future qualitative work might help to address.

\subsubsection*{Takeaway Points}

Motivation is both a psychological construct and a clinical signal. While this review shows that motivation influences behaviour and likely outcome, the field lacks standardised reporting, theoretical frameworks, and quantitative synthesis.

Until better data are available, clinicians must adopt a multidisciplinary lens: understanding that IIFO, particularly in vulnerable populations, reflects complex interactions between psychiatric illness, systemic factors (e.g., detention or displacement), and individual agency.

Improving reporting of motivation, psychiatric background, object characteristics, and outcomes in future publications will enable stronger evidence synthesis and support more tailored intervention strategies for this challenging clinical presentation.
