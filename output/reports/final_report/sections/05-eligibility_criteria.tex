
\subsection*{Eligibility Criteria}

\textbf{Population, or participants and conditions of interest:}
\begin{itemize}
  \item Any human.
  \item Any age group.
\end{itemize}

\textbf{Interventions or exposures:}
\begin{itemize}
  \item Humans that have:
    \begin{itemize}
      \item Intentionally
      \item Ingested a foreign object through the oral cavity (mouth).
    \end{itemize}
\end{itemize}

\textbf{Comparisons or control group:}
\begin{itemize}
  \item \textbf{Motivation/reason for ingestion:} protest, suicidal intent, self-harm, psychiatric and other documented motivations.
  \item \textbf{Intervention details:} number of ingestions, management strategies (Conservative, Endoscopic, Surgical), time to intervention.
  \item \textbf{Object characteristics:} multiple objects, blunt objects, sharp-pointed objects, long objects (>6cm), short objects (<6cm), food bolus.
  \item \textbf{Radiological findings:} location of foreign object (e.g., oesophagus, stomach, ileo-caecal region).
  \item \textbf{Setting/location.}
\end{itemize}

\textbf{Outcomes of interest:}
\begin{itemize}
  \item \textbf{Endoscopic intervention:} undergoing a minimally invasive medical procedure using an endoscope (Goyal \& Neumann 2020).
  \item \textbf{Surgical intervention:} any operative procedure involving an incision to retrieve or manage complications from foreign objects (Smith et al. 2019).
  \item \textbf{Conservative management:} any case not undergoing endoscopic or surgical intervention.
  \item Complication rates.
  \item Mortality rates.
\end{itemize}

\textbf{Setting:} Any setting.

\textbf{Study designs:}
\begin{itemize}
  \item Observational studies (cohort, case-control, cross-sectional).
  \item Case series.
  \item Clinical trials.
  \item Case reports.
\end{itemize}

\subsection*{Exclusion Criteria}

\begin{table}[ht]
\centering
\caption{Exclusion Criteria}
\label{tab:exclusion}
\renewcommand{\arraystretch}{1.2}
\begin{tabular}{|c|>{\RaggedRight\arraybackslash}p{0.85\linewidth}|}
\hline
\textbf{\#} & \textbf{Exclusion Criterion} \\
\hline
1 & Full text not available in English. \\
\hline
2 & Studies not focusing on intentional self-ingestion (into the gastrointestinal tract) of foreign object via the oral cavity (mouth) or where unclear if ingested. \\
\hline
3 & Studies focussing solely on accidental ingestion. \\
\hline
4 & Non-Human/ animal studies. \\
\hline
5 & Reviews, editorials, commentaries, and opinion pieces without original empirical data. \\
\hline
6 & Duplicate publications or studies with overlapping data sets (the most comprehensive or recent study will be included). \\
\hline
7 & Studies focusing on ingestion or co-ingestion of substances (e.g. poisons, medications) rather than physical foreign objects. \\
\hline
8 & Ingestions undertaken in controlled environment as part of voluntary study. \\
\hline
9 & Ingestions not explicitly stated to be intentional and history not suggestive of deliberate ingestion (i.e. Age < 8, no history of previous ingestions, no psychiatric co-morbidities, not a prisoner/detainee/vulnerable group). \\
\hline
10 & Does not meet inclusion criteria. \\
\hline
11 & Ingestions where death resulted from other means (i.e. suicide) \\
\hline
12 & Studies before the advent of Endoscopy (1906) \\
\hline
\end{tabular}
\end{table}
