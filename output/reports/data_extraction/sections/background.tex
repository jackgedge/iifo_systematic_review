\section{Background}

In 1635 Daniel Schwaban performed a gastrotomy on a man who swallowed a knife \cite{Moehlau_1895}. Gorsauld in 1738 was apparently the first surgeon to perform cervical esophagotomy for removal of a foreign body (FB) \cite{Saint_1929}/ In 1906, Jose Goyanes carried out an extraction of a coin impacted in the esophagus by means of a rigid esophagoscope \cite{Barros_1991}. Early in the 20th century, with the development of the rigid esophagoscope, the first large scale FB extraction was reported \cite{Lerche_1991} and other series describing technical aspects appeared in the literature \cite{Jackson_1957}.

The most impressive FBI case was reported by Chalk in a psychiatric patient who over time swallowed 2,533 foreign bodies accounting for 21268 grams in weight \cite{Chalk_1928}. The largest object reported was 28 cm in length \cite{Ricote_1985}

Psychosis, Malingering, Pica, Personality Disorder are considered the main psychiatric conditions that manifest intentional ingestion of foreign objects \cite{Poynter_2011,Gitlin_2007}. 

The term `pica' was coined by Pare from the Latin word \enquote{magpie}, meaning unusual appetites \cite{Kaplan_2009}. It has been reported on all continents \cite{McLoughlin_1987}.

Malingering can take many forms. Much IIFO data come from prison populations where transfer from prisoner is a primary goal \cite{Poynter_2011,Gitlin_2007,Losanoff_1997e}. Generally, a brief medical intervention to manage the ingestion, with minimal reinforcement of the behavior, followed by an immediate discharge back to prison is the best management \cite{Blaho_1998}.

Patients can report anxiety, restlessness prior to ingestion, then a sense of relief afterwards with OCD \cite{Poynter_2011}

The scant literature on IIFO in people with personality disorder is summarised by Gitlin et. al \cite{Gitlin_2007}, however none of these studies focussed on IIFO. Typical of borderline pathology, it often involves perceived abandonment or rejection. One should not presume that the act was a suicide attempt. Rather, it may have functioned as an affect regulator and a (albeit self-damaging) means of seeking help \cite{Poynter_2011}

Swallowing behavior in and of itself is not necessarily of suicidal intent \cite{Poynter_2011}.

The following psychiatric diagnoses have been associated with recurrent FBIs: Pica, personality disorder (noted to be associated with the highest risk patients), impulse control disorder, OCD, autism spectrum disorder, factitious disorder, Munchausen syndrome, intellectual disability, psychosis, and malingering \cite{Guinan_2019e,Gitlin_2007}.

In extreme cases of repeated IIFO, some authors view the condition as a condition that could warrant a palliative approach. Essentially, a palliative approach would be guided by an awareness that the patient has a limited functional prognosis and lifespan as a result of a severe treatment-resistant mental illness. Multiple medical and surgical interventions that may also contribute to further deterioration of her physical health with the introduction of complications. A palliative perspective may help to establish a baseline and guide both short-term and long-term management treatment-resistant repeated IIFO presents at the hospital \cite{Jaini_2023}.

Pica is seen most frequently in patients with borderline personality disorder, and it has been proposed that it is a form of affect regulation\cite{Guinan_2019e}

Possible reasons patients may perform this self-injurious behavior include immediate relief from psychiatric symptoms, punishment of self and/or others, or command auditory hallucinations \cite{Tromans_2019}.

Previously examined factors that influence the need and timing of endoscopic intervention for FBI include the patient’s age, comorbid conditions, the characteristics of the ingested item (size, shape, content, anatomic location), and longer duration since the time of ingestion.\cite{Ikenberry_2011}

Prisoners do not only swallow objects for secondary gain. Karp et al. reported that "The most common motive for swallowing was suicidal ideation with command hallucinations, reported by ten patients (10/19). Other patients' motives were recorded as suicidal ideation without command hallucinations (2/19), command hallucinations without suicidal ideation (2/19), depression with a desire to harm but not kill themselves (2/19), and manipulation of the medicolegal system (3/19). \cite{Karp_1991}

In prison populations, on multivariate analyses, hospital admission was associated with number of items ingested (OR 1.3, P < 0.05). The need for endoscopy was independently associated with ingestion of multiple objects (OR 3.3, P < 0.05). Surgical therapy was significantly associated with increasing number of ingested items (OR 1.07 per item, P < 0.05). Endoscopy is associated with significantly lower odds of surgery (OR 0.13, P < 0.01) \cite{Dalal_2013}. 

One UK acute NHS Trust reported increased incidence of IIFO during the COVID-19 pandemic\cite{Jones_2023d}

