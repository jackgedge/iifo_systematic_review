\section{Abstract}

\subsection*{Background}
Intentional ingestion of foreign objects (IIFO) is a clinically distinct form of self-harm, yet its outcome profile and drivers of morbidity remain poorly defined. Whether patient motivation modifies clinical outcomes has never been systematically evaluated.

\subsection*{Objectives}
To synthesise all available evidence on outcomes following intentional foreign-object ingestion, and to determine whether patient motivation, object characteristics, or demographic factors influence the need for intervention, risk of complications, or mortality.

\subsection*{Methods}
A comprehensive search of PubMed, Embase, CENTRAL, Web of Science, Scopus, PsycINFO, and Google Scholar (1\textsuperscript{st} January 1906--31\textsuperscript{st} March 2025) identified studies reporting non-accidental ingestion of true foreign bodies in humans. 

Human studies of any design, any age, reporting intentional ingestion of non-digestible foreign objects that reported outcomes of endoscopy, surgery, conservative management, complications and mortality were included. Accidental or substance ingestion, animal studies, non-English full texts, pre-1906 reports, and studies lacking motivation, object, or outcome data were excluded. 
