\subsection*{Study Selection}

A total of 673 records were identified through initial database searches: PubMed (317), WoS (277), Embase (25), SCOPUS (24), PsycINFO (16), and Cochrane (14).
 
Following the removal of duplicate records—based on combinations of publication year, title, author, and DOI—313 duplicates were excluded. This left 360 unique database records for screening: PubMed (258), Web of Science (65), Cochrane (14), SCOPUS (12), Embase (9), PsycINFO (2).
A Google Scholar search yielded 135 results.
3 duplicates were removed manually. Thus, 132 records proceeded to screening.
Database records (360) and Google Scholar records (132) were then merged, yielding a total of 492 records.

Title and abstract review was then undertaken. JGE reviewed all 492 records. A random sample of 49 records was generated for independent screening MS.
After title and abstract screening, Cohen's Kappa was calculated for inter-reviewer agreement between JGE and MS, yielding a value of 0.38, indicating fair agreement.
Where JGE and MS disagreed, 16 records were reviewed by GC.
75 studies were excluded as they were studies not focusing on intentional self-ingestion (into the gastrointestinal tract) of foreign object via the oral cavity (mouth) or where unclear if ingested. 30 studies were excluded as they were non-human/ animal studies. 27 studies were excluded as they were reviews, editorials, commentaries, and opinion pieces without original empirical data. 18 studies were excluded as they were studies focussing solely on accidental ingestion. 16 studies were excluded as they were studies focusing on ingestion or co-ingestion of substances (e.g. poisons, medications) rather than physical foreign objects. 9 studies were excluded as they were not available in english. 1 study was excluded as it studied ingestions undertaken in controlled environment as part of voluntary study. In total, 176 records were excluded, leaving 316 for full text review.

During full text review, JGE reviewed all 316 records. 
A random sample of 32 records was generated for independent review by MS. 
Inter-reviewer agreement was calculated using Cohen's Kappa, yielding a value of 0.45, indicating fair agreement.   
Where JGE and MS disagreed, 5 records were reviewed by GC.   
94 records were excluded as they were ingestions not explicitly stated to be intentional and history not suggestive of deliberate ingestion (i.e. age < 8, no history of previous ingestions, no psychiatric co-morbidities, not a prisoner/detainee/vulnerable group). 46 records were excluded as they were reviews, editorials, commentaries, and opinion pieces without original empirical data. 24 records were excluded as they were full text not available in english. 19 records were excluded as they were studies not focusing on intentional self-ingestion (into the gastrointestinal tract) of foreign object via the oral cavity (mouth) or where unclear if ingested. 11 records were excluded as they were studies focussing solely on accidental ingestion. 9 records were excluded as they were ingestions where death resulted from other means (i.e. suicide) 7 records were excluded as they were studies focusing on ingestion or co-ingestion of substances (e.g. poisons, medications) rather than physical foreign objects. 6 records were excluded as they were duplicate publications or studies with overlapping data sets (the most comprehensive or recent study will be included). 6 records were excluded as they were does not meet inclusion criteria. 1 records were excluded as they were non-human/ animal studies. 1 records were excluded as they were studies before the advent of endoscopy (1906) In total, 224 records were excluded during full text review. 92 records were included and proceeded to bibliography search.  

The bibliographies of the 92 included from full text review were searched by JGE manually. A list of included papers were collated using Python Pandas \cite{ThePandasDevelopmentTeam_2020}, ensuring each included item had its bibliography searched. Relevant bibliography items were identified; compared to the eligibility criteria; and collated in Zotero \cite{Stillman_2025}. The bibliography search results were then exported from Zotero as a CSV and input into Pandas for analysis, manipulation and duplicate removal.

In total, 204 records were identified during bibliography searching.
Using \textit{Python Pandas}, bibliography search records were then programatically compared to title and abstract screen and full text review records. In this process, 12 duplicates were identified.
194 full text bibliography search records were reviewed by JGE. 121 bibliography search records were excluded, leaving 73 for inclusion. 

Therefore, a total of 165 records were included in this study and proceeded to bias assessment. This process is illustrated in Figure~\ref{fig:prisma}.
