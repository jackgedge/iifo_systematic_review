\section{Results}

\subsection{Age and Gender}

\begin{table}[H]
\centering
\caption{Summary statistics for age in years.}
\label{tab:age-summary}
\begin{tabular}{lr}
\toprule
Statistic & Age (Years) \\
\midrule
Count & 119.0 \\
Mean & 30.5 \\
Std & 11.1 \\
Min & 4.0 \\
25\% & 23.0 \\
50\% & 29.0 \\
75\% & 36.0 \\
Max & 62.0 \\
\bottomrule
\end{tabular}

\end{table}
\FloatBarrier


\begin{figure}[H]
    \centering
    \includegraphics[width=0.8\textwidth]{figures/kde_age_plot.png}
    \caption{Kernel density estimate (KDE) plot showing the distribution of age among included cases.}
    \label{fig:age-kde}
\end{figure}


\begin{table}[H]
\centering
\caption{Counts and percentages of 1s in Gender Column (sorted descending)}
\label{tab:gender-summary}
\begin{tabular}{lrr}
\toprule
Gender & Count (N) & Percentage (\%) \\
\midrule
Male & 119 & 68.0 \\
Female & 48 & 27.4 \\
Unknown & 8 & 4.6 \\
\bottomrule
\end{tabular}

\end{table}
\FloatBarrier




\FloatBarrier
\newpage

\subsection{Subgroup Data}
\begin{table}[H]
        \centering
        \caption{Counts and percentages of 1s for population summary (sorted descending by percentage).}
        \begin{tabular}{lrrr}
\toprule
 & Count (1s) & Valid N & Percentage (\%) \\
\midrule
Psych Hx & 57 & 90 & 63.3 \\
Previous Ingestions & 36 & 90 & 40.0 \\
Is Prisoner & 25 & 90 & 27.8 \\
Is Psych Inpat & 8 & 90 & 8.9 \\
Severe Disability Hx & 5 & 90 & 5.6 \\
Under Influence Alcohol & 3 & 90 & 3.3 \\
Is Displaced Person & 2 & 90 & 2.2 \\
\bottomrule
\end{tabular}
        \label{tab:population_summary}
        \end{table}
        \FloatBarrier

\begin{table}[H]
\centering
\begin{table}
\caption{Summary of motivations for ingestion.}
\label{tab:motivation-summary}
\begin{tabular}{llrr}
\toprule
Motivation & Motivation\_Long & Count & Percentage \\
\midrule
Motivation\_Psychiatric & Psychiatric & 34.0 & 35.4 \\
Motivation\_Intent\_To\_Harm & Self-Harm & 20.0 & 20.8 \\
Motivation\_Psychosocial & Psychosocial & 17.0 & 17.7 \\
Motivation\_Protest & Protest & 9.0 & 9.4 \\
Motivation\_Other & Other & 9.0 & 9.4 \\
Motivation\_Other\_Psych\_Hx & Other (with Psychiatric History) & 5.0 & 5.2 \\
Motivation\_Other\_Severe\_Disability\_Hx & Other (with Severe Disability History) & 2.0 & 2.1 \\
\bottomrule
\end{tabular}
\end{table}

\end{table}
\FloatBarrier

\begin{table}[H]
        \centering
        \caption{Counts and percentages of 1s for object summary (sorted descending by percentage).}
        \begin{tabular}{lrrr}
\toprule
 & Count (1s) & Valid N & Percentage (\%) \\
\midrule
Object Diameter Large & 48 & 60 & 80.0 \\
Object Long & 29 & 60 & 48.3 \\
Object Short & 28 & 60 & 46.7 \\
Object Sharp & 25 & 60 & 41.7 \\
Object Multiple & 24 & 60 & 40.0 \\
Object Short Sharp & 14 & 60 & 23.3 \\
Object Long Sharp & 10 & 60 & 16.7 \\
Object Magnet & 3 & 60 & 5.0 \\
Object Button Battery & 0 & 60 & 0.0 \\
\bottomrule
\end{tabular}
        \label{tab:object_summary}
        \end{table}
        \FloatBarrier
\begin{table}[H]
        \centering
        \caption{Counts and percentages of 1s for outcome summary (sorted descending by percentage).}
        \begin{tabular}{lrrr}
\toprule
 & Count (1s) & Valid N & Percentage (\%) \\
\midrule
Outcome Surgery & 72 & 117 & 61.5 \\
Outcome Injury Needing Intervention & 64 & 117 & 54.7 \\
Outcome Endoscopy & 45 & 116 & 38.8 \\
Outcome Other & 45 & 117 & 38.5 \\
Outcome Perforation & 34 & 117 & 29.1 \\
Outcome Obstruction & 17 & 117 & 14.5 \\
Outcome Conservative & 11 & 117 & 9.4 \\
Outcome Endoscopy Surgery & 11 & 117 & 9.4 \\
Outcome Death & 6 & 117 & 5.1 \\
\bottomrule
\end{tabular}
        \label{tab:outcome_summary}
        \end{table}
        \FloatBarrier

From Table~\ref{tab:population_summary}, we observe that 39.7\% of the population had a psychiatric history, 20.7\% were prisoners, and 19.0\% had a history of previous ingestions. Smaller proportions were psychiatric inpatients (6.9\%), had a history of severe disability (5.2\%), were displaced persons (3.4\%), or were under the influence of alcohol (3.4\%) at the time of presentation.

From Table~\ref{tab:motivation_summary}, the most frequently recorded motivations were unknown (34.5\%), intent to harm (32.8\%), and psychiatric reasons (31.0\%). Psychosocial motivations accounted for 20.7\% of cases, followed by protest-related ingestion (13.8\%), and motivations classified as 'other' (12.1\%). Additional subcategories included 'Other Psych Hx' (6.9\%).

From Table~\ref{tab:object_summary}, the most commonly ingested object types were long objects (44.8\%), sharp objects (41.4\%), and multiple objects (39.7\%). Less frequently, ingestions involved long sharp objects (17.2\%), magnets (5.2\%), while no cases involved button batteries (0.0\%).

From Table~\ref{tab:outcome_summary}, surgical intervention was the most common outcome (72.4\%), followed by injuries requiring intervention (53.4\%) and perforation (36.2\%). Endoscopy was performed in 26.3\% of cases. Other outcomes included miscellaneous findings (22.4\%), conservative management (10.3\%), death (8.6\%), and obstruction (8.6\%).


\begin{table}[H]
\centering
\renewcommand{\arraystretch}{1.2}
\caption{List of comments made during data extraction in Motivation\_Other\_Long where Motivation\_Other marked 'Y'.}
\label{tab:motivation-other-summary}
\begin{tabular}{p{3cm}p{12cm}}
\toprule
Patient\ ID & Free Text Motivation\\
\midrule
3-012 & Delirium Tremens \\
3-014 & Cleaning Oesophagus with sponge on wire \\
113-001 & "Reported that thermometer ingestion and suicide attempts were her best chance to obtain narcotic medications. History of narcotic use since teens, multiple admissions and surgeries related to similar behavior." \\
168-001 & "Foreign body ingested in police custody to conceal object, likely related to narcotics smuggling. Lighter was double-wrapped in cellophane and retained for 17 months without prior symptoms." \\
217-001 & Smuggling \\
238-001 & "A 29-year-old mentally retarded female patient was admitted to the emergency service with a two-day history of abdominal pain, nausea, vomiting, and failure to eliminate feces or pass gas. Patient history revealed that the patient had undergone two surgeries due to repeated foreign body ingestion within the last 6 months." \\
260-001 & "history of cerebral palsy and self-destructive behaviour" \\
328-001 & "PICA", "anxiety and an empty prescription for alprazolam as the primary trigger leading to the ingestions.", "past psychiatric history included major depressive disorder, generalized anxiety disorder, posttraumatic stress disorder, borderline personality disorder, and pica with a history of more than twenty admissions for ingestion behaviors often requiring endoscopic retrieval" \\
349-001 & "adjustment disorder who developed a gastrocolic fistula following the deliberate ingestion of multiple magnets.", "Although she acknowledged this was a wrong and harmful act, she refused to explain her action." \\
\bottomrule
\end{tabular}
\end{table}
\FloatBarrier



\FloatBarrier
\newpage

\subsection{Correlations}

\begin{figure}[H]
    \centering
    \includegraphics[width=0.8\textwidth]{figures/correlation_matrix.png}
    \caption{Correlation matrix showing correlation between variables.}
    \label{fig:correlation-matrix}
\end{figure}

\FloatBarrier


\begin{table}[H]
\centering
\caption{Top 30 strongest pairwise Pearson correlations between variables}
\label{tab:top-correlations}
\begin{tabular}{llr}
\toprule
Variable A & Variable B & Correlation \\
\midrule
Object Long & Object Short & -0.916000 \\
Motivation Other & Motivation Other Psych Hx & 0.760000 \\
Severe Disability Hx & Motivation Other Severe Disability Hx & 0.624000 \\
Object Short & Object Short Sharp & 0.617000 \\
Object Sharp & Object Short Sharp & 0.616000 \\
Object Long & Object Diameter Large & 0.601000 \\
Psych Hx & Motivation Psychiatric & 0.597000 \\
Outcome Endoscopy & Outcome Surgery & -0.595000 \\
Outcome Injury Needing Intervention & Outcome Perforation & 0.591000 \\
Motivation Other Psych Hx & Motivation Other Severe Disability Hx & 0.567000 \\
Object Long & Object Short Sharp & -0.564000 \\
Object Long & Object Long Sharp & 0.550000 \\
Object Diameter Large & Object Short & -0.505000 \\
Object Long Sharp & Object Short & -0.497000 \\
Object Sharp & Object Long Sharp & 0.496000 \\
Outcome Surgery & Outcome Injury Needing Intervention & 0.496000 \\
Object Multiple & Object Short & 0.486000 \\
Object Diameter Large & Object Short Sharp & -0.475000 \\
Object Multiple & Object Short Sharp & 0.458000 \\
Object Long & Object Multiple & -0.450000 \\
Motivation Other & Motivation Other Severe Disability Hx & 0.431000 \\
Outcome Surgery & Outcome Perforation & 0.425000 \\
Outcome Surgery & Outcome Conservative & -0.414000 \\
Outcome Endoscopy & Outcome Endoscopy Surgery & 0.413000 \\
Outcome Endoscopy & Outcome Perforation & -0.392000 \\
Psych Hx & Previous Ingestions & 0.365000 \\
Psych Hx & Motivation Psychosocial & -0.364000 \\
Motivation Psychosocial & Motivation Unknown & -0.354000 \\
Motivation Intent To Harm & Motivation Unknown & -0.353000 \\
Previous Ingestions & Outcome Surgery & -0.342000 \\
\bottomrule
\end{tabular}

\end{table}
\FloatBarrier


From Table~\ref{tab:top-correlations}, Psychiatric History showed a strong positive correlation with Psychiatric Motivation ($r = 0.83$), as expected. Injury requiring intervention was strongly associated with perforation ($r = 0.70$), consistent with it being the most common complication. Being a prisoner was also notably associated with protest-related motivation ($r = 0.54$). Surgical intervention correlated positively with both injury needing intervention ($r = 0.51$) and perforation ($r = 0.47$), while showing a negative correlation with conservative management ($r = -0.55$) and psychiatric motivation ($r = -0.42$). Endoscopy was negatively correlated with both perforation ($r = -0.37$) and sharp object ingestion ($r = -0.43$). Conservative management showed a negative correlation with injury needing intervention ($r = -0.36$), and a positive correlation with previous ingestions ($r = 0.41$). Additionally, psychiatric motivation was positively correlated with multiple object ingestion ($r = 0.37$).