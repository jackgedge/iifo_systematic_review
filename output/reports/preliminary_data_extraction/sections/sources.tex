\section{Sources}

\subsection{Sample Study Designs}

\begin{figure}[H]
    \centering
    \includegraphics[width=0.8\columnwidth]{figures/study_design_plot.png}
    \caption{Bar plot showing the distribution of study design characteristics where data has been extracted.}
    \label{fig:study-design}
\end{figure}

Figure \ref{fig:study-design} shows that the majority of papers included in this 30-paper sample are case reports. This is likely because extracting granular data from larger case series and retrospective cohort studies is more challenging.

To address this, aggregate data will first be generated from the included case reports and case series before attempting data extraction from larger studies with differing designs. This approach aims to reduce data heterogeneity and enable the aggregate data from larger studies to be more easily analysed alongside the data from case series.

Furthermore, the more detailed data available in case reports and case series is generally preferred over the broader aggregate data from large studies. In some cases, larger studies may include the same case reports or series already included in the review. These larger studies may therefore meet exclusion criteria and be removed before final data extraction begins.

\FloatBarrier

\subsection{Sample Publication/Case Years}

\begin{figure}[H]
    \centering
    \includegraphics[width=0.8\columnwidth]{figures/publications_per_year_plot.png}
    \caption{Histogram showing the distribution of publication dates of papers data collected from.}
    \label{fig:publication-years}
\end{figure}
\FloatBarrier

\FloatBarrier

\begin{figure}[H]
    \centering
    \includegraphics[width=0.8\columnwidth]{figures/cases_per_year_plot.png}
    \caption{Histogram showing the distribution of case dates.}
    \label{fig:case-years}
\end{figure}

Figures \ref{fig:publication-years} and \ref{fig:case-years} show that, while most papers with extractable data were published after 1990, there is one large historical case series from before 1900. This single study may skew the surgical intervention and mortality rates for the overall population.
