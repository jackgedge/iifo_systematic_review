\subsection*{Rationale}
The global displacement crisis has reached unprecedented levels, with over 100 million forcibly displaced individuals reported by the United Nations High Commissioner for Refugees (UNHCR) as of May 2024 \cite{UNHCR_2022}. Refugees and asylum seekers often endure extreme hardships, compelling them to seek asylum in foreign countries \cite{UNHCR_2010, AmnestyInternational_2024}. This vulnerable population frequently faces compounded mental health challenges due to traumatic pre-migration experiences, hazardous journeys, and difficult post-migration realities, including detention and instability of legal status \cite{Athwal_2015, Sundvall_2015, Nickerson_2019, Bevione_2024}.

Self-harm, encompassing various behaviours where individuals inflict harm on themselves, is a particularly alarming manifestation of these mental health challenges. Rates of self-harm are significantly elevated among asylum seekers and refugees compared to general populations, especially among those who are detained, with rates up to 216 times higher in offshore detention facilities than in the general population \cite{vonWerthern_2018, Hedrick_2019, GlobalDetentionProject_2024}.

Methods of suicide and self-harm among refugees differ based on available means, cultural factors, and motivating factors \cite{Ajdacic-Gross_2008}. Common methods include cutting, self-battery, attempted hanging, self-poisoning by medication or chemicals, and intentional ingestion of foreign objects \cite{Hedrick_2019}.

Intentional ingestion of foreign objects (IIFO) is defined as non-accidental ingestion of a true foreign body (non-nutritive items)\cite{Becq_2021}. Most ingested foreign bodies (80--90\%) pass spontaneously, but 10--20\% require endoscopic removal and up to 1\% need surgery. Timely assessment and intervention are critical \cite{Ikenberry_2011, Birk_2016}. In refugee contexts, however, geographic isolation and limited access to advanced care complicate timely management, potentially increasing morbidity and mortality \cite{Vujkovic_2019}.

Globally, rates of IIFO are increasing. In the United States, rates doubled in 2017, with 14\% of cases deemed intentional \cite{Hsieh_2020}. A 2009 review found intentional ingestions in up to 92\% of adults from lower socioeconomic populations, suggesting that rates may be even higher among refugees and asylum seekers \cite{Palta_2009}.

Management of IIFO has been evolving since 1635, when Daniel Schwaban recorded the first gastrotomy on a man who had swallowed a knife \cite{Moehlau_1895}. In 1738, Gorsauld is credited as the first surgeon to perform a cervical esophagotomy for the removal of a foreign body (FB) \cite{Saint_1929}. In 1906, José Goyanes extracted a coin impacted in the esophagus using a rigid esophagoscope for the first time \cite{Barros_1991}. The early 20\textsuperscript{th} century saw the emergence of rigid esophagoscopy as the first large-scale method for foreign body extraction, with further case series detailing technical refinements appearing in the literature \cite{Lerche_1911, Jackson_1957}. Among the most extraordinary documented cases is that by Chalk, who reported a psychiatric patient ingesting 2,533 objects weighing a total of 21,268 grams \cite{Chalk_1928}. The largest single ingested item reported measured 28 cm in length \cite{Ricote_1985}.

Clinical outcomes are influenced by various factors, including patient age, comorbidities, object characteristics (size, shape, composition, anatomical location), and the time elapsed since ingestion and current guidance advises invasive foreign object extraction guidance based on these factors \cite{Ikenberry_2011}.

Literature to date largely focuses on IIFO in prisons or psychiatric contexts, with sparse data from displaced or asylum-seeking populations. In detention, where traditional communication channels are obstructed, ingestion may serve as a form of protest or distress signal \cite{Puggioni_2014}. Conversely, in psychiatric settings, ingestion may reflect mental illness or affective dysregulation \cite{Gitlin_2007, Tromans_2019, Al-Faham_2020b, Pantazopoulos_2022, Aitchison_2024}.

Psychiatric conditions most frequently associated with intentional ingestion of foreign objects (IIFO) include psychosis, malingering, pica, and personality disorders \cite{Poynter_2011, Gitlin_2007}.

Malingering can present in various forms, particularly in prison populations where manipulation to trigger medical transfer is a noted motivation \cite{Poynter_2011, Gitlin_2007, Losanoff_1997e}. In such cases, the optimal management often involves brief medical intervention with minimal reinforcement, followed by prompt return to custody \cite{Blaho_1998}. In contrast, individuals with obsessive-compulsive disorder (OCD) may describe escalating anxiety prior to ingestion followed by a sense of relief afterward \cite{Poynter_2011}.

In cases involving borderline personality disorder, Gitlin et al. \cite{Gitlin_2007} suggest that IIFO may function as an affect regulation strategy, particularly during episodes of perceived abandonment. While such behaviour may appear life-threatening, it should not be presumed to indicate suicidal intent \cite{Poynter_2011}.

In rare and severe cases, some authors have proposed a palliative care approach to repeated IIFO, recognising the limited prognosis associated with treatment-resistant psychiatric illness and the cumulative harms of repeated surgical intervention \cite{Jaini_2023}.

Despite the rising prevalence, the heterogeneity in populations engaging in, and the potential severity of IIFO, there is limited research exploring how motivations for ingestion differ across populations and how these motivations may influence clinical outcomes \cite{Bhugra_2010, Haase_2022, Hedrick_2023a}. Varying motivations likely influence clinical management, including decisions around the need for endoscopic or surgical intervention. For instance, if ingestion is primarily intended as protest, patients may avoid behaviours that risk severe harm, potentially lowering the threshold for conservative management.

This systematic review aimed to address these gaps in the literature by evaluating how motivation for IIFO influences clinical outcomes. Specifically, we aim to examine how different motivations impact rates of endoscopic and surgical interventions, in the hope of informing future clinical strategies and healthcare responses.