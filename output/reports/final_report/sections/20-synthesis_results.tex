\subsection*{Synthesis}

\subsubsection*{Key Associations Between Motivations and Outcomes}

\paragraph*{Psychiatric Motivation} Pooled prevalence was 50.4\% (95\% CI: 40.2--60.6\%). Psychiatric motivation was more common in females (45.7\% vs 32.7\%, $p < 0.001$) and associated with psychiatric inpatient status, which increased the likelihood of surgery (OR~1.40, $p = 0.044$).

\paragraph*{Intent to Harm Motivation} Pooled prevalence was 25.5\% (95\% CI: 10.0--51.6\%). This motivation was more common in psychiatric inpatients (25.0\% vs 23.3\%, $p < 0.001$), but no direct associations with outcomes were observed in meta-regression.

\paragraph*{Psychosocial Motivation} Associated with an increased likelihood of surgery (OR~1.08, $p = 0.044$). Case examples included ingestion for weight loss, dares, or games. Psychosocial motivation was associated with lower intent-to-harm (4.3\% vs 29.9\%, $p < 0.001$).

\paragraph*{Protest Motivation} Pooled prevalence was 60.3\% (95\% CI: 2.0--99.1\%). Protest motivation was less common in females (2.9\% vs 14.5\%, $p = 0.045$) and showed no direct outcome associations.

\paragraph*{Other Motivation} Associated with an increased likelihood of surgery (OR~1.16, $p = 0.044$).


\subsubsection*{Outcome-Specific Findings}

\begin{table}[h!]
\centering
\begin{tabular}{|l|l|p{4cm}|}
\hline
\textbf{Outcome} & \textbf{Pooled Prevalence} & \textbf{Key Influencers} \\
\hline
Conservative & 34.8% (95\% CI: 7.5%–78.0\%) & 
Endoscopy & 45.1\% (95\% CI: 0.2--99.7\%) & Reduced in ages 18--25 (OR~0.29) and with sharp object ingestion (OR~0.34) \\
Surgery & 28.5\% (95\% CI: 6.5--69.6\%) & Increased in females, with psychosocial/other motivation, and psychiatric inpatient status \\
Death & 3.4\% (95\% CI: 1.1--10.1\%) & No motivation-linked associations \\
Complications & 34.8\% (95\% CI: 7.5--78.0\%) & No motivation-linked associations \\
\hline
\end{tabular}
\caption{Summary of pooled outcome prevalence and key influencing factors.}
\end{table}

\subsubsection*{Methodological Notes}
Meta-analyses used restricted maximum likelihood (REML) estimation with Hartung-Knapp adjustments due to small study sizes. Heterogeneity was moderate to high (I$^2$ up to 51.2\%), and insufficient data precluded some subgroup analyses. The findings should be interpreted with caution due to descriptive nature and high heterogeneity.

\subsubsection*{Conclusion}

Psychiatric and psychosocial motivations were most strongly associated with surgical interventions, while protest and intent-to-harm motivations showed no direct links to outcomes. Female gender and psychiatric inpatient status were important confounders. The high heterogeneity across studies limits the strength of conclusions and highlights the need for further research.
\subsection*{Sensitivity Analyses}
Formal sensitivity analyses were not performed due to limited study numbers and heterogeneity; meta-analyses were primarily descriptive, with aggregate case reports included to enhance robustness.