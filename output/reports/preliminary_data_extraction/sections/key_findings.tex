\newpage
\section{Key Findings}

\begin{enumerate}
    \item \textbf{High Rates of Unknown Motivation} \\
    Motivation was recorded as 'Unknown' in 34.5\% of cases, making it the most frequently assigned category. This likely reflects poor reporting of motivational factors in the primary literature. This interpretation is further supported by the fact that, during full-text screening, 80 out of 158 papers were excluded because the ingestion was not described as explicitly intentional.

    \item \textbf{High Surgery Rates} \\
    In the general population, surgery is required in approximately 1\% of foreign body ingestion cases \cite{Birk_2016}. In contrast, our population demonstrates a markedly higher surgery rate of 72.4\%. One likely contributor is the inclusion of a historic case series, which may reflect pre-endoscopy era practices and skew the data. However, the intentional nature of ingestion in this population is also likely to increase surgical interventions—particularly given that 41\% of objects ingested are sharp. The population is predominantly composed of individuals with a psychiatric history, prisoners, and those with prior ingestions, which may further contribute to the high rate.

    \item \textbf{Slightly Higher Endoscopy Rates} \\
    Despite the high rate of surgical intervention and the influence of older data, the endoscopy rate remains above the expected global average of 10–20\% \cite{Birk_2016}. This may reflect the high-risk nature of ingestions (e.g., sharp objects) in this cohort.

    \item \textbf{Low Rates of Conservative Management} \\
    In the general population, 80–99\% of ingested objects pass spontaneously without the need for intervention \cite{Birk_2016}. In our dataset, only 10.3\% of cases were managed conservatively. Correlational analysis did not reveal strong associations with endoscopy, making it difficult to determine the reasons for low conservative management. One possibility is that sharp object ingestion—common in this population—may lead clinicians to opt for emergent endoscopy, in accordance with clinical guidelines (noting that object location was not captured in this dataset).

    \item \textbf{Ambiguity in Documented Motivation} \\
    'Motivation Other' was recorded in 34.5\% of cases, making it the joint most common motivation. This presents an opportunity to further differentiate motivations—for example, by creating subgroups such as 'Motivation Other with Psychiatric History' to identify likely psychiatric drivers. Additionally, the 'Motivation Other Long' field provides free-text reasons for ingestion, which could be explored for richer insights (as shown in Table~\ref{tab:motivation-other-summary}). To address this, a new variable—\textit{Motivation\_Psychosocial}—was created, as described earlier in the text. This field captures underlying psychosocial drivers inferred from free-text entries, providing a basis for more detailed subgroup analysis.

\end{enumerate}