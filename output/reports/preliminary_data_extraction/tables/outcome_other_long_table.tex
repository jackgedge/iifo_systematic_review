
\begin{table}[H]
\centering
\renewcommand{\arraystretch}{1.2}
\caption{List of comments made during data extraction where Outcome\_Other marked 'Y'.}
\label{tab:outcome-other-summary}
\begin{tabular}{p{3cm}p{12cm}}
\toprule
Patient\ ID & Free Text \\
\midrule
3-008 & Silver coffee spoon from a cavalry man. In situ 1 week. "stomach and abdominal surfaces everywhere adherent". \\
3-009 & "swallowed a brass fork with four tines, ten inches long without the handle, which remained in the stomach two and a half years", ""Fifteen months later an abscess appeared in the right hypochondrium. Two months later a large amount of pus was discharged through the mouth. Une month later opening of the abscess outwardly. After repeated closing and breaking out of the wound, the sharp end of the fork appeared. For eleven months the condition continued the same, without the fork protruding any farther. Then for the first time the opening of the abscess was enlarged, and the fork removed. Convulsions followed, with trifling haemorrhage. For thirty hours' food was discharged through the wound. Fifteen days later it was healed." \\
3-012 & "An iron table fork, with wooden handle, twenty-one centimeters long, was swallowed by a peasant with delirium tremens, and remained in his stomach six and a half months.", a small fluctuating spot with a fistula, secreting fetid pus, in which the metallic body could be felt. An incision was made at this point, and the fork extracted. The patient died three months later with caries of the sternum and ribs. Autopsy showed complete adhesion of the stomach wound with the abdominal parietes.2" \\
39-001 & "31 year old severely mentally and physically handicapped man living a in a hospital for the handicapped", "5 bells in the mediastinum and an electric battery in the centre of the abdomen" \\
61-001 & Interintestinal adhesions \\
85-001 & "lodged near the ileocecal valve and an inflammatory mass had formed around the intraluminal coin, causing a 10 \_x0001\_ 7 cm fibrous tumor to completely obstruct the small bowel" \\
99-001 & "Subsequent post-mortem confirmed a mediastinal abscess and aortoesophageal fistula but no significant clot in the gastrointestinal tract" \\
113-001 & "to obtain narcotic medications", "thermometer had moved to the mediastinum intact, and there were no signs of mediastinitis (Fig 2). She had an uneventful recovery, and the next day she was transferred to the psychiatric ward." \\
148-001 & "attempt to impress his friends", "a median sternotomy was performed and pericardiotomy evacuated 150 ml of blood. A needle found embedded within the left ventricle was surgically removed", "n the small bowel, with no peritonism clinically. An exploratory laparotomy was performed six days post-sternotomy and the needle, which was embedded within the pylorus removed through an enterotomy. No diaphragmatic or hollow viscous injuries were noted intraoperatively. The patient recovered uneventfully." \\
168-001 & "presence of an ulcer in the angular part of the stomach (Forrest III) as well as a large foreign body wrapped in dark-colored cellophane in the middle of the stomach." \\
214-001 & "attempting suicide by cutting his wrist. Incidentally, he was found to have ingested multiple magnets which stayed in the stomach", "Severe antral erosion is observed, likely related to pressure effect of the magnet clump, but there was no sign of perforation" \\
261-001 & "a case of a schizophrenic who ingested large pieces of computer circuit boards, which impacted at the mid-esophagus, in the stomach, and in the cecum. Endoscopic removal of the esophageal object was unsuccessful, and the foreign objects were removed by esophagotomy and laparotomy.", "complicated by respiratory failure requiring continued ventilator support for 6 days. A postoperative contrast esophagram showed contrast extravasation at the esophagotomy site. It was managed with antibiotics and continued chest tube drainage. Seven days after the operation, CT of the chest showed fluid and gas collections in the paraesophageal and right lower hemithorax that were not being drained by the chest tubes. Thoracentesis confirmed empyema with pleural fl uid growingActinomyces meyeri and Streptococcus mitis. Antibiotics were switched to piperacillin-tazobactam and fluconazole and given for 14 days.", "nd delirium, which were treated with antipsychotics. A follow-up chest x-ray 25 days after the operation showed resolution of empyema. A repeat esophagram at 29 days showed no leak. Th e patient was discharged to a nursing home 53 days after admission" \\
273-001 & "chronic psychosis and under specific psychotic treatment", "pneumoperitoneum and free intra-abdominal fluid" \\
\bottomrule
\end{tabular}
\end{table}
\FloatBarrier
