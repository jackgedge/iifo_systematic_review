
\begin{table}[H]
\centering
\renewcommand{\arraystretch}{1.2}
\caption{List of comments made during data extraction where Outcome\_Other marked 'Y'.}
\label{tab:outcome-other-summary}
\begin{tabular}{p{3cm}p{12cm}}
\toprule
Patient\ ID & Free Text \\
\midrule
3-008 & "stomach and abdominal surfaces everywhere adherent". \\
3-009 & ""Fifteen months later an abscess appeared in the right hypochondrium. Two months later a large amount of pus was discharged through the mouth. Une month later opening of the abscess outwardly. After repeated closing and breaking out of the wound, the sharp end of the fork appeared. For eleven months the condition continued the same, without the fork protruding any farther. Then for the first time the opening of the abscess was enlarged, and the fork removed. Convulsions followed, with trifling haemorrhage. For thirty hours' food was discharged through the wound. Fifteen days later it was healed." \\
3-012 & "small fluctuating spot with a fistula, secreting fetid pus, in which the metallic body could be felt.", "caries of the sternum and ribs. Autopsy showed complete adhesion of the stomach wound with the abdominal parietes.2" \\
39-001 & "Fluroscopy" \\
61-001 & Interintestinal adhesions \\
85-001 & "lodged near the ileocecal valve and an inflammatory mass had formed around the intraluminal coin, causing a 10 \_x0001\_ 7 cm fibrous tumor to completely obstruct the small bowel" \\
99-001 & "Subsequent post-mortem confirmed a mediastinal abscess and aortoesophageal fistula but no significant clot in the gastrointestinal tract" \\
113-001 & "Thermometer in Mediastinum" \\
148-001 & "a median sternotomy was performed and pericardiotomy evacuated 150 ml of blood. A needle found embedded within the left ventricle was surgically removed", "n the small bowel, with no peritonism clinically. An exploratory laparotomy was performed six days post-sternotomy and the needle, which was embedded within the pylorus removed through an enterotomy. No diaphragmatic or hollow viscous injuries were noted intraoperatively. The patient recovered uneventfully." \\
168-001 & "presence of an ulcer in the angular part of the stomach (Forrest III) as well as a large foreign body wrapped in dark-colored cellophane in the middle of the stomach." \\
214-001 & Severe antral erosion is observed, likely related to pressure effect of the magnet clump, but there was no sign of perforation \\
261-001 & "complicated by respiratory failure requiring continued ventilator support for 6 days. A postoperative contrast esophagram showed contrast extravasation at the esophagotomy site. It was managed with antibiotics and continued chest tube drainage. Seven days after the operation, CT of the chest showed fluid and gas collections in the paraesophageal and right lower hemithorax that were not being drained by the chest tubes. Thoracentesis confirmed empyema with pleural fl uid growingActinomyces meyeri and Streptococcus mitis. Antibiotics were switched to piperacillin-tazobactam and fluconazole and given for 14 days.", "nd delirium, which were treated with antipsychotics. A follow-up chest x-ray 25 days after the operation showed resolution of empyema. A repeat esophagram at 29 days showed no leak. Th e patient was discharged to a nursing home 53 days after admission" \\
273-001 & "pneumoperitoneum and free intra-abdominal fluid" \\
300-001 & "early mucosal appendicitis" \\
322-001 & "gastro-oesophagoscopy showed oedema and a haematoma of the left piriform sinus. The oesophagus showed no signs of perforation. Due to the presence of laryngeal oedema and haematoma the patient was left intubated and admitted to the intensive care unit.", "Two days later, a follow-up CT scan still showed a small foreign body at the posterior wall of hypopharynx (Figure 3). A second endoscopic procedure was required to remove it", "After removal, the oedema reduced and the patient could be extubated safely and was transferred" \\
327-001 & "perforated sigmoid colon secondary to a linear metallic foreign body or 2 adjacent foreign bodies with surrounding inflammatory changes and a 3.5-cm abscess", "transrectal 10F pigtail drain placed by interventional radiology to drain the pelvic abscess. Under general anesthesia and fluoroscopy support, flexible sigmoidoscopy was performed." \\
349-001 & "gastrocolic fistula" \\
360-001 & "During the upper endoscopy, the razor blade was detected in the antrum and was embedded in the mucosa" \\
369-001 & Retroperitoneal Abscess "onsidering the severe adherence of viscera on the right side of the abdomen, this area was inspected, and a retroperitoneal abscess was observed posterior to the ascending colon on the psoas muscle", "The posterior wall of the ascending colon was macerated to the presence of an adjacent abscess, and a small perforation was present" \\
370-001 & "hypodermic needle FB that transmigrated to the left atrium and presented as a left atrial mass \\
373-001 & "thermometer was outside the esophageal wall and was located in the left tracheoesophageal groove and wrapped in the fibrous membrane. The foreign body was removed successfully (Figure 3). The incision primarily healed and he was discharged after 5 d with no complications." \\
380-001 & "fluoroscopy", "transferred to the psychiatric hospital on day three postoperative because of the high risk of suicidal ideation and poor insight" \\
382-001 & "severe ulceration." \\
386-001 & "foreign body ingestion mimicking IBD" \\
399-001 & Septic shock, "On opening the thoracic cavity, approximately 300 ml of dark red effusion was observed in both thoracic cavities. Additionally, 60 ml of dark red effusion was present in the pericardium. On opening the abdominal cavity, approximately 400 ml of dark red effusion was present in the abdominal cavity, and a long hard FB was located in the duodenal area. The visceral surface of the liver adhered tightly to the duodenum. In the duodenum, an 11.9 cm long hard toothbrush handle was detected, with one end of the toothbrush perforating the duodenum and penetrating the right lobe of the liver (Figure 1 ). The liver was incised along the toothbrush, and the depth of the liver penetration was approximately 6.5 cm, with a large abscess present on the right lobe of the liver. The total handle was 14.5 cm long, only 11.9 cm of it protruded into the duodenum (Figure 2 ). The diameter of the duodenum rupture was approximately 1.0 cm. No other FBs or perforations were documented in the digestive tract" \\
405-001 & "emergent tracheostomy under local anesthesia" \\
406-001 & "medical admission was complicated by the repeated swallowing of mask wires and popsicle sticks, minimal oral intake and repeat vaginal packing all with a constant companion present" \\
409-001 & "n episode of temporary laryngospasm that required intubation and monitoring in the intensive care unit post operatively \\
416-001 & "foreign metallic body (sewing needle, Fig. 5) was seen on the pancreas surface and was removed gently and there was no organ damage of perforation during the operation" \\
421-002 & "lodged in the distal duodenum" \\
421-004 & "CT examination disclosed an entire toothbrush that had been entrapped in the ascending colon (Fig. 5b). It was removed by performing laparoscopic surgery." \\
431-001 & "chronically embedded with ulceration and unable to be removed endoscopically." \\
443-002 & "sigmoid rupture and damage to the left ovary", "patient underwent a laparotomy not only to remove the knife but also to repair the left ovary and sigmoid" \\
451-001 & "fistula" \\
460-001 & Conservative management with H. pylori induced gastroduodenitis diagnosed by endoscopy. \\
465-001 & partial oesophageal obstruction, "impaction of two screws", "Bedside laryngoscopy demonstrated two screws in the vallecula" \\
465-003 & Partial oesophageal obstruction ""Neck radiographs showed 4.5 cm linear overlapping radiodensities (razorblades) at the level of the hyoid bone (A and B" \\
465-004 & "extensive subcutaneous soft tissue emphysema in the neck. The patient underwent laryngoscopy under general anesthesia, which was negative for a foreign body in the neck. The patient was subsequently managed conservatively with imaging follow-up" \\
465-005 & "a soda can pull-tab in the vallecula" \\
465-007 & "successful C4 through T2 laminectomy for epidural abscess evacuation" \\
471-001 & "Postoperatively, the patient remained in the intensivecare unit (ICU) due to difficult weaning from the ventilator and the presence of acute kidney injury requiring multiple renal replacement therapies. She was extubated successfully and transferred to the ward." \\
475-001 & "Pharyngeal perforation" and "Retro-esophageal foreign body (Fig. 2A) that had entirely crossed the posterior pharyngeal wall, accounting for it not being seen on initial endoscopy" \\
476-001 & "escaped from the surgical ward without completing the follow-up and treatment course" \\
482-001 & "Following surgery, the patient was transferred to the intensive care unit (ICU)" \\
484-001 & "The patient presented a good clinical evolution, maintaining suicidal ideas and with a high risk of escape, so contacted her psychiatric center where she was referred on the third postoperative day" \\
499-001 & "passed the remaining 37 coins in the stool." \\
504-002 & "complicated by a faecal fistula." \\
504-004 & "Following an unsuccessful attempt at endoscopic removal, the pins were removed by gastrotomy which was complicated by a subphrenic abscess." \\
504-009 & "mediastinal abscesses which were the result of an oesophageal tear at the level of the azygos arch and which required a thoracotomy for drainage." \\
\bottomrule
\end{tabular}
\end{table}
\FloatBarrier
