\section{Synthesis Expanded}
\label{appendix:synthesis_results}
\subsection*{Univariate Association Testing}
A full table of case-level univariate association testing results for is available in Table~\ref{tab:univariate_results}.

In the \textit{Age Group} subgroup, being between 41 and 60 years of age was significantly associated with reduced odds of undergoing surgery (OR = 0.18, 95\% CI [0.04, 0.76], \emph{p} = 0.019). Subgroup analysis (\emph{n} = 61 vs (\emph{n} = 11)) revealed several distinguishing characteristics. Notably, there was a significant increase in the proportion of male patients (82\% vs 56\%, \emph{p} = 0.020), and a higher likelihood of ingesting multiple objects (32\% vs 24\%, \emph{p} = 0.002). Psychiatric motivation and previous ingestion were more common in this subgroup, though not statistically significant. There were also numerical trends toward higher rates of psychiatric inpatient status and psychiatric history, though these did not reach statistical significance. Fewer cases involved sharp or large-diameter objects, and no individuals in this group were detained, but these differences were also non-significant. 

\textit{Intent to harm} motivation was significantly associated with increased odds of undergoing surgery (OR = 5.40, 95\% CI [1.36, 21.43], \emph{p} = 0.024), while another documented motivation was associated with decreased odds (OR = 0.14, 95\% CI [0.03, 0.75], \emph{p} = 0.023). Subgroup analysis (\emph{n} = 21 vs \emph{n} = 51) revealed higher proportions of large-diameter object ingestions (34\%), sharp object ingestions (23\%), and male patients (71\%). Psychosocial motivation was markedly underrepresented (3\%) compared to the broader population. 

\textit{Other} motivation was significantly associated with reduced odds of undergoing surgery (OR = 0.14, 95\% CI [0.03, 0.75], \emph{p} = 0.023). Among cases in this \enquote{other} motivation subgroup (\emph{n} = 9 vs \emph{n} = 63), there were significantly fewer instances of \textit{intent to harm}, \textit{protest}, or \textit{psychosocial} motivations—each underrepresented by approximately 20\%. This group included a higher proportion of female patients (44\%) and individuals with a history of prior ingestion (31\%), as well as a greater presence of severe disability (15\%).

Sharp object ingestion was significantly associated with reduced odds of undergoing endoscopy (OR = 0.34, 95\% CI [0.13, 0.90], \emph{p} = 0.048). Subgroup analysis (\emph{n} = 34 vs \emph{n} = 38) revealed markedly lower proportions of cases involving large-diameter objects and magnets—both typically considered more suitable for endoscopic retrieval. There was also a lower proportion of female patients, and a 15\% lower incidence of severe disability. 

No other significant associations were identified among demographic or gender characteristics. This subgroup comparison is visualised in Figure~\ref{fig:nested_subgroup_plots}.

\begin{figure}[h]
\centering
\includegraphics[width=0.95\linewidth]{figures/nested_univariate_subgroup_plots.png}
\caption{Nested bar plots showing percentage differences in population characteristics between key subgroups and the comparison group. Each subplot visualises the relative increase or decrease of specific variables within a given subgroup, highlighting significant deviations (\( p < 0.05 \)) from the broader case report population.}
\label{fig:nested_subgroup_plots}
\end{figure}


\begin{table*}[h]
\caption{Univariate association test results.}
\label{tab:univariate_results}
\centering
\begin{adjustbox}{width=\textwidth}
\renewcommand{\arraystretch}{1.2}
\small
\begin{tabularx}{\textwidth}{p{4cm}>{\centering\arraybackslash}X>{\centering\arraybackslash}X>{\centering\arraybackslash}X>{\centering\arraybackslash}X>{\centering\arraybackslash}X}
\toprule
\textbf{Variable} & \textbf{Conservative} & \textbf{Endoscopy} & \textbf{Surgery} & \textbf{Death} & \textbf{Complication} \\
\midrule

\multicolumn{6}{l}{\textit{Gender}} \\
\hspace{1em}Male & 0.23 [0.04, 1.30] (p=0.110) & 1.42 [0.54, 3.72] (p=0.632) & 0.93 [0.36, 2.46] (p=1.000) & --- & 1.41 [0.52, 3.81] (p=0.671) \\
\hspace{1em}Female & 4.57 [0.82, 25.41] (p=0.101) & 0.78 [0.30, 2.03] (p=0.786) & 0.97 [0.37, 2.57] (p=1.000) & --- & 0.84 [0.31, 2.28] (p=0.932) \\
\hspace{1em}Unknown & --- & --- & --- & --- & --- \\

\multicolumn{6}{l}{\textit{Age Group}} \\
\hspace{1em}\textless{}18 & 1.96 [0.34, 11.45] (p=0.602) & 0.33 [0.08, 1.33] (p=0.194) & 2.45 [0.61, 9.84] (p=0.328) & --- & 1.84 [0.46, 7.44] (p=0.522) \\
\hspace{1em}18--25 & 1.23 [0.22, 6.94] (p=1.000) & 0.29 [0.08, 0.98] (p=0.074) & 2.80 [0.82, 9.62] (p=0.163) & --- & 1.41 [0.44, 4.56] (p=0.773) \\
\hspace{1em}26--40 & 0.28 [0.03, 2.51] (p=0.409) & 2.25 [0.84, 6.04] (p=0.171) & 0.93 [0.34, 2.51] (p=1.000) & --- & 0.83 [0.30, 2.31] (p=0.930) \\
\hspace{1em}41--60 & 2.49 [0.42, 14.83] (p=0.289) & 2.70 [0.71, 10.22] (p=0.188) & \textbf{0.18 [0.04, 0.76] (p=0.019)*} & --- & 0.54 [0.15, 2.00] (p=0.488) \\
\hspace{1em}60+ & --- & 0.65 [0.06, 7.51] (p=1.000) & 1.29 [0.11, 14.88] (p=1.000) & --- & 1.00 [0.09, 11.61] (p=1.000) \\
\hspace{1em}Unknown & --- & --- & --- & --- & 0.49 [0.03, 8.18] (p=1.000) \\

\multicolumn{6}{l}{\textit{Demographic}} \\
\hspace{1em}Detained Person & --- & 0.99 [0.28, 3.53] (p=1.000) & 0.86 [0.24, 3.08] (p=1.000) & --- & 0.63 [0.17, 2.26] (p=0.510) \\
\hspace{1em}Psychiatric Inpatient & --- & 0.44 [0.04, 4.54] (p=0.634) & 2.25 [0.22, 22.99] (p=0.634) & --- & 0.15 [0.01, 1.50] (p=0.103) \\
\hspace{1em}Displaced Person & --- & --- & 0.67 [0.03, 12.84] (p=1.000) & --- & 0.50 [0.03, 9.77] (p=1.000) \\
\hspace{1em}Under Influence of Alcohol & --- & 0.88 [0.07, 10.69] (p=1.000) & 1.00 [0.08, 12.27] (p=1.000) & --- & --- \\
\hspace{1em}Psychiatric History & 0.96 [0.20, 4.70] (p=1.000) & 0.80 [0.29, 2.20] (p=0.861) & 1.35 [0.48, 3.74] (p=0.756) & 0.71 [0.04, 11.97] (p=1.000) & 0.70 [0.24, 2.03] (p=0.696) \\
\hspace{1em}Severely Disabled & --- & 4.13 [0.74, 23.06] (p=0.115) & 0.81 [0.17, 3.93] (p=1.000) & --- & 1.31 [0.23, 7.35] (p=1.000) \\
\hspace{1em}Previous Ingestor & 1.69 [0.30, 9.38] (p=0.665) & 0.83 [0.26, 2.65] (p=0.986) & 0.74 [0.23, 2.36] (p=0.832) & 1.61 [0.09, 27.40] (p=1.000) & 0.34 [0.10, 1.23] (p=0.179) \\

\multicolumn{6}{l}{\textit{Motivation}} \\
\hspace{1em}Intent to harm & --- & 0.49 [0.16, 1.55] (p=0.347) & \textbf{5.40 [1.36, 21.43] (p=0.024)*} & --- & 0.84 [0.28, 2.56] (p=0.988) \\
\hspace{1em}Protest & --- & 0.33 [0.06, 1.74] (p=0.279) & 5.50 [0.64, 47.15] (p=0.137) & --- & 4.71 [0.55, 40.44] (p=0.251) \\
\hspace{1em}Psychiatric & 7.07 [0.80, 62.31] (p=0.105) & 1.44 [0.55, 3.77] (p=0.624) & 0.47 [0.17, 1.27] (p=0.212) & --- & 0.67 [0.24, 1.85] (p=0.608) \\
\hspace{1em}Psychosocial & 1.20 [0.21, 6.84] (p=1.000) & 0.89 [0.29, 2.72] (p=1.000) & 0.63 [0.21, 1.93] (p=0.603) & --- & 0.94 [0.30, 2.99] (p=1.000) \\
\hspace{1em}Other & 1.19 [0.13, 11.18] (p=1.000) & 3.04 [0.70, 13.29] (p=0.161) & \textbf{0.14 [0.03, 0.75] (p=0.023)*} & 7.75 [0.44, 136.41] (p=0.236) & 0.58 [0.14, 2.40] (p=0.469) \\

\multicolumn{6}{l}{\textit{Object}} \\
\hspace{1em}Button Battery & --- & --- & --- & --- & --- \\
\hspace{1em}Magnet & --- & 1.07 [0.26, 4.35] (p=1.000) & 2.46 [0.47, 12.80] (p=0.467) & --- & --- \\
\hspace{1em}Long (\textgreater{}5cm) & 0.45 [0.08, 2.51] (p=0.446) & 0.80 [0.31, 2.05] (p=0.820) & 2.43 [0.90, 6.56] (p=0.128) & 1.23 [0.07, 20.40] (p=1.000) & 1.88 [0.67, 5.24] (p=0.342) \\
\hspace{1em}Large (\textgreater{}2.5cm) Diameter & 0.23 [0.05, 1.17] (p=0.081) & 1.41 [0.48, 4.16] (p=0.727) & 1.65 [0.57, 4.80] (p=0.518) & --- & 0.92 [0.30, 2.85] (p=1.000) \\
\hspace{1em}Sharp & 1.56 [0.32, 7.51] (p=0.700) & \textbf{0.34 [0.13, 0.90] (p=0.048)*} & 2.16 [0.82, 5.72] (p=0.187) & 1.12 [0.07, 18.65] (p=1.000) & 1.09 [0.41, 2.90] (p=1.000) \\
\hspace{1em}Multiple & 4.26 [0.48, 37.48] (p=0.235) & 0.50 [0.19, 1.30] (p=0.233) & 1.03 [0.39, 2.71] (p=1.000) & --- & 2.60 [0.95, 7.13] (p=0.104) \\
\multicolumn{6}{l}{\small OR: Odds Ratio; CI: Confidence Interval; p: p-value. * indicates $p < 0.05$. Bold = statistically significant. --- = missing or unstable estimate.} \\\\
\bottomrule
\end{tabularx}
\end{adjustbox}
\end{table*}

\subsection*{Logistic Regression}
Grouped univariate linear regression results are shown in Table~\ref{tab:univariate_logistic_regression}.

In univariate logistic regression, being the 18--25 years age group was significantly associated with reduced odds of undergoing endoscopy (OR = 0.29, 95\% CI [0.08, 0.98], \emph{p} = 0.046). Subgroup analysis (\emph{n} = 18 vs \emph{n} = 54) revealed that several factors were over represented in this group; male gender, detained, sharp object ingestion and surgery; and several were significantly over represented: psychiatric history (31\% vs 46\%, \emph{p} < 0.001), psychiatric motivation (23\% vs 44\%, \emph{p} < 0.001), previous ingestions (13\% vs 25\%, \emph{p} < 0.001), and magnetic object ingestion (0\% vs 7\%, \emph{p} = 0.003). The proportion of female patients was also lower (28\% vs 43\%, \emph{p} = 0.004). 

The 41--60 years was significantly associated with reduced odds of undergoing surgery (aOR = 0.18, 95\% CI [0.04, 0.76], \emph{p} = 0.020). Subgroup analysis (\emph{n} = 11) vs \emph{n} = 61) revealed the subgroup had higher rates of males (81\% vs 55\%), psychiatric motivation (50\% vs 36\%), previous ingestions (31\% vs 21\%), and long-object ingestion (19\% vs 18\%), endoscopy (35\% vs 21\%, all \emph{p} < 0.001). 

In contrast, they were significantly less likely to be female (18\% vs 42\%), have \textit{intent to harm} motivation (14\% vs 25\%), be prisoners (0\% vs 17\%), or report \textit{protest} motivation (7\% vs 11\%). Magnet ingestion was also lower in this group (3\% vs 6\%, \emph{p} = 0.045).
Intent to harm was significantly associated with increased odds of undergoing surgery (OR = 5.77, 95\% CI [1.51, 22.03], p = 0.010). The same subgroup analysis undertaken on the unvariate association tests analysis (\emph{n} = 21 vs \emph{n} = 51) revealed higher proportions of large-diameter object ingestions (34\%), sharp object ingestions (23\%), and male patients (71\%). There were fewer females (24\% vs 45\%), less psychiatric motivation (17\% vs 51\%), less psychosocial motivation (3\% vs 29\%), fewer individuals with severe disability history (0\% vs 12\%), and fewer people with previous ingestions (20\% vs 24\%) in the harm group compared to the no-harm group.

In the object subgroup, sharp object ingestion was significantly associated with reduced odds of undergoing endoscopy (OR = 0.34, 95\% CI [0.13, 0.90], \emph{p} = 0.029).

To further explore this finding, we descriptively compared individuals who ingested sharp objects (\emph{n} = 34) with those who did not (\emph{n} = 38). Magnet ingestion was more common in the non-sharp group (12.5\% vs 0\%). The sharp object group also had a higher proportion of males (71\% vs 50\%) and more individuals with psychiatric motivation (46\% vs 30\%). Previous ingestions and prisoner status were more frequently documented among those with sharp object ingestion. Other variables such as female gender, psychosocial motivation, and severe disability history were less common in the sharp object group.

Subgroup comparison is presented visually in Figure~\ref{fig:nested_univariate_logistic_regression_subgroup_plots} in Appendix~\ref{appendix:logistic_regression}.

\begin{figure}[h]
\centering
\includegraphics[width=0.95\linewidth]{figures/nested_univariate_logstic_regression_subgroup_plots.png}
\caption{Nested bar plots showing percentage differences in predictor characteristics between significant subgroups. Significant deviations are highlighted (\( p < 0.05 \)). All data is from case reports.}
\label{fig:nested_univariate_logistic_regression_subgroup_plots}
\end{figure}

\begin{table*}[htbp]
\caption{Grouped case-level univariate logistic regression results.}
\label{tab:univariate_logistic_regression}
\centering
\begin{adjustbox}{width=\textwidth}
\renewcommand{\arraystretch}{1.2}
\small
\begin{tabularx}{\textwidth}{p{4cm}>{\centering\arraybackslash}X>{\centering\arraybackslash}X>{\centering\arraybackslash}X>{\centering\arraybackslash}X>{\centering\arraybackslash}X}
\toprule
\textbf{Variable} & \textbf{Conservative} & \textbf{Endoscopy} & \textbf{Surgery} & \textbf{Death} & \textbf{Complication} \\
\midrule

\multicolumn{6}{l}{\textit{Gender}} \\
\hspace{1em}Male & 0.23 [0.04, 1.30] (p=0.097) & 1.42 [0.54, 3.72] (p=0.471) & 0.93 [0.36, 2.46] (p=0.891) & --- & 1.41 [0.52, 3.81] (p=0.497) \\
\hspace{1em}Female & 4.57 [0.82, 25.41] (p=0.083) & 0.78 [0.30, 2.03] (p=0.607) & 0.97 [0.37, 2.57] (p=0.956) & --- & 0.84 [0.31, 2.28] (p=0.733) \\
\hspace{1em}Unknown & --- & --- & --- & --- & --- \\

\multicolumn{6}{l}{\textit{Age Group}} \\
\hspace{1em}\textless{}18 & 1.96 [0.34, 11.45] (p=0.453) & 0.33 [0.08, 1.33] (p=0.120) & 2.45 [0.61, 9.84] (p=0.206) & --- & 1.84 [0.46, 7.44] (p=0.391) \\
\hspace{1em}18--25 & 1.22 [0.22, 6.94] (p=0.819) & \textbf{0.29 [0.08, 0.98] (p=0.046)*} & 2.80 [0.82, 9.62] (p=0.102) & --- & 1.41 [0.44, 4.56] (p=0.565) \\
\hspace{1em}26--40 & 0.28 [0.03, 2.51] (p=0.258) & 2.25 [0.84, 6.04] (p=0.109) & 0.93 [0.34, 2.51] (p=0.888) & --- & 0.83 [0.30, 2.31] (p=0.726) \\
\hspace{1em}41--60 & 2.49 [0.42, 14.83] (p=0.317) & 2.70 [0.71, 10.22] (p=0.144) & \textbf{0.18 [0.04, 0.76] (p=0.020)*} & --- & 0.54 [0.15, 2.00] (p=0.359) \\
\hspace{1em}60+ & --- & 0.65 [0.06, 7.51] (p=0.730) & 1.29 [0.11, 14.88] (p=0.841) & --- & 1.00 [0.09, 11.61] (p=1.000) \\
\hspace{1em}Unknown & --- & --- & --- & --- & 0.49 [0.03, 8.18] (p=0.619) \\

\multicolumn{6}{l}{\textit{Demographic}} \\
\hspace{1em}Detained Person & --- & 0.93 [0.27, 3.28] (p=0.915) & 0.87 [0.25, 3.07] (p=0.829) & --- & 0.65 [0.18, 2.31] (p=0.504) \\
\hspace{1em}Psychiatric Inpatient & --- & 0.42 [0.04, 4.27] (p=0.465) & 1.98 [0.20, 20.00] (p=0.564) & --- & 0.15 [0.01, 1.52] (p=0.108) \\
\hspace{1em}Displaced Person & --- & --- & 0.63 [0.04, 10.46] (p=0.746) & --- & 0.49 [0.03, 8.18] (p=0.619) \\
\hspace{1em}Under Influence of Alcohol & --- & 0.65 [0.06, 7.51] (p=0.730) & 1.29 [0.11, 14.88] (p=0.841) & --- & --- \\
\hspace{1em}Psychiatric History & 1.37 [0.28, 6.64] (p=0.692) & 1.12 [0.44, 2.85] (p=0.812) & 1.00 [0.39, 2.58] (p=1.000) & 1.00 [0.06, 16.63] (p=1.000) & 0.60 [0.22, 1.63] (p=0.319) \\
\hspace{1em}Severely Disabled & --- & 3.75 [0.68, 20.80] (p=0.131) & 0.83 [0.17, 4.04] (p=0.821) & --- & 1.28 [0.23, 7.13] (p=0.779) \\
\hspace{1em}Previous Ingestor & 2.30 [0.46, 11.37] (p=0.308) & 0.95 [0.33, 2.74] (p=0.922) & 0.62 [0.21, 1.79] (p=0.379) & 2.89 [0.17, 48.62] (p=0.461) & 0.59 [0.20, 1.76] (p=0.347) \\

\multicolumn{6}{l}{\textit{Motivation}} \\
\hspace{1em}Intent to harm & --- & 0.42 [0.14, 1.24] (p=0.116) & \textbf{5.77 [1.51, 22.03] (p=0.010)*} & --- & 0.74 [0.26, 2.15] (p=0.583) \\
\hspace{1em}Protest & --- & 0.33 [0.06, 1.74] (p=0.193) & 6.00 [0.71, 50.89] (p=0.100) & --- & 4.60 [0.54, 39.14] (p=0.162) \\
\hspace{1em}Psychiatric & 7.93 [0.90, 69.66] (p=0.062) & 1.36 [0.53, 3.48] (p=0.517) & 0.52 [0.20, 1.36] (p=0.181) & --- & 0.66 [0.25, 1.76] (p=0.405) \\
\hspace{1em}Psychosocial & 1.33 [0.23, 7.58] (p=0.746) & 0.90 [0.30, 2.72] (p=0.858) & 0.64 [0.21, 1.93] (p=0.431) & --- & 0.89 [0.28, 2.80] (p=0.844) \\
\hspace{1em}Other & 1.19 [0.13, 11.18] (p=0.881) & 3.04 [0.70, 13.29] (p=0.140) & \textbf{0.14 [0.03, 0.75] (p=0.021)*} & 7.75 [0.44, 136.41] (p=0.162) & 0.58 [0.14, 2.40] (p=0.453) \\

\multicolumn{6}{l}{\textit{Object}} \\
\hspace{1em}Button Battery & --- & --- & --- & --- & --- \\
\hspace{1em}Magnet & --- & 1.07 [0.26, 4.35] (p=0.928) & 2.46 [0.47, 12.80] (p=0.285) & --- & --- \\
\hspace{1em}Long (\textgreater{}5cm) & 0.47 [0.08, 2.58] (p=0.383) & 0.84 [0.33, 2.14] (p=0.710) & 2.31 [0.86, 6.22] (p=0.097) & 1.26 [0.08, 20.93] (p=0.873) & 2.00 [0.72, 5.55] (p=0.183) \\
\hspace{1em}Large (\textgreater{}2.5cm) Diameter & 0.27 [0.05, 1.31] (p=0.104) & 1.33 [0.47, 3.77] (p=0.586) & 1.67 [0.59, 4.67] (p=0.332) & --- & 1.00 [0.34, 2.94] (p=1.000) \\
\hspace{1em}Sharp & 1.56 [0.32, 7.51] (p=0.582) & \textbf{0.34 [0.13, 0.90] (p=0.029)*} & 2.16 [0.82, 5.72] (p=0.121) & 1.12 [0.07, 18.65] (p=0.936) & 1.09 [0.41, 2.90] (p=0.867) \\
\hspace{1em}Multiple & 4.26 [0.48, 37.48] (p=0.191) & 0.50 [0.19, 1.30] (p=0.153) & 1.03 [0.39, 2.71] (p=0.956) & --- & 2.60 [0.95, 7.13] (p=0.063) \\
\multicolumn{6}{l}{\small OR: Odds Ratio; CI: Confidence Interval; p: p-value. * indicates $p < 0.05$. Bold = statistically significant. --- = missing or unstable estimate.} \\\\
\bottomrule
\end{tabularx}
\end{adjustbox}
\end{table*}



\subsection*{Meta-analysis}
Given that series numbers were small, case reports were flattened into a series and presented alongside other case series. This is demonstrated in Table~\ref{tab:series_results}. 

Meta-analyses of proportions were then performed on all series-data -- including collapsed case reports -- using random-effects models with restricted maximum likelihood (REML) estimation and Hartung-Knapp (HK) adjustments for confidence intervals, appropriate for small numbers of studies \cite{Tanriver-Ayder_2021}. In general, REML yielded wider confidence intervals and lower heterogeneity estimates than the DerSimonian \& Laird (DL) method, reflecting greater uncertainty in pooled estimates from small samples.

Herein, where cases in subgroups were present as cases in case reports, case-level analysis was undertaken, as outlined in the methodolgy and similarly to previous rounds of analysis shown above.

A plot of aggregate series-level meta-analysis of pooled variable and outcome proportions is shown in Figure~\ref{fig:reml_meta}. For comparison, DL estimates were also calculated for outcomes, yielding narrower confidence intervals and higher I\textsuperscript{2} values. These are included in Figure~\ref{fig:ma_comparison}.

\begin{figure}[htbp]
\centering
\includegraphics[width=\columnwidth]{figures/full_meta_analysis_plot.png}
\caption{Comparison of DL vs REML for meta-analysis of proportions.}
\label{fig:ma_comparison}
\end{figure}

In the demographic subgroup, meta-analyses of proportions were performed for psychiatric history and prior ingestion. The pooled proportion of patients with a psychiatric history was 42.5\% (95\% CI 2.8\%--95.0\%), with moderate heterogeneity ($I^2$ = 34.3\%). For prior ingestion, the pooled proportion was 24.0\% (95\% CI 16.2\%--34.1\%), with moderate heterogeneity ($I^2$ = 49.2\%). Meta-analysis could not be performed for other demographic variables due to insufficient data.

In the motivation subgroup, pooled estimates were available for intent to harm, protest, and psychiatric motivation. The proportion of patients with intent to harm was 25.5\% (95\% CI 10.0\%--51.6\%), with moderate heterogeneity ($I^2$ = 27.2\%). Protest motivation was reported in 60.3\% (95\% CI 2.0\%--99.1\%) of cases, and psychiatric motivation in 50.4\% (95\% CI 40.2\%--60.6\%), both with moderate heterogeneity ($I^2$ = 26.6\% and 33.3\%, respectively).

For the object subgroup, meta-analyses were conducted for long objects, sharp objects, and multiple object ingestion. The pooled proportion was 52.2\% (95\% CI 40.4\%--63.7\%) for long objects, 61.8\% (95\% CI 18.4\%--92.1\%) for sharp objects, and 39.2\% (95\% CI 1.5\%--96.5\%) for multiple objects. Heterogeneity was substantial for long and multiple objects ($I^2$ = 50.6\% and 51.2\%, respectively), and moderate for sharp objects ($I^2$ = 45.3\%).

In terms of outcomes, proportions were pooled for endoscopy, surgery, death, complications, and conservative management. Endoscopy was performed in 45.1\% of cases (95\% CI 0.2\%--99.7\%), surgery in 28.5\% (95\% CI 6.5\%--69.6\%), death occurred in 3.4\% (95\% CI 1.1\%--10.1\%), complications were reported in 34.8\% (95\% CI 7.5\%--78.0\%), and conservative management was used in 50.0\% of cases (95\% CI 0.7\%--99.3\%). Heterogeneity was moderate for endoscopy, complications, and conservative management ($I^2$ = 37.2\%, 50.1\%, and 32.3\%, respectively), and low for surgery and death ($I^2$ = 17.6\% and <0.5\%).

Meta-analysis could not be performed for the \textit{gender} subgroup or for other variables not mentioned above due to insufficient series-level data.

\begin{figure}[H]
    \includegraphics[width=\linewidth]{figures/ma_comparison_plot.png}
\caption{DL vs REML meta-analysis of outcome proportions.}
\label{fig:ma_comparison_plot}
\end{figure}
\subsection*{Meta Regression} A full table of grouped aggregate series-level results for univariate meta-regression is available in Table~\ref{tab:univariate_meta_regression}. 
In the gender subgroup, female gender was associated with an increased likelihood of surgery (OR = 1.05, 95\% CI [1.00, 1.10], p = 0.044). As all female cases were drawn from case reports, further case-level subgroup analysis was conducted comparing females (\emph{n} = 28) with non-females (\emph{n} = 54). 

Females were significantly more likely to have a positive psychiatric history (58.6\% vs 35.2\%, \emph{p} = 0.000006) and psychiatric motivation (45.7\% vs 32.7\%, \emph{p} = 0.000025). They were also less likely to be prisoners (0\% vs 22.2\%, \emph{p} = 0.000053) and less likely to report protest (2.9\% vs 14.5\%, \emph{p} = 0.045). Trends toward lower prevalence of intent to harm (14.3\% vs 29.1\%, \emph{p} = 0.111) and alcohol use (0\% vs 5.6\%, \emph{p} = 0.232) among females were noted, although these did not reach statistical significance.


Unknown gender was associated with an increased likelihood of surgery (OR = 3.81, 95\% CI [1.06, 13.68], p = 0.044). This is likely a finding as a result of publication bias. The single female cases in this review is from to a 29 year old female with borderline personality disorder and 6 documented laparotomies for repeated IIFO. \cite{fjbuilsRepeatedBehaviorDeliberate2024}. All other comparisons in this subgroup were non-significant. 
In the demographic subgroup, psychiatric inpatient status was associated with an increased likelihood of surgery (OR = 1.40, 95\% CI [1.01, 1.92], p = 0.044). 

All four psychiatric inpatients were cases from case reports, so case-level analysis was undertaken on psychiatric inpatients (\emph{n} == 4) vs non-psychiatric inpatients (\emph{n} = 68). Psychiatric inpatients were significantly more likely to exhibit intent to harm (25.0\% vs 23.3\%, \emph{p} < 0.00001), although the absolute difference was small. They were also significantly less likely to have psychosocial (0\% vs 19.8\%, \emph{p} < 0.00001), protest (0\% vs 10.5\%, \emph{p} = 0.001), or other motivations (0\% vs 10.5\%, \emph{p} = 0.001). No psychiatric inpatients were prisoners (vs 17.1\% among non-inpatients, \emph{p} < 0.001) or ingested magnets (0\% vs 5.6\%, \emph{p} = 0.001). All other comparisons in this subgroup were non-significant. 

In the motivation subgroup, psychosocial motivation was associated with an increased likelihood of surgery (OR = 1.08, 95\% CI [1.00, 1.17], p = 0.044). All cases that were motivated in this manner came from case reports, so case-level analysis was persued. 

Several illustrative cases reflected the diverse contexts and consequences of psychosocially motivated ingestion. These included: \enquote{a jackass and a fish} (a 28 year old male) in the Netherlands who drunkenly ingested a catfish, which preceeded to erect and lock the spines of its pectoral in the man's hypopharynx, leading to laryngeal oedema, intubation, endoscopy and a prolonged intensive care stay \cite{Benoist_2019e}; 16 year old female in Israel who ingested toilet roll \enquote{as a means of weightloss} (managed conservatively) \cite{Goldman_1998f}; a 19 year old male in New Zealand who underwent a median sternotomy and pericardiotomy for a penetrating left ventricular injury after ingested 8 sewing needles \enquote{in an attempt to impress his friends} \cite{Sobnach_2011f}; a 12 year old girl in Sweden who underwent laparoscopy for a double colonic perforation after swallowing multiple magnetic balls whilst \enquote{playing a game with her brother} \cite{Naji_2012f}; and a 33 year old female in Poland who, \enquote{motivated by the aim to induce weight loss}, ingested a water-filled balloon, resulting in bowel obstruction and laparotomy \cite{Wnęk_2015f}.

Statistical comparison (\emph{n} = 17 vs \emph{n} = 55) showed that psychosocially motivated cases were significantly less likely to present with intent to harm (4.3\% vs 29.9\%, \emph{p} < 0.00001), and had lower rates of protest and other motivations. They also had lower prevalence of severe disability (\emph{p} = 0.009) and psychiatric inpatient status (\emph{p} = 0.107), though these did not all remain significant after correction. Interestingly, previous ingestion history was slightly more common among psychosocial cases (25.0\% vs 22.7\%, \emph{p} < 0.0001), though the clinical relevance of this difference is unclear.

% egin{figure}[b]
%\centering
%\includegraphics[width=	extwidth]{figures/series_pies.png}
%\caption{Pie charts showing distribution of variables for all ingestions.}
%\label{fig:variable_pie_charts}
%\end{figure}
