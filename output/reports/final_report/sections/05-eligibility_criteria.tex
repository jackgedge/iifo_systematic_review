\subsection*{} 
\vspace{-1em}
This study was conducted according to the Preferred Reporting Items for Systematic Reviews and Meta-Analyses (PRISMA) guidelines \cite{Page_2021}. Ethical approval was not required as all analysis was based on published data. Eligibility criteria were structured using the PICOS (Population, Intervention, Comparator, Outcome, Studies) framework.


\subsection*{Eligibility Criteria}

\begin{table}[H]
\renewcommand{\arraystretch}{1.05}
\centering
\begin{tabular}{p{3.0cm} p{5.0cm}}
\toprule
\textbf{Category} & \textbf{Details} \\
\midrule

Population & Any human; any age group. \\
\midrule

Interventions or exposures & Non-accidental ingestion of a true foreign body (non-nutritive items). \\
\midrule

Comparators / Control group & 
\textbf{Demographics:} \newline
Gender, age, detained person, psychiatric inpatient, displaced person, under influence of alcohol, psychiatric history, severely disabled, previous ingestion. \newline
\textbf{Motivation:} \newline
Intent to harm, psychiatric, psychosocial, protest, other. \newline
\textbf{Object characteristics:} \newline
Button battery, magnet, long ($>$5 cm), large diameter ($>$2.5 cm), multiple, blunt objects, sharp-pointed objects. \\
\midrule

Outcomes of interest & Endoscopic intervention, surgical intervention, conservative management, complication rates, mortality. \\
\midrule

Setting & Any setting. \\
\midrule

Study designs & Any design. \\
\bottomrule
\end{tabular}
\caption{Inclusion criteria structured using the PICO framework.}
\label{tab:inclusion-criteria-intro}
\end{table}

This review included studies involving human participants of any age who had non-accidental ingestion of a true foreign body (non-nutritive items). Studies were only included if they reported on all of the following data: demographic/population data; motivations for ingestion; object characteristics; and outcomes (whether conservative, endoscopic, or surgical treatment was used). All settings were eligible, and a wide range of study designs were accepted.

Studies were excluded if the full text was not available in English, as this would limit the ability to assess methodology and extract reliable data. Accidental ingestions, non-human studies, and ingestions undertaken in controlled research settings were excluded to ensure that the review focused solely on real-world, intentional ingestion events relevant to clinical practice.

Ingestions that were not explicitly intentional, or where intent could not be confidently inferred (e.g., young children with no relevant history or psychiatric comorbidities), were excluded to maintain a clear focus on non-accidental ingestion as the exposure of interest. Studies involving the ingestion or co-ingestion of substances such as drugs or poisons were also excluded to avoid confounding physiological effects and to ensure consistency with the review’s focus on physical foreign bodies. Similarly, studies in which the primary cause of death was unrelated to the ingestion event—such as concurrent suicide by other means—were excluded to avoid misattributing outcomes to foreign body ingestion.

Studies published before the advent of endoscopy (1906) were excluded to reduce historical bias and to avoid skewing the intervention data toward surgical outcomes, given that endoscopy was not available as a clinical management option at that time. Finally, studies were excluded if they lacked original empirical data (e.g., reviews, editorials, or commentaries), involved duplicate or overlapping datasets (with only the most comprehensive or recent version retained), or failed to report on key variables such as outcomes, motivations, or object characteristics—essential for subgroup analyses and interpretation of results.

A full list of eligibility criteria is shown in Table~\ref{tab:inclusion-criteria-intro}. This is reproduced for in a larger format for clarity in Appendix~\ref{tab:inclusion-criteria}. A full list of exclusion criteria is available in Appendix~\ref{tab:exclusion-criteria}.
