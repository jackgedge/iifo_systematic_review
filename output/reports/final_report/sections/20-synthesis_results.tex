\subsection*{Synthesis}
\subsubsection*{Case Reports}
\paragraph*{Chi-Squared Tests} A full table of grouped case-level chi-squared test results for the is available in Table~\ref{tab:grouped_chi2_wide}. 
In the age group subgroup, 41--60 age group was significantly associated with surgery (OR = 0.19, 95\% CI [0.04, 0.78], p = 0.034); 41--60 age group was significantly associated with death (OR = nan, 95\% CI [nan, nan], p = 0.018). In the motivation subgroup, intent to harm motivation was significantly associated with surgery (OR = 5.10, 95\% CI [1.28, 20.33], p = 0.032); other motivation was significantly associated with surgery (OR = 0.15, 95\% CI [0.03, 0.77], p = 0.031). There were no significant associations with outcomes in the gender, object, and population subgroups.
\paragraph*{{Multivariate Logistic Regression}} A full table of grouped series level logistic regression results is available in Table~\ref{{tab:grouped_logistic_wide}}. In the motivation subgroup, intent to harm motivation was significantly associated with surgery (OR = 15.95, 95\% CI [1.32, 192.48], p = 0.029). In the object subgroup, long (\textgreater{}5cm) object ingestion was significantly associated with surgery (OR = 7.66, 95\% CI [1.04, 56.32], p = 0.045); multiple object ingestion was significantly associated with complication (OR = 14.79, 95\% CI [2.08, 105.31], p = 0.007); long (\textgreater{}5cm) object ingestion was significantly associated with complication (OR = 16.50, 95\% CI [1.76, 154.43], p = 0.014). In the population subgroup, detained person was significantly associated with complication (OR = 0.04, 95\% CI [0.00, 0.69], p = 0.028). There were no significant associations with outcomes in the age group, and gender subgroups.
\subsubsection*{Case Series} 
\paragraph*{Meta-analysis of Proportions} A plot of case series meta-analysis of pooled outcome proportions is shown in Figure~\ref{fig:meta}. We performed meta-analyses of proportions for endoscopy, surgery, and conservative. The pooled proportion of patients undergoing endoscopy was 41.6\% (95\% CI 0.6\%--98.9\%), with substantial heterogeneity ($I^2$ = 94.9\%). The pooled proportion of patients undergoing surgery was 17.8\% (95\% CI 10.4\%--28.8\%), with low heterogeneity ($I^2$ = 17.4\%). The pooled proportion of patients conservative management was 76.2\% (95\% CI 60.2\%--87.1\%), with low heterogeneity ($I^2$ <0.5\%). Meta-analyses could not be performed for death and complication because fewer than two studies reported data on these outcomes.\paragraph*{Meta Regression} A full table of grouped aggregate series-level results for univariate meta-regression is available in Table~\ref{tab:series_meta_regression}.In the gender subgroup, male gender was associated with a reduced likelihood of conservative management (OR = 0.87, 95\% CI [0.78, 0.97], p = 0.040). All other comparisons in this subgroup were non-significant.In the population subgroup, being a detained person was associated with an increased likelihood of conservative management (OR = 1.62, 95\% CI [1.11, 2.37], p = 0.040). All other comparisons in this subgroup were non-significant.In the motivation subgroup, intent to harm was associated with a reduced likelihood of conservative management (OR = 0.82, 95\% CI [0.77, 0.87], p = 0.014); intent to harm was associated with an increased likelihood of undergoing surgery (OR = 1.12, 95\% CI [1.05, 1.18], p = 0.015); psychiatric motivation was associated with an increased likelihood of undergoing surgery (OR = 1.07, 95\% CI [1.03, 1.11], p = 0.017); psychosocial motivation was associated with a reduced likelihood of conservative management (OR = 0.82, 95\% CI [0.70, 0.96], p = 0.040); another documented motivation was associated with a reduced likelihood of conservative management (OR = 0.69, 95\% CI [0.51, 0.92], p = 0.040); another documented motivation was associated with an increased likelihood of undergoing surgery (OR = 1.25, 95\% CI [1.02, 1.52], p = 0.041); psychosocial motivation was associated with an increased likelihood of undergoing surgery (OR = 1.12, 95\% CI [1.01, 1.25], p = 0.041). All other comparisons in this subgroup were non-significant.In the object subgroup, Sharp object ingestion was associated with a reduced likelihood of conservative management (OR = 0.80, 95\% CI [0.65, 1.00], p = 0.049). All other comparisons in this subgroup were non-significant.\paragraph*{Investigation of Heterogeneity}

We explored heterogeneity across study results using two complementary approaches.

First, meta-analyses of proportions were conducted for each outcome of interest across included case series. Between-study heterogeneity was quantified using the $I^2$ statistic and $\tau^2$ variance. The degree of heterogeneity varied substantially across outcomes:
\begin{itemize}
\item Substantial heterogeneity ($I^2 = 95\\%$) was observed for endoscopy.
\item Heterogeneity was moderate for surgery ($I^2 = 17\\%$).
\item Heterogeneity was low ($I^2 = 0\\%$) for conservative management.
\item Meta-analyses could not be meaningfully conducted for death and complications due to insufficient data.
\end{itemize}

Second, to investigate potential sources of heterogeneity, we performed univariate meta-regressions for each outcome. Given the small number of case series available, we combined series-level data with collapsed aggregate case reports to increase the number of contributing studies. Meta-regressions examined associations between outcome proportions and patient-level predictors including gender, population characteristics, ingestion motivations, and object type.

Significant associations between certain predictors and outcomes were identified, particularly in relation to motivation and surgery or conservative management outcomes. This suggests that clinical heterogeneity in patient characteristics likely contributed to observed between-study differences.

No additional subgroup or sensitivity analyses were performed, given the limited number of studies and inconsistent reporting of potential study-level moderators. \paragraph*{Sensitivity Analyses}

Formal sensitivity analyses were not performed due to the limited number of included case series and the small number of studies available for several outcomes. Given the small number of eligible series and heterogeneity in reporting, the meta-analyses of proportions and meta-regressions were considered primarily descriptive in nature.

To partially address robustness, we incorporated aggregate case report data into the meta-regression models to increase the effective number of series contributing to each outcome. This allowed exploratory assessment of predictor-outcome relationships across a larger pooled dataset. However, no additional sensitivity analyses (e.g. leave-one-out analyses, exclusion of small studies, or alternative meta-analytic models) were conducted, as such analyses would not have been statistically meaningful in the context of the available data. 