\subsection{Discussion}

\subsubsection{Protest and intent-to-harm in detention could modulate outcomes}
Motivations differ in custodial settings and with this, so do outcomes. Among the detained persons reported in case reports ($n = 71$) there was a pooled surgery rate of 58\%. 

Intent-to-harm motivation was dominant (80\%) (associated with a near six-fold in surgery rates in this review), with protest motivation rates of 55\%. Sharp objects were ingested in 67\% of ingestions. Alongside this, meta-regression of three, all male case series ($n = 90$) prison populations in three countries suggests that protest motivation was associated with a near 80\% reduction in the odds of undergoing surgery, despite a pooled estimate of $\approx$90\% of ingested objects being sharp (wide confidence intervals with moderate heterogeneity). 

In one series \cite{Lee_2007} with higher rates protest motivation(97\% of $n = 52$) and lower rates of intent-to-harm (0\%), with 64\% of objects ingested being sharp, there was a surgery rate of 12\%. Whereas, in another \cite{Karp_1991b} ($n = 19$) with lower rates of protest motivation (16\%) and higher (12\%) intent-to-harm rate, surgery rates rose to 26\% ($n = 6$). One caveat to this is that endoscopy rates were not reported in the latter series \cite{Karp_1991b} and there were 16 years separating the two publications. However, the third series, published in 2016 -- 9 years after the later study and 25 years after the earlier -- where rates of protest were 90\% with 84\% sharp object ingestion, 10\% intent-to-harm rate and 0\% psychiatric motivation rate, the surgery rate was between the rates of the other two series at 21\%, conservative rates 79\%, with one death (out of 3 reported in this review). 

Publication bias likely skews this result. Including one outstanding series \cite{Losanoff_1996} and an accompanying case report of $5 + 1$ \cite{Losanoff_1997e} cases of \enquote{gastrointestinal cross} ingestion. Five inmates in one prison in Bulgaria ingested a total of 20 crosses underwent laparotomy for a similar number of perforations.

\subsubsection{Middle-age, psychiatric history, recurrent ingestion reduce the odds of surgery, but increase mortality}
Two deaths occurred in this age group, outside of detention settings, but with psychiatric motivation and history. One 54 year old man with 15 year history of ingestion who died after repeatedly refusing surgery \cite{Kumar_2001}, who died after ingesting metal secondary to acuphagia from 3,4-Methylenedioxymethamphetamine (MDMA) use \cite{Emamhadi_2018}.

This likely indicates a cohort significant psychiatric co-morbidities, indicated by higher rates of recurrent ingestion, but present late or refuse treatment.  

\begin{figure}[H]
\centering
\includegraphics[width=\columnwidth]{figures/ma_series_only_plot.png}
\caption{Meta-Analysis of case series. Prop = Pooled Proportion; Tau2 = $\tau^2$; N = $n$; I2 = $I^2$; REML and HK estimated 95\% Confidence Intervals}
\label{fig:ma_series_only}
\end{figure}

\subsection{Limitations}

\subsubsection{Publication bias, funny titles and countertransference anger}
Both of the above effects could be explained by publication bias, but deaths were outstanding cases. 

Elsewhere, there appears to be a competition amongst authors (unbeknowst to this one) to publish case reports with the wittiest title. Examples include: \enquote{Loose Screws}, in reference to a female with an extensive psychiatric history who -- as advised by her rastafarian -- ingested nails in a milkshake \cite{Hardy_2023g}; \textit{Ingested Magnets: the force within} \cite{Tay_2004}; \textit{Now You See It, Endo You Don't: Case of the Disappearing Knife} \cite{Fine_2013}; and \textit{A Jackass And A Fish} \cite{Benoist_2019e}. The list goes on. Joking aside, extreme cases of IIFO are remarkable, publication-worthy and interesting to read, likely leading to publication bias.

Furthermore, extreme cases of recurrent intentional ingestion cause 'countertransference anger' \cite{Palese_2012} among clinicians who have to endure sometimes years of repeat encounters with prolific recurrent ingesters, who are often complex patients with complex mental health needs. This is well described by Gitlin \textit{et al.} \cite{Gitlin_2007} who describes how clinicians are \enquote{held hostage} by ingesters \enquote{in the midst of their self-harming process} after an initiating an \enquote{ongoing injurious sequence}. 

\subsubsection{Selection Bias}
This review was conducted by a single reviewer (JGE), which introduces the potential for selection bias and inconsistent application of eligibility criteria. Although inclusion decisions were guided by a predefined protocol and supervision by an experienced reviewer (GC), dual independent screening was not performed due to resource limitations. This may have affected reproducibility and sensitivity of the search and screening processes.

\subsubsection{Unmeasured confounding}
The review did not record item location. This could have led to a reduced perception of ingestion risk from underestimation of high-risk object rates and introduced an unmeasured confounding factor. This reduced rates of perceived high risk ingestions. 

Time to presentation was not recorded. Ingesters that present early have increased endoscopy rates. This may have skewed endoscopy rates in some cases reports and case series, but the effect cannot be quantified in this review. 

Also, this review did not record any alteration of objects that ingesters may have undertaken. Although, in clinical practice, this is likely difficult to assess practically. Items may appear radiopaque and sharp imaging, but wrapped by the ingester to reduce morbidity \cite{Ataya_2013,Albeldawi_2014}. Perhaps one day motivation could serve as a surrogate marker.

\subsubsection{Information Bias}
Information bias from missing data in studies likely distorted true effects in the case series analysis. This is particularly in age group and object characteristics analysis, but also endoscopic outcomes and demographic characteristics.  

\begin{figure}[H]
\centering
\includegraphics[width=\columnwidth]{figures/ma_all_plot.png}
\caption{Meta-Analysis of case series and pooled case reports. Meta-analysis of case series. Prop = Pooled Proportion; Tau2 = $\tau^2$; N = $n$; I2 = $I^2$; REML and HK estimated 95\% Confidence Intervals}
\label{fig:ma_all}
\end{figure}

\subsection{Conclusion}

This review represents the first systematic synthesis of outcomes following intentional ingestion of foreign objects (IIFO) with a focus on patient motivation. Across both case reports and case series, the nature of the ingestion—particularly protest versus intent-to-harm—appears to modulate clinical decision-making, especially in detained populations. While protest-motivated ingestion was often managed conservatively, intent-to-harm cases were associated with significantly increased rates of surgical intervention. 

Middle-aged adults (40--64 years) were significantly less likely to undergo surgery, and subgroup analysis suggests that psychiatric history, prior ingestion, and multiple object ingestion may contribute to more conservative management strategies and refusal of treatment, modulating this effect. Sharp object ingestion, traditionally viewed as high-risk, was not uniformly associated with surgical outcomes, particularly in the context of protest motivations and detention settings. However, outside of protest motivations in this age group, those with psychiatric co-morbidities had a tendency towards increased mortality.

Findings must be interpreted in light of limitations including small sample sizes, publication bias, selection bias, missing data, and unmeasured confounding factors such as object location, wrapping, and timing of presentation. Nonetheless, this review highlights the complexity of IIFO and the influence of both psychosocial context and clinical framing on treatment decisions.

Future research should focus on prospective, standardised reporting of IIFO cases to better understand how patient characteristics and motivations influence management. Greater clarity around risk stratification may help clinicians provide safer, more consistent care for this often-marginalised population.